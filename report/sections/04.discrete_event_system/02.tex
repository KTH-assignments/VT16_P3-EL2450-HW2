\section{Question 2}

The set of states of the discrete event system that models the complete
manufacturing process $Q$ is the set of combination of all states of the DES of
machines $M_1$, $M_2$ and buffer $B$:

$$Q \equiv Q_{M_1} \times Q_{M_2} \times Q_{B}$$

State set $Q$ is defined in table \ref{tbl:04.02_all_states}.


\begin{table}[H]\centering
    \begin{tabular}{r|lll}
    state notation of \\ the complete DES & M1         & M2         & B     \\ \hline
    (I,I,E)                            & Idle       & Idle       & Empty \\
    (I,I,F)                            & Idle       & Idle       & Full  \\
    (I,P,E)                            & Idle       & Processing & Empty \\
    (I,P,F)                            & Idle       & Processing & Full  \\
    (I,D,E)                            & Idle       & Down       & Empty \\
    (I,D,F)                            & Idle       & Down       & Full  \\
    (P,I,E)                            & Processing & Idle       & Empty \\
    (P,I,F)                            & Processing & Idle       & Full  \\
    (P,P,E)                            & Processing & Processing & Empty \\
    (P,P,F)                            & Processing & Processing & Full  \\
    (P,D,E)                            & Processing & Down       & Empty \\
    (P,D,F)                            & Processing & Down       & Full  \\
    (D,I,E)                            & Down       & Idle       & Empty \\
    (D,I,F)                            & Down       & Idle       & Full  \\
    (D,P,E)                            & Down       & Processing & Empty \\
    (D,P,F)                            & Down       & Processing & Full  \\
    (D,D,E)                            & Down       & Down       & Empty \\
    (D,D,F)                            & Down       & Down       & Full  \\
    \end{tabular}
    \caption{The complete set of all state combinations between machines $M_1$
      and $M_2$ and buffer $B$.}
    \label{tbl:04.02_all_states}
\end{table}

The set of events of the overall DES is the union of the sets of the three
separate DES:

\begin{align*}
  E &= E_{M_1} \cup E_{M_2} \cup E_{B} = \\
    & \{\texttt{START(1), END(1), BREAKDOWN(1), END(1)}, \\
    &\ \ \texttt{START(2), END(2), BREAKDOWN(2), END(2)}, \\
    &\ \ \texttt{PLACED, TAKEN}\}
\end{align*}

where the number $X$ inside parentheses denotes an event pertaining to machine
$M_X$.

The initial state $q_0$ shall be state $q_0 = (I,I,E)$, that is, both $M_1$ and
$M_2$ are idle, and buffer $B$ is empty. Furthermore, the marked state shall be
the initial state $Q_m \equiv \{q_0\}$.

Since there are $18$ states, the maximum number of transitions between them is
$18^2$. However, it is assumed that transitions between states are caused by
single events $e \in E$, i.e. two events cannot happen
at the same time. This fact decreases the number of allowed transitions from
$18^2$ to $66$.

The transition function $\delta()$ is shown in table \ref{tbl:04.02_delta}.
The state machine diagram is illustrated in figure \ref{fig:fsm_2} of the
appendix. From this and the size of the number of transitions we can discern
that synthesis of separate processes into one, even simple ones like these in
this exercise, can lead to a DES whose complexity is higher than that of the
separate DES.

\begin{table}\centering
  \begin{tabular}{ll}
    $\delta$((I,I,E), $\texttt{PLACED})$ = (I,I,F)       & $\delta$((P,P,F), $\texttt{END(1)})$ = (I,P,F)       \\
    $\delta$((I,I,E), $\texttt{START(2)})$ = (I,P,E)     & $\delta$((P,P,F), $\texttt{END(2)})$ = (P,I,F)       \\
    $\delta$((I,I,E), $\texttt{START(1)})$ = (P,I,E)     & $\delta$((P,P,F), $\texttt{TAKEN})$ = (P,P,E)        \\
    $\delta$((I,I,F), $\texttt{TAKEN})$ = (I,I,E)        & $\delta$((P,P,F), $\texttt{BREAKDOWN(2)})$ = (P,D,F) \\
    $\delta$((I,I,F), $\texttt{START(2)})$ = (I,P,F)     & $\delta$((P,P,F), $\texttt{BREAKDOWN(1)})$ = (D,P,F) \\
    $\delta$((I,I,F), $\texttt{START(1)})$ = (P,I,F)     & $\delta$((P,D,E), $\texttt{END(1)})$ = (I,D,E)       \\
    $\delta$((I,P,E), $\texttt{END(2)})$ = (I,I,E)       & $\delta$((P,D,E), $\texttt{REPAIR(2)})$ = (P,I,E)    \\
    $\delta$((I,P,E), $\texttt{PLACED})$ = (I,P,F)       & $\delta$((P,D,E), $\texttt{PLACED})$ = (P,D,F)       \\
    $\delta$((I,P,E), $\texttt{BREAKDOWN(2)})$ = (I,D,E) & $\delta$((P,D,E), $\texttt{BREAKDOWN(1)})$ = (D,D,E) \\
    $\delta$((I,P,E), $\texttt{START(1)})$ = (P,P,E)     & $\delta$((P,D,F), $\texttt{END(1)})$ = (I,D,F)       \\
    $\delta$((I,P,F), $\texttt{END(2)})$ = (I,I,F)       & $\delta$((P,D,F), $\texttt{REPAIR(2)})$ = (P,I,F)    \\
    $\delta$((I,P,F), $\texttt{TAKEN})$ = (I,P,E)        & $\delta$((P,D,F), $\texttt{TAKEN})$ = (P,D,E)         \\
    $\delta$((I,P,F), $\texttt{BREAKDOWN(2)})$ = (I,D,F) & $\delta$((P,D,F), $\texttt{BREAKDOWN(1)})$ = (D,D,F)  \\
    $\delta$((I,P,F), $\texttt{START(1)})$ = (P,P,F)     & $\delta$((D,I,E), $\texttt{REPAIR(1)})$ = (I,I,E)     \\
    $\delta$((I,D,E), $\texttt{REPAIR(1)})$ = (I,I,E)    & $\delta$((D,I,E), $\texttt{PLACED})$ = (D,I,F)        \\
    $\delta$((I,D,E), $\texttt{PLACED})$ = (I,D,F)       & $\delta$((D,I,E), $\texttt{START(2)})$ = (D,P,E)      \\
    $\delta$((I,D,E), $\texttt{START(1)})$ = (P,D,E)     & $\delta$((D,I,F), $\texttt{REPAIR(1)})$ = (I,I,F)     \\
    $\delta$((I,D,F), $\texttt{REPAIR(2)})$ = (I,I,F)    & $\delta$((D,I,F), $\texttt{TAKEN})$ = (D,I,E)         \\
    $\delta$((I,D,F), $\texttt{TAKEN})$ = (I,D,E)        & $\delta$((D,I,F), $\texttt{START(2)})$ = (D,P,F)      \\
    $\delta$((I,D,F), $\texttt{START(1)})$ = (P,D,F)     & $\delta$((D,P,E), $\texttt{REPAIR(1)})$ = (I,P,E)     \\
    $\delta$((P,I,E), $\texttt{END(1)})$ = (I,I,E)       & $\delta$((D,P,E), $\texttt{END(2)})$ = (D,I,E)        \\
    $\delta$((P,I,E), $\texttt{PLACED})$ = (P,I,F)       & $\delta$((D,P,E), $\texttt{PLACED})$ = (D,P,F)        \\
    $\delta$((P,I,E), $\texttt{START(2)})$ = (P,P,E)     & $\delta$((D,P,E), $\texttt{BREAKDOWN(2)})$ = (D,D,E)  \\
    $\delta$((P,I,E), $\texttt{BREAKDOWN(1)})$ = (D,I,E) & $\delta$((D,P,F), $\texttt{REPAIR(1)})$ = (I,P,F)     \\
    $\delta$((P,I,F), $\texttt{END(1)})$ = (I,I,F)       & $\delta$((D,P,F), $\texttt{END(2)})$ = (D,I,F)        \\
    $\delta$((P,I,F), $\texttt{TAKEN})$ = (P,I,E)        & $\delta$((D,P,F), $\texttt{TAKEN})$ = (D,P,E)         \\
    $\delta$((P,I,F), $\texttt{END(2)})$ = (P,P,F)       & $\delta$((D,P,F), $\texttt{BREAKDOWN(2)})$ = (D,D,F)  \\
    $\delta$((P,I,F), $\texttt{BREAKDOWN(1)})$ = (D,I,F) & $\delta$((D,D,E), $\texttt{REPAIR(1)})$ = (I,D,E)     \\
    $\delta$((P,P,E), $\texttt{END(1)})$ = (I,P,E)       & $\delta$((D,D,E), $\texttt{REPAIR(2)})$ = (D,I,E)     \\
    $\delta$((P,P,E), $\texttt{END(2)})$ = (P,I,E)       & $\delta$((D,D,E), $\texttt{PLACED})$ = (D,D,F)        \\
    $\delta$((P,P,E), $\texttt{PLACED})$ = (P,P,F)       & $\delta$((D,D,F), $\texttt{REPAIR(1)})$ = (I,D,F)     \\
    $\delta$((P,P,E), $\texttt{BREAKDOWN(2)})$ = (P,D,E) & $\delta$((D,D,F), $\texttt{REPAIR(2)})$ = (D,I,F)     \\
    $\delta$((P,P,E), $\texttt{BREAKDOWN(1)})$ = (D,P,E) & $\delta$((D,D,F), $\texttt{TAKEN})$ = (D,D,E)         \\
  \end{tabular}
  \caption{Allowed transitions between states.}
  \label{tbl:04.02_delta}
\end{table}
