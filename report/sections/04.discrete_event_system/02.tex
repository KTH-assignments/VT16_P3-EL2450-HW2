\section{Question 2}

The set of states of the discrete event system that models the complete
manufacturing process $Q$ will be a subset of the set of states comprised by
the combination of all states of the DES of machines $M_1$, $M_2$ and buffer $M$:

$$Q \subseteq Q_{M_1} \times Q_{M_2} \times Q_{B}$$

Table \ref{tbl:04.02_all_states} features the complete set of states that result
from the combination of states of $M_1$, $M_2$ and $B$.
%Here we can see that
%some states do not have a meaning in the process. For example state $(I,I,F)$,
%where both machines are idle but the buffer is full, does not correspond to
%real-life conditions.

\begin{table}[H]\centering
    \begin{tabular}{r|lll}
    state notation of \\ the complete DES & M1         & M2         & B     \\ \hline
    (I,I,E)                            & Idle       & Idle       & Empty \\
    (I,I,F)                            & Idle       & Idle       & Full  \\
    (I,P,E)                            & Idle       & Processing & Empty \\
    (I,P,F)                            & Idle       & Processing & Full  \\
    (I,D,E)                            & Idle       & Down       & Empty \\
    (I,D,F)                            & Idle       & Down       & Full  \\
    (P,I,E)                            & Processing & Idle       & Empty \\
    (P,I,F)                            & Processing & Idle       & Full  \\
    (P,P,E)                            & Processing & Processing & Empty \\
    (P,P,F)                            & Processing & Processing & Full  \\
    (P,D,E)                            & Processing & Down       & Empty \\
    (P,D,F)                            & Processing & Down       & Full  \\
    (D,I,E)                            & Down       & Idle       & Empty \\
    (D,I,F)                            & Down       & Idle       & Full  \\
    (D,P,E)                            & Down       & Processing & Empty \\
    (D,P,F)                            & Down       & Processing & Full  \\
    (D,D,E)                            & Down       & Down       & Empty \\
    (D,D,F)                            & Down       & Down       & Full  \\
    \end{tabular}
    \caption{The complete set of all state combinations between machines $M_1$
      and $M_2$ and buffer $B$.}
    \label{tbl:04.02_all_states}
\end{table}

The set of events of the overall DES is the union of the sets of the three
separate DES:

\begin{align*}
  E &= E_{M_1} \cup E_{M_2} \cup E_{B} = \\
    & \{\texttt{START(1), END(1), BREAKDOWN(1), END(1)}, \\
    &\ \ \texttt{START(2), END(2), BREAKDOWN(2), END(2)}, \\
    &\ \ \texttt{PLACED, TAKEN}\}
\end{align*}

where the number $X$ inside parentheses denotes an event pertaining to machine
$M_X$.

The initial state $q_0$ shall be state $q_0 = (I,I,E)$, that is, both $M_1$ and
$M_2$ are idle, and buffer $B$ is empty. Furthermore, the marked state shall be
the initial state $Q_m \equiv \{q_0\}$.

The transition function $\delta()$ is shown in tables \ref{tbl:04.02_delta_1}
and \ref{tbl:04.02_delta_2}.

\begin{table}[H]\centering
  \begin{tabular}{l|l}
    $\delta$((I,I,E), $\texttt{PLACED})$ = (I,I,F)       & $\delta$((P,I,E), $\texttt{START(2)})$ = (P,P,E)     \\
    $\delta$((I,I,E), $\texttt{START(2)})$ = (I,P,E)     & $\delta$((P,I,E), $\texttt{BREAKDOWN(1)})$ = (D,I,E) \\
    $\delta$((I,I,E), $\texttt{START(1)})$ = (P,I,E)     & $\delta$((P,I,F), $\texttt{END(1)})$ = (I,I,F)       \\
    $\delta$((I,I,F), $\texttt{TAKEN})$ = (I,I,E)        & $\delta$((P,I,F), $\texttt{TAKEN})$ = (P,I,E)        \\
    $\delta$((I,I,F), $\texttt{START(2)})$ = (I,P,F)     & $\delta$((P,I,F), $\texttt{END(2)})$ = (P,P,F)       \\
    $\delta$((I,I,F), $\texttt{START(1)})$ = (P,I,F)     & $\delta$((P,I,F), $\texttt{BREAKDOWN(1)})$ = (D,I,F) \\
    $\delta$((I,P,E), $\texttt{END(2)})$ = (I,I,E)       & $\delta$((P,P,E), $\texttt{END(1)})$ = (I,P,E)       \\
    $\delta$((I,P,E), $\texttt{PLACED})$ = (I,P,F)       & $\delta$((P,P,E), $\texttt{END(2)})$ = (P,I,E)       \\
    $\delta$((I,P,E), $\texttt{BREAKDOWN(2)})$ = (I,D,E) & $\delta$((P,P,E), $\texttt{PLACED})$ = (P,P,F)       \\
    $\delta$((I,P,E), $\texttt{START(1)})$ = (P,P,E)     & $\delta$((P,P,E), $\texttt{BREAKDOWN(2)})$ = (P,D,E) \\
    $\delta$((I,P,F), $\texttt{END(2)})$ = (I,I,F)       & $\delta$((P,P,E), $\texttt{BREAKDOWN(1)})$ = (D,P,E) \\
    $\delta$((I,P,F), $\texttt{TAKEN})$ = (I,P,E)        & $\delta$((P,P,F), $\texttt{END(1)})$ = (I,P,F)       \\
    $\delta$((I,P,F), $\texttt{BREAKDOWN(2)})$ = (I,D,F) & $\delta$((P,P,F), $\texttt{END(2)})$ = (P,I,F)       \\
    $\delta$((I,P,F), $\texttt{START(1)})$ = (P,P,F)     & $\delta$((P,P,F), $\texttt{TAKEN})$ = (P,P,E)        \\
    $\delta$((I,D,E), $\texttt{REPAIR(1)})$ = (I,I,E)    & $\delta$((P,P,F), $\texttt{BREAKDOWN(2)})$ = (P,D,F) \\
    $\delta$((I,D,E), $\texttt{PLACED})$ = (I,D,F)       & $\delta$((P,P,F), $\texttt{BREAKDOWN(1)})$ = (D,P,F) \\
    $\delta$((I,D,E), $\texttt{START(1)})$ = (P,D,E)     & $\delta$((P,D,E), $\texttt{END(1)})$ = (I,D,E)       \\
    $\delta$((I,D,F), $\texttt{REPAIR(2)})$ = (I,I,F)    & $\delta$((P,D,E), $\texttt{REPAIR(2)})$ = (P,I,E)    \\
    $\delta$((I,D,F), $\texttt{TAKEN})$ = (I,D,E)        & $\delta$((P,D,E), $\texttt{PLACED})$ = (P,D,F)       \\
    $\delta$((I,D,F), $\texttt{START(1)})$ = (P,D,F)     & $\delta$((P,D,E), $\texttt{BREAKDOWN(1)})$ = (D,D,E) \\
    $\delta$((P,I,E), $\texttt{END(1)})$ = (I,I,E)       & $\delta$((P,D,F), $\texttt{END(1)})$ = (I,D,F)       \\
    $\delta$((P,I,E), $\texttt{PLACED})$ = (P,I,F)       & $\delta$((P,D,F), $\texttt{REPAIR(2)})$ = (P,I,F)    \\
  \end{tabular}
  \caption{Allowed transitions between states (part I).}
  \label{tbl:04.02_delta_1}
\end{table}

\begin{table}[H]\centering
  \begin{tabular}{l}
    $\delta$((P,D,F), $\texttt{TAKEN})$ = (P,D,E)         \\
    $\delta$((P,D,F), $\texttt{BREAKDOWN(1)})$ = (D,D,F)  \\
    $\delta$((D,I,E), $\texttt{REPAIR(1)})$ = (I,I,E)     \\
    $\delta$((D,I,E), $\texttt{PLACED})$ = (D,I,F)        \\
    $\delta$((D,I,E), $\texttt{START(2)})$ = (D,P,E)      \\
    $\delta$((D,I,F), $\texttt{REPAIR(1)})$ = (I,I,F)     \\
    $\delta$((D,I,F), $\texttt{TAKEN})$ = (D,I,E)         \\
    $\delta$((D,I,F), $\texttt{START(2)})$ = (D,P,F)      \\
    $\delta$((D,P,E), $\texttt{REPAIR(1)})$ = (I,P,E)     \\
    $\delta$((D,P,E), $\texttt{END(2)})$ = (D,I,E)        \\
    $\delta$((D,P,E), $\texttt{PLACED})$ = (D,P,F)        \\
    $\delta$((D,P,E), $\texttt{BREAKDOWN(2)})$ = (D,D,E)  \\
    $\delta$((D,P,F), $\texttt{REPAIR(1)})$ = (I,P,F)     \\
    $\delta$((D,P,F), $\texttt{END(2)})$ = (D,I,F)        \\
    $\delta$((D,P,F), $\texttt{TAKEN})$ = (D,P,E)         \\
    $\delta$((D,P,F), $\texttt{BREAKDOWN(2)})$ = (D,D,F)  \\
    $\delta$((D,D,E), $\texttt{REPAIR(1)})$ = (I,D,E)     \\
    $\delta$((D,D,E), $\texttt{REPAIR(2)})$ = (D,I,E)     \\
    $\delta$((D,D,E), $\texttt{PLACED})$ = (D,D,F)        \\
    $\delta$((D,D,F), $\texttt{REPAIR(1)})$ = (I,D,F)     \\
    $\delta$((D,D,F), $\texttt{REPAIR(2)})$ = (D,I,F)     \\
    $\delta$((D,D,F), $\texttt{TAKEN})$ = (D,D,E)         \\
  \end{tabular}
  \caption{Allowed transitions between states (part II).}
  \label{tbl:04.02_delta_2}
\end{table}

\begin{sidewaysfigure}
  \begin{figure}[H]\centering
    \scalebox{0.8}{\begin{tikzpicture}[->,>=stealth',shorten >=1pt,auto,node distance=4.8cm,
                    semithick]

  \node[initial,state,accepting] (01)           {(I,I,E)};
  \node[state]         (02) [right of=01]       {(I,I,F)};
  \node[state]         (03) [right of=02]       {(I,P,E)};
  \node[state]         (04) [right of=03]       {(I,P,F)};
  \node[state]         (05) [right of=04]       {(I,D,E)};
  \node[state]         (06) [right of=05]       {(I,D,F)};
  \node[state]         (07) [below of=01]       {(P,I,E)};
  \node[state]         (08) [right of=07]       {(P,I,F)};
  \node[state]         (09) [right of=08]       {(P,P,E)};
  \node[state]         (10) [right of=09]       {(P,P,F)};
  \node[state]         (11) [right of=10]       {(P,D,E)};
  \node[state]         (12) [right of=11]       {(P,D,F)};
  \node[state]         (13) [below of=07]       {(D,I,E)};
  \node[state]         (14) [right of=13]       {(D,I,F)};
  \node[state]         (15) [right of=14]       {(D,P,E)};
  \node[state]         (16) [right of=15]       {(D,P,F)};
  \node[state]         (17) [right of=16]       {(D,D,E)};
  \node[state]         (18) [right of=17]       {(D,D,F)};

  \path (01) edge [bend left] node {\texttt{PLACED}}        (02)
        (01) edge [bend left] node {\texttt{START(2)}}      (03)
        (01) edge [bend left] node {\texttt{START(1)}}      (07);
  \path (02) edge [bend left] node {\texttt{TAKEN}}         (01);
        %(02) edge [bend left] node {\texttt{START(2)}}      (04)
        %(02) edge [bend left] node {\texttt{START(1)}}      (08);
  \path (03) edge [bend left] node {\texttt{END(2)}}        (01)
        (03) edge [bend left] node {\texttt{PLACED}}        (04)
        (03) edge [bend left] node {\texttt{BREAKDOWN(2)}}  (05)
        (03) edge [bend left] node {\texttt{START(1)}}      (09);
  \path (04) edge [bend left] node {\texttt{END(2)}}        (02)
        (04) edge [bend left] node {\texttt{TAKEN}}         (03)
        (04) edge [bend left] node {\texttt{BREAKDOWN(2)}}  (06);
        %(04) edge [bend left] node {\texttt{START(1)}}      (10);
  \path (05) edge [bend left] node {\texttt{REPAIR(1)}}     (01)
        (05) edge [bend left] node {\texttt{PLACED}}        (06);
        %(05) edge [bend left] node {\texttt{START(1)}}      (11);
  \path (06) edge [bend right] node {\texttt{REPAIR(2)}}     (02)
        (06) edge [bend left] node {\texttt{TAKEN}}         (05);
        %(06) edge [bend left] node {\texttt{START(1)}}      (12);
  \path (07) edge [bend left] node {\texttt{END(1)}}        (01)
        (07) edge [bend left] node {\texttt{PLACED}}        (08)
        (07) edge [bend left] node {\texttt{START(2)}}      (09)
        (07) edge [bend left] node {\texttt{BREAKDOWN(1)}}  (13);
  \path (08) edge [bend left] node {\texttt{END(1)}}        (02)
        (08) edge [bend left] node {\texttt{TAKEN}}         (07)
        %(08) edge [bend left] node {\texttt{START(2)}}        (10)
        (08) edge [bend left] node {\texttt{BREAKDOWN(1)}}  (14);
  \path (09) edge [bend left] node {\texttt{END(1)}}        (03)
        (09) edge [bend left] node {\texttt{END(2)}}        (07)
        (09) edge [bend left] node {\texttt{PLACED}}        (10)
        (09) edge [bend left] node {\texttt{BREAKDOWN(2)}}  (11)
        (09) edge [bend left] node {\texttt{BREAKDOWN(1)}}  (15);
  \path (10) edge [bend left] node {\texttt{END(1)}}        (04)
        (10) edge [bend left] node {\texttt{END(2)}}        (08)
        (10) edge [bend left] node {\texttt{TAKEN}}         (09)
        (10) edge [bend left] node {\texttt{BREAKDOWN(2)}}  (12)
        (10) edge [bend left] node {\texttt{BREAKDOWN(1)}}  (16);
  \path (11) edge [bend left] node {\texttt{END(1)}}        (05)
        (11) edge [bend left] node {\texttt{REPAIR(2)}}     (07)
        (11) edge [bend left] node {\texttt{PLACED}}        (12)
        (11) edge [bend left] node {\texttt{BREAKDOWN(1)}}  (17);
  \path (12) edge [bend right] node {\texttt{END(1)}}        (06)
        (12) edge [bend left] node {\texttt{REPAIR(2)}}     (08)
        (12) edge [bend left] node {\texttt{TAKEN}}         (11)
        (12) edge [bend left] node {\texttt{BREAKDOWN(1)}}  (18);
  \path (13) edge [bend left] node {\texttt{REPAIR(1)}}     (01)
        (13) edge [bend left] node {\texttt{PLACED}}        (14)
        (13) edge [bend left] node {\texttt{START(2)}}      (15);
  \path (14) edge [bend left] node {\texttt{REPAIR(1)}}     (02)
        (14) edge [bend left] node {\texttt{TAKEN}}         (13);
        %(14) edge [bend left] node {\texttt{START(2)}}      (16);
  \path (15) edge [bend left] node {\texttt{REPAIR(1)}}     (03)
        (15) edge [bend left] node {\texttt{END(2)}}        (13)
        (15) edge [bend left] node {\texttt{PLACED}}        (16)
        (15) edge [bend left] node {\texttt{BREAKDOWN(2)}}  (17);
  \path (16) edge [bend left] node {\texttt{REPAIR(1)}}     (04)
        (16) edge [bend left] node {\texttt{END(2)}}        (14)
        (16) edge [bend left] node {\texttt{TAKEN}}         (15)
        (16) edge [bend left] node {\texttt{BREAKDOWN(2)}}  (18);
  \path %(17) edge [bend left] node {\texttt{REPAIR(1)}}     (05)
        (17) edge [bend left] node {\texttt{REPAIR(2)}}     (13)
        (17) edge [bend left] node {\texttt{PLACED}}        (18);
  \path %(18) edge [bend left] node {\texttt{REPAIR(1)}}     (06)
        (18) edge [bend left] node {\texttt{REPAIR(2)}}     (14)
        (18) edge [bend left] node {\texttt{TAKEN}}         (17);
\end{tikzpicture}
}
    \caption{}
    \label{}
  \end{figure}
\end{sidewaysfigure}
