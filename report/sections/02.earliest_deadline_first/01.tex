\subsubsection{Question 1}

In contrast to Rate Monitoring scheduling, Earliest Deadline First scheduling
assigns dynamic priorities to tasks, introducing a degree of flexibility. The
task whose deadline is closest to the current timestep is given the highest
priority and is executed for one time unit.

EDF is more lenient than RM, and this can be seen in the condition that
identifies it:
$$U \leq 1 \Leftrightarrow \sigma \text{ is feasible}$$
whereas in RM
$$U \leq n (2^{1/n}-1)\ (< 1 \text{ for } n \geq 2) \Rightarrow \sigma \text{ is feasible}$$

This means that if $U \leq 1$ for a certain collection of tasks $\{J_i\}$,
there is always a feasible schedule for $\{J_i\}$, where the processor is
utilized to the fullest it can be, whereas the former certainty does not hold
with RM. However, if certain tasks are indeed more important than others and
there are extra requirements per their execution, it is possible that EDF
introduces delays between their release and start times due to the
indiscrimination it shows to absolute task priorities.
