\subsubsection{Question 6}

Figures \ref{fig:02.6.6.1}, \ref{fig:02.6.6.2}, \ref{fig:02.6.6.3},
\ref{fig:02.6.10.1}, and \ref{fig:02.6.10.2} feature the comparison of the
course of the angular displacement of every stable pendulum, between the two
considered scheduling policies. Figures \ref{fig:02.6.6.1_diff},
\ref{fig:02.6.6.2_diff}, \ref{fig:02.6.6.3_diff}, \ref{fig:02.6.10.1_diff} and
\ref{fig:02.6.10.2_diff} graphically illustrate the evolution of the difference
in angular displacement between the two considered scheduling policies for the
two considered execution times for all stable penduli.

With execution time $C_i = 6$ ms the behaviour of all three penduli is nearly
identical with respect to the two different scheduling policies. Scheduling
under EDF makes the system have a higher overshoot but its peak difference from
that of RM's is by an order of magnitude of $10^{-3}$ degrees. Hence their
difference is negligible.

With execution time $C_i = 10$ ms there are two things to consider. First that
pendulum $P_3$ is not controllable under RM but it is under EDF. Hence, if the
purpose at hand is to control all three penduli, EDF is conclusively the only
choice there is. On the other hand, though, EDF makes the system behave
oscillatory in a degree larger than RM, with higher overshoot, rise and settling
times.

\noindent\makebox[\textwidth][c]{%
\begin{minipage}{\linewidth}
  \begin{minipage}{0.45\linewidth}
    \begin{figure}[H]\centering
      \scalebox{0.7}{% This file was created by matlab2tikz.
%
%The latest updates can be retrieved from
%  http://www.mathworks.com/matlabcentral/fileexchange/22022-matlab2tikz-matlab2tikz
%where you can also make suggestions and rate matlab2tikz.
%
\definecolor{mycolor1}{rgb}{0.00000,0.44700,0.74100}%
\definecolor{mycolor2}{rgb}{0.85000,0.32500,0.09800}%
%
\begin{tikzpicture}

\begin{axis}[%
width=4.133in,
height=3.26in,
at={(0.693in,0.44in)},
scale only axis,
xmin=0,
xmax=1,
xmajorgrids,
ymin=-0.1,
ymax=0.2,
ymajorgrids,
axis background/.style={fill=white}
]
\pgfplotsset{max space between ticks=50}
\addplot [color=mycolor1,solid,forget plot]
  table[row sep=crcr]{%
0	0.15314\\
3.15544362088405e-30	0.15314\\
0.000656101980281985	0.153143230512962\\
0.00393661188169191	0.153256312778436\\
0.00599999999999994	0.153410244700375\\
0.006	0.153410244700375\\
0.012	0.152025843789547\\
0.0120000000000001	0.152025843789547\\
0.018	0.146785790333147\\
0.0180000000000001	0.146785790333147\\
0.0199999999999998	0.144179493489919\\
0.02	0.144179493489918\\
0.026	0.133767457951154\\
0.0260000000000002	0.133767457951153\\
0.0289999999999998	0.127361824092399\\
0.029	0.127361824092398\\
0.0319999999999996	0.120505893144387\\
0.0349999999999991	0.113193617075476\\
0.035	0.113193617075474\\
0.0399999999999996	0.0999746854104061\\
0.04	0.0999746854104049\\
0.0449999999999996	0.0854376461080997\\
0.0459999999999996	0.0823689879834041\\
0.046	0.0823689879834027\\
0.047	0.0792808135350665\\
0.0470000000000004	0.0792808135350652\\
0.0490000000000003	0.0731494760963212\\
0.0510000000000002	0.0670761531370235\\
0.055	0.0550940482536636\\
0.0579999999999996	0.0462422041576017\\
0.058	0.0462422041576004\\
0.0599999999999996	0.0404007870498711\\
0.06	0.0404007870498698\\
0.0619999999999995	0.0346045456083864\\
0.0639999999999991	0.0288512070597887\\
0.0659999999999991	0.0231385157858626\\
0.066	0.02313851578586\\
0.0699999999999991	0.0121295381531389\\
0.07	0.0121295381531366\\
0.0700000000000009	0.0121295381531342\\
0.074	0.00186375329871775\\
0.076	-0.00299551806340728\\
0.0760000000000009	-0.0029955180634094\\
0.08	-0.0121763447641911\\
0.0800000000000009	-0.0121763447641931\\
0.0839999999999999	-0.0206520990256193\\
0.086	-0.0246297706860801\\
0.0860000000000009	-0.0246297706860818\\
0.0869999999999991	-0.0265502128214379\\
0.087	-0.0265502128214396\\
0.0880000000000004	-0.0284199307246156\\
0.0890000000000009	-0.0302391076551602\\
0.0910000000000017	-0.0337265468553195\\
0.0929999999999991	-0.0370138976863233\\
0.093	-0.0370138976863247\\
0.0970000000000017	-0.0429934114909054\\
0.0999999999999991	-0.0469618347664145\\
0.1	-0.0469618347664156\\
0.104000000000002	-0.0515714390916565\\
0.104999999999999	-0.0526028899776093\\
0.105	-0.0526028899776102\\
0.105999999999999	-0.053586170072105\\
0.106	-0.0535861700721058\\
0.106999999999999	-0.05452585761117\\
0.107999999999998	-0.0554265265666187\\
0.109999999999997	-0.0571111580107877\\
0.111999999999999	-0.058640724871287\\
0.112	-0.0586407248712876\\
0.115999999999997	-0.0612370029316692\\
0.115999999999998	-0.0612370029316702\\
0.116	-0.0612370029316711\\
0.119999999999997	-0.063219432648722\\
0.119999999999998	-0.0632194326487227\\
0.12	-0.0632194326487234\\
0.123999999999997	-0.0645911231830114\\
0.125999999999999	-0.0650486508548064\\
0.126	-0.0650486508548066\\
0.127999999999998	-0.0653835388327254\\
0.128	-0.0653835388327257\\
0.129999999999998	-0.0656252314014744\\
0.131999999999996	-0.0657738233185884\\
0.135999999999993	-0.0657919017428447\\
0.139999999999998	-0.0654378024812815\\
0.14	-0.0654378024812812\\
0.144999999999998	-0.0644709017753106\\
0.145	-0.0644709017753102\\
0.145999999999998	-0.0642074456912333\\
0.146	-0.0642074456912328\\
0.146999999999999	-0.0639273473319215\\
0.147999999999998	-0.0636373505175543\\
0.149999999999997	-0.0630275468598978\\
0.151999999999998	-0.0623777956425841\\
0.152	-0.0623777956425835\\
0.155999999999997	-0.0609574158314218\\
0.157999999999998	-0.0601862303730122\\
0.158	-0.0601862303730115\\
0.16	-0.0593739834145983\\
0.160000000000002	-0.0593739834145976\\
0.162000000000002	-0.0585203565157101\\
0.164000000000002	-0.0576250150091663\\
0.166	-0.0566876078731048\\
0.166000000000002	-0.056687607873104\\
0.170000000000002	-0.0547623883316001\\
0.174	-0.0528189567458071\\
0.174000000000001	-0.0528189567458063\\
0.175	-0.0523298967241606\\
0.175000000000002	-0.0523298967241597\\
0.176000000000001	-0.0518394601814495\\
0.177	-0.051347599048052\\
0.178999999999998	-0.0503594100348185\\
0.179999999999998	-0.0498629853004126\\
0.18	-0.0498629853004117\\
0.183999999999997	-0.0478606143562568\\
0.186	-0.0468487381919073\\
0.186000000000002	-0.0468487381919064\\
0.189999999999998	-0.0448460616191702\\
0.192	-0.0438655851705541\\
0.192000000000002	-0.0438655851705532\\
0.195999999999998	-0.0419444338237903\\
0.199999999999995	-0.0400738489287693\\
0.199999999999997	-0.0400738489287682\\
0.2	-0.040073848928767\\
0.202999999999998	-0.0387023267293148\\
0.203	-0.038702326729314\\
0.205999999999998	-0.0373563865838793\\
0.206	-0.0373563865838785\\
0.208999999999998	-0.0360456737975151\\
0.209999999999998	-0.0356188041094374\\
0.21	-0.0356188041094366\\
0.211999999999998	-0.0347798648300974\\
0.212	-0.0347798648300967\\
0.213999999999998	-0.0339603857274477\\
0.215999999999997	-0.0331600455233567\\
0.217999999999998	-0.0323785304413606\\
0.218	-0.0323785304413599\\
0.219999999999998	-0.031615534089747\\
0.22	-0.0316155340897463\\
0.221999999999998	-0.0308707573368921\\
0.223999999999996	-0.030143908190053\\
0.225999999999998	-0.0294347016852164\\
0.226	-0.0294347016852158\\
0.229999999999996	-0.0280719343105698\\
0.231999999999998	-0.0274187948368041\\
0.232	-0.0274187948368036\\
0.235999999999996	-0.0261677239027614\\
0.237999999999998	-0.0255693019550791\\
0.238	-0.0255693019550785\\
0.239999999999998	-0.0249886405204315\\
0.24	-0.024988640520431\\
0.241999999999998	-0.0244255119513503\\
0.243999999999996	-0.0238796954709565\\
0.245	-0.0236132121289339\\
0.245000000000002	-0.0236132121289334\\
0.245999999999998	-0.0233509770892397\\
0.246	-0.0233509770892393\\
0.246999999999999	-0.0230927460166266\\
0.247999999999998	-0.0228382749662833\\
0.249999999999997	-0.022340513535244\\
0.252	-0.021857497646821\\
0.252000000000003	-0.0218574976468202\\
0.256	-0.020934950743504\\
0.259999999999997	-0.020069187370574\\
0.26	-0.0200691873705733\\
0.260999999999996	-0.019861457897877\\
0.261	-0.0198614578978763\\
0.261999999999998	-0.0196571717576136\\
0.262999999999996	-0.0194563089176112\\
0.264999999999993	-0.0190647747253879\\
0.265999999999997	-0.0188740649980124\\
0.266	-0.0188740649980117\\
0.269999999999993	-0.0181390028091065\\
0.271999999999997	-0.0177870605763722\\
0.272	-0.0177870605763716\\
0.275999999999993	-0.0171136638873358\\
0.279999999999986	-0.0164800440753242\\
0.279999999999993	-0.0164800440753232\\
0.28	-0.0164800440753221\\
0.285999999999996	-0.015602038849948\\
0.286	-0.0156020388499475\\
0.289999999999996	-0.0150588753922395\\
0.29	-0.015058875392239\\
0.293999999999996	-0.0145427244440733\\
0.295999999999997	-0.0142945240673435\\
0.296	-0.0142945240673431\\
0.297999999999997	-0.0140527764903384\\
0.298	-0.014052776490338\\
0.299999999999997	-0.0138173869362699\\
0.3	-0.0138173869362695\\
0.301999999999997	-0.013588263119703\\
0.303999999999993	-0.013365315212369\\
0.305999999999997	-0.0131484558066773\\
0.306	-0.0131484558066769\\
0.309999999999993	-0.0127295733960827\\
0.313999999999986	-0.0123278675486211\\
0.314999999999997	-0.0122300505122881\\
0.315	-0.0122300505122877\\
0.318999999999997	-0.0118489324500033\\
0.319	-0.011848932450003\\
0.319999999999996	-0.0117561437676456\\
0.32	-0.0117561437676453\\
0.320999999999998	-0.0116643330830954\\
0.321999999999996	-0.0115734913976006\\
0.323999999999993	-0.0113946795041558\\
0.325999999999996	-0.0112196379834197\\
0.326	-0.0112196379834194\\
0.329999999999993	-0.0108791202968776\\
0.331	-0.0107957814892731\\
0.331000000000004	-0.0107957814892728\\
0.333	-0.0106311951898568\\
0.333000000000004	-0.0106311951898565\\
0.335	-0.0104693436533348\\
0.336999999999996	-0.0103101634251705\\
0.339999999999996	-0.0100762655513816\\
0.34	-0.0100762655513813\\
0.343999999999993	-0.00977317734588367\\
0.345999999999997	-0.00962526027168728\\
0.346	-0.00962526027168701\\
0.347999999999997	-0.0094796035653446\\
0.348	-0.00947960356534435\\
0.349999999999997	-0.00933607010573833\\
0.35	-0.00933607010573808\\
0.351999999999997	-0.00919460362000974\\
0.353999999999993	-0.00905514864605245\\
0.354	-0.00905514864605196\\
0.357999999999993	-0.00878205530667275\\
0.359999999999996	-0.00864830987431721\\
0.36	-0.00864830987431697\\
0.363999999999993	-0.00838615928986414\\
0.365999999999996	-0.0082576513606088\\
0.366	-0.00825765136060858\\
0.369999999999993	-0.00800583848808932\\
0.373999999999986	-0.00776084841505878\\
0.376999999999997	-0.00758135178724119\\
0.377	-0.00758135178724098\\
0.379999999999997	-0.00740531830369136\\
0.38	-0.00740531830369115\\
0.382999999999996	-0.0072325926860709\\
0.384999999999997	-0.00711920461136385\\
0.385	-0.00711920461136365\\
0.385999999999997	-0.00706302255390143\\
0.386	-0.00706302255390123\\
0.386999999999998	-0.00700720978319619\\
0.387999999999996	-0.0069517961486822\\
0.388999999999997	-0.00689677621923768\\
0.389	-0.00689677621923748\\
0.390999999999997	-0.00678789594355975\\
0.392999999999993	-0.00668052626950994\\
0.394999999999997	-0.00657462510239691\\
0.395	-0.00657462510239673\\
0.398999999999993	-0.00636706277512728\\
0.399999999999997	-0.00631602581069908\\
0.4	-0.0063160258106989\\
0.403999999999993	-0.00611514234622662\\
0.405999999999997	-0.00601659044848844\\
0.406	-0.00601659044848827\\
0.409999999999993	-0.00582364296420372\\
0.411999999999997	-0.00572931417869244\\
0.412	-0.00572931417869227\\
0.415999999999993	-0.00554476159888032\\
0.419999999999986	-0.00536544202393063\\
0.419999999999996	-0.00536544202393015\\
0.42	-0.00536544202392999\\
0.426	-0.00510566147102732\\
0.426000000000004	-0.00510566147102717\\
0.432000000000004	-0.00485715183959725\\
0.432000000000007	-0.00485715183959711\\
0.434999999999997	-0.00473723959737629\\
0.435	-0.00473723959737615\\
0.43799999999999	-0.00462008052529514\\
0.439999999999997	-0.00454345288002267\\
0.44	-0.00454345288002254\\
0.44299999999999	-0.00443065390890581\\
0.445999999999979	-0.00432033765129838\\
0.445999999999995	-0.00432033765129781\\
0.446	-0.00432033765129762\\
0.447	-0.00428412121621806\\
0.447000000000004	-0.00428412121621793\\
0.448000000000004	-0.00424820789644331\\
0.449000000000004	-0.00421259417148631\\
0.451000000000004	-0.00414225157298479\\
0.454999999999997	-0.00400500985662403\\
0.455	-0.00400500985662391\\
0.459	-0.00387218082291413\\
0.459999999999997	-0.00383963848575863\\
0.46	-0.00383963848575852\\
0.463999999999997	-0.00371203313013307\\
0.464	-0.00371203313013296\\
0.465999999999997	-0.00364972516082666\\
0.466	-0.00364972516082655\\
0.467999999999996	-0.00358843606042134\\
0.469999999999993	-0.00352819689626121\\
0.471999999999997	-0.00346898405135086\\
0.472	-0.00346898405135076\\
0.473	-0.00343975522865681\\
0.473000000000004	-0.0034397552286567\\
0.474000000000004	-0.00341077431143422\\
0.475000000000004	-0.00338203845914245\\
0.477000000000004	-0.0033252907074444\\
0.479999999999997	-0.00324193742241739\\
0.48	-0.00324193742241729\\
0.484	-0.00313398671204486\\
0.485999999999997	-0.00308132950301097\\
0.486	-0.00308132950301088\\
0.489999999999997	-0.00297868852361778\\
0.49	-0.00297868852361769\\
0.492999999999997	-0.00290402940427869\\
0.493	-0.0029040294042786\\
0.495999999999997	-0.00283128690125538\\
0.498999999999993	-0.00276039684460079\\
0.499	-0.00276039684460062\\
0.499999999999997	-0.00273716762542179\\
0.5	-0.00273716762542171\\
0.500999999999998	-0.00271413499589151\\
0.501999999999997	-0.00269129669818229\\
0.503999999999993	-0.00264619416349113\\
0.505999999999993	-0.00260184233872669\\
0.506	-0.00260184233872653\\
0.507999999999993	-0.00255824664999363\\
0.508	-0.00255824664999347\\
0.509999999999993	-0.00251541281983331\\
0.511999999999986	-0.00247332405507247\\
0.515999999999972	-0.00239131600358102\\
0.519999999999993	-0.00231209387396655\\
0.52	-0.00231209387396641\\
0.521999999999993	-0.00227348853762208\\
0.522	-0.00227348853762195\\
0.523999999999993	-0.00223553342237412\\
0.524999999999993	-0.00221679503685007\\
0.525	-0.00221679503684994\\
0.525999999999993	-0.00219821364546351\\
0.526	-0.00219821364546338\\
0.526999999999998	-0.00217979147657175\\
0.527999999999997	-0.00216153077413183\\
0.529999999999993	-0.0021254866253021\\
0.531999999999993	-0.00209006706772744\\
0.532	-0.00209006706772732\\
0.535999999999993	-0.00202104642112781\\
0.538	-0.00198741827228713\\
0.538000000000007	-0.00198741827228701\\
0.539999999999993	-0.00195436058493492\\
0.54	-0.0019543605849348\\
0.541999999999986	-0.0019218603988513\\
0.543999999999972	-0.00188990497220944\\
0.545999999999993	-0.00185848177676129\\
0.546	-0.00185848177676118\\
0.549999999999972	-0.00179723505039184\\
0.550999999999993	-0.00178225327967243\\
0.551	-0.00178225327967232\\
0.554999999999972	-0.00172360178577363\\
0.556999999999993	-0.00169502131546648\\
0.557	-0.00169502131546638\\
0.559999999999993	-0.00165305062286593\\
0.56	-0.00165305062286584\\
0.562999999999993	-0.00161212710515334\\
0.565999999999986	-0.0015722146615644\\
0.565999999999993	-0.0015722146615643\\
0.566	-0.00157221466156421\\
0.571999999999986	-0.00149538666278546\\
0.571999999999993	-0.00149538666278537\\
0.572	-0.00149538666278528\\
0.577999999999986	-0.00142239911146773\\
0.579999999999993	-0.00139887846362935\\
0.58	-0.00139887846362927\\
0.585999999999986	-0.00133061300714212\\
0.585999999999993	-0.00133061300714204\\
0.586	-0.00133061300714195\\
0.591999999999986	-0.00126570205787977\\
0.591999999999993	-0.0012657020578797\\
0.592	-0.00126570205787962\\
0.594999999999993	-0.00123446833080083\\
0.595	-0.00123446833080076\\
0.597999999999993	-0.001204012128928\\
0.599999999999993	-0.00118412665739079\\
0.6	-0.00118412665739072\\
0.602999999999993	-0.00115490695370076\\
0.605999999999986	-0.00112639458960772\\
0.606	-0.00112639458960759\\
0.606999999999993	-0.00111704573803569\\
0.607	-0.00111704573803563\\
0.607999999999999	-0.00110777666040905\\
0.608999999999997	-0.00109858644810464\\
0.609000000000004	-0.00109858644810457\\
0.611	-0.00108043902415802\\
0.612999999999997	-0.00106259635147641\\
0.614999999999997	-0.00104505143477157\\
0.615000000000004	-0.00104505143477151\\
0.618999999999997	-0.00101082746954429\\
0.619999999999993	-0.00100244696549607\\
0.62	-0.00100244696549601\\
0.623999999999993	-0.000969602255118335\\
0.625999999999993	-0.000953575039215723\\
0.626	-0.000953575039215667\\
0.629999999999993	-0.000922314872334188\\
0.63	-0.000922314872334133\\
0.633999999999993	-0.000892096646802706\\
0.635999999999993	-0.00087736338990332\\
0.636	-0.000877363389903268\\
0.637999999999993	-0.00086287296953141\\
0.638	-0.000862872969531359\\
0.639999999999993	-0.000848619704738106\\
0.64	-0.000848619704738056\\
0.641999999999993	-0.000834598007474457\\
0.643999999999986	-0.000820802380517159\\
0.645999999999993	-0.000807227415237722\\
0.646	-0.000807227415237674\\
0.649999999999986	-0.000780749480107166\\
0.65	-0.000780749480107074\\
0.650000000000007	-0.000780749480107028\\
0.653999999999993	-0.000755153889262814\\
0.657999999999979	-0.000730400499641502\\
0.659999999999993	-0.00071832744769925\\
0.66	-0.000718327447699207\\
0.664999999999993	-0.00068899255537763\\
0.665	-0.00068899255537759\\
0.665999999999993	-0.000683266293745905\\
0.666	-0.000683266293745864\\
0.666999999999998	-0.000677587299523604\\
0.667000000000006	-0.000677587299523564\\
0.668000000000004	-0.000671956698024208\\
0.669000000000002	-0.000666373937390249\\
0.670999999999998	-0.00065534975467938\\
0.673000000000005	-0.000644510429317838\\
0.673000000000013	-0.0006445104293178\\
0.677000000000005	-0.000623369423314114\\
0.678	-0.00061819315304099\\
0.678000000000007	-0.000618193153040954\\
0.679999999999993	-0.000607967823786801\\
0.68	-0.000607967823786765\\
0.681999999999986	-0.000597908775795358\\
0.683999999999972	-0.000588012065377077\\
0.686	-0.00057827381248852\\
0.686000000000007	-0.000578273812488486\\
0.689999999999979	-0.000559280295869667\\
0.69399999999995	-0.000540921264653169\\
0.695999999999993	-0.000531970644914858\\
0.696	-0.000531970644914827\\
0.699999999999993	-0.000514509651731863\\
0.7	-0.000514509651731833\\
0.703999999999993	-0.000497612928330438\\
0.705999999999993	-0.000489367852763903\\
0.706	-0.000489367852763874\\
0.707999999999993	-0.000481258764641516\\
0.708	-0.000481258764641487\\
0.709999999999993	-0.000473287275702884\\
0.711999999999986	-0.000465450260694147\\
0.713999999999993	-0.000457744647083473\\
0.714	-0.000457744647083446\\
0.717999999999986	-0.00044271559051157\\
0.719999999999993	-0.000435386255349687\\
0.72	-0.000435386255349661\\
0.723999999999986	-0.000421083602853629\\
0.724999999999993	-0.000417580061591119\\
0.725	-0.000417580061591095\\
0.725999999999993	-0.000414104678101229\\
0.726	-0.000414104678101205\\
0.726999999999999	-0.000410658112946585\\
0.727999999999997	-0.000407241029488636\\
0.729999999999993	-0.000400493970880571\\
0.731999999999993	-0.000393860857066824\\
0.732	-0.0003938608570668\\
0.734999999999993	-0.000384119156214144\\
0.735	-0.000384119156214121\\
0.737999999999993	-0.000374619396602752\\
0.74	-0.000368416488381077\\
0.740000000000007	-0.000368416488381055\\
0.743	-0.000359301443482657\\
0.745999999999993	-0.000350406446841923\\
0.746000000000007	-0.000350406446841882\\
0.746999999999993	-0.000347489822750103\\
0.747	-0.000347489822750083\\
0.747999999999999	-0.000344598163069743\\
0.748999999999997	-0.000341731184339105\\
0.750999999999993	-0.000336070148143451\\
0.753999999999993	-0.00032775675327346\\
0.754	-0.000327756753273441\\
0.757999999999993	-0.000316993431197263\\
0.759999999999993	-0.000311744671891618\\
0.76	-0.000311744671891599\\
0.763999999999993	-0.00030150266708328\\
0.766	-0.000296505406162441\\
0.766000000000007	-0.000296505406162424\\
0.77	-0.000286759795580549\\
0.770000000000007	-0.000286759795580532\\
0.774	-0.000277340867113982\\
0.776	-0.000272749272956306\\
0.776000000000007	-0.00027274927295629\\
0.779999999999993	-0.000263792823436101\\
0.78	-0.000263792823436085\\
0.782999999999993	-0.000257266983859888\\
0.783	-0.000257266983859873\\
0.785999999999993	-0.000250898875352722\\
0.786000000000001	-0.000250898875352707\\
0.788999999999994	-0.000244688191897789\\
0.791999999999987	-0.000238634766356024\\
0.792	-0.000238634766355996\\
0.792000000000008	-0.000238634766355982\\
0.797999999999994	-0.000226978462881162\\
0.799999999999993	-0.000223220926701678\\
0.8	-0.000223220926701665\\
0.804999999999993	-0.000214092868699917\\
0.805000000000001	-0.000214092868699904\\
0.805999999999993	-0.00021231138540505\\
0.806	-0.000212311385405038\\
0.806999999999994	-0.000210544699048974\\
0.807999999999987	-0.000208793134689887\\
0.809999999999973	-0.000205334686741833\\
0.811999999999993	-0.00020193468566305\\
0.812	-0.000201934685663038\\
0.815999999999973	-0.000195304714652854\\
0.817999999999993	-0.000192072145415192\\
0.818000000000001	-0.000192072145415181\\
0.819999999999993	-0.000188892823460906\\
0.82	-0.000188892823460895\\
0.821999999999993	-0.000185765502341274\\
0.823999999999986	-0.000182688955977668\\
0.825999999999993	-0.000179661978198278\\
0.826	-0.000179661978198267\\
0.829999999999986	-0.000173758748879205\\
0.833999999999972	-0.00016805330439474\\
0.839999999999993	-0.000159846491338248\\
0.84	-0.000159846491338238\\
0.840999999999993	-0.000158517949099927\\
0.841000000000001	-0.000158517949099918\\
0.841999999999994	-0.000157200273203245\\
0.842999999999987	-0.000155893334441578\\
0.844999999999973	-0.000153311156988693\\
0.845999999999993	-0.000152035665212466\\
0.846	-0.000152035665212457\\
0.849999999999973	-0.000147040493086817\\
0.851999999999993	-0.000144606132107409\\
0.852	-0.000144606132107401\\
0.855999999999973	-0.000139859087997034\\
0.857999999999993	-0.000137544543768725\\
0.858	-0.000137544543768717\\
0.86	-0.000135268103104821\\
0.860000000000007	-0.000135268103104813\\
0.862000000000007	-0.000133028873529955\\
0.864000000000007	-0.000130825977145306\\
0.866	-0.000128658550296913\\
0.866000000000007	-0.000128658550296905\\
0.87	-0.00012443157155467\\
0.870000000000007	-0.000124431571554662\\
0.874	-0.00012034615959215\\
0.874999999999994	-0.000119346170807423\\
0.875000000000001	-0.000119346170807415\\
0.876	-0.000118354530440419\\
0.876000000000007	-0.000118354530440412\\
0.877000000000007	-0.000117371141303731\\
0.878000000000006	-0.000116395907011291\\
0.879999999999998	-0.000114469521416621\\
0.880000000000006	-0.000114469521416614\\
0.882000000000005	-0.00011257461841458\\
0.884000000000004	-0.000110710455106177\\
0.886000000000005	-0.000108876300639045\\
0.886000000000013	-0.000108876300639038\\
0.888000000000007	-0.000107072464325324\\
0.888000000000014	-0.000107072464325317\\
0.890000000000009	-0.000105299267363138\\
0.892000000000004	-0.000103556014563583\\
0.895999999999993	-0.000100156619146666\\
0.898999999999993	-9.76806754765567e-05\\
0.899000000000001	-9.76806754765509e-05\\
0.899999999999993	-9.68689464247173e-05\\
0.9	-9.68689464247115e-05\\
0.900999999999994	-9.60638770080667e-05\\
0.901999999999987	-9.52653882843476e-05\\
0.903999999999973	-9.36878405121959e-05\\
0.905999999999993	-9.21356846245868e-05\\
0.906	-9.21356846245813e-05\\
0.909999999999973	-8.91086105786474e-05\\
0.909999999999987	-8.91086105786371e-05\\
0.910000000000001	-8.91086105786269e-05\\
0.910999999999993	-8.83678627950749e-05\\
0.911000000000001	-8.83678627950696e-05\\
0.911999999999994	-8.76333771689773e-05\\
0.912999999999987	-8.69050817103412e-05\\
0.914999999999973	-8.54667763730249e-05\\
0.916999999999993	-8.40523828899803e-05\\
0.917000000000001	-8.40523828899753e-05\\
0.919999999999993	-8.19744169124745e-05\\
0.92	-8.19744169124697e-05\\
0.922999999999993	-7.99471740251249e-05\\
0.925999999999986	-7.79688658915029e-05\\
0.925999999999993	-7.79688658914982e-05\\
0.926	-7.79688658914935e-05\\
0.927999999999993	-7.66770543173713e-05\\
0.928000000000001	-7.66770543173668e-05\\
0.929999999999994	-7.54071848162338e-05\\
0.931999999999987	-7.41587595094216e-05\\
0.933999999999994	-7.29312889469987e-05\\
0.934000000000001	-7.29312889469944e-05\\
0.937999999999987	-7.05372951732963e-05\\
0.939999999999993	-6.93698333873553e-05\\
0.940000000000001	-6.93698333873512e-05\\
0.943999999999987	-6.70916913265192e-05\\
0.944999999999994	-6.65336590015102e-05\\
0.945000000000001	-6.65336590015063e-05\\
0.945999999999993	-6.59801178959441e-05\\
0.946000000000001	-6.59801178959402e-05\\
0.946999999999994	-6.54311699460332e-05\\
0.947999999999987	-6.4886917534072e-05\\
0.949999999999973	-6.38122864093837e-05\\
0.952	-6.27558032086044e-05\\
0.952000000000008	-6.27558032086007e-05\\
0.95599999999998	-6.06956307635859e-05\\
0.956999999999994	-6.01912913038984e-05\\
0.957000000000001	-6.01912913038949e-05\\
0.96	-5.87031691039845e-05\\
0.960000000000008	-5.87031691039811e-05\\
0.963000000000007	-5.7251374393749e-05\\
0.966000000000007	-5.5834626469541e-05\\
0.966000000000014	-5.58346264695377e-05\\
0.969000000000007	-5.44528649761358e-05\\
0.969000000000014	-5.44528649761326e-05\\
0.972000000000007	-5.31060604792873e-05\\
0.975	-5.17930248741832e-05\\
0.979999999999994	-4.96765412407009e-05\\
0.980000000000001	-4.96765412406979e-05\\
0.985999999999987	-4.72490603426931e-05\\
0.986000000000001	-4.72490603426876e-05\\
0.991999999999987	-4.49400389565575e-05\\
0.992000000000001	-4.49400389565522e-05\\
0.997999999999987	-4.2745353414389e-05\\
0.998000000000001	-4.2745353414384e-05\\
0.999999999999993	-4.20378496315467e-05\\
1	-4.20378496315442e-05\\
1.00199999999999	-4.13419087183369e-05\\
1.00399999999999	-4.06572578167357e-05\\
1.00599999999999	-3.99836285067314e-05\\
1.006	-3.99836285067267e-05\\
1.00999999999999	-3.86698957830971e-05\\
1.01399999999997	-3.7400163425386e-05\\
1.01499999999999	-3.70893705411058e-05\\
1.015	-3.70893705411014e-05\\
1.01999999999999	-3.55737274406711e-05\\
1.02	-3.55737274406669e-05\\
1.02499999999999	-3.41192521045063e-05\\
1.02599999999999	-3.38353809429035e-05\\
1.026	-3.38353809428995e-05\\
1.02699999999999	-3.35538656362035e-05\\
1.027	-3.35538656361996e-05\\
1.02799999999999	-3.32747585970089e-05\\
1.02899999999999	-3.29980324674835e-05\\
1.03099999999997	-3.24516146799697e-05\\
1.03299999999999	-3.19143980484644e-05\\
1.033	-3.19143980484606e-05\\
1.03699999999997	-3.08667293197848e-05\\
1.04	-3.01035897688743e-05\\
1.04000000000001	-3.01035897688707e-05\\
1.04399999999999	-2.91149397143452e-05\\
1.044	-2.91149397143417e-05\\
1.046	-2.86325503086822e-05\\
1.04600000000001	-2.86325503086788e-05\\
1.04800000000001	-2.81581355412429e-05\\
1.05	-2.76917801641151e-05\\
1.05000000000001	-2.76917801641118e-05\\
1.05200000000001	-2.72333013420966e-05\\
1.05200000000002	-2.72333013420933e-05\\
1.05400000000002	-2.6782519328018e-05\\
1.05600000000002	-2.63392573922493e-05\\
1.05800000000001	-2.59033417522502e-05\\
1.05800000000002	-2.59033417522472e-05\\
1.05999999999999	-2.54746015079691e-05\\
1.06	-2.54746015079661e-05\\
1.06199999999996	-2.50528685723951e-05\\
1.06399999999992	-2.46379776034624e-05\\
1.06599999999999	-2.4229765941555e-05\\
1.066	-2.42297659415522e-05\\
1.06999999999992	-2.34336593969743e-05\\
1.07299999999999	-2.28541444037821e-05\\
1.073	-2.28541444037794e-05\\
1.07699999999992	-2.21039056379277e-05\\
1.07899999999999	-2.17380751785288e-05\\
1.079	-2.17380751785262e-05\\
1.07999999999999	-2.15574190380294e-05\\
1.08	-2.15574190380269e-05\\
1.08099999999999	-2.13782453005339e-05\\
1.08199999999999	-2.12005364044807e-05\\
1.08399999999997	-2.0849443609276e-05\\
1.08499999999999	-2.06760252987325e-05\\
1.085	-2.06760252987301e-05\\
1.08599999999999	-2.05040030047781e-05\\
1.086	-2.05040030047756e-05\\
1.08699999999999	-2.03334083321418e-05\\
1.08799999999999	-2.01642730242068e-05\\
1.08999999999997	-1.98303143357649e-05\\
1.09199999999999	-1.95019960096621e-05\\
1.092	-1.95019960096598e-05\\
1.09599999999997	-1.88617677393808e-05\\
1.09999999999995	-1.82425841049671e-05\\
1.09999999999997	-1.82425841049629e-05\\
1.1	-1.82425841049587e-05\\
1.10199999999999	-1.79405792911033e-05\\
1.102	-1.79405792911012e-05\\
1.10399999999999	-1.76434739863176e-05\\
1.10599999999997	-1.73511516913527e-05\\
1.10599999999999	-1.73511516913506e-05\\
1.106	-1.73511516913485e-05\\
1.10999999999997	-1.67810558343024e-05\\
1.11199999999999	-1.65032228222119e-05\\
1.112	-1.650322282221e-05\\
1.11599999999997	-1.59614419673838e-05\\
1.11999999999994	-1.54374695865326e-05\\
1.12	-1.54374695865252e-05\\
1.12000000000001	-1.54374695865234e-05\\
1.126	-1.46831115774926e-05\\
1.12600000000001	-1.46831115774908e-05\\
1.13099999999999	-1.40826238203242e-05\\
1.131	-1.40826238203226e-05\\
1.13599999999997	-1.35070951443084e-05\\
1.13699999999999	-1.33948593391259e-05\\
1.137	-1.33948593391243e-05\\
1.138	-1.32835542934168e-05\\
1.13800000000001	-1.32835542934153e-05\\
1.13900000000001	-1.31731690990796e-05\\
1.14	-1.30636929362533e-05\\
1.14000000000001	-1.30636929362518e-05\\
1.14100000000001	-1.29551150751291e-05\\
1.14200000000001	-1.2847424873952e-05\\
1.14400000000001	-1.26346653183609e-05\\
1.146	-1.24253308562818e-05\\
1.14600000000001	-1.24253308562803e-05\\
1.15000000000001	-1.20170802836402e-05\\
1.15400000000001	-1.16225029090269e-05\\
1.15499999999999	-1.15259220208211e-05\\
1.155	-1.15259220208197e-05\\
1.15999999999999	-1.10549259906287e-05\\
1.16	-1.10549259906273e-05\\
1.16499999999999	-1.06029379259719e-05\\
1.16599999999999	-1.05147229995218e-05\\
1.166	-1.05147229995205e-05\\
1.17099999999999	-1.00847077903749e-05\\
1.17199999999999	-1.00008824866871e-05\\
1.172	-1.0000882486686e-05\\
1.173	-9.91776376127359e-06\\
1.17300000000001	-9.91776376127241e-06\\
1.17400000000001	-9.83534346834612e-06\\
1.17500000000001	-9.75361352913734e-06\\
1.17700000000001	-9.59219273721292e-06\\
1.17999999999999	-9.35504109962058e-06\\
1.18	-9.35504109961947e-06\\
1.184	-9.04780999261662e-06\\
1.18599999999999	-8.89790339866367e-06\\
1.186	-8.89790339866261e-06\\
1.18899999999999	-8.67770341137901e-06\\
1.189	-8.67770341137798e-06\\
1.18999999999999	-8.605550877653e-06\\
1.19	-8.60555087765198e-06\\
1.19099999999999	-8.53401023820592e-06\\
1.19199999999999	-8.46307448122891e-06\\
1.19399999999997	-8.32298986351776e-06\\
1.19499999999999	-8.253827272776e-06\\
1.195	-8.25382727277502e-06\\
1.19899999999997	-7.98288418551996e-06\\
1.19999999999999	-7.9165420383979e-06\\
1.2	-7.91654203839696e-06\\
1.20399999999997	-7.65655286685725e-06\\
1.20599999999999	-7.52969692603209e-06\\
1.206	-7.52969692603119e-06\\
1.20699999999999	-7.46704952725967e-06\\
1.207	-7.46704952725878e-06\\
1.20799999999999	-7.40493804190913e-06\\
1.20899999999999	-7.34335638217564e-06\\
1.21099999999997	-7.22175844832158e-06\\
1.21499999999995	-6.98465832657863e-06\\
1.21799999999999	-6.81198437866188e-06\\
1.218	-6.81198437866107e-06\\
1.21999999999999	-6.69923636456424e-06\\
1.22	-6.69923636456344e-06\\
1.22199999999999	-6.58833104067165e-06\\
1.22399999999997	-6.47922491941166e-06\\
1.22499999999999	-6.42533316522846e-06\\
1.225	-6.4253331652277e-06\\
1.22599999999999	-6.37187522519907e-06\\
1.226	-6.37187522519831e-06\\
1.22699999999999	-6.31886092470963e-06\\
1.22799999999999	-6.26630013223346e-06\\
1.22999999999997	-6.162518509245e-06\\
1.23	-6.16251850924362e-06\\
1.23399999999997	-5.96017360953826e-06\\
1.236	-5.86153100298906e-06\\
1.23600000000001	-5.86153100298837e-06\\
1.23999999999999	-5.66911209764873e-06\\
1.24	-5.66911209764806e-06\\
1.24399999999997	-5.48293117039364e-06\\
1.24599999999999	-5.39208832924915e-06\\
1.246	-5.39208832924851e-06\\
1.247	-5.3472258855794e-06\\
1.24700000000001	-5.34722588557876e-06\\
1.24800000000001	-5.30274721588461e-06\\
1.24900000000001	-5.25864796062516e-06\\
1.25100000000001	-5.17157044143064e-06\\
1.253	-5.08595918892056e-06\\
1.25300000000001	-5.08595918891996e-06\\
1.25700000000001	-4.91900179081807e-06\\
1.25999999999999	-4.79738711104327e-06\\
1.26	-4.7973871110427e-06\\
1.264	-4.63983471739208e-06\\
1.266	-4.56296053514804e-06\\
1.26600000000001	-4.56296053514749e-06\\
1.27000000000001	-4.41303812031228e-06\\
1.272	-4.33997426195933e-06\\
1.27200000000001	-4.33997426195882e-06\\
1.276	-4.1974980090753e-06\\
1.27600000000001	-4.1974980090748e-06\\
1.27999999999999	-4.05970495298293e-06\\
1.28000000000001	-4.05970495298245e-06\\
1.28399999999999	-3.92637898476647e-06\\
1.28599999999999	-3.8613255491427e-06\\
1.28600000000001	-3.86132554914224e-06\\
1.288	-3.79734751734759e-06\\
1.28800000000001	-3.79734751734714e-06\\
1.29	-3.73445632307131e-06\\
1.29199999999999	-3.67262730957186e-06\\
1.29499999999999	-3.58182246443054e-06\\
1.295	-3.58182246443011e-06\\
1.29899999999998	-3.4642441154725e-06\\
1.29999999999999	-3.43545430836002e-06\\
1.3	-3.43545430835962e-06\\
1.30399999999998	-3.32262954280066e-06\\
1.30499999999999	-3.29499314455648e-06\\
1.305	-3.29499314455609e-06\\
1.30599999999999	-3.26757921303967e-06\\
1.306	-3.26757921303928e-06\\
1.30699999999999	-3.24039278655083e-06\\
1.30799999999999	-3.21343892548606e-06\\
1.30999999999997	-3.16021835512765e-06\\
1.31199999999999	-3.10789663663485e-06\\
1.312	-3.10789663663448e-06\\
1.31299999999999	-3.08206642307353e-06\\
1.313	-3.08206642307317e-06\\
1.31399999999999	-3.05645325740209e-06\\
1.31499999999999	-3.03105462916777e-06\\
1.31699999999997	-2.98089104847349e-06\\
1.32	-2.90719314528659e-06\\
1.32000000000001	-2.90719314528625e-06\\
1.32399999999999	-2.81171716776449e-06\\
1.326	-2.76513179276059e-06\\
1.32600000000001	-2.76513179276026e-06\\
1.32999999999999	-2.67427955175119e-06\\
1.33	-2.67427955175087e-06\\
1.33399999999997	-2.5864701596918e-06\\
1.334	-2.58647015969125e-06\\
1.33799999999997	-2.50156589997259e-06\\
1.34	-2.46016140980652e-06\\
1.34000000000001	-2.46016140980623e-06\\
1.34399999999999	-2.37936653954714e-06\\
1.346	-2.33994448350719e-06\\
1.34600000000001	-2.33994448350691e-06\\
1.348	-2.30117411703354e-06\\
1.34800000000002	-2.30117411703327e-06\\
1.35	-2.2630623686039e-06\\
1.35199999999999	-2.22559429635624e-06\\
1.35599999999997	-2.15253066939083e-06\\
1.35999999999999	-2.08186864303797e-06\\
1.36	-2.08186864303772e-06\\
1.36299999999999	-2.03038137130965e-06\\
1.363	-2.0303813713094e-06\\
1.36499999999999	-1.99674987847508e-06\\
1.365	-1.99674987847484e-06\\
1.36599999999999	-1.98013717334081e-06\\
1.366	-1.98013717334057e-06\\
1.36699999999999	-1.96366233502023e-06\\
1.36799999999999	-1.94732843012561e-06\\
1.36999999999997	-1.91507703069938e-06\\
1.37199999999999	-1.88337033077675e-06\\
1.372	-1.88337033077653e-06\\
1.37599999999997	-1.82154151594818e-06\\
1.378	-1.79139516081199e-06\\
1.37800000000001	-1.79139516081177e-06\\
1.37999999999999	-1.76174501564124e-06\\
1.38	-1.76174501564103e-06\\
1.38199999999997	-1.73257945613772e-06\\
1.38399999999994	-1.70388704782844e-06\\
1.38599999999999	-1.6756565417437e-06\\
1.386	-1.6756565417435e-06\\
1.38999999999994	-1.62060053378322e-06\\
1.39199999999999	-1.59376929333304e-06\\
1.392	-1.59376929333285e-06\\
1.39599999999994	-1.54144773956234e-06\\
1.39799999999999	-1.51593691336975e-06\\
1.398	-1.51593691336957e-06\\
1.39999999999999	-1.49084599610554e-06\\
1.4	-1.49084599610536e-06\\
1.40199999999999	-1.46616515091044e-06\\
1.40399999999997	-1.44188470156147e-06\\
1.40599999999999	-1.417995128815e-06\\
1.406	-1.41799512881483e-06\\
1.40999999999997	-1.37140494339974e-06\\
1.412	-1.3486994745809e-06\\
1.41200000000001	-1.34869947458074e-06\\
1.41599999999999	-1.30442327294668e-06\\
1.41999999999996	-1.26160243065508e-06\\
1.41999999999998	-1.26160243065486e-06\\
1.42	-1.26160243065463e-06\\
1.42099999999999	-1.25111670074059e-06\\
1.421	-1.25111670074044e-06\\
1.42199999999999	-1.24071669552288e-06\\
1.42299999999999	-1.23040139521421e-06\\
1.42499999999997	-1.21002087349958e-06\\
1.42599999999999	-1.19995365455298e-06\\
1.426	-1.19995365455284e-06\\
1.42999999999997	-1.16052752262025e-06\\
1.43199999999999	-1.14131341554508e-06\\
1.432	-1.14131341554494e-06\\
1.43499999999999	-1.11309472112466e-06\\
1.435	-1.11309472112453e-06\\
1.43799999999998	-1.08557690582272e-06\\
1.43999999999999	-1.06760906050339e-06\\
1.44	-1.06760906050326e-06\\
1.44299999999999	-1.04120572847741e-06\\
1.44599999999997	-1.01543985858412e-06\\
1.44599999999999	-1.01543985858399e-06\\
1.446	-1.01543985858386e-06\\
1.44699999999999	-1.00699134970299e-06\\
1.447	-1.00699134970287e-06\\
1.44799999999999	-9.98615113406417e-07\\
1.44899999999999	-9.90310328594109e-07\\
1.44999999999999	-9.82076181309267e-07\\
1.45	-9.8207618130915e-07\\
1.45199999999999	-9.65816578044685e-07\\
1.45399999999997	-9.49829929014372e-07\\
1.45599999999999	-9.34109966591029e-07\\
1.456	-9.34109966590918e-07\\
1.45999999999997	-9.03445551795481e-07\\
1.46	-9.03445551795266e-07\\
1.46000000000001	-9.03445551795159e-07\\
1.46399999999999	-8.73775240816882e-07\\
1.466	-8.59298273552407e-07\\
1.46600000000001	-8.59298273552305e-07\\
1.46999999999999	-8.31064843041291e-07\\
1.47	-8.31064843041193e-07\\
1.47399999999997	-8.03777013842215e-07\\
1.47599999999999	-7.9047426952453e-07\\
1.476	-7.90474269524436e-07\\
1.47899999999998	-7.70931916184081e-07\\
1.479	-7.7093191618399e-07\\
1.47999999999999	-7.64525042745671e-07\\
1.48	-7.6452504274558e-07\\
1.48099999999999	-7.5817074056425e-07\\
1.48199999999999	-7.51868386844822e-07\\
1.48399999999997	-7.39417059037786e-07\\
1.48599999999999	-7.27166177733579e-07\\
1.486	-7.27166177733492e-07\\
1.48999999999997	-7.03274129112764e-07\\
1.491	-6.97427583679543e-07\\
1.49100000000001	-6.9742758367946e-07\\
1.49499999999999	-6.74530076827283e-07\\
1.49899999999996	-6.52387687655539e-07\\
1.49999999999999	-6.46965982423159e-07\\
1.5	-6.46965982423082e-07\\
1.50499999999999	-6.20514293217049e-07\\
1.505	-6.20514293216975e-07\\
1.50599999999999	-6.15351693284912e-07\\
1.506	-6.15351693284839e-07\\
1.50699999999999	-6.10231937160525e-07\\
1.50799999999999	-6.05155977796369e-07\\
1.508	-6.05155977796297e-07\\
1.50999999999999	-5.95133463605008e-07\\
1.51199999999997	-5.85280221372093e-07\\
1.51399999999999	-5.75592388096068e-07\\
1.514	-5.75592388096e-07\\
1.51799999999997	-5.56697819398575e-07\\
1.518	-5.56697819398449e-07\\
1.51999999999999	-5.47483676287149e-07\\
1.52	-5.47483676287084e-07\\
1.52199999999999	-5.38420124002452e-07\\
1.52399999999997	-5.29503609142893e-07\\
1.52599999999999	-5.20730635954505e-07\\
1.526	-5.20730635954443e-07\\
1.52999999999997	-5.03621311631648e-07\\
1.53399999999994	-4.87085017267981e-07\\
1.537	-4.75043212647212e-07\\
1.53700000000001	-4.75043212647155e-07\\
1.53999999999999	-4.63298508781781e-07\\
1.54	-4.63298508781726e-07\\
1.54299999999997	-4.51840545727677e-07\\
1.54599999999994	-4.40659215131324e-07\\
1.54599999999997	-4.40659215131222e-07\\
1.546	-4.40659215131119e-07\\
1.549	-4.29754029904995e-07\\
1.54900000000001	-4.29754029904943e-07\\
1.55200000000001	-4.19124747018758e-07\\
1.55500000000001	-4.0876198968307e-07\\
1.55500000000003	-4.08761989683021e-07\\
1.55999999999999	-3.92058280632787e-07\\
1.56	-3.9205828063274e-07\\
1.56499999999996	-3.76028684233724e-07\\
1.56599999999998	-3.72900173437115e-07\\
1.566	-3.72900173437071e-07\\
1.57099999999996	-3.57649839494969e-07\\
1.57199999999998	-3.54677005726727e-07\\
1.572	-3.54677005726684e-07\\
1.57499999999999	-3.45907703142011e-07\\
1.575	-3.4590770314197e-07\\
1.57799999999999	-3.37356207341001e-07\\
1.57999999999999	-3.31772480888699e-07\\
1.58	-3.3177248088866e-07\\
1.58299999999999	-3.23567324385399e-07\\
1.58599999999997	-3.15560267031899e-07\\
1.586	-3.15560267031826e-07\\
1.58699999999999	-3.12934790181142e-07\\
1.587	-3.12934790181105e-07\\
1.58799999999999	-3.10331772917966e-07\\
1.58899999999999	-3.07750960075479e-07\\
1.59099999999997	-3.02654938017935e-07\\
1.59499999999995	-2.92718360082872e-07\\
1.59499999999997	-2.92718360082807e-07\\
1.595	-2.92718360082741e-07\\
1.59999999999999	-2.80756674565536e-07\\
1.6	-2.80756674565503e-07\\
1.60499999999999	-2.69277727841909e-07\\
1.60599999999999	-2.6703737144704e-07\\
1.606	-2.67037371447009e-07\\
1.60699999999998	-2.64815607488212e-07\\
1.607	-2.64815607488181e-07\\
1.60799999999999	-2.62612849545219e-07\\
1.60899999999999	-2.60428881705593e-07\\
1.60999999999998	-2.58263489917374e-07\\
1.61	-2.58263489917343e-07\\
1.61199999999999	-2.53987587373152e-07\\
1.61399999999997	-2.49783465539479e-07\\
1.61599999999999	-2.45649476174206e-07\\
1.616	-2.45649476174176e-07\\
1.61999999999997	-2.37585438822793e-07\\
1.61999999999999	-2.37585438822765e-07\\
1.62	-2.37585438822736e-07\\
1.62399999999997	-2.29782828005434e-07\\
1.624	-2.29782828005387e-07\\
1.62599999999999	-2.25975717896285e-07\\
1.626	-2.25975717896258e-07\\
1.62799999999999	-2.22231543436072e-07\\
1.62999999999998	-2.18550973721736e-07\\
1.63199999999999	-2.14932565771757e-07\\
1.632	-2.14932565771731e-07\\
1.63599999999998	-2.07876584623407e-07\\
1.63999999999995	-2.01052533672594e-07\\
1.63999999999998	-2.01052533672553e-07\\
1.64	-2.01052533672513e-07\\
1.645	-1.92832350190112e-07\\
1.64500000000001	-1.92832350190089e-07\\
1.64599999999999	-1.91228009603193e-07\\
1.646	-1.9122800960317e-07\\
1.64699999999999	-1.89636983250257e-07\\
1.64799999999999	-1.88059567272314e-07\\
1.64999999999997	-1.84944949347465e-07\\
1.65199999999999	-1.81882934730689e-07\\
1.652	-1.81882934730667e-07\\
1.65299999999998	-1.80371277726006e-07\\
1.653	-1.80371277725984e-07\\
1.65399999999999	-1.78872322966436e-07\\
1.65499999999997	-1.77385923533193e-07\\
1.65699999999995	-1.74450209131484e-07\\
1.65899999999998	-1.71562983562702e-07\\
1.659	-1.71562983562682e-07\\
1.65999999999999	-1.70137199630488e-07\\
1.66	-1.70137199630468e-07\\
1.66099999999999	-1.68723114896976e-07\\
1.66199999999998	-1.67320590761841e-07\\
1.66399999999995	-1.64549675558426e-07\\
1.66599999999999	-1.61823367672233e-07\\
1.666	-1.61823367672213e-07\\
1.66999999999995	-1.56506437501136e-07\\
1.67399999999991	-1.51367583340531e-07\\
1.67999999999998	-1.43975637475592e-07\\
1.68	-1.43975637475575e-07\\
1.68199999999998	-1.41592131198525e-07\\
1.682	-1.41592131198508e-07\\
1.68399999999998	-1.39247292537327e-07\\
1.68599999999997	-1.36940202047131e-07\\
1.68599999999999	-1.36940202047113e-07\\
1.686	-1.36940202047097e-07\\
1.68999999999997	-1.32440842628641e-07\\
1.69199999999999	-1.30248104736845e-07\\
1.692	-1.3024810473683e-07\\
1.69599999999997	-1.25972213947321e-07\\
1.69799999999999	-1.23887384669785e-07\\
1.698	-1.2388738466977e-07\\
1.69999999999999	-1.21836871736504e-07\\
1.7	-1.2183687173649e-07\\
1.70199999999999	-1.19819871246762e-07\\
1.70399999999998	-1.17835592427537e-07\\
1.70599999999999	-1.15883257334686e-07\\
1.706	-1.15883257334672e-07\\
1.70999999999998	-1.120757529046e-07\\
1.71099999999998	-1.11144030893268e-07\\
1.711	-1.11144030893255e-07\\
1.71499999999997	-1.07495019465674e-07\\
1.715	-1.07495019465651e-07\\
1.71699999999998	-1.05715990593976e-07\\
1.717	-1.05715990593963e-07\\
1.71899999999998	-1.03966345733328e-07\\
1.71999999999999	-1.0310232750308e-07\\
1.72	-1.03102327503068e-07\\
1.72199999999999	-1.0139547602115e-07\\
1.72399999999997	-9.97163146824921e-08\\
1.72599999999999	-9.80641851654392e-08\\
1.726	-9.80641851654276e-08\\
1.72899999999998	-9.56373482015685e-08\\
1.729	-9.56373482015572e-08\\
1.73199999999998	-9.3271910434298e-08\\
1.73499999999997	-9.09657851878987e-08\\
1.73999999999998	-8.72485462338861e-08\\
1.74	-8.72485462338757e-08\\
1.74599999999997	-8.29851061635154e-08\\
1.74599999999998	-8.2985106163504e-08\\
1.746	-8.29851061634924e-08\\
1.74999999999998	-8.02585158957089e-08\\
1.75	-8.02585158956994e-08\\
1.75199999999998	-7.89297272399456e-08\\
1.752	-7.89297272399362e-08\\
1.75399999999998	-7.76232453150041e-08\\
1.75599999999997	-7.63385579094703e-08\\
1.75799999999998	-7.50751613566996e-08\\
1.758	-7.50751613566907e-08\\
1.75999999999999	-7.38325603440556e-08\\
1.76	-7.38325603440468e-08\\
1.76199999999999	-7.26102677115558e-08\\
1.76399999999998	-7.14078042545381e-08\\
1.76599999999999	-7.02246985425808e-08\\
1.766	-7.02246985425725e-08\\
1.76899999999998	-6.84868174352463e-08\\
1.769	-6.84868174352382e-08\\
1.77199999999998	-6.67929048782489e-08\\
1.77499999999996	-6.51414665831872e-08\\
1.77499999999998	-6.51414665831779e-08\\
1.775	-6.51414665831687e-08\\
1.78	-6.24795162960429e-08\\
1.78000000000002	-6.24795162960355e-08\\
1.78499999999998	-5.99249944977103e-08\\
1.785	-5.99249944977032e-08\\
1.78600000000001	-5.94264261716752e-08\\
1.78600000000003	-5.94264261716681e-08\\
1.78700000000005	-5.89319954056393e-08\\
1.78800000000006	-5.844179422904e-08\\
1.79000000000009	-5.74738888747106e-08\\
1.79200000000003	-5.65223306377233e-08\\
1.79200000000004	-5.65223306377166e-08\\
1.79600000000011	-5.4666769552787e-08\\
1.79799999999998	-5.37620392249574e-08\\
1.798	-5.3762039224951e-08\\
1.8	-5.28722007818405e-08\\
1.80000000000002	-5.28722007818342e-08\\
1.80200000000002	-5.19969053635938e-08\\
1.80400000000002	-5.11358098071856e-08\\
1.806	-5.02885765168103e-08\\
1.80600000000002	-5.02885765168044e-08\\
1.80999999999998	-4.86362758853376e-08\\
1.81	-4.86362758853319e-08\\
1.81399999999997	-4.70393145376682e-08\\
1.81799999999994	-4.5495187865846e-08\\
1.82	-4.47421775600963e-08\\
1.82000000000001	-4.47421775600909e-08\\
1.826	-4.25558305674522e-08\\
1.82600000000001	-4.25558305674471e-08\\
1.827	-4.22017639727711e-08\\
1.82700000000001	-4.2201763972766e-08\\
1.828	-4.18507262334433e-08\\
1.82899999999999	-4.15026829381711e-08\\
1.83099999999996	-4.08154435224708e-08\\
1.833	-4.01397762908001e-08\\
1.83300000000001	-4.01397762907953e-08\\
1.83699999999996	-3.88221032371646e-08\\
1.838	-3.84995090314187e-08\\
1.83800000000001	-3.84995090314142e-08\\
1.83999999999999	-3.7862287234108e-08\\
1.84	-3.78622872341036e-08\\
1.84199999999998	-3.72354798358852e-08\\
1.84399999999995	-3.661884109443e-08\\
1.84599999999999	-3.60121292541403e-08\\
1.846	-3.6012129254136e-08\\
1.84999999999995	-3.48289010127955e-08\\
1.8539999999999	-3.36853017613536e-08\\
1.85499999999998	-3.34053824559417e-08\\
1.855	-3.34053824559377e-08\\
1.85599999999998	-3.31278001395999e-08\\
1.856	-3.31278001395959e-08\\
1.85699999999999	-3.28525276188241e-08\\
1.85799999999997	-3.25795379013569e-08\\
1.85999999999995	-3.20403000741265e-08\\
1.85999999999998	-3.20403000741197e-08\\
1.86	-3.20403000741128e-08\\
1.86399999999995	-3.09880554724835e-08\\
1.86599999999999	-3.04746361629106e-08\\
1.866	-3.0474636162907e-08\\
1.86799999999998	-2.99697042288428e-08\\
1.868	-2.99697042288392e-08\\
1.86999999999998	-2.94733499061388e-08\\
1.87199999999996	-2.8985378597119e-08\\
1.87299999999998	-2.8744476661222e-08\\
1.873	-2.87444766612186e-08\\
1.87699999999996	-2.78008784236872e-08\\
1.87999999999999	-2.71135450339443e-08\\
1.88	-2.71135450339411e-08\\
1.88399999999997	-2.6223101397762e-08\\
1.88499999999998	-2.6004987447609e-08\\
1.885	-2.60049874476059e-08\\
1.88599999999999	-2.57886292639487e-08\\
1.886	-2.57886292639456e-08\\
1.88699999999999	-2.55740666104413e-08\\
1.88799999999998	-2.53613394251298e-08\\
1.88999999999996	-2.49413082390308e-08\\
1.88999999999998	-2.49413082390265e-08\\
1.89	-2.49413082390222e-08\\
1.89199999999999	-2.45283710308024e-08\\
1.892	-2.45283710307995e-08\\
1.89399999999999	-2.41223659101517e-08\\
1.89599999999998	-2.37231337011574e-08\\
1.89799999999999	-2.3330517883296e-08\\
1.898	-2.33305178832933e-08\\
1.89999999999999	-2.29443645321499e-08\\
1.9	-2.29443645321472e-08\\
1.90199999999999	-2.25645222568405e-08\\
1.90399999999997	-2.21908421387009e-08\\
1.90599999999999	-2.18231776750027e-08\\
1.906	-2.18231776750001e-08\\
1.90999999999997	-2.11061470385496e-08\\
1.91399999999994	-2.04131313792367e-08\\
1.91399999999997	-2.04131313792321e-08\\
1.914	-2.04131313792275e-08\\
1.91999999999998	-1.94162682509089e-08\\
1.92	-1.94162682509066e-08\\
1.92499999999998	-1.86224195884769e-08\\
1.925	-1.86224195884746e-08\\
1.92599999999998	-1.84674834285151e-08\\
1.926	-1.84674834285129e-08\\
1.92699999999999	-1.83138330655611e-08\\
1.92799999999997	-1.81614970988778e-08\\
1.92999999999995	-1.78607087596418e-08\\
1.93199999999998	-1.75650004855758e-08\\
1.932	-1.75650004855737e-08\\
1.93599999999995	-1.69883623520299e-08\\
1.9399999999999	-1.6430678322065e-08\\
1.93999999999999	-1.64306783220532e-08\\
1.94	-1.64306783220513e-08\\
1.94299999999998	-1.60243266720874e-08\\
1.943	-1.60243266720855e-08\\
1.94599999999998	-1.56277856946937e-08\\
1.946	-1.56277856946909e-08\\
1.94899999999998	-1.52410380965178e-08\\
1.95199999999997	-1.48640752475541e-08\\
1.95199999999998	-1.48640752475519e-08\\
1.952	-1.48640752475497e-08\\
1.95799999999997	-1.41381819826827e-08\\
1.95999999999998	-1.39041748989391e-08\\
1.96	-1.39041748989374e-08\\
1.96599999999996	-1.32247410378825e-08\\
1.96599999999998	-1.32247410378806e-08\\
1.966	-1.32247410378786e-08\\
1.97199999999996	-1.25784643932505e-08\\
1.97199999999998	-1.25784643932487e-08\\
1.972	-1.25784643932468e-08\\
1.97799999999996	-1.19641898799819e-08\\
1.97799999999998	-1.19641898799801e-08\\
1.978	-1.19641898799783e-08\\
1.98	-1.17661654674142e-08\\
1.98000000000002	-1.17661654674129e-08\\
1.98200000000002	-1.1571377459055e-08\\
1.98400000000002	-1.13797494842373e-08\\
1.986	-1.1191206414481e-08\\
1.98600000000002	-1.11912064144796e-08\\
1.99000000000002	-1.08235038741963e-08\\
1.99400000000003	-1.04681165234722e-08\\
1.995	-1.03811282010725e-08\\
1.99500000000001	-1.03811282010712e-08\\
1.99999999999999	-9.95691227066246e-09\\
2	-9.95691227066128e-09\\
2.00099999999997	-9.87415601350494e-09\\
2.001	-9.8741560135026e-09\\
2.00199999999999	-9.79207631615642e-09\\
2.00299999999997	-9.71066513323937e-09\\
2.00499999999995	-9.5498164588615e-09\\
2.00599999999997	-9.47036320226029e-09\\
2.006	-9.47036320225804e-09\\
2.00999999999995	-9.15920134246831e-09\\
2.01199999999997	-9.00755833363719e-09\\
2.012	-9.00755833363505e-09\\
2.01299999999998	-8.93269513242115e-09\\
2.01300000000001	-8.93269513241903e-09\\
2.014	-8.8584609963783e-09\\
2.01499999999999	-8.78484864951351e-09\\
2.01699999999996	-8.63946052444411e-09\\
2.01999999999997	-8.42586332904913e-09\\
2.02	-8.42586332904713e-09\\
2.02399999999995	-8.14914715867654e-09\\
2.02599999999997	-8.01412966850449e-09\\
2.026	-8.01412966850259e-09\\
2.02999999999995	-7.75081437303826e-09\\
2.03	-7.75081437303539e-09\\
2.03399999999995	-7.49631810219567e-09\\
2.03599999999997	-7.37225184595871e-09\\
2.036	-7.37225184595696e-09\\
2.03999999999995	-7.13023987250886e-09\\
2.04	-7.13023987250574e-09\\
2.04399999999995	-6.89607364036707e-09\\
2.04599999999997	-6.78181744450519e-09\\
2.046	-6.78181744450358e-09\\
2.04799999999997	-6.6694500268658e-09\\
2.048	-6.66945002686422e-09\\
2.04999999999996	-6.55899146850636e-09\\
2.05199999999993	-6.45039846371201e-09\\
2.05599999999987	-6.23863953344532e-09\\
2.05899999999997	-6.08440591461964e-09\\
2.059	-6.0844059146182e-09\\
2.05999999999997	-6.0338411109779e-09\\
2.06	-6.03384111097646e-09\\
2.06099999999999	-5.98369121641239e-09\\
2.06199999999998	-5.93395131340365e-09\\
2.06399999999995	-5.83568202164657e-09\\
2.06499999999997	-5.78714300130417e-09\\
2.065	-5.7871430013028e-09\\
2.06599999999997	-5.73899470875154e-09\\
2.066	-5.73899470875018e-09\\
2.06699999999999	-5.69124599299032e-09\\
2.06799999999998	-5.64390574183142e-09\\
2.06999999999995	-5.55043211354764e-09\\
2.07099999999997	-5.5042895748624e-09\\
2.071	-5.5042895748611e-09\\
2.07499999999995	-5.32357617654407e-09\\
2.07699999999997	-5.23547166947827e-09\\
2.077	-5.23547166947703e-09\\
2.07999999999997	-5.10603279297846e-09\\
2.08	-5.10603279297725e-09\\
2.08299999999998	-4.97975409164269e-09\\
2.08599999999995	-4.85652416845909e-09\\
2.086	-4.85652416845716e-09\\
2.08799999999997	-4.77605678737089e-09\\
2.088	-4.77605678736975e-09\\
2.08999999999997	-4.69695635996074e-09\\
2.09199999999993	-4.61919187326626e-09\\
2.09399999999997	-4.5427328394189e-09\\
2.094	-4.54273283941783e-09\\
2.09799999999993	-4.39361172793363e-09\\
2.09999999999997	-4.32089118677496e-09\\
2.1	-4.32089118677394e-09\\
2.10399999999993	-4.17898757236493e-09\\
2.10599999999997	-4.10974886524149e-09\\
2.106	-4.10974886524051e-09\\
2.10999999999993	-3.9747173918502e-09\\
2.112	-3.9089105511927e-09\\
2.11200000000003	-3.90891055119177e-09\\
2.11599999999996	-3.7805856543079e-09\\
2.11699999999997	-3.74917121285253e-09\\
2.117	-3.74917121285164e-09\\
2.11999999999997	-3.6564787983741e-09\\
2.12	-3.65647879837323e-09\\
2.12299999999998	-3.56604941544754e-09\\
2.12599999999995	-3.47780329160975e-09\\
2.126	-3.47780329160834e-09\\
2.12899999999997	-3.39173658257802e-09\\
2.129	-3.39173658257721e-09\\
2.13199999999996	-3.30784737023787e-09\\
2.13499999999993	-3.22606165039854e-09\\
2.13499999999997	-3.22606165039763e-09\\
2.135	-3.22606165039672e-09\\
2.13999999999997	-3.09423140212098e-09\\
2.14	-3.09423140212025e-09\\
2.14499999999998	-2.96772143511769e-09\\
2.14599999999997	-2.94303037062004e-09\\
2.146	-2.94303037061934e-09\\
2.15099999999997	-2.82267055597101e-09\\
2.15199999999997	-2.79920813716559e-09\\
2.152	-2.79920813716493e-09\\
2.15299999999997	-2.77594348809086e-09\\
2.15299999999999	-2.7759434880902e-09\\
2.15399999999998	-2.7528743287676e-09\\
2.15499999999997	-2.72999839797709e-09\\
2.15699999999995	-2.68481727239451e-09\\
2.15999999999997	-2.61843934977484e-09\\
2.16	-2.61843934977422e-09\\
2.16399999999995	-2.53244643901385e-09\\
2.16599999999997	-2.49048811406982e-09\\
2.166	-2.49048811406922e-09\\
2.16999999999995	-2.40865968783442e-09\\
2.17	-2.40865968783348e-09\\
2.17399999999995	-2.32957188058798e-09\\
2.17499999999997	-2.3102135230745e-09\\
2.175	-2.31021352307395e-09\\
2.17899999999995	-2.23437754616511e-09\\
2.17999999999997	-2.21580862427865e-09\\
2.18	-2.21580862427812e-09\\
2.18399999999995	-2.1430386234173e-09\\
2.18599999999997	-2.10753212276636e-09\\
2.186	-2.10753212276586e-09\\
2.187	-2.08999735254141e-09\\
2.18700000000002	-2.08999735254091e-09\\
2.18800000000002	-2.07261258291589e-09\\
2.18800000000005	-2.0726125829154e-09\\
2.18900000000004	-2.05537610993588e-09\\
2.19000000000004	-2.03828624424824e-09\\
2.19200000000003	-2.0045396490152e-09\\
2.19600000000001	-1.93873298938135e-09\\
2.2	-1.87508939275339e-09\\
2.20000000000003	-1.87508939275295e-09\\
2.20399999999997	-1.81350904616928e-09\\
2.204	-1.81350904616885e-09\\
2.20499999999997	-1.79842495606749e-09\\
2.205	-1.79842495606706e-09\\
2.20599999999999	-1.78346228942824e-09\\
2.20600000000003	-1.78346228942761e-09\\
2.20700000000002	-1.76862379607557e-09\\
2.20800000000001	-1.75391223806097e-09\\
2.20999999999998	-1.72486417280099e-09\\
2.21200000000003	-1.69630670524138e-09\\
2.21200000000006	-1.69630670524097e-09\\
2.21600000000001	-1.64061896760432e-09\\
2.21800000000003	-1.61346686492962e-09\\
2.21800000000006	-1.61346686492923e-09\\
2.21999999999997	-1.58676168658682e-09\\
2.22	-1.58676168658644e-09\\
2.22199999999992	-1.5604929628474e-09\\
2.22399999999983	-1.53465039495448e-09\\
2.226	-1.50922385122686e-09\\
2.22600000000003	-1.5092238512265e-09\\
2.22999999999986	-1.45963621734595e-09\\
2.23299999999997	-1.42353954729606e-09\\
2.233	-1.42353954729572e-09\\
2.23699999999983	-1.37680885100811e-09\\
2.23899999999997	-1.35402207548029e-09\\
2.239	-1.35402207547997e-09\\
2.24	-1.3427693979148e-09\\
2.24000000000003	-1.34276939791449e-09\\
2.24100000000003	-1.33160905363882e-09\\
2.24200000000003	-1.32053994878075e-09\\
2.24400000000004	-1.29867112660388e-09\\
2.246	-1.27715435762888e-09\\
2.24600000000003	-1.27715435762858e-09\\
2.25000000000004	-1.23519168767677e-09\\
2.252	-1.21474141291048e-09\\
2.25200000000003	-1.2147414129102e-09\\
2.25600000000004	-1.1748628930994e-09\\
2.25999999999997	-1.13629518082873e-09\\
2.26	-1.13629518082846e-09\\
2.26199999999997	-1.11748389727971e-09\\
2.262	-1.11748389727945e-09\\
2.26399999999996	-1.09897778804608e-09\\
2.26599999999993	-1.08076959773465e-09\\
2.26599999999997	-1.08076959773433e-09\\
2.266	-1.08076959773401e-09\\
2.26999999999994	-1.04525942035653e-09\\
2.272	-1.02795373209538e-09\\
2.27200000000003	-1.02795373209513e-09\\
2.27499999999997	-1.0025378267038e-09\\
2.275	-1.00253782670357e-09\\
2.27799999999994	-9.77753186500347e-10\\
2.27999999999997	-9.61569976625494e-10\\
2.28	-9.61569976625266e-10\\
2.28299999999994	-9.37789125064582e-10\\
2.28599999999989	-9.14582420524831e-10\\
2.28599999999997	-9.14582420524162e-10\\
2.286	-9.14582420523945e-10\\
2.28699999999997	-9.06973050074798e-10\\
2.287	-9.06973050074582e-10\\
2.28799999999999	-8.99428773825644e-10\\
2.28899999999997	-8.91948852231551e-10\\
2.29099999999995	-8.77179146803042e-10\\
2.291	-8.77179146802704e-10\\
2.29499999999995	-8.48380148770387e-10\\
2.29699999999997	-8.34339565446173e-10\\
2.297	-8.34339565445975e-10\\
2.29999999999997	-8.13711820144677e-10\\
2.3	-8.13711820144484e-10\\
2.30299999999998	-7.93587689280769e-10\\
2.30599999999995	-7.73949420311691e-10\\
2.306	-7.73949420311389e-10\\
2.30999999999997	-7.48520243251078e-10\\
2.31	-7.48520243250901e-10\\
2.31399999999997	-7.23942746985813e-10\\
2.31599999999997	-7.11961282814105e-10\\
2.316	-7.11961282813937e-10\\
2.31999999999997	-6.8858943394354e-10\\
2.32	-6.88589433943377e-10\\
2.32399999999997	-6.65975272691227e-10\\
2.32599999999997	-6.54941196533885e-10\\
2.326	-6.54941196533729e-10\\
2.32999999999997	-6.33422199093801e-10\\
2.33199999999997	-6.22935034878605e-10\\
2.332	-6.22935034878457e-10\\
2.33599999999997	-6.02484816432953e-10\\
2.33999999999994	-5.8270679694935e-10\\
2.34	-5.82706796949037e-10\\
2.34000000000003	-5.82706796948898e-10\\
2.34499999999997	-5.58882392611338e-10\\
2.345	-5.58882392611205e-10\\
2.34600000000003	-5.54232562236764e-10\\
2.34600000000006	-5.54232562236632e-10\\
2.34700000000009	-5.49621320258463e-10\\
2.34800000000012	-5.45049524976744e-10\\
2.34899999999997	-5.40516728308045e-10\\
2.349	-5.40516728307916e-10\\
2.35100000000006	-5.3156635754316e-10\\
2.35300000000011	-5.22766698960443e-10\\
2.35499999999997	-5.1411430262079e-10\\
2.355	-5.14114302620668e-10\\
2.35800000000003	-5.01404419943085e-10\\
2.35800000000006	-5.01404419942966e-10\\
2.35999999999997	-4.93105461514733e-10\\
2.36	-4.93105461514616e-10\\
2.36199999999992	-4.84942136631149e-10\\
2.36399999999983	-4.76911244847572e-10\\
2.36599999999997	-4.69009637621144e-10\\
2.366	-4.69009637621032e-10\\
2.36999999999983	-4.5359967822399e-10\\
2.37399999999966	-4.38705833216138e-10\\
2.37799999999997	-4.24304743067939e-10\\
2.378	-4.24304743067838e-10\\
2.37999999999997	-4.17281893849117e-10\\
2.38	-4.17281893849018e-10\\
2.38199999999997	-4.10373822587615e-10\\
2.38399999999995	-4.03577820529392e-10\\
2.38599999999997	-3.96891223275574e-10\\
2.386	-3.9689122327548e-10\\
2.38999999999995	-3.83850814005746e-10\\
2.39	-3.83850814005596e-10\\
2.39399999999995	-3.7124715754564e-10\\
2.396	-3.65102908496369e-10\\
2.39600000000002	-3.65102908496282e-10\\
2.39999999999997	-3.5311752358106e-10\\
2.4	-3.53117523580976e-10\\
2.40399999999995	-3.41520690657151e-10\\
2.40599999999997	-3.35862274411286e-10\\
2.406	-3.35862274411206e-10\\
2.40699999999997	-3.3306788389636e-10\\
2.407	-3.33067883896281e-10\\
2.40799999999998	-3.30297397901298e-10\\
2.40899999999997	-3.27550544879331e-10\\
2.41099999999995	-3.2212666316368e-10\\
2.41299999999997	-3.16794112295603e-10\\
2.413	-3.16794112295528e-10\\
2.41499999999997	-3.11550801656969e-10\\
2.415	-3.11550801656895e-10\\
2.41699999999997	-3.06394675615195e-10\\
2.41899999999995	-3.01323712690688e-10\\
2.41999999999997	-2.98819544831066e-10\\
2.42	-2.98819544830995e-10\\
2.42399999999995	-2.89005927311466e-10\\
2.426	-2.84217591289028e-10\\
2.42600000000003	-2.8421759128896e-10\\
2.427	-2.81852887070435e-10\\
2.42700000000003	-2.81852887070368e-10\\
2.42800000000002	-2.7950841163425e-10\\
2.429	-2.77183935188748e-10\\
2.43099999999998	-2.72594069915909e-10\\
2.43499999999993	-2.63644431648391e-10\\
2.43599999999997	-2.61453675973893e-10\\
2.436	-2.61453675973831e-10\\
2.43999999999997	-2.52870827190714e-10\\
2.44	-2.52870827190655e-10\\
2.44399999999998	-2.44566224572082e-10\\
2.446	-2.40514178746721e-10\\
2.44600000000003	-2.40514178746664e-10\\
2.448	-2.36529117612343e-10\\
2.44800000000002	-2.36529117612287e-10\\
2.44999999999999	-2.32611753335474e-10\\
2.45000000000002	-2.32611753335419e-10\\
2.45199999999998	-2.28760550097895e-10\\
2.45399999999995	-2.24973998030777e-10\\
2.45600000000002	-2.2125061260164e-10\\
2.45600000000005	-2.21250612601587e-10\\
2.45999999999998	-2.13987526869586e-10\\
2.46000000000001	-2.13987526869535e-10\\
2.46399999999994	-2.0695990176863e-10\\
2.46499999999997	-2.05238486724146e-10\\
2.465	-2.05238486724097e-10\\
2.46599999999998	-2.03530928669529e-10\\
2.46600000000001	-2.03530928669481e-10\\
2.46699999999999	-2.01837541430481e-10\\
2.46799999999998	-2.00158640209034e-10\\
2.46999999999996	-1.96843639023328e-10\\
2.47199999999998	-1.93584625453377e-10\\
2.47200000000001	-1.93584625453331e-10\\
2.47599999999996	-1.8722947185324e-10\\
2.47999999999991	-1.8108321224008e-10\\
2.47999999999997	-1.81083212239983e-10\\
2.48	-1.8108321223994e-10\\
2.48499999999997	-1.73679489420597e-10\\
2.485	-1.73679489420556e-10\\
2.48599999999997	-1.72234498194054e-10\\
2.486	-1.72234498194013e-10\\
2.48699999999999	-1.70801498779927e-10\\
2.48799999999998	-1.6938075790536e-10\\
2.48999999999995	-1.66575495973076e-10\\
2.49199999999997	-1.63817612583412e-10\\
2.492	-1.63817612583373e-10\\
2.494	-1.61106026512522e-10\\
2.49400000000002	-1.61106026512484e-10\\
2.49600000000002	-1.58439674686919e-10\\
2.49800000000001	-1.55817511752828e-10\\
2.49999999999997	-1.53238509681235e-10\\
2.5	-1.53238509681199e-10\\
2.50399999999999	-1.48205960281369e-10\\
2.50599999999997	-1.4575043992247e-10\\
2.506	-1.45750439922435e-10\\
2.50999999999999	-1.40961607983283e-10\\
2.51399999999998	-1.36333164851405e-10\\
2.51999999999997	-1.29675415867662e-10\\
2.52	-1.29675415867631e-10\\
2.52299999999997	-1.26468377306643e-10\\
2.523	-1.26468377306613e-10\\
2.52599999999996	-1.23338767260756e-10\\
2.526	-1.23338767260718e-10\\
2.52899999999997	-1.20286449246537e-10\\
2.53199999999993	-1.17311355154062e-10\\
2.53199999999997	-1.1731135515403e-10\\
2.532	-1.17311355153998e-10\\
2.53799999999993	-1.11582406611418e-10\\
2.53799999999997	-1.11582406611387e-10\\
2.538	-1.11582406611357e-10\\
2.53999999999997	-1.09735558587977e-10\\
2.54	-1.09735558587951e-10\\
2.54199999999998	-1.07918894451924e-10\\
2.54399999999995	-1.06131701942516e-10\\
2.54599999999997	-1.04373280384019e-10\\
2.546	-1.04373280383994e-10\\
2.54999999999995	-1.00943952141367e-10\\
2.55199999999997	-9.92726880172142e-11\\
2.552	-9.92726880171907e-11\\
2.55499999999997	-9.68181950081224e-11\\
2.555	-9.68181950080995e-11\\
2.55799999999998	-9.44246652417958e-11\\
2.55800000000001	-9.44246652417734e-11\\
2.55999999999997	-9.28618023468607e-11\\
2.56	-9.28618023468387e-11\\
2.56199999999997	-9.13244820113373e-11\\
2.56399999999994	-8.98121015262943e-11\\
2.56599999999997	-8.83240679571029e-11\\
2.566	-8.83240679570819e-11\\
2.56999999999994	-8.54220587545775e-11\\
2.56999999999997	-8.54220587545537e-11\\
2.57	-8.542205875453e-11\\
2.57399999999994	-8.26172444016043e-11\\
2.57799999999988	-7.99052259437233e-11\\
2.57999999999997	-7.85826804004676e-11\\
2.58	-7.85826804004489e-11\\
2.58099999999997	-7.79295453177888e-11\\
2.581	-7.79295453177703e-11\\
2.58199999999998	-7.72817498546224e-11\\
2.58299999999997	-7.6639230490473e-11\\
2.58499999999995	-7.53697686757906e-11\\
2.58599999999997	-7.47427018024479e-11\\
2.586	-7.47427018024301e-11\\
2.58999999999995	-7.22869271377689e-11\\
2.59	-7.22869271377405e-11\\
2.59199999999997	-7.10901189570533e-11\\
2.592	-7.10901189570364e-11\\
2.59399999999997	-6.99134019158138e-11\\
2.59599999999995	-6.87563146774701e-11\\
2.59799999999997	-6.76184036014927e-11\\
2.598	-6.76184036014767e-11\\
2.59999999999997	-6.64992225716506e-11\\
2.6	-6.64992225716348e-11\\
2.60199999999997	-6.53983328145846e-11\\
2.60399999999995	-6.4315302722118e-11\\
2.60599999999997	-6.32497076881122e-11\\
2.606	-6.32497076880971e-11\\
2.60999999999995	-6.11715512140176e-11\\
2.61	-6.11715512139918e-11\\
2.61399999999994	-5.91629968985426e-11\\
2.61599999999997	-5.81838320031065e-11\\
2.616	-5.81838320030927e-11\\
2.61999999999994	-5.62738071831348e-11\\
2.61999999999997	-5.62738071831202e-11\\
2.62	-5.62738071831056e-11\\
2.62399999999995	-5.44257031031198e-11\\
2.62499999999997	-5.39730104619526e-11\\
2.625	-5.39730104619398e-11\\
2.62599999999997	-5.35239618909736e-11\\
2.626	-5.35239618909609e-11\\
2.62699999999999	-5.3078639919152e-11\\
2.62799999999998	-5.26371274374117e-11\\
2.62999999999995	-5.1765358222005e-11\\
2.63199999999997	-5.09083124642615e-11\\
2.632	-5.09083124642495e-11\\
2.63599999999995	-4.92370529507171e-11\\
2.63899999999997	-4.8019799030268e-11\\
2.639	-4.80197990302566e-11\\
2.63999999999997	-4.76207277578018e-11\\
2.64	-4.76207277577905e-11\\
2.64099999999999	-4.72249310447358e-11\\
2.64199999999998	-4.68323700973984e-11\\
2.64399999999995	-4.60568019130215e-11\\
2.64599999999997	-4.52937191401674e-11\\
2.646	-4.52937191401566e-11\\
2.64999999999995	-4.38055314566428e-11\\
2.65099999999997	-4.34413618941943e-11\\
2.651	-4.3441361894184e-11\\
2.65499999999995	-4.20151222225282e-11\\
2.6589999999999	-4.06359173167093e-11\\
2.65999999999997	-4.0298210198285e-11\\
2.66	-4.02982101982755e-11\\
2.66599999999997	-3.8329019717519e-11\\
2.666	-3.83290197175099e-11\\
2.66799999999997	-3.76939490893432e-11\\
2.668	-3.76939490893342e-11\\
2.66999999999996	-3.70696668342045e-11\\
2.67199999999993	-3.64559281893091e-11\\
2.674	-3.58524925360319e-11\\
2.67400000000002	-3.58524925360234e-11\\
2.67799999999996	-3.46755878484471e-11\\
2.67799999999999	-3.46755878484367e-11\\
2.67800000000003	-3.46755878484263e-11\\
2.67999999999997	-3.4101657404004e-11\\
2.68	-3.41016574039959e-11\\
2.68199999999995	-3.35371069596809e-11\\
2.68399999999989	-3.29817151812338e-11\\
2.68599999999997	-3.24352643251454e-11\\
2.686	-3.24352643251377e-11\\
2.68999999999989	-3.13695589229914e-11\\
2.69399999999978	-3.03395464044841e-11\\
2.69499999999997	-3.00874297748843e-11\\
2.695	-3.00874297748772e-11\\
2.69699999999997	-2.9589486614518e-11\\
2.697	-2.9589486614511e-11\\
2.69899999999996	-2.90997679520281e-11\\
2.69999999999997	-2.88579326735708e-11\\
2.7	-2.8857932673564e-11\\
2.70199999999997	-2.83801917082186e-11\\
2.70399999999993	-2.7910201104252e-11\\
2.70599999999997	-2.74477766000323e-11\\
2.706	-2.74477766000258e-11\\
2.70899999999997	-2.67685145519481e-11\\
2.709	-2.67685145519418e-11\\
2.71199999999996	-2.61064378960786e-11\\
2.71299999999997	-2.58894632791713e-11\\
2.713	-2.58894632791652e-11\\
2.71599999999997	-2.52493944997663e-11\\
2.71899999999993	-2.46251710184165e-11\\
2.71999999999997	-2.44205221295759e-11\\
2.72	-2.44205221295701e-11\\
2.72599999999993	-2.32272021553652e-11\\
2.726	-2.32272021553531e-11\\
2.72600000000002	-2.32272021553476e-11\\
2.72999999999997	-2.24640403493058e-11\\
2.73	-2.24640403493005e-11\\
2.73199999999999	-2.20921176729254e-11\\
2.73200000000002	-2.20921176729202e-11\\
2.73400000000002	-2.17264385622146e-11\\
2.73600000000001	-2.13668596512321e-11\\
2.73799999999999	-2.10132399656852e-11\\
2.73800000000002	-2.10132399656802e-11\\
2.73999999999997	-2.06654408695227e-11\\
2.74	-2.06654408695178e-11\\
2.74199999999995	-2.03233260085986e-11\\
2.7439999999999	-1.99867612554305e-11\\
2.74599999999997	-1.96556146584942e-11\\
2.746	-1.96556146584896e-11\\
2.7499999999999	-1.90098022981566e-11\\
2.7539999999998	-1.83856196540069e-11\\
2.75499999999997	-1.82328381861385e-11\\
2.755	-1.82328381861342e-11\\
2.75999999999997	-1.74877688394403e-11\\
2.76	-1.74877688394362e-11\\
2.76499999999998	-1.67727683211569e-11\\
2.76500000000001	-1.67727683211529e-11\\
2.76599999999997	-1.66332210244393e-11\\
2.766	-1.66332210244353e-11\\
2.76699999999999	-1.64948318143686e-11\\
2.76700000000002	-1.64948318143647e-11\\
2.76800000000001	-1.63576264496215e-11\\
2.76899999999999	-1.6221591482649e-11\\
2.77099999999997	-1.59529795240545e-11\\
2.77299999999999	-1.56888906282281e-11\\
2.77300000000002	-1.56888906282244e-11\\
2.77699999999997	-1.51738696156618e-11\\
2.77999999999997	-1.47987193404728e-11\\
2.78	-1.47987193404693e-11\\
2.78399999999995	-1.43127103959542e-11\\
2.784	-1.43127103959493e-11\\
2.786	-1.40755731595766e-11\\
2.78600000000003	-1.40755731595733e-11\\
2.78800000000004	-1.38423560582958e-11\\
2.79000000000004	-1.36131007688331e-11\\
2.792	-1.33877174107761e-11\\
2.79200000000003	-1.33877174107729e-11\\
2.79600000000004	-1.29482145279694e-11\\
2.79999999999997	-1.25231581111642e-11\\
2.8	-1.25231581111612e-11\\
2.80400000000001	-1.21118815194777e-11\\
2.80599999999997	-1.19112082695944e-11\\
2.806	-1.19112082695916e-11\\
2.81000000000001	-1.15198490757353e-11\\
2.81199999999997	-1.13291223402327e-11\\
2.812	-1.132912234023e-11\\
2.81299999999997	-1.12349642640189e-11\\
2.813	-1.12349642640162e-11\\
2.81399999999998	-1.11415973851309e-11\\
2.81499999999997	-1.10490125522957e-11\\
2.81699999999995	-1.08661528034292e-11\\
2.81899999999997	-1.06863133265479e-11\\
2.819	-1.06863133265454e-11\\
2.81999999999997	-1.05975041114888e-11\\
2.82	-1.05975041114863e-11\\
2.82099999999999	-1.05094236160115e-11\\
2.82199999999998	-1.04220632069827e-11\\
2.82399999999995	-1.02494684690849e-11\\
2.82599999999997	-1.00796522313614e-11\\
2.826	-1.00796522313591e-11\\
2.82999999999995	-9.74847133930087e-12\\
2.83399999999991	-9.42838244394114e-12\\
2.83499999999997	-9.35003414035548e-12\\
2.835	-9.35003414035326e-12\\
2.83999999999997	-8.9679529873308e-12\\
2.84	-8.96795298732868e-12\\
2.84199999999997	-8.81948918069547e-12\\
2.842	-8.81948918069338e-12\\
2.84399999999996	-8.67343389414479e-12\\
2.84599999999992	-8.52972986527916e-12\\
2.846	-8.52972986527369e-12\\
2.84600000000003	-8.52972986527166e-12\\
2.84999999999996	-8.24947381186411e-12\\
2.852	-8.11289257653891e-12\\
2.85200000000003	-8.11289257653699e-12\\
2.853	-8.04546507966469e-12\\
2.85300000000003	-8.04546507966278e-12\\
2.85400000000002	-7.97860416675544e-12\\
2.85500000000001	-7.91230328449694e-12\\
2.85699999999998	-7.78135567347434e-12\\
2.85999999999997	-7.58897378222225e-12\\
2.86	-7.58897378222045e-12\\
2.86399999999995	-7.33974213794908e-12\\
2.86599999999997	-7.2181351098903e-12\\
2.866	-7.21813510988859e-12\\
2.86999999999995	-6.98097331339068e-12\\
2.87	-6.98097331338786e-12\\
2.87099999999997	-6.92293822240178e-12\\
2.871	-6.92293822240014e-12\\
2.87199999999998	-6.86539382557288e-12\\
2.87299999999997	-6.80833448273911e-12\\
2.87499999999995	-6.69564863640815e-12\\
2.87699999999997	-6.5848365044872e-12\\
2.87699999999999	-6.58483650448564e-12\\
2.87999999999997	-6.42203668486506e-12\\
2.88	-6.42203668486354e-12\\
2.88299999999998	-6.26321152850448e-12\\
2.88599999999996	-6.10822092747244e-12\\
2.886	-6.10822092747029e-12\\
2.888	-6.00701427704255e-12\\
2.88800000000003	-6.00701427704112e-12\\
2.89000000000003	-5.90752689225108e-12\\
2.89200000000003	-5.80971976700665e-12\\
2.89600000000002	-5.61899355726707e-12\\
2.89999999999997	-5.43453651256152e-12\\
2.9	-5.43453651256023e-12\\
2.90499999999997	-5.21234141278977e-12\\
2.905	-5.21234141278853e-12\\
2.90599999999997	-5.16897539578636e-12\\
2.90599999999999	-5.16897539578513e-12\\
2.90699999999998	-5.12596926827081e-12\\
2.90799999999997	-5.08333103506783e-12\\
2.90999999999995	-4.99914157128256e-12\\
2.91199999999997	-4.91637399762571e-12\\
2.91199999999999	-4.91637399762454e-12\\
2.91599999999995	-4.75497527004384e-12\\
2.91799999999997	-4.67628083911203e-12\\
2.91799999999999	-4.67628083911092e-12\\
2.91999999999997	-4.59888172059877e-12\\
2.92	-4.59888172059768e-12\\
2.92199999999998	-4.52274757028833e-12\\
2.92399999999996	-4.4478485394864e-12\\
2.926	-4.37415526373796e-12\\
2.92600000000003	-4.37415526373692e-12\\
2.92899999999997	-4.26590614425663e-12\\
2.92899999999999	-4.26590614425561e-12\\
2.93199999999993	-4.16039573744367e-12\\
2.93499999999987	-4.05753096707997e-12\\
2.93499999999997	-4.05753096707663e-12\\
2.935	-4.05753096707567e-12\\
2.93999999999997	-3.89172343705658e-12\\
2.94	-3.89172343705565e-12\\
2.94499999999998	-3.73260741181077e-12\\
2.94599999999997	-3.70155259351973e-12\\
2.946	-3.70155259351885e-12\\
2.95099999999998	-3.55017183098313e-12\\
2.95199999999997	-3.52066232352339e-12\\
2.952	-3.52066232352256e-12\\
2.95699999999998	-3.37678891908506e-12\\
2.95799999999999	-3.34872932312758e-12\\
2.95800000000002	-3.34872932312679e-12\\
2.95999999999997	-3.29330307596154e-12\\
2.96	-3.29330307596076e-12\\
2.96199999999995	-3.23878268489432e-12\\
2.9639999999999	-3.18514677467459e-12\\
2.96599999999997	-3.13237431713458e-12\\
2.966	-3.13237431713384e-12\\
2.9699999999999	-3.02945583423346e-12\\
2.96999999999999	-3.02945583423118e-12\\
2.97000000000002	-3.02945583423046e-12\\
2.97399999999992	-2.92998432398885e-12\\
2.97499999999997	-2.90563663848692e-12\\
2.975	-2.90563663848623e-12\\
2.9789999999999	-2.8102550693505e-12\\
2.97999999999997	-2.78690028473393e-12\\
2.98	-2.78690028473327e-12\\
2.9839999999999	-2.69537490124323e-12\\
2.986	-2.65071712949066e-12\\
2.98600000000003	-2.65071712949004e-12\\
2.98699999999997	-2.62866303370326e-12\\
2.98699999999999	-2.62866303370264e-12\\
2.98799999999998	-2.60679759892924e-12\\
2.98899999999997	-2.58511868204772e-12\\
2.99099999999995	-2.54231192099552e-12\\
2.99299999999999	-2.50022596798987e-12\\
2.99300000000002	-2.50022596798928e-12\\
2.99699999999997	-2.41815076342239e-12\\
2.99999999999997	-2.35836575492963e-12\\
3	-2.35836575492907e-12\\
3.00399999999995	-2.28091399538719e-12\\
3.00599999999997	-2.24312313487969e-12\\
3.006	-2.24312313487915e-12\\
3.00999999999995	-2.16942222697924e-12\\
3.01	-2.16942222697836e-12\\
3.01399999999995	-2.09818972907157e-12\\
3.01599999999997	-2.06346407547313e-12\\
3.01599999999999	-2.06346407547264e-12\\
3.01999999999995	-1.99572588285542e-12\\
3.02	-1.99572588285449e-12\\
3.02399999999995	-1.93018368208641e-12\\
3.02599999999997	-1.89820382544615e-12\\
3.026	-1.8982038254457e-12\\
3.02799999999997	-1.86675263033961e-12\\
3.02799999999999	-1.86675263033917e-12\\
3.02999999999996	-1.83583571738237e-12\\
3.03199999999992	-1.80544096547638e-12\\
3.03599999999985	-1.74617047966347e-12\\
3.03999999999997	-1.68884821298193e-12\\
3.04	-1.68884821298153e-12\\
3.04499999999997	-1.61979838735784e-12\\
3.04499999999999	-1.61979838735746e-12\\
3.04599999999997	-1.60632187092433e-12\\
3.046	-1.60632187092394e-12\\
3.04699999999998	-1.59295719451876e-12\\
3.04799999999996	-1.57970684573812e-12\\
3.04999999999992	-1.55354394744051e-12\\
3.05199999999997	-1.52782291876538e-12\\
3.052	-1.52782291876501e-12\\
3.05599999999992	-1.47766630407571e-12\\
3.05799999999997	-1.45321105396152e-12\\
3.058	-1.45321105396117e-12\\
3.05999999999997	-1.42915833791045e-12\\
3.06	-1.42915833791011e-12\\
3.06199999999997	-1.40549872608842e-12\\
3.06399999999995	-1.38222294265119e-12\\
3.066	-1.35932186223749e-12\\
3.06600000000003	-1.35932186223717e-12\\
3.06999999999997	-1.31465946519602e-12\\
3.07399999999992	-1.27149291285293e-12\\
3.07399999999996	-1.27149291285253e-12\\
3.07399999999999	-1.27149291285213e-12\\
3.07999999999997	-1.20940031253451e-12\\
3.07999999999999	-1.20940031253423e-12\\
3.08599999999997	-1.15030241352425e-12\\
3.08599999999999	-1.15030241352398e-12\\
3.09199999999997	-1.09408856547161e-12\\
3.09199999999999	-1.09408856547135e-12\\
3.09799999999997	-1.04065829776661e-12\\
3.09999999999997	-1.02343391820491e-12\\
3.1	-1.02343391820467e-12\\
3.10299999999997	-9.98123092079175e-13\\
3.10299999999999	-9.98123092078938e-13\\
3.10599999999996	-9.73423351898789e-13\\
3.106	-9.73423351898428e-13\\
3.10600000000003	-9.73423351898197e-13\\
3.10899999999999	-9.49333621632712e-13\\
3.11199999999996	-9.25853363729107e-13\\
3.11199999999999	-9.25853363728823e-13\\
3.11200000000003	-9.25853363728541e-13\\
3.11499999999997	-9.02961864733072e-13\\
3.115	-9.02961864732858e-13\\
3.11799999999995	-8.80638931519867e-13\\
3.11999999999997	-8.66063101079356e-13\\
3.12	-8.66063101079151e-13\\
3.12299999999995	-8.44644256320587e-13\\
3.12599999999989	-8.23742532070308e-13\\
3.12599999999995	-8.23742532069926e-13\\
3.126	-8.2374253206954e-13\\
3.127	-8.168889539345e-13\\
3.12700000000003	-8.16888953934306e-13\\
3.12800000000003	-8.100940045908e-13\\
3.12900000000003	-8.03357017947864e-13\\
3.13100000000003	-7.90054297200907e-13\\
3.13199999999997	-7.83487259267933e-13\\
3.13199999999999	-7.83487259267747e-13\\
3.13599999999999	-7.57766301886672e-13\\
3.13799999999997	-7.45225334957743e-13\\
3.13799999999999	-7.45225334957566e-13\\
3.13999999999997	-7.32890792672652e-13\\
3.14	-7.32890792672478e-13\\
3.14199999999998	-7.20757839291721e-13\\
3.14399999999996	-7.08821718042662e-13\\
3.14599999999997	-6.97077749323092e-13\\
3.146	-6.97077749322926e-13\\
3.14999999999996	-6.74174297402267e-13\\
3.15	-6.74174297402037e-13\\
3.15399999999996	-6.52037933862976e-13\\
3.15599999999997	-6.41246515415471e-13\\
3.156	-6.41246515415319e-13\\
3.15999999999996	-6.20196049680316e-13\\
3.16	-6.20196049680107e-13\\
3.16099999999997	-6.15041328819762e-13\\
3.16099999999999	-6.15041328819617e-13\\
3.16199999999996	-6.09928749483303e-13\\
3.16299999999992	-6.04857810565804e-13\\
3.16499999999985	-5.94838870000723e-13\\
3.16599999999997	-5.89889886373646e-13\\
3.166	-5.89889886373506e-13\\
3.16999999999986	-5.70508239652694e-13\\
3.17199999999997	-5.61062701470501e-13\\
3.172	-5.61062701470368e-13\\
3.17599999999986	-5.42643678443368e-13\\
3.17999999999972	-5.24830088953951e-13\\
3.17999999999997	-5.24830088952841e-13\\
3.18	-5.24830088952716e-13\\
3.18499999999997	-5.03372017321785e-13\\
3.185	-5.03372017321665e-13\\
3.18599999999997	-4.99184026224789e-13\\
3.186	-4.99184026224671e-13\\
3.18699999999997	-4.95030790773876e-13\\
3.18799999999994	-4.90913084019566e-13\\
3.18999999999989	-4.82782645730676e-13\\
3.18999999999994	-4.82782645730459e-13\\
3.18999999999999	-4.82782645730243e-13\\
3.19399999999988	-4.66930584543919e-13\\
3.19599999999999	-4.59202746805088e-13\\
3.19600000000002	-4.5920274680498e-13\\
3.198	-4.51602980907584e-13\\
3.19800000000003	-4.51602980907477e-13\\
3.19999999999998	-4.44128307373594e-13\\
3.20000000000001	-4.44128307373489e-13\\
3.20199999999996	-4.36775795727619e-13\\
3.20399999999991	-4.29542563408687e-13\\
3.20599999999998	-4.22425774599864e-13\\
3.20600000000001	-4.22425774599764e-13\\
3.20999999999991	-4.0854639240905e-13\\
3.21399999999982	-3.95131862110923e-13\\
3.21899999999997	-3.78985412216736e-13\\
3.21899999999999	-3.78985412216646e-13\\
3.21999999999997	-3.75835832227105e-13\\
3.22	-3.75835832227016e-13\\
3.22099999999998	-3.7271209609046e-13\\
3.22199999999996	-3.69613897504357e-13\\
3.22399999999992	-3.63492900833956e-13\\
3.22599999999997	-3.57470442455363e-13\\
3.226	-3.57470442455278e-13\\
3.22999999999992	-3.45725257405476e-13\\
3.23099999999999	-3.42851131416931e-13\\
3.23100000000002	-3.4285113141685e-13\\
3.23499999999994	-3.31594857074074e-13\\
3.23699999999999	-3.26107004424169e-13\\
3.23700000000002	-3.26107004424092e-13\\
3.23999999999997	-3.18044516990916e-13\\
3.24	-3.1804451699084e-13\\
3.24299999999996	-3.10178870521291e-13\\
3.24599999999991	-3.02503126319347e-13\\
3.24599999999997	-3.02503126319178e-13\\
3.246	-3.02503126319107e-13\\
3.24799999999997	-2.9749097490681e-13\\
3.24799999999999	-2.97490974906739e-13\\
3.24999999999996	-2.92563968292626e-13\\
3.25199999999992	-2.87720174741591e-13\\
3.25399999999997	-2.82957695225625e-13\\
3.25399999999999	-2.82957695225558e-13\\
3.25499999999998	-2.80606363528261e-13\\
3.255	-2.80606363528195e-13\\
3.25599999999998	-2.78274662626113e-13\\
3.25699999999997	-2.75962363979509e-13\\
3.25899999999993	-2.7139506881488e-13\\
3.25999999999997	-2.69139624645909e-13\\
3.26	-2.69139624645845e-13\\
3.26399999999993	-2.60300733781437e-13\\
3.26599999999997	-2.55987993911811e-13\\
3.266	-2.5598799391175e-13\\
3.267	-2.53858161310821e-13\\
3.26700000000003	-2.5385816131076e-13\\
3.26800000000003	-2.51746548290973e-13\\
3.26900000000003	-2.49652947884312e-13\\
3.27100000000003	-2.45518965877227e-13\\
3.27500000000003	-2.37458240481723e-13\\
3.27699999999997	-2.33528336635854e-13\\
3.27699999999999	-2.33528336635799e-13\\
3.27999999999997	-2.27754712171856e-13\\
3.28	-2.27754712171802e-13\\
3.28299999999998	-2.22122047733471e-13\\
3.28599999999996	-2.16625374116037e-13\\
3.286	-2.16625374115966e-13\\
3.28899999999999	-2.11264451868165e-13\\
3.28900000000002	-2.11264451868115e-13\\
3.29	-2.09507848264976e-13\\
3.29000000000003	-2.09507848264927e-13\\
3.29100000000001	-2.07766141715919e-13\\
3.29199999999999	-2.0603916151365e-13\\
3.29399999999996	-2.0262870452799e-13\\
3.296	-1.99275140225088e-13\\
3.29600000000003	-1.9927514022504e-13\\
3.29999999999996	-1.92733452427004e-13\\
3.3	-1.92733452426933e-13\\
3.30000000000003	-1.92733452426887e-13\\
3.30399999999996	-1.86403838397922e-13\\
3.30599999999997	-1.83315444255107e-13\\
3.30599999999999	-1.83315444255064e-13\\
3.30999999999992	-1.77292362245008e-13\\
3.31199999999999	-1.74357046588822e-13\\
3.31200000000002	-1.74357046588781e-13\\
3.31599999999995	-1.68633111585208e-13\\
3.31999999999988	-1.6309732240834e-13\\
3.32	-1.63097322408168e-13\\
3.32000000000003	-1.63097322408129e-13\\
3.32499999999998	-1.56428966172374e-13\\
3.325	-1.56428966172337e-13\\
3.326	-1.55127497102571e-13\\
3.32600000000003	-1.55127497102534e-13\\
3.32700000000003	-1.5383682877187e-13\\
3.32800000000003	-1.5255720141494e-13\\
3.33000000000002	-1.50030569024964e-13\\
3.332	-1.47546609356722e-13\\
3.33200000000003	-1.47546609356687e-13\\
3.33499999999999	-1.43898555434172e-13\\
3.33500000000002	-1.43898555434138e-13\\
3.33799999999998	-1.4034110971077e-13\\
3.33999999999997	-1.38018264178142e-13\\
3.34	-1.38018264178109e-13\\
3.34299999999997	-1.34604896542496e-13\\
3.34599999999993	-1.31273938690481e-13\\
3.34599999999997	-1.3127393869044e-13\\
3.346	-1.312739386904e-13\\
3.34699999999999	-1.30181733107836e-13\\
3.34700000000002	-1.30181733107805e-13\\
3.34800000000001	-1.29098870775082e-13\\
3.349	-1.28025245542062e-13\\
3.35099999999999	-1.25905286364392e-13\\
3.35499999999995	-1.21771642638291e-13\\
3.35999999999997	-1.16795548586236e-13\\
3.36	-1.16795548586208e-13\\
3.36399999999997	-1.12959832653229e-13\\
3.36399999999999	-1.12959832653202e-13\\
3.36599999999997	-1.11088280606513e-13\\
3.366	-1.11088280606486e-13\\
3.36799999999998	-1.09247667339592e-13\\
3.36999999999996	-1.07438321776991e-13\\
3.37199999999997	-1.05659534557801e-13\\
3.372	-1.05659534557776e-13\\
3.37599999999996	-1.02190857378047e-13\\
3.37799999999997	-1.00499607508959e-13\\
3.378	-1.00499607508935e-13\\
3.37999999999997	-9.88361956569031e-14\\
3.38	-9.88361956568796e-14\\
3.38199999999997	-9.71999696835991e-14\\
3.38399999999995	-9.55902881001927e-14\\
3.38599999999997	-9.40065198247906e-14\\
3.386	-9.40065198247683e-14\\
3.38999999999995	-9.09178058297786e-14\\
3.39299999999997	-8.86694168400394e-14\\
3.39299999999999	-8.86694168400184e-14\\
3.39499999999998	-8.72018349754664e-14\\
3.395	-8.72018349754457e-14\\
3.39699999999999	-8.57586557334059e-14\\
3.39899999999997	-8.43393133097318e-14\\
3.399	-8.43393133097118e-14\\
3.39999999999997	-8.36384066509526e-14\\
3.4	-8.36384066509327e-14\\
3.40099999999998	-8.2943251246912e-14\\
3.40199999999996	-8.2253778962549e-14\\
3.40399999999991	-8.08916140013595e-14\\
3.40599999999997	-7.95513777253065e-14\\
3.406	-7.95513777252876e-14\\
3.40999999999991	-7.69376073858217e-14\\
3.41099999999997	-7.62980001566494e-14\\
3.411	-7.62980001566313e-14\\
3.41499999999991	-7.37930318716043e-14\\
3.41899999999982	-7.13706728423389e-14\\
3.41999999999997	-7.07775427831177e-14\\
3.42	-7.07775427831009e-14\\
3.42199999999997	-6.96058256817e-14\\
3.42199999999999	-6.96058256816835e-14\\
3.42399999999996	-6.845311734377e-14\\
3.42599999999992	-6.73189657758643e-14\\
3.426	-6.73189657758173e-14\\
3.42600000000003	-6.73189657758014e-14\\
3.42999999999996	-6.5107108216986e-14\\
3.43	-6.51071082169594e-14\\
3.432	-6.4029171686277e-14\\
3.43200000000003	-6.40291716862618e-14\\
3.43400000000003	-6.29693307644826e-14\\
3.43600000000003	-6.19271699367673e-14\\
3.438	-6.09022806199836e-14\\
3.43800000000003	-6.09022806199691e-14\\
3.43999999999997	-5.98942610078512e-14\\
3.44	-5.9894261007837e-14\\
3.44199999999995	-5.89027159077013e-14\\
3.44399999999989	-5.79272565803615e-14\\
3.44599999999997	-5.69675005931895e-14\\
3.446	-5.69675005931759e-14\\
3.44999999999989	-5.50957547092801e-14\\
3.45099999999996	-5.46377258685123e-14\\
3.45099999999999	-5.46377258684993e-14\\
3.45499999999988	-5.28438941738728e-14\\
3.45699999999996	-5.19693344635255e-14\\
3.45699999999999	-5.19693344635132e-14\\
3.45999999999997	-5.0684473675495e-14\\
3.46	-5.0684473675483e-14\\
3.46299999999998	-4.94309820184898e-14\\
3.46499999999998	-4.86122010398904e-14\\
3.465	-4.86122010398789e-14\\
3.46599999999997	-4.82077537214988e-14\\
3.466	-4.82077537214873e-14\\
3.46699999999997	-4.78066628608809e-14\\
3.46799999999994	-4.74090031169614e-14\\
3.46999999999988	-4.66238214124858e-14\\
3.47199999999997	-4.58519007745066e-14\\
3.472	-4.58519007744957e-14\\
3.47599999999988	-4.43466372594216e-14\\
3.47999999999976	-4.28908517766138e-14\\
3.47999999999996	-4.28908517765403e-14\\
3.47999999999999	-4.28908517765302e-14\\
3.48599999999996	-4.07949706858099e-14\\
3.48599999999999	-4.07949706858002e-14\\
3.49199999999996	-3.88013712199406e-14\\
3.49199999999999	-3.88013712199313e-14\\
3.49799999999996	-3.69064902060786e-14\\
3.5	-3.62956351371981e-14\\
3.50000000000003	-3.62956351371895e-14\\
3.506	-3.45220323403653e-14\\
3.50600000000003	-3.45220323403571e-14\\
3.50899999999999	-3.36677006041787e-14\\
3.50900000000002	-3.36677006041707e-14\\
3.51199999999998	-3.28349835499721e-14\\
3.51499999999995	-3.20231465832783e-14\\
3.51499999999998	-3.20231465832682e-14\\
3.51500000000002	-3.20231465832582e-14\\
3.518	-3.12314735301807e-14\\
3.51800000000003	-3.12314735301733e-14\\
3.51999999999997	-3.0714548092292e-14\\
3.52	-3.07145480922847e-14\\
3.52199999999995	-3.0206070995269e-14\\
3.52399999999989	-2.9705842889841e-14\\
3.52599999999997	-2.92136676596237e-14\\
3.526	-2.92136676596168e-14\\
3.52999999999989	-2.82538122659013e-14\\
3.53399999999978	-2.7326104599682e-14\\
3.53499999999998	-2.70990291750021e-14\\
3.535	-2.70990291749957e-14\\
3.53799999999997	-2.64290896338166e-14\\
3.53799999999999	-2.64290896338103e-14\\
3.53999999999997	-2.59916504995883e-14\\
3.54	-2.59916504995821e-14\\
3.54199999999998	-2.55613606265209e-14\\
3.54399999999996	-2.51380513187718e-14\\
3.54599999999997	-2.47215566162945e-14\\
3.546	-2.47215566162886e-14\\
3.54999999999996	-2.39092957345257e-14\\
3.54999999999999	-2.39092957345199e-14\\
3.553	-2.33180210404949e-14\\
3.55300000000003	-2.33180210404894e-14\\
3.55600000000004	-2.27415263838088e-14\\
3.55900000000005	-2.21793032085386e-14\\
3.55999999999997	-2.19949808440081e-14\\
3.56	-2.19949808440029e-14\\
3.56599999999999	-2.0920186055034e-14\\
3.56600000000002	-2.09201860550291e-14\\
3.56699999999996	-2.07461290838185e-14\\
3.56699999999999	-2.07461290838135e-14\\
3.56799999999996	-2.05735610793579e-14\\
3.56899999999992	-2.0402465125275e-14\\
3.56999999999998	-2.02328244523391e-14\\
3.57	-2.02328244523343e-14\\
3.57199999999993	-1.98978425849201e-14\\
3.57299999999996	-1.97324685586537e-14\\
3.57299999999999	-1.9732468558649e-14\\
3.57499999999992	-1.94058732776885e-14\\
3.57699999999985	-1.90847085490819e-14\\
3.57899999999996	-1.87688484589389e-14\\
3.57899999999999	-1.87688484589345e-14\\
3.57999999999997	-1.86128688756542e-14\\
3.58	-1.86128688756497e-14\\
3.58099999999998	-1.84581691750974e-14\\
3.58199999999997	-1.83047341945102e-14\\
3.58399999999993	-1.80015983628171e-14\\
3.58599999999997	-1.77033425349953e-14\\
3.586	-1.77033425349911e-14\\
3.58999999999993	-1.7121674772293e-14\\
3.59399999999986	-1.6559488378084e-14\\
3.59599999999996	-1.62854239876795e-14\\
3.59599999999999	-1.62854239876756e-14\\
3.59999999999997	-1.5750815585786e-14\\
3.6	-1.57508155857823e-14\\
3.60399999999998	-1.52335385782119e-14\\
3.60499999999998	-1.510683170208e-14\\
3.605	-1.51068317020764e-14\\
3.60599999999997	-1.49811447867518e-14\\
3.606	-1.49811447867483e-14\\
3.60699999999997	-1.48565009317784e-14\\
3.60799999999994	-1.47329233380123e-14\\
3.60799999999999	-1.47329233380059e-14\\
3.60999999999993	-1.44889185898453e-14\\
3.61199999999987	-1.42490348799036e-14\\
3.61399999999996	-1.40131781608352e-14\\
3.61399999999999	-1.40131781608319e-14\\
3.61799999999987	-1.35531773694731e-14\\
3.61999999999997	-1.3328852952031e-14\\
3.62	-1.33288529520278e-14\\
3.62399999999988	-1.28911163083524e-14\\
3.62499999999996	-1.27838928248907e-14\\
3.62499999999999	-1.27838928248877e-14\\
3.62599999999997	-1.2677532465497e-14\\
3.626	-1.2677532465494e-14\\
3.62699999999998	-1.25720547777366e-14\\
3.62799999999997	-1.24674793949086e-14\\
3.62999999999993	-1.22609946335711e-14\\
3.63199999999997	-1.20579972283313e-14\\
3.632	-1.20579972283284e-14\\
3.63599999999993	-1.16621474834296e-14\\
3.63999999999985	-1.12793093253031e-14\\
3.63999999999997	-1.12793093252916e-14\\
3.64	-1.12793093252889e-14\\
3.64599999999997	-1.0728140718954e-14\\
3.646	-1.07281407189515e-14\\
3.65199999999997	-1.0203869831953e-14\\
3.652	-1.02038698319506e-14\\
3.65399999999996	-1.00349705829278e-14\\
3.65399999999999	-1.00349705829254e-14\\
3.65599999999995	-9.86888888692641e-15\\
3.65799999999992	-9.70555962818327e-15\\
3.65999999999996	-9.54491877282795e-15\\
3.65999999999999	-9.54491877282568e-15\\
3.66399999999992	-9.23145138601018e-15\\
3.66599999999996	-9.078501958419e-15\\
3.66599999999999	-9.07850195841685e-15\\
3.66999999999992	-8.78021524244453e-15\\
3.67399999999984	-8.49191882287093e-15\\
3.67499999999998	-8.42135237794935e-15\\
3.67500000000001	-8.42135237794735e-15\\
3.67999999999997	-8.07722101321503e-15\\
3.68	-8.07722101321311e-15\\
3.68299999999996	-7.87746103813164e-15\\
3.68299999999999	-7.87746103812977e-15\\
3.68599999999995	-7.68252391869859e-15\\
3.686	-7.68252391869545e-15\\
3.68899999999996	-7.49240116457305e-15\\
3.69199999999993	-7.30708853334999e-15\\
3.69199999999997	-7.30708853334701e-15\\
3.692	-7.30708853334528e-15\\
3.69299999999998	-7.24635819728954e-15\\
3.69300000000001	-7.24635819728782e-15\\
3.69399999999998	-7.18613817066278e-15\\
3.69499999999995	-7.12642255025906e-15\\
3.6969999999999	-7.00848116597157e-15\\
3.69999999999997	-6.83520739341388e-15\\
3.7	-6.83520739341226e-15\\
3.7039999999999	-6.61073040669021e-15\\
3.70599999999997	-6.50120186224144e-15\\
3.706	-6.50120186223989e-15\\
3.7099999999999	-6.28759589893387e-15\\
3.70999999999998	-6.28759589892967e-15\\
3.71000000000001	-6.28759589892818e-15\\
3.71199999999996	-6.18349621582644e-15\\
3.71199999999999	-6.18349621582497e-15\\
3.71399999999995	-6.08114408197546e-15\\
3.71599999999991	-5.98049936981793e-15\\
3.71799999999996	-5.88152262120288e-15\\
3.71799999999999	-5.88152262120149e-15\\
3.71999999999997	-5.78417503245618e-15\\
3.72	-5.7841750324548e-15\\
3.72199999999998	-5.68841843859074e-15\\
3.72399999999997	-5.5942152978558e-15\\
3.72599999999997	-5.50152867754692e-15\\
3.726	-5.50152867754561e-15\\
3.728	-5.41037426708697e-15\\
3.72800000000003	-5.41037426708569e-15\\
3.73000000000004	-5.32076835659003e-15\\
3.73200000000004	-5.23267581570431e-15\\
3.73600000000004	-5.0608932753967e-15\\
3.74	-4.89475722217972e-15\\
3.74000000000003	-4.89475722217856e-15\\
3.74099999999996	-4.854074750526e-15\\
3.74099999999999	-4.85407475052485e-15\\
3.74199999999996	-4.81372487303449e-15\\
3.74299999999992	-4.77370363314156e-15\\
3.74499999999985	-4.69463140798249e-15\\
3.745	-4.69463140797648e-15\\
3.74500000000003	-4.69463140797537e-15\\
3.746	-4.65557267266386e-15\\
3.74600000000003	-4.65557267266276e-15\\
3.747	-4.61683808101855e-15\\
3.74799999999997	-4.57843484298726e-15\\
3.74999999999991	-4.50260740420368e-15\\
3.752	-4.42806062786829e-15\\
3.75200000000003	-4.42806062786724e-15\\
3.75599999999991	-4.28269264958756e-15\\
3.758	-4.21181445554549e-15\\
3.75800000000003	-4.21181445554449e-15\\
3.75999999999997	-4.14210291844037e-15\\
3.76	-4.14210291843939e-15\\
3.76199999999995	-4.07353070795797e-15\\
3.76399999999989	-4.00607094009237e-15\\
3.76599999999997	-3.93969716697781e-15\\
3.766	-3.93969716697688e-15\\
3.76999999999989	-3.81025297675158e-15\\
3.76999999999996	-3.81025297674926e-15\\
3.76999999999999	-3.81025297674836e-15\\
3.77399999999988	-3.6851441651295e-15\\
3.77599999999999	-3.62415395167592e-15\\
3.77600000000002	-3.62415395167506e-15\\
3.77999999999991	-3.50518234435319e-15\\
3.77999999999996	-3.50518234435183e-15\\
3.78	-3.50518234435045e-15\\
3.78399999999989	-3.39006765550459e-15\\
3.78599999999997	-3.33390000777525e-15\\
3.786	-3.33390000777446e-15\\
3.78999999999989	-3.22436011969777e-15\\
3.79199999999997	-3.17097646195372e-15\\
3.792	-3.17097646195297e-15\\
3.79599999999989	-3.06687706580533e-15\\
3.79899999999996	-2.99105676558978e-15\\
3.79899999999999	-2.99105676558907e-15\\
3.79999999999997	-2.96619941810204e-15\\
3.8	-2.96619941810134e-15\\
3.80099999999999	-2.94154603633986e-15\\
3.80199999999997	-2.91709420392314e-15\\
3.80399999999993	-2.86878562050751e-15\\
3.80599999999997	-2.82125472828752e-15\\
3.806	-2.82125472828685e-15\\
3.80999999999993	-2.72855850856725e-15\\
3.81099999999996	-2.70587511846295e-15\\
3.81099999999999	-2.70587511846231e-15\\
3.81499999999992	-2.61703751590995e-15\\
3.81499999999998	-2.61703751590871e-15\\
3.81500000000001	-2.61703751590809e-15\\
3.81899999999993	-2.53112961439891e-15\\
3.81999999999997	-2.5100945176302e-15\\
3.82	-2.51009451762961e-15\\
3.82399999999993	-2.4276597904262e-15\\
3.826	-2.3874376018712e-15\\
3.82600000000003	-2.38743760187063e-15\\
3.82700000000001	-2.36757400446317e-15\\
3.82700000000003	-2.36757400446261e-15\\
3.82799999999998	-2.34788032952707e-15\\
3.82800000000001	-2.34788032952651e-15\\
3.82899999999997	-2.32835464680266e-15\\
3.82999999999993	-2.30899504257081e-15\\
3.83199999999986	-2.27076649590433e-15\\
3.83399999999998	-2.23317970187517e-15\\
3.83400000000001	-2.23317970187464e-15\\
3.83799999999985	-2.15987267428425e-15\\
3.84	-2.12412370040751e-15\\
3.84000000000003	-2.124123700407e-15\\
3.84399999999988	-2.05436475025752e-15\\
3.846	-2.02032742470752e-15\\
3.84600000000003	-2.02032742470704e-15\\
3.84999999999988	-1.95394677708567e-15\\
3.84999999999998	-1.95394677708404e-15\\
3.85000000000001	-1.95394677708358e-15\\
3.85399999999985	-1.88978936741942e-15\\
3.85599999999998	-1.85851280056189e-15\\
3.85600000000001	-1.85851280056145e-15\\
3.85699999999996	-1.84306965300618e-15\\
3.85699999999999	-1.84306965300574e-15\\
3.85799999999995	-1.82775457374714e-15\\
3.85899999999992	-1.81256606169043e-15\\
3.85999999999997	-1.79750262819955e-15\\
3.86	-1.79750262819912e-15\\
3.86199999999993	-1.7677451035562e-15\\
3.86399999999985	-1.73847033334046e-15\\
3.86599999999997	-1.70966684026331e-15\\
3.866	-1.7096668402629e-15\\
3.86899999999996	-1.66735697232078e-15\\
3.86899999999999	-1.66735697232038e-15\\
3.87199999999995	-1.62611754789958e-15\\
3.87499999999991	-1.58591218755792e-15\\
3.87999999999997	-1.52110524272874e-15\\
3.88	-1.52110524272838e-15\\
3.88499999999998	-1.45891371561352e-15\\
3.88500000000001	-1.45891371561317e-15\\
3.88599999999999	-1.44677573913382e-15\\
3.88600000000002	-1.44677573913348e-15\\
3.88700000000001	-1.43473849429686e-15\\
3.88799999999999	-1.42280422162054e-15\\
3.88999999999996	-1.39923992399512e-15\\
3.89199999999999	-1.37607360778336e-15\\
3.89200000000002	-1.37607360778304e-15\\
3.89599999999996	-1.33089874285555e-15\\
3.89799999999999	-1.30887248315541e-15\\
3.89800000000002	-1.3088724831551e-15\\
3.89999999999997	-1.28720877626961e-15\\
3.9	-1.2872087762693e-15\\
3.90199999999996	-1.26589912897186e-15\\
3.90399999999991	-1.2449351867308e-15\\
3.90599999999997	-1.22430873055122e-15\\
3.906	-1.22430873055093e-15\\
3.90999999999991	-1.1840823762004e-15\\
3.91399999999982	-1.1452032934228e-15\\
3.91499999999996	-1.1356868427643e-15\\
3.91499999999999	-1.13568684276403e-15\\
3.91999999999997	-1.08927797019302e-15\\
3.92	-1.08927797019276e-15\\
3.92499999999999	-1.04474202454565e-15\\
3.92599999999997	-1.03604990382504e-15\\
3.926	-1.0360499038248e-15\\
3.92699999999996	-1.02742991792166e-15\\
3.92699999999999	-1.02742991792142e-15\\
3.92799999999995	-1.01888367144116e-15\\
3.92899999999992	-1.01041032668775e-15\\
3.93099999999984	-9.9367902752157e-16\\
3.93299999999996	-9.77229460851304e-16\\
3.93299999999999	-9.77229460851072e-16\\
3.93699999999984	-9.45149836863486e-16\\
3.93999999999997	-9.21782480064383e-16\\
3.94	-9.21782480064165e-16\\
3.94399999999985	-8.91509959540591e-16\\
3.94399999999996	-8.91509959539745e-16\\
3.94399999999999	-8.91509959539533e-16\\
3.94599999999997	-8.76739157620247e-16\\
3.946	-8.76739157620039e-16\\
3.94799999999999	-8.62212533044435e-16\\
3.94999999999997	-8.47932681771328e-16\\
3.95199999999997	-8.3389400533016e-16\\
3.952	-8.33894005329962e-16\\
3.95499999999998	-8.13276179175535e-16\\
3.95500000000001	-8.13276179175342e-16\\
3.95799999999998	-7.9317044637687e-16\\
3.95999999999998	-7.80042344024914e-16\\
3.96	-7.80042344024729e-16\\
3.96299999999998	-7.60750902121826e-16\\
3.96599999999995	-7.41925218320583e-16\\
3.966	-7.41925218320275e-16\\
3.96600000000003	-7.41925218320098e-16\\
3.96700000000001	-7.35752364191756e-16\\
3.96700000000003	-7.35752364191582e-16\\
3.96800000000001	-7.29632315616045e-16\\
3.96899999999998	-7.23564472661093e-16\\
3.97099999999993	-7.11583029824479e-16\\
3.97299999999996	-6.99803338310683e-16\\
3.97299999999999	-6.99803338310517e-16\\
3.97699999999989	-6.76830813729727e-16\\
3.97899999999996	-6.65628974193142e-16\\
3.97899999999999	-6.65628974192985e-16\\
3.97999999999997	-6.60097226696182e-16\\
3.98	-6.60097226696025e-16\\
3.98099999999999	-6.54610869678583e-16\\
3.98199999999997	-6.49169365399763e-16\\
3.98399999999994	-6.38418786105708e-16\\
3.98599999999997	-6.27841274007706e-16\\
3.986	-6.27841274007557e-16\\
3.98999999999994	-6.07212681963767e-16\\
3.98999999999997	-6.07212681963586e-16\\
3.99000000000001	-6.07212681963406e-16\\
3.99399999999994	-5.87274988109447e-16\\
3.99599999999998	-5.77555415298363e-16\\
3.99600000000001	-5.77555415298226e-16\\
3.99999999999994	-5.58595763669535e-16\\
3.99999999999997	-5.58595763669384e-16\\
4	-5.58595763669235e-16\\
4.00199999999993	-5.49348251645655e-16\\
4.00199999999999	-5.49348251645395e-16\\
4.00399999999992	-5.40250761446967e-16\\
4.00599999999986	-5.31299726366615e-16\\
4.00599999999993	-5.31299726366292e-16\\
4.006	-5.31299726365964e-16\\
4.00999999999987	-5.13843140178483e-16\\
4.01199999999995	-5.05335769587673e-16\\
4.012	-5.05335769587433e-16\\
4.01599999999987	-4.88746195574398e-16\\
4.01999999999973	-4.72701921091856e-16\\
4.01999999999995	-4.72701921091018e-16\\
4.02	-4.72701921090794e-16\\
4.02500000000001	-4.53375148620804e-16\\
4.02500000000006	-4.53375148620589e-16\\
4.02599999999995	-4.49603125095344e-16\\
4.026	-4.4960312509513e-16\\
4.02699999999996	-4.4586240515004e-16\\
4.02799999999992	-4.42153685053331e-16\\
4.02999999999985	-4.34830793533185e-16\\
4.03099999999993	-4.31215904376339e-16\\
4.03099999999999	-4.31215904376135e-16\\
4.03499999999984	-4.17058492955532e-16\\
4.03699999999994	-4.10156228006739e-16\\
4.03699999999999	-4.10156228006544e-16\\
4.03799999999995	-4.06748014353535e-16\\
4.038	-4.06748014353342e-16\\
4.03899999999996	-4.03367966944358e-16\\
4.03999999999991	-4.00015754486846e-16\\
4.04	-4.00015754486544e-16\\
4.04199999999991	-3.93393522915121e-16\\
4.04399999999982	-3.86878723384874e-16\\
4.046	-3.80468801742839e-16\\
4.04600000000006	-3.80468801742658e-16\\
4.04999999999988	-3.67967973820632e-16\\
4.05399999999969	-3.55885826894424e-16\\
4.05999999999994	-3.38506354350168e-16\\
4.06	-3.38506354350008e-16\\
};
\pgfplotsset{max space between ticks=50}
\addplot [color=mycolor2,solid,forget plot]
  table[row sep=crcr]{%
0	0.15314\\
3.15544362088405e-30	0.15314\\
0.000656101980281985	0.153143230512962\\
0.00393661188169191	0.153256312778436\\
0.00599999999999994	0.153410244700375\\
0.006	0.153410244700375\\
0.012	0.152025843789547\\
0.0120000000000001	0.152025843789547\\
0.018	0.146785790333147\\
0.0180000000000001	0.146785790333147\\
0.0199999999999998	0.144179493489919\\
0.02	0.144179493489918\\
0.026	0.133767457951154\\
0.0260000000000002	0.133767457951153\\
0.0289999999999998	0.127361824092399\\
0.029	0.127361824092398\\
0.0319999999999996	0.120505893144387\\
0.0349999999999991	0.113193617075476\\
0.035	0.113193617075474\\
0.0399999999999996	0.0999746854104061\\
0.04	0.0999746854104049\\
0.0449999999999996	0.0854376461080997\\
0.0459999999999996	0.0823689879834041\\
0.046	0.0823689879834027\\
0.047	0.0792808135350665\\
0.0470000000000004	0.0792808135350652\\
0.0490000000000003	0.0731494760963212\\
0.0510000000000002	0.0670761531370235\\
0.055	0.0550940482536636\\
0.0579999999999996	0.0462422041576017\\
0.058	0.0462422041576004\\
0.0599999999999996	0.0404007870498711\\
0.06	0.0404007870498698\\
0.0619999999999995	0.0346045456083864\\
0.0639999999999991	0.0288512070597887\\
0.0659999999999991	0.0231385157858626\\
0.066	0.02313851578586\\
0.0699999999999991	0.0121295381531389\\
0.07	0.0121295381531366\\
0.0700000000000009	0.0121295381531342\\
0.074	0.00186375329871775\\
0.076	-0.00299551806340728\\
0.0760000000000009	-0.0029955180634094\\
0.08	-0.0121763447641911\\
0.0800000000000009	-0.0121763447641931\\
0.0839999999999999	-0.0206520990256193\\
0.086	-0.0246297706860801\\
0.0860000000000009	-0.0246297706860818\\
0.0869999999999991	-0.0265502128214379\\
0.087	-0.0265502128214396\\
0.0880000000000004	-0.0284199307246156\\
0.0890000000000009	-0.0302391076551602\\
0.0910000000000017	-0.0337265468553195\\
0.0929999999999991	-0.0370138976863233\\
0.093	-0.0370138976863247\\
0.0970000000000017	-0.0429934114909054\\
0.0999999999999991	-0.0469618347664145\\
0.1	-0.0469618347664156\\
0.104000000000002	-0.0515714390916565\\
0.104999999999999	-0.0526028899776093\\
0.105	-0.0526028899776102\\
0.105999999999999	-0.053586170072105\\
0.106	-0.0535861700721058\\
0.106999999999999	-0.05452585761117\\
0.107999999999998	-0.0554265265666187\\
0.109999999999997	-0.0571111580107877\\
0.111999999999999	-0.058640724871287\\
0.112	-0.0586407248712876\\
0.115999999999997	-0.0612370029316692\\
0.115999999999998	-0.0612370029316702\\
0.116	-0.0612370029316711\\
0.119999999999997	-0.063219432648722\\
0.119999999999998	-0.0632194326487227\\
0.12	-0.0632194326487234\\
0.123999999999997	-0.0645911231830114\\
0.125999999999999	-0.0650486508548064\\
0.126	-0.0650486508548066\\
0.127999999999998	-0.0653835388327254\\
0.128	-0.0653835388327257\\
0.129999999999998	-0.0656252314014744\\
0.131999999999996	-0.0657738233185884\\
0.135999999999993	-0.0657919017428447\\
0.139999999999998	-0.0654378024812815\\
0.14	-0.0654378024812812\\
0.144999999999998	-0.0644709017753106\\
0.145	-0.0644709017753102\\
0.145999999999998	-0.0642074456912333\\
0.146	-0.0642074456912328\\
0.146999999999999	-0.0639273473319215\\
0.147999999999998	-0.0636373505175543\\
0.149999999999997	-0.0630275468598978\\
0.151999999999998	-0.0623777956425841\\
0.152	-0.0623777956425835\\
0.155999999999997	-0.0609574158314218\\
0.157999999999998	-0.0601862303730122\\
0.158	-0.0601862303730115\\
0.16	-0.0593739834145983\\
0.160000000000002	-0.0593739834145976\\
0.162000000000002	-0.0585203565157101\\
0.164000000000002	-0.0576250150091663\\
0.166	-0.0566876078731048\\
0.166000000000002	-0.056687607873104\\
0.170000000000002	-0.0547623883316001\\
0.174	-0.0528189567458071\\
0.174000000000001	-0.0528189567458063\\
0.175	-0.0523298967241606\\
0.175000000000002	-0.0523298967241597\\
0.176000000000001	-0.0518394601814495\\
0.177	-0.051347599048052\\
0.178999999999998	-0.0503594100348185\\
0.179999999999998	-0.0498629853004126\\
0.18	-0.0498629853004117\\
0.183999999999997	-0.0478606143562568\\
0.186	-0.0468487381919073\\
0.186000000000002	-0.0468487381919064\\
0.189999999999998	-0.0448460616191702\\
0.192	-0.0438655851705541\\
0.192000000000002	-0.0438655851705532\\
0.195999999999998	-0.0419444338237903\\
0.199999999999995	-0.0400738489287693\\
0.199999999999997	-0.0400738489287682\\
0.2	-0.040073848928767\\
0.202999999999998	-0.0387023267293148\\
0.203	-0.038702326729314\\
0.205999999999998	-0.0373563865838793\\
0.206	-0.0373563865838785\\
0.208999999999998	-0.0360456737975151\\
0.209999999999998	-0.0356188041094374\\
0.21	-0.0356188041094366\\
0.211999999999998	-0.0347798648300974\\
0.212	-0.0347798648300967\\
0.213999999999998	-0.0339603857274477\\
0.215999999999997	-0.0331600455233567\\
0.217999999999998	-0.0323785304413606\\
0.218	-0.0323785304413599\\
0.219999999999998	-0.031615534089747\\
0.22	-0.0316155340897463\\
0.221999999999998	-0.0308707573368921\\
0.223999999999996	-0.030143908190053\\
0.225999999999998	-0.0294347016852164\\
0.226	-0.0294347016852158\\
0.229999999999996	-0.0280719343105698\\
0.231999999999998	-0.0274187948368041\\
0.232	-0.0274187948368036\\
0.235999999999996	-0.0261677239027614\\
0.237999999999998	-0.0255693019550791\\
0.238	-0.0255693019550785\\
0.239999999999998	-0.0249886405204315\\
0.24	-0.024988640520431\\
0.241999999999998	-0.0244255119513503\\
0.243999999999996	-0.0238796954709565\\
0.245	-0.0236132121289339\\
0.245000000000002	-0.0236132121289334\\
0.245999999999998	-0.0233509770892397\\
0.246	-0.0233509770892393\\
0.246999999999999	-0.0230927460166266\\
0.247999999999998	-0.0228382749662833\\
0.249999999999997	-0.022340513535244\\
0.252	-0.021857497646821\\
0.252000000000003	-0.0218574976468202\\
0.256	-0.020934950743504\\
0.259999999999997	-0.020069187370574\\
0.26	-0.0200691873705733\\
0.260999999999996	-0.019861457897877\\
0.261	-0.0198614578978763\\
0.261999999999998	-0.0196571717576136\\
0.262999999999996	-0.0194563089176112\\
0.264999999999993	-0.0190647747253879\\
0.265999999999997	-0.0188740649980124\\
0.266	-0.0188740649980117\\
0.269999999999993	-0.0181390028091065\\
0.271999999999997	-0.0177870605763722\\
0.272	-0.0177870605763716\\
0.275999999999993	-0.0171136638873358\\
0.279999999999986	-0.0164800440753242\\
0.279999999999993	-0.0164800440753232\\
0.28	-0.0164800440753221\\
0.285999999999996	-0.015602038849948\\
0.286	-0.0156020388499475\\
0.289999999999996	-0.0150588753922395\\
0.29	-0.015058875392239\\
0.291999999999996	-0.0147974749263449\\
0.292	-0.0147974749263445\\
0.293999999999996	-0.0145427244433615\\
0.295999999999993	-0.0142945240673259\\
0.297999999999996	-0.0140527764903207\\
0.298	-0.0140527764903202\\
0.299999999999996	-0.0138173869356104\\
0.3	-0.01381738693561\\
0.301999999999996	-0.0135882631190604\\
0.303999999999993	-0.013365315211743\\
0.305999999999996	-0.0131484558060676\\
0.306	-0.0131484558060673\\
0.309999999999993	-0.0127295733955093\\
0.313999999999986	-0.0123278675480914\\
0.314999999999997	-0.0122300505117705\\
0.315	-0.0122300505117702\\
0.318999999999997	-0.011848932449539\\
0.319	-0.0118489324495386\\
0.319999999999996	-0.0117561437671958\\
0.32	-0.0117561437671955\\
0.320999999999998	-0.0116643330826606\\
0.321999999999996	-0.0115734913971812\\
0.323999999999993	-0.0113946795037688\\
0.325999999999996	-0.0112196379830671\\
0.326	-0.0112196379830668\\
0.329999999999993	-0.0108791202965964\\
0.331	-0.01079578148901\\
0.331000000000004	-0.0107957814890097\\
0.333	-0.0106311951896303\\
0.333000000000004	-0.01063119518963\\
0.335	-0.0104693436531452\\
0.336999999999996	-0.0103101634250184\\
0.339999999999996	-0.0100762655512865\\
0.34	-0.0100762655512863\\
0.343999999999993	-0.00977317734586651\\
0.345999999999997	-0.00962526027170982\\
0.346	-0.00962526027170956\\
0.347999999999997	-0.00947960356540665\\
0.348	-0.00947960356540639\\
0.349999999999997	-0.00933607010583894\\
0.35	-0.00933607010583868\\
0.351999999999997	-0.00919460362014799\\
0.353999999999993	-0.00905514864622743\\
0.354	-0.00905514864622694\\
0.357999999999993	-0.00878205530691853\\
0.359999999999996	-0.00864830987459709\\
0.36	-0.00864830987459685\\
0.363999999999993	-0.00838615929020967\\
0.365999999999996	-0.00825765136098592\\
0.366	-0.00825765136098569\\
0.369999999999993	-0.00800583848852589\\
0.373999999999986	-0.00776084841554911\\
0.376999999999997	-0.00758135178776813\\
0.377	-0.00758135178776792\\
0.379999999999997	-0.00740531830425179\\
0.38	-0.00740531830425158\\
0.382999999999996	-0.00723259268666171\\
0.384999999999997	-0.00711920461197321\\
0.385	-0.00711920461197301\\
0.385999999999997	-0.00706302255451955\\
0.386	-0.00706302255451935\\
0.386999999999998	-0.00700720978382274\\
0.387999999999996	-0.00695179614931686\\
0.388999999999997	-0.00689677621988011\\
0.389	-0.00689677621987991\\
0.390999999999997	-0.00678789594421677\\
0.392999999999993	-0.00668052627018024\\
0.394999999999997	-0.00657462510307921\\
0.395	-0.00657462510307902\\
0.398999999999993	-0.00636706277582973\\
0.399999999999997	-0.00631602581140577\\
0.4	-0.00631602581140559\\
0.403999999999993	-0.00611514234694707\\
0.405999999999997	-0.00601659044921387\\
0.406	-0.0060165904492137\\
0.409999999999993	-0.00582364296493607\\
0.411999999999997	-0.00572931417942692\\
0.412	-0.00572931417942676\\
0.415999999999993	-0.00554476159961642\\
0.419999999999986	-0.00536544202466481\\
0.419999999999996	-0.00536544202466433\\
0.42	-0.00536544202466417\\
0.426	-0.005105661471752\\
0.426000000000004	-0.00510566147175185\\
0.432000000000004	-0.00485715184030643\\
0.432000000000007	-0.00485715184030629\\
0.434999999999997	-0.0047372395980762\\
0.435	-0.00473723959807606\\
0.43799999999999	-0.00462008052598476\\
0.439999999999997	-0.00454345288070487\\
0.44	-0.00454345288070474\\
0.44299999999999	-0.00443065390957599\\
0.445999999999979	-0.0043203376519555\\
0.445999999999995	-0.00432033765195493\\
0.446	-0.00432033765195474\\
0.447	-0.00428412121687065\\
0.447000000000004	-0.00428412121687052\\
0.448000000000004	-0.00424820789709132\\
0.449000000000004	-0.00421259417212973\\
0.451000000000004	-0.00414225157361891\\
0.454999999999997	-0.00400500985723923\\
0.455	-0.00400500985723911\\
0.459	-0.00387218082350991\\
0.459999999999997	-0.00383963848634947\\
0.46	-0.00383963848634935\\
0.463999999999997	-0.0037120331307038\\
0.464	-0.00371203313070369\\
0.465999999999997	-0.00364972516138714\\
0.466	-0.00364972516138703\\
0.466999999999997	-0.00361894787712166\\
0.467	-0.00361894787712155\\
0.467999999999998	-0.00358843606096643\\
0.468999999999997	-0.00355818672244122\\
0.470999999999993	-0.0034984636447164\\
0.472999999999997	-0.00343975522928542\\
0.473	-0.00343975522928531\\
0.476999999999993	-0.00332529070805253\\
0.479999999999997	-0.00324193742300783\\
0.48	-0.00324193742300773\\
0.483999999999993	-0.00313398671261266\\
0.485999999999997	-0.00308132950356731\\
0.486	-0.00308132950356722\\
0.489999999999993	-0.00297868852415155\\
0.49	-0.00297868852415136\\
0.490000000000004	-0.00297868852415127\\
0.492999999999997	-0.00290402940479505\\
0.493	-0.00290402940479497\\
0.495999999999993	-0.00283128690175424\\
0.498999999999986	-0.00276039684508185\\
0.498999999999993	-0.00276039684508169\\
0.499	-0.00276039684508152\\
0.499999999999997	-0.00273716762589667\\
0.5	-0.00273716762589659\\
0.500999999999998	-0.00271413499636033\\
0.501999999999997	-0.00269129669864501\\
0.503999999999993	-0.00264619416394156\\
0.505999999999993	-0.00260184233916467\\
0.506	-0.00260184233916451\\
0.507999999999997	-0.00255824665041905\\
0.508000000000004	-0.0025582466504189\\
0.51	-0.00251541282024629\\
0.511999999999997	-0.0024733240554731\\
0.51599999999999	-0.00239131600395722\\
0.519999999999993	-0.00231209387431912\\
0.52	-0.00231209387431898\\
0.521999999999993	-0.00227348853796277\\
0.522	-0.00227348853796264\\
0.523999999999993	-0.00223553342270299\\
0.524999999999993	-0.00221679503717306\\
0.525	-0.00221679503717292\\
0.526	-0.00219821364578049\\
0.526000000000007	-0.00219821364578036\\
0.527000000000007	-0.00217979147688287\\
0.528000000000007	-0.00216153077443716\\
0.530000000000007	-0.00212548662559605\\
0.532	-0.00209006706801044\\
0.532000000000007	-0.00209006706801032\\
0.536000000000007	-0.00202104642138933\\
0.538	-0.00198741827253858\\
0.538000000000007	-0.00198741827253846\\
0.539999999999993	-0.00195436058517631\\
0.54	-0.00195436058517619\\
0.541999999999986	-0.00192186039908287\\
0.543999999999972	-0.00188990497243143\\
0.546	-0.0018584817769738\\
0.546000000000007	-0.00185848177697369\\
0.549999999999979	-0.00179723505058651\\
0.550999999999993	-0.00178225327986291\\
0.551	-0.00178225327986281\\
0.554999999999972	-0.00172360178594778\\
0.556999999999993	-0.0016950213156329\\
0.557	-0.0016950213156328\\
0.559999999999993	-0.00165305062302131\\
0.56	-0.00165305062302121\\
0.562999999999993	-0.00161212710529833\\
0.565999999999986	-0.00157221466169964\\
0.565999999999993	-0.00157221466169954\\
0.566	-0.00157221466169945\\
0.571999999999986	-0.00149538666290301\\
0.571999999999993	-0.00149538666290292\\
0.572	-0.00149538666290283\\
0.577999999999986	-0.00142239911156992\\
0.579999999999993	-0.00139887846372692\\
0.58	-0.00139887846372684\\
0.585999999999986	-0.00133061300722733\\
0.585999999999993	-0.00133061300722725\\
0.586	-0.00133061300722717\\
0.591999999999986	-0.00126570205795457\\
0.591999999999993	-0.0012657020579545\\
0.592	-0.00126570205795442\\
0.594999999999993	-0.00123446833087104\\
0.595	-0.00123446833087096\\
0.597999999999993	-0.00120401212899401\\
0.599999999999993	-0.00118412665745423\\
0.6	-0.00118412665745416\\
0.602999999999993	-0.00115490695376066\\
0.605999999999986	-0.00112639458966449\\
0.606	-0.00112639458966436\\
0.606999999999993	-0.00111704573809149\\
0.607	-0.00111704573809143\\
0.607999999999999	-0.00110777666046391\\
0.608999999999997	-0.00109858644815858\\
0.609000000000004	-0.00109858644815852\\
0.611	-0.00108043902421022\\
0.612999999999997	-0.00106259635152698\\
0.614999999999997	-0.0010450514348206\\
0.615000000000004	-0.00104505143482053\\
0.618999999999997	-0.00101082746959056\\
0.619999999999993	-0.00100244696554173\\
0.62	-0.00100244696554167\\
0.623999999999993	-0.00096960225516176\\
0.625999999999993	-0.000953575039258187\\
0.626	-0.000953575039258131\\
0.629999999999993	-0.000922314872374938\\
0.63	-0.000922314872374883\\
0.633999999999993	-0.000892096646841945\\
0.635999999999993	-0.00087736338994188\\
0.636	-0.000877363389941828\\
0.637999999999993	-0.00086287296956934\\
0.638	-0.000862872969569289\\
0.639999999999993	-0.000848619704775458\\
0.64	-0.000848619704775408\\
0.641999999999993	-0.000834598007511279\\
0.643999999999986	-0.000820802380553502\\
0.645999999999993	-0.000807227415273633\\
0.646	-0.000807227415273585\\
0.649999999999986	-0.0007807494801423\\
0.65	-0.000780749480142208\\
0.650000000000007	-0.000780749480142162\\
0.653999999999993	-0.00075515388929724\\
0.657999999999979	-0.000730400499675287\\
0.659999999999993	-0.000718327447732739\\
0.66	-0.000718327447732696\\
0.664999999999993	-0.000688992555410453\\
0.665	-0.000688992555410412\\
0.665999999999993	-0.000683266293778606\\
0.666	-0.000683266293778566\\
0.666999999999998	-0.000677587299556187\\
0.667000000000006	-0.000677587299556147\\
0.668000000000004	-0.000671956698056672\\
0.669000000000002	-0.000666373937422595\\
0.670999999999998	-0.00065534975471149\\
0.673000000000005	-0.000644510429349712\\
0.673000000000013	-0.000644510429349674\\
0.677000000000005	-0.000623369423345519\\
0.678	-0.000618193153072279\\
0.678000000000007	-0.000618193153072243\\
0.679999999999993	-0.000607967823817857\\
0.68	-0.00060796782381782\\
0.681999999999986	-0.000597908775826181\\
0.683999999999972	-0.000588012065407668\\
0.686	-0.000578273812518879\\
0.686000000000007	-0.000578273812518844\\
0.689999999999979	-0.000559280295899554\\
0.69399999999995	-0.000540921264682568\\
0.695999999999993	-0.000531970644944006\\
0.696	-0.000531970644943975\\
0.699999999999993	-0.000514509651760496\\
0.7	-0.000514509651760466\\
0.703999999999993	-0.000497612928358538\\
0.705999999999993	-0.000489367852791729\\
0.706	-0.0004893678527917\\
0.707999999999993	-0.000481258764669065\\
0.708	-0.000481258764669036\\
0.709999999999993	-0.000473287275730152\\
0.711999999999986	-0.00046545026072113\\
0.713999999999993	-0.000457744647110169\\
0.714	-0.000457744647110142\\
0.717999999999986	-0.000442715590537679\\
0.719999999999993	-0.000435386255375497\\
0.72	-0.000435386255375471\\
0.723999999999986	-0.000421083602878829\\
0.724999999999993	-0.000417580061616165\\
0.725	-0.00041758006161614\\
0.725999999999993	-0.000414104678126119\\
0.726	-0.000414104678126094\\
0.726999999999999	-0.000410658112971318\\
0.727999999999997	-0.000407241029513212\\
0.729999999999993	-0.000400493970904834\\
0.731999999999993	-0.000393860857090771\\
0.732	-0.000393860857090748\\
0.734999999999993	-0.000384119156237618\\
0.735	-0.000384119156237595\\
0.737999999999993	-0.000374619396625751\\
0.74	-0.000368416488403757\\
0.740000000000007	-0.000368416488403735\\
0.743	-0.000359301443504857\\
0.745999999999993	-0.000350406446863639\\
0.746000000000007	-0.000350406446863598\\
0.746999999999993	-0.000347489822771658\\
0.747	-0.000347489822771637\\
0.747999999999999	-0.000344598163091137\\
0.748999999999997	-0.000341731184360338\\
0.750999999999993	-0.000336070148164364\\
0.753999999999993	-0.000327756753293898\\
0.754	-0.000327756753293879\\
0.757999999999993	-0.000316993431217072\\
0.759999999999993	-0.000311744671911115\\
0.76	-0.000311744671911096\\
0.763999999999993	-0.000301502667102156\\
0.766	-0.000296505406181009\\
0.766000000000007	-0.000296505406180992\\
0.77	-0.000286759795598509\\
0.770000000000007	-0.000286759795598492\\
0.774	-0.000277340867131346\\
0.776	-0.000272749272973379\\
0.776000000000007	-0.000272749272973363\\
0.779999999999993	-0.000263792823452599\\
0.78	-0.000263792823452584\\
0.782999999999993	-0.000257266983875963\\
0.783	-0.000257266983875948\\
0.785999999999993	-0.00025089887536838\\
0.786000000000001	-0.000250898875368365\\
0.788999999999994	-0.000244688191913039\\
0.791999999999987	-0.000238634766370875\\
0.792	-0.000238634766370847\\
0.792000000000008	-0.000238634766370833\\
0.797999999999994	-0.000226978462895243\\
0.799999999999993	-0.00022322092671551\\
0.8	-0.000223220926715497\\
0.804999999999993	-0.000214092868713142\\
0.805000000000001	-0.00021409286871313\\
0.805999999999993	-0.000212311385418157\\
0.806	-0.000212311385418145\\
0.806999999999994	-0.000210544699061964\\
0.807999999999987	-0.00020879313470276\\
0.809999999999973	-0.000205334686754476\\
0.811999999999993	-0.000201934685675467\\
0.812	-0.000201934685675455\\
0.815999999999973	-0.000195304714664832\\
0.817999999999993	-0.000192072145426957\\
0.818000000000001	-0.000192072145426945\\
0.819999999999993	-0.000188892823472461\\
0.82	-0.00018889282347245\\
0.821999999999993	-0.000185765502352623\\
0.823999999999986	-0.000182688955988815\\
0.825999999999993	-0.000179661978209226\\
0.826	-0.000179661978209215\\
0.829999999999986	-0.000173758748889766\\
0.833999999999972	-0.00016805330440493\\
0.839999999999993	-0.000159846491347907\\
0.84	-0.000159846491347897\\
0.840999999999993	-0.000158517949109501\\
0.841000000000001	-0.000158517949109491\\
0.841999999999994	-0.000157200273212734\\
0.842999999999987	-0.000155893334450983\\
0.844999999999973	-0.000153311156997933\\
0.845999999999993	-0.000152035665221624\\
0.846	-0.000152035665221615\\
0.849999999999973	-0.000147040493095658\\
0.851999999999993	-0.000144606132116096\\
0.852	-0.000144606132116087\\
0.855999999999973	-0.000139859088005421\\
0.857999999999993	-0.000137544543776967\\
0.858	-0.000137544543776958\\
0.86	-0.00013526810311292\\
0.860000000000007	-0.000135268103112912\\
0.862000000000007	-0.000133028873537915\\
0.864000000000007	-0.000130825977153128\\
0.866	-0.000128658550304601\\
0.866000000000007	-0.000128658550304593\\
0.87	-0.000124431571562097\\
0.870000000000007	-0.000124431571562089\\
0.874	-0.000120346159599326\\
0.874999999999994	-0.000119346170814537\\
0.875000000000001	-0.00011934617081453\\
0.876	-0.000118354530447473\\
0.876000000000007	-0.000118354530447466\\
0.877000000000007	-0.000117371141310725\\
0.878000000000006	-0.000116395907018226\\
0.879999999999998	-0.000114469521423438\\
0.880000000000006	-0.000114469521423431\\
0.882000000000005	-0.000112574618421282\\
0.884000000000004	-0.000110710455112767\\
0.886000000000005	-0.000108876300645523\\
0.886000000000013	-0.000108876300645517\\
0.888000000000007	-0.000107072464331694\\
0.888000000000014	-0.000107072464331687\\
0.890000000000009	-0.000105299267369401\\
0.892000000000004	-0.000103556014569742\\
0.895999999999993	-0.000100156619152621\\
0.898999999999993	-9.76806754823634e-05\\
0.899000000000001	-9.76806754823576e-05\\
0.899999999999993	-9.68689464304756e-05\\
0.9	-9.68689464304698e-05\\
0.900999999999994	-9.6063877013777e-05\\
0.901999999999987	-9.52653882900104e-05\\
0.903999999999973	-9.36878405177649e-05\\
0.905999999999993	-9.21356846300636e-05\\
0.906	-9.21356846300581e-05\\
0.909999999999973	-8.91086105839449e-05\\
0.909999999999987	-8.91086105839346e-05\\
0.910000000000001	-8.91086105839244e-05\\
0.910999999999993	-8.83678628003285e-05\\
0.911000000000001	-8.83678628003233e-05\\
0.911999999999994	-8.76333771741875e-05\\
0.912999999999987	-8.69050817155083e-05\\
0.914999999999973	-8.54667763781069e-05\\
0.916999999999993	-8.40523828949788e-05\\
0.917000000000001	-8.40523828949738e-05\\
0.919999999999993	-8.19744169173503e-05\\
0.92	-8.19744169173455e-05\\
0.922999999999993	-7.99471740298812e-05\\
0.925999999999986	-7.79688658961427e-05\\
0.925999999999993	-7.7968865896138e-05\\
0.926	-7.79688658961333e-05\\
0.927999999999993	-7.66770543219352e-05\\
0.928000000000001	-7.66770543219306e-05\\
0.929999999999994	-7.5407184820723e-05\\
0.931999999999987	-7.41587595138373e-05\\
0.933999999999994	-7.29312889513422e-05\\
0.934000000000001	-7.29312889513379e-05\\
0.937999999999987	-7.0537295177499e-05\\
0.939999999999993	-6.93698333914894e-05\\
0.940000000000001	-6.93698333914852e-05\\
0.943999999999987	-6.70916913305193e-05\\
0.944999999999994	-6.65336590054775e-05\\
0.945000000000001	-6.65336590054735e-05\\
0.945999999999993	-6.59801178998789e-05\\
0.946000000000001	-6.59801178998749e-05\\
0.946999999999994	-6.54311699499357e-05\\
0.947999999999987	-6.48869175379424e-05\\
0.949999999999973	-6.3812286413191e-05\\
0.952	-6.27558032123494e-05\\
0.952000000000008	-6.27558032123457e-05\\
0.95599999999998	-6.06956307672097e-05\\
0.956999999999994	-6.01912913074925e-05\\
0.957000000000001	-6.01912913074889e-05\\
0.96	-5.87031691074909e-05\\
0.960000000000008	-5.87031691074874e-05\\
0.963000000000007	-5.72513743971698e-05\\
0.966000000000007	-5.58346264728782e-05\\
0.966000000000014	-5.58346264728749e-05\\
0.969000000000007	-5.44528649793914e-05\\
0.969000000000014	-5.44528649793881e-05\\
0.972000000000007	-5.31060604824633e-05\\
0.975	-5.17930248772815e-05\\
0.979999999999994	-4.9676541243674e-05\\
0.980000000000001	-4.9676541243671e-05\\
0.985999999999987	-4.72490603455223e-05\\
0.986000000000001	-4.72490603455167e-05\\
0.991999999999987	-4.49400389592495e-05\\
0.992000000000001	-4.49400389592443e-05\\
0.997999999999987	-4.27453534169505e-05\\
0.998000000000001	-4.27453534169455e-05\\
0.999999999999993	-4.20378496340661e-05\\
1	-4.20378496340636e-05\\
1.00199999999999	-4.13419087208149e-05\\
1.00399999999999	-4.06572578191729e-05\\
1.00599999999999	-3.99836285091284e-05\\
1.006	-3.99836285091237e-05\\
1.00999999999999	-3.86698957854157e-05\\
1.01399999999997	-3.74001634276287e-05\\
1.01499999999999	-3.708937054333e-05\\
1.015	-3.70893705433256e-05\\
1.01999999999999	-3.55737274428047e-05\\
1.02	-3.55737274428004e-05\\
1.02499999999999	-3.41192521065527e-05\\
1.02599999999999	-3.3835380944933e-05\\
1.026	-3.3835380944929e-05\\
1.02699999999999	-3.35538656382162e-05\\
1.027	-3.35538656382122e-05\\
1.02799999999999	-3.32747585990048e-05\\
1.02899999999999	-3.29980324694628e-05\\
1.03099999999997	-3.24516146819163e-05\\
1.03299999999999	-3.19143980503788e-05\\
1.033	-3.1914398050375e-05\\
1.03699999999997	-3.08667293216363e-05\\
1.04	-3.01035897706801e-05\\
1.04000000000001	-3.01035897706765e-05\\
1.04399999999999	-2.91149397160916e-05\\
1.044	-2.91149397160882e-05\\
1.046	-2.86325503103997e-05\\
1.04600000000001	-2.86325503103963e-05\\
1.04800000000001	-2.81581355429319e-05\\
1.05	-2.76917801657761e-05\\
1.05000000000001	-2.76917801657728e-05\\
1.05200000000001	-2.723330134373e-05\\
1.05200000000002	-2.72333013437268e-05\\
1.05400000000002	-2.67825193296244e-05\\
1.05600000000002	-2.63392573938291e-05\\
1.05800000000001	-2.59033417538038e-05\\
1.05800000000002	-2.59033417538007e-05\\
1.05999999999999	-2.5474601509497e-05\\
1.06	-2.54746015094939e-05\\
1.06199999999996	-2.50528685738976e-05\\
1.06399999999992	-2.46379776049399e-05\\
1.06599999999999	-2.42297659430081e-05\\
1.066	-2.42297659430052e-05\\
1.06999999999992	-2.34336593983795e-05\\
1.07299999999999	-2.28541444051524e-05\\
1.073	-2.28541444051497e-05\\
1.07699999999992	-2.2103905639253e-05\\
1.07899999999999	-2.17380751798321e-05\\
1.079	-2.17380751798295e-05\\
1.07999999999999	-2.15574190393219e-05\\
1.08	-2.15574190393193e-05\\
1.08099999999999	-2.13782453018156e-05\\
1.08199999999999	-2.12005364057517e-05\\
1.08399999999997	-2.08494436105259e-05\\
1.08499999999999	-2.0676025299972e-05\\
1.085	-2.06760252999696e-05\\
1.08599999999999	-2.05040030060072e-05\\
1.086	-2.05040030060048e-05\\
1.08699999999999	-2.03334083333607e-05\\
1.08799999999999	-2.01642730254155e-05\\
1.08999999999997	-1.98303143369535e-05\\
1.09199999999999	-1.9501996010831e-05\\
1.092	-1.95019960108287e-05\\
1.09599999999997	-1.88617677405113e-05\\
1.09999999999995	-1.82425841060603e-05\\
1.09999999999997	-1.82425841060561e-05\\
1.1	-1.8242584106052e-05\\
1.10199999999999	-1.79405792921785e-05\\
1.102	-1.79405792921763e-05\\
1.10399999999999	-1.76434739873749e-05\\
1.10599999999997	-1.73511516923925e-05\\
1.10599999999999	-1.73511516923903e-05\\
1.106	-1.73511516923883e-05\\
1.10999999999997	-1.6781055835308e-05\\
1.11199999999999	-1.65032228232008e-05\\
1.112	-1.65032228231988e-05\\
1.11599999999997	-1.59614419683402e-05\\
1.11999999999994	-1.54374695874575e-05\\
1.12	-1.54374695874501e-05\\
1.12000000000001	-1.54374695874483e-05\\
1.126	-1.46831115783722e-05\\
1.12600000000001	-1.46831115783705e-05\\
1.13099999999999	-1.40826238211678e-05\\
1.131	-1.40826238211662e-05\\
1.132	-1.39655673521018e-05\\
1.13200000000001	-1.39655673521001e-05\\
1.13300000000001	-1.38494975788315e-05\\
1.13400000000001	-1.37344031242957e-05\\
1.13600000000001	-1.35070951440944e-05\\
1.138	-1.32835542945336e-05\\
1.13800000000001	-1.3283554294532e-05\\
1.13999999999999	-1.30636929367676e-05\\
1.14	-1.30636929367661e-05\\
1.14199999999997	-1.28474248744612e-05\\
1.14399999999994	-1.26346653188649e-05\\
1.14599999999999	-1.24253308567719e-05\\
1.146	-1.24253308567705e-05\\
1.14999999999994	-1.2017080284122e-05\\
1.15399999999989	-1.16225029095039e-05\\
1.15499999999999	-1.15259220212841e-05\\
1.155	-1.15259220212827e-05\\
1.15999999999999	-1.10549259910857e-05\\
1.16	-1.10549259910844e-05\\
1.16499999999999	-1.06029379264291e-05\\
1.16599999999999	-1.05147229999796e-05\\
1.166	-1.05147229999784e-05\\
1.17099999999999	-1.00847077908383e-05\\
1.17199999999999	-1.00008824871519e-05\\
1.172	-1.00008824871507e-05\\
1.173	-9.91776376173988e-06\\
1.17300000000001	-9.9177637617387e-06\\
1.17400000000001	-9.83534346881401e-06\\
1.17500000000001	-9.75361352960694e-06\\
1.17700000000001	-9.59219273768623e-06\\
1.17999999999999	-9.35504110010023e-06\\
1.18	-9.35504110009912e-06\\
1.184	-9.04780999310616e-06\\
1.18599999999999	-8.89790339915877e-06\\
1.186	-8.89790339915772e-06\\
1.18899999999999	-8.6777034118826e-06\\
1.189	-8.67770341188157e-06\\
1.18999999999999	-8.60555087815935e-06\\
1.19	-8.60555087815833e-06\\
1.19099999999999	-8.53401023871499e-06\\
1.19199999999999	-8.46307448174067e-06\\
1.19399999999997	-8.32298986403477e-06\\
1.19499999999999	-8.2538272732956e-06\\
1.195	-8.25382727329462e-06\\
1.19899999999997	-7.98288418604953e-06\\
1.19999999999999	-7.91654203892987e-06\\
1.2	-7.91654203892893e-06\\
1.20399999999997	-7.65655286739852e-06\\
1.20599999999999	-7.52969692657781e-06\\
1.206	-7.52969692657691e-06\\
1.20699999999999	-7.46704952780753e-06\\
1.207	-7.46704952780665e-06\\
1.20799999999999	-7.40493804245904e-06\\
1.20899999999999	-7.34335638272751e-06\\
1.21099999999997	-7.22175844887707e-06\\
1.21499999999995	-6.98465832714022e-06\\
1.21799999999999	-6.81198437922706e-06\\
1.218	-6.81198437922625e-06\\
1.21999999999999	-6.69923636513134e-06\\
1.22	-6.69923636513054e-06\\
1.22199999999999	-6.5883310412403e-06\\
1.22399999999997	-6.47922491998147e-06\\
1.22499999999999	-6.42533316579872e-06\\
1.225	-6.42533316579796e-06\\
1.22599999999999	-6.37187522576968e-06\\
1.226	-6.37187522576892e-06\\
1.22699999999999	-6.3188609252805e-06\\
1.22799999999999	-6.26630013280449e-06\\
1.22999999999997	-6.16251850981608e-06\\
1.23	-6.1625185098147e-06\\
1.23399999999997	-5.96017361010833e-06\\
1.236	-5.86153100355807e-06\\
1.23600000000001	-5.86153100355737e-06\\
1.23999999999999	-5.66911209821449e-06\\
1.24	-5.66911209821382e-06\\
1.24399999999997	-5.48293117095466e-06\\
1.24599999999999	-5.39208832980723e-06\\
1.246	-5.39208832980659e-06\\
1.247	-5.34722588613589e-06\\
1.24700000000001	-5.34722588613526e-06\\
1.24800000000001	-5.30274721643946e-06\\
1.24900000000001	-5.25864796117829e-06\\
1.25100000000001	-5.17157044198016e-06\\
1.253	-5.08595918946622e-06\\
1.25300000000001	-5.08595918946562e-06\\
1.25700000000001	-4.91900179135527e-06\\
1.25999999999999	-4.79738711157346e-06\\
1.26	-4.79738711157289e-06\\
1.264	-4.63983471791204e-06\\
1.266	-4.56296053566249e-06\\
1.26600000000001	-4.56296053566195e-06\\
1.27000000000001	-4.41303812081527e-06\\
1.272	-4.33997426245643e-06\\
1.27200000000001	-4.33997426245592e-06\\
1.276	-4.19749800956031e-06\\
1.27600000000001	-4.19749800955981e-06\\
1.27999999999999	-4.05970495345541e-06\\
1.28000000000001	-4.05970495345493e-06\\
1.28399999999999	-3.92637898522597e-06\\
1.28599999999999	-3.86132554959552e-06\\
1.28600000000001	-3.86132554959506e-06\\
1.288	-3.79734751779369e-06\\
1.28800000000001	-3.79734751779324e-06\\
1.29	-3.73445632351069e-06\\
1.29199999999999	-3.67262731000453e-06\\
1.29499999999999	-3.58182246485316e-06\\
1.295	-3.58182246485274e-06\\
1.29899999999998	-3.46424411588173e-06\\
1.29999999999999	-3.4354543087659e-06\\
1.3	-3.4354543087655e-06\\
1.30399999999998	-3.32262954319313e-06\\
1.30499999999999	-3.29499314494559e-06\\
1.305	-3.2949931449452e-06\\
1.30599999999999	-3.26757921342542e-06\\
1.306	-3.26757921342503e-06\\
1.30699999999999	-3.24039278693323e-06\\
1.30700000000001	-3.24039278693284e-06\\
1.308	-3.21343892586474e-06\\
1.30899999999999	-3.18671498846163e-06\\
1.31099999999998	-3.13394643003336e-06\\
1.31300000000001	-3.08206642353151e-06\\
1.31300000000002	-3.08206642353115e-06\\
1.31699999999999	-2.98089104891791e-06\\
1.31999999999999	-2.9071931457201e-06\\
1.32	-2.90719314571975e-06\\
1.32399999999997	-2.81171716818259e-06\\
1.32599999999999	-2.76513179317114e-06\\
1.326	-2.76513179317081e-06\\
1.32999999999997	-2.67427955214637e-06\\
1.33	-2.67427955214579e-06\\
1.33399999999997	-2.5864701600707e-06\\
1.334	-2.58647016007012e-06\\
1.33799999999997	-2.50156590033478e-06\\
1.34	-2.46016141016012e-06\\
1.34000000000001	-2.46016141015982e-06\\
1.34399999999999	-2.37936653988299e-06\\
1.346	-2.33994448383389e-06\\
1.34600000000001	-2.33994448383361e-06\\
1.348	-2.30117411735104e-06\\
1.34800000000001	-2.30117411735077e-06\\
1.35	-2.26306236891223e-06\\
1.35199999999999	-2.22559429665541e-06\\
1.35599999999996	-2.15253066967173e-06\\
1.35999999999999	-2.08186864330043e-06\\
1.36	-2.08186864330018e-06\\
1.36299999999999	-2.0303813715584e-06\\
1.363	-2.03038137155816e-06\\
1.36499999999999	-1.99674987871469e-06\\
1.365	-1.99674987871446e-06\\
1.36599999999999	-1.98013717357585e-06\\
1.366	-1.98013717357561e-06\\
1.36699999999999	-1.96366233525072e-06\\
1.36799999999999	-1.9473284303516e-06\\
1.36999999999997	-1.91507703091651e-06\\
1.37199999999999	-1.8833703309852e-06\\
1.372	-1.88337033098498e-06\\
1.37599999999997	-1.82154151613983e-06\\
1.378	-1.79139516099549e-06\\
1.37800000000001	-1.79139516099528e-06\\
1.37999999999999	-1.76174501581677e-06\\
1.38	-1.76174501581656e-06\\
1.38199999999997	-1.73257945630545e-06\\
1.38399999999994	-1.70388704798853e-06\\
1.38599999999999	-1.67565654189632e-06\\
1.386	-1.67565654189612e-06\\
1.38999999999994	-1.62060053392155e-06\\
1.39199999999999	-1.59376929346458e-06\\
1.392	-1.59376929346439e-06\\
1.39599999999994	-1.54144773968102e-06\\
1.39799999999999	-1.51593691348235e-06\\
1.398	-1.51593691348217e-06\\
1.39999999999999	-1.49084599621229e-06\\
1.4	-1.49084599621211e-06\\
1.40199999999999	-1.46616515101157e-06\\
1.40399999999997	-1.4418847016572e-06\\
1.40599999999999	-1.41799512890556e-06\\
1.406	-1.41799512890539e-06\\
1.40999999999997	-1.3714049434806e-06\\
1.412	-1.34869947465724e-06\\
1.41200000000001	-1.34869947465708e-06\\
1.41599999999999	-1.30442327301461e-06\\
1.41999999999996	-1.26160243071545e-06\\
1.41999999999998	-1.26160243071522e-06\\
1.42	-1.261602430715e-06\\
1.42099999999999	-1.25111670079919e-06\\
1.421	-1.25111670079905e-06\\
1.42199999999999	-1.24071669557977e-06\\
1.42299999999999	-1.23040139526944e-06\\
1.42499999999997	-1.21002087355164e-06\\
1.42599999999999	-1.19995365460354e-06\\
1.426	-1.1999536546034e-06\\
1.42999999999997	-1.16052752266519e-06\\
1.43199999999999	-1.14131341558744e-06\\
1.432	-1.1413134155873e-06\\
1.43499999999999	-1.11309472116344e-06\\
1.435	-1.11309472116331e-06\\
1.43799999999998	-1.08557690585826e-06\\
1.43999999999999	-1.06760906053696e-06\\
1.44	-1.06760906053683e-06\\
1.44299999999999	-1.0412057285083e-06\\
1.44599999999997	-1.01543985861267e-06\\
1.44599999999999	-1.01543985861253e-06\\
1.446	-1.0154398586124e-06\\
1.44699999999999	-1.00699134973081e-06\\
1.447	-1.00699134973069e-06\\
1.44799999999999	-9.98615113433553e-07\\
1.44899999999999	-9.90310328620577e-07\\
1.44999999999999	-9.82076181335089e-07\\
1.45	-9.82076181334972e-07\\
1.45199999999999	-9.65816578069283e-07\\
1.45399999999997	-9.49829929037836e-07\\
1.45599999999999	-9.34109966613451e-07\\
1.456	-9.3410996661334e-07\\
1.45999999999997	-9.03445551816085e-07\\
1.46	-9.03445551815871e-07\\
1.46000000000001	-9.03445551815764e-07\\
1.46399999999999	-8.73775240836027e-07\\
1.466	-8.59298273570955e-07\\
1.46600000000001	-8.59298273570853e-07\\
1.46999999999999	-8.3106484305882e-07\\
1.47	-8.31064843058722e-07\\
1.47399999999997	-8.03777013858898e-07\\
1.47599999999999	-7.90474269540854e-07\\
1.476	-7.90474269540761e-07\\
1.47899999999998	-7.70931916199948e-07\\
1.479	-7.70931916199856e-07\\
1.47999999999999	-7.64525042761406e-07\\
1.48	-7.64525042761315e-07\\
1.48099999999999	-7.58170740579864e-07\\
1.48199999999999	-7.51868386860326e-07\\
1.48399999999997	-7.39417059053101e-07\\
1.48599999999999	-7.27166177748748e-07\\
1.486	-7.27166177748662e-07\\
1.48999999999997	-7.0327412912771e-07\\
1.491	-6.97427583694441e-07\\
1.49100000000001	-6.97427583694358e-07\\
1.49499999999999	-6.74530076842024e-07\\
1.49899999999996	-6.52387687670176e-07\\
1.49999999999999	-6.46965982437778e-07\\
1.5	-6.46965982437701e-07\\
1.50499999999999	-6.20514293231628e-07\\
1.505	-6.20514293231554e-07\\
1.50599999999999	-6.15351693299492e-07\\
1.506	-6.15351693299419e-07\\
1.50699999999999	-6.10231937175109e-07\\
1.50799999999999	-6.05155977810956e-07\\
1.508	-6.05155977810884e-07\\
1.50999999999999	-5.951334636196e-07\\
1.51199999999997	-5.85280221386688e-07\\
1.51399999999999	-5.75592388110665e-07\\
1.514	-5.75592388110597e-07\\
1.51799999999997	-5.56697819413171e-07\\
1.518	-5.56697819413046e-07\\
1.51999999999999	-5.47483676301743e-07\\
1.52	-5.47483676301678e-07\\
1.52199999999999	-5.38420124017042e-07\\
1.52399999999997	-5.29503609157478e-07\\
1.52599999999999	-5.20730635969083e-07\\
1.526	-5.20730635969022e-07\\
1.52999999999997	-5.03621311646199e-07\\
1.53399999999994	-4.8708501728248e-07\\
1.537	-4.75043212661656e-07\\
1.53700000000001	-4.750432126616e-07\\
1.53999999999999	-4.63298508796156e-07\\
1.54	-4.63298508796101e-07\\
1.54299999999997	-4.51840545741969e-07\\
1.54599999999994	-4.40659215145519e-07\\
1.54599999999997	-4.40659215145417e-07\\
1.546	-4.40659215145314e-07\\
1.549	-4.2975402991908e-07\\
1.54900000000001	-4.29754029919029e-07\\
1.55200000000001	-4.1912474703272e-07\\
1.55500000000001	-4.08761989696897e-07\\
1.55500000000003	-4.08761989696848e-07\\
1.55999999999999	-3.9205828064636e-07\\
1.56	-3.92058280646314e-07\\
1.56499999999996	-3.76028684247008e-07\\
1.56599999999998	-3.72900173450336e-07\\
1.566	-3.72900173450292e-07\\
1.57099999999996	-3.57649839507866e-07\\
1.57199999999998	-3.54677005739556e-07\\
1.572	-3.54677005739514e-07\\
1.57499999999999	-3.45907703154635e-07\\
1.575	-3.45907703154594e-07\\
1.57799999999999	-3.37356207353412e-07\\
1.57999999999999	-3.31772480900963e-07\\
1.58	-3.31772480900924e-07\\
1.58299999999999	-3.23567324397437e-07\\
1.58599999999997	-3.15560267043704e-07\\
1.586	-3.15560267043631e-07\\
1.58699999999999	-3.12934790192868e-07\\
1.587	-3.1293479019283e-07\\
1.58799999999999	-3.10331772929613e-07\\
1.58899999999999	-3.07750960087046e-07\\
1.59099999999997	-3.02654938029343e-07\\
1.59499999999995	-2.92718360093961e-07\\
1.59499999999997	-2.92718360093896e-07\\
1.595	-2.9271836009383e-07\\
1.59999999999999	-2.80756674576221e-07\\
1.6	-2.80756674576188e-07\\
1.60499999999999	-2.69277727852186e-07\\
1.60599999999999	-2.67037371457235e-07\\
1.606	-2.67037371457203e-07\\
1.60699999999998	-2.64815607498324e-07\\
1.607	-2.64815607498293e-07\\
1.60799999999999	-2.62612849555249e-07\\
1.60899999999999	-2.60428881715541e-07\\
1.60999999999998	-2.58263489927241e-07\\
1.61	-2.5826348992721e-07\\
1.61199999999999	-2.53987587382857e-07\\
1.61399999999997	-2.49783465549024e-07\\
1.61599999999999	-2.45649476183591e-07\\
1.616	-2.45649476183562e-07\\
1.61999999999997	-2.37585438831863e-07\\
1.61999999999999	-2.37585438831834e-07\\
1.62	-2.37585438831805e-07\\
1.62399999999997	-2.29782828014192e-07\\
1.624	-2.29782828014144e-07\\
1.62599999999999	-2.25975717904888e-07\\
1.626	-2.25975717904861e-07\\
1.62799999999999	-2.22231543444522e-07\\
1.62999999999998	-2.18550973730036e-07\\
1.63199999999999	-2.14932565779908e-07\\
1.632	-2.14932565779882e-07\\
1.63599999999998	-2.07876584631267e-07\\
1.63999999999995	-2.01052533680171e-07\\
1.63999999999998	-2.0105253368013e-07\\
1.64	-2.0105253368009e-07\\
1.645	-1.92832350197346e-07\\
1.64500000000001	-1.92832350197323e-07\\
1.64599999999999	-1.9122800961036e-07\\
1.646	-1.91228009610337e-07\\
1.64699999999999	-1.89636983257357e-07\\
1.64799999999999	-1.88059567279348e-07\\
1.64999999999997	-1.84944949354369e-07\\
1.65199999999999	-1.81882934737465e-07\\
1.652	-1.81882934737443e-07\\
1.65299999999998	-1.80371277732718e-07\\
1.653	-1.80371277732697e-07\\
1.65399999999999	-1.78872322973086e-07\\
1.65499999999997	-1.77385923539781e-07\\
1.65699999999995	-1.7445020913795e-07\\
1.65899999999998	-1.71562983569048e-07\\
1.659	-1.71562983569027e-07\\
1.65999999999999	-1.70137199636775e-07\\
1.66	-1.70137199636754e-07\\
1.66099999999999	-1.68723114903204e-07\\
1.66199999999998	-1.67320590768011e-07\\
1.66399999999995	-1.64549675564482e-07\\
1.66599999999999	-1.61823367678176e-07\\
1.666	-1.61823367678156e-07\\
1.66999999999995	-1.56506437506861e-07\\
1.67399999999991	-1.51367583346046e-07\\
1.67999999999998	-1.43975637480808e-07\\
1.68	-1.43975637480791e-07\\
1.68199999999998	-1.41592131203646e-07\\
1.682	-1.41592131203629e-07\\
1.68399999999998	-1.39247292542355e-07\\
1.68599999999997	-1.36940202052067e-07\\
1.68599999999999	-1.36940202052049e-07\\
1.686	-1.36940202052033e-07\\
1.68999999999997	-1.32440842633399e-07\\
1.69199999999999	-1.30248104741518e-07\\
1.692	-1.30248104741502e-07\\
1.69599999999997	-1.25972213951827e-07\\
1.69799999999999	-1.23887384674211e-07\\
1.698	-1.23887384674196e-07\\
1.69999999999999	-1.21836871740851e-07\\
1.7	-1.21836871740837e-07\\
1.70199999999999	-1.19819871251032e-07\\
1.70399999999998	-1.17835592431731e-07\\
1.70599999999999	-1.15883257338806e-07\\
1.706	-1.15883257338792e-07\\
1.70999999999998	-1.12075752908577e-07\\
1.71099999999998	-1.1114403089721e-07\\
1.711	-1.11144030897197e-07\\
1.71499999999997	-1.0749501946948e-07\\
1.715	-1.07495019469458e-07\\
1.71699999999998	-1.05715990597716e-07\\
1.717	-1.05715990597704e-07\\
1.71899999999998	-1.03966345737004e-07\\
1.71999999999999	-1.03102327506724e-07\\
1.72	-1.03102327506712e-07\\
1.72199999999999	-1.01395476024732e-07\\
1.72399999999997	-9.97163146860125e-08\\
1.72599999999999	-9.80641851688996e-08\\
1.726	-9.8064185168888e-08\\
1.72899999999998	-9.56373482049413e-08\\
1.729	-9.56373482049299e-08\\
1.73199999999998	-9.32719104375856e-08\\
1.73499999999997	-9.09657851911036e-08\\
1.73999999999998	-8.72485462369585e-08\\
1.74	-8.72485462369481e-08\\
1.74599999999997	-8.29851061664371e-08\\
1.74599999999998	-8.29851061664257e-08\\
1.746	-8.29851061664141e-08\\
1.74999999999998	-8.02585158985349e-08\\
1.75	-8.02585158985254e-08\\
1.75199999999998	-7.89297272427249e-08\\
1.752	-7.89297272427156e-08\\
1.75399999999998	-7.76232453177377e-08\\
1.75599999999997	-7.6338557912159e-08\\
1.75799999999998	-7.50751613593442e-08\\
1.758	-7.50751613593353e-08\\
1.75999999999999	-7.38325603466569e-08\\
1.76	-7.38325603466481e-08\\
1.76199999999999	-7.26102677141145e-08\\
1.76399999999998	-7.1407804257055e-08\\
1.76599999999999	-7.02246985450567e-08\\
1.766	-7.02246985450484e-08\\
1.76899999999998	-6.8486817437662e-08\\
1.769	-6.84868174376539e-08\\
1.77199999999998	-6.67929048806059e-08\\
1.77499999999996	-6.5141466585487e-08\\
1.77499999999998	-6.51414665854777e-08\\
1.775	-6.51414665854685e-08\\
1.78	-6.24795162982505e-08\\
1.78000000000002	-6.24795162982431e-08\\
1.78499999999998	-5.99249944998296e-08\\
1.785	-5.99249944998225e-08\\
1.78600000000001	-5.94264261737771e-08\\
1.78600000000003	-5.94264261737701e-08\\
1.78700000000005	-5.89319954077242e-08\\
1.78800000000006	-5.84417942311079e-08\\
1.79000000000009	-5.7473888876745e-08\\
1.79200000000003	-5.65223306397248e-08\\
1.79200000000004	-5.65223306397181e-08\\
1.79600000000011	-5.46667695547242e-08\\
1.79799999999998	-5.37620392268632e-08\\
1.798	-5.37620392268568e-08\\
1.8	-5.28722007837155e-08\\
1.80000000000002	-5.28722007837092e-08\\
1.80200000000002	-5.19969053654383e-08\\
1.80400000000002	-5.11358098090003e-08\\
1.806	-5.02885765185956e-08\\
1.80600000000002	-5.02885765185896e-08\\
1.80999999999998	-4.86362758870654e-08\\
1.81	-4.86362758870596e-08\\
1.81399999999997	-4.70393145393403e-08\\
1.81799999999994	-4.54951878674641e-08\\
1.82	-4.47421775616881e-08\\
1.82000000000001	-4.47421775616828e-08\\
1.826	-4.25558305689674e-08\\
1.82600000000001	-4.25558305689624e-08\\
1.827	-4.22017639742739e-08\\
1.82700000000001	-4.22017639742689e-08\\
1.828	-4.18507262349338e-08\\
1.82899999999999	-4.15026829396494e-08\\
1.83099999999996	-4.0815443523925e-08\\
1.83200000000001	-4.04761800455945e-08\\
1.83200000000003	-4.04761800455897e-08\\
1.83599999999998	-3.91473950511301e-08\\
1.83800000000001	-3.84995090327869e-08\\
1.83800000000003	-3.84995090327824e-08\\
1.83999999999999	-3.78622872354582e-08\\
1.84	-3.78622872354538e-08\\
1.84199999999996	-3.72354798372176e-08\\
1.84399999999992	-3.66188410957449e-08\\
1.84599999999999	-3.60121292554251e-08\\
1.846	-3.60121292554209e-08\\
1.84999999999992	-3.48289010140466e-08\\
1.85399999999984	-3.36853017625717e-08\\
1.85499999999998	-3.34053824571342e-08\\
1.855	-3.34053824571302e-08\\
1.85599999999998	-3.31278001407826e-08\\
1.856	-3.31278001407786e-08\\
1.85699999999999	-3.2852527619997e-08\\
1.85799999999997	-3.25795379025201e-08\\
1.85999999999995	-3.20403000752706e-08\\
1.85999999999998	-3.20403000752638e-08\\
1.86	-3.20403000752569e-08\\
1.86399999999995	-3.09880554735901e-08\\
1.86599999999999	-3.04746361639989e-08\\
1.866	-3.04746361639953e-08\\
1.86799999999998	-2.99697042299132e-08\\
1.868	-2.99697042299096e-08\\
1.86999999999998	-2.94733499071916e-08\\
1.87199999999996	-2.89853785981544e-08\\
1.87299999999998	-2.87444766622487e-08\\
1.873	-2.87444766622453e-08\\
1.87699999999996	-2.78008784246803e-08\\
1.87999999999999	-2.71135450349128e-08\\
1.88	-2.71135450349096e-08\\
1.88399999999997	-2.62231013986987e-08\\
1.88499999999998	-2.60049874485378e-08\\
1.885	-2.60049874485347e-08\\
1.886	-2.57886292648668e-08\\
1.88600000000002	-2.57886292648638e-08\\
1.88700000000002	-2.55740666113488e-08\\
1.88800000000002	-2.53613394260269e-08\\
1.88999999999998	-2.4941308239916e-08\\
1.89	-2.4941308239913e-08\\
1.892	-2.45283710327764e-08\\
1.89200000000002	-2.45283710327735e-08\\
1.89400000000002	-2.41223659120891e-08\\
1.89600000000003	-2.37231337030588e-08\\
1.898	-2.33305178851724e-08\\
1.89800000000002	-2.33305178851696e-08\\
1.9	-2.29443645339956e-08\\
1.90000000000002	-2.29443645339929e-08\\
1.902	-2.2564522258656e-08\\
1.90399999999998	-2.2190842140487e-08\\
1.906	-2.18231776767582e-08\\
1.90600000000002	-2.18231776767556e-08\\
1.90999999999998	-2.11061470402425e-08\\
1.91399999999995	-2.04131313808562e-08\\
1.91399999999997	-2.04131313808518e-08\\
1.914	-2.04131313808474e-08\\
1.91999999999998	-1.94162682523969e-08\\
1.92	-1.94162682523946e-08\\
1.92499999999998	-1.86224195898359e-08\\
1.925	-1.86224195898337e-08\\
1.92599999999998	-1.84674834298463e-08\\
1.926	-1.84674834298441e-08\\
1.92699999999999	-1.8313833066864e-08\\
1.92799999999997	-1.81614971001522e-08\\
1.92999999999995	-1.78607087608587e-08\\
1.93199999999998	-1.75650004867345e-08\\
1.932	-1.75650004867324e-08\\
1.93599999999995	-1.69883623530699e-08\\
1.9399999999999	-1.64306783229831e-08\\
1.93999999999999	-1.64306783229713e-08\\
1.94	-1.64306783229694e-08\\
1.94299999999998	-1.60243266729119e-08\\
1.943	-1.602432667291e-08\\
1.94599999999998	-1.56277856954226e-08\\
1.946	-1.56277856954198e-08\\
1.94899999999998	-1.52410380971515e-08\\
1.95199999999997	-1.48640752480954e-08\\
1.95199999999998	-1.48640752480932e-08\\
1.952	-1.4864075248091e-08\\
1.95799999999997	-1.41381819830471e-08\\
1.95999999999998	-1.39041748992468e-08\\
1.96	-1.39041748992451e-08\\
1.96599999999996	-1.32247410380267e-08\\
1.96599999999998	-1.32247410380248e-08\\
1.966	-1.32247410380228e-08\\
1.97199999999996	-1.2578464393245e-08\\
1.97199999999998	-1.25784643932432e-08\\
1.972	-1.25784643932414e-08\\
1.97799999999996	-1.19641898798447e-08\\
1.97799999999998	-1.1964189879843e-08\\
1.978	-1.19641898798412e-08\\
1.98	-1.17661654672371e-08\\
1.98000000000002	-1.17661654672357e-08\\
1.98200000000002	-1.15713774588398e-08\\
1.98400000000002	-1.13797494839859e-08\\
1.986	-1.11912064141954e-08\\
1.98600000000002	-1.11912064141941e-08\\
1.99000000000002	-1.08235038738479e-08\\
1.99400000000003	-1.04681165230686e-08\\
1.995	-1.03811282006563e-08\\
1.99500000000001	-1.03811282006551e-08\\
1.99999999999999	-9.95691227019017e-09\\
2	-9.95691227018899e-09\\
2.00099999999997	-9.87415601302283e-09\\
2.001	-9.87415601302048e-09\\
2.00199999999999	-9.79207631566493e-09\\
2.00299999999997	-9.71066513273897e-09\\
2.00499999999995	-9.54981645834466e-09\\
2.00599999999997	-9.47036320173591e-09\\
2.006	-9.47036320173366e-09\\
2.00999999999995	-9.15920134191744e-09\\
2.01199999999997	-9.00755833307511e-09\\
2.012	-9.00755833307297e-09\\
2.01299999999998	-8.93269513185397e-09\\
2.01300000000001	-8.93269513185185e-09\\
2.014	-8.85846099580636e-09\\
2.01499999999999	-8.78484864893714e-09\\
2.01699999999996	-8.63946052385991e-09\\
2.01999999999997	-8.42586332845568e-09\\
2.02	-8.42586332845369e-09\\
2.02399999999995	-8.14914715807546e-09\\
2.02599999999997	-8.0141296679016e-09\\
2.026	-8.01412966789969e-09\\
2.02999999999995	-7.7508143724346e-09\\
2.03	-7.75081437243172e-09\\
2.03399999999995	-7.49631810159424e-09\\
2.03599999999997	-7.37225184535954e-09\\
2.036	-7.37225184535779e-09\\
2.03999999999995	-7.13023987191646e-09\\
2.04	-7.13023987191334e-09\\
2.04399999999995	-6.89607363978448e-09\\
2.04599999999997	-6.78181744392865e-09\\
2.046	-6.78181744392704e-09\\
2.04799999999997	-6.66945002629583e-09\\
2.048	-6.66945002629424e-09\\
2.04999999999996	-6.55899146794323e-09\\
2.05199999999993	-6.450398463156e-09\\
2.05599999999987	-6.23863953290441e-09\\
2.05899999999997	-6.08440591409081e-09\\
2.059	-6.08440591408937e-09\\
2.05999999999997	-6.03384111045324e-09\\
2.06	-6.03384111045181e-09\\
2.06099999999999	-5.98369121589198e-09\\
2.06199999999998	-5.93395131288756e-09\\
2.06399999999995	-5.83568202113935e-09\\
2.06499999999997	-5.78714300080149e-09\\
2.065	-5.78714300080012e-09\\
2.06599999999997	-5.73899470825349e-09\\
2.066	-5.73899470825212e-09\\
2.06699999999999	-5.69124599249692e-09\\
2.06799999999998	-5.64390574134267e-09\\
2.06999999999995	-5.55043211306818e-09\\
2.07099999999997	-5.50428957438758e-09\\
2.071	-5.50428957438628e-09\\
2.07499999999995	-5.32357617608778e-09\\
2.07699999999997	-5.23547166903123e-09\\
2.077	-5.23547166902999e-09\\
2.07999999999997	-5.10603279254529e-09\\
2.08	-5.10603279254408e-09\\
2.08299999999998	-4.9797540912234e-09\\
2.08599999999995	-4.8565241680537e-09\\
2.086	-4.85652416805177e-09\\
2.08799999999997	-4.77605678697469e-09\\
2.088	-4.77605678697356e-09\\
2.08999999999997	-4.69695635957358e-09\\
2.09199999999993	-4.61919187288798e-09\\
2.09399999999997	-4.54273283904935e-09\\
2.094	-4.54273283904828e-09\\
2.09799999999993	-4.3936117275811e-09\\
2.09999999999997	-4.32089118643073e-09\\
2.1	-4.3208911864297e-09\\
2.10399999999993	-4.17898757203687e-09\\
2.10599999999997	-4.10974886492132e-09\\
2.106	-4.10974886492035e-09\\
2.10999999999993	-3.9747173915453e-09\\
2.112	-3.90891055089514e-09\\
2.11200000000003	-3.90891055089421e-09\\
2.11599999999996	-3.78058565402447e-09\\
2.11699999999997	-3.74917121257253e-09\\
2.117	-3.74917121257164e-09\\
2.11999999999997	-3.65647879810409e-09\\
2.12	-3.65647879810322e-09\\
2.12299999999998	-3.56604941518713e-09\\
2.12599999999995	-3.47780329135854e-09\\
2.126	-3.47780329135713e-09\\
2.12899999999997	-3.39173658233563e-09\\
2.129	-3.39173658233483e-09\\
2.13199999999996	-3.30784737000392e-09\\
2.13499999999993	-3.22606165017266e-09\\
2.13499999999997	-3.22606165017175e-09\\
2.135	-3.22606165017085e-09\\
2.13999999999997	-3.09423140190775e-09\\
2.14	-3.09423140190701e-09\\
2.14499999999998	-2.9677214349161e-09\\
2.14599999999997	-2.94303037042066e-09\\
2.146	-2.94303037041996e-09\\
2.14699999999997	-2.91854421395811e-09\\
2.14699999999999	-2.91854421395741e-09\\
2.14799999999998	-2.89426752361201e-09\\
2.14899999999997	-2.87019791980234e-09\\
2.15099999999995	-2.82267055556964e-09\\
2.15299999999999	-2.77594348797241e-09\\
2.15300000000002	-2.77594348797175e-09\\
2.15699999999997	-2.68481727228376e-09\\
2.15999999999997	-2.61843934966851e-09\\
2.16	-2.61843934966789e-09\\
2.16399999999995	-2.53244643891227e-09\\
2.16599999999997	-2.49048811397047e-09\\
2.166	-2.49048811396988e-09\\
2.16999999999995	-2.40865968773891e-09\\
2.17	-2.40865968773797e-09\\
2.17399999999995	-2.32957188049521e-09\\
2.17499999999997	-2.31021352298225e-09\\
2.175	-2.3102135229817e-09\\
2.17899999999995	-2.23437754607426e-09\\
2.17999999999997	-2.21580862418798e-09\\
2.18	-2.21580862418745e-09\\
2.18399999999995	-2.14303862332668e-09\\
2.18599999999997	-2.10753212267536e-09\\
2.186	-2.10753212267486e-09\\
2.187	-2.08999735245014e-09\\
2.18700000000002	-2.08999735244965e-09\\
2.18800000000002	-2.07261258282432e-09\\
2.18800000000005	-2.07261258282383e-09\\
2.18900000000004	-2.05537610984398e-09\\
2.19000000000004	-2.03828624415599e-09\\
2.19200000000003	-2.00453964892216e-09\\
2.19600000000001	-1.93873298928637e-09\\
2.2	-1.87508939265602e-09\\
2.20000000000003	-1.87508939265558e-09\\
2.20399999999997	-1.81350904606906e-09\\
2.204	-1.81350904606863e-09\\
2.20499999999997	-1.79842495596648e-09\\
2.205	-1.79842495596605e-09\\
2.20599999999999	-1.78346228932641e-09\\
2.20600000000003	-1.78346228932579e-09\\
2.20700000000002	-1.76862379597292e-09\\
2.20800000000001	-1.7539122379575e-09\\
2.20999999999998	-1.7248641726959e-09\\
2.21200000000003	-1.69630670513471e-09\\
2.21200000000006	-1.69630670513431e-09\\
2.21600000000001	-1.64061896749459e-09\\
2.21800000000003	-1.6134668648184e-09\\
2.21800000000006	-1.61346686481801e-09\\
2.21999999999997	-1.58676168647414e-09\\
2.22	-1.58676168647377e-09\\
2.22199999999992	-1.56049296273329e-09\\
2.22399999999983	-1.53465039483897e-09\\
2.226	-1.50922385110997e-09\\
2.22600000000003	-1.50922385110961e-09\\
2.22999999999986	-1.45963621722653e-09\\
2.23299999999997	-1.42353954717499e-09\\
2.233	-1.42353954717465e-09\\
2.23699999999983	-1.37680885088518e-09\\
2.23899999999997	-1.35402207535656e-09\\
2.239	-1.35402207535624e-09\\
2.24	-1.34276939779072e-09\\
2.24000000000003	-1.3427693977904e-09\\
2.24100000000003	-1.3316090535144e-09\\
2.24200000000003	-1.32053994865601e-09\\
2.24400000000004	-1.29867112647859e-09\\
2.246	-1.27715435750312e-09\\
2.24600000000003	-1.27715435750281e-09\\
2.25000000000004	-1.23519168755036e-09\\
2.252	-1.21474141278389e-09\\
2.25200000000003	-1.2147414127836e-09\\
2.25600000000004	-1.17486289297273e-09\\
2.25999999999997	-1.13629518070236e-09\\
2.26	-1.13629518070209e-09\\
2.26199999999997	-1.11748389715364e-09\\
2.262	-1.11748389715338e-09\\
2.26399999999996	-1.0989777879204e-09\\
2.26599999999993	-1.08076959760946e-09\\
2.26599999999997	-1.08076959760914e-09\\
2.266	-1.08076959760882e-09\\
2.26999999999994	-1.04525942023253e-09\\
2.272	-1.02795373197208e-09\\
2.27200000000003	-1.02795373197183e-09\\
2.27499999999997	-1.00253782658167e-09\\
2.275	-1.00253782658143e-09\\
2.27799999999994	-9.77753186379527e-10\\
2.27999999999997	-9.61569976505633e-10\\
2.28	-9.61569976505405e-10\\
2.28299999999994	-9.37789124946285e-10\\
2.28599999999989	-9.14582420408248e-10\\
2.28599999999997	-9.1458242040758e-10\\
2.286	-9.14582420407362e-10\\
2.28699999999997	-9.06973049958816e-10\\
2.287	-9.069730499586e-10\\
2.28799999999999	-8.99428773710269e-10\\
2.28899999999997	-8.91948852116791e-10\\
2.29099999999995	-8.77179146689534e-10\\
2.291	-8.77179146689196e-10\\
2.29499999999995	-8.48380148659471e-10\\
2.29699999999997	-8.34339565336599e-10\\
2.297	-8.34339565336401e-10\\
2.29999999999997	-8.13711820037174e-10\\
2.3	-8.13711820036981e-10\\
2.30299999999998	-7.93587689175409e-10\\
2.30599999999995	-7.73949420208547e-10\\
2.306	-7.73949420208246e-10\\
2.30999999999997	-7.48520243150941e-10\\
2.31	-7.48520243150763e-10\\
2.31399999999997	-7.23942746888686e-10\\
2.31599999999997	-7.11961282718487e-10\\
2.316	-7.11961282718318e-10\\
2.31999999999997	-6.88589433850948e-10\\
2.32	-6.88589433850785e-10\\
2.32399999999997	-6.65975272601678e-10\\
2.32599999999997	-6.54941196445864e-10\\
2.326	-6.54941196445709e-10\\
2.32999999999997	-6.33422199008811e-10\\
2.33199999999997	-6.22935034795106e-10\\
2.332	-6.22935034794958e-10\\
2.33599999999997	-6.02484816352391e-10\\
2.33999999999994	-5.82706796871669e-10\\
2.34	-5.82706796871356e-10\\
2.34000000000003	-5.82706796871218e-10\\
2.34499999999997	-5.58882392537186e-10\\
2.345	-5.58882392537054e-10\\
2.34600000000003	-5.54232562163309e-10\\
2.34600000000006	-5.54232562163177e-10\\
2.34700000000009	-5.49621320185701e-10\\
2.34800000000012	-5.45049524904668e-10\\
2.34899999999997	-5.40516728236649e-10\\
2.349	-5.40516728236521e-10\\
2.35100000000006	-5.31566357473108e-10\\
2.35300000000011	-5.22766698891713e-10\\
2.35499999999997	-5.14114302553359e-10\\
2.355	-5.14114302553237e-10\\
2.35800000000003	-5.01404419877563e-10\\
2.35800000000006	-5.01404419877444e-10\\
2.35999999999997	-4.93105461450458e-10\\
2.36	-4.93105461450341e-10\\
2.36199999999992	-4.849421365681e-10\\
2.36399999999983	-4.76911244785728e-10\\
2.36599999999997	-4.69009637560487e-10\\
2.366	-4.69009637560376e-10\\
2.36999999999983	-4.5359967816564e-10\\
2.37399999999966	-4.3870583316e-10\\
2.37799999999997	-4.24304743013925e-10\\
2.378	-4.24304743013824e-10\\
2.37999999999997	-4.17281893796132e-10\\
2.38	-4.17281893796033e-10\\
2.38199999999997	-4.10373822535638e-10\\
2.38399999999995	-4.03577820478402e-10\\
2.38599999999997	-3.9689122322555e-10\\
2.386	-3.96891223225456e-10\\
2.38999999999995	-3.83850813957594e-10\\
2.39	-3.83850813957444e-10\\
2.39399999999995	-3.71247157499278e-10\\
2.396	-3.65102908450872e-10\\
2.39600000000002	-3.65102908450786e-10\\
2.39999999999997	-3.53117523537237e-10\\
2.4	-3.53117523537153e-10\\
2.40399999999995	-3.41520690614927e-10\\
2.40599999999997	-3.35862274369834e-10\\
2.406	-3.35862274369754e-10\\
2.40699999999997	-3.33067883855288e-10\\
2.407	-3.33067883855209e-10\\
2.40799999999998	-3.30297397860602e-10\\
2.40899999999997	-3.27550544839006e-10\\
2.41099999999995	-3.22126663124085e-10\\
2.41299999999997	-3.16794112256723e-10\\
2.413	-3.16794112256648e-10\\
2.41499999999997	-3.11550801618787e-10\\
2.415	-3.11550801618713e-10\\
2.41699999999997	-3.06394675577696e-10\\
2.41899999999995	-3.01323712653857e-10\\
2.41999999999997	-2.98819544794563e-10\\
2.42	-2.98819544794493e-10\\
2.42399999999995	-2.8900592727624e-10\\
2.426	-2.84217591254418e-10\\
2.42600000000003	-2.8421759125435e-10\\
2.427	-2.81852887036128e-10\\
2.42700000000003	-2.81852887036061e-10\\
2.42800000000002	-2.79508411600243e-10\\
2.429	-2.77183935155039e-10\\
2.43099999999998	-2.72594069882784e-10\\
2.43499999999993	-2.63644431616398e-10\\
2.43599999999997	-2.61453675942175e-10\\
2.436	-2.61453675942113e-10\\
2.43999999999997	-2.52870827160071e-10\\
2.44	-2.52870827160011e-10\\
2.44399999999998	-2.44566224542467e-10\\
2.446	-2.40514178717604e-10\\
2.44600000000003	-2.40514178717547e-10\\
2.448	-2.36529117583715e-10\\
2.44800000000002	-2.36529117583659e-10\\
2.44999999999999	-2.32611753307324e-10\\
2.45000000000002	-2.32611753307269e-10\\
2.45199999999998	-2.28760550070214e-10\\
2.45399999999995	-2.24973998003557e-10\\
2.45600000000002	-2.21250612574871e-10\\
2.45600000000005	-2.21250612574819e-10\\
2.45999999999998	-2.13987526843694e-10\\
2.46000000000001	-2.13987526843643e-10\\
2.46399999999994	-2.06959901743582e-10\\
2.46499999999997	-2.05238486699303e-10\\
2.465	-2.05238486699255e-10\\
2.46599999999998	-2.03530928644891e-10\\
2.46600000000001	-2.03530928644842e-10\\
2.46699999999999	-2.01837541406044e-10\\
2.46799999999998	-2.00158640184797e-10\\
2.46999999999996	-1.96843638999486e-10\\
2.47199999999998	-1.93584625429924e-10\\
2.47200000000001	-1.93584625429878e-10\\
2.47599999999996	-1.87229471830543e-10\\
2.47999999999991	-1.81083212218112e-10\\
2.47999999999997	-1.81083212218014e-10\\
2.48	-1.81083212217971e-10\\
2.48499999999997	-1.73679489399505e-10\\
2.485	-1.73679489399464e-10\\
2.48599999999997	-1.72234498173133e-10\\
2.486	-1.72234498173092e-10\\
2.48699999999999	-1.70801498759175e-10\\
2.48799999999998	-1.69380757884777e-10\\
2.48999999999995	-1.66575495952824e-10\\
2.49199999999997	-1.63817612563486e-10\\
2.492	-1.63817612563447e-10\\
2.494	-1.61106026492917e-10\\
2.49400000000002	-1.61106026492879e-10\\
2.49600000000002	-1.58439674667629e-10\\
2.49800000000001	-1.55817511733848e-10\\
2.49999999999997	-1.53238509662561e-10\\
2.5	-1.53238509662525e-10\\
2.50399999999999	-1.48205960263291e-10\\
2.50599999999997	-1.45750439904683e-10\\
2.506	-1.45750439904648e-10\\
2.50999999999999	-1.40961607966064e-10\\
2.51399999999998	-1.36333164834736e-10\\
2.51999999999997	-1.29675415851786e-10\\
2.52	-1.29675415851756e-10\\
2.52299999999997	-1.2646837729115e-10\\
2.523	-1.2646837729112e-10\\
2.52599999999996	-1.23338767245638e-10\\
2.526	-1.233387672456e-10\\
2.52899999999997	-1.20286449231785e-10\\
2.53199999999993	-1.17311355139667e-10\\
2.53199999999997	-1.17311355139635e-10\\
2.532	-1.17311355139603e-10\\
2.53799999999993	-1.11582406597712e-10\\
2.53799999999997	-1.11582406597681e-10\\
2.538	-1.11582406597651e-10\\
2.53999999999997	-1.09735558574493e-10\\
2.54	-1.09735558574467e-10\\
2.54199999999998	-1.0791889443866e-10\\
2.54399999999995	-1.06131701929468e-10\\
2.54599999999997	-1.04373280371184e-10\\
2.546	-1.04373280371159e-10\\
2.54999999999995	-1.00943952128948e-10\\
2.55199999999997	-9.92726880049976e-11\\
2.552	-9.9272688004974e-11\\
2.55499999999997	-9.68181949962042e-11\\
2.555	-9.68181949961812e-11\\
2.55799999999998	-9.44246652301689e-11\\
2.55800000000001	-9.44246652301465e-11\\
2.55999999999997	-9.28618023354243e-11\\
2.56	-9.28618023354023e-11\\
2.56199999999997	-9.13244820000885e-11\\
2.56399999999994	-8.98121015152303e-11\\
2.56599999999997	-8.83240679462208e-11\\
2.566	-8.83240679461998e-11\\
2.56999999999994	-8.54220587440507e-11\\
2.56999999999997	-8.5422058744027e-11\\
2.57	-8.54220587440032e-11\\
2.57399999999994	-8.26172443914214e-11\\
2.57799999999988	-7.99052259338734e-11\\
2.57999999999997	-7.85826803907801e-11\\
2.58	-7.85826803907615e-11\\
2.58099999999997	-7.79295453081817e-11\\
2.581	-7.79295453081632e-11\\
2.58199999999998	-7.72817498450949e-11\\
2.58299999999997	-7.66392304810246e-11\\
2.58499999999995	-7.53697686664986e-11\\
2.58599999999997	-7.47427017932331e-11\\
2.586	-7.47427017932153e-11\\
2.58999999999995	-7.22869271288568e-11\\
2.59	-7.22869271288283e-11\\
2.59199999999997	-7.10901189482887e-11\\
2.592	-7.10901189482719e-11\\
2.59399999999997	-6.99134019071945e-11\\
2.59599999999995	-6.87563146689936e-11\\
2.59799999999997	-6.76184035931566e-11\\
2.598	-6.76184035931406e-11\\
2.59999999999997	-6.64992225634527e-11\\
2.6	-6.64992225634369e-11\\
2.60199999999997	-6.53983328065227e-11\\
2.60399999999995	-6.43153027141899e-11\\
2.60599999999997	-6.32497076803159e-11\\
2.606	-6.32497076803009e-11\\
2.60999999999995	-6.11715512064784e-11\\
2.61	-6.11715512064525e-11\\
2.61399999999994	-5.91629968912518e-11\\
2.61599999999997	-5.81838319959368e-11\\
2.616	-5.8183831995923e-11\\
2.61999999999994	-5.62738071762015e-11\\
2.61999999999997	-5.62738071761869e-11\\
2.62	-5.62738071761723e-11\\
2.62399999999995	-5.44257030964152e-11\\
2.62499999999997	-5.3973010455304e-11\\
2.625	-5.39730104552912e-11\\
2.62599999999997	-5.35239618843806e-11\\
2.626	-5.35239618843679e-11\\
2.62699999999999	-5.30786399126141e-11\\
2.62799999999998	-5.26371274309284e-11\\
2.62999999999995	-5.17653582156297e-11\\
2.63199999999997	-5.09083124579923e-11\\
2.632	-5.09083124579802e-11\\
2.63599999999995	-4.92370529446546e-11\\
2.63899999999997	-4.80197990243561e-11\\
2.639	-4.80197990243447e-11\\
2.63999999999997	-4.76207277519393e-11\\
2.64	-4.7620727751928e-11\\
2.64099999999999	-4.72249310389222e-11\\
2.64199999999998	-4.68323700916334e-11\\
2.64399999999995	-4.60568019073524e-11\\
2.64599999999997	-4.52937191345926e-11\\
2.646	-4.52937191345819e-11\\
2.64999999999995	-4.38055314512521e-11\\
2.65099999999997	-4.34413618888486e-11\\
2.651	-4.34413618888383e-11\\
2.65499999999995	-4.20151222173587e-11\\
2.6589999999999	-4.06359173117102e-11\\
2.65999999999997	-4.02982101933277e-11\\
2.66	-4.02982101933181e-11\\
2.66599999999997	-3.83290197128048e-11\\
2.666	-3.83290197127957e-11\\
2.66799999999997	-3.76939490847073e-11\\
2.668	-3.76939490846984e-11\\
2.66999999999996	-3.70696668296457e-11\\
2.67199999999993	-3.64559281848261e-11\\
2.67199999999996	-3.64559281848151e-11\\
2.672	-3.64559281848042e-11\\
2.67599999999993	-3.52591232972855e-11\\
2.67799999999997	-3.46755878441791e-11\\
2.678	-3.46755878441709e-11\\
2.67999999999997	-3.41016573998113e-11\\
2.68	-3.41016573998032e-11\\
2.68199999999998	-3.35371069555499e-11\\
2.68399999999995	-3.29817151771637e-11\\
2.68599999999997	-3.2435264321158e-11\\
2.686	-3.24352643211503e-11\\
2.68999999999995	-3.13695589191206e-11\\
2.69399999999989	-3.03395464007265e-11\\
2.69499999999997	-3.00874297711861e-11\\
2.695	-3.0087429771179e-11\\
2.69699999999997	-2.95894866108812e-11\\
2.697	-2.95894866108742e-11\\
2.69899999999996	-2.90997679484516e-11\\
2.69999999999997	-2.88579326700239e-11\\
2.7	-2.88579326700171e-11\\
2.70199999999997	-2.83801917047306e-11\\
2.70399999999993	-2.79102011008219e-11\\
2.70599999999997	-2.74477765966589e-11\\
2.706	-2.74477765966524e-11\\
2.70899999999997	-2.67685145486583e-11\\
2.709	-2.67685145486519e-11\\
2.71199999999996	-2.61064378928702e-11\\
2.71299999999997	-2.58894632759895e-11\\
2.713	-2.58894632759833e-11\\
2.71599999999997	-2.52493944966632e-11\\
2.71899999999993	-2.46251710153901e-11\\
2.71999999999997	-2.44205221265747e-11\\
2.72	-2.44205221265689e-11\\
2.72599999999993	-2.32272021525107e-11\\
2.726	-2.32272021524986e-11\\
2.72600000000002	-2.32272021524931e-11\\
2.72999999999997	-2.24640403465451e-11\\
2.73	-2.24640403465398e-11\\
2.73199999999999	-2.20921176702104e-11\\
2.73200000000002	-2.20921176702051e-11\\
2.73400000000002	-2.17264385595445e-11\\
2.73600000000001	-2.13668596486061e-11\\
2.73799999999999	-2.10132399631027e-11\\
2.73800000000002	-2.10132399630977e-11\\
2.73999999999997	-2.06654408669829e-11\\
2.74	-2.0665440866978e-11\\
2.74199999999995	-2.03233260061008e-11\\
2.7439999999999	-1.99867612529741e-11\\
2.74599999999997	-1.96556146560785e-11\\
2.746	-1.96556146560738e-11\\
2.7499999999999	-1.90098022958202e-11\\
2.7539999999998	-1.83856196517471e-11\\
2.75499999999997	-1.82328381838975e-11\\
2.755	-1.82328381838931e-11\\
2.75999999999997	-1.74877688372908e-11\\
2.76	-1.74877688372867e-11\\
2.76499999999998	-1.67727683190952e-11\\
2.76500000000001	-1.67727683190912e-11\\
2.76599999999997	-1.66332210223947e-11\\
2.766	-1.66332210223908e-11\\
2.76699999999999	-1.64948318123411e-11\\
2.76700000000002	-1.64948318123371e-11\\
2.76800000000001	-1.63576264476108e-11\\
2.76899999999999	-1.6221591480655e-11\\
2.77099999999997	-1.59529795220935e-11\\
2.77299999999999	-1.56888906262995e-11\\
2.77300000000002	-1.56888906262958e-11\\
2.77699999999997	-1.51738696137965e-11\\
2.77999999999997	-1.47987193386536e-11\\
2.78	-1.47987193386501e-11\\
2.78399999999995	-1.43127103941947e-11\\
2.784	-1.43127103941897e-11\\
2.786	-1.40755731578462e-11\\
2.78600000000003	-1.40755731578429e-11\\
2.78800000000004	-1.3842356056594e-11\\
2.79000000000004	-1.36131007671595e-11\\
2.792	-1.33877174091302e-11\\
2.79200000000003	-1.3387717409127e-11\\
2.79600000000004	-1.29482145263775e-11\\
2.79999999999997	-1.25231581096245e-11\\
2.8	-1.25231581096216e-11\\
2.80400000000001	-1.21118815179885e-11\\
2.80599999999997	-1.19112082681299e-11\\
2.806	-1.19112082681271e-11\\
2.81000000000001	-1.15198490743189e-11\\
2.81199999999997	-1.13291223388397e-11\\
2.812	-1.1329122338837e-11\\
2.81299999999997	-1.12349642626375e-11\\
2.813	-1.12349642626349e-11\\
2.81399999999998	-1.1141597383761e-11\\
2.81499999999997	-1.10490125509372e-11\\
2.81699999999995	-1.08661528020932e-11\\
2.81899999999997	-1.0686313325234e-11\\
2.819	-1.06863133252314e-11\\
2.81999999999997	-1.05975041101858e-11\\
2.82	-1.05975041101833e-11\\
2.82099999999999	-1.05094236147193e-11\\
2.82199999999998	-1.04220632057012e-11\\
2.82399999999995	-1.02494684678246e-11\\
2.82599999999997	-1.00796522301221e-11\\
2.826	-1.00796522301197e-11\\
2.82999999999995	-9.74847133810219e-12\\
2.83399999999991	-9.4283824427818e-12\\
2.83499999999997	-9.35003413920577e-12\\
2.835	-9.35003413920355e-12\\
2.83999999999997	-8.96795298622807e-12\\
2.84	-8.96795298622594e-12\\
2.84199999999997	-8.81948917961099e-12\\
2.842	-8.8194891796089e-12\\
2.84399999999996	-8.67343389307826e-12\\
2.84599999999992	-8.52972986423031e-12\\
2.846	-8.52972986422483e-12\\
2.84600000000003	-8.5297298642228e-12\\
2.847	-8.45876209530731e-12\\
2.84700000000003	-8.4587620953053e-12\\
2.84800000000002	-8.38840141777417e-12\\
2.84900000000001	-8.31864093528458e-12\\
2.85099999999998	-8.18089326457407e-12\\
2.853	-8.04546507866881e-12\\
2.85300000000003	-8.0454650786669e-12\\
2.85699999999998	-7.78135567251753e-12\\
2.85999999999997	-7.58897378128907e-12\\
2.86	-7.58897378128727e-12\\
2.86399999999995	-7.33974213704654e-12\\
2.86599999999997	-7.21813510900272e-12\\
2.866	-7.218135109001e-12\\
2.86999999999995	-6.98097331253226e-12\\
2.87	-6.98097331252944e-12\\
2.87099999999997	-6.9229382215505e-12\\
2.871	-6.92293822154886e-12\\
2.87199999999998	-6.86539382472867e-12\\
2.87299999999997	-6.80833448190192e-12\\
2.87499999999995	-6.69564863558482e-12\\
2.87699999999997	-6.5848365036775e-12\\
2.87699999999999	-6.58483650367594e-12\\
2.87999999999997	-6.42203668407538e-12\\
2.88	-6.42203668407385e-12\\
2.88299999999998	-6.26321152773433e-12\\
2.88599999999996	-6.10822092672135e-12\\
2.886	-6.1082209267192e-12\\
2.888	-6.00701427630391e-12\\
2.88800000000003	-6.00701427630248e-12\\
2.89000000000003	-5.90752689152468e-12\\
2.89200000000003	-5.80971976629227e-12\\
2.89600000000002	-5.61899355657615e-12\\
2.89999999999997	-5.43453651189328e-12\\
2.9	-5.43453651189199e-12\\
2.90499999999997	-5.21234141214886e-12\\
2.905	-5.21234141214762e-12\\
2.90599999999997	-5.16897539515079e-12\\
2.90599999999999	-5.16897539514956e-12\\
2.90699999999998	-5.12596926764053e-12\\
2.90799999999997	-5.08333103444278e-12\\
2.90999999999995	-4.99914157066787e-12\\
2.91199999999997	-4.9163739970212e-12\\
2.91199999999999	-4.91637399702003e-12\\
2.91599999999995	-4.75497526945917e-12\\
2.91799999999997	-4.67628083853704e-12\\
2.91799999999999	-4.67628083853594e-12\\
2.91999999999997	-4.5988817200333e-12\\
2.92	-4.59888172003221e-12\\
2.92199999999998	-4.52274756973222e-12\\
2.92399999999996	-4.44784853893951e-12\\
2.926	-4.37415526320013e-12\\
2.92600000000003	-4.37415526319909e-12\\
2.92899999999997	-4.26590614373211e-12\\
2.92899999999999	-4.2659061437311e-12\\
2.93199999999993	-4.16039573693212e-12\\
2.93499999999987	-4.05753096658108e-12\\
2.93499999999997	-4.05753096657774e-12\\
2.935	-4.05753096657678e-12\\
2.93999999999997	-3.89172343657807e-12\\
2.94	-3.89172343657715e-12\\
2.94499999999998	-3.73260741135184e-12\\
2.94599999999997	-3.70155259306461e-12\\
2.946	-3.70155259306373e-12\\
2.95099999999998	-3.55017183054663e-12\\
2.95199999999997	-3.52066232309052e-12\\
2.952	-3.52066232308968e-12\\
2.95699999999998	-3.37678891866988e-12\\
2.95799999999999	-3.34872932271585e-12\\
2.95800000000002	-3.34872932271505e-12\\
2.95999999999997	-3.29330307555662e-12\\
2.96	-3.29330307555584e-12\\
2.96199999999995	-3.2387826844961e-12\\
2.9639999999999	-3.18514677428297e-12\\
2.96599999999997	-3.13237431674945e-12\\
2.966	-3.13237431674871e-12\\
2.9699999999999	-3.02945583386099e-12\\
2.96999999999999	-3.02945583385871e-12\\
2.97000000000002	-3.02945583385799e-12\\
2.97399999999992	-2.92998432362861e-12\\
2.97499999999997	-2.90563663812967e-12\\
2.975	-2.90563663812898e-12\\
2.9789999999999	-2.81025506900498e-12\\
2.97999999999997	-2.78690028439128e-12\\
2.98	-2.78690028439062e-12\\
2.9839999999999	-2.69537490091183e-12\\
2.986	-2.65071712916476e-12\\
2.98600000000003	-2.65071712916413e-12\\
2.98699999999997	-2.62866303338006e-12\\
2.98699999999999	-2.62866303337944e-12\\
2.98799999999998	-2.60679759860873e-12\\
2.98899999999997	-2.58511868172988e-12\\
2.99099999999995	-2.54231192068294e-12\\
2.99299999999999	-2.50022596768247e-12\\
2.99300000000002	-2.50022596768187e-12\\
2.99699999999997	-2.41815076312507e-12\\
2.99999999999997	-2.35836575463967e-12\\
3	-2.35836575463911e-12\\
3.00399999999995	-2.28091399510675e-12\\
3.00599999999997	-2.24312313460389e-12\\
3.006	-2.24312313460336e-12\\
3.00999999999995	-2.16942222671251e-12\\
3.01	-2.16942222671163e-12\\
3.01399999999995	-2.0981897288136e-12\\
3.01599999999997	-2.06346407521942e-12\\
3.01599999999999	-2.06346407521893e-12\\
3.01999999999995	-1.99572588261004e-12\\
3.02	-1.99572588260911e-12\\
3.02399999999995	-1.93018368184909e-12\\
3.02599999999997	-1.89820382521276e-12\\
3.026	-1.89820382521231e-12\\
3.02799999999997	-1.86675263011009e-12\\
3.02799999999999	-1.86675263010965e-12\\
3.02999999999996	-1.83583571715665e-12\\
3.03199999999992	-1.8054409652544e-12\\
3.03599999999985	-1.74617047944877e-12\\
3.03999999999997	-1.68884821277428e-12\\
3.04	-1.68884821277388e-12\\
3.04499999999997	-1.61979838715869e-12\\
3.04499999999999	-1.6197983871583e-12\\
3.04599999999997	-1.60632187072683e-12\\
3.046	-1.60632187072644e-12\\
3.04699999999998	-1.59295719432291e-12\\
3.04799999999996	-1.57970684554389e-12\\
3.04999999999992	-1.5535439472495e-12\\
3.05199999999997	-1.52782291857753e-12\\
3.052	-1.52782291857716e-12\\
3.05599999999992	-1.47766630389402e-12\\
3.05799999999997	-1.45321105378284e-12\\
3.058	-1.45321105378249e-12\\
3.05999999999997	-1.42915833773473e-12\\
3.06	-1.42915833773439e-12\\
3.06199999999997	-1.40549872591561e-12\\
3.06399999999995	-1.38222294248124e-12\\
3.066	-1.35932186207036e-12\\
3.06600000000003	-1.35932186207004e-12\\
3.06999999999997	-1.31465946503438e-12\\
3.07399999999992	-1.2714929126966e-12\\
3.07399999999996	-1.2714929126962e-12\\
3.07399999999999	-1.27149291269579e-12\\
3.07999999999997	-1.20940031238581e-12\\
3.07999999999999	-1.20940031238553e-12\\
3.08599999999997	-1.15030241338282e-12\\
3.08599999999999	-1.15030241338254e-12\\
3.09199999999997	-1.09408856533709e-12\\
3.09199999999999	-1.09408856533683e-12\\
3.09799999999997	-1.04065829763865e-12\\
3.09999999999997	-1.02343391807908e-12\\
3.1	-1.02343391807884e-12\\
3.10299999999997	-9.98123091956453e-13\\
3.10299999999999	-9.98123091956216e-13\\
3.10599999999996	-9.73423351779104e-13\\
3.106	-9.73423351778743e-13\\
3.10600000000003	-9.73423351778511e-13\\
3.10899999999999	-9.49333621515988e-13\\
3.11199999999996	-9.25853363615271e-13\\
3.11199999999999	-9.25853363614986e-13\\
3.11200000000003	-9.25853363614704e-13\\
3.11499999999997	-9.0296186462205e-13\\
3.115	-9.02961864621836e-13\\
3.11799999999995	-8.80638931411589e-13\\
3.11999999999997	-8.66063100972871e-13\\
3.12	-8.66063100972666e-13\\
3.12299999999995	-8.44644256216736e-13\\
3.12599999999989	-8.23742531969026e-13\\
3.12599999999995	-8.23742531968644e-13\\
3.126	-8.23742531968259e-13\\
3.127	-8.16888953834061e-13\\
3.12700000000003	-8.16888953833867e-13\\
3.12800000000003	-8.10094004491197e-13\\
3.12900000000003	-8.03357017849089e-13\\
3.13100000000003	-7.90054297103768e-13\\
3.13199999999997	-7.83487259171601e-13\\
3.13199999999999	-7.83487259171415e-13\\
3.13599999999999	-7.57766301793502e-13\\
3.13799999999997	-7.45225334866115e-13\\
3.13799999999999	-7.45225334865939e-13\\
3.13999999999997	-7.32890792582541e-13\\
3.14	-7.32890792582367e-13\\
3.14199999999998	-7.20757839203102e-13\\
3.14399999999996	-7.08821717955511e-13\\
3.14599999999997	-6.97077749237384e-13\\
3.146	-6.97077749237219e-13\\
3.14999999999996	-6.74174297319376e-13\\
3.15	-6.74174297319146e-13\\
3.15399999999996	-6.52037933782806e-13\\
3.15599999999997	-6.41246515336628e-13\\
3.156	-6.41246515336476e-13\\
3.15999999999996	-6.20196049604061e-13\\
3.16	-6.20196049603852e-13\\
3.16099999999997	-6.15041328744142e-13\\
3.16099999999999	-6.15041328743996e-13\\
3.16199999999996	-6.0992874940831e-13\\
3.16299999999992	-6.04857810491435e-13\\
3.16499999999985	-5.94838869927586e-13\\
3.16599999999997	-5.89889886301117e-13\\
3.166	-5.89889886300977e-13\\
3.16999999999986	-5.70508239582549e-13\\
3.17199999999997	-5.61062701401517e-13\\
3.172	-5.61062701401383e-13\\
3.17599999999986	-5.42643678376649e-13\\
3.17999999999972	-5.24830088889422e-13\\
3.17999999999997	-5.24830088888312e-13\\
3.18	-5.24830088888187e-13\\
3.18499999999997	-5.03372017259894e-13\\
3.185	-5.03372017259775e-13\\
3.18599999999997	-4.99184026163413e-13\\
3.186	-4.99184026163295e-13\\
3.18699999999997	-4.95030790713011e-13\\
3.18799999999994	-4.90913083959207e-13\\
3.18999999999989	-4.82782645671317e-13\\
3.18999999999994	-4.827826456711e-13\\
3.18999999999999	-4.82782645670883e-13\\
3.19399999999988	-4.66930584486508e-13\\
3.19599999999999	-4.59202746748628e-13\\
3.19600000000002	-4.59202746748519e-13\\
3.198	-4.51602980852058e-13\\
3.19800000000003	-4.51602980851951e-13\\
3.19999999999998	-4.44128307318988e-13\\
3.20000000000001	-4.44128307318882e-13\\
3.20199999999996	-4.36775795673916e-13\\
3.20399999999991	-4.29542563355874e-13\\
3.20599999999998	-4.22425774547926e-13\\
3.20600000000001	-4.22425774547826e-13\\
3.20999999999991	-4.08546392358818e-13\\
3.21399999999982	-3.9513186206234e-13\\
3.21899999999997	-3.78985412170138e-13\\
3.21899999999999	-3.78985412170048e-13\\
3.21999999999997	-3.75835832180895e-13\\
3.22	-3.75835832180806e-13\\
3.22099999999998	-3.72712096044634e-13\\
3.22199999999996	-3.69613897458912e-13\\
3.22399999999992	-3.63492900789264e-13\\
3.22599999999997	-3.57470442411411e-13\\
3.226	-3.57470442411326e-13\\
3.22999999999992	-3.45725257362968e-13\\
3.23099999999999	-3.42851131374777e-13\\
3.23100000000002	-3.42851131374696e-13\\
3.23499999999994	-3.31594857033304e-13\\
3.23699999999999	-3.26107004384073e-13\\
3.23700000000002	-3.26107004383996e-13\\
3.23999999999997	-3.18044516951811e-13\\
3.24	-3.18044516951736e-13\\
3.24299999999996	-3.10178870483153e-13\\
3.24599999999991	-3.02503126282153e-13\\
3.24599999999997	-3.02503126281985e-13\\
3.246	-3.02503126281913e-13\\
3.24799999999997	-2.97490974870232e-13\\
3.24799999999999	-2.97490974870162e-13\\
3.24999999999996	-2.92563968256655e-13\\
3.25199999999992	-2.87720174706215e-13\\
3.25399999999997	-2.82957695190835e-13\\
3.25399999999999	-2.82957695190767e-13\\
3.25499999999998	-2.8060636349376e-13\\
3.255	-2.80606363493693e-13\\
3.25599999999998	-2.78274662591899e-13\\
3.25699999999997	-2.75962363945579e-13\\
3.25899999999993	-2.71395068781511e-13\\
3.25999999999997	-2.69139624612817e-13\\
3.26	-2.69139624612753e-13\\
3.26399999999993	-2.60300733749432e-13\\
3.26599999999997	-2.55987993880336e-13\\
3.266	-2.55987993880275e-13\\
3.267	-2.53858161279608e-13\\
3.26700000000003	-2.53858161279548e-13\\
3.26800000000003	-2.5174654826002e-13\\
3.26900000000003	-2.49652947853616e-13\\
3.27100000000003	-2.4551896584704e-13\\
3.27500000000003	-2.37458240452527e-13\\
3.27699999999997	-2.33528336607141e-13\\
3.27699999999999	-2.33528336607086e-13\\
3.27999999999997	-2.27754712143853e-13\\
3.28	-2.27754712143798e-13\\
3.28299999999998	-2.2212204770616e-13\\
3.28599999999996	-2.16625374089402e-13\\
3.286	-2.16625374089331e-13\\
3.28899999999999	-2.1126445184219e-13\\
3.28900000000002	-2.11264451842139e-13\\
3.29	-2.09507848239217e-13\\
3.29000000000003	-2.09507848239167e-13\\
3.29100000000001	-2.07766141690374e-13\\
3.29199999999999	-2.06039161488317e-13\\
3.29399999999996	-2.02628704503076e-13\\
3.296	-1.99275140200586e-13\\
3.29600000000003	-1.99275140200539e-13\\
3.29999999999996	-1.92733452403307e-13\\
3.3	-1.92733452403236e-13\\
3.30000000000003	-1.9273345240319e-13\\
3.30399999999996	-1.86403838375003e-13\\
3.30599999999997	-1.83315444232568e-13\\
3.30599999999999	-1.83315444232524e-13\\
3.30999999999992	-1.77292362223209e-13\\
3.31199999999999	-1.74357046567384e-13\\
3.31200000000002	-1.74357046567343e-13\\
3.31599999999995	-1.68633111564474e-13\\
3.31999999999988	-1.63097322388287e-13\\
3.32	-1.63097322388114e-13\\
3.32000000000003	-1.63097322388076e-13\\
3.32499999999998	-1.56428966153141e-13\\
3.325	-1.56428966153104e-13\\
3.326	-1.55127497083498e-13\\
3.32600000000003	-1.55127497083461e-13\\
3.32700000000003	-1.53836828752955e-13\\
3.32800000000003	-1.52557201396182e-13\\
3.33000000000002	-1.50030569006517e-13\\
3.332	-1.47546609338581e-13\\
3.33200000000003	-1.47546609338546e-13\\
3.33499999999999	-1.43898555416479e-13\\
3.33500000000002	-1.43898555416445e-13\\
3.33799999999998	-1.40341109693515e-13\\
3.33999999999997	-1.38018264161172e-13\\
3.34	-1.38018264161139e-13\\
3.34299999999997	-1.34604896525945e-13\\
3.34599999999993	-1.3127393867434e-13\\
3.34599999999997	-1.31273938674299e-13\\
3.346	-1.31273938674259e-13\\
3.34699999999999	-1.30181733091829e-13\\
3.34700000000002	-1.30181733091799e-13\\
3.34800000000001	-1.29098870759209e-13\\
3.349	-1.28025245526321e-13\\
3.35099999999999	-1.25905286348911e-13\\
3.35499999999995	-1.21771642623318e-13\\
3.35999999999997	-1.16795548571875e-13\\
3.36	-1.16795548571848e-13\\
3.36399999999997	-1.1295983263934e-13\\
3.36399999999999	-1.12959832639314e-13\\
3.36599999999997	-1.11088280592854e-13\\
3.366	-1.11088280592828e-13\\
3.36799999999998	-1.09247667326159e-13\\
3.36999999999996	-1.07438321763781e-13\\
3.37199999999997	-1.0565953454481e-13\\
3.372	-1.05659534544785e-13\\
3.37599999999996	-1.02190857365482e-13\\
3.37799999999997	-1.00499607496602e-13\\
3.378	-1.00499607496579e-13\\
3.37999999999997	-9.88361956447507e-14\\
3.38	-9.88361956447273e-14\\
3.38199999999997	-9.71999696716479e-14\\
3.38399999999995	-9.55902880884394e-14\\
3.38599999999997	-9.40065198132321e-14\\
3.386	-9.40065198132098e-14\\
3.38999999999995	-9.09178058185998e-14\\
3.39299999999997	-8.86694168291371e-14\\
3.39299999999999	-8.8669416829116e-14\\
3.39499999999998	-8.72018349647445e-14\\
3.395	-8.72018349647238e-14\\
3.39699999999999	-8.57586557228615e-14\\
3.39899999999997	-8.43393132993619e-14\\
3.399	-8.43393132993419e-14\\
3.39999999999997	-8.36384066406688e-14\\
3.4	-8.3638406640649e-14\\
3.40099999999998	-8.29432512367137e-14\\
3.40199999999996	-8.22537789524355e-14\\
3.40399999999991	-8.08916139914134e-14\\
3.40599999999997	-7.95513777155253e-14\\
3.406	-7.95513777155064e-14\\
3.40999999999991	-7.69376073763618e-14\\
3.41099999999997	-7.62980001472681e-14\\
3.411	-7.629800014725e-14\\
3.41499999999991	-7.3793031862531e-14\\
3.41899999999982	-7.13706728335635e-14\\
3.41999999999997	-7.07775427744152e-14\\
3.42	-7.07775427743984e-14\\
3.42199999999997	-6.96058256731416e-14\\
3.42199999999999	-6.96058256731251e-14\\
3.42399999999996	-6.84531173353533e-14\\
3.42599999999992	-6.73189657675871e-14\\
3.426	-6.73189657675401e-14\\
3.42600000000003	-6.73189657675241e-14\\
3.42999999999996	-6.51071082089807e-14\\
3.43	-6.51071082089541e-14\\
3.432	-6.40291716784043e-14\\
3.43200000000003	-6.40291716783891e-14\\
3.43400000000003	-6.29693307567402e-14\\
3.43600000000003	-6.1927169929153e-14\\
3.438	-6.09022806124953e-14\\
3.43800000000003	-6.09022806124808e-14\\
3.43999999999997	-5.98942610004869e-14\\
3.44	-5.98942610004726e-14\\
3.44199999999995	-5.89027159004588e-14\\
3.44399999999989	-5.7927256573239e-14\\
3.44599999999997	-5.69675005861849e-14\\
3.446	-5.69675005861714e-14\\
3.44999999999989	-5.50957547025057e-14\\
3.45099999999996	-5.46377258617942e-14\\
3.45099999999999	-5.46377258617812e-14\\
3.45499999999988	-5.28438941673753e-14\\
3.45699999999996	-5.19693344571355e-14\\
3.45699999999999	-5.19693344571232e-14\\
3.45999999999997	-5.0684473669263e-14\\
3.46	-5.0684473669251e-14\\
3.46299999999998	-4.9430982012412e-14\\
3.46499999999998	-4.86122010339132e-14\\
3.465	-4.86122010339017e-14\\
3.46599999999997	-4.82077537155713e-14\\
3.466	-4.82077537155599e-14\\
3.46699999999997	-4.78066628550027e-14\\
3.46799999999994	-4.74090031111321e-14\\
3.46999999999988	-4.6623821406753e-14\\
3.47199999999997	-4.58519007688688e-14\\
3.472	-4.58519007688579e-14\\
3.47599999999988	-4.43466372539689e-14\\
3.47999999999976	-4.289085177134e-14\\
3.47999999999996	-4.28908517712666e-14\\
3.47999999999999	-4.28908517712564e-14\\
3.48599999999996	-4.07949706807939e-14\\
3.48599999999999	-4.07949706807842e-14\\
3.49199999999996	-3.88013712151697e-14\\
3.49199999999999	-3.88013712151603e-14\\
3.49799999999996	-3.69064902015406e-14\\
3.5	-3.62956351327353e-14\\
3.50000000000003	-3.62956351327267e-14\\
3.506	-3.45220323361206e-14\\
3.50600000000003	-3.45220323361124e-14\\
3.50899999999999	-3.3667700600039e-14\\
3.50900000000002	-3.3667700600031e-14\\
3.51199999999998	-3.28349835459348e-14\\
3.51200000000003	-3.2834983545922e-14\\
3.51499999999999	-3.20231465793283e-14\\
3.51799999999995	-3.12314735372487e-14\\
3.51799999999999	-3.12314735372389e-14\\
3.51800000000003	-3.12314735372292e-14\\
3.51999999999997	-3.07145480897088e-14\\
3.52	-3.07145480897015e-14\\
3.52199999999995	-3.02060709927274e-14\\
3.52399999999989	-2.97058428873405e-14\\
3.52599999999997	-2.92136676571637e-14\\
3.526	-2.92136676571568e-14\\
3.52999999999989	-2.82538122635131e-14\\
3.53399999999978	-2.73261045973481e-14\\
3.53499999999998	-2.70990291726791e-14\\
3.535	-2.70990291726727e-14\\
3.53799999999997	-2.64290896315196e-14\\
3.53799999999999	-2.64290896315134e-14\\
3.53999999999997	-2.59916504973033e-14\\
3.54	-2.59916504972971e-14\\
3.54199999999998	-2.55613606242436e-14\\
3.54399999999996	-2.51380513164979e-14\\
3.54599999999997	-2.47215566140195e-14\\
3.546	-2.47215566140137e-14\\
3.54999999999996	-2.39092957322407e-14\\
3.54999999999999	-2.39092957322349e-14\\
3.553	-2.33180210381976e-14\\
3.55300000000003	-2.33180210381921e-14\\
3.55600000000004	-2.2741526381495e-14\\
3.55900000000005	-2.2179303206204e-14\\
3.55999999999997	-2.19949808416657e-14\\
3.56	-2.19949808416605e-14\\
3.56599999999999	-2.09201860526347e-14\\
3.56600000000002	-2.09201860526297e-14\\
3.56699999999996	-2.07461290814083e-14\\
3.56699999999999	-2.07461290814034e-14\\
3.56799999999996	-2.05735610769371e-14\\
3.56899999999992	-2.04024651228437e-14\\
3.56999999999998	-2.02328244498974e-14\\
3.57	-2.02328244498926e-14\\
3.57199999999993	-1.9897842582458e-14\\
3.57299999999996	-1.97324685561817e-14\\
3.57299999999999	-1.9732468556177e-14\\
3.57499999999992	-1.9405873275197e-14\\
3.57699999999985	-1.90847085465714e-14\\
3.57899999999996	-1.87688484564101e-14\\
3.57899999999999	-1.87688484564056e-14\\
3.57999999999997	-1.86128688731163e-14\\
3.58	-1.86128688731119e-14\\
3.58099999999998	-1.84581691725508e-14\\
3.58199999999997	-1.83047341919548e-14\\
3.58399999999993	-1.80015983602447e-14\\
3.58599999999997	-1.77033425324064e-14\\
3.586	-1.77033425324022e-14\\
3.58999999999993	-1.71216747696748e-14\\
3.59399999999986	-1.65594883754431e-14\\
3.59599999999996	-1.62854239850297e-14\\
3.59599999999999	-1.62854239850258e-14\\
3.59999999999997	-1.57508155831232e-14\\
3.6	-1.57508155831194e-14\\
3.60399999999998	-1.52335385755425e-14\\
3.60499999999998	-1.51068316994099e-14\\
3.605	-1.51068316994063e-14\\
3.60599999999997	-1.49811447840816e-14\\
3.606	-1.4981144784078e-14\\
3.60699999999997	-1.48565009291083e-14\\
3.60799999999994	-1.47329233353427e-14\\
3.60799999999999	-1.47329233353364e-14\\
3.60999999999993	-1.44889185871781e-14\\
3.61199999999987	-1.42490348772403e-14\\
3.61399999999996	-1.40131781581774e-14\\
3.61399999999999	-1.40131781581741e-14\\
3.61799999999987	-1.35531773668312e-14\\
3.61999999999997	-1.33288529493994e-14\\
3.62	-1.33288529493962e-14\\
3.62399999999988	-1.28911163057463e-14\\
3.62499999999996	-1.2783892822292e-14\\
3.62499999999999	-1.27838928222889e-14\\
3.62599999999997	-1.2677532462906e-14\\
3.626	-1.2677532462903e-14\\
3.62699999999998	-1.25720547751537e-14\\
3.62799999999997	-1.24674793923341e-14\\
3.62999999999993	-1.22609946310141e-14\\
3.63199999999997	-1.20579972257929e-14\\
3.632	-1.20579972257901e-14\\
3.63599999999993	-1.16621474809316e-14\\
3.63999999999985	-1.12793093228497e-14\\
3.63999999999997	-1.12793093228382e-14\\
3.64	-1.12793093228355e-14\\
3.64599999999997	-1.07281407165758e-14\\
3.646	-1.07281407165732e-14\\
3.65199999999997	-1.02038698296567e-14\\
3.652	-1.02038698296543e-14\\
3.65399999999996	-1.00349705806596e-14\\
3.65399999999999	-1.00349705806573e-14\\
3.65599999999995	-9.86888888468687e-15\\
3.65799999999992	-9.7055596259728e-15\\
3.65999999999996	-9.54491877064701e-15\\
3.65999999999999	-9.54491877064474e-15\\
3.66399999999992	-9.23145138388975e-15\\
3.66599999999996	-9.07850195632958e-15\\
3.66599999999999	-9.07850195632742e-15\\
3.66999999999992	-8.78021524041745e-15\\
3.67399999999984	-8.49191882090594e-15\\
3.67499999999998	-8.42135237599986e-15\\
3.67500000000001	-8.42135237599786e-15\\
3.67999999999997	-8.0772210113429e-15\\
3.68	-8.07722101134098e-15\\
3.68299999999996	-7.8774610363059e-15\\
3.68299999999999	-7.87746103630403e-15\\
3.68599999999995	-7.68252391691928e-15\\
3.686	-7.68252391691614e-15\\
3.68699999999998	-7.61860494339898e-15\\
3.68700000000001	-7.61860494339717e-15\\
3.68799999999998	-7.55523276241712e-15\\
3.68899999999995	-7.49240116276722e-15\\
3.6909999999999	-7.36833512723599e-15\\
3.69299999999998	-7.24635819623115e-15\\
3.693	-7.24635819622943e-15\\
3.6969999999999	-7.00848116495709e-15\\
3.69999999999997	-6.83520739242763e-15\\
3.7	-6.83520739242601e-15\\
3.7039999999999	-6.61073040574145e-15\\
3.70599999999997	-6.50120186131134e-15\\
3.706	-6.5012018613098e-15\\
3.7099999999999	-6.28759589803682e-15\\
3.70999999999998	-6.28759589803265e-15\\
3.71000000000001	-6.28759589803115e-15\\
3.71199999999996	-6.18349621494276e-15\\
3.71199999999999	-6.18349621494129e-15\\
3.71399999999995	-6.08114408110302e-15\\
3.71599999999991	-5.98049936895461e-15\\
3.71799999999996	-5.88152262034658e-15\\
3.71799999999999	-5.88152262034519e-15\\
3.71999999999997	-5.7841750316048e-15\\
3.72	-5.78417503160343e-15\\
3.72199999999998	-5.6884184377422e-15\\
3.72399999999997	-5.59421529700798e-15\\
3.72599999999997	-5.50152867669773e-15\\
3.726	-5.50152867669643e-15\\
3.728	-5.41037426623489e-15\\
3.72800000000003	-5.41037426623361e-15\\
3.73000000000004	-5.32076835573408e-15\\
3.73200000000004	-5.23267581484353e-15\\
3.73600000000004	-5.06089327452333e-15\\
3.74	-4.89475722128985e-15\\
3.74000000000003	-4.89475722128869e-15\\
3.74099999999996	-4.8540747496314e-15\\
3.74099999999999	-4.85407474963024e-15\\
3.74199999999996	-4.8137248721349e-15\\
3.74299999999992	-4.77370363223674e-15\\
3.74499999999985	-4.69463140706647e-15\\
3.745	-4.69463140706046e-15\\
3.74500000000003	-4.69463140705934e-15\\
3.746	-4.65557267174186e-15\\
3.74600000000003	-4.65557267174076e-15\\
3.747	-4.61683808009048e-15\\
3.74799999999997	-4.57843484205317e-15\\
3.74999999999991	-4.50260740325773e-15\\
3.752	-4.42806062691068e-15\\
3.75200000000003	-4.42806062690963e-15\\
3.75599999999991	-4.28269264860725e-15\\
3.758	-4.21181445455413e-15\\
3.75800000000003	-4.21181445455313e-15\\
3.75999999999997	-4.14210291743815e-15\\
3.76	-4.14210291743717e-15\\
3.76199999999995	-4.07353070694509e-15\\
3.76399999999989	-4.006070939069e-15\\
3.76599999999997	-3.93969716594415e-15\\
3.766	-3.93969716594322e-15\\
3.76999999999989	-3.81025297569902e-15\\
3.76999999999996	-3.8102529756967e-15\\
3.76999999999999	-3.8102529756958e-15\\
3.77399999999988	-3.68514416406106e-15\\
3.77599999999999	-3.62415395060065e-15\\
3.77600000000002	-3.6241539505998e-15\\
3.77999999999991	-3.50518234326653e-15\\
3.77999999999996	-3.50518234326517e-15\\
3.78	-3.50518234326379e-15\\
3.78399999999989	-3.3900676544095e-15\\
3.78599999999997	-3.33390000667705e-15\\
3.786	-3.33390000667626e-15\\
3.78999999999989	-3.22436011859564e-15\\
3.79199999999997	-3.17097646085078e-15\\
3.792	-3.17097646085003e-15\\
3.79599999999989	-3.0668770647031e-15\\
3.79899999999996	-2.99105676449011e-15\\
3.79899999999999	-2.9910567644894e-15\\
3.79999999999997	-2.96619941700361e-15\\
3.8	-2.96619941700291e-15\\
3.80099999999999	-2.94154603524286e-15\\
3.80199999999997	-2.91709420282777e-15\\
3.80399999999993	-2.86878561941597e-15\\
3.80599999999997	-2.8212547272006e-15\\
3.806	-2.82125472719993e-15\\
3.80999999999993	-2.7285585074914e-15\\
3.81099999999996	-2.70587511739019e-15\\
3.81099999999999	-2.70587511738955e-15\\
3.81499999999992	-2.61703751485092e-15\\
3.81499999999998	-2.61703751484968e-15\\
3.81500000000001	-2.61703751484906e-15\\
3.81899999999993	-2.53112961335576e-15\\
3.81999999999997	-2.51009451659137e-15\\
3.82	-2.51009451659077e-15\\
3.82399999999993	-2.42765978940598e-15\\
3.826	-2.38743760086111e-15\\
3.82600000000003	-2.38743760086055e-15\\
3.82700000000001	-2.36757400345833e-15\\
3.82700000000003	-2.36757400345776e-15\\
3.82799999999998	-2.34788032852752e-15\\
3.82800000000001	-2.34788032852696e-15\\
3.82899999999997	-2.32835464580846e-15\\
3.82999999999993	-2.30899504158203e-15\\
3.83199999999986	-2.27076649492656e-15\\
3.83399999999998	-2.23317970090865e-15\\
3.83400000000001	-2.23317970090812e-15\\
3.83799999999985	-2.15987267334099e-15\\
3.84	-2.12412369947627e-15\\
3.84000000000003	-2.12412369947576e-15\\
3.84399999999988	-2.0543647493511e-15\\
3.846	-2.02032742381392e-15\\
3.84600000000003	-2.02032742381344e-15\\
3.84999999999988	-1.95394677621798e-15\\
3.84999999999998	-1.95394677621635e-15\\
3.85000000000001	-1.95394677621589e-15\\
3.85399999999985	-1.88978936657767e-15\\
3.85599999999998	-1.85851279973314e-15\\
3.85600000000001	-1.8585127997327e-15\\
3.85699999999996	-1.84306965218393e-15\\
3.85699999999999	-1.8430696521835e-15\\
3.85799999999995	-1.82775457293141e-15\\
3.85899999999992	-1.81256606088122e-15\\
3.85999999999997	-1.79750262739686e-15\\
3.86	-1.79750262739644e-15\\
3.86199999999993	-1.76774510276659e-15\\
3.86399999999985	-1.73847033256396e-15\\
3.86599999999997	-1.70966683949997e-15\\
3.866	-1.70966683949956e-15\\
3.86899999999996	-1.66735697157706e-15\\
3.86899999999999	-1.66735697157666e-15\\
3.87199999999995	-1.62611754717516e-15\\
3.87499999999991	-1.58591218685251e-15\\
3.87999999999997	-1.52110524205442e-15\\
3.88	-1.52110524205405e-15\\
3.88499999999998	-1.45891371496958e-15\\
3.88500000000001	-1.45891371496924e-15\\
3.88599999999999	-1.44677573849589e-15\\
3.88600000000002	-1.44677573849555e-15\\
3.88700000000001	-1.4347384936649e-15\\
3.88799999999999	-1.42280422099449e-15\\
3.88999999999996	-1.39923992338075e-15\\
3.89199999999999	-1.37607360718049e-15\\
3.89200000000002	-1.37607360718016e-15\\
3.89599999999996	-1.33089874227508e-15\\
3.89799999999999	-1.30887248258588e-15\\
3.89800000000002	-1.30887248258557e-15\\
3.89999999999997	-1.28720877571083e-15\\
3.9	-1.28720877571052e-15\\
3.90199999999996	-1.26589912842366e-15\\
3.90399999999991	-1.24493518619302e-15\\
3.90599999999997	-1.22430873002368e-15\\
3.906	-1.22430873002339e-15\\
3.90999999999991	-1.18408237569279e-15\\
3.91399999999982	-1.14520329293432e-15\\
3.91499999999996	-1.13568684228049e-15\\
3.91499999999999	-1.13568684228022e-15\\
3.91999999999997	-1.08927796973181e-15\\
3.92	-1.08927796973155e-15\\
3.92499999999999	-1.04474202410593e-15\\
3.92599999999997	-1.03604990338948e-15\\
3.926	-1.03604990338924e-15\\
3.92699999999996	-1.02742991749022e-15\\
3.92699999999999	-1.02742991748998e-15\\
3.92799999999995	-1.01888367101379e-15\\
3.92899999999992	-1.01041032626442e-15\\
3.93099999999984	-9.93679027106171e-16\\
3.93299999999996	-9.77229460443666e-16\\
3.93299999999999	-9.77229460443434e-16\\
3.93699999999984	-9.45149836470864e-16\\
3.93999999999997	-9.21782479682593e-16\\
3.94	-9.21782479682374e-16\\
3.94399999999985	-8.91509959172686e-16\\
3.94399999999996	-8.9150995917184e-16\\
3.94399999999999	-8.91509959171628e-16\\
3.94599999999997	-8.76739157259053e-16\\
3.946	-8.76739157258845e-16\\
3.94799999999999	-8.62212532689804e-16\\
3.94999999999997	-8.47932681423118e-16\\
3.95199999999997	-8.3389400498823e-16\\
3.952	-8.33894004988032e-16\\
3.95499999999998	-8.13276178842771e-16\\
3.95500000000001	-8.13276178842578e-16\\
3.95799999999998	-7.93170446052971e-16\\
3.95999999999998	-7.80042343706762e-16\\
3.96	-7.80042343706577e-16\\
3.96299999999998	-7.60750901812056e-16\\
3.96599999999995	-7.41925218018914e-16\\
3.966	-7.41925218018606e-16\\
3.96600000000003	-7.4192521801843e-16\\
3.96700000000001	-7.35752363892729e-16\\
3.96700000000003	-7.35752363892554e-16\\
3.96800000000001	-7.29632315319631e-16\\
3.96899999999998	-7.23564472367265e-16\\
3.97099999999993	-7.11583029535743e-16\\
3.97299999999996	-6.99803338026933e-16\\
3.97299999999999	-6.99803338026767e-16\\
3.97699999999989	-6.7683081345564e-16\\
3.97899999999996	-6.65628973923737e-16\\
3.97899999999999	-6.65628973923579e-16\\
3.97999999999997	-6.6009722642908e-16\\
3.98	-6.60097226428923e-16\\
3.98099999999999	-6.5461086941376e-16\\
3.98199999999997	-6.49169365137196e-16\\
3.98399999999994	-6.3841878584758e-16\\
3.98599999999997	-6.27841273753923e-16\\
3.986	-6.27841273753774e-16\\
3.98999999999994	-6.07212681718417e-16\\
3.98999999999997	-6.07212681718237e-16\\
3.99000000000001	-6.07212681718057e-16\\
3.99399999999994	-5.87274987872209e-16\\
3.99599999999998	-5.77555415065065e-16\\
3.99600000000001	-5.77555415064928e-16\\
3.99999999999994	-5.58595763443891e-16\\
3.99999999999997	-5.5859576344374e-16\\
4	-5.58595763443591e-16\\
4.00199999999993	-5.49348251423728e-16\\
4.00199999999999	-5.49348251423468e-16\\
4.00399999999992	-5.40250761228686e-16\\
4.00599999999986	-5.3129972615191e-16\\
4.00599999999993	-5.31299726151587e-16\\
4.006	-5.3129972615126e-16\\
4.00999999999987	-5.13843139970736e-16\\
4.01199999999995	-5.05335769383313e-16\\
4.012	-5.05335769383073e-16\\
4.01599999999987	-4.88746195376633e-16\\
4.01999999999973	-4.72701920900456e-16\\
4.01999999999995	-4.72701920899619e-16\\
4.02	-4.72701920899395e-16\\
4.02500000000001	-4.53375148437054e-16\\
4.02500000000006	-4.53375148436839e-16\\
4.02599999999995	-4.49603124913084e-16\\
4.026	-4.49603124912871e-16\\
4.02699999999996	-4.45862404969258e-16\\
4.02799999999992	-4.42153684874015e-16\\
4.02999999999985	-4.34830793356763e-16\\
4.03099999999993	-4.31215904201346e-16\\
4.03099999999999	-4.31215904201141e-16\\
4.03499999999984	-4.17058492786135e-16\\
4.03699999999994	-4.10156227840072e-16\\
4.03699999999999	-4.10156227839877e-16\\
4.03799999999995	-4.06748014188217e-16\\
4.038	-4.06748014188024e-16\\
4.03899999999996	-4.03367966780376e-16\\
4.03999999999991	-4.00015754324191e-16\\
4.04	-4.00015754323889e-16\\
4.04199999999991	-3.93393522755086e-16\\
4.04399999999982	-3.86878723227418e-16\\
4.046	-3.80468801587922e-16\\
4.04600000000006	-3.80468801587741e-16\\
4.04999999999988	-3.6796797367067e-16\\
4.05399999999969	-3.55885826749259e-16\\
4.05999999999994	-3.38506354211919e-16\\
4.06	-3.38506354211758e-16\\
};
\end{axis}
\end{tikzpicture}%
}
      \caption{Evolution of the angular displacement of pendulum $P_1$ for
        $C_1 = 6$ ms. \texttt{Blue}: RM scheduling, \texttt{Red}: EDF scheduling}
      \label{fig:02.6.6.1}
    \end{figure}
  \end{minipage}
  \hfill
  \begin{minipage}{0.45\linewidth}
    \begin{figure}[H]\centering
      \scalebox{0.7}{% This file was created by matlab2tikz.
%
%The latest updates can be retrieved from
%  http://www.mathworks.com/matlabcentral/fileexchange/22022-matlab2tikz-matlab2tikz
%where you can also make suggestions and rate matlab2tikz.
%
\definecolor{mycolor1}{rgb}{0.00000,0.44700,0.74100}%
%
\begin{tikzpicture}

\begin{axis}[%
width=4.133in,
height=3.26in,
at={(0.693in,0.44in)},
scale only axis,
xmin=0,
xmax=1.2,
xmajorgrids,
ymin=-0.0001,
ymax=0.0006,
ymajorgrids,
axis background/.style={fill=white}
]
\pgfplotsset{max space between ticks=50}
\addplot [color=mycolor1,solid,forget plot]
  table[row sep=crcr]{%
0	0\\
3.15544362088405e-30	0\\
0.000656101980281985	0\\
0.00393661188169191	0\\
0.00599999999999994	0\\
0.006	0\\
0.012	0\\
0.0120000000000001	0\\
0.018	0\\
0.0180000000000001	0\\
0.0199999999999998	0\\
0.02	0\\
0.026	0\\
0.0260000000000002	0\\
0.0289999999999998	0\\
0.029	0\\
0.0319999999999996	0\\
0.0349999999999991	0\\
0.035	0\\
0.0399999999999996	0\\
0.04	0\\
0.0449999999999996	0\\
0.0459999999999996	0\\
0.046	0\\
0.047	0\\
0.0470000000000004	0\\
0.0490000000000003	0\\
0.0510000000000002	0\\
0.055	0\\
0.0579999999999996	0\\
0.058	0\\
0.0599999999999996	0\\
0.06	0\\
0.0619999999999995	0\\
0.0639999999999991	0\\
0.0659999999999991	0\\
0.066	0\\
0.0699999999999991	0\\
0.07	0\\
0.0700000000000009	0\\
0.074	0\\
0.076	0\\
0.0760000000000009	0\\
0.08	0\\
0.0800000000000009	0\\
0.0839999999999999	0\\
0.086	0\\
0.0860000000000009	0\\
0.0869999999999991	0\\
0.087	0\\
0.0880000000000004	0\\
0.0890000000000009	0\\
0.0910000000000017	0\\
0.0929999999999991	0\\
0.093	0\\
0.0970000000000017	0\\
0.0999999999999991	0\\
0.1	0\\
0.104000000000002	0\\
0.104999999999999	0\\
0.105	0\\
0.105999999999999	0\\
0.106	0\\
0.106999999999999	0\\
0.107999999999998	0\\
0.109999999999997	0\\
0.111999999999999	0\\
0.112	0\\
0.115999999999997	0\\
0.115999999999998	0\\
0.116	0\\
0.119999999999997	0\\
0.119999999999998	0\\
0.12	0\\
0.123999999999997	0\\
0.125999999999999	0\\
0.126	0\\
0.127999999999998	0\\
0.128	0\\
0.129999999999998	0\\
0.131999999999996	0\\
0.135999999999993	0\\
0.139999999999998	0\\
0.14	0\\
0.144999999999998	0\\
0.145	0\\
0.145999999999998	0\\
0.146	0\\
0.146999999999999	0\\
0.147999999999998	0\\
0.149999999999997	0\\
0.151999999999998	0\\
0.152	0\\
0.155999999999997	0\\
0.157999999999998	0\\
0.158	0\\
0.16	0\\
0.160000000000002	0\\
0.162000000000002	0\\
0.164000000000002	0\\
0.166	0\\
0.166000000000002	0\\
0.170000000000002	0\\
0.174	0\\
0.174000000000001	0\\
0.175	0\\
0.175000000000002	0\\
0.176000000000001	0\\
0.177	0\\
0.178999999999998	0\\
0.179999999999998	0\\
0.18	0\\
0.183999999999997	0\\
0.186	0\\
0.186000000000002	0\\
0.189999999999998	0\\
0.192	0\\
0.192000000000002	0\\
0.195999999999998	0\\
0.199999999999995	0\\
0.199999999999997	0\\
0.2	0\\
0.202999999999998	0\\
0.203	0\\
0.205999999999998	0\\
0.206	0\\
0.208999999999998	0\\
0.209999999999998	0\\
0.21	0\\
0.211999999999998	0\\
0.212	0\\
0.213999999999998	0\\
0.215999999999997	0\\
0.217999999999998	0\\
0.218	0\\
0.219999999999998	0\\
0.22	0\\
0.221999999999998	0\\
0.223999999999996	0\\
0.225999999999998	0\\
0.226	0\\
0.229999999999996	0\\
0.231999999999998	0\\
0.232	0\\
0.235999999999996	0\\
0.237999999999998	0\\
0.238	0\\
0.239999999999998	0\\
0.24	0\\
0.241999999999998	0\\
0.243999999999996	0\\
0.245	0\\
0.245000000000002	0\\
0.245999999999998	0\\
0.246	0\\
0.246999999999999	0\\
0.247999999999998	0\\
0.249999999999997	0\\
0.252	0\\
0.252000000000003	0\\
0.256	0\\
0.259999999999997	0\\
0.26	0\\
0.260999999999996	0\\
0.261	0\\
0.261999999999998	0\\
0.262999999999996	0\\
0.264999999999993	0\\
0.265999999999997	0\\
0.266	0\\
0.269999999999993	0\\
0.271999999999997	0\\
0.272	0\\
0.275999999999993	0\\
0.279999999999986	0\\
0.279999999999993	0\\
0.28	0\\
0.285999999999996	0\\
0.286	0\\
0.289999999999996	0\\
0.29	0\\
0.291999999999996	0.000254750482271644\\
0.292	0.000502950859000992\\
0.293999999999996	0.00024820037601846\\
0.295999999999993	0.000241747576987557\\
0.297999999999996	-1.72934583320128e-14\\
0.298	0.000235389554050298\\
0.299999999999996	-6.59141144443431e-13\\
0.3	0.00022912381590693\\
0.301999999999996	0.000222947906691365\\
0.303999999999993	0.000216859405065684\\
0.305999999999996	-6.09286926467334e-13\\
0.306	0.000418882409984507\\
0.309999999999993	0.000401705846888282\\
0.313999999999986	9.78170358033482e-05\\
0.314999999999997	-5.17204334915533e-13\\
0.315	0.000381118061766883\\
0.318999999999997	-4.63988222842993e-13\\
0.319	9.2788681893009e-05\\
0.319999999999996	-4.49487669307302e-13\\
0.32	9.18106841000816e-05\\
0.320999999999998	9.08416850599857e-05\\
0.321999999999996	0.000178811893025415\\
0.323999999999993	0.000175041520349094\\
0.325999999999996	-3.52343154652601e-13\\
0.326	0.000340517686189184\\
0.329999999999993	8.33388073232348e-05\\
0.331	-2.62807137163534e-13\\
0.331000000000004	0.00016458629915294\\
0.333	-2.26254778801227e-13\\
0.333000000000004	0.000161851536295186\\
0.335	0.000159180227974714\\
0.336999999999996	0.000233897873636882\\
0.339999999999996	-9.47315142996175e-14\\
0.34	0.000303088205402606\\
0.343999999999993	0.000147917074179231\\
0.345999999999997	2.28064095386671e-14\\
0.346	0.000145656706364959\\
0.347999999999997	6.2302593639707e-14\\
0.348	0.000143533459668062\\
0.349999999999997	1.00862027063719e-13\\
0.35	0.000141466485828946\\
0.351999999999997	0.000139454974095534\\
0.353999999999993	1.75472483765482e-13\\
0.354	0.000273093339554191\\
0.357999999999993	0.000133745432601327\\
0.359999999999996	2.80116208006831e-13\\
0.36	0.000262150584732713\\
0.363999999999993	0.000128507929600868\\
0.365999999999996	3.77342254664903e-13\\
0.366	0.000251812872896377\\
0.369999999999993	0.000244990073467109\\
0.373999999999986	0.000179496628307924\\
0.376999999999997	5.27154708773736e-13\\
0.377	0.000176033484076564\\
0.379999999999997	5.6063834130704e-13\\
0.38	0.000172725618180685\\
0.382999999999996	0.000113388075297859\\
0.384999999999997	6.09561012776538e-13\\
0.385	5.6182058071585e-05\\
0.385999999999997	6.18323101053697e-13\\
0.386	5.58127713231613e-05\\
0.386999999999998	5.5413635140545e-05\\
0.387999999999996	5.50199300791802e-05\\
0.388999999999997	6.4262744431387e-13\\
0.389	0.000108880276320161\\
0.390999999999997	0.000107369674706829\\
0.392999999999993	0.000105901167783328\\
0.394999999999997	6.82485318259651e-13\\
0.395	0.00020756232795174\\
0.398999999999993	5.10369651306446e-05\\
0.399999999999997	7.06864254629291e-13\\
0.4	0.000200883465178965\\
0.403999999999993	9.85518984586318e-05\\
0.405999999999997	7.25606207063745e-13\\
0.406	0.000192947485009984\\
0.409999999999993	9.43287862436287e-05\\
0.411999999999997	7.34651055267488e-13\\
0.412	0.000184552580546437\\
0.415999999999993	0.000179319575685793\\
0.419999999999986	7.34662330970082e-13\\
0.419999999999996	7.34344876573978e-13\\
0.42	0.000259780553636858\\
0.426	7.24832520393459e-13\\
0.426000000000004	0.000248509632154599\\
0.432000000000004	7.09327561965178e-13\\
0.432000000000007	0.000119912242930002\\
0.434999999999997	7.00058067071296e-13\\
0.435	0.000117159072780928\\
0.43799999999999	7.66276459620921e-05\\
0.439999999999997	6.82333529955503e-13\\
0.44	0.00011279897179893\\
0.44299999999999	0.000110316258277618\\
0.445999999999979	6.57695252426205e-13\\
0.445999999999995	6.57316215346704e-13\\
0.446	3.62164357366798e-05\\
0.447	6.52711391879723e-13\\
0.447000000000004	3.59133204272085e-05\\
0.448000000000004	3.56137256050061e-05\\
0.449000000000004	7.03425991449413e-05\\
0.451000000000004	0.000137241716994886\\
0.454999999999997	6.15322029440257e-13\\
0.455	0.000132829034324977\\
0.459	3.25423377512718e-05\\
0.459999999999997	5.90950031964521e-13\\
0.46	0.000127605356216285\\
0.463999999999997	5.70846321601426e-13\\
0.464	6.23079698770278e-05\\
0.465999999999997	5.60582830155809e-13\\
0.466	6.12891009656846e-05\\
0.466999999999997	9.07509808604446e-05\\
0.467	0.000149963825770687\\
0.467999999999998	0.000119452009615675\\
0.468999999999997	0.000118431493784418\\
0.470999999999993	5.87084160596982e-05\\
0.472999999999997	2.89809178512009e-05\\
0.473	5.77167701428685e-05\\
0.476999999999993	6.08129432227988e-13\\
0.479999999999997	5.90443926390405e-13\\
0.48	5.90443926390405e-13\\
0.483999999999993	5.67807953433253e-13\\
0.485999999999997	5.56344467023129e-13\\
0.486	5.56344467023129e-13\\
0.489999999999993	5.3376313785547e-13\\
0.49	5.33670330149505e-13\\
0.490000000000004	7.46591198725831e-05\\
0.492999999999997	5.16448429160876e-13\\
0.493	7.27425035395883e-05\\
0.495999999999993	7.08900571534548e-05\\
0.498999999999986	4.81232675236809e-13\\
0.498999999999993	2.32292196598958e-05\\
0.499	2.32292196598134e-05\\
0.499999999999997	2.30326300051592e-05\\
0.5	4.58709277143003e-05\\
0.500999999999998	6.79408328691945e-05\\
0.501999999999997	8.94543599183196e-05\\
0.503999999999993	4.43518252150241e-05\\
0.505999999999993	4.35956891710423e-05\\
0.506	4.35956891710393e-05\\
0.507999999999997	4.28338305857399e-05\\
0.508000000000004	8.49225953464291e-05\\
0.51	0.000124096816665271\\
0.511999999999997	0.000161230181506548\\
0.51599999999999	7.92221299908053e-05\\
0.519999999999993	3.86053366970376e-05\\
0.52	3.86053366970354e-05\\
0.521999999999993	3.79551155886568e-05\\
0.522	5.66935011125673e-05\\
0.523999999999993	1.87383858530525e-05\\
0.524999999999993	1.85813917095499e-05\\
0.525	1.85813917095486e-05\\
0.526	1.84221692087444e-05\\
0.526000000000007	3.668287164853e-05\\
0.527000000000007	5.43048515807632e-05\\
0.528000000000007	7.14637067097132e-05\\
0.530000000000007	3.54195578687292e-05\\
0.532	6.90206468826355e-05\\
0.532000000000007	0.000102648795723191\\
0.536000000000007	3.3628149102322e-05\\
0.538	3.3057687603655e-05\\
0.538000000000007	3.30576876036528e-05\\
0.539999999999993	3.25001863250018e-05\\
0.54	6.4455612966747e-05\\
0.541999999999986	6.33786223215849e-05\\
0.543999999999972	3.14231956702504e-05\\
0.546	6.12467265819576e-05\\
0.546000000000007	7.62284973012623e-05\\
0.549999999999979	1.49817709141863e-05\\
0.550999999999993	5.86514940892886e-05\\
0.551	8.72319643963294e-05\\
0.554999999999972	2.85804704813989e-05\\
0.556999999999993	4.19706927669659e-05\\
0.557	4.19706927669635e-05\\
0.559999999999993	4.0923517867977e-05\\
0.56	8.08359614568165e-05\\
0.562999999999993	3.99124437340218e-05\\
0.565999999999986	1.35428343886468e-13\\
0.565999999999993	7.68279989140836e-05\\
0.566	7.68279989140789e-05\\
0.571999999999986	1.17733730750635e-13\\
0.571999999999993	7.29875514351886e-05\\
0.572	9.65081992734833e-05\\
0.577999999999986	2.35206479406566e-05\\
0.579999999999993	6.82654565848069e-05\\
0.58	6.82654565848041e-05\\
0.585999999999986	8.5379403200192e-14\\
0.585999999999993	6.491094934748e-05\\
0.586	6.49109493474759e-05\\
0.591999999999986	7.49526309073989e-14\\
0.591999999999993	3.12337271536648e-05\\
0.592	3.12337271536618e-05\\
0.594999999999993	3.04562019430368e-05\\
0.595	5.0341673480172e-05\\
0.597999999999993	1.98854716032875e-05\\
0.599999999999993	2.92197037534718e-05\\
0.6	5.77320678464325e-05\\
0.602999999999993	2.85123641530661e-05\\
0.605999999999986	9.34885162879782e-06\\
0.606	9.34885162873255e-06\\
0.606999999999993	9.26907768244176e-06\\
0.607	1.84592899867915e-05\\
0.607999999999999	9.1902123593409e-06\\
0.608999999999997	1.8147424000563e-05\\
0.609000000000004	3.59900966821034e-05\\
0.611	3.53875894386535e-05\\
0.612999999999997	1.75449167554706e-05\\
0.614999999999997	3.42239652763104e-05\\
0.615000000000004	4.2604469324464e-05\\
0.618999999999997	8.38050409455201e-06\\
0.619999999999993	3.28447104233896e-05\\
0.62	4.88719263259426e-05\\
0.623999999999993	1.60272159460933e-05\\
0.625999999999993	3.1260166923999e-05\\
0.626	3.12601669239981e-05\\
0.629999999999993	3.02182255722318e-05\\
0.63	4.49514824715627e-05\\
0.633999999999993	1.47332569386774e-05\\
0.635999999999993	1.44904204104698e-05\\
0.636	1.44904204104691e-05\\
0.637999999999993	1.42532648312344e-05\\
0.638	1.42532648312334e-05\\
0.639999999999993	1.4021697301001e-05\\
0.64	2.78173242582483e-05\\
0.641999999999993	2.73705922735573e-05\\
0.643999999999986	1.35749653158278e-05\\
0.645999999999993	2.64779351664673e-05\\
0.646	2.64779351665112e-05\\
0.649999999999986	3.52722408572537e-14\\
0.65	2.55955908793945e-05\\
0.650000000000007	5.03489805006599e-05\\
0.653999999999993	3.68264415979901e-05\\
0.657999999999979	1.207305197608e-05\\
0.659999999999993	2.93348923551086e-05\\
0.66	2.93348923551069e-05\\
0.664999999999993	5.72626166454849e-06\\
0.665	5.72626166454816e-06\\
0.665999999999993	5.67899425500239e-06\\
0.666	5.67899425500207e-06\\
0.666999999999998	5.63060153197859e-06\\
0.667000000000006	1.12133621658974e-05\\
0.668000000000004	1.66069433772923e-05\\
0.669000000000002	2.18635081047568e-05\\
0.670999999999998	1.08393253936899e-05\\
0.673000000000005	2.11410060355987e-05\\
0.673000000000013	2.63172763086838e-05\\
0.677000000000005	5.17627030456581e-06\\
0.678	1.02253292854788e-05\\
0.678000000000007	1.02253292854782e-05\\
0.679999999999993	1.00590480224988e-05\\
0.68	1.99557584407437e-05\\
0.681999999999986	1.9634963337661e-05\\
0.683999999999972	9.73825291918215e-06\\
0.686	1.89935166492117e-05\\
0.686000000000007	3.73525478656751e-05\\
0.689999999999979	2.73096509846955e-05\\
0.69399999999995	8.95061976774079e-06\\
0.695999999999993	1.74609932121431e-05\\
0.696	1.74609932121422e-05\\
0.699999999999993	1.68967234300586e-05\\
0.7	2.51417989965631e-05\\
0.703999999999993	8.24507559466436e-06\\
0.705999999999993	8.10908815021322e-06\\
0.706	8.10908815021273e-06\\
0.707999999999993	7.97148896618065e-06\\
0.708	1.58085039748895e-05\\
0.709999999999993	1.5542628646679e-05\\
0.711999999999986	7.7056136376845e-06\\
0.713999999999993	1.50290565985986e-05\\
0.714	2.23583917604549e-05\\
0.717999999999986	7.32933518801838e-06\\
0.719999999999993	1.43026525218685e-05\\
0.72	1.78061937843518e-05\\
0.723999999999986	3.50354128773449e-06\\
0.724999999999993	3.47538351493554e-06\\
0.725	3.47538351493527e-06\\
0.725999999999993	3.44656517953403e-06\\
0.726	6.86364863745838e-06\\
0.726999999999999	1.01641420907463e-05\\
0.727999999999997	1.33801724463884e-05\\
0.729999999999993	6.63311383803327e-06\\
0.731999999999993	9.74170087662715e-06\\
0.732	9.74170087662644e-06\\
0.734999999999993	9.49975963486559e-06\\
0.735	1.57026678565178e-05\\
0.737999999999993	6.20290824469518e-06\\
0.74	9.11504492109985e-06\\
0.740000000000007	1.80100415618122e-05\\
0.743	8.89499666297471e-06\\
0.745999999999993	2.91662411353575e-06\\
0.746000000000007	2.91662411351537e-06\\
0.746999999999993	2.89165970191461e-06\\
0.747	5.75863843253234e-06\\
0.747999999999999	8.52801494768566e-06\\
0.748999999999997	1.39744310868773e-05\\
0.750999999999993	8.31339489092343e-06\\
0.753999999999993	1.0763322096635e-05\\
0.754	1.60120814022608e-05\\
0.757999999999993	5.24875932547285e-06\\
0.759999999999993	1.02420048278352e-05\\
0.76	1.52392657486551e-05\\
0.763999999999993	4.99726093973268e-06\\
0.766	9.74561060045992e-06\\
0.766000000000007	9.74561060045927e-06\\
0.77	9.41892848452708e-06\\
0.770000000000007	1.40105226421857e-05\\
0.774	4.59159417505678e-06\\
0.776	8.95644953727775e-06\\
0.776000000000007	8.95644953727715e-06\\
0.779999999999993	6.52583959271128e-06\\
0.78	6.52583959271085e-06\\
0.782999999999993	6.36810852324148e-06\\
0.783	6.36810852324115e-06\\
0.785999999999993	6.21068347059092e-06\\
0.786000000000001	1.22641090123414e-05\\
0.788999999999994	6.05342555704295e-06\\
0.791999999999987	1.48927365665324e-14\\
0.792	1.16563034896847e-05\\
0.792000000000008	1.54138396691549e-05\\
0.797999999999994	3.75753619357821e-06\\
0.799999999999993	9.12805801559305e-06\\
0.8	9.12805801559248e-06\\
0.804999999999993	1.78148330809183e-06\\
0.805000000000001	1.78148330809172e-06\\
0.805999999999993	1.76668636918303e-06\\
0.806	3.51825072825789e-06\\
0.806999999999994	5.21001232013096e-06\\
0.807999999999987	6.85844903970992e-06\\
0.809999999999973	3.40000109143792e-06\\
0.811999999999993	6.62997102261325e-06\\
0.812	9.86254026026325e-06\\
0.815999999999973	3.23256924965166e-06\\
0.817999999999993	3.17932196605068e-06\\
0.818000000000001	3.17932196605052e-06\\
0.819999999999993	3.12732113118668e-06\\
0.82	6.20386749478129e-06\\
0.821999999999993	6.10352415434518e-06\\
0.823999999999986	3.02697779054766e-06\\
0.825999999999993	5.9032293300205e-06\\
0.826	1.16086738144749e-05\\
0.829999999999986	1.39122575515186e-05\\
0.833999999999972	8.20681306669139e-06\\
0.839999999999993	1.32854224797958e-06\\
0.84	1.32854224797949e-06\\
0.840999999999993	1.31767590625593e-06\\
0.841000000000001	2.62461466791372e-06\\
0.841999999999994	3.88911622404049e-06\\
0.842999999999987	3.85766923851683e-06\\
0.844999999999973	1.27549178547618e-06\\
0.845999999999993	4.99517213480721e-06\\
0.846	7.42953311420601e-06\\
0.849999999999973	2.434360988257e-06\\
0.851999999999993	4.74704411906196e-06\\
0.852	7.0615883473622e-06\\
0.855999999999973	2.31454423670386e-06\\
0.857999999999993	2.27644067214515e-06\\
0.858	2.27644067214502e-06\\
0.86	2.23922958296493e-06\\
0.860000000000007	4.44212596760647e-06\\
0.862000000000007	4.37032324100207e-06\\
0.864000000000007	2.16742685622313e-06\\
0.866	4.22697874993098e-06\\
0.866000000000007	4.22697874993074e-06\\
0.87	4.08541196994691e-06\\
0.870000000000007	5.08540075466692e-06\\
0.874	9.99988791910487e-07\\
0.874999999999994	9.91640374118049e-07\\
0.875000000000001	9.91640374117995e-07\\
0.876	9.83389143742336e-07\\
0.876000000000007	1.95862343617517e-06\\
0.877000000000007	2.90161989410431e-06\\
0.878000000000006	1.92638560161174e-06\\
0.879999999999998	1.89490300885827e-06\\
0.880000000000006	3.75906631725377e-06\\
0.882000000000005	3.69831778223734e-06\\
0.884000000000004	1.83415447372862e-06\\
0.886000000000005	1.80383632019923e-06\\
0.886000000000013	1.80383632019914e-06\\
0.888000000000007	1.77319696855571e-06\\
0.888000000000014	3.51644976810377e-06\\
0.890000000000009	5.14264822273486e-06\\
0.892000000000004	5.87533909318519e-06\\
0.895999999999993	2.47594367606996e-06\\
0.898999999999993	8.11729057646112e-07\\
0.899000000000001	8.11729057646057e-07\\
0.899999999999993	8.05069422408906e-07\\
0.9	1.60355814612217e-06\\
0.900999999999994	2.37603650158101e-06\\
0.901999999999987	3.12970366542363e-06\\
0.903999999999973	1.55215589318359e-06\\
0.905999999999993	3.02707405141619e-06\\
0.906	3.02707405142102e-06\\
0.909999999999973	5.31805231362287e-15\\
0.909999999999987	7.40747788859734e-07\\
0.910000000000001	7.40747788854719e-07\\
0.910999999999993	7.34485631351169e-07\\
0.911000000000001	1.46278108998206e-06\\
0.911999999999994	2.16660080116261e-06\\
0.912999999999987	2.85269882552801e-06\\
0.914999999999973	1.41439348813164e-06\\
0.916999999999993	2.07796598250425e-06\\
0.917000000000001	2.07796598250411e-06\\
0.919999999999993	2.02724289222548e-06\\
0.92	4.00555102584262e-06\\
0.922999999999993	1.97830813838303e-06\\
0.925999999999986	4.64926220352518e-15\\
0.925999999999993	1.29181157876669e-06\\
0.926	1.29181157876654e-06\\
0.927999999999993	1.26986950570137e-06\\
0.928000000000001	2.51829481250905e-06\\
0.929999999999994	2.47589587372429e-06\\
0.931999999999987	1.22747056684296e-06\\
0.933999999999994	2.39399377804589e-06\\
0.934000000000001	3.56145556398258e-06\\
0.937999999999987	1.16746179014785e-06\\
0.939999999999993	2.2781420649702e-06\\
0.940000000000001	2.83617438997503e-06\\
0.943999999999987	5.58032329012992e-07\\
0.944999999999994	5.5354110953334e-07\\
0.945000000000001	5.53541109533326e-07\\
0.945999999999993	5.48947953845638e-07\\
0.946000000000001	1.09320036580293e-06\\
0.946999999999994	1.61888354055192e-06\\
0.947999999999987	2.13111432933805e-06\\
0.949999999999973	1.0564832045903e-06\\
0.952	2.06017244876353e-06\\
0.952000000000008	2.56451190844728e-06\\
0.95599999999998	5.04339463314799e-07\\
0.956999999999994	1.48812220350794e-06\\
0.957000000000001	1.48812220350786e-06\\
0.96	1.45179471374187e-06\\
0.960000000000008	2.86854263794638e-06\\
0.963000000000007	1.41674792763205e-06\\
0.966000000000007	1.38176149674242e-06\\
0.966000000000014	1.38176149674234e-06\\
0.969000000000007	1.3468045001041e-06\\
0.969000000000014	2.65984010520498e-06\\
0.972000000000007	3.42951924176241e-06\\
0.975	2.11648363658361e-06\\
0.979999999999994	2.4274809009809e-06\\
0.980000000000001	2.42748090098347e-06\\
0.985999999999987	2.30902138896479e-06\\
0.986000000000001	2.30902138896452e-06\\
0.991999999999987	2.19468554486052e-06\\
0.992000000000001	2.19468554486027e-06\\
0.997999999999987	7.07503785403881e-07\\
0.998000000000001	7.07503785401388e-07\\
0.999999999999993	6.95940915729139e-07\\
1	1.38059181732785e-06\\
1.00199999999999	1.35828021408341e-06\\
1.00399999999999	6.73629312446154e-07\\
1.00599999999999	1.3137327260313e-06\\
1.006	2.58346508373771e-06\\
1.00999999999999	1.58052524430985e-06\\
1.01399999999997	3.10792886527256e-07\\
1.01499999999999	1.51564310265886e-06\\
1.015	1.51564310265867e-06\\
1.01999999999999	1.4544753382984e-06\\
1.02	1.73834649989691e-06\\
1.02499999999999	2.8387116365323e-07\\
1.02599999999999	2.81515308729484e-07\\
1.026	2.8151530872945e-07\\
1.02699999999999	2.79107041207307e-07\\
1.027	5.55833170728695e-07\\
1.02799999999999	8.23143919035037e-07\\
1.02899999999999	1.08363442099842e-06\\
1.03099999999997	5.37216633455706e-07\\
1.03299999999999	1.04766873059397e-06\\
1.033	1.81080828150068e-06\\
1.03699999999997	7.63139552765624e-07\\
1.04	9.88650056334894e-07\\
1.04000000000001	9.88650056334778e-07\\
1.04399999999999	4.82389407409441e-07\\
1.044	4.82389407409384e-07\\
1.046	4.74414769156814e-07\\
1.04600000000001	9.40770146281225e-07\\
1.04800000000001	4.66355378820103e-07\\
1.05	4.58478823679526e-07\\
1.05000000000001	4.58478823679469e-07\\
1.05200000000001	4.50782015712044e-07\\
1.05200000000002	8.94043951477479e-07\\
1.05400000000002	8.79177577374127e-07\\
1.05600000000002	4.35915641581902e-07\\
1.05800000000001	4.28740245834668e-07\\
1.05800000000002	4.28740245834617e-07\\
1.05999999999999	4.21732937101834e-07\\
1.06	8.36623906031561e-07\\
1.06199999999996	8.23102632342568e-07\\
1.06399999999992	4.08211663387762e-07\\
1.06599999999999	7.96106546033759e-07\\
1.066	1.37562153922309e-06\\
1.06999999999992	5.79514994600081e-07\\
1.07299999999999	7.50238767224709e-07\\
1.073	1.11606922662096e-06\\
1.07699999999992	3.65830460726805e-07\\
1.07899999999999	1.80656141802625e-07\\
1.079	1.80656141802608e-07\\
1.07999999999999	1.79173738787916e-07\\
1.08	3.56882634838591e-07\\
1.08099999999999	5.2880169253962e-07\\
1.08199999999999	5.24511107019145e-07\\
1.08399999999997	1.73418311795779e-07\\
1.08499999999999	1.72022295193934e-07\\
1.085	1.72022295193917e-07\\
1.08599999999999	1.70594673865408e-07\\
1.086	3.39729981797945e-07\\
1.08699999999999	5.03093997595808e-07\\
1.08799999999999	6.62277015753433e-07\\
1.08999999999997	3.28318327293725e-07\\
1.09199999999999	6.40228271450222e-07\\
1.092	1.25941190586167e-06\\
1.09599999999997	6.1918363554843e-07\\
1.09999999999995	1.10167137998968e-15\\
1.09999999999997	3.02004814952827e-07\\
1.1	3.02004814950764e-07\\
1.10199999999999	2.97105305860844e-07\\
1.102	5.89427600823626e-07\\
1.10399999999999	2.92322296024341e-07\\
1.10599999999997	1.04397149562271e-15\\
1.10599999999999	5.70095858087896e-07\\
1.106	8.47928870176326e-07\\
1.10999999999997	2.77833013098009e-07\\
1.11199999999999	5.41780855816927e-07\\
1.112	1.06575323666624e-06\\
1.11599999999997	5.23972381814986e-07\\
1.11999999999994	9.34104546100253e-16\\
1.12	7.54358009957495e-07\\
1.12000000000001	7.54358009957405e-07\\
1.126	6.00487758048004e-07\\
1.12600000000001	6.00487758047931e-07\\
1.13099999999999	5.75528676859457e-07\\
1.131	6.8776448204023e-07\\
1.132	5.70708012977454e-07\\
1.13200000000001	6.82013058683307e-07\\
1.13300000000001	5.65943285416188e-07\\
1.13400000000001	5.6123402521602e-07\\
1.13600000000001	4.4340220784106e-07\\
1.138	2.19861358281812e-07\\
1.13800000000001	3.28439219402909e-07\\
1.13999999999999	2.16268062815632e-07\\
1.14	4.29027618405134e-07\\
1.14199999999997	4.22094018179425e-07\\
1.14399999999994	2.09334462584648e-07\\
1.14599999999999	4.08250573131726e-07\\
1.146	8.02827947743517e-07\\
1.14999999999994	4.9115826330094e-07\\
1.15399999999989	9.65808886841834e-08\\
1.15499999999999	4.70996030655447e-07\\
1.155	4.70996030655391e-07\\
1.15999999999999	4.51988065113812e-07\\
1.16	5.40202991562663e-07\\
1.16499999999999	8.82149269085525e-08\\
1.16599999999999	4.30015209604688e-07\\
1.166	5.13840513291239e-07\\
1.17099999999999	8.38253041523645e-08\\
1.17199999999999	8.31187258783301e-08\\
1.172	8.311872587832e-08\\
1.173	8.24202933937586e-08\\
1.17300000000001	1.64150232601357e-07\\
1.17400000000001	2.43150731601092e-07\\
1.17500000000001	3.98572429986359e-07\\
1.17700000000001	2.37151638066759e-07\\
1.17999999999999	3.07231107483611e-07\\
1.18	4.57137701435453e-07\\
1.184	1.49906594443548e-07\\
1.18599999999999	2.20199987779763e-07\\
1.186	2.20199987779735e-07\\
1.18899999999999	7.21525342295983e-08\\
1.189	7.21525342295899e-08\\
1.18999999999999	7.15406399534345e-08\\
1.19	1.42476396929417e-07\\
1.19099999999999	2.1102037519723e-07\\
1.19199999999999	2.09247208964666e-07\\
1.19399999999997	6.91625912597499e-08\\
1.19499999999999	2.70943087775639e-07\\
1.195	3.37285234896719e-07\\
1.19899999999997	6.63421476525654e-08\\
1.19999999999999	2.59989172072623e-07\\
1.2	3.86845112896846e-07\\
1.20399999999997	1.26855941367326e-07\\
1.20599999999999	6.26473993181351e-08\\
1.206	6.26473993181284e-08\\
1.20699999999999	6.21114858984045e-08\\
1.207	1.23693145631008e-07\\
1.20799999999999	1.83179594137461e-07\\
1.20899999999999	3.58698056148874e-07\\
1.21099999999997	4.09774070215191e-07\\
1.21499999999995	1.72673948479153e-07\\
1.21799999999999	1.12748014662821e-07\\
1.218	1.12748014662808e-07\\
1.21999999999999	1.10905324459683e-07\\
1.22	2.20011445718886e-07\\
1.22199999999999	1.62997876011839e-07\\
1.22399999999997	5.38917547537746e-08\\
1.22499999999999	5.34579405996526e-08\\
1.225	5.3457940599645e-08\\
1.22599999999999	5.30143010600456e-08\\
1.226	1.05575093535467e-07\\
1.22699999999999	1.56342416035505e-07\\
1.22799999999999	1.03781623560875e-07\\
1.22999999999997	2.02344900277826e-07\\
1.23	3.00987506825638e-07\\
1.23399999999997	9.86426071199605e-08\\
1.236	1.92418905909337e-07\\
1.23600000000001	1.92418905909314e-07\\
1.23999999999999	1.86180927820855e-07\\
1.24	2.77023768964674e-07\\
1.24399999999997	9.08428417061526e-08\\
1.24599999999999	4.48624442278365e-08\\
1.246	4.48624442278306e-08\\
1.247	4.44786702512792e-08\\
1.24700000000001	8.85779255101042e-08\\
1.24800000000001	1.31176775008823e-07\\
1.24900000000001	1.72688772257727e-07\\
1.25100000000001	8.56112530601992e-08\\
1.253	1.66957398648158e-07\\
1.25300000000001	2.88572078422353e-07\\
1.25700000000001	1.21614680312569e-07\\
1.25999999999999	1.57552394181382e-07\\
1.26	2.34426576424856e-07\\
1.264	7.68741827645453e-08\\
1.266	1.49922415350208e-07\\
1.26600000000001	2.22986273702618e-07\\
1.27000000000001	7.30638588564547e-08\\
1.272	1.42476253381134e-07\\
1.27200000000001	1.42476253381117e-07\\
1.276	1.37793056577384e-07\\
1.27600000000001	1.37793056577368e-07\\
1.27999999999999	1.33325968688934e-07\\
1.28000000000001	1.98379404312229e-07\\
1.28399999999999	6.50534360837272e-08\\
1.28599999999999	6.39780322479258e-08\\
1.28600000000001	6.39780322479186e-08\\
1.288	6.28911947223767e-08\\
1.28800000000001	1.2472020822138e-07\\
1.29	1.5263385908015e-07\\
1.29199999999999	9.08048455744143e-08\\
1.29499999999999	1.1757834938066e-07\\
1.295	1.46368156492714e-07\\
1.29899999999998	2.87898075221162e-08\\
1.29999999999999	1.1282476596524e-07\\
1.3	1.40461164209015e-07\\
1.30399999999998	2.76363986370425e-08\\
1.30499999999999	2.74139319059244e-08\\
1.305	2.74139319059214e-08\\
1.30599999999999	2.71864268745895e-08\\
1.306	5.41402879389705e-08\\
1.30699999999999	8.01744318055743e-08\\
1.30700000000001	1.32496150297992e-07\\
1.308	1.05542289230263e-07\\
1.30899999999999	1.04648565388095e-07\\
1.31099999999998	5.18800069601914e-08\\
1.31300000000001	2.56131661294244e-08\\
1.31300000000002	5.1011794363374e-08\\
1.31699999999999	4.44412776435914e-16\\
1.31999999999999	4.33506803723541e-16\\
1.32	4.33506803723541e-16\\
1.32399999999997	4.18091227599986e-16\\
1.32599999999999	4.10543740523477e-16\\
1.326	4.10543317007003e-16\\
1.32999999999997	3.95179409893231e-16\\
1.33	3.94911747481899e-16\\
1.33399999999997	3.78903248295266e-16\\
1.334	3.78874025658586e-16\\
1.33799999999997	3.62188323630623e-16\\
1.34	3.53596021413676e-16\\
1.34000000000001	3.53596021413676e-16\\
1.34399999999999	3.35841787322752e-16\\
1.346	3.26698913690089e-16\\
1.34600000000001	3.26698913690089e-16\\
1.348	3.17500982915855e-16\\
1.34800000000001	3.17500559399381e-16\\
1.35	3.08325498514722e-16\\
1.35199999999999	2.99170342904324e-16\\
1.35599999999996	2.80904500913259e-16\\
1.35999999999999	2.62459087937376e-16\\
1.36	2.62458664420902e-16\\
1.36299999999999	2.4875282430138e-16\\
1.363	2.4875282430138e-16\\
1.36499999999999	2.3961376231698e-16\\
1.365	2.3961376231698e-16\\
1.36599999999999	2.35041901984175e-16\\
1.366	2.35041901984175e-16\\
1.36699999999999	2.30493335057419e-16\\
1.36799999999999	2.25991778459236e-16\\
1.36999999999997	2.17130543281535e-16\\
1.37199999999999	2.08453961286336e-16\\
1.372	2.0845438480281e-16\\
1.37599999999997	1.9164289838218e-16\\
1.378	1.83501641209645e-16\\
1.37800000000001	1.83501641209645e-16\\
1.37999999999999	1.75532119967166e-16\\
1.38	1.75532119967166e-16\\
1.38199999999997	1.67731370039428e-16\\
1.38399999999994	1.60096426811114e-16\\
1.38599999999999	1.52623902150437e-16\\
1.386	1.52623902150437e-16\\
1.38999999999994	1.38327680066679e-16\\
1.39199999999999	1.31541675609751e-16\\
1.392	1.31541675609751e-16\\
1.39599999999994	1.18679903822168e-16\\
1.39799999999999	1.12599266052066e-16\\
1.398	1.12598842535593e-16\\
1.39999999999999	1.0674817421067e-16\\
1.4	1.06748385968907e-16\\
1.40199999999999	1.01125781265033e-16\\
1.40399999999997	9.57291225998393e-17\\
1.40599999999999	9.05558688744836e-17\\
1.406	9.05558688744836e-17\\
1.40999999999997	8.08658119578944e-17\\
1.412	7.63435030525037e-17\\
1.41200000000001	7.63437148107405e-17\\
1.41599999999999	6.79358540180575e-17\\
1.41999999999996	6.0365708810209e-17\\
1.41999999999998	6.03659205684458e-17\\
1.42	6.0365708810209e-17\\
1.42099999999999	5.8602821488736e-17\\
1.421	5.8602821488736e-17\\
1.42199999999999	5.68911796605719e-17\\
1.42299999999999	5.52309950839535e-17\\
1.42499999999997	5.20637271359328e-17\\
1.42599999999999	5.05564320062938e-17\\
1.426	5.05564320062938e-17\\
1.42999999999997	4.49386977418665e-17\\
1.43199999999999	4.23609647251348e-17\\
1.432	4.23609647251348e-17\\
1.43499999999999	3.87805564570909e-17\\
1.435	3.87803446988541e-17\\
1.43799999999998	3.55404436756064e-17\\
1.43999999999999	3.35679156996879e-17\\
1.44	3.35679156996879e-17\\
1.44299999999999	3.08891740039962e-17\\
1.44599999999997	2.8543739773049e-17\\
1.44599999999999	2.8543739773049e-17\\
1.446	2.8543739773049e-17\\
1.44699999999999	2.78288439655664e-17\\
1.447	2.78286322073296e-17\\
1.44799999999999	2.71368180476596e-17\\
1.44899999999999	2.64678737775656e-17\\
1.44999999999999	2.58215876388105e-17\\
1.45	2.58215876388105e-17\\
1.45199999999999	2.45976250300281e-17\\
1.45399999999997	2.34644008257202e-17\\
1.45599999999999	2.24214915094134e-17\\
1.456	2.24214915094134e-17\\
1.45999999999997	2.06048175957897e-17\\
1.46	2.06048175957897e-17\\
1.46000000000001	2.06048175957897e-17\\
1.46399999999999	1.91449563111969e-17\\
1.466	1.8547903962501e-17\\
1.46600000000001	1.8547903962501e-17\\
1.46999999999999	1.75291350851909e-17\\
1.47	1.75291350851909e-17\\
1.47399999999997	1.66830550500023e-17\\
1.47599999999999	1.63243365968401e-17\\
1.476	1.63244424759585e-17\\
1.47899999999998	1.58664094097307e-17\\
1.479	1.58664094097307e-17\\
1.47999999999999	1.57348016655511e-17\\
1.48	1.57350134237879e-17\\
1.48099999999999	1.56143112288042e-17\\
1.48199999999999	1.55041969456611e-17\\
1.48399999999997	1.5315732114897e-17\\
1.48599999999999	1.51694071732589e-17\\
1.486	1.51695130523773e-17\\
1.48999999999997	1.49458963543021e-17\\
1.491	1.48983566301375e-17\\
1.49100000000001	1.48983566301375e-17\\
1.49499999999999	1.4741020260185e-17\\
1.49899999999996	1.46367293285543e-17\\
1.49999999999999	1.4618941636662e-17\\
1.5	1.4618941636662e-17\\
1.50499999999999	1.45789193299042e-17\\
1.505	1.45789193299042e-17\\
1.50599999999999	1.45807192749171e-17\\
1.506	1.45807192749171e-17\\
1.50699999999999	1.45841074067061e-17\\
1.50799999999999	1.45870720220215e-17\\
1.508	1.45869661429031e-17\\
1.50999999999999	1.45918365823498e-17\\
1.51199999999997	1.45952247141388e-17\\
1.51399999999999	1.45970246591518e-17\\
1.514	1.45971305382702e-17\\
1.51799999999997	1.45966011426781e-17\\
1.518	1.45966011426781e-17\\
1.51999999999999	1.45941659229548e-17\\
1.52	1.45942718020732e-17\\
1.52199999999999	1.45903542746921e-17\\
1.52399999999997	1.45850603187718e-17\\
1.52599999999999	1.45782840551938e-17\\
1.526	1.45782840551938e-17\\
1.52999999999997	1.45510731217632e-17\\
1.53399999999994	1.44992982328623e-17\\
1.537	1.44441881517316e-17\\
1.53700000000001	1.44442410912908e-17\\
1.53999999999999	1.43752079060895e-17\\
1.54	1.43752079060895e-17\\
1.54299999999997	1.4292304556377e-17\\
1.54599999999994	1.41953192839164e-17\\
1.54599999999997	1.41953192839164e-17\\
1.546	1.41953192839164e-17\\
1.549	1.40850461820957e-17\\
1.54900000000001	1.40850991216549e-17\\
1.55200000000001	1.39623852234215e-17\\
1.55500000000001	1.38270717100976e-17\\
1.55500000000003	1.38270717100976e-17\\
1.55999999999999	1.35733324028357e-17\\
1.56	1.35733853423949e-17\\
1.56499999999996	1.32838059535524e-17\\
1.56599999999998	1.322149609237e-17\\
1.566	1.322149609237e-17\\
1.57099999999996	1.28970824735716e-17\\
1.57199999999998	1.28298492333833e-17\\
1.572	1.28298492333833e-17\\
1.57499999999999	1.26237025898453e-17\\
1.575	1.26237025898453e-17\\
1.57799999999999	1.24104091058148e-17\\
1.57999999999999	1.22641371037358e-17\\
1.58	1.22640841641766e-17\\
1.58299999999999	1.20384557628517e-17\\
1.58599999999997	1.18050981858832e-17\\
1.586	1.18050981858832e-17\\
1.58699999999999	1.17260064844333e-17\\
1.587	1.17259535448741e-17\\
1.58799999999999	1.1646808903865e-17\\
1.58899999999999	1.15674525046191e-17\\
1.59099999999997	1.14084220687722e-17\\
1.59499999999995	1.10887200707429e-17\\
1.59499999999997	1.10887200707429e-17\\
1.595	1.10886671311837e-17\\
1.59999999999999	1.0685320629613e-17\\
1.6	1.06852147504946e-17\\
1.60499999999999	1.0276680172122e-17\\
1.60599999999999	1.01942532784423e-17\\
1.606	1.01943062180015e-17\\
1.60699999999998	1.0111932263881e-17\\
1.607	1.0111932263881e-17\\
1.60799999999999	1.00298759471158e-17\\
1.60899999999999	9.94819020726494e-18\\
1.60999999999998	9.86682210476933e-18\\
1.61	9.86682210476933e-18\\
1.61199999999999	9.70503881184376e-18\\
1.61399999999997	9.54447312877986e-18\\
1.61599999999999	9.38507211601844e-18\\
1.616	9.38507211601844e-18\\
1.61999999999997	9.06952587338662e-18\\
1.61999999999999	9.06952587338662e-18\\
1.62	9.06952587338662e-18\\
1.62399999999997	8.75776480923783e-18\\
1.624	8.75776480923783e-18\\
1.62599999999999	8.60320776614352e-18\\
1.626	8.60320776614352e-18\\
1.62799999999999	8.45015950048651e-18\\
1.62999999999998	8.29938763587524e-18\\
1.63199999999999	8.15075982341172e-18\\
1.632	8.15078629319132e-18\\
1.63599999999998	7.85985694558907e-18\\
1.63999999999995	7.57697441098573e-18\\
1.63999999999998	7.57700088076534e-18\\
1.64	7.57700088076534e-18\\
1.645	7.23403194646615e-18\\
1.64500000000001	7.23400547668655e-18\\
1.64599999999999	7.16677223649824e-18\\
1.646	7.16677223649824e-18\\
1.64699999999999	7.10004192212236e-18\\
1.64799999999999	7.03392041267732e-18\\
1.64999999999997	6.90358321791856e-18\\
1.65199999999999	6.77560183354436e-18\\
1.652	6.77560183354436e-18\\
1.65299999999998	6.71252434875352e-18\\
1.653	6.71252434875352e-18\\
1.65399999999999	6.65002919911391e-18\\
1.65499999999997	6.58808991484594e-18\\
1.65699999999995	6.46590541220451e-18\\
1.65899999999998	6.34594437104962e-18\\
1.659	6.34594437104962e-18\\
1.65999999999999	6.28678441363982e-18\\
1.66	6.28675794386022e-18\\
1.66099999999999	6.22815385182207e-18\\
1.66199999999998	6.17007915537594e-18\\
1.66399999999995	6.0555179492598e-18\\
1.66599999999999	5.94299491617298e-18\\
1.666	5.94299491617298e-18\\
1.66999999999995	5.72451335534058e-18\\
1.67399999999991	5.51492564045434e-18\\
1.67999999999998	5.2163729963268e-18\\
1.68	5.2163729963268e-18\\
1.68199999999998	5.12092297108308e-18\\
1.682	5.12094944086269e-18\\
1.68399999999998	5.02748464908909e-18\\
1.68599999999997	4.93595215122643e-18\\
1.68599999999999	4.93592568144683e-18\\
1.686	4.93595215122643e-18\\
1.68999999999997	4.75860462789506e-18\\
1.69199999999999	4.67268372330795e-18\\
1.692	4.67268372330795e-18\\
1.69599999999997	4.50629468873168e-18\\
1.69799999999999	4.42572067962412e-18\\
1.698	4.42572067962412e-18\\
1.69999999999999	4.34689367597026e-18\\
1.7	4.34689367597026e-18\\
1.70199999999999	4.26968132887211e-18\\
1.70399999999998	4.19418951744807e-18\\
1.70599999999999	4.12032559746954e-18\\
1.706	4.12032559746954e-18\\
1.70999999999998	3.97713732471416e-18\\
1.71099999999998	3.94225015519912e-18\\
1.711	3.94225015519912e-18\\
1.71499999999997	3.80630136716481e-18\\
1.715	3.80630136716481e-18\\
1.71699999999998	3.74040485084638e-18\\
1.717	3.74043132062599e-18\\
1.71899999999998	3.67588476306725e-18\\
1.71999999999999	3.64413426243501e-18\\
1.72	3.64413426243501e-18\\
1.72199999999999	3.5816126430158e-18\\
1.72399999999997	3.52037480790728e-18\\
1.72599999999999	3.46042075710944e-18\\
1.726	3.46042075710944e-18\\
1.72899999999998	3.37275284706862e-18\\
1.729	3.37273961217881e-18\\
1.73199999999998	3.28758633120016e-18\\
1.73499999999997	3.20490797461426e-18\\
1.73999999999998	3.07240025792816e-18\\
1.74	3.07240025792816e-18\\
1.74599999999997	2.9217078026557e-18\\
1.74599999999998	2.9217210375455e-18\\
1.746	2.9217210375455e-18\\
1.74999999999998	2.82594014005676e-18\\
1.75	2.82595337494656e-18\\
1.75199999999998	2.77935332795777e-18\\
1.752	2.77935332795777e-18\\
1.75399999999998	2.73361354880604e-18\\
1.75599999999997	2.68869433282196e-18\\
1.75799999999998	2.64458244511573e-18\\
1.758	2.64459568000554e-18\\
1.75999999999999	2.60126465079756e-18\\
1.76	2.60127788568736e-18\\
1.76199999999999	2.55874094986743e-18\\
1.76399999999998	2.51694516787635e-18\\
1.76599999999999	2.47591700949372e-18\\
1.766	2.47591700949372e-18\\
1.76899999999998	2.41567179112026e-18\\
1.769	2.41567179112026e-18\\
1.77199999999998	2.3569882897433e-18\\
1.77499999999996	2.29978709602403e-18\\
1.77499999999998	2.29977386113423e-18\\
1.775	2.29977386113423e-18\\
1.78	2.20761932345092e-18\\
1.78000000000002	2.20763255834072e-18\\
1.78499999999998	2.11923672936086e-18\\
1.785	2.11923672936086e-18\\
1.78600000000001	2.10199828539525e-18\\
1.78600000000003	2.10198505050545e-18\\
1.78700000000005	2.08489219032766e-18\\
1.78800000000006	2.06793829649277e-18\\
1.79000000000009	2.03444079040682e-18\\
1.79200000000003	2.00149253224761e-18\\
1.79200000000004	2.00149253224761e-18\\
1.79600000000011	1.93721067248489e-18\\
1.79799999999998	1.90583074876708e-18\\
1.798	1.90583074876708e-18\\
1.8	1.8749537508617e-18\\
1.80000000000002	1.8749669857515e-18\\
1.80200000000002	1.84457306132385e-18\\
1.80400000000002	1.81466221037393e-18\\
1.806	1.78520796312214e-18\\
1.80600000000002	1.78520796312214e-18\\
1.80999999999998	1.72771560182726e-18\\
1.81	1.72770898438236e-18\\
1.81399999999997	1.67204965532489e-18\\
1.81799999999994	1.61814394916603e-18\\
1.82	1.59182637079705e-18\\
1.82000000000001	1.59181975335215e-18\\
1.826	1.51525591585424e-18\\
1.82600000000001	1.51525591585424e-18\\
1.827	1.50283497177614e-18\\
1.82700000000001	1.50283497177614e-18\\
1.828	1.49051990681645e-18\\
1.82899999999999	1.47829748608537e-18\\
1.83099999999996	1.45416366453352e-18\\
1.83200000000001	3.36403754794391e-10\\
1.83200000000003	3.36403754794351e-10\\
1.83599999999998	3.25291813965517e-10\\
1.83800000000001	1.36816996805251e-18\\
1.83800000000003	1.36816996805251e-18\\
1.83999999999999	1.35019698770296e-18\\
1.84	1.35019698770296e-18\\
1.84199999999996	1.33242253070042e-18\\
1.84399999999992	1.31484659704489e-18\\
1.84599999999999	1.28483648442147e-18\\
1.846	1.28483648442147e-18\\
1.84999999999992	1.2510941328742e-18\\
1.85399999999984	1.21815910960479e-18\\
1.85499999999998	1.19249665828095e-18\\
1.855	1.19249665828095e-18\\
1.85599999999998	1.18270283982832e-18\\
1.856	1.18271607471812e-18\\
1.85699999999999	1.17294210860019e-18\\
1.85799999999997	1.16324755182107e-18\\
1.85999999999995	1.14409004883434e-18\\
1.85999999999998	1.14403710927514e-18\\
1.86	1.14403049183024e-18\\
1.86399999999995	1.10661545836324e-18\\
1.86599999999999	1.08833807554827e-18\\
1.866	1.08834469299317e-18\\
1.86799999999998	1.07040479986812e-18\\
1.868	1.07041141731302e-18\\
1.86999999999998	1.05277592665339e-18\\
1.87199999999996	1.03543160356938e-18\\
1.87299999999998	1.02668003268857e-18\\
1.873	1.02668003268857e-18\\
1.87699999999996	9.93122969598516e-19\\
1.87999999999999	9.68479604789336e-19\\
1.88	9.68482913511786e-19\\
1.88399999999997	9.36762191381603e-19\\
1.88499999999998	9.28821257501094e-19\\
1.885	9.28824566223544e-19\\
1.886	9.18120849097108e-19\\
1.88600000000002	9.18120849097108e-19\\
1.88700000000002	9.07532937256429e-19\\
1.88800000000002	8.97050904534157e-19\\
1.88999999999998	8.85175899660346e-19\\
1.89	8.86519240975132e-19\\
1.892	-4.12937206245817e-10\\
1.89200000000002	1.97110853398731e-18\\
1.89400000000002	-4.06005118710392e-10\\
1.89600000000003	-3.99232207092912e-10\\
1.898	-3.92615815984957e-10\\
1.89800000000002	1.87359386593466e-18\\
1.9	-3.8615334929769e-10\\
1.90000000000002	1.84292200882119e-18\\
1.902	-3.79842273491206e-10\\
1.90399999999998	-3.73680116353532e-10\\
1.906	-3.67664461942791e-10\\
1.90600000000002	1.75290490584072e-18\\
1.90999999999998	-7.17030634757582e-10\\
1.91399999999995	-6.9301565769336e-10\\
1.91399999999997	1.61509330706694e-18\\
1.914	1.61531830019355e-18\\
1.91999999999998	-9.96863126830583e-10\\
1.92	1.48567924587179e-18\\
1.92499999999998	-7.93848661070687e-10\\
1.925	1.35686406344013e-18\\
1.92599999999998	-1.54936158628327e-10\\
1.926	1.3290079291318e-18\\
1.92699999999999	-1.53650361648972e-10\\
1.92799999999997	-1.52335965408897e-10\\
1.92999999999995	-3.00788338019038e-10\\
1.93199999999998	-2.95708272907363e-10\\
1.932	1.1565937109737e-18\\
1.93599999999995	-5.76638132503809e-10\\
1.9399999999999	-5.5768402904684e-10\\
1.93999999999999	9.06295475060049e-19\\
1.94	9.1612899818208e-19\\
1.94299999999998	-4.06351649139419e-10\\
1.943	8.22564944734982e-19\\
1.94599999999998	-3.96540976662915e-10\\
1.946	7.261289202011e-19\\
1.94899999999998	-3.86747597539456e-10\\
1.95199999999997	-3.76962848422434e-10\\
1.95199999999998	5.39091803174284e-19\\
1.952	5.39062024672232e-19\\
1.95799999999997	-7.25893264502657e-10\\
1.95999999999998	-2.34007083435913e-10\\
1.96	3.06058481005845e-19\\
1.96599999999996	-6.79433860910682e-10\\
1.96599999999998	1.42240323773394e-19\\
1.966	1.42215508355017e-19\\
1.97199999999996	-6.46276644633615e-10\\
1.97199999999998	-7.26430013944069e-21\\
1.972	-7.26099141699048e-21\\
1.97799999999996	-6.14274513402104e-10\\
1.97799999999998	-1.38893551015004e-19\\
1.978	-1.38870389957852e-19\\
1.98	-1.98024412741183e-10\\
1.98000000000002	-1.78500613105268e-19\\
1.98200000000002	-1.9478800857303e-10\\
1.98400000000002	-1.91627975069041e-10\\
1.986	-1.8854307004188e-10\\
1.98600000000002	-2.86896014935442e-19\\
1.99000000000002	-3.67702540631715e-10\\
1.99400000000003	-3.55387351127637e-10\\
1.995	-8.69883228158737e-11\\
1.99500000000001	-4.17398645816708e-19\\
1.99999999999999	-4.24215930881063e-10\\
2	-4.73466602096777e-19\\
2.00099999999997	-8.27562576384512e-11\\
2.001	-4.84459832437607e-19\\
2.00199999999999	-8.20796978376681e-11\\
2.00299999999997	-8.14111834174536e-11\\
2.00499999999995	-1.60848674894713e-10\\
2.00599999999997	-7.94532571255869e-11\\
2.006	-5.26624536981885e-19\\
2.00999999999995	-3.11161860340601e-10\\
2.01199999999997	-1.51643009393202e-10\\
2.012	-5.64221550183645e-19\\
2.01299999999998	-7.48632017810837e-11\\
2.01300000000001	-5.69300439144721e-19\\
2.014	-7.42341366126713e-11\\
2.01499999999999	-7.3612347441164e-11\\
2.01699999999996	-1.45388125653602e-10\\
2.01999999999997	-2.13597195988427e-10\\
2.02	-5.95444309585072e-19\\
2.02399999999995	-2.7671617097167e-10\\
2.02599999999997	-1.35017490774946e-10\\
2.026	-6.04793104868146e-19\\
2.02999999999995	-2.63315296067987e-10\\
2.03	-6.06540110321858e-19\\
2.03399999999995	-2.54496271441149e-10\\
2.03599999999997	-1.24066256836133e-10\\
2.036	-6.00922726782011e-19\\
2.03999999999995	-2.420119740405e-10\\
2.04	-5.95518755840202e-19\\
2.04399999999995	-2.34166232721265e-10\\
2.04599999999997	-1.14256196438415e-10\\
2.046	-5.78152098879651e-19\\
2.04799999999997	-1.12367418207757e-10\\
2.048	-5.71558642216991e-19\\
2.04999999999996	-1.10458558920991e-10\\
2.05199999999993	-1.08593005350359e-10\\
2.05599999999987	-2.11758930807599e-10\\
2.05899999999997	-1.5423361935451e-10\\
2.059	-5.30273230663856e-19\\
2.05999999999997	-5.05648041649596e-11\\
2.06	-5.26088523944951e-19\\
2.06099999999999	-5.01498950844819e-11\\
2.06199999999998	-4.97399035248369e-11\\
2.06399999999995	-9.82692922643021e-11\\
2.06499999999997	-4.85390208450776e-11\\
2.065	-5.04052432426538e-19\\
2.06599999999997	-4.81482930493138e-11\\
2.066	-4.99419393815629e-19\\
2.06699999999999	-4.77487162532553e-11\\
2.06799999999998	-4.73402516476549e-11\\
2.06999999999995	-9.34736287632321e-11\\
2.07099999999997	-4.61425391600595e-11\\
2.071	-4.76125160585523e-19\\
2.07499999999995	-1.8071339877332e-10\\
2.07699999999997	-8.81045075128372e-11\\
2.077	-4.48283915528213e-19\\
2.07999999999997	-1.29438876931741e-10\\
2.08	-4.34379836611809e-19\\
2.08299999999998	-1.26278701753848e-10\\
2.08599999999995	-1.23229923588988e-10\\
2.086	-4.07318622872137e-19\\
2.08799999999997	-8.04673814824732e-11\\
2.088	-3.97330416975559e-19\\
2.08999999999997	-7.91004277961736e-11\\
2.09199999999993	-7.77644870727562e-11\\
2.09399999999997	-7.64590342169109e-11\\
2.094	-3.70627372441122e-19\\
2.09799999999993	-1.4912111183673e-10\\
2.09999999999997	-7.27205415029016e-11\\
2.1	-3.45261878957184e-19\\
2.10399999999993	-1.41903614737063e-10\\
2.10599999999997	-6.92387074436094e-11\\
2.106	-3.21142946656362e-19\\
2.10999999999993	-1.35031473695218e-10\\
2.112	-6.58068409550555e-11\\
2.11200000000003	-2.98483160956085e-19\\
2.11599999999996	-1.28324897167297e-10\\
2.11699999999997	-3.14144417353698e-11\\
2.117	-2.80890683688307e-19\\
2.11999999999997	-9.26924147475475e-11\\
2.12	-2.70878489553965e-19\\
2.12299999999998	-9.04293831861064e-11\\
2.12599999999995	-8.82461240890036e-11\\
2.126	-2.52614755219101e-19\\
2.12899999999997	-8.60667092727105e-11\\
2.129	-2.43189445729365e-19\\
2.13199999999996	-8.3889212573288e-11\\
2.13499999999993	-8.17857200652025e-11\\
2.13499999999997	-2.26782731869682e-19\\
2.135	-2.26783559050295e-19\\
2.13999999999997	-1.31830248488974e-10\\
2.14	-2.13973426493992e-19\\
2.14499999999998	-1.2650996720415e-10\\
2.14599999999997	-2.46910646970314e-11\\
2.146	-2.00077619383714e-19\\
2.14699999999997	-2.4486156661234e-11\\
2.14699999999999	9.5873657986406e-11\\
2.14799999999998	9.5059386446415e-11\\
2.14899999999997	7.09897826374119e-11\\
2.15099999999995	4.67270674787769e-11\\
2.15299999999999	-1.17793414359695e-19\\
2.15300000000002	2.30691592041505e-11\\
2.15699999999997	-4.51811256933329e-11\\
2.15999999999997	-6.63779227259979e-11\\
2.16	-1.06946181396981e-19\\
2.16399999999995	-8.59929108619416e-11\\
2.16599999999997	-4.19583250433836e-11\\
2.166	-9.99374800668191e-20\\
2.16999999999995	-8.18284263303184e-11\\
2.17	-9.64509137849081e-20\\
2.17399999999995	-7.90878073382663e-11\\
2.17499999999997	-1.9358357605729e-11\\
2.175	-9.27956026580363e-20\\
2.17899999999995	-7.58359769996876e-11\\
2.17999999999997	-1.85689219771346e-11\\
2.18	-9.11945945824399e-20\\
2.18399999999995	-7.27700009514384e-11\\
2.18599999999997	-3.55065007419419e-11\\
2.186	-9.14977562769406e-20\\
2.187	-1.75347703157146e-11\\
2.18700000000002	-9.1764522024489e-20\\
2.18800000000002	-1.73847697165944e-11\\
2.18800000000005	-9.20594119128641e-20\\
2.18900000000004	-1.7236473071413e-11\\
2.19000000000004	-1.7089865779885e-11\\
2.19200000000003	-3.37465953260862e-11\\
2.19600000000001	-6.58066597288318e-11\\
2.2	-6.36435967253317e-11\\
2.20000000000003	-9.7817829747152e-20\\
2.20399999999997	-6.15803466838921e-11\\
2.204	-1.00657333994893e-19\\
2.20499999999997	-1.50840902023755e-11\\
2.205	-1.01440260444675e-19\\
2.20599999999999	-1.49626667406497e-11\\
2.20600000000003	-1.02456658622349e-19\\
2.20700000000002	-1.48384934546957e-11\\
2.20800000000001	-1.47115581180762e-11\\
2.20999999999998	-2.90480653650654e-11\\
2.21200000000003	-2.85574676662715e-11\\
2.21200000000006	-1.07065088610035e-19\\
2.21600000000001	-5.56877377463829e-11\\
2.21800000000003	-2.71521027859243e-11\\
2.21800000000006	-1.11600726703817e-19\\
2.21999999999997	-2.67051784550905e-11\\
2.22	-1.13053876244919e-19\\
2.22199999999992	-2.62687238531513e-11\\
2.22399999999983	-2.58425680084351e-11\\
2.226	-2.54265438445052e-11\\
2.22600000000003	-1.17243959637807e-19\\
2.22999999999986	-4.95876339999642e-11\\
2.23299999999997	-3.60966701709579e-11\\
2.233	-1.21402403372264e-19\\
2.23699999999983	-4.67306964105415e-11\\
2.23899999999997	-2.27867756515465e-11\\
2.239	-1.24044625043911e-19\\
2.24	-1.12526776892452e-11\\
2.24000000000003	-1.24403207839453e-19\\
2.24100000000003	-1.11603444000897e-11\\
2.24200000000003	-1.10691049828092e-11\\
2.24400000000004	-2.186882230216e-11\\
2.246	-2.15167691007643e-11\\
2.24600000000003	-1.26065220485225e-19\\
2.25000000000004	-4.19626700782182e-11\\
2.252	-2.04502748928751e-11\\
2.25200000000003	-1.26879579798284e-19\\
2.25600000000004	-3.98785199374648e-11\\
2.25999999999997	-3.85677123970366e-11\\
2.26	-1.26634113951509e-19\\
2.26199999999997	-1.88112836748152e-11\\
2.262	-1.26334881364918e-19\\
2.26399999999996	-1.85061093590453e-11\\
2.26599999999993	-1.82081904366237e-11\\
2.26599999999997	-1.25514111402112e-19\\
2.266	-1.25509975499049e-19\\
2.26999999999994	-3.55101775014796e-11\\
2.272	-1.73056883844537e-11\\
2.27200000000003	-1.23543973978164e-19\\
2.27499999999997	-2.54159055134612e-11\\
2.275	-1.22371652255023e-19\\
2.27799999999994	-2.47846403240397e-11\\
2.27999999999997	-1.61832099947144e-11\\
2.28	-1.2008884055953e-19\\
2.28299999999994	-2.37808516789812e-11\\
2.28599999999989	-2.32067046563336e-11\\
2.28599999999997	-1.1725109407059e-19\\
2.286	-1.16799763648866e-19\\
2.28699999999997	-7.6093705651293e-12\\
2.287	-1.16197265970197e-19\\
2.28799999999999	-7.54427636431311e-12\\
2.28899999999997	-7.47992170885228e-12\\
2.29099999999995	-1.47697055420167e-11\\
2.291	-1.13846315271745e-19\\
2.29499999999995	-2.87989981432332e-11\\
2.29699999999997	-1.40405834337875e-11\\
2.297	-1.09772037164615e-19\\
2.29999999999997	-2.06277454088011e-11\\
2.3	-1.07695917224684e-19\\
2.30299999999998	-2.01241309690754e-11\\
2.30599999999995	-1.96382690722212e-11\\
2.306	-1.03444932659197e-19\\
2.30999999999997	-2.54291771604481e-11\\
2.31	-1.00314984618873e-19\\
2.31399999999997	-2.45774963622142e-11\\
2.31599999999997	-1.19814642673259e-11\\
2.316	-9.57869979457578e-20\\
2.31999999999997	-2.33718489629884e-11\\
2.32	-9.27552776031743e-20\\
2.32399999999997	-2.26141613416991e-11\\
2.32599999999997	-1.10340762453628e-11\\
2.326	-8.81759023345042e-20\\
2.32999999999997	-2.15189975249185e-11\\
2.33199999999997	-1.04871642986947e-11\\
2.332	-8.36465714928935e-20\\
2.33599999999997	-2.04502185260658e-11\\
2.33999999999994	-1.97780195612837e-11\\
2.34	-7.79938259818593e-20\\
2.34000000000003	-7.78188772823043e-20\\
2.34499999999997	-2.38244044117119e-11\\
2.345	-7.42844379224418e-20\\
2.34600000000003	-4.64983044789674e-12\\
2.34600000000006	-7.35865042806002e-20\\
2.34700000000009	-4.61124205093114e-12\\
2.34800000000012	-4.57179535379523e-12\\
2.34899999999997	-4.5327967400948e-12\\
2.349	-7.15237226280461e-20\\
2.35100000000006	-8.9503708348082e-12\\
2.35300000000011	-8.79965865144667e-12\\
2.35499999999997	-8.65239640708342e-12\\
2.355	-6.75525319047556e-20\\
2.35800000000003	-1.27098827431055e-11\\
2.35800000000006	-6.56409175091455e-20\\
2.35999999999997	-8.29895849250798e-12\\
2.36	-6.4392701964803e-20\\
2.36199999999992	-8.1633249465166e-12\\
2.36399999999983	-8.03089184542129e-12\\
2.36599999999997	-7.90160728708476e-12\\
2.366	-6.07684101109019e-20\\
2.36999999999983	-1.54099594553925e-11\\
2.37399999999966	-1.48938450639899e-11\\
2.37799999999997	-1.44010902022126e-11\\
2.378	-5.41145175647371e-20\\
2.37999999999997	-7.02284927170601e-12\\
2.38	-5.3083850521496e-20\\
2.38199999999997	-6.90807131338002e-12\\
2.38399999999995	-6.79600210921361e-12\\
2.38599999999997	-6.6865973038412e-12\\
2.386	-5.01176842424574e-20\\
2.38999999999995	-1.30404093178859e-11\\
2.39	-4.83013507136558e-20\\
2.39399999999995	-1.26036565063182e-11\\
2.396	-6.14424909476756e-12\\
2.39600000000002	-4.55829250280769e-20\\
2.39999999999997	-1.19853849590449e-11\\
2.4	-4.39066952155264e-20\\
2.40399999999995	-1.15968329660496e-11\\
2.40599999999997	-5.65841628731682e-12\\
2.406	-4.15313943877921e-20\\
2.40699999999997	-2.79439055591817e-12\\
2.407	-4.11511497999592e-20\\
2.40799999999998	-2.77048603567906e-12\\
2.40899999999997	-2.74685306229216e-12\\
2.41099999999995	-5.42388175524633e-12\\
2.41299999999997	-5.33255090695671e-12\\
2.413	-3.8955553960302e-20\\
2.41499999999997	-5.24331067674105e-12\\
2.415	-3.82556557645056e-20\\
2.41699999999997	-5.15612607919868e-12\\
2.41899999999995	-5.07096296133844e-12\\
2.41999999999997	-2.50416789612413e-12\\
2.42	-3.65733771937258e-20\\
2.42399999999995	-9.81361755475535e-12\\
2.426	-4.78833605704782e-12\\
2.42600000000003	-3.46771173382363e-20\\
2.427	-2.36470425283179e-12\\
2.42700000000003	-3.43733869570645e-20\\
2.42800000000002	-2.34447547012448e-12\\
2.429	-2.32447647921114e-12\\
2.43099999999998	-4.5898653059648e-12\\
2.43499999999993	-8.94963829951105e-12\\
2.43599999999997	-2.1907557062154e-12\\
2.436	-3.17792968573099e-20\\
2.43999999999997	-8.58284881375953e-12\\
2.44	-3.07033933743199e-20\\
2.44399999999998	-8.30460264818717e-12\\
2.446	-4.05204585447743e-12\\
2.44600000000003	-2.9173522831403e-20\\
2.448	-3.98506116294896e-12\\
2.44800000000002	-2.86845556918076e-20\\
2.44999999999999	-3.91736430496275e-12\\
2.45000000000002	-2.820499773168e-20\\
2.45199999999998	-3.85120326520457e-12\\
2.45399999999995	-3.7865520943379e-12\\
2.45600000000002	-3.72338545590597e-12\\
2.45600000000005	-2.68212279644554e-20\\
2.45999999999998	-7.2630857578931e-12\\
2.46000000000001	-2.59424261118002e-20\\
2.46399999999994	-7.02762512595326e-12\\
2.46499999999997	-1.72141506932654e-12\\
2.465	-2.48909244568804e-20\\
2.46599999999998	-1.70755807920612e-12\\
2.46600000000001	-2.46866625443681e-20\\
2.46699999999999	-1.69338726343626e-12\\
2.46799999999998	-1.67890124568349e-12\\
2.46999999999996	-3.3150012095471e-12\\
2.47199999999998	-3.25901359340393e-12\\
2.47200000000001	-2.34991413774717e-20\\
2.47599999999996	-6.35515362278848e-12\\
2.47999999999991	-6.14625963512868e-12\\
2.47999999999997	-2.20656632253114e-20\\
2.48	-2.20113019494302e-20\\
2.48499999999997	-7.40372284043446e-12\\
2.485	-2.11330946328403e-20\\
2.48599999999997	-1.44499124742269e-12\\
2.486	-2.09619716436184e-20\\
2.48699999999999	-1.43299943483791e-12\\
2.48799999999998	-1.42074089515048e-12\\
2.48999999999995	-2.80526195253667e-12\\
2.49199999999997	-2.75788340959021e-12\\
2.492	-1.99650122103389e-20\\
2.494	-2.71158609045634e-12\\
2.49400000000002	-1.9643652542362e-20\\
2.49600000000002	-2.66635184485516e-12\\
2.49800000000001	-2.62216295307043e-12\\
2.49999999999997	-2.57900209026679e-12\\
2.5	-1.87106445101905e-20\\
2.50399999999999	-5.03254941790832e-12\\
2.50599999999997	-2.45552037668617e-12\\
2.506	-1.7821787243214e-20\\
2.50999999999999	-4.78883195637112e-12\\
2.51399999999998	-4.62844314854734e-12\\
2.51999999999997	-6.65774899961853e-12\\
2.52	-1.59064763842416e-20\\
2.52299999999997	-3.20703857648121e-12\\
2.523	-1.55227421281994e-20\\
2.52599999999996	-3.12961006097512e-12\\
2.526	-1.51563011168384e-20\\
2.52899999999997	-3.05231802893337e-12\\
2.53199999999993	-2.97509410687044e-12\\
2.53199999999997	-1.44272965032378e-20\\
2.532	-1.44276842441499e-20\\
2.53799999999993	-5.72894855628681e-12\\
2.53799999999997	-1.37369108844857e-20\\
2.538	-1.37364972941794e-20\\
2.53999999999997	-1.84684803686411e-12\\
2.54	-1.35096300864897e-20\\
2.54199999999998	-1.81666414929057e-12\\
2.54399999999995	-1.78719252245596e-12\\
2.54599999999997	-1.75842157133255e-12\\
2.546	-1.28601382092707e-20\\
2.54999999999995	-3.42932825504675e-12\\
2.55199999999997	-1.67126413636961e-12\\
2.552	-1.22401921895564e-20\\
2.55499999999997	-2.45449302098652e-12\\
2.555	-1.19412439463009e-20\\
2.55799999999998	-2.39352977793063e-12\\
2.55800000000001	-1.16493008888579e-20\\
2.55999999999997	-1.56286290634906e-12\\
2.56	-1.14584289625113e-20\\
2.56199999999997	-1.53732034675024e-12\\
2.56399999999994	-1.51238049610698e-12\\
2.56599999999997	-1.48803358007351e-12\\
2.566	-1.09030547293644e-20\\
2.56999999999994	-2.90200921303119e-12\\
2.56999999999997	-1.05504948426578e-20\\
2.57	-1.05504948426578e-20\\
2.57399999999994	-2.80481436310854e-12\\
2.57799999999988	-2.71201846773096e-12\\
2.57999999999997	-1.32254555294319e-12\\
2.58	-9.7060597595148e-21\\
2.58099999999997	-6.5313509226724e-13\\
2.581	-9.62560352024695e-21\\
2.58199999999998	-6.47795472675371e-13\\
2.58299999999997	-6.42519373597737e-13\\
2.58499999999995	-1.26946182397447e-12\\
2.58599999999997	-6.27066882557564e-13\\
2.586	-9.23257640701062e-21\\
2.58999999999995	-2.45577467357339e-12\\
2.59	-8.94052995199112e-21\\
2.59199999999997	-1.19680818945173e-12\\
2.592	-8.78138815695416e-21\\
2.59399999999997	-1.17671704984194e-12\\
2.59599999999995	-1.15708724682026e-12\\
2.59799999999997	-1.13791108431352e-12\\
2.598	-8.35212019313035e-21\\
2.59999999999997	-1.11918103802397e-12\\
2.6	-8.21367083810428e-21\\
2.60199999999997	-1.10088976511206e-12\\
2.60399999999995	-1.08303010039468e-12\\
2.60599999999997	-1.06559504180206e-12\\
2.606	-7.81128624418845e-21\\
2.60999999999995	-2.07815648161874e-12\\
2.61	-7.56510953907443e-21\\
2.61399999999994	-2.00855432273999e-12\\
2.61599999999997	-9.79164902605728e-13\\
2.616	-7.18344969691217e-21\\
2.61999999999994	-1.91002482689121e-12\\
2.61999999999997	-6.9479294045333e-21\\
2.62	-6.9479294045333e-21\\
2.62399999999995	-1.84810408669044e-12\\
2.62499999999997	-4.52692647815827e-13\\
2.625	-6.66140825751171e-21\\
2.62599999999997	-4.49048577559168e-13\\
2.626	-6.60571573783216e-21\\
2.62699999999999	-4.45321978346767e-13\\
2.62799999999998	-4.41512488223589e-13\\
2.62999999999995	-8.71769221781979e-13\\
2.63199999999997	-8.57045764012763e-13\\
2.632	-6.28136400248334e-21\\
2.63599999999995	-1.67125951959486e-12\\
2.63899999999997	-1.217253926361e-12\\
2.639	-5.92329819482445e-21\\
2.63999999999997	-3.99071278317344e-13\\
2.64	-5.87384184148173e-21\\
2.64099999999999	-3.95796718868339e-13\\
2.64199999999998	-3.92560953102384e-13\\
2.64399999999995	-7.75568190046022e-13\\
2.64599999999997	-7.6308277842885e-13\\
2.646	-5.58550477452164e-21\\
2.64999999999995	-1.48818768890454e-12\\
2.65099999999997	-3.64169567794203e-13\\
2.651	-5.35603324037207e-21\\
2.65499999999995	-1.42623967682535e-12\\
2.6589999999999	-1.37920491081794e-12\\
2.65999999999997	-3.37707123381611e-13\\
2.66	-4.96691584799976e-21\\
2.66599999999997	-1.96919048547067e-12\\
2.666	-4.7233305446485e-21\\
2.66799999999997	-6.35070632802611e-13\\
2.668	-4.64478716054718e-21\\
2.66999999999996	-6.24282259688504e-13\\
2.67199999999993	-6.13738649378448e-13\\
2.67199999999996	-4.49402056921231e-21\\
2.672	6.03435648772303e-13\\
2.67599999999993	-5.93369238737916e-13\\
2.67799999999997	-4.26799992918074e-21\\
2.678	-4.26581565537571e-21\\
2.67999999999997	-5.73930448614993e-13\\
2.68	-4.20083027850202e-21\\
2.68199999999998	-5.64550448446068e-13\\
2.68399999999995	-5.55391782517187e-13\\
2.68599999999997	-5.4645086007577e-13\\
2.686	-3.99508848817506e-21\\
2.68999999999995	-1.0657054060171e-12\\
2.69399999999989	-1.03001252226488e-12\\
2.69499999999997	-2.52116633298013e-13\\
2.695	-3.70536847862836e-21\\
2.69699999999997	-4.97943163995975e-13\\
2.697	-3.64384692056973e-21\\
2.69899999999996	-4.89718666059376e-13\\
2.69999999999997	-2.4183528200415e-13\\
2.7	-3.55374239493827e-21\\
2.70199999999997	-4.77740968833352e-13\\
2.70399999999993	-4.69990607396756e-13\\
2.70599999999997	-4.62424507593096e-13\\
2.706	-3.37992137520276e-21\\
2.70899999999997	-6.79262051367478e-13\\
2.709	-3.29621134344725e-21\\
2.71199999999996	-6.62076659071573e-13\\
2.71299999999997	-2.16974620089182e-13\\
2.713	-3.18799608604485e-21\\
2.71599999999997	-6.40068782502e-13\\
2.71899999999993	-6.24223484376276e-13\\
2.71999999999997	-2.04648891841797e-13\\
2.72	-3.00696308534364e-21\\
2.72599999999993	-1.1933199770594e-12\\
2.726	-2.86658148810545e-21\\
2.72600000000002	-2.86002543551611e-21\\
2.72999999999997	-7.63161808802509e-13\\
2.73	-2.76607581250599e-21\\
2.73199999999999	-3.71922679090143e-13\\
2.73200000000002	-2.72029007313147e-21\\
2.73400000000002	-3.6567911337568e-13\\
2.73600000000001	-3.59578913608509e-13\\
2.73799999999999	-3.53619688129396e-13\\
2.73800000000002	-2.58748881072549e-21\\
2.73999999999997	-3.47799098697249e-13\\
2.74	-2.54468221402587e-21\\
2.74199999999995	-3.42114863416966e-13\\
2.7439999999999	-3.36564755624484e-13\\
2.74599999999997	-3.31146599352053e-13\\
2.746	-2.42041448286112e-21\\
2.7499999999999	-6.4581236266938e-13\\
2.7539999999998	-6.24182646409535e-13\\
2.75499999999997	-1.52781470109393e-13\\
2.755	-2.24533007398691e-21\\
2.75999999999997	-7.45069348843339e-13\\
2.76	-2.15364873531277e-21\\
2.76499999999998	-7.15000520340944e-13\\
2.76500000000001	-2.06567032234951e-21\\
2.76599999999997	-1.39547298758178e-13\\
2.766	-2.04850309346477e-21\\
2.76699999999999	-1.38389212094298e-13\\
2.76700000000002	-2.03147157389927e-21\\
2.76800000000001	-1.37205366753916e-13\\
2.76899999999999	-1.36034968966462e-13\\
2.77099999999997	-2.68611960555482e-13\\
2.77299999999999	-2.64088897755002e-13\\
2.77300000000002	-1.93228744857534e-21\\
2.77699999999997	-5.15021014427911e-13\\
2.77999999999997	-3.7515027700819e-13\\
2.78	-1.82274094737461e-21\\
2.78399999999995	-4.860089462746e-13\\
2.784	-1.76452164941766e-21\\
2.786	-2.37137238103021e-13\\
2.78600000000003	-1.733744714517e-21\\
2.78800000000004	-2.33217102979288e-13\\
2.79000000000004	-2.29255291136282e-13\\
2.792	-2.25383359702906e-13\\
2.79200000000003	-1.64907825517823e-21\\
2.79600000000004	-4.39502884395394e-13\\
2.79999999999997	-4.2505641834489e-13\\
2.8	-1.54265468390732e-21\\
2.80400000000001	-4.11276593172695e-13\\
2.80599999999997	-2.00673251347779e-13\\
2.806	-1.46732308702817e-21\\
2.81000000000001	-3.91359195272636e-13\\
2.81199999999997	-1.90726736895641e-13\\
2.812	-1.39564756383302e-21\\
2.81299999999997	-9.41580775924601e-14\\
2.813	-1.38405411056021e-21\\
2.81399999999998	-9.33668802552697e-14\\
2.81499999999997	-9.2584834193694e-14\\
2.81699999999995	-1.82859750202558e-13\\
2.81899999999997	-1.79839478195255e-13\\
2.819	-1.31649671896936e-21\\
2.81999999999997	-8.88092163596395e-14\\
2.82	-1.30556080966004e-21\\
2.82099999999999	-8.8080496766984e-14\\
2.82199999999998	-8.7360410310339e-14\\
2.82399999999995	-1.72594739158064e-13\\
2.82599999999997	-1.6981623896284e-13\\
2.826	-1.24179196989816e-21\\
2.82999999999995	-3.31180893256868e-13\\
2.83399999999991	-3.20088896519072e-13\\
2.83499999999997	-7.83483047353677e-14\\
2.835	-1.15192331998825e-21\\
2.83999999999997	-3.82081154125194e-13\\
2.84	-1.10486126677796e-21\\
2.84199999999997	-1.48463807717687e-13\\
2.842	-1.08657443600943e-21\\
2.84399999999996	-1.46055287615114e-13\\
2.84599999999992	-1.43704029914483e-13\\
2.846	-1.05433216357833e-21\\
2.84600000000003	-1.05088288504747e-21\\
2.847	-7.09677699643506e-14\\
2.84700000000003	2.09288283441186e-13\\
2.84800000000002	2.75508841235254e-13\\
2.84900000000001	2.0574835874759e-13\\
2.85099999999998	1.35428184909379e-13\\
2.853	-9.93968981494701e-22\\
2.85300000000003	6.68609119114558e-14\\
2.85699999999998	-1.30947611979404e-13\\
2.85999999999997	-1.92381892185277e-13\\
2.86	-9.34985510823415e-22\\
2.86399999999995	-2.49231645173912e-13\\
2.86599999999997	-1.21607028946359e-13\\
2.866	-8.89296706676932e-22\\
2.86999999999995	-2.37161797356326e-13\\
2.87	-8.61242036096886e-22\\
2.87099999999997	-5.80350918373591e-14\\
2.871	-8.52924993531607e-22\\
2.87199999999998	-5.75443976714675e-14\\
2.87299999999997	-5.70593436709622e-14\\
2.87499999999995	-1.12685847154285e-13\\
2.87699999999997	-1.10812132730654e-13\\
2.87699999999999	-8.11263848109917e-22\\
2.87999999999997	-1.62799820410263e-13\\
2.88	-7.9120633384264e-22\\
2.88299999999998	-1.58825157129208e-13\\
2.88599999999996	-1.54990601783129e-13\\
2.886	-7.53243267370431e-22\\
2.888	-1.01206651166377e-13\\
2.88800000000003	-7.40068154293537e-22\\
2.89000000000003	-9.94873855164452e-14\\
2.89200000000003	-9.78071259588153e-14\\
2.89600000000002	-1.90726210430494e-13\\
2.89999999999997	-1.84457045373793e-13\\
2.9	-6.69525965472816e-22\\
2.90499999999997	-2.22195100411363e-13\\
2.905	-6.42147418108305e-22\\
2.90599999999997	-4.33660176377486e-14\\
2.90599999999999	-6.36803863662955e-22\\
2.90699999999998	-4.30061281446079e-14\\
2.90799999999997	-4.26382338280294e-14\\
2.90999999999995	-8.41894643999595e-14\\
2.91199999999997	-8.27675742613635e-14\\
2.91199999999999	-6.05677961941379e-22\\
2.91599999999995	-1.61398728165368e-13\\
2.91799999999997	-7.86944315067914e-14\\
2.91799999999999	-5.76094138139104e-22\\
2.91999999999997	-7.73991190776162e-14\\
2.92	-5.66557327287736e-22\\
2.92199999999998	-7.61341508654563e-14\\
2.92399999999996	-7.48990313488191e-14\\
2.926	-7.36932762862737e-14\\
2.92600000000003	-5.38869394987084e-22\\
2.92899999999997	-1.08249120004809e-13\\
2.92899999999999	-5.255302998161e-22\\
2.93199999999993	-1.0551040732349e-13\\
2.93499999999987	-1.02864770862588e-13\\
2.93499999999997	-5.02231917758234e-22\\
2.935	-4.99854581290711e-22\\
2.93999999999997	-1.65807530497599e-13\\
2.94	-4.79427097569772e-22\\
2.94499999999998	-1.59116025703819e-13\\
2.94599999999997	-3.10548187461651e-14\\
2.946	-4.55997852954062e-22\\
2.95099999999998	-1.51380762972219e-13\\
2.95199999999997	-2.95095078926174e-14\\
2.952	-4.33709616751662e-22\\
2.95699999999998	-1.43873404852678e-13\\
2.95799999999999	-2.80595963692134e-14\\
2.95800000000002	-4.12526846045629e-22\\
2.95999999999997	-5.5426247570162e-14\\
2.96	-4.05699374817799e-22\\
2.96199999999995	-5.45203914646572e-14\\
2.9639999999999	-5.36359106113462e-14\\
2.96599999999997	-5.27724579251377e-14\\
2.966	-3.85874101201019e-22\\
2.9699999999999	-1.02918483272851e-13\\
2.96999999999999	-3.74754822752003e-22\\
2.97000000000002	-3.73194165580662e-22\\
2.97399999999992	-9.94715106018538e-14\\
2.97499999999997	-2.43476858591801e-14\\
2.975	-3.57939387965665e-22\\
2.9789999999999	-9.53815694812586e-14\\
2.97999999999997	-2.33547849592156e-14\\
2.98	-3.43313073745751e-22\\
2.9839999999999	-9.15253838214441e-14\\
2.986	-4.46577720784714e-14\\
2.98600000000003	-3.26537624740977e-22\\
2.98699999999997	-2.2054096109971e-14\\
2.98699999999999	-3.2382020718177e-22\\
2.98799999999998	-2.18654350939085e-14\\
2.98899999999997	-2.16789171993578e-14\\
2.99099999999995	-4.28067613647803e-14\\
2.99299999999999	-4.20859533130534e-14\\
2.99300000000002	-3.07997550689209e-22\\
2.99699999999997	-8.20752048642049e-14\\
2.99999999999997	-5.97850087827142e-14\\
3	-2.90521744661892e-22\\
3.00399999999995	-7.74517598223231e-14\\
3.00599999999997	-3.77908607832978e-14\\
3.006	-2.76326388309945e-22\\
3.00999999999995	-7.37009081666485e-14\\
3.01	-2.67609084032243e-22\\
3.01399999999995	-7.12324981647626e-14\\
3.01599999999997	-3.4725653852154e-14\\
3.01599999999999	-2.54194460162749e-22\\
3.01999999999995	-6.77381928625918e-14\\
3.02	-2.46309587155786e-22\\
3.02399999999995	-6.55422010053954e-14\\
3.02599999999997	-3.19798568736495e-14\\
3.026	-2.33838062275702e-22\\
3.02799999999997	-3.1451195335611e-14\\
3.02799999999999	-2.29963480431844e-22\\
3.02999999999996	-3.0916913182516e-14\\
3.03199999999992	-3.03947521279748e-14\\
3.03599999999985	-5.92704860276047e-14\\
3.03999999999997	-5.7322266889183e-14\\
3.04	-2.08048040960591e-22\\
3.04499999999997	-6.90498258228451e-14\\
3.04499999999999	-1.99542378597429e-22\\
3.04599999999997	-1.34765166306347e-14\\
3.046	-1.97882564765746e-22\\
3.04699999999998	-1.33646766010387e-14\\
3.04799999999996	-1.3250348974872e-14\\
3.04999999999992	-2.61628984886174e-14\\
3.05199999999997	-2.57210288629874e-14\\
3.052	-1.88211660182265e-22\\
3.05599999999992	-5.01566148709906e-14\\
3.05799999999997	-2.44552502928667e-14\\
3.058	-1.79020786925942e-22\\
3.05999999999997	-2.40527162264414e-14\\
3.06	-1.76057598173991e-22\\
3.06199999999997	-2.36596119944948e-14\\
3.06399999999995	-2.32757836071793e-14\\
3.066	-2.29010805808338e-14\\
3.06600000000003	-1.67455000582796e-22\\
3.06999999999997	-4.46623972027911e-14\\
3.07399999999992	-4.31665524994231e-14\\
3.07399999999996	-1.56734166310676e-22\\
3.07399999999999	-1.56734368259068e-22\\
3.07999999999997	-6.20926004663135e-14\\
3.07999999999999	-1.48986416261702e-22\\
3.08599999999997	-5.90978991514096e-14\\
3.08599999999999	-1.41706378687991e-22\\
3.09199999999997	-5.62138481868882e-14\\
3.09199999999999	-1.3478116443856e-22\\
3.09799999999997	-5.34302678326972e-14\\
3.09999999999997	-1.7224379687526e-14\\
3.1	-1.26076986806321e-22\\
3.10299999999997	-2.53108262482193e-14\\
3.10299999999999	-1.229591055863e-22\\
3.10599999999996	-2.46997402998342e-14\\
3.106	-1.20046807829067e-22\\
3.10600000000003	-1.19916753064789e-22\\
3.10899999999999	-2.40897303822088e-14\\
3.11199999999996	-2.34802580174414e-14\\
3.11199999999999	-1.14121642015516e-22\\
3.11200000000003	-1.14118208892856e-22\\
3.11499999999997	-2.28914991064907e-14\\
3.115	-1.1123630436858e-22\\
3.11799999999995	-2.23229333212691e-14\\
3.11999999999997	-1.45758305469955e-14\\
3.12	-1.06690748993177e-22\\
3.12299999999995	-2.14188448624152e-14\\
3.12599999999989	-2.09017243515612e-14\\
3.12599999999995	-1.0166384962607e-22\\
3.126	-1.01666979826142e-22\\
3.127	-6.85357823547967e-15\\
3.12700000000003	-1.00632802112059e-22\\
3.12800000000003	-6.79494944310928e-15\\
3.12900000000003	-6.73698674171106e-15\\
3.13100000000003	-1.33027208440964e-14\\
3.13199999999997	-6.56703802930638e-15\\
3.13199999999999	-9.65180026562302e-23\\
3.13599999999999	-2.5720957474245e-14\\
3.13799999999997	-1.25409670205566e-14\\
3.13799999999999	-9.18044262189025e-23\\
3.13999999999997	-1.23345423750256e-14\\
3.14	-9.02848655452806e-23\\
3.14199999999998	-1.21329534693762e-14\\
3.14399999999996	-1.19361213362102e-14\\
3.14599999999997	-1.17439688052783e-14\\
3.146	-8.58731009794033e-23\\
3.14999999999996	-2.29034520035505e-14\\
3.15	-8.31213521936007e-23\\
3.15399999999996	-2.21363636192309e-14\\
3.15599999999997	-1.07914185263474e-14\\
3.156	-7.89949407052472e-23\\
3.15999999999996	-2.10504658112578e-14\\
3.16	-7.64638205374168e-23\\
3.16099999999997	-5.15472093596526e-15\\
3.16099999999999	-7.5766795663338e-23\\
3.16199999999996	-5.11257941130608e-15\\
3.16299999999992	-5.07093899186783e-15\\
3.16499999999985	-1.00189406382178e-14\\
3.16599999999997	-4.94898369960523e-15\\
3.166	-7.26685034373154e-23\\
3.16999999999986	-1.93816467909573e-14\\
3.17199999999997	-9.44553825117721e-15\\
3.172	-6.91171399944318e-23\\
3.17599999999986	-1.84190230937188e-14\\
3.17999999999972	-1.78135895539458e-14\\
3.17999999999997	-6.56397906371196e-23\\
3.18	-6.46536766402699e-23\\
3.18499999999997	-2.14580716928221e-14\\
3.185	-6.20101721924381e-23\\
3.18599999999997	-4.18799115825216e-15\\
3.186	-6.14943959999429e-23\\
3.18699999999997	-4.15323551165964e-15\\
3.18799999999994	-4.11770681466959e-15\\
3.18999999999989	-8.13043834824902e-15\\
3.18999999999994	-5.95757853042495e-23\\
3.18999999999999	-5.95759872526412e-23\\
3.19399999999988	-1.58520612437341e-14\\
3.19599999999999	-7.72783779529022e-15\\
3.19600000000002	-5.65689756996836e-23\\
3.198	-7.59976595292143e-15\\
3.19800000000003	-5.56328439297887e-23\\
3.19999999999998	-7.47467358848928e-15\\
3.20000000000001	-5.47120602376657e-23\\
3.20199999999996	-7.3525116995726e-15\\
3.20399999999991	-7.23323237174492e-15\\
3.20599999999998	-7.11678886076137e-15\\
3.20600000000001	-5.20386674278569e-23\\
3.20999999999991	-1.38793822409457e-14\\
3.21399999999982	-1.341453034671e-14\\
3.21899999999997	-1.61464499407844e-14\\
3.21899999999999	-4.66872369952293e-23\\
3.21999999999997	-3.14958003575066e-15\\
3.22	-4.62991931605074e-23\\
3.22099999999998	-3.12373618238164e-15\\
3.22199999999996	-3.09819863154875e-15\\
3.22399999999992	-6.12099671509272e-15\\
3.22599999999997	-6.02245842254511e-15\\
3.226	-4.40368158149805e-23\\
3.22999999999992	-1.17451850923101e-14\\
3.23099999999999	-2.87412603069908e-15\\
3.23100000000002	-4.22358400574737e-23\\
3.23499999999994	-1.12562743835462e-14\\
3.23699999999999	-5.48785269000128e-15\\
3.23700000000002	-4.01731391842763e-23\\
3.23999999999997	-8.0624874722802e-15\\
3.24	-3.91799569937158e-23\\
3.24299999999996	-7.86564650768708e-15\\
3.24599999999991	-7.67574423913768e-15\\
3.24599999999997	-3.7362068058401e-23\\
3.246	-3.72655367271509e-23\\
3.24799999999997	-5.01215144887412e-15\\
3.24799999999999	-3.66480795194163e-23\\
3.24999999999996	-4.92700665008475e-15\\
3.25199999999992	-4.84379358641103e-15\\
3.25399999999997	-4.7624795507568e-15\\
3.25399999999999	-3.48577060524757e-23\\
3.25499999999998	-2.35133173179837e-15\\
3.255	-3.45680615716275e-23\\
3.25599999999998	-2.33170093629583e-15\\
3.25699999999997	-2.31229868053467e-15\\
3.25899999999993	-4.56729519799835e-15\\
3.25999999999997	-2.25544420206266e-15\\
3.26	-3.31554830585281e-23\\
3.26399999999993	-8.83889089641289e-15\\
3.26599999999997	-4.31273990110107e-15\\
3.266	-3.15354530600172e-23\\
3.267	-2.12983263214203e-15\\
3.26700000000003	-3.12730716120535e-23\\
3.26800000000003	-2.11161305074038e-15\\
3.26900000000003	-2.09360043735666e-15\\
3.27100000000003	-4.13398203727241e-15\\
3.27500000000003	-8.06072542470068e-15\\
3.27699999999997	-3.9299038745821e-15\\
3.27699999999999	-2.87686086319323e-23\\
3.27999999999997	-5.77362449194617e-15\\
3.28	-2.8057396883334e-23\\
3.28299999999998	-5.63266446564174e-15\\
3.28599999999996	-5.49667364406834e-15\\
3.286	-2.67065136039101e-23\\
3.28899999999999	-5.36092227377627e-15\\
3.28900000000002	-2.60259222802089e-23\\
3.29	-1.75660362889877e-15\\
3.29000000000003	-2.58095345784631e-23\\
3.29100000000001	-1.7417065745531e-15\\
3.29199999999999	-1.72698022760235e-15\\
3.29399999999996	-3.41045701057376e-15\\
3.296	-3.35356432740438e-15\\
3.29600000000003	-2.45488969865965e-23\\
3.29999999999996	-6.54168782173381e-15\\
3.3	-2.3768038286398e-23\\
3.30000000000003	-2.37430724164696e-23\\
3.30399999999996	-6.32961405188467e-15\\
3.30599999999997	-3.08839416535399e-15\\
3.30599999999999	-2.25829041494919e-23\\
3.30999999999992	-6.02308203185488e-15\\
3.31199999999999	-2.93531567762314e-15\\
3.31200000000002	-2.14792561886515e-23\\
3.31599999999995	-5.72393502430697e-15\\
3.31999999999988	-5.53578919692153e-15\\
3.32	-2.02257372775936e-23\\
3.32000000000003	-2.00921484164599e-23\\
3.32499999999998	-6.6683562549883e-15\\
3.325	-1.92707233330713e-23\\
3.326	-1.30146908883968e-15\\
3.32600000000003	-1.91104015535814e-23\\
3.32700000000003	-1.29066834957937e-15\\
3.32800000000003	-1.27962737568767e-15\\
3.33000000000002	-2.52663240842292e-15\\
3.332	-2.4839596863827e-15\\
3.33200000000003	-1.81764659724467e-23\\
3.33499999999999	-3.64805394020833e-15\\
3.33500000000002	-1.77271055572838e-23\\
3.33799999999998	-3.5574457406234e-15\\
3.33999999999997	-2.322845549598e-15\\
3.34	-1.70027166761247e-23\\
3.34299999999997	-3.41336765216405e-15\\
3.34599999999993	-3.33095786815518e-15\\
3.34599999999997	-1.61816197588727e-23\\
3.346	-1.61811401314424e-23\\
3.34699999999999	-1.09220559857034e-15\\
3.34700000000002	-1.60373276329769e-23\\
3.34800000000001	-1.082862348596e-15\\
3.349	-1.07362524876166e-15\\
3.35099999999999	-2.11995919315047e-15\\
3.35499999999995	-4.13364374107346e-15\\
3.35999999999997	-4.97609406641548e-15\\
3.36	-1.4388217066056e-23\\
3.36399999999997	-3.83571594686783e-15\\
3.36399999999999	-1.39157209382649e-23\\
3.36599999999997	-1.87155206034842e-15\\
3.366	-1.36851716055486e-23\\
3.36799999999998	-1.84061328032698e-15\\
3.36999999999996	-1.80934557581044e-15\\
3.37199999999997	-1.77878723218177e-15\\
3.372	-1.30163942627639e-23\\
3.37599999999996	-3.46867719229406e-15\\
3.37799999999997	-1.69124988144435e-15\\
3.378	-1.23807238344496e-23\\
3.37999999999997	-1.66341186418474e-15\\
3.38	-1.21758093257094e-23\\
3.38199999999997	-1.63622598523176e-15\\
3.38399999999995	-1.60968159515965e-15\\
3.38599999999997	-1.58376828696058e-15\\
3.386	-1.1580856741879e-23\\
3.38999999999995	-3.08871400616846e-15\\
3.39299999999997	-2.24838900064155e-15\\
3.39299999999999	-1.09233506437081e-23\\
3.39499999999998	-1.46758187527382e-15\\
3.395	-1.07425437242315e-23\\
3.39699999999999	-1.44317925258427e-15\\
3.39899999999997	-1.41934243404399e-15\\
3.399	-1.03899418322594e-23\\
3.39999999999997	-7.00906669043044e-16\\
3.4	-1.03035962730175e-23\\
3.40099999999998	-6.95155414218966e-16\\
3.40199999999996	-6.89472294476565e-16\\
3.40399999999991	-1.36216497113552e-15\\
3.40599999999997	-1.34023628583419e-15\\
3.406	-9.80013893241822e-24\\
3.40999999999991	-2.6137703489258e-15\\
3.41099999999997	-6.39607238553556e-16\\
3.411	-9.39933448369353e-24\\
3.41499999999991	-2.50496829410024e-15\\
3.41899999999982	-2.42235903804074e-15\\
3.41999999999997	-5.93130067923769e-16\\
3.42	-8.7192732745206e-24\\
3.42199999999997	-1.17171710995924e-15\\
3.42199999999999	-8.57493066152688e-24\\
3.42399999999996	-1.15270834633025e-15\\
3.42599999999992	-1.1341515761829e-15\\
3.426	-8.32421173318592e-24\\
3.42600000000003	-8.29323789860332e-24\\
3.42999999999996	-2.21185756682067e-15\\
3.43	-8.0318283272748e-24\\
3.432	-1.07793653855516e-15\\
3.43200000000003	-7.88794009816249e-24\\
3.43400000000003	-1.05984092952163e-15\\
3.43600000000003	-1.0421608353296e-15\\
3.438	-1.02488932427206e-15\\
3.43800000000003	-7.50273616269945e-24\\
3.43999999999997	-1.00801961948229e-15\\
3.44	-7.3785757671049e-24\\
3.44199999999995	-9.91545107378178e-16\\
3.44399999999989	-9.75459334462297e-16\\
3.44599999999997	-9.59755994176549e-16\\
3.446	-7.01804740075718e-24\\
3.44999999999989	-1.87174589067023e-15\\
3.45099999999996	-4.58028847485905e-16\\
3.45099999999999	-6.73102824900156e-24\\
3.45499999999988	-1.79383170112401e-15\\
3.45699999999996	-8.74559716737271e-16\\
3.45699999999999	-6.40230044346511e-24\\
3.45999999999997	-1.28486079425016e-15\\
3.46	-6.24401076966708e-24\\
3.46299999999998	-1.253491663071e-15\\
3.46499999999998	-8.18780984576616e-16\\
3.465	-5.98874800251204e-24\\
3.46599999999997	-4.0444732430757e-16\\
3.466	-5.93892354773828e-24\\
3.46699999999997	-4.01090866484625e-16\\
3.46799999999994	-3.97659749748799e-16\\
3.46999999999988	-7.85181710208316e-16\\
3.47199999999997	-7.71920643616991e-16\\
3.472	-5.64869846526385e-24\\
3.47599999999988	-1.50526352052688e-15\\
3.47999999999976	-1.45578548808156e-15\\
3.47999999999996	-5.3471832055139e-24\\
3.47999999999999	-5.28391656091517e-24\\
3.48599999999996	-2.09588109573631e-15\\
3.48599999999999	-5.0257318529672e-24\\
3.49199999999996	-1.99359947063052e-15\\
3.49199999999999	-4.78027620458587e-24\\
3.49799999999996	-1.89488101839064e-15\\
3.5	-6.10855073343264e-16\\
3.50000000000003	-4.47144031563546e-24\\
3.506	-1.77360280106898e-15\\
3.50600000000003	-4.25295108843821e-24\\
3.50899999999999	-8.54331740318104e-16\\
3.50900000000002	-4.14770442190724e-24\\
3.51199999999998	-8.32717058235931e-16\\
3.51200000000003	-4.05009392893882e-24\\
3.51499999999999	-3.94995277018644e-24\\
3.51799999999995	-7.91673046019423e-16\\
3.51799999999999	-7.91673046019284e-16\\
3.51800000000003	7.04847849903698e-24\\
3.51999999999997	-5.16925440464543e-16\\
3.52	-2.59049930588821e-24\\
3.52199999999995	-5.08477099557319e-16\\
3.52399999999989	-5.00228107928441e-16\\
3.52599999999997	-4.92175232677288e-16\\
3.526	-2.46691951191991e-24\\
3.52999999999989	-9.59855396103668e-16\\
3.53399999999978	-9.27707668553242e-16\\
3.53499999999998	-2.27075427002875e-16\\
3.535	-2.32944945513247e-24\\
3.53799999999997	-6.69939543476046e-16\\
3.53799999999999	-2.3032087859812e-24\\
3.53999999999997	-4.37439136507006e-16\\
3.54	-2.29112659235684e-24\\
3.54199999999998	-4.30289875338507e-16\\
3.54399999999996	-4.23309310023056e-16\\
3.54599999999997	-4.16494704752309e-16\\
3.546	-2.28084300159637e-24\\
3.54999999999996	-8.1226088404789e-16\\
3.54999999999999	-2.29082051432561e-24\\
3.553	-5.91274696322269e-16\\
3.55300000000003	-2.30285222085204e-24\\
3.55600000000004	-5.76494658994397e-16\\
3.55900000000005	-5.62223177604835e-16\\
3.55999999999997	-1.84322366872866e-16\\
3.56	-2.34762481038876e-24\\
3.56599999999999	-1.07479479136818e-15\\
3.56600000000002	-2.40428395604536e-24\\
3.56699999999996	-1.74056973620724e-16\\
3.56699999999999	-2.41506610689792e-24\\
3.56799999999996	-1.72568006876427e-16\\
3.56899999999992	-1.71095956514256e-16\\
3.56999999999998	-1.69640675377647e-16\\
3.57	-2.44654165701624e-24\\
3.57199999999993	-3.34981869876282e-16\\
3.57299999999996	-1.65374028738453e-16\\
3.57299999999999	-2.47671716436275e-24\\
3.57499999999992	-3.26595283452025e-16\\
3.57699999999985	-3.21164731117102e-16\\
3.57899999999996	-3.15860092671813e-16\\
3.57899999999999	-2.5332942684852e-24\\
3.57999999999997	-1.55979585818134e-16\\
3.58	-2.54224310659403e-24\\
3.58099999999998	-1.54699703098975e-16\\
3.58199999999997	-1.53434983142669e-16\\
3.58399999999993	-3.0313583426547e-16\\
3.58599999999997	-2.98255830410775e-16\\
3.586	-2.59315618941699e-24\\
3.58999999999993	-5.81667765316328e-16\\
3.59399999999986	-5.62186396849912e-16\\
3.59599999999996	-2.74064393054274e-16\\
3.59599999999999	-2.65369021984003e-24\\
3.59999999999997	-5.34608404552465e-16\\
3.6	-2.66657074070048e-24\\
3.60399999999998	-5.17277010239741e-16\\
3.60499999999998	-1.26706878802015e-16\\
3.605	-2.67362000174954e-24\\
3.60599999999997	-1.25686917994809e-16\\
3.606	-2.67382826102852e-24\\
3.60699999999997	-1.24643857640025e-16\\
3.60799999999994	-1.23577596435623e-16\\
3.60799999999999	-2.67587614393847e-24\\
3.60999999999993	-2.44004750827812e-16\\
3.61199999999987	-2.39883712605044e-16\\
3.61399999999996	-2.35856721726145e-16\\
3.61399999999999	-2.66109604601825e-24\\
3.61799999999987	-4.60000794000718e-16\\
3.61999999999997	-2.2432442007372e-16\\
3.62	-2.63477964622008e-24\\
3.62399999999988	-4.37736646281576e-16\\
3.62499999999996	-1.07223486060429e-16\\
3.62499999999999	-2.60179894949459e-24\\
3.62599999999997	-1.06360361981705e-16\\
3.626	-2.59399395969834e-24\\
3.62699999999998	-1.05477690340284e-16\\
3.62799999999997	-1.04575385402491e-16\\
3.62999999999993	-2.06484763894447e-16\\
3.63199999999997	-2.02997407778139e-16\\
3.632	-2.54123652007897e-24\\
3.63599999999993	-3.95849747396893e-16\\
3.63999999999985	-3.82838160579886e-16\\
3.63999999999997	-2.4649094943334e-24\\
3.64	-2.45608845169122e-24\\
3.64599999999997	-5.51168608713187e-16\\
3.646	-2.38083585449857e-24\\
3.65199999999997	-5.24270889294779e-16\\
3.652	-2.29874698870127e-24\\
3.65399999999996	-1.68899251290985e-16\\
3.65399999999999	-2.27053574500876e-24\\
3.65599999999995	-1.66081698238539e-16\\
3.65799999999992	-1.63329260953614e-16\\
3.65999999999996	-1.60640857536265e-16\\
3.65999999999999	-2.18320410963536e-24\\
3.66399999999992	-3.13467388935934e-16\\
3.66599999999996	-1.52949429680598e-16\\
3.66599999999999	-2.09158107093756e-24\\
3.66999999999992	-2.98286717999392e-16\\
3.67399999999984	-2.88296421538587e-16\\
3.67499999999998	-7.05664468710721e-17\\
3.67500000000001	-1.95149041822298e-24\\
3.67999999999997	-3.44131366604455e-16\\
3.68	-1.87404478771381e-24\\
3.68299999999996	-1.99759976907205e-16\\
3.68299999999999	-1.82760612394526e-24\\
3.68599999999995	-1.94937121210494e-16\\
3.686	-1.78244699256498e-24\\
3.68699999999998	-6.39189752964751e-17\\
3.68700000000001	1.26203778824115e-16\\
3.68799999999998	2.48144229067125e-16\\
3.68899999999995	1.85312629420202e-16\\
3.6909999999999	6.12465938907059e-17\\
3.69299999999998	-1.05838943298606e-24\\
3.693	-1.05838785526425e-24\\
3.6969999999999	-1.77657005705696e-16\\
3.69999999999997	-2.91215157831432e-16\\
3.7	-1.73273773545558e-16\\
3.7039999999999	-2.24476987672425e-16\\
3.70599999999997	-3.34005532100912e-16\\
3.706	-1.09528545380411e-16\\
3.7099999999999	-2.13605964204618e-16\\
3.70999999999998	-2.13605964207247e-16\\
3.71000000000001	-9.02714833088845e-25\\
3.71199999999996	-1.04099683986916e-16\\
3.71199999999999	-1.04099683986891e-16\\
3.71399999999995	-1.02352134723425e-16\\
3.71599999999991	-2.02996846870362e-16\\
3.71799999999996	-1.99621461628881e-16\\
3.71799999999999	-9.89767494727448e-17\\
3.71999999999997	-9.73475895980789e-17\\
3.72	-9.73475895980544e-17\\
3.72199999999998	-9.57565947139776e-17\\
3.72399999999997	-1.8995973544682e-16\\
3.72599999999997	-1.86889761893009e-16\\
3.726	-9.26866211593717e-17\\
3.728	-9.11544113120215e-17\\
3.72800000000003	-9.11544113119987e-17\\
3.73000000000004	-8.96059113528959e-17\\
3.73200000000004	-1.77698452242162e-16\\
3.73600000000004	-2.598750820667e-16\\
3.74	-3.37918594414458e-16\\
3.74000000000003	-1.66136054108006e-16\\
3.74099999999996	-4.06824725483292e-17\\
3.74099999999999	-4.0682472548319e-17\\
3.74199999999996	-4.03498783911048e-17\\
3.74299999999992	-8.03711182881174e-17\\
3.74499999999985	-1.19093465968026e-16\\
3.745	-7.90722260811003e-17\\
3.74500000000003	-9.23147908255879e-25\\
3.746	-3.90587362346191e-17\\
3.74600000000003	-3.90587362346112e-17\\
3.747	-3.87345925733768e-17\\
3.74799999999997	-7.71378306095819e-17\\
3.74999999999991	-1.14230677760823e-16\\
3.752	-1.50374216076576e-16\\
3.75200000000003	-7.45467772940519e-17\\
3.75599999999991	-1.45367979261046e-16\\
3.758	-2.16246173313113e-16\\
3.75800000000003	-7.08781950344263e-17\\
3.75999999999997	-6.97115381073411e-17\\
3.76	-6.97115381073237e-17\\
3.76199999999995	-6.85722114952833e-17\\
3.76399999999989	-1.36031979370381e-16\\
3.76599999999997	-1.33833542013816e-16\\
3.766	-6.63737741491516e-17\\
3.76999999999989	-1.29444191278796e-16\\
3.76999999999996	-1.29444191280178e-16\\
3.76999999999999	-1.05578382541612e-24\\
3.77399999999988	-1.25108812688206e-16\\
3.77599999999999	-1.86099026147705e-16\\
3.77600000000002	-6.09902145297066e-17\\
3.77999999999991	-1.18971608409383e-16\\
3.77999999999996	-1.18971608409888e-16\\
3.78	-1.08939916073985e-24\\
3.78399999999989	-1.15114689942324e-16\\
3.78599999999997	-1.712823376734e-16\\
3.786	-5.61676488283352e-17\\
3.78999999999989	-1.09539889179611e-16\\
3.79199999999997	-1.62923546923672e-16\\
3.792	-5.33836588477352e-17\\
3.79599999999989	-1.04099397250618e-16\\
3.79899999999996	-1.79919697462857e-16\\
3.79899999999999	-7.58203013159293e-17\\
3.79999999999997	-2.4857348586169e-17\\
3.8	-2.48573485861635e-17\\
3.80099999999999	-2.46533828591813e-17\\
3.80199999999997	-4.91052152735701e-17\\
3.80399999999993	-7.27604169238814e-17\\
3.80599999999997	-9.58394767225339e-17\\
3.806	-4.75308933075751e-17\\
3.80999999999993	-9.26962207961236e-17\\
3.81099999999996	-1.15379610896659e-16\\
3.81099999999999	-2.26833911777034e-17\\
3.81499999999992	-8.88376036120345e-17\\
3.81499999999998	-8.8837603612634e-17\\
3.81500000000001	-1.06089209420788e-24\\
3.81899999999993	-8.59079025529454e-17\\
3.81999999999997	-1.06942999316719e-16\\
3.82	-2.10350978081355e-17\\
3.82399999999993	-8.24347282242177e-17\\
3.826	-1.22656916768492e-16\\
3.82600000000003	-4.02221895656556e-17\\
3.82700000000001	-1.98635984128695e-17\\
3.82700000000003	-1.98635984128644e-17\\
3.82799999999998	-1.96936759356492e-17\\
3.82800000000001	-1.96936759356445e-17\\
3.82899999999997	-1.95256837186068e-17\\
3.82999999999993	-3.88852879444878e-17\\
3.83199999999986	-5.75881518761011e-17\\
3.83399999999998	-7.58153416621549e-17\\
3.83400000000001	-3.75867949962072e-17\\
3.83799999999985	-7.33070285341806e-17\\
3.84	-1.09056002398375e-16\\
3.84000000000003	-3.57489748084855e-17\\
3.84399999999988	-6.9758951056405e-17\\
3.846	-1.03796276593085e-16\\
3.84600000000003	-3.40373264440822e-17\\
3.84999999999988	-6.63806484895329e-17\\
3.84999999999998	-6.63806484906827e-17\\
3.85000000000001	-8.6977870700451e-25\\
3.85399999999985	-6.41574105063658e-17\\
3.85599999999998	-9.54339773504345e-17\\
3.85600000000001	-3.12765676867149e-17\\
3.85699999999996	-1.54431483779544e-17\\
3.85699999999999	-1.54431483779508e-17\\
3.85799999999995	-1.53150800747656e-17\\
3.85899999999992	-3.05035921245193e-17\\
3.85999999999997	-3.02519463502763e-17\\
3.86	-1.50634342939936e-17\\
3.86199999999993	-2.9757525432957e-17\\
3.86399999999985	-5.90322956351599e-17\\
3.86599999999997	-5.80782640562303e-17\\
3.866	-2.88034938408929e-17\\
3.86899999999996	-4.23098686862513e-17\\
3.86899999999999	-4.23098686862412e-17\\
3.87199999999995	-4.12394251456209e-17\\
3.87499999999991	-8.14447854678787e-17\\
3.87999999999997	-1.05012305845166e-16\\
3.88	-6.48069455038619e-17\\
3.88499999999998	-6.21915277591608e-17\\
3.88500000000001	-6.21915277591464e-17\\
3.88599999999999	-1.21379771176224e-17\\
3.88600000000002	-1.21379771176194e-17\\
3.88700000000001	-1.20372454689222e-17\\
3.88799999999999	-2.39715181389844e-17\\
3.88999999999996	-3.54985709161083e-17\\
3.89199999999999	-4.67306144400551e-17\\
3.89200000000002	-2.31663168149601e-17\\
3.89599999999996	-4.51748655082828e-17\\
3.89799999999999	-6.72011251971588e-17\\
3.89800000000002	-2.20262602699797e-17\\
3.89999999999997	-2.16637074445878e-17\\
3.9	-2.16637074445829e-17\\
3.90199999999996	-2.13096478459441e-17\\
3.90399999999991	-4.22735900762863e-17\\
3.90599999999997	-4.15903989481807e-17\\
3.906	-2.06264567074109e-17\\
3.90999999999991	-4.02263548584337e-17\\
3.91399999999982	-7.9105437616615e-17\\
3.91499999999996	-4.83955339199129e-17\\
3.91499999999999	-9.5164511425783e-18\\
3.91999999999997	-4.6408873032494e-17\\
3.92	-4.6408873032483e-17\\
3.92499999999999	-4.45359460870912e-17\\
3.92599999999997	-5.32280668032763e-17\\
3.926	-8.69212115641739e-18\\
3.92699999999996	-8.61998633482437e-18\\
3.92699999999999	-8.6199863348222e-18\\
3.92799999999995	-8.54624690786621e-18\\
3.92899999999992	-1.70195916569997e-17\\
3.93099999999984	-2.52046443349875e-17\\
3.93299999999996	-3.31808662440835e-17\\
3.93299999999999	-1.64495670781361e-17\\
3.93699999999984	-3.20796243804395e-17\\
3.93999999999997	-5.54469811684794e-17\\
3.94	-2.33673571811124e-17\\
3.94399999999985	-3.02725208916971e-17\\
3.94399999999996	-3.02725208923249e-17\\
3.94399999999999	-3.68962867146351e-25\\
3.94599999999997	-1.47708022806921e-17\\
3.946	-1.47708022806884e-17\\
3.94799999999999	-1.45266249304434e-17\\
3.94999999999997	-2.88064761969212e-17\\
3.95199999999997	-2.83185280562049e-17\\
3.952	-1.40386767832959e-17\\
3.95499999999998	-2.0617826487389e-17\\
3.95500000000001	-2.06178264873839e-17\\
3.95799999999998	-2.01057331225641e-17\\
3.95999999999998	-3.32338354685801e-17\\
3.96	-1.31281026702929e-17\\
3.96299999999998	-1.92914422128582e-17\\
3.96599999999995	-3.81171260058146e-17\\
3.966	-1.882568410322e-17\\
3.96600000000003	-3.02153251007052e-25\\
3.96700000000001	-6.17285442754606e-18\\
3.96700000000003	-6.17285442754458e-18\\
3.96800000000001	-6.12004887212521e-18\\
3.96899999999998	-1.21878918243171e-17\\
3.97099999999993	-1.8049286080302e-17\\
3.97299999999996	-2.37611346341598e-17\\
3.97299999999999	-1.17796917977127e-17\\
3.97699999999989	-2.29725248550422e-17\\
3.97899999999996	-3.41743643867794e-17\\
3.97899999999999	-1.12018398061473e-17\\
3.97999999999997	-5.5317477640626e-18\\
3.98	-5.53174776406132e-18\\
3.98099999999999	-5.48635728242159e-18\\
3.98199999999997	-1.09278615588295e-17\\
3.98399999999994	-1.61920838310028e-17\\
3.98599999999997	-2.13280916458399e-17\\
3.986	-1.05775123519342e-17\\
3.98999999999994	-2.0628592289289e-17\\
3.98999999999997	-2.062859228932e-17\\
3.99000000000001	-2.45710056023262e-25\\
3.99399999999994	-1.99376940913773e-17\\
3.99599999999998	-2.96572668983412e-17\\
3.99600000000001	-9.7195730445188e-18\\
3.99999999999994	-1.89596518544726e-17\\
3.99999999999997	-1.89596518544868e-17\\
4	-2.25944554397271e-25\\
4.00199999999993	-9.24751224565562e-18\\
4.00199999999999	-9.24751224576734e-18\\
4.00399999999992	-9.09749041696885e-18\\
4.00599999999986	-1.80485254934843e-17\\
4.00599999999993	-8.95103529538028e-18\\
4.006	-2.1535508282081e-25\\
4.00999999999987	-1.74565863955551e-17\\
4.01199999999995	-2.5963956982651e-17\\
4.012	-8.50737079540923e-18\\
4.01599999999987	-1.65895742110397e-17\\
4.01999999999973	-3.26338486869768e-17\\
4.01999999999995	-1.60442746747795e-17\\
4.02	-1.92461156059939e-25\\
4.02500000000001	-1.93267726539639e-17\\
4.02500000000006	-1.93267726539549e-17\\
4.02599999999995	-3.77202370772e-18\\
4.026	-3.77202370771823e-18\\
4.02699999999996	-3.74072012608552e-18\\
4.02799999999992	-7.44944022111481e-18\\
4.02999999999985	-1.10316117932771e-17\\
4.03099999999993	-1.09377808519858e-17\\
4.03099999999999	-3.61488933204424e-18\\
4.03499999999984	-1.41574115902039e-17\\
4.03699999999994	-2.10596765360628e-17\\
4.03699999999999	-6.90226511565471e-18\\
4.03799999999995	-3.40821381852235e-18\\
4.038	-3.40821381852067e-18\\
4.03899999999996	-3.38004757315912e-18\\
4.03999999999991	-6.73226002915132e-18\\
4.04	-3.35221262046921e-18\\
4.04199999999991	-6.62223173175986e-18\\
4.04399999999982	-1.31370312591254e-17\\
4.046	-1.29247213271986e-17\\
4.04600000000006	-6.40992179713233e-18\\
4.04999999999988	-1.2500828072169e-17\\
4.05399999999969	-2.45829749933999e-17\\
4.05999999999994	-2.94616196087134e-17\\
4.06	-1.73794726826654e-17\\
};
\end{axis}
\end{tikzpicture}%
}
      \caption{The evolution of the difference in angular displacement between
        RM and EDF of pendulum $P_1$ for execution time $C_1 = 6$ ms.}
      \label{fig:02.6.6.1_diff}
    \end{figure}
  \end{minipage}
\end{minipage}
}

\noindent\makebox[\textwidth][c]{%
\begin{minipage}{\linewidth}
  \begin{minipage}{0.45\linewidth}
    \begin{figure}[H]\centering
      \scalebox{0.7}{% This file was created by matlab2tikz.
%
%The latest updates can be retrieved from
%  http://www.mathworks.com/matlabcentral/fileexchange/22022-matlab2tikz-matlab2tikz
%where you can also make suggestions and rate matlab2tikz.
%
\definecolor{mycolor1}{rgb}{0.00000,0.44700,0.74100}%
\definecolor{mycolor2}{rgb}{0.85000,0.32500,0.09800}%
%
\begin{tikzpicture}

\begin{axis}[%
width=4.133in,
height=3.26in,
at={(0.693in,0.44in)},
scale only axis,
xmin=0,
xmax=1,
xmajorgrids,
ymin=-0.1,
ymax=0.2,
ymajorgrids,
axis background/.style={fill=white}
]
\pgfplotsset{max space between ticks=50}
\addplot [color=mycolor1,solid,forget plot]
  table[row sep=crcr]{%
0	0.15313\\
3.15544362088405e-30	0.15313\\
0.000656101980281985	0.153131614989962\\
0.00393661188169191	0.153188143215565\\
0.00599999999999994	0.153265080494076\\
0.006	0.153265080494076\\
0.012	0.153670560289007\\
0.0120000000000001	0.153670560289007\\
0.018	0.153071664853654\\
0.0180000000000001	0.153071664853654\\
0.0199999999999998	0.152365329512728\\
0.02	0.152365329512728\\
0.026	0.148724274488254\\
0.0260000000000002	0.148724274488254\\
0.0289999999999998	0.146046083995443\\
0.029	0.146046083995443\\
0.0319999999999996	0.142794627452511\\
0.0349999999999991	0.138968470912041\\
0.035	0.13896847091204\\
0.0399999999999996	0.131789348388764\\
0.04	0.131789348388763\\
0.0449999999999996	0.123959478342115\\
0.0459999999999996	0.122314584640995\\
0.046	0.122314584640994\\
0.047	0.120643199324198\\
0.0470000000000004	0.120643199324197\\
0.0490000000000003	0.117220624955326\\
0.0510000000000002	0.113691084393681\\
0.055	0.106308316385691\\
0.0579999999999996	0.100485154924159\\
0.058	0.100485154924158\\
0.0599999999999996	0.0964653994777478\\
0.06	0.0964653994777469\\
0.0619999999999995	0.092334609798713\\
0.0639999999999991	0.0880919761882657\\
0.0659999999999991	0.0837366670740184\\
0.066	0.0837366670740165\\
0.0699999999999991	0.0746845854767947\\
0.07	0.0746845854767927\\
0.0700000000000009	0.0746845854767906\\
0.074	0.0654820973483735\\
0.076	0.0609387524318376\\
0.0760000000000009	0.0609387524318356\\
0.08	0.0519634081808075\\
0.0800000000000009	0.0519634081808055\\
0.0839999999999999	0.0431306724709095\\
0.086	0.0387656144357706\\
0.0860000000000009	0.0387656144357686\\
0.0869999999999991	0.0365955374773486\\
0.087	0.0365955374773467\\
0.0880000000000004	0.0344336199800889\\
0.0890000000000009	0.03227975600762\\
0.0910000000000017	0.0279957668674395\\
0.0929999999999991	0.0237427303784554\\
0.093	0.0237427303784535\\
0.0970000000000017	0.0154879858559104\\
0.0999999999999991	0.00958444869935146\\
0.1	0.00958444869934975\\
0.104000000000002	0.00209147738315415\\
0.104999999999999	0.000285191714757269\\
0.105	0.000285191714755677\\
0.105999999999999	-0.00149449006990914\\
0.106	-0.00149449006991071\\
0.106999999999999	-0.00324765517394124\\
0.107999999999998	-0.00497438950448962\\
0.109999999999997	-0.00834890299548232\\
0.111999999999999	-0.0116186919605986\\
0.112	-0.0116186919606\\
0.115999999999997	-0.0178466394573827\\
0.115999999999998	-0.0178466394573853\\
0.116	-0.0178466394573879\\
0.119999999999997	-0.0236631150271634\\
0.119999999999998	-0.0236631150271658\\
0.12	-0.0236631150271682\\
0.123999999999997	-0.0290726790847185\\
0.125999999999999	-0.031626207097874\\
0.126	-0.0316262070978751\\
0.127999999999998	-0.03407957300888\\
0.128	-0.0340795730088821\\
0.129999999999998	-0.0364271843737179\\
0.131999999999996	-0.0386634280235806\\
0.135999999999993	-0.0428035436794524\\
0.139999999999998	-0.046503166088059\\
0.14	-0.0465031660880606\\
0.144999999999998	-0.0505126366966201\\
0.145	-0.0505126366966214\\
0.145999999999998	-0.0512329261213502\\
0.146	-0.0512329261213514\\
0.146999999999999	-0.0519260994409184\\
0.147999999999998	-0.0525921906227423\\
0.149999999999997	-0.0538432558021337\\
0.151999999999998	-0.0549863668732067\\
0.152	-0.0549863668732077\\
0.155999999999997	-0.0570007720155776\\
0.157999999999998	-0.0578852528495309\\
0.158	-0.0578852528495317\\
0.16	-0.0586881556806868\\
0.160000000000002	-0.0586881556806875\\
0.162000000000002	-0.0594096378799837\\
0.164000000000002	-0.0600498408609338\\
0.166	-0.0606088901058539\\
0.166000000000002	-0.0606088901058543\\
0.170000000000002	-0.0614839498018867\\
0.174	-0.0620355030603464\\
0.174000000000001	-0.0620355030603465\\
0.175	-0.0621228997391115\\
0.175000000000002	-0.0621228997391117\\
0.176000000000001	-0.0621901098124187\\
0.177	-0.0622371365737359\\
0.178999999999998	-0.0622706483888091\\
0.179999999999998	-0.0622571350847035\\
0.18	-0.0622571350847035\\
0.183999999999997	-0.0620804705828559\\
0.186	-0.0619304203383477\\
0.186000000000002	-0.0619304203383475\\
0.189999999999998	-0.0615067368047821\\
0.192	-0.061233020472304\\
0.192000000000002	-0.0612330204723037\\
0.195999999999998	-0.0605615704193962\\
0.199999999999995	-0.0597243116296848\\
0.199999999999997	-0.0597243116296842\\
0.2	-0.0597243116296836\\
0.202999999999998	-0.0589871655511613\\
0.203	-0.0589871655511608\\
0.205999999999998	-0.0581560603911522\\
0.206	-0.0581560603911517\\
0.208999999999998	-0.0572306296136106\\
0.209999999999998	-0.0569011221000455\\
0.21	-0.0569011221000449\\
0.211999999999998	-0.0562104650884198\\
0.212	-0.0562104650884192\\
0.213999999999998	-0.0554939446732612\\
0.215999999999997	-0.0547678599943136\\
0.217999999999998	-0.0540320687361973\\
0.218	-0.0540320687361967\\
0.219999999999998	-0.0532864266819164\\
0.22	-0.0532864266819157\\
0.221999999999998	-0.0525307876837086\\
0.223999999999996	-0.0517650036334558\\
0.225999999999998	-0.0509889244345494\\
0.226	-0.0509889244345487\\
0.229999999999996	-0.0494052700894414\\
0.231999999999998	-0.0485973845409472\\
0.232	-0.0485973845409465\\
0.235999999999996	-0.0469487049222538\\
0.237999999999998	-0.0461075877045604\\
0.238	-0.0461075877045596\\
0.239999999999998	-0.0452638608140846\\
0.24	-0.0452638608140839\\
0.241999999999998	-0.0444261532252999\\
0.243999999999996	-0.0435943007442838\\
0.245	-0.0431805191730594\\
0.245000000000002	-0.0431805191730586\\
0.245999999999998	-0.042768140324533\\
0.246	-0.0427681403245322\\
0.246999999999999	-0.0423571439924296\\
0.247999999999998	-0.0419475100374713\\
0.249999999999997	-0.0411322490361785\\
0.252	-0.0403221975263589\\
0.252000000000003	-0.0403221975263574\\
0.256	-0.0387170888829923\\
0.259999999999997	-0.0371309256194401\\
0.26	-0.0371309256194387\\
0.260999999999996	-0.0367371989641445\\
0.261	-0.0367371989641431\\
0.261999999999998	-0.0363445593118938\\
0.262999999999996	-0.0359529874189366\\
0.264999999999993	-0.0351729702138744\\
0.265999999999997	-0.0347844866804151\\
0.266	-0.0347844866804137\\
0.269999999999993	-0.0332402760675451\\
0.271999999999997	-0.032473740201349\\
0.272	-0.0324737402013477\\
0.275999999999993	-0.0309768515277924\\
0.279999999999986	-0.0295444303379096\\
0.279999999999993	-0.0295444303379072\\
0.28	-0.0295444303379047\\
0.285999999999996	-0.0275142319192832\\
0.286	-0.0275142319192821\\
0.289999999999996	-0.0262381743224773\\
0.29	-0.0262381743224762\\
0.293999999999996	-0.0250228688460333\\
0.295999999999997	-0.0244376985423952\\
0.296	-0.0244376985423942\\
0.297999999999997	-0.023865912771368\\
0.298	-0.023865912771367\\
0.299999999999997	-0.0233059496077691\\
0.3	-0.0233059496077682\\
0.301999999999997	-0.0227576992966336\\
0.303999999999993	-0.0222210543791462\\
0.305999999999997	-0.0216959096708285\\
0.306	-0.0216959096708276\\
0.309999999999993	-0.0206797113972297\\
0.313999999999986	-0.0197083077239614\\
0.314999999999997	-0.0194723659426118\\
0.315	-0.019472365942611\\
0.318999999999997	-0.0185558885254103\\
0.319	-0.0185558885254095\\
0.319999999999996	-0.0183335354663476\\
0.32	-0.0183335354663468\\
0.320999999999998	-0.0181138669354534\\
0.321999999999996	-0.0178968721687158\\
0.323999999999993	-0.0174708615270537\\
0.325999999999996	-0.0170554200416272\\
0.326	-0.0170554200416265\\
0.329999999999993	-0.0162559208851037\\
0.331	-0.0160625270729795\\
0.331000000000004	-0.0160625270729788\\
0.333	-0.0156829269477488\\
0.333000000000004	-0.0156829269477482\\
0.335	-0.0153124997481444\\
0.336999999999996	-0.0149511728689865\\
0.339999999999996	-0.0144260912138601\\
0.34	-0.0144260912138595\\
0.343999999999993	-0.0137571872671349\\
0.345999999999997	-0.0134359608475538\\
0.346	-0.0134359608475532\\
0.347999999999997	-0.0131234669407986\\
0.348	-0.0131234669407981\\
0.349999999999997	-0.0128196442972705\\
0.35	-0.01281964429727\\
0.351999999999997	-0.0125244333665395\\
0.353999999999993	-0.0122377762863185\\
0.354	-0.0122377762863175\\
0.357999999999993	-0.0116869790778428\\
0.359999999999996	-0.0114220006465959\\
0.36	-0.0114220006465954\\
0.363999999999993	-0.0109126244375425\\
0.365999999999996	-0.0106681268200104\\
0.366	-0.01066812682001\\
0.369999999999993	-0.0101992729386525\\
0.373999999999986	-0.00975697004641091\\
0.376999999999997	-0.00944245756938025\\
0.377	-0.00944245756937989\\
0.379999999999997	-0.00914254582941242\\
0.38	-0.00914254582941208\\
0.382999999999996	-0.00885710256496239\\
0.384999999999997	-0.00867478266822029\\
0.385	-0.00867478266821997\\
0.385999999999997	-0.00858600188607707\\
0.386	-0.00858600188607676\\
0.386999999999998	-0.00849880138853904\\
0.387999999999996	-0.0084131769027382\\
0.388999999999997	-0.00832912423305362\\
0.389	-0.00832912423305333\\
0.390999999999997	-0.00816487980833682\\
0.392999999999993	-0.00800519778581065\\
0.394999999999997	-0.00785004686718034\\
0.395	-0.00785004686718007\\
0.398999999999993	-0.00755321758399373\\
0.399999999999997	-0.00748179575418615\\
0.4	-0.0074817957541859\\
0.403999999999993	-0.00720714540431904\\
0.405999999999997	-0.00707639549649106\\
0.406	-0.00707639549649083\\
0.409999999999993	-0.00682791807339543\\
0.411999999999997	-0.00671014185557849\\
0.412	-0.00671014185557829\\
0.415999999999993	-0.00648439614930216\\
0.419999999999986	-0.00626957769808391\\
0.419999999999996	-0.00626957769808335\\
0.42	-0.00626957769808316\\
0.426	-0.00596747249584989\\
0.426000000000004	-0.00596747249584972\\
0.432000000000004	-0.00568904363910817\\
0.432000000000007	-0.00568904363910801\\
0.434999999999997	-0.00555855272545515\\
0.435	-0.005558552725455\\
0.43799999999999	-0.00543379990561904\\
0.439999999999997	-0.00535379191588978\\
0.44	-0.00535379191588964\\
0.44299999999999	-0.00523848156133828\\
0.445999999999979	-0.0051287681435457\\
0.445999999999995	-0.00512876814354515\\
0.446	-0.00512876814354496\\
0.447	-0.00509343233425499\\
0.447000000000004	-0.00509343233425487\\
0.448000000000004	-0.00505858406985652\\
0.449000000000004	-0.00502409443325188\\
0.451000000000004	-0.00495618430091737\\
0.454999999999997	-0.00482459437707987\\
0.455	-0.00482459437707975\\
0.459	-0.00469855938950006\\
0.459999999999997	-0.00466790692656755\\
0.46	-0.00466790692656744\\
0.463999999999997	-0.00454867714962721\\
0.464	-0.0045486771496271\\
0.465999999999997	-0.00449106991554863\\
0.466	-0.00449106991554853\\
0.467999999999996	-0.00443478597566644\\
0.469999999999993	-0.00437981429810855\\
0.471999999999997	-0.00432614410821348\\
0.472	-0.00432614410821338\\
0.473	-0.0042997379678403\\
0.473000000000004	-0.00429973796784021\\
0.474000000000004	-0.00427354165958215\\
0.475000000000004	-0.00424755389979012\\
0.477000000000004	-0.00419619894213898\\
0.479999999999997	-0.00412069911279833\\
0.48	-0.00412069911279824\\
0.484	-0.00402284195400596\\
0.485999999999997	-0.00397509572992924\\
0.486	-0.00397509572992915\\
0.489999999999997	-0.00388192119864539\\
0.49	-0.00388192119864531\\
0.492999999999997	-0.00381403259657121\\
0.493	-0.00381403259657113\\
0.495999999999997	-0.00374781834107435\\
0.498999999999993	-0.00368324923058202\\
0.499	-0.00368324923058187\\
0.499999999999997	-0.00366204063867869\\
0.5	-0.00366204063867862\\
0.500999999999998	-0.00364091823100526\\
0.501999999999997	-0.00361988097247265\\
0.503999999999993	-0.00357805778363198\\
0.505999999999993	-0.00353656287465056\\
0.506	-0.00353656287465042\\
0.507999999999993	-0.00349538811242098\\
0.508	-0.00349538811242084\\
0.509999999999993	-0.00345452542658466\\
0.511999999999986	-0.00341396680789743\\
0.515999999999972	-0.00333373003153357\\
0.519999999999993	-0.00325461487274754\\
0.52	-0.0032546148727474\\
0.521999999999993	-0.00321545848261091\\
0.522	-0.00321545848261077\\
0.523999999999993	-0.00317655930231894\\
0.524999999999993	-0.00315720378110358\\
0.525	-0.00315720378110344\\
0.525999999999993	-0.00313790970692414\\
0.526	-0.00313790970692401\\
0.526999999999998	-0.00311867613436781\\
0.527999999999997	-0.00309950212098513\\
0.529999999999993	-0.00306132901649075\\
0.531999999999993	-0.00302338291136547\\
0.532	-0.00302338291136533\\
0.535999999999993	-0.00294848395299221\\
0.538	-0.00291160190528123\\
0.538000000000007	-0.0029116019052811\\
0.539999999999993	-0.00287508848201567\\
0.54	-0.00287508848201555\\
0.541999999999986	-0.00283893652647114\\
0.543999999999972	-0.00280313895272453\\
0.545999999999993	-0.00276768874431281\\
0.546	-0.00276768874431269\\
0.549999999999972	-0.00269780269691938\\
0.550999999999993	-0.00268053751206524\\
0.551	-0.00268053751206511\\
0.554999999999972	-0.00261227669971175\\
0.556999999999993	-0.00257861455645162\\
0.557	-0.0025786145564515\\
0.559999999999993	-0.00252860911213904\\
0.56	-0.00252860911213893\\
0.562999999999993	-0.0024791062239712\\
0.565999999999986	-0.00243008406036295\\
0.565999999999993	-0.00243008406036284\\
0.566	-0.00243008406036272\\
0.571999999999986	-0.00233339563102444\\
0.571999999999993	-0.00233339563102433\\
0.572	-0.00233339563102421\\
0.577999999999986	-0.00223837324065878\\
0.579999999999993	-0.00220704028678049\\
0.58	-0.00220704028678038\\
0.585999999999986	-0.00211397896728288\\
0.585999999999993	-0.00211397896728277\\
0.586	-0.00211397896728266\\
0.591999999999986	-0.00202219657886334\\
0.591999999999993	-0.00202219657886323\\
0.592	-0.00202219657886312\\
0.594999999999993	-0.00197709382754081\\
0.595	-0.0019770938275407\\
0.597999999999993	-0.00193296945867172\\
0.599999999999993	-0.0019040871313196\\
0.6	-0.0019040871313195\\
0.602999999999993	-0.00186155036532712\\
0.605999999999986	-0.00181994102509126\\
0.606	-0.00181994102509107\\
0.606999999999993	-0.00180627415626971\\
0.607	-0.00180627415626962\\
0.607999999999999	-0.00179270762290004\\
0.608999999999997	-0.00177924076016245\\
0.609000000000004	-0.00177924076016235\\
0.611	-0.0017526034119188\\
0.612999999999997	-0.00172635689053332\\
0.614999999999997	-0.0017004960515865\\
0.615000000000004	-0.00170049605158641\\
0.618999999999997	-0.00164966916666219\\
0.619999999999993	-0.00163712001132626\\
0.62	-0.00163712001132617\\
0.623999999999993	-0.00158753520550816\\
0.625999999999993	-0.00156310139838846\\
0.626	-0.00156310139838837\\
0.629999999999993	-0.00151492702933171\\
0.63	-0.00151492702933163\\
0.633999999999993	-0.00146764558624051\\
0.635999999999993	-0.00144432810065704\\
0.636	-0.00144432810065695\\
0.637999999999993	-0.00142121999801122\\
0.638	-0.00142121999801114\\
0.639999999999993	-0.00139831674905593\\
0.64	-0.00139831674905585\\
0.641999999999993	-0.00137561386466581\\
0.643999999999986	-0.00135310689500291\\
0.645999999999993	-0.00133079142861429\\
0.646	-0.00133079142861422\\
0.649999999999986	-0.00128671754680051\\
0.65	-0.00128671754680036\\
0.650000000000007	-0.00128671754680028\\
0.653999999999993	-0.00124352608126631\\
0.657999999999979	-0.00120135158592858\\
0.659999999999993	-0.00118063533791182\\
0.66	-0.00118063533791175\\
0.664999999999993	-0.00112989427952784\\
0.665	-0.00112989427952777\\
0.665999999999993	-0.00111992200875428\\
0.666	-0.00111992200875421\\
0.666999999999998	-0.00111000723812208\\
0.667000000000006	-0.00111000723812201\\
0.668000000000004	-0.00110014948178418\\
0.669000000000002	-0.0010903482567086\\
0.670999999999998	-0.00107091348204279\\
0.673000000000005	-0.00105169910483379\\
0.673000000000013	-0.00105169910483372\\
0.677000000000005	-0.00101407582451287\\
0.678	-0.00100485171107809\\
0.678000000000007	-0.00100485171107802\\
0.679999999999993	-0.000986618355865464\\
0.68	-0.0009866183558654\\
0.681999999999986	-0.000968668523877228\\
0.683999999999972	-0.000950998696876605\\
0.686	-0.000933605411508941\\
0.686000000000007	-0.000933605411508879\\
0.689999999999979	-0.000899634882709027\\
0.69399999999995	-0.000866730302830468\\
0.695999999999993	-0.000850669649253334\\
0.696	-0.000850669649253277\\
0.699999999999993	-0.000819315875150975\\
0.7	-0.00081931587515092\\
0.703999999999993	-0.000788965076111516\\
0.705999999999993	-0.000774158325460929\\
0.706	-0.000774158325460877\\
0.707999999999993	-0.000759593455176628\\
0.708	-0.000759593455176577\\
0.709999999999993	-0.000745281733509318\\
0.711999999999986	-0.000731234478308297\\
0.713999999999993	-0.000717448936257202\\
0.714	-0.000717448936257153\\
0.717999999999986	-0.000690652234377396\\
0.719999999999993	-0.000677635822290894\\
0.72	-0.000677635822290849\\
0.723999999999986	-0.000652354119068287\\
0.724999999999993	-0.000646188365736588\\
0.725	-0.000646188365736545\\
0.725999999999993	-0.000640083872616837\\
0.726	-0.000640083872616794\\
0.726999999999999	-0.000634040340593207\\
0.727999999999997	-0.000628057473525747\\
0.729999999999993	-0.000616272564570708\\
0.731999999999993	-0.000604726835864012\\
0.732	-0.000604726835863972\\
0.734999999999993	-0.000587851930833831\\
0.735	-0.000587851930833791\\
0.737999999999993	-0.000571502983697621\\
0.74	-0.000560892314416705\\
0.740000000000007	-0.000560892314416668\\
0.743	-0.000545404054311648\\
0.745999999999993	-0.000530423031895794\\
0.746000000000007	-0.000530423031895725\\
0.746999999999993	-0.000525540928906708\\
0.747	-0.000525540928906674\\
0.747999999999999	-0.000520714210543487\\
0.748999999999997	-0.000515942640275208\\
0.750999999999993	-0.000506564011482599\\
0.753999999999993	-0.000492903921019896\\
0.754	-0.000492903921019864\\
0.757999999999993	-0.000475442476622148\\
0.759999999999993	-0.000467030182100566\\
0.76	-0.000467030182100537\\
0.763999999999993	-0.00045074052161536\\
0.766	-0.000442836543384021\\
0.766000000000007	-0.000442836543383993\\
0.77	-0.000427502544636809\\
0.770000000000007	-0.000427502544636782\\
0.774	-0.000412790522106836\\
0.776	-0.000405664128438062\\
0.776000000000007	-0.000405664128438037\\
0.779999999999993	-0.00039186359205892\\
0.78	-0.000391863592058896\\
0.782999999999993	-0.000381903598373912\\
0.783	-0.000381903598373889\\
0.785999999999993	-0.000372273346438775\\
0.786000000000001	-0.000372273346438752\\
0.788999999999994	-0.000362968589154366\\
0.791999999999987	-0.000353985222971942\\
0.792	-0.000353985222971901\\
0.792000000000008	-0.00035398522297188\\
0.797999999999994	-0.000336911043278193\\
0.799999999999993	-0.000331471725798996\\
0.8	-0.000331471725798977\\
0.804999999999993	-0.000318414712141896\\
0.805000000000001	-0.000318414712141878\\
0.805999999999993	-0.0003158950735722\\
0.806	-0.000315895073572183\\
0.806999999999994	-0.000313405731496958\\
0.807999999999987	-0.000310946563932357\\
0.809999999999973	-0.000306118271808874\\
0.811999999999993	-0.000301409250837741\\
0.812	-0.000301409250837724\\
0.815999999999973	-0.000292345353633015\\
0.817999999999993	-0.00028798870084009\\
0.818000000000001	-0.000287988700840075\\
0.819999999999993	-0.000283723658083559\\
0.82	-0.000283723658083544\\
0.821999999999993	-0.000279525281738471\\
0.823999999999986	-0.000275392748906758\\
0.825999999999993	-0.000271325249596005\\
0.826	-0.000271325249595991\\
0.829999999999986	-0.000263382175170138\\
0.833999999999972	-0.000255689830680818\\
0.839999999999993	-0.0002446082918977\\
0.84	-0.000244608291897687\\
0.840999999999993	-0.000242813514315119\\
0.841000000000001	-0.000242813514315106\\
0.841999999999994	-0.000241033398545753\\
0.842999999999987	-0.000239267857343117\\
0.844999999999973	-0.000235780153300426\\
0.845999999999993	-0.000234057819560533\\
0.846	-0.000234057819560521\\
0.849999999999973	-0.000227309978716701\\
0.851999999999993	-0.000224019799905309\\
0.852	-0.000224019799905297\\
0.855999999999973	-0.000217612731891902\\
0.857999999999993	-0.000214496844502343\\
0.858	-0.000214496844502332\\
0.86	-0.000211438570089865\\
0.860000000000007	-0.000211438570089854\\
0.862000000000007	-0.000208437309224881\\
0.864000000000007	-0.000205492473648578\\
0.866	-0.000202603486161819\\
0.866000000000007	-0.000202603486161809\\
0.87	-0.000196990801298907\\
0.870000000000007	-0.000196990801298898\\
0.874	-0.000191594854001679\\
0.874999999999994	-0.000190279232949771\\
0.875000000000001	-0.000190279232949762\\
0.876	-0.000188976828297866\\
0.876000000000007	-0.000188976828297856\\
0.877000000000007	-0.000187684412002462\\
0.878000000000006	-0.000186398756507051\\
0.879999999999998	-0.000183847476254879\\
0.880000000000006	-0.00018384747625487\\
0.882000000000005	-0.000181322487482216\\
0.884000000000004	-0.000178823295283206\\
0.886000000000005	-0.000176349409806542\\
0.886000000000013	-0.000176349409806533\\
0.888000000000007	-0.000173900346164411\\
0.888000000000014	-0.000173900346164402\\
0.890000000000009	-0.000171475624334087\\
0.892000000000004	-0.000169074769060716\\
0.895999999999993	-0.000164342780469229\\
0.898999999999993	-0.000160852971891248\\
0.899000000000001	-0.00016085297189124\\
0.899999999999993	-0.000159700670206876\\
0.9	-0.000159700670206867\\
0.900999999999994	-0.000158553758126382\\
0.901999999999987	-0.000157412179438012\\
0.903999999999973	-0.000155144798745878\\
0.905999999999993	-0.000152898083715084\\
0.906	-0.000152898083715076\\
0.909999999999973	-0.000148464893155957\\
0.909999999999987	-0.000148464893155942\\
0.910000000000001	-0.000148464893155927\\
0.910999999999993	-0.000147368828266066\\
0.911000000000001	-0.000147368828266059\\
0.911999999999994	-0.00014627788771615\\
0.912999999999987	-0.000145192357059511\\
0.914999999999973	-0.000143037312923647\\
0.916999999999993	-0.000140903273461481\\
0.917000000000001	-0.000140903273461474\\
0.919999999999993	-0.00013774068415686\\
0.92	-0.000137740684156853\\
0.922999999999993	-0.000134623020437661\\
0.925999999999986	-0.000131548907363069\\
0.925999999999993	-0.000131548907363062\\
0.926	-0.000131548907363055\\
0.927999999999993	-0.000129523023169475\\
0.928000000000001	-0.000129523023169468\\
0.929999999999994	-0.000127515494816387\\
0.931999999999987	-0.000125525928812399\\
0.933999999999994	-0.000123553935194896\\
0.934000000000001	-0.000123553935194889\\
0.937999999999987	-0.000119676323222335\\
0.939999999999993	-0.000117773744853171\\
0.940000000000001	-0.000117773744853165\\
0.943999999999987	-0.000114039178723854\\
0.944999999999994	-0.000113119976282754\\
0.945000000000001	-0.000113119976282747\\
0.945999999999993	-0.000112206458973796\\
0.946000000000001	-0.000112206458973789\\
0.946999999999994	-0.000111298582035277\\
0.947999999999987	-0.000110396300980206\\
0.949999999999973	-0.000108608349944615\\
0.952	-0.000106842255421689\\
0.952000000000008	-0.000106842255421683\\
0.95599999999998	-0.000103374255495249\\
0.956999999999994	-0.000102520380047639\\
0.957000000000001	-0.000102520380047633\\
0.96	-9.99895821087766e-05\\
0.960000000000008	-9.99895821087707e-05\\
0.963000000000007	-9.75041613143291e-05\\
0.966000000000007	-9.5063021552972e-05\\
0.966000000000014	-9.50630215529663e-05\\
0.969000000000007	-9.26650862343915e-05\\
0.969000000000014	-9.26650862343859e-05\\
0.972000000000007	-9.03149750231814e-05\\
0.975	-8.80173286505207e-05\\
0.979999999999994	-8.4301764671992e-05\\
0.980000000000001	-8.43017646719868e-05\\
0.985999999999987	-8.00253586065973e-05\\
0.986000000000001	-8.00253586065876e-05\\
0.991999999999987	-7.59406814815088e-05\\
0.992000000000001	-7.59406814814995e-05\\
0.997999999999987	-7.20486910896441e-05\\
0.998000000000001	-7.20486910896352e-05\\
0.999999999999993	-7.07947977279235e-05\\
1	-7.07947977279191e-05\\
1.00199999999999	-6.95622096195735e-05\\
1.00399999999999	-6.83506851673719e-05\\
1.00599999999999	-6.71599869078441e-05\\
1.006	-6.71599869078357e-05\\
1.00999999999999	-6.48401394844127e-05\\
1.01399999999997	-6.26008484692461e-05\\
1.01499999999999	-6.20534055317706e-05\\
1.015	-6.20534055317629e-05\\
1.01999999999999	-5.93891314832457e-05\\
1.02	-5.93891314832383e-05\\
1.02499999999999	-5.68440533010911e-05\\
1.02599999999999	-5.6349064251296e-05\\
1.026	-5.6349064251289e-05\\
1.02699999999999	-5.5858693604283e-05\\
1.027	-5.58586936042761e-05\\
1.02799999999999	-5.53728531834919e-05\\
1.02899999999999	-5.48914550322382e-05\\
1.03099999999997	-5.39418914004163e-05\\
1.03299999999999	-5.30098165906625e-05\\
1.033	-5.30098165906559e-05\\
1.03699999999997	-5.11974060740385e-05\\
1.04	-4.98826721685848e-05\\
1.04000000000001	-4.98826721685787e-05\\
1.04399999999999	-4.81880232058742e-05\\
1.044	-4.81880232058683e-05\\
1.046	-4.73653225225227e-05\\
1.04600000000001	-4.73653225225169e-05\\
1.04800000000001	-4.65588216338842e-05\\
1.05	-4.57683624626513e-05\\
1.05000000000001	-4.57683624626457e-05\\
1.05200000000001	-4.49937900762732e-05\\
1.05200000000002	-4.49937900762678e-05\\
1.05400000000002	-4.42348729009988e-05\\
1.05600000000002	-4.3491382431542e-05\\
1.05800000000001	-4.27631729408992e-05\\
1.05800000000002	-4.27631729408941e-05\\
1.05999999999999	-4.20501016981892e-05\\
1.06	-4.20501016981842e-05\\
1.06199999999996	-4.13520289396309e-05\\
1.06399999999992	-4.06688178402438e-05\\
1.06599999999999	-4.00003344879761e-05\\
1.066	-4.00003344879714e-05\\
1.06999999999992	-3.87070297887418e-05\\
1.07299999999999	-3.77747578454953e-05\\
1.073	-3.77747578454909e-05\\
1.07699999999992	-3.65812223238851e-05\\
1.07899999999999	-3.60054022074721e-05\\
1.079	-3.60054022074681e-05\\
1.07999999999999	-3.57221447570039e-05\\
1.08	-3.57221447569999e-05\\
1.08099999999999	-3.54412578530156e-05\\
1.08199999999999	-3.51627277317338e-05\\
1.08399999999997	-3.46126833599653e-05\\
1.08499999999999	-3.4341142156937e-05\\
1.085	-3.43411421569332e-05\\
1.08599999999999	-3.40719038307299e-05\\
1.086	-3.40719038307261e-05\\
1.08699999999999	-3.38049551888542e-05\\
1.08799999999999	-3.35402831505268e-05\\
1.08999999999997	-3.30177171194932e-05\\
1.09199999999999	-3.25041033126607e-05\\
1.092	-3.25041033126571e-05\\
1.09599999999997	-3.15033314284924e-05\\
1.09999999999995	-3.05371828412418e-05\\
1.09999999999997	-3.05371828412353e-05\\
1.1	-3.05371828412287e-05\\
1.10199999999999	-3.00668545108411e-05\\
1.102	-3.00668545108378e-05\\
1.10399999999999	-2.9604900033819e-05\\
1.10599999999997	-2.91512288582763e-05\\
1.10599999999999	-2.9151228858273e-05\\
1.106	-2.91512288582698e-05\\
1.10999999999997	-2.82683823337324e-05\\
1.11199999999999	-2.78390339436935e-05\\
1.112	-2.78390339436904e-05\\
1.11599999999997	-2.70045342993367e-05\\
1.11999999999994	-2.62020670142234e-05\\
1.12	-2.62020670142121e-05\\
1.12000000000001	-2.62020670142093e-05\\
1.126	-2.505705718079e-05\\
1.12600000000001	-2.50570571807874e-05\\
1.13099999999999	-2.41553998898001e-05\\
1.131	-2.41553998897976e-05\\
1.13599999999997	-2.33003044539298e-05\\
1.13699999999999	-2.31347796109702e-05\\
1.137	-2.31347796109679e-05\\
1.138	-2.29706780724201e-05\\
1.13800000000001	-2.29706780724178e-05\\
1.13900000000001	-2.28076028209267e-05\\
1.14	-2.26455458654222e-05\\
1.14000000000001	-2.26455458654199e-05\\
1.14100000000001	-2.2484499265074e-05\\
1.14200000000001	-2.23244551285734e-05\\
1.14400000000001	-2.20073429270659e-05\\
1.146	-2.16941471056864e-05\\
1.14600000000001	-2.16941471056842e-05\\
1.15000000000001	-2.10792598103952e-05\\
1.15400000000001	-2.04793111393561e-05\\
1.15499999999999	-2.0331602754049e-05\\
1.155	-2.03316027540469e-05\\
1.15999999999999	-1.96063721793395e-05\\
1.16	-1.96063721793375e-05\\
1.16499999999999	-1.89026795963069e-05\\
1.16599999999999	-1.87644493195945e-05\\
1.166	-1.87644493195926e-05\\
1.17099999999999	-1.80854665808059e-05\\
1.17199999999999	-1.79520571834687e-05\\
1.172	-1.79520571834668e-05\\
1.173	-1.78196115072008e-05\\
1.17300000000001	-1.7819611507199e-05\\
1.17400000000001	-1.76883064102212e-05\\
1.17500000000001	-1.75581354582677e-05\\
1.17700000000001	-1.73011705310835e-05\\
1.17999999999999	-1.69240715573422e-05\\
1.18	-1.69240715573404e-05\\
1.184	-1.6436599074425e-05\\
1.18599999999999	-1.61993234522913e-05\\
1.186	-1.61993234522896e-05\\
1.18899999999999	-1.5851354695347e-05\\
1.189	-1.58513546953454e-05\\
1.18999999999999	-1.57374609041885e-05\\
1.19	-1.57374609041868e-05\\
1.19099999999999	-1.56246056662662e-05\\
1.19199999999999	-1.55127834516205e-05\\
1.19399999999997	-1.52922162252599e-05\\
1.19499999999999	-1.51834604056034e-05\\
1.195	-1.51834604056019e-05\\
1.19899999999997	-1.47547700599766e-05\\
1.19999999999999	-1.4648921426791e-05\\
1.2	-1.46489214267895e-05\\
1.20399999999997	-1.42306671027404e-05\\
1.20599999999999	-1.40245524215676e-05\\
1.206	-1.40245524215661e-05\\
1.20699999999999	-1.39222304626545e-05\\
1.207	-1.39222304626531e-05\\
1.20799999999999	-1.38203920652582e-05\\
1.20899999999999	-1.37190322391849e-05\\
1.21099999999997	-1.35177284576024e-05\\
1.21499999999995	-1.31206467589369e-05\\
1.21799999999999	-1.28275175591633e-05\\
1.218	-1.2827517559162e-05\\
1.21999999999999	-1.26342556732136e-05\\
1.22	-1.26342556732123e-05\\
1.22199999999999	-1.24426756123699e-05\\
1.22399999999997	-1.22527398234283e-05\\
1.22499999999999	-1.2158376884686e-05\\
1.225	-1.21583768846847e-05\\
1.22599999999999	-1.20644110781907e-05\\
1.226	-1.20644110781894e-05\\
1.22699999999999	-1.19708377996829e-05\\
1.22799999999999	-1.18776524639688e-05\\
1.22999999999997	-1.16924273753032e-05\\
1.23	-1.16924273753007e-05\\
1.23399999999997	-1.13273556843818e-05\\
1.236	-1.11476682242628e-05\\
1.23600000000001	-1.11476682242615e-05\\
1.23999999999999	-1.07938139717061e-05\\
1.24	-1.07938139717049e-05\\
1.24399999999997	-1.04470900073049e-05\\
1.24599999999999	-1.02763167176051e-05\\
1.246	-1.02763167176038e-05\\
1.247	-1.01915625468847e-05\\
1.24700000000001	-1.01915625468835e-05\\
1.24800000000001	-1.01072244815172e-05\\
1.24900000000001	-1.00232983888436e-05\\
1.25100000000001	-9.85666569197846e-06\\
1.253	-9.69163179563864e-06\\
1.25300000000001	-9.69163179563747e-06\\
1.25700000000001	-9.3676450052173e-06\\
1.25999999999999	-9.13046943887784e-06\\
1.26	-9.13046943887673e-06\\
1.264	-8.82178918179261e-06\\
1.266	-8.67061822590467e-06\\
1.26600000000001	-8.6706182259036e-06\\
1.27000000000001	-8.37446650407949e-06\\
1.272	-8.22942769125908e-06\\
1.27200000000001	-8.22942769125805e-06\\
1.276	-7.94528201625836e-06\\
1.27600000000001	-7.94528201625736e-06\\
1.27999999999999	-7.66886059805219e-06\\
1.28000000000001	-7.66886059805122e-06\\
1.28399999999999	-7.39994670808966e-06\\
1.28599999999999	-7.26823897252217e-06\\
1.28600000000001	-7.26823897252124e-06\\
1.288	-7.13832950393264e-06\\
1.28800000000001	-7.13832950393173e-06\\
1.29	-7.01037836194254e-06\\
1.29199999999999	-6.8845459898406e-06\\
1.29499999999999	-6.69971628604371e-06\\
1.295	-6.69971628604284e-06\\
1.29899999999998	-6.4604656111749e-06\\
1.29999999999999	-6.4019170575923e-06\\
1.3	-6.40191705759147e-06\\
1.30399999999998	-6.17269323372248e-06\\
1.30499999999999	-6.11661583531408e-06\\
1.305	-6.11661583531329e-06\\
1.30599999999999	-6.06102432834729e-06\\
1.306	-6.06102432834651e-06\\
1.30699999999999	-6.00591598887775e-06\\
1.30799999999999	-5.95128811653486e-06\\
1.30999999999997	-5.84346308953184e-06\\
1.31199999999999	-5.7375281132148e-06\\
1.312	-5.73752811321405e-06\\
1.31299999999999	-5.68528172104826e-06\\
1.313	-5.68528172104752e-06\\
1.31399999999999	-5.63353774481781e-06\\
1.31499999999999	-5.58229364901029e-06\\
1.31699999999997	-5.48129507914927e-06\\
1.32	-5.333484342734e-06\\
1.32000000000001	-5.33348434273331e-06\\
1.32399999999999	-5.14318508225873e-06\\
1.326	-5.05089983583614e-06\\
1.32600000000001	-5.05089983583549e-06\\
1.32999999999999	-4.87196766631394e-06\\
1.33	-4.87196766631331e-06\\
1.33399999999997	-4.70043696349007e-06\\
1.334	-4.70043696348899e-06\\
1.33799999999997	-4.5361732385068e-06\\
1.34	-4.45672602563384e-06\\
1.34000000000001	-4.45672602563328e-06\\
1.34399999999999	-4.30251287421099e-06\\
1.346	-4.22756417611948e-06\\
1.34600000000001	-4.22756417611896e-06\\
1.348	-4.15403438256845e-06\\
1.34800000000002	-4.15403438256793e-06\\
1.35	-4.08190908153086e-06\\
1.35199999999999	-4.0111741361671e-06\\
1.35599999999997	-3.87382012515426e-06\\
1.35999999999999	-3.74186465546198e-06\\
1.36	-3.74186465546152e-06\\
1.36299999999999	-3.64637845226643e-06\\
1.363	-3.64637845226598e-06\\
1.36499999999999	-3.5843548015264e-06\\
1.365	-3.58435480152596e-06\\
1.36599999999999	-3.55382853857513e-06\\
1.366	-3.5538285385747e-06\\
1.36699999999999	-3.52362398423309e-06\\
1.36799999999999	-3.4937396584704e-06\\
1.36999999999997	-3.43492585094958e-06\\
1.37199999999999	-3.3773755882781e-06\\
1.372	-3.3773755882777e-06\\
1.37599999999997	-3.26595316195702e-06\\
1.378	-3.21204224464791e-06\\
1.37800000000001	-3.21204224464753e-06\\
1.37999999999999	-3.15931735747209e-06\\
1.38	-3.15931735747172e-06\\
1.38199999999997	-3.10776816621658e-06\\
1.38399999999994	-3.05738456703884e-06\\
1.38599999999999	-3.00815668455678e-06\\
1.386	-3.00815668455643e-06\\
1.38999999999994	-2.91312969919946e-06\\
1.39199999999999	-2.86731197066199e-06\\
1.392	-2.86731197066167e-06\\
1.39599999999994	-2.77902313813485e-06\\
1.39799999999999	-2.73653472918714e-06\\
1.398	-2.73653472918684e-06\\
1.39999999999999	-2.69492703274025e-06\\
1.4	-2.69492703273996e-06\\
1.40199999999999	-2.6539797769404e-06\\
1.40399999999997	-2.61368493596654e-06\\
1.40599999999999	-2.57403461187361e-06\\
1.406	-2.57403461187333e-06\\
1.40999999999997	-2.49663655290851e-06\\
1.412	-2.45887364771557e-06\\
1.41200000000001	-2.4588736477153e-06\\
1.41599999999999	-2.38518307631144e-06\\
1.41999999999996	-2.31389154170394e-06\\
1.41999999999998	-2.31389154170356e-06\\
1.42	-2.31389154170319e-06\\
1.42099999999999	-2.2964369163176e-06\\
1.421	-2.29643691631735e-06\\
1.42199999999999	-2.27912787316759e-06\\
1.42299999999999	-2.26196356390797e-06\\
1.42499999999997	-2.22806578989156e-06\\
1.42599999999999	-2.21133066412117e-06\\
1.426	-2.21133066412093e-06\\
1.42999999999997	-2.14579618269136e-06\\
1.43199999999999	-2.11386133961809e-06\\
1.432	-2.11386133961787e-06\\
1.43499999999999	-2.06704218373801e-06\\
1.435	-2.06704218373779e-06\\
1.43799999999998	-2.02155265098667e-06\\
1.43999999999999	-1.99195504449106e-06\\
1.44	-1.99195504449085e-06\\
1.44299999999999	-1.94863725445236e-06\\
1.44599999999997	-1.9065968690407e-06\\
1.44599999999999	-1.90659686904048e-06\\
1.446	-1.90659686904027e-06\\
1.44699999999999	-1.89286405427437e-06\\
1.447	-1.89286405427417e-06\\
1.44799999999999	-1.87927043587937e-06\\
1.44899999999999	-1.86581534770759e-06\\
1.44999999999999	-1.8524981304562e-06\\
1.45	-1.85249813045601e-06\\
1.45199999999999	-1.82627470525601e-06\\
1.45399999999997	-1.80059502040327e-06\\
1.45599999999999	-1.77545404258007e-06\\
1.456	-1.77545404257989e-06\\
1.45999999999997	-1.72626273934275e-06\\
1.46	-1.72626273934241e-06\\
1.46000000000001	-1.72626273934223e-06\\
1.46399999999999	-1.67815636448379e-06\\
1.466	-1.65449820593373e-06\\
1.46600000000001	-1.65449820593357e-06\\
1.46999999999999	-1.60794876037299e-06\\
1.47	-1.60794876037282e-06\\
1.47399999999997	-1.56239147897107e-06\\
1.47599999999999	-1.53997370793431e-06\\
1.476	-1.53997370793415e-06\\
1.47899999999998	-1.50678576380868e-06\\
1.479	-1.50678576380852e-06\\
1.47999999999999	-1.49583793460443e-06\\
1.48	-1.49583793460428e-06\\
1.48099999999999	-1.4849466184123e-06\\
1.48199999999999	-1.4741112815527e-06\\
1.48399999999997	-1.45260642481291e-06\\
1.48599999999999	-1.43131914934989e-06\\
1.486	-1.43131914934974e-06\\
1.48999999999997	-1.38938069465798e-06\\
1.491	-1.37902560040018e-06\\
1.49100000000001	-1.37902560040003e-06\\
1.49499999999999	-1.33813803681195e-06\\
1.49899999999996	-1.2980910088061e-06\\
1.49999999999999	-1.28820689169125e-06\\
1.5	-1.28820689169111e-06\\
1.50499999999999	-1.23952795957562e-06\\
1.505	-1.23952795957548e-06\\
1.50599999999999	-1.22993716448838e-06\\
1.506	-1.22993716448824e-06\\
1.50699999999999	-1.2203935978135e-06\\
1.50799999999999	-1.21089679189402e-06\\
1.508	-1.21089679189388e-06\\
1.50999999999999	-1.19204160320765e-06\\
1.51199999999997	-1.17336790275205e-06\\
1.51399999999999	-1.15487203040965e-06\\
1.514	-1.15487203040952e-06\\
1.51799999999997	-1.11852615852831e-06\\
1.518	-1.11852615852807e-06\\
1.51999999999999	-1.10070074731822e-06\\
1.52	-1.1007007473181e-06\\
1.52199999999999	-1.08310234577188e-06\\
1.52399999999997	-1.06572750453432e-06\\
1.52599999999999	-1.04857281806936e-06\\
1.526	-1.04857281806924e-06\\
1.52999999999997	-1.01491050251353e-06\\
1.53399999999994	-9.82089006112928e-07\\
1.537	-9.58009166376067e-07\\
1.53700000000001	-9.58009166375954e-07\\
1.53999999999999	-9.34377179497663e-07\\
1.54	-9.34377179497552e-07\\
1.54299999999997	-9.11182623717409e-07\\
1.54599999999994	-8.88415269517888e-07\\
1.54599999999997	-8.8841526951768e-07\\
1.546	-8.88415269517469e-07\\
1.549	-8.66065076050485e-07\\
1.54900000000001	-8.6606507605038e-07\\
1.55200000000001	-8.44169805071282e-07\\
1.55500000000001	-8.2276741873344e-07\\
1.55500000000003	-8.2276741873334e-07\\
1.55999999999999	-7.88166534418296e-07\\
1.56	-7.88166534418199e-07\\
1.56499999999996	-7.5486625256255e-07\\
1.56599999999998	-7.48358650859211e-07\\
1.566	-7.48358650859118e-07\\
1.57099999999996	-7.16565376563523e-07\\
1.57199999999998	-7.10353487514413e-07\\
1.572	-7.10353487514325e-07\\
1.57499999999999	-6.92022111635327e-07\\
1.575	-6.92022111635241e-07\\
1.57799999999999	-6.74149041057575e-07\\
1.57999999999999	-6.6248437400273e-07\\
1.58	-6.62484374002648e-07\\
1.58299999999999	-6.45357727541423e-07\\
1.58599999999997	-6.28668806610422e-07\\
1.586	-6.2866880661027e-07\\
1.58699999999999	-6.23201827672381e-07\\
1.587	-6.23201827672304e-07\\
1.58799999999999	-6.17782397695095e-07\\
1.58899999999999	-6.12410251102481e-07\\
1.59099999999997	-6.01806757430007e-07\\
1.59499999999995	-5.81155741959902e-07\\
1.59499999999997	-5.81155741959766e-07\\
1.595	-5.81155741959629e-07\\
1.59999999999999	-5.5636368732e-07\\
1.6	-5.56363687319931e-07\\
1.60499999999999	-5.32678587721969e-07\\
1.60599999999999	-5.28071825399014e-07\\
1.606	-5.28071825398949e-07\\
1.60699999999998	-5.23507950624845e-07\\
1.607	-5.23507950624781e-07\\
1.60799999999999	-5.18986046486511e-07\\
1.60899999999999	-5.14505198108044e-07\\
1.60999999999998	-5.10065185926705e-07\\
1.61	-5.10065185926642e-07\\
1.61199999999999	-5.01306801900782e-07\\
1.61399999999997	-4.92709177737343e-07\\
1.61599999999999	-4.84270628268706e-07\\
1.616	-4.84270628268647e-07\\
1.61999999999997	-4.67864168369017e-07\\
1.61999999999999	-4.67864168368959e-07\\
1.62	-4.67864168368901e-07\\
1.62399999999997	-4.52074558690693e-07\\
1.624	-4.52074558690597e-07\\
1.62599999999999	-4.44407185347908e-07\\
1.626	-4.44407185347854e-07\\
1.62799999999999	-4.36889419365828e-07\\
1.62999999999998	-4.29519787233023e-07\\
1.63199999999999	-4.22296844473021e-07\\
1.632	-4.22296844472971e-07\\
1.63599999999998	-4.08282867872059e-07\\
1.63999999999995	-3.94833977063589e-07\\
1.63999999999998	-3.9483397706351e-07\\
1.64	-3.94833977063431e-07\\
1.645	-3.7880149278047e-07\\
1.64500000000001	-3.78801492780426e-07\\
1.64599999999999	-3.75697229345322e-07\\
1.646	-3.75697229345278e-07\\
1.64699999999999	-3.72626684913019e-07\\
1.64799999999999	-3.6958970901955e-07\\
1.64999999999997	-3.63615869236855e-07\\
1.65199999999999	-3.57774539101445e-07\\
1.652	-3.57774539101404e-07\\
1.65299999999998	-3.54903206356646e-07\\
1.653	-3.54903206356605e-07\\
1.65399999999999	-3.5206457369935e-07\\
1.65499999999997	-3.49258502033321e-07\\
1.65699999999995	-3.43743493271411e-07\\
1.65899999999998	-3.38357099107699e-07\\
1.659	-3.38357099107662e-07\\
1.65999999999999	-3.35706895155976e-07\\
1.66	-3.35706895155938e-07\\
1.66099999999999	-3.33078637780421e-07\\
1.66199999999998	-3.30472198193459e-07\\
1.66399999999995	-3.25324262583774e-07\\
1.66599999999999	-3.20262079311505e-07\\
1.666	-3.20262079311469e-07\\
1.66999999999995	-3.10391017593686e-07\\
1.67399999999991	-3.00851273618371e-07\\
1.67999999999998	-2.8714658623336e-07\\
1.68	-2.87146586233329e-07\\
1.68199999999998	-2.8273607409928e-07\\
1.682	-2.82736074099249e-07\\
1.68399999999998	-2.78402966953575e-07\\
1.68599999999997	-2.74146415424678e-07\\
1.68599999999999	-2.74146415424646e-07\\
1.686	-2.74146415424616e-07\\
1.68999999999997	-2.65859656867926e-07\\
1.69199999999999	-2.6182782560612e-07\\
1.692	-2.61827825606092e-07\\
1.69599999999997	-2.53989063491912e-07\\
1.69799999999999	-2.50182035085901e-07\\
1.698	-2.50182035085874e-07\\
1.69999999999999	-2.46448908652388e-07\\
1.7	-2.46448908652361e-07\\
1.70199999999999	-2.4278895248907e-07\\
1.70399999999998	-2.39201449230298e-07\\
1.70599999999999	-2.35685695711517e-07\\
1.706	-2.35685695711493e-07\\
1.70999999999998	-2.28866695438398e-07\\
1.71099999999998	-2.27205729073886e-07\\
1.711	-2.27205729073862e-07\\
1.71499999999997	-2.20734556705414e-07\\
1.715	-2.20734556705374e-07\\
1.71699999999998	-2.17601479500977e-07\\
1.717	-2.17601479500955e-07\\
1.71899999999998	-2.14521553882354e-07\\
1.71999999999999	-2.12995947288766e-07\\
1.72	-2.12995947288744e-07\\
1.72199999999999	-2.09973072754497e-07\\
1.72399999999997	-2.06987490116208e-07\\
1.72599999999999	-2.04038614188135e-07\\
1.726	-2.04038614188114e-07\\
1.72899999999998	-1.99682862971758e-07\\
1.729	-1.99682862971738e-07\\
1.73199999999998	-1.9540648208593e-07\\
1.73499999999997	-1.91207585576567e-07\\
1.73999999999998	-1.84376597821633e-07\\
1.74	-1.84376597821614e-07\\
1.74599999999997	-1.76444984375117e-07\\
1.74599999999998	-1.76444984375096e-07\\
1.746	-1.76444984375074e-07\\
1.74999999999998	-1.7131170700702e-07\\
1.75	-1.71311707007002e-07\\
1.75199999999998	-1.68789889596657e-07\\
1.752	-1.68789889596639e-07\\
1.75399999999998	-1.66304323092843e-07\\
1.75599999999997	-1.63861551359575e-07\\
1.75799999999998	-1.61461095604116e-07\\
1.758	-1.61461095604099e-07\\
1.75999999999999	-1.59102485331164e-07\\
1.76	-1.59102485331148e-07\\
1.76199999999999	-1.56785258247086e-07\\
1.76399999999998	-1.54508960166368e-07\\
1.76599999999999	-1.52273144925767e-07\\
1.766	-1.52273144925751e-07\\
1.76899999999998	-1.48994370921664e-07\\
1.769	-1.48994370921648e-07\\
1.77199999999998	-1.45804253148329e-07\\
1.77499999999996	-1.42701384711502e-07\\
1.77499999999998	-1.42701384711485e-07\\
1.775	-1.42701384711467e-07\\
1.78	-1.37665426175532e-07\\
1.78000000000002	-1.37665426175517e-07\\
1.78499999999998	-1.3275247359739e-07\\
1.785	-1.32752473597376e-07\\
1.78600000000001	-1.31784111555373e-07\\
1.78600000000003	-1.31784111555359e-07\\
1.78700000000005	-1.308203810859e-07\\
1.78800000000006	-1.29861234963934e-07\\
1.79000000000009	-1.27956507991403e-07\\
1.79200000000003	-1.26069557304039e-07\\
1.79200000000004	-1.26069557304025e-07\\
1.79600000000011	-1.22347508808765e-07\\
1.79799999999998	-1.20511681465223e-07\\
1.798	-1.2051168146521e-07\\
1.8	-1.1869217119805e-07\\
1.80000000000002	-1.18692171198038e-07\\
1.80200000000002	-1.16888621378719e-07\\
1.80400000000002	-1.15100678504502e-07\\
1.806	-1.1332799213172e-07\\
1.80600000000002	-1.13327992131707e-07\\
1.80999999999998	-1.09827002009612e-07\\
1.81	-1.09827002009599e-07\\
1.81399999999997	-1.06391543839114e-07\\
1.81799999999994	-1.0302756182855e-07\\
1.82	-1.01371547996724e-07\\
1.82000000000001	-1.01371547996712e-07\\
1.826	-9.65035300846211e-08\\
1.82600000000001	-9.65035300846097e-08\\
1.827	-9.570636256889e-08\\
1.82700000000001	-9.57063625688787e-08\\
1.828	-9.4913138466805e-08\\
1.82899999999999	-9.41238189069571e-08\\
1.83099999999996	-9.25567388961018e-08\\
1.833	-9.10048153825517e-08\\
1.83300000000001	-9.10048153825407e-08\\
1.83699999999996	-8.79584212989152e-08\\
1.838	-8.72099487175452e-08\\
1.83800000000001	-8.72099487175346e-08\\
1.83999999999999	-8.57284970573063e-08\\
1.84	-8.57284970572959e-08\\
1.84199999999998	-8.4267461847356e-08\\
1.84399999999995	-8.28265567191319e-08\\
1.84599999999999	-8.14054992496207e-08\\
1.846	-8.14054992496107e-08\\
1.84999999999995	-7.8621817000051e-08\\
1.8539999999999	-7.59142325481688e-08\\
1.85499999999998	-7.52489766459447e-08\\
1.855	-7.52489766459353e-08\\
1.85599999999998	-7.45883112950838e-08\\
1.856	-7.45883112950744e-08\\
1.85699999999999	-7.39322041301608e-08\\
1.85799999999997	-7.32806229943321e-08\\
1.85999999999995	-7.19909113197179e-08\\
1.85999999999998	-7.19909113197016e-08\\
1.86	-7.19909113196851e-08\\
1.86399999999995	-6.9464410172314e-08\\
1.86599999999999	-6.82271254963768e-08\\
1.866	-6.8227125496368e-08\\
1.86799999999998	-6.70068269449261e-08\\
1.868	-6.70068269449175e-08\\
1.86999999999998	-6.58049953999871e-08\\
1.87199999999996	-6.46231153615943e-08\\
1.87299999999998	-6.40395846139472e-08\\
1.873	-6.40395846139389e-08\\
1.87699999999996	-6.17541923890512e-08\\
1.87999999999999	-6.00905349373025e-08\\
1.88	-6.00905349372948e-08\\
1.88399999999997	-5.79381123220036e-08\\
1.88499999999998	-5.74115778577305e-08\\
1.885	-5.7411577857723e-08\\
1.88599999999999	-5.68896199298675e-08\\
1.886	-5.68896199298601e-08\\
1.88699999999999	-5.63722129628449e-08\\
1.88799999999998	-5.58593316031364e-08\\
1.88999999999996	-5.48470454010066e-08\\
1.88999999999998	-5.48470454009962e-08\\
1.89	-5.48470454009858e-08\\
1.89199999999999	-5.38525629114445e-08\\
1.892	-5.38525629114375e-08\\
1.89399999999999	-5.28763777946159e-08\\
1.89599999999998	-5.19189872949678e-08\\
1.89799999999999	-5.09802037602512e-08\\
1.898	-5.09802037602446e-08\\
1.89999999999999	-5.00598431865559e-08\\
1.9	-5.00598431865495e-08\\
1.90199999999999	-4.91577251808642e-08\\
1.90399999999997	-4.82736729245355e-08\\
1.90599999999999	-4.74075131398962e-08\\
1.906	-4.74075131398902e-08\\
1.90999999999997	-4.57281953820331e-08\\
1.91399999999994	-4.41184552355144e-08\\
1.91399999999997	-4.41184552355037e-08\\
1.914	-4.4118455235493e-08\\
1.91999999999998	-4.18315568172844e-08\\
1.92	-4.18315568172792e-08\\
1.92499999999998	-4.00312896122251e-08\\
1.925	-4.00312896122201e-08\\
1.92599999999998	-3.9681347303424e-08\\
1.926	-3.96813473034191e-08\\
1.92699999999999	-3.93347349788759e-08\\
1.92799999999997	-3.89914356537493e-08\\
1.92999999999995	-3.83147088763167e-08\\
1.93199999999998	-3.76510343300422e-08\\
1.932	-3.76510343300375e-08\\
1.93599999999995	-3.63623241367501e-08\\
1.9399999999999	-3.51242946585884e-08\\
1.93999999999999	-3.51242946585623e-08\\
1.94	-3.5124294658558e-08\\
1.94299999999998	-3.42284463599272e-08\\
1.943	-3.4228446359923e-08\\
1.94599999999998	-3.33601640875806e-08\\
1.946	-3.33601640875746e-08\\
1.94899999999998	-3.25190649014224e-08\\
1.95199999999997	-3.17047778630876e-08\\
1.95199999999998	-3.1704777863083e-08\\
1.952	-3.17047778630782e-08\\
1.95799999999997	-3.01537344030671e-08\\
1.95999999999998	-2.96590999767079e-08\\
1.96	-2.96590999767044e-08\\
1.96599999999996	-2.82409794610751e-08\\
1.96599999999998	-2.82409794610711e-08\\
1.966	-2.82409794610671e-08\\
1.97199999999996	-2.69196096332222e-08\\
1.97199999999998	-2.69196096332185e-08\\
1.972	-2.69196096332148e-08\\
1.97799999999996	-2.56926592541164e-08\\
1.97799999999998	-2.56926592541129e-08\\
1.978	-2.56926592541095e-08\\
1.98	-2.53022877167275e-08\\
1.98000000000002	-2.53022877167247e-08\\
1.98200000000002	-2.49181062335062e-08\\
1.98400000000002	-2.45400395017882e-08\\
1.986	-2.41680134190381e-08\\
1.98600000000002	-2.41680134190354e-08\\
1.99000000000002	-2.34417926981472e-08\\
1.99400000000003	-2.27388746763394e-08\\
1.995	-2.25667212356414e-08\\
1.99500000000001	-2.2566721235639e-08\\
1.99999999999999	-2.17269895028307e-08\\
2	-2.17269895028283e-08\\
2.00099999999997	-2.15631926451752e-08\\
2.001	-2.15631926451706e-08\\
2.00199999999999	-2.14007600655816e-08\\
2.00299999999997	-2.12396838044754e-08\\
2.00499999999995	-2.09215687327087e-08\\
2.00599999999997	-2.07645143341947e-08\\
2.006	-2.07645143341902e-08\\
2.00999999999995	-2.01494721786688e-08\\
2.01199999999997	-1.98497511074405e-08\\
2.012	-1.98497511074363e-08\\
2.01299999999998	-1.97018778116177e-08\\
2.01300000000001	-1.97018778116135e-08\\
2.014	-1.95554038006506e-08\\
2.01499999999999	-1.94103218971492e-08\\
2.01699999999996	-1.91243060442274e-08\\
2.01999999999997	-1.8705547543061e-08\\
2.02	-1.87055475430571e-08\\
2.02399999999995	-1.81660788469888e-08\\
2.02599999999997	-1.79043149499689e-08\\
2.026	-1.79043149499652e-08\\
2.02999999999995	-1.73964715158294e-08\\
2.03	-1.73964715158239e-08\\
2.03399999999995	-1.69092104518742e-08\\
2.03599999999997	-1.66731782996926e-08\\
2.036	-1.66731782996893e-08\\
2.03999999999995	-1.62113389343112e-08\\
2.04	-1.62113389343052e-08\\
2.04399999999995	-1.57596717406334e-08\\
2.04599999999997	-1.55375417384341e-08\\
2.046	-1.5537541738431e-08\\
2.04799999999997	-1.53178225881549e-08\\
2.048	-1.53178225881518e-08\\
2.04999999999996	-1.5100471224279e-08\\
2.05199999999993	-1.48854450451002e-08\\
2.05599999999987	-1.44622001050315e-08\\
2.05899999999997	-1.41505597784908e-08\\
2.059	-1.41505597784879e-08\\
2.05999999999997	-1.40477559166967e-08\\
2.06	-1.40477559166937e-08\\
2.06099999999999	-1.39454817618426e-08\\
2.06199999999998	-1.38437323013294e-08\\
2.06399999999995	-1.3641787545807e-08\\
2.06499999999997	-1.35415823553696e-08\\
2.065	-1.35415823553667e-08\\
2.06599999999997	-1.34418820681208e-08\\
2.066	-1.3441882068118e-08\\
2.06699999999999	-1.33426817988148e-08\\
2.06799999999998	-1.32439766865307e-08\\
2.06999999999995	-1.30480326107648e-08\\
2.07099999999997	-1.29507840458923e-08\\
2.071	-1.29507840458895e-08\\
2.07499999999995	-1.25667888675737e-08\\
2.07699999999997	-1.23777679812538e-08\\
2.077	-1.23777679812511e-08\\
2.07999999999997	-1.20978531934828e-08\\
2.08	-1.20978531934801e-08\\
2.08299999999998	-1.18221686661598e-08\\
2.08599999999995	-1.15505928178903e-08\\
2.086	-1.1550592817886e-08\\
2.08799999999997	-1.13717656326952e-08\\
2.088	-1.13717656326927e-08\\
2.08999999999997	-1.11946762101953e-08\\
2.09199999999993	-1.10192898394639e-08\\
2.09399999999997	-1.08455721440888e-08\\
2.094	-1.08455721440863e-08\\
2.09799999999993	-1.05042011474244e-08\\
2.09999999999997	-1.03367794824425e-08\\
2.1	-1.03367794824401e-08\\
2.10399999999993	-1.00082997373021e-08\\
2.10599999999997	-9.84717727383331e-09\\
2.106	-9.84717727383104e-09\\
2.10999999999993	-9.53100925676381e-09\\
2.112	-9.37590173299745e-09\\
2.11200000000003	-9.37590173299526e-09\\
2.11599999999996	-9.0714876416268e-09\\
2.11699999999997	-8.99657086389484e-09\\
2.117	-8.99657086389271e-09\\
2.11999999999997	-8.7746098902634e-09\\
2.12	-8.77460989026132e-09\\
2.12299999999998	-8.55675705892501e-09\\
2.12599999999995	-8.34291629321021e-09\\
2.126	-8.3429162932068e-09\\
2.12899999999997	-8.13299328515979e-09\\
2.129	-8.13299328515782e-09\\
2.13199999999996	-7.92734331050279e-09\\
2.13499999999993	-7.72632352728159e-09\\
2.13499999999997	-7.72632352727935e-09\\
2.135	-7.72632352727712e-09\\
2.13999999999997	-7.40133999134691e-09\\
2.14	-7.4013399913451e-09\\
2.14499999999998	-7.08857451889077e-09\\
2.14599999999997	-7.02745361420819e-09\\
2.146	-7.02745361420646e-09\\
2.15099999999997	-6.72884530835135e-09\\
2.15199999999997	-6.67050240601952e-09\\
2.152	-6.67050240601787e-09\\
2.15299999999997	-6.61262934049074e-09\\
2.15299999999999	-6.6126293404891e-09\\
2.15399999999998	-6.55524024944396e-09\\
2.15499999999997	-6.49833232067931e-09\\
2.15699999999995	-6.38594881943226e-09\\
2.15999999999997	-6.22091333110968e-09\\
2.16	-6.22091333110813e-09\\
2.16399999999995	-6.00736083516805e-09\\
2.16599999999997	-5.90332089117342e-09\\
2.166	-5.90332089117196e-09\\
2.16999999999995	-5.70061164661789e-09\\
2.17	-5.70061164661556e-09\\
2.17399999999995	-5.50493139749837e-09\\
2.17499999999997	-5.45709151705487e-09\\
2.175	-5.45709151705351e-09\\
2.17899999999995	-5.26998239498095e-09\\
2.17999999999997	-5.22425625387275e-09\\
2.18	-5.22425625387146e-09\\
2.18399999999995	-5.04548902676957e-09\\
2.18599999999997	-4.95855722717335e-09\\
2.186	-4.95855722717213e-09\\
2.187	-4.91569677406088e-09\\
2.18700000000002	-4.91569677405967e-09\\
2.18800000000002	-4.87323061084445e-09\\
2.18800000000005	-4.87323061084325e-09\\
2.18900000000004	-4.83115012216696e-09\\
2.19000000000004	-4.78945324607722e-09\\
2.19200000000003	-4.70720217773331e-09\\
2.19600000000001	-4.54721474042109e-09\\
2.2	-4.3931428588351e-09\\
2.20000000000003	-4.39314285883403e-09\\
2.20399999999997	-4.24486573457977e-09\\
2.204	-4.24486573457874e-09\\
2.20499999999997	-4.20868816236106e-09\\
2.205	-4.20868816236003e-09\\
2.20599999999999	-4.17286370200183e-09\\
2.20600000000003	-4.17286370200033e-09\\
2.20700000000002	-4.13739059806238e-09\\
2.20800000000001	-4.1022671123555e-09\\
2.20999999999998	-4.03306212845063e-09\\
2.21200000000003	-3.96523518583935e-09\\
2.21200000000006	-3.9652351858384e-09\\
2.21600000000001	-3.83363827357977e-09\\
2.21800000000003	-3.76983645035347e-09\\
2.21800000000006	-3.76983645035257e-09\\
2.21999999999997	-3.70734895525977e-09\\
2.22	-3.70734895525889e-09\\
2.22199999999992	-3.64616354058833e-09\\
2.22399999999983	-3.58626821375995e-09\\
2.226	-3.5276512350512e-09\\
2.22600000000003	-3.52765123505038e-09\\
2.22999999999986	-3.41420661407492e-09\\
2.23299999999997	-3.33239516206389e-09\\
2.233	-3.33239516206313e-09\\
2.23699999999983	-3.22760731251082e-09\\
2.23899999999997	-3.1770305369125e-09\\
2.239	-3.17703053691179e-09\\
2.24	-3.15214595769945e-09\\
2.24000000000003	-3.15214595769874e-09\\
2.24100000000003	-3.12746748377689e-09\\
2.24200000000003	-3.10299390587212e-09\\
2.24400000000004	-3.05465665125412e-09\\
2.246	-3.00712471955739e-09\\
2.24600000000003	-3.00712471955672e-09\\
2.25000000000004	-2.91443971545888e-09\\
2.252	-2.86926847643928e-09\\
2.25200000000003	-2.86926847643864e-09\\
2.25600000000004	-2.78122425430432e-09\\
2.25999999999997	-2.69618709682006e-09\\
2.26	-2.69618709681947e-09\\
2.26199999999997	-2.65477524117238e-09\\
2.262	-2.65477524117179e-09\\
2.26399999999996	-2.61409033046825e-09\\
2.26599999999993	-2.57412439030736e-09\\
2.26599999999997	-2.57412439030666e-09\\
2.266	-2.57412439030596e-09\\
2.26999999999994	-2.49631822718897e-09\\
2.272	-2.45846275394815e-09\\
2.27200000000003	-2.45846275394762e-09\\
2.27499999999997	-2.40299836655627e-09\\
2.275	-2.40299836655575e-09\\
2.27799999999994	-2.34911905792272e-09\\
2.27999999999997	-2.31406835553695e-09\\
2.28	-2.31406835553646e-09\\
2.28299999999994	-2.26277838160786e-09\\
2.28599999999989	-2.21301164714088e-09\\
2.28599999999997	-2.21301164713945e-09\\
2.286	-2.21301164713898e-09\\
2.28699999999997	-2.19675742060556e-09\\
2.287	-2.1967574206051e-09\\
2.28799999999999	-2.18066920240034e-09\\
2.28899999999997	-2.16474620412977e-09\\
2.29099999999995	-2.13339275452833e-09\\
2.291	-2.13339275452761e-09\\
2.29499999999995	-2.07263470216078e-09\\
2.29699999999997	-2.04321819060171e-09\\
2.297	-2.04321819060129e-09\\
2.29999999999997	-1.99997676668582e-09\\
2.3	-1.99997676668541e-09\\
2.30299999999998	-1.95753560360859e-09\\
2.30599999999995	-1.91587598412145e-09\\
2.306	-1.91587598412081e-09\\
2.30999999999997	-1.86151385276234e-09\\
2.31	-1.86151385276196e-09\\
2.31399999999997	-1.80846588606633e-09\\
2.31599999999997	-1.78242167746164e-09\\
2.316	-1.78242167746127e-09\\
2.31999999999997	-1.73126728506668e-09\\
2.32	-1.73126728506632e-09\\
2.32399999999997	-1.68132493726019e-09\\
2.32599999999997	-1.65679600710987e-09\\
2.326	-1.65679600710953e-09\\
2.32999999999997	-1.60859859469415e-09\\
2.33199999999997	-1.58492066554941e-09\\
2.332	-1.58492066554908e-09\\
2.33599999999997	-1.53864721584874e-09\\
2.33999999999994	-1.49396290685296e-09\\
2.34	-1.49396290685225e-09\\
2.34000000000003	-1.49396290685194e-09\\
2.34499999999997	-1.44028899244637e-09\\
2.345	-1.44028899244607e-09\\
2.34600000000003	-1.42983977710715e-09\\
2.34600000000006	-1.42983977710685e-09\\
2.34700000000009	-1.41948454541527e-09\\
2.34800000000012	-1.40922278993983e-09\\
2.34899999999997	-1.39905400785384e-09\\
2.349	-1.39905400785355e-09\\
2.35100000000006	-1.37899337528491e-09\\
2.35300000000011	-1.35929871576225e-09\\
2.35499999999997	-1.33996616905825e-09\\
2.355	-1.33996616905798e-09\\
2.35800000000003	-1.31145318556236e-09\\
2.35800000000006	-1.3114531855621e-09\\
2.35999999999997	-1.29268083131363e-09\\
2.36	-1.29268083131337e-09\\
2.36199999999992	-1.27409322013992e-09\\
2.36399999999983	-1.25568670880826e-09\\
2.36599999999997	-1.23745768956927e-09\\
2.366	-1.23745768956901e-09\\
2.36999999999983	-1.20151786974349e-09\\
2.37399999999966	-1.16624558198903e-09\\
2.37799999999997	-1.13161317065069e-09\\
2.378	-1.13161317065044e-09\\
2.37999999999997	-1.11452841420626e-09\\
2.38	-1.11452841420602e-09\\
2.38199999999997	-1.09759348264727e-09\\
2.38399999999995	-1.08080505641673e-09\\
2.38599999999997	-1.06415984491784e-09\\
2.386	-1.06415984491761e-09\\
2.38999999999995	-1.03128604354357e-09\\
2.39	-1.03128604354319e-09\\
2.39399999999995	-9.99027452896263e-10\\
2.396	-9.83151367853786e-10\\
2.39600000000002	-9.83151367853561e-10\\
2.39999999999997	-9.51890058947727e-10\\
2.4	-9.51890058947507e-10\\
2.40399999999995	-9.21262859377403e-10\\
2.40599999999997	-9.06179527525357e-10\\
2.406	-9.06179527525144e-10\\
2.40699999999997	-8.98694130278034e-10\\
2.407	-8.98694130277822e-10\\
2.40799999999998	-8.9124575583242e-10\\
2.40899999999997	-8.83834039210157e-10\\
2.41099999999995	-8.69119128511101e-10\\
2.41299999999997	-8.54546514001179e-10\\
2.413	-8.54546514000972e-10\\
2.41499999999997	-8.40144325727746e-10\\
2.415	-8.40144325727543e-10\\
2.41699999999997	-8.25940727141565e-10\\
2.41899999999995	-8.11932934282373e-10\\
2.41999999999997	-8.05001605517956e-10\\
2.42	-8.0500160551776e-10\\
2.42399999999995	-7.77752177356834e-10\\
2.426	-7.64408330174143e-10\\
2.42600000000003	-7.64408330173954e-10\\
2.427	-7.57805473075876e-10\\
2.42700000000003	-7.57805473075689e-10\\
2.42800000000002	-7.51248228174591e-10\\
2.429	-7.44736274158145e-10\\
2.43099999999998	-7.31846964634642e-10\\
2.43499999999993	-7.06597943149877e-10\\
2.43599999999997	-7.00394213140809e-10\\
2.436	-7.00394213140633e-10\\
2.43999999999997	-6.76004274925558e-10\\
2.44	-6.76004274925388e-10\\
2.44399999999998	-6.52280040411578e-10\\
2.446	-6.40661729127687e-10\\
2.44600000000003	-6.40661729127523e-10\\
2.448	-6.2920290829421e-10\\
2.44800000000002	-6.29202908294048e-10\\
2.44999999999999	-6.1791748794341e-10\\
2.45000000000002	-6.17917487943251e-10\\
2.45199999999998	-6.06819412079602e-10\\
2.45399999999995	-5.95906505444543e-10\\
2.45600000000002	-5.85176629065746e-10\\
2.45600000000005	-5.85176629065595e-10\\
2.45999999999998	-5.64257590217451e-10\\
2.46000000000001	-5.64257590217305e-10\\
2.46399999999994	-5.44045893928809e-10\\
2.46499999999997	-5.39101623049226e-10\\
2.465	-5.39101623049086e-10\\
2.46599999999998	-5.34200325724678e-10\\
2.46600000000001	-5.3420032572454e-10\\
2.46699999999999	-5.29341761795003e-10\\
2.46799999999998	-5.24525693185172e-10\\
2.46999999999996	-5.15020100042042e-10\\
2.47199999999998	-5.05681683162023e-10\\
2.47200000000001	-5.05681683161891e-10\\
2.47599999999996	-4.87524963431531e-10\\
2.47999999999991	-4.70067172426743e-10\\
2.47999999999997	-4.70067172426467e-10\\
2.48	-4.70067172426346e-10\\
2.48499999999997	-4.49207000838113e-10\\
2.485	-4.49207000837998e-10\\
2.48599999999997	-4.45161182251697e-10\\
2.486	-4.45161182251583e-10\\
2.48699999999999	-4.41156968210199e-10\\
2.48799999999998	-4.37194162497634e-10\\
2.48999999999995	-4.29392001365374e-10\\
2.49199999999997	-4.21753169600935e-10\\
2.492	-4.21753169600827e-10\\
2.494	-4.14276169974353e-10\\
2.49400000000002	-4.14276169974248e-10\\
2.49600000000002	-4.06959536974705e-10\\
2.49800000000001	-3.99801836513459e-10\\
2.49999999999997	-3.92801665653694e-10\\
2.5	-3.92801665653595e-10\\
2.50399999999999	-3.79214213353523e-10\\
2.50599999999997	-3.7261070893629e-10\\
2.506	-3.72610708936197e-10\\
2.50999999999999	-3.59777671791511e-10\\
2.51399999999998	-3.47434919367978e-10\\
2.51999999999997	-3.29819008564371e-10\\
2.52	-3.2981900856429e-10\\
2.52299999999997	-3.21406813448657e-10\\
2.523	-3.21406813448579e-10\\
2.52599999999996	-3.132534735627e-10\\
2.526	-3.13253473562602e-10\\
2.52899999999997	-3.0535539302427e-10\\
2.53199999999993	-2.97709088651961e-10\\
2.53199999999997	-2.9770908865188e-10\\
2.532	-2.97709088651798e-10\\
2.53799999999993	-2.83144521197966e-10\\
2.53799999999997	-2.83144521197889e-10\\
2.538	-2.83144521197813e-10\\
2.53999999999997	-2.78499822431848e-10\\
2.54	-2.78499822431783e-10\\
2.54199999999998	-2.73958667841102e-10\\
2.54399999999995	-2.69520167322178e-10\\
2.54599999999997	-2.65183450911658e-10\\
2.546	-2.65183450911598e-10\\
2.54999999999995	-2.56811990173398e-10\\
2.55199999999997	-2.52775605006605e-10\\
2.552	-2.52775605006549e-10\\
2.55499999999997	-2.46905476792075e-10\\
2.555	-2.46905476792021e-10\\
2.55799999999998	-2.41254394060164e-10\\
2.55800000000001	-2.41254394060112e-10\\
2.55999999999997	-2.37588763523654e-10\\
2.56	-2.37588763523603e-10\\
2.56199999999997	-2.33981259811801e-10\\
2.56399999999994	-2.30431175841995e-10\\
2.56599999999997	-2.26937815783919e-10\\
2.566	-2.2693781578387e-10\\
2.56999999999994	-2.20118539560074e-10\\
2.56999999999997	-2.20118539560019e-10\\
2.57	-2.20118539559963e-10\\
2.57399999999994	-2.13518084475836e-10\\
2.57799999999988	-2.07131275436497e-10\\
2.57999999999997	-2.04016416830315e-10\\
2.58	-2.04016416830271e-10\\
2.58099999999997	-2.02478355246968e-10\\
2.581	-2.02478355246924e-10\\
2.58199999999998	-2.00953104792171e-10\\
2.58299999999997	-1.99440590710864e-10\\
2.58499999999995	-1.96453475835079e-10\\
2.58599999999997	-1.94978728669949e-10\\
2.586	-1.94978728669907e-10\\
2.58999999999995	-1.89203462789397e-10\\
2.59	-1.89203462789331e-10\\
2.59199999999997	-1.8638907515443e-10\\
2.592	-1.8638907515439e-10\\
2.59399999999997	-1.83625148761516e-10\\
2.59599999999995	-1.8091351158901e-10\\
2.59799999999997	-1.78253632145454e-10\\
2.598	-1.78253632145417e-10\\
2.59999999999997	-1.75644989087918e-10\\
2.6	-1.75644989087881e-10\\
2.60199999999997	-1.73087071115691e-10\\
2.60399999999995	-1.70579376866916e-10\\
2.60599999999997	-1.68121414823735e-10\\
2.606	-1.68121414823701e-10\\
2.60999999999995	-1.63352769942347e-10\\
2.61	-1.63352769942288e-10\\
2.61399999999994	-1.58777397607867e-10\\
2.61599999999997	-1.56561061763077e-10\\
2.616	-1.56561061763046e-10\\
2.61999999999994	-1.52224403568707e-10\\
2.61999999999997	-1.52224403568674e-10\\
2.62	-1.52224403568641e-10\\
2.62399999999995	-1.47983262659608e-10\\
2.62499999999997	-1.46937510718826e-10\\
2.625	-1.46937510718797e-10\\
2.62599999999997	-1.45897469183968e-10\\
2.626	-1.45897469183939e-10\\
2.62699999999999	-1.44863087093697e-10\\
2.62799999999998	-1.43834313762169e-10\\
2.62999999999995	-1.41793392007215e-10\\
2.63199999999997	-1.3977430389051e-10\\
2.632	-1.39774303890481e-10\\
2.63599999999995	-1.35800049789669e-10\\
2.63899999999997	-1.32873759498341e-10\\
2.639	-1.32873759498314e-10\\
2.63999999999997	-1.31908435437998e-10\\
2.64	-1.31908435437971e-10\\
2.64099999999999	-1.30948085356726e-10\\
2.64199999999998	-1.29992662196324e-10\\
2.64399999999995	-1.28096409613474e-10\\
2.64599999999997	-1.26219306016573e-10\\
2.646	-1.26219306016546e-10\\
2.64999999999995	-1.22521077794882e-10\\
2.65099999999997	-1.21607918423308e-10\\
2.651	-1.21607918423282e-10\\
2.65499999999995	-1.18002221972096e-10\\
2.6589999999999	-1.14470596266435e-10\\
2.65999999999997	-1.13598937650055e-10\\
2.66	-1.1359893765003e-10\\
2.66599999999997	-1.0846018701723e-10\\
2.666	-1.08460187017206e-10\\
2.66799999999997	-1.0678100744891e-10\\
2.668	-1.06781007448886e-10\\
2.66999999999996	-1.05118145273492e-10\\
2.67199999999993	-1.03471274556863e-10\\
2.674	-1.01840072505916e-10\\
2.67400000000002	-1.01840072505893e-10\\
2.67799999999996	-9.86346122859581e-11\\
2.67799999999999	-9.86346122859297e-11\\
2.67800000000003	-9.86346122859012e-11\\
2.67999999999997	-9.70625291353826e-11\\
2.68	-9.70625291353604e-11\\
2.68199999999995	-9.55104651203672e-11\\
2.68399999999989	-9.39781160303321e-11\\
2.68599999999997	-9.24651815188273e-11\\
2.686	-9.24651815188059e-11\\
2.68999999999989	-8.94963738240988e-11\\
2.69399999999978	-8.66017143377782e-11\\
2.69499999999997	-8.58893678996115e-11\\
2.695	-8.58893678995913e-11\\
2.69699999999997	-8.44780120371611e-11\\
2.697	-8.44780120371412e-11\\
2.69899999999996	-8.30842087250671e-11\\
2.69999999999997	-8.23938037625289e-11\\
2.7	-8.23938037625094e-11\\
2.70199999999997	-8.10258180730562e-11\\
2.70399999999993	-7.96747082713699e-11\\
2.70599999999997	-7.83402095346764e-11\\
2.706	-7.83402095346575e-11\\
2.70899999999997	-7.63690358812078e-11\\
2.709	-7.63690358811893e-11\\
2.71199999999996	-7.44379854343939e-11\\
2.71299999999997	-7.38040050030584e-11\\
2.713	-7.38040050030405e-11\\
2.71599999999997	-7.19307379400007e-11\\
2.71899999999993	-7.00998421219007e-11\\
2.71999999999997	-6.94988192540647e-11\\
2.72	-6.94988192540476e-11\\
2.72599999999993	-6.59880275171684e-11\\
2.726	-6.59880275171329e-11\\
2.72600000000002	-6.59880275171166e-11\\
2.72999999999997	-6.3736214360205e-11\\
2.73	-6.37362143601892e-11\\
2.73199999999999	-6.26362549594678e-11\\
2.73200000000002	-6.26362549594523e-11\\
2.73400000000002	-6.15539426691592e-11\\
2.73600000000001	-6.04897029300394e-11\\
2.73799999999999	-5.94433271470043e-11\\
2.73800000000002	-5.94433271469896e-11\\
2.73999999999997	-5.84146102277692e-11\\
2.74	-5.84146102277547e-11\\
2.74199999999995	-5.74033505411959e-11\\
2.7439999999999	-5.6409349876465e-11\\
2.74599999999997	-5.54324134055397e-11\\
2.746	-5.54324134055259e-11\\
2.7499999999999	-5.35289704250801e-11\\
2.7539999999998	-5.16915292023018e-11\\
2.75499999999997	-5.12423117193521e-11\\
2.755	-5.12423117193394e-11\\
2.75999999999997	-4.90559830973219e-11\\
2.76	-4.90559830973097e-11\\
2.76499999999998	-4.69673060585418e-11\\
2.76500000000001	-4.69673060585302e-11\\
2.76599999999997	-4.65610615858433e-11\\
2.766	-4.65610615858318e-11\\
2.76699999999999	-4.61586005422954e-11\\
2.76700000000002	-4.61586005422841e-11\\
2.76800000000001	-4.57598418596641e-11\\
2.76899999999999	-4.53647046520195e-11\\
2.77099999999997	-4.45852173909519e-11\\
2.77299999999999	-4.38199859765541e-11\\
2.77300000000002	-4.38199859765433e-11\\
2.77699999999997	-4.23316935039945e-11\\
2.77999999999997	-4.12518002455e-11\\
2.78	-4.125180024549e-11\\
2.78399999999995	-3.98594718766973e-11\\
2.784	-3.98594718766831e-11\\
2.786	-3.91833696361644e-11\\
2.78600000000003	-3.91833696361549e-11\\
2.78800000000004	-3.85204644638442e-11\\
2.79000000000004	-3.7870626427743e-11\\
2.792	-3.72337281571029e-11\\
2.79200000000003	-3.72337281570939e-11\\
2.79600000000004	-3.59980266101386e-11\\
2.79999999999997	-3.48121634897285e-11\\
2.8	-3.48121634897202e-11\\
2.80400000000001	-3.36752090183786e-11\\
2.80599999999997	-3.31247924750994e-11\\
2.806	-3.31247924750917e-11\\
2.81000000000001	-3.20595413366134e-11\\
2.81199999999997	-3.15444979480709e-11\\
2.812	-3.15444979480637e-11\\
2.81299999999997	-3.12913272580302e-11\\
2.813	-3.12913272580231e-11\\
2.81399999999998	-3.10410406498908e-11\\
2.81499999999997	-3.07936258593316e-11\\
2.81699999999995	-3.03073633775614e-11\\
2.81899999999997	-2.98324445033773e-11\\
2.819	-2.98324445033706e-11\\
2.81999999999997	-2.95987768408143e-11\\
2.82	-2.95987768408077e-11\\
2.82099999999999	-2.9367044519258e-11\\
2.82199999999998	-2.91372361835686e-11\\
2.82399999999995	-2.8683346520877e-11\\
2.82599999999997	-2.82370188885987e-11\\
2.826	-2.82370188885924e-11\\
2.82999999999995	-2.73667012552836e-11\\
2.83399999999991	-2.65256009097085e-11\\
2.83499999999997	-2.63198132451408e-11\\
2.835	-2.6319813245135e-11\\
2.83999999999997	-2.53172963516026e-11\\
2.84	-2.5317296351597e-11\\
2.84199999999997	-2.49284365997138e-11\\
2.842	-2.49284365997083e-11\\
2.84399999999996	-2.45464029050013e-11\\
2.84599999999992	-2.41711203868222e-11\\
2.846	-2.41711203868079e-11\\
2.84600000000003	-2.41711203868026e-11\\
2.84999999999996	-2.34405159625682e-11\\
2.852	-2.30850508554508e-11\\
2.85200000000003	-2.30850508554457e-11\\
2.853	-2.29097784263452e-11\\
2.85300000000003	-2.29097784263402e-11\\
2.85400000000002	-2.27361766550507e-11\\
2.85500000000001	-2.25642370348889e-11\\
2.85699999999998	-2.22253106286781e-11\\
2.85999999999997	-2.17291791508379e-11\\
2.86	-2.17291791508333e-11\\
2.86399999999995	-2.10902136703638e-11\\
2.86599999999997	-2.07802510346248e-11\\
2.866	-2.07802510346205e-11\\
2.86999999999995	-2.01790621927587e-11\\
2.87	-2.01790621927515e-11\\
2.87099999999997	-2.00326253336902e-11\\
2.871	-2.00326253336861e-11\\
2.87199999999998	-1.98877181513316e-11\\
2.87299999999997	-1.97443335450681e-11\\
2.87499999999995	-1.94621040315792e-11\\
2.87699999999997	-1.9185881475139e-11\\
2.87699999999999	-1.91858814751352e-11\\
2.87999999999997	-1.87798423171719e-11\\
2.88	-1.8779842317168e-11\\
2.88299999999998	-1.83813176899147e-11\\
2.88599999999996	-1.79901318374748e-11\\
2.886	-1.79901318374693e-11\\
2.888	-1.77333330394983e-11\\
2.88800000000003	-1.77333330394947e-11\\
2.89000000000003	-1.74796688467695e-11\\
2.89200000000003	-1.7229089539088e-11\\
2.89600000000002	-1.67369897164925e-11\\
2.89999999999997	-1.62566477350344e-11\\
2.9	-1.62566477350311e-11\\
2.90499999999997	-1.56721818402349e-11\\
2.905	-1.56721818402316e-11\\
2.90599999999997	-1.55573593362218e-11\\
2.90599999999999	-1.55573593362185e-11\\
2.90699999999998	-1.54432138507151e-11\\
2.90799999999997	-1.5329739790323e-11\\
2.90999999999995	-1.51047837364932e-11\\
2.91199999999997	-1.48824470824713e-11\\
2.91199999999999	-1.48824470824682e-11\\
2.91599999999995	-1.44479376816201e-11\\
2.91799999999997	-1.42362996181168e-11\\
2.91799999999999	-1.42362996181138e-11\\
2.91999999999997	-1.40283504266273e-11\\
2.92	-1.40283504266244e-11\\
2.92199999999998	-1.38240493485795e-11\\
2.92399999999996	-1.36233563401696e-11\\
2.926	-1.34262320647849e-11\\
2.92600000000003	-1.34262320647821e-11\\
2.92899999999997	-1.31371526975295e-11\\
2.92899999999999	-1.31371526975268e-11\\
2.93199999999993	-1.28558887201996e-11\\
2.93499999999987	-1.25823160907694e-11\\
2.93499999999997	-1.25823160907606e-11\\
2.935	-1.2582316090758e-11\\
2.93999999999997	-1.21383054085953e-11\\
2.94	-1.21383054085928e-11\\
2.94499999999998	-1.17051378374286e-11\\
2.94599999999997	-1.16197585696554e-11\\
2.946	-1.1619758569653e-11\\
2.95099999999998	-1.11989033510574e-11\\
2.95199999999997	-1.11159116605299e-11\\
2.952	-1.11159116605276e-11\\
2.95699999999998	-1.07066262967569e-11\\
2.95799999999999	-1.06258757646096e-11\\
2.95800000000002	-1.06258757646073e-11\\
2.95999999999997	-1.04654494788889e-11\\
2.96	-1.04654494788866e-11\\
2.96199999999995	-1.03064300573495e-11\\
2.9639999999999	-1.01487863313495e-11\\
2.96599999999997	-9.99248740210029e-12\\
2.966	-9.99248740209808e-12\\
2.9699999999999	-9.68380165162971e-12\\
2.96999999999999	-9.68380165162281e-12\\
2.97000000000002	-9.68380165162063e-12\\
2.97399999999992	-9.38089276145319e-12\\
2.97499999999997	-9.30615929125593e-12\\
2.975	-9.30615929125381e-12\\
2.9789999999999	-9.01109055694533e-12\\
2.97999999999997	-8.93827158689449e-12\\
2.98	-8.93827158689242e-12\\
2.9839999999999	-8.6506815123428e-12\\
2.986	-8.50904869864449e-12\\
2.98600000000003	-8.50904869864249e-12\\
2.98699999999997	-8.43876065658802e-12\\
2.98699999999999	-8.43876065658603e-12\\
2.98799999999998	-8.36882025989018e-12\\
2.98899999999997	-8.2992240813984e-12\\
2.99099999999995	-8.16105075482004e-12\\
2.99299999999999	-8.02421359434202e-12\\
2.99300000000002	-8.02421359434009e-12\\
2.99699999999997	-7.75560464782525e-12\\
2.99999999999997	-7.55898584961571e-12\\
3	-7.55898584961387e-12\\
3.00399999999995	-7.30311313697392e-12\\
3.00599999999997	-7.17781412614375e-12\\
3.006	-7.17781412614198e-12\\
3.00999999999995	-6.93236799595898e-12\\
3.01	-6.93236799595605e-12\\
3.01399999999995	-6.69363124445467e-12\\
3.01599999999997	-6.57672018985643e-12\\
3.01599999999999	-6.57672018985478e-12\\
3.01999999999995	-6.34769814408939e-12\\
3.02	-6.34769814408623e-12\\
3.02399999999995	-6.12492706589027e-12\\
3.02599999999997	-6.01583086939761e-12\\
3.026	-6.01583086939607e-12\\
3.02799999999997	-5.90823229118074e-12\\
3.02799999999999	-5.90823229117922e-12\\
3.02999999999996	-5.8022619464486e-12\\
3.03199999999992	-5.69805076940972e-12\\
3.03599999999985	-5.4948245563101e-12\\
3.03999999999997	-5.29839430992907e-12\\
3.04	-5.2983943099277e-12\\
3.04499999999997	-5.06217920525298e-12\\
3.04499999999999	-5.06217920525167e-12\\
3.04599999999997	-5.01615591103689e-12\\
3.046	-5.01615591103559e-12\\
3.04699999999998	-4.97053388410342e-12\\
3.04799999999996	-4.92531088886532e-12\\
3.04999999999992	-4.8360531491721e-12\\
3.05199999999997	-4.74836519709039e-12\\
3.052	-4.74836519708916e-12\\
3.05599999999992	-4.57787317217652e-12\\
3.05799999999997	-4.49509641932757e-12\\
3.058	-4.4950964193264e-12\\
3.05999999999997	-4.41394409972527e-12\\
3.06	-4.41394409972413e-12\\
3.06199999999997	-4.3344003073063e-12\\
3.06399999999995	-4.25644945117825e-12\\
3.066	-4.1800762526682e-12\\
3.06600000000003	-4.18007625266713e-12\\
3.06999999999997	-4.03200325750283e-12\\
3.07399999999992	-3.89006522494165e-12\\
3.07399999999996	-3.89006522494033e-12\\
3.07399999999999	-3.89006522493902e-12\\
3.07999999999997	-3.68841908442741e-12\\
3.07999999999999	-3.68841908442649e-12\\
3.08599999999997	-3.49882547894769e-12\\
3.08599999999999	-3.49882547894682e-12\\
3.09199999999997	-3.3198038282741e-12\\
3.09199999999999	-3.31980382827327e-12\\
3.09799999999997	-3.15103829179038e-12\\
3.09999999999997	-3.09701027590395e-12\\
3.1	-3.09701027590319e-12\\
3.10299999999997	-3.01801953502208e-12\\
3.10299999999999	-3.01801953502134e-12\\
3.10599999999996	-2.94145944854362e-12\\
3.106	-2.9414594485425e-12\\
3.10600000000003	-2.94145944854179e-12\\
3.10899999999999	-2.86729625194502e-12\\
3.11199999999996	-2.7954972380575e-12\\
3.11199999999999	-2.79549723805663e-12\\
3.11200000000003	-2.79549723805577e-12\\
3.11499999999997	-2.72599809685364e-12\\
3.115	-2.72599809685299e-12\\
3.11799999999995	-2.65873553332895e-12\\
3.11999999999997	-2.61512167829872e-12\\
3.12	-2.61512167829811e-12\\
3.12299999999995	-2.55152130524519e-12\\
3.12599999999989	-2.49008056252698e-12\\
3.12599999999995	-2.49008056252586e-12\\
3.126	-2.49008056252473e-12\\
3.127	-2.47007552219259e-12\\
3.12700000000003	-2.47007552219202e-12\\
3.12800000000003	-2.45030644217331e-12\\
3.12900000000003	-2.43077235369364e-12\\
3.13100000000003	-2.3924053341222e-12\\
3.13199999999997	-2.37357052302229e-12\\
3.13199999999999	-2.37357052302176e-12\\
3.13599999999999	-2.30053453391936e-12\\
3.13799999999997	-2.26538600604482e-12\\
3.13799999999999	-2.26538600604433e-12\\
3.13999999999997	-2.23096562398205e-12\\
3.14	-2.23096562398156e-12\\
3.14199999999998	-2.19709105438571e-12\\
3.14399999999996	-2.16375565770895e-12\\
3.14599999999997	-2.13095290008445e-12\\
3.146	-2.13095290008399e-12\\
3.14999999999996	-2.06691968746139e-12\\
3.15	-2.06691968746074e-12\\
3.15399999999996	-2.00494121117471e-12\\
3.15599999999997	-1.97470725153823e-12\\
3.156	-1.97470725153781e-12\\
3.15999999999996	-1.91572025843586e-12\\
3.16	-1.91572025843527e-12\\
3.16099999999997	-1.90127781252388e-12\\
3.16099999999999	-1.90127781252347e-12\\
3.16199999999996	-1.88695566329608e-12\\
3.16299999999992	-1.87275310895104e-12\\
3.16499999999985	-1.84470400702205e-12\\
3.16599999999997	-1.8308560850115e-12\\
3.166	-1.83085608501111e-12\\
3.16999999999986	-1.77662615712271e-12\\
3.17199999999997	-1.75019896850808e-12\\
3.172	-1.75019896850771e-12\\
3.17599999999986	-1.69878325375639e-12\\
3.17999999999972	-1.64931166116149e-12\\
3.17999999999997	-1.64931166115842e-12\\
3.18	-1.64931166115807e-12\\
3.18499999999997	-1.59014714295145e-12\\
3.185	-1.59014714295112e-12\\
3.18599999999997	-1.57866506083961e-12\\
3.186	-1.57866506083929e-12\\
3.18699999999997	-1.56729859313016e-12\\
3.18799999999994	-1.55604718283906e-12\\
3.18999999999989	-1.53388733483639e-12\\
3.18999999999994	-1.5338873348358e-12\\
3.18999999999999	-1.53388733483521e-12\\
3.19399999999988	-1.49092444414231e-12\\
3.19599999999999	-1.47011298057184e-12\\
3.19600000000002	-1.47011298057154e-12\\
3.198	-1.44963820941492e-12\\
3.19800000000003	-1.44963820941463e-12\\
3.19999999999998	-1.42939162335557e-12\\
3.20000000000001	-1.42939162335529e-12\\
3.20199999999996	-1.4093692539842e-12\\
3.20399999999991	-1.38956717685257e-12\\
3.20599999999998	-1.3699815106759e-12\\
3.20600000000001	-1.36998151067562e-12\\
3.20999999999991	-1.33144409750937e-12\\
3.21399999999982	-1.29372679916477e-12\\
3.21899999999997	-1.24768847019449e-12\\
3.21899999999999	-1.24768847019424e-12\\
3.21999999999997	-1.2386240472449e-12\\
3.22	-1.23862404724464e-12\\
3.22099999999998	-1.22960633023258e-12\\
3.22199999999996	-1.22063487718652e-12\\
3.22399999999992	-1.20282900682581e-12\\
3.22599999999997	-1.18520294614256e-12\\
3.226	-1.18520294614231e-12\\
3.22999999999992	-1.15047646937924e-12\\
3.23099999999999	-1.141901874859e-12\\
3.23100000000002	-1.14190187485876e-12\\
3.23499999999994	-1.1080442748024e-12\\
3.23699999999999	-1.09137793497014e-12\\
3.23700000000002	-1.0913779349699e-12\\
3.23999999999997	-1.06669730088544e-12\\
3.24	-1.06669730088521e-12\\
3.24299999999996	-1.04238965486689e-12\\
3.24599999999991	-1.01844427684422e-12\\
3.24599999999997	-1.01844427684369e-12\\
3.246	-1.01844427684346e-12\\
3.24799999999997	-1.0026767300233e-12\\
3.24799999999999	-1.00267673002307e-12\\
3.24999999999996	-9.87062404126876e-13\\
3.25199999999992	-9.71598238623281e-13\\
3.25399999999997	-9.56281202475678e-13\\
3.25399999999999	-9.56281202475462e-13\\
3.25499999999998	-9.48683499302722e-13\\
3.255	-9.48683499302507e-13\\
3.25599999999998	-9.41134616569704e-13\\
3.25699999999997	-9.33634184373027e-13\\
3.25899999999993	-9.1877720385342e-13\\
3.25999999999997	-9.11419927528205e-13\\
3.26	-9.11419927527996e-13\\
3.26399999999993	-8.82457196716489e-13\\
3.26599999999997	-8.68250696761139e-13\\
3.266	-8.68250696760939e-13\\
3.267	-8.61214939072229e-13\\
3.26700000000003	-8.6121493907203e-13\\
3.26800000000003	-8.54223715845782e-13\\
3.26900000000003	-8.47276684504138e-13\\
3.27100000000003	-8.33513837999691e-13\\
3.27500000000003	-8.06503612785155e-13\\
3.27699999999997	-7.9325093978526e-13\\
3.27699999999999	-7.93250939785073e-13\\
3.27999999999997	-7.73680162750712e-13\\
3.28	-7.73680162750528e-13\\
3.28299999999998	-7.54471604941536e-13\\
3.28599999999996	-7.35616794788954e-13\\
3.286	-7.35616794788708e-13\\
3.28899999999999	-7.17107416952931e-13\\
3.28900000000002	-7.17107416952757e-13\\
3.29	-7.1101735941356e-13\\
3.29000000000003	-7.11017359413387e-13\\
3.29100000000001	-7.04973251618005e-13\\
3.29199999999999	-6.98974797402188e-13\\
3.29399999999996	-6.87113676230055e-13\\
3.296	-6.75431671071039e-13\\
3.29600000000003	-6.75431671070874e-13\\
3.29999999999996	-6.52595884650585e-13\\
3.3	-6.52595884650338e-13\\
3.30000000000003	-6.52595884650178e-13\\
3.30399999999996	-6.3044953370589e-13\\
3.30599999999997	-6.19629449579047e-13\\
3.30599999999999	-6.19629449578895e-13\\
3.30999999999992	-5.98484859774678e-13\\
3.31199999999999	-5.8815620967493e-13\\
3.31200000000002	-5.88156209674785e-13\\
3.31599999999995	-5.68000025083738e-13\\
3.31999999999988	-5.48514845707144e-13\\
3.32	-5.48514845706538e-13\\
3.32000000000003	-5.48514845706402e-13\\
3.32499999999998	-5.25078747394788e-13\\
3.325	-5.25078747394657e-13\\
3.326	-5.20511932882717e-13\\
3.32600000000003	-5.20511932882588e-13\\
3.32700000000003	-5.1598472671187e-13\\
3.32800000000003	-5.11496907038435e-13\\
3.33000000000002	-5.02638549486335e-13\\
3.332	-4.93935123954123e-13\\
3.33200000000003	-4.93935123954e-13\\
3.33499999999999	-4.81166759776665e-13\\
3.33500000000002	-4.81166759776546e-13\\
3.33799999999998	-4.68737530369277e-13\\
3.33999999999997	-4.60637072643467e-13\\
3.34	-4.60637072643353e-13\\
3.34299999999997	-4.48760959518755e-13\\
3.34599999999993	-4.37209689678633e-13\\
3.34599999999997	-4.37209689678492e-13\\
3.346	-4.37209689678352e-13\\
3.34699999999999	-4.33430569199851e-13\\
3.34700000000002	-4.33430569199744e-13\\
3.34800000000001	-4.29686214014326e-13\\
3.349	-4.25975864581989e-13\\
3.35099999999999	-4.18656457402482e-13\\
3.35499999999995	-4.0441782308688e-13\\
3.35999999999997	-3.87355577165823e-13\\
3.36	-3.87355577165728e-13\\
3.36399999999997	-3.74281574790268e-13\\
3.36399999999999	-3.74281574790177e-13\\
3.36599999999997	-3.67932955903292e-13\\
3.366	-3.67932955903203e-13\\
3.36799999999998	-3.6170825786299e-13\\
3.36999999999996	-3.55606260604363e-13\\
3.37199999999997	-3.49625768112222e-13\\
3.372	-3.49625768112138e-13\\
3.37599999999996	-3.38022496189186e-13\\
3.37799999999997	-3.32396908482526e-13\\
3.378	-3.32396908482447e-13\\
3.37999999999997	-3.26887208460771e-13\\
3.38	-3.26887208460694e-13\\
3.38199999999997	-3.21492316208679e-13\\
3.38399999999995	-3.1621117430639e-13\\
3.38599999999997	-3.11042747629482e-13\\
3.386	-3.11042747629409e-13\\
3.38999999999995	-3.01040009751384e-13\\
3.39299999999997	-2.93826458140032e-13\\
3.39299999999999	-2.93826458139964e-13\\
3.39499999999998	-2.89153028123512e-13\\
3.395	-2.89153028123446e-13\\
3.39699999999999	-2.84587009855958e-13\\
3.39899999999997	-2.80127508380005e-13\\
3.399	-2.80127508379942e-13\\
3.39999999999997	-2.77933362507385e-13\\
3.4	-2.77933362507323e-13\\
3.40099999999998	-2.75757389539134e-13\\
3.40199999999996	-2.73599482850152e-13\\
3.40399999999991	-2.6933744623844e-13\\
3.40599999999997	-2.65146417296355e-13\\
3.406	-2.65146417296296e-13\\
3.40999999999991	-2.56974110372578e-13\\
3.41099999999997	-2.54974146530968e-13\\
3.411	-2.54974146530911e-13\\
3.41499999999991	-2.47143802649475e-13\\
3.41899999999982	-2.3958010242293e-13\\
3.41999999999997	-2.377301409291e-13\\
3.42	-2.37730140929048e-13\\
3.42199999999997	-2.34078736883186e-13\\
3.42199999999999	-2.34078736883135e-13\\
3.42399999999996	-2.304914297469e-13\\
3.42599999999992	-2.26967516339711e-13\\
3.426	-2.26967516339565e-13\\
3.42600000000003	-2.26967516339516e-13\\
3.42999999999996	-2.20107120209086e-13\\
3.43	-2.20107120209004e-13\\
3.432	-2.16769292823825e-13\\
3.43200000000003	-2.16769292823778e-13\\
3.43400000000003	-2.13493354232011e-13\\
3.43600000000003	-2.10279846981791e-13\\
3.438	-2.07128141213328e-13\\
3.43800000000003	-2.07128141213284e-13\\
3.43999999999997	-2.04037619184366e-13\\
3.44	-2.04037619184323e-13\\
3.44199999999995	-2.01007675144722e-13\\
3.44399999999989	-1.98037715213592e-13\\
3.44599999999997	-1.95127157267125e-13\\
3.446	-1.95127157267084e-13\\
3.44999999999989	-1.89481976955495e-13\\
3.45099999999996	-1.88106930681509e-13\\
3.45099999999999	-1.8810693068147e-13\\
3.45499999999988	-1.82749719397263e-13\\
3.45699999999996	-1.80155981707531e-13\\
3.45699999999999	-1.80155981707494e-13\\
3.45999999999997	-1.76343262411956e-13\\
3.46	-1.7634326241192e-13\\
3.46299999999998	-1.72601104764111e-13\\
3.46499999999998	-1.70144717666367e-13\\
3.465	-1.70144717666332e-13\\
3.46599999999997	-1.68927858410524e-13\\
3.466	-1.68927858410489e-13\\
3.46699999999997	-1.67718475725644e-13\\
3.46799999999994	-1.66516510351418e-13\\
3.46999999999988	-1.64134596309215e-13\\
3.47199999999997	-1.6178164941937e-13\\
3.472	-1.61781649419336e-13\\
3.47599999999988	-1.57160817952207e-13\\
3.47999999999976	-1.52650392979641e-13\\
3.47999999999996	-1.52650392979412e-13\\
3.47999999999999	-1.52650392979381e-13\\
3.48599999999996	-1.46084054779771e-13\\
3.48599999999999	-1.4608405477974e-13\\
3.49199999999996	-1.39746609218652e-13\\
3.49199999999999	-1.39746609218622e-13\\
3.49799999999996	-1.33679265825626e-13\\
3.5	-1.31726616884913e-13\\
3.50000000000003	-1.31726616884886e-13\\
3.506	-1.26072708008889e-13\\
3.50600000000003	-1.26072708008863e-13\\
3.50899999999999	-1.2335824434972e-13\\
3.50900000000002	-1.23358244349694e-13\\
3.51199999999998	-1.20717167431565e-13\\
3.51499999999995	-1.18148312487092e-13\\
3.51499999999998	-1.1814831248706e-13\\
3.51500000000002	-1.18148312487028e-13\\
3.518	-1.15634247311025e-13\\
3.51800000000003	-1.15634247311001e-13\\
3.51999999999997	-1.13979039258338e-13\\
3.52	-1.13979039258315e-13\\
3.52199999999995	-1.12340120548718e-13\\
3.52399999999989	-1.10717169948701e-13\\
3.52599999999997	-1.09109869353589e-13\\
3.526	-1.09109869353566e-13\\
3.52999999999989	-1.05940961045359e-13\\
3.53399999999978	-1.02830911036443e-13\\
3.53499999999998	-1.02062308393732e-13\\
3.535	-1.0206230839371e-13\\
3.53799999999997	-9.97772808476234e-14\\
3.53799999999999	-9.97772808476019e-14\\
3.53999999999997	-9.82708733568416e-14\\
3.54	-9.82708733568203e-14\\
3.54199999999998	-9.67776763583061e-14\\
3.54399999999996	-9.5297397180491e-14\\
3.54599999999997	-9.38297456829367e-14\\
3.546	-9.38297456829159e-14\\
3.54999999999996	-9.09311774946338e-14\\
3.54999999999999	-9.09311774946127e-14\\
3.553	-8.87923015406012e-14\\
3.55300000000003	-8.87923015405811e-14\\
3.55600000000004	-8.66870187720262e-14\\
3.55900000000005	-8.46144007247041e-14\\
3.55999999999997	-8.39306284738931e-14\\
3.56	-8.39306284738737e-14\\
3.56599999999999	-7.99002130614967e-14\\
3.56600000000002	-7.99002130614779e-14\\
3.56699999999996	-7.92402062992367e-14\\
3.56699999999999	-7.9240206299218e-14\\
3.56799999999996	-7.85834639392896e-14\\
3.56899999999992	-7.79299537984646e-14\\
3.56999999999998	-7.72796438545315e-14\\
3.57	-7.7279643854513e-14\\
3.57199999999993	-7.59884972517226e-14\\
3.57299999999996	-7.53475973262711e-14\\
3.57299999999999	-7.53475973262529e-14\\
3.57499999999992	-7.40777193373298e-14\\
3.57699999999985	-7.28253515077975e-14\\
3.57899999999996	-7.15902483686228e-14\\
3.57899999999999	-7.15902483686054e-14\\
3.57999999999997	-7.09790952759618e-14\\
3.58	-7.09790952759445e-14\\
3.58099999999998	-7.03721678368933e-14\\
3.58199999999997	-6.97694363113005e-14\\
3.58399999999993	-6.85764430687827e-14\\
3.58599999999997	-6.73998817170701e-14\\
3.586	-6.73998817170535e-14\\
3.58999999999993	-6.50951354265461e-14\\
3.59399999999986	-6.28533903997532e-14\\
3.59599999999996	-6.17555921885506e-14\\
3.59599999999999	-6.17555921885352e-14\\
3.59999999999997	-5.96050685121926e-14\\
3.6	-5.96050685121776e-14\\
3.60399999999998	-5.75132416300556e-14\\
3.60499999999998	-5.69992630875212e-14\\
3.605	-5.69992630875066e-14\\
3.60599999999997	-5.64888251834409e-14\\
3.606	-5.64888251834264e-14\\
3.60699999999997	-5.59819029067255e-14\\
3.60799999999994	-5.54784714176222e-14\\
3.60799999999999	-5.54784714175963e-14\\
3.60999999999993	-5.44834068114571e-14\\
3.61199999999987	-5.35048608415422e-14\\
3.61399999999996	-5.25426417089934e-14\\
3.61399999999999	-5.25426417089799e-14\\
3.61799999999987	-5.06664327306191e-14\\
3.61999999999997	-4.97520751404169e-14\\
3.62	-4.9752075140404e-14\\
3.62399999999988	-4.79699576405413e-14\\
3.62499999999996	-4.75340084912547e-14\\
3.62499999999999	-4.75340084912423e-14\\
3.62599999999997	-4.71018484285682e-14\\
3.626	-4.7101848428556e-14\\
3.62699999999998	-4.66734562769173e-14\\
3.62799999999997	-4.62488110446081e-14\\
3.62999999999993	-4.54106782897531e-14\\
3.63199999999997	-4.45872858867035e-14\\
3.632	-4.45872858866919e-14\\
3.63599999999993	-4.29863608493592e-14\\
3.63999999999985	-4.14470621278111e-14\\
3.63999999999997	-4.1447062127765e-14\\
3.64	-4.14470621277543e-14\\
3.64599999999997	-3.92510362891824e-14\\
3.646	-3.92510362891723e-14\\
3.65199999999997	-3.71870909114119e-14\\
3.652	-3.71870909114025e-14\\
3.65399999999996	-3.65278243738834e-14\\
3.65399999999999	-3.65278243738742e-14\\
3.65599999999995	-3.5882697769789e-14\\
3.65799999999992	-3.52515846490476e-14\\
3.65999999999996	-3.46343613110151e-14\\
3.65999999999999	-3.46343613110064e-14\\
3.66399999999992	-3.34363201442009e-14\\
3.66599999999996	-3.28540718960251e-14\\
3.66599999999999	-3.28540718960169e-14\\
3.66999999999992	-3.17225494368373e-14\\
3.67399999999984	-3.06342566306588e-14\\
3.67499999999998	-3.03688373308038e-14\\
3.67500000000001	-3.03688373307963e-14\\
3.67999999999997	-2.90810155675151e-14\\
3.68	-2.9081015567508e-14\\
3.68299999999996	-2.833929024112e-14\\
3.68299999999999	-2.83392902411131e-14\\
3.68599999999995	-2.76203888282483e-14\\
3.686	-2.76203888282368e-14\\
3.68899999999996	-2.69239942808392e-14\\
3.69199999999993	-2.62497994776264e-14\\
3.69199999999997	-2.62497994776156e-14\\
3.692	-2.62497994776094e-14\\
3.69299999999998	-2.60299154232021e-14\\
3.69300000000001	-2.60299154231958e-14\\
3.69399999999998	-2.58123859995534e-14\\
3.69499999999995	-2.55972005467561e-14\\
3.6969999999999	-2.51738194915144e-14\\
3.69999999999997	-2.45560677918906e-14\\
3.7	-2.45560677918848e-14\\
3.7039999999999	-2.3764307876104e-14\\
3.70599999999997	-2.33819281238275e-14\\
3.706	-2.33819281238222e-14\\
3.7099999999999	-2.26437942786869e-14\\
3.70999999999998	-2.26437942786725e-14\\
3.71000000000001	-2.26437942786674e-14\\
3.71199999999996	-2.22878955086735e-14\\
3.71199999999999	-2.22878955086685e-14\\
3.71399999999995	-2.19406819257382e-14\\
3.71599999999991	-2.16020854746506e-14\\
3.71799999999996	-2.12720397891771e-14\\
3.71799999999999	-2.12720397891725e-14\\
3.71999999999997	-2.09488314138285e-14\\
3.72	-2.0948831413824e-14\\
3.72199999999998	-2.06307482331831e-14\\
3.72399999999997	-2.03177279017053e-14\\
3.72599999999997	-2.00097090661947e-14\\
3.726	-2.00097090661904e-14\\
3.728	-1.97066313541854e-14\\
3.72800000000003	-1.97066313541811e-14\\
3.73000000000004	-1.94084353616775e-14\\
3.73200000000004	-1.91150626411012e-14\\
3.73600000000004	-1.85425579418699e-14\\
3.74	-1.79886683773474e-14\\
3.74000000000003	-1.79886683773435e-14\\
3.74099999999996	-1.7853053394309e-14\\
3.74099999999999	-1.78530533943051e-14\\
3.74199999999996	-1.77185680019975e-14\\
3.74299999999992	-1.75852056090617e-14\\
3.74499999999985	-1.73218237368769e-14\\
3.745	-1.73218237368569e-14\\
3.74500000000003	-1.73218237368531e-14\\
3.746	-1.71917913517187e-14\\
3.74600000000003	-1.71917913517151e-14\\
3.747	-1.70628561539129e-14\\
3.74799999999997	-1.69350118254708e-14\\
3.74999999999991	-1.66825707722222e-14\\
3.752	-1.64344187125295e-14\\
3.75200000000003	-1.6434418712526e-14\\
3.75599999999991	-1.59516236749897e-14\\
3.758	-1.57170950130466e-14\\
3.75800000000003	-1.57170950130433e-14\\
3.75999999999997	-1.54870839998064e-14\\
3.76	-1.54870839998032e-14\\
3.76199999999995	-1.52615455525109e-14\\
3.76399999999989	-1.50404354647483e-14\\
3.76599999999997	-1.48237103980698e-14\\
3.766	-1.48237103980668e-14\\
3.76999999999989	-1.44032462646985e-14\\
3.76999999999996	-1.4403246264691e-14\\
3.76999999999999	-1.44032462646881e-14\\
3.77399999999988	-1.39998234868054e-14\\
3.77599999999999	-1.380440324591e-14\\
3.77600000000002	-1.38044032459072e-14\\
3.77999999999991	-1.34220285315939e-14\\
3.77999999999996	-1.34220285315895e-14\\
3.78	-1.3422028531585e-14\\
3.78399999999989	-1.3048075834305e-14\\
3.78599999999997	-1.28641658659008e-14\\
3.786	-1.28641658658982e-14\\
3.78999999999989	-1.25022984483233e-14\\
3.79199999999997	-1.23242700717147e-14\\
3.792	-1.23242700717122e-14\\
3.79599999999989	-1.19738495058769e-14\\
3.79899999999996	-1.17158307017958e-14\\
3.79899999999999	-1.17158307017934e-14\\
3.79999999999997	-1.16307155103961e-14\\
3.8	-1.16307155103936e-14\\
3.80099999999999	-1.15460388882036e-14\\
3.80199999999997	-1.1461796685971e-14\\
3.80399999999993	-1.12945990509233e-14\\
3.80599999999997	-1.11290898338637e-14\\
3.806	-1.11290898338614e-14\\
3.80999999999993	-1.08030072175168e-14\\
3.81099999999996	-1.07224915281335e-14\\
3.81099999999999	-1.07224915281312e-14\\
3.81499999999992	-1.04045676888103e-14\\
3.81499999999998	-1.04045676888059e-14\\
3.81500000000001	-1.04045676888036e-14\\
3.81899999999993	-1.0093174867736e-14\\
3.81999999999997	-1.00163184131473e-14\\
3.82	-1.00163184131451e-14\\
3.82399999999993	-9.71274661367409e-15\\
3.826	-9.563221127408e-15\\
3.82600000000003	-9.56322112740588e-15\\
3.82700000000001	-9.48901061764938e-15\\
3.82700000000003	-9.48901061764728e-15\\
3.82799999999998	-9.41516340912175e-15\\
3.82800000000001	-9.41516340911966e-15\\
3.82899999999997	-9.34167588322551e-15\\
3.82999999999993	-9.26854443906041e-15\\
3.83199999999986	-9.12333547936829e-15\\
3.83399999999998	-8.9795080685137e-15\\
3.83400000000001	-8.97950806851167e-15\\
3.83799999999985	-8.69687414224253e-15\\
3.84	-8.55825940375652e-15\\
3.84000000000003	-8.55825940375456e-15\\
3.84399999999988	-8.28629852652514e-15\\
3.846	-8.15289908236819e-15\\
3.84600000000003	-8.15289908236631e-15\\
3.84999999999988	-7.89113144466315e-15\\
3.84999999999998	-7.89113144465669e-15\\
3.85000000000001	-7.89113144465485e-15\\
3.85399999999985	-7.63590165374497e-15\\
3.85599999999998	-7.51067571944e-15\\
3.85600000000001	-7.51067571943823e-15\\
3.85699999999996	-7.44864920347337e-15\\
3.85699999999999	-7.44864920347161e-15\\
3.85799999999995	-7.38700959624349e-15\\
3.85899999999992	-7.32575387734175e-15\\
3.85999999999997	-7.26487904521555e-15\\
3.86	-7.26487904521383e-15\\
3.86199999999993	-7.14426012833152e-15\\
3.86399999999985	-7.02512920389251e-15\\
3.86599999999997	-6.90746292176486e-15\\
3.866	-6.9074629217632e-15\\
3.86899999999996	-6.73365933031322e-15\\
3.86899999999999	-6.73365933031159e-15\\
3.87199999999995	-6.56339350943366e-15\\
3.87499999999991	-6.39696116990132e-15\\
3.87999999999997	-6.12789417999429e-15\\
3.88	-6.12789417999279e-15\\
3.88499999999998	-5.86894290397029e-15\\
3.88500000000001	-5.86894290396885e-15\\
3.88599999999999	-5.81833840285625e-15\\
3.88600000000002	-5.81833840285482e-15\\
3.88700000000001	-5.76812332246034e-15\\
3.88799999999999	-5.71829520212548e-15\\
3.88999999999996	-5.61979009412465e-15\\
3.89199999999999	-5.52280377146847e-15\\
3.89200000000002	-5.5228037714671e-15\\
3.89599999999996	-5.33353661615564e-15\\
3.89799999999999	-5.24127490318989e-15\\
3.89800000000002	-5.24127490318859e-15\\
3.89999999999997	-5.15057021903449e-15\\
3.9	-5.15057021903321e-15\\
3.90199999999996	-5.06140478533817e-15\\
3.90399999999991	-4.97376112533022e-15\\
3.90599999999997	-4.88762206050934e-15\\
3.906	-4.88762206050813e-15\\
3.90999999999991	-4.71979047420756e-15\\
3.91399999999982	-4.5577784378095e-15\\
3.91499999999996	-4.51816974827689e-15\\
3.91499999999999	-4.51816974827577e-15\\
3.91999999999997	-4.3253953917768e-15\\
3.92	-4.32539539177573e-15\\
3.92499999999999	-4.14123123426453e-15\\
3.92599999999997	-4.10541158897463e-15\\
3.926	-4.10541158897361e-15\\
3.92699999999996	-4.06992553858963e-15\\
3.92699999999999	-4.06992553858863e-15\\
3.92799999999995	-4.03476593523774e-15\\
3.92899999999992	-3.99992564689076e-15\\
3.93099999999984	-3.93119620204696e-15\\
3.93299999999996	-3.86372373281414e-15\\
3.93299999999999	-3.86372373281319e-15\\
3.93699999999984	-3.73249706594185e-15\\
3.93999999999997	-3.63728003351394e-15\\
3.94	-3.63728003351305e-15\\
3.94399999999985	-3.51451477489942e-15\\
3.94399999999996	-3.51451477489599e-15\\
3.94399999999999	-3.51451477489514e-15\\
3.94599999999997	-3.45490106012126e-15\\
3.946	-3.45490106012042e-15\\
3.94799999999999	-3.39645096567762e-15\\
3.94999999999997	-3.3391530351195e-15\\
3.95199999999997	-3.28299603782984e-15\\
3.952	-3.28299603782905e-15\\
3.95499999999998	-3.20086449267831e-15\\
3.95500000000001	-3.20086449267755e-15\\
3.95799999999998	-3.12121654816652e-15\\
3.95999999999998	-3.06948030581747e-15\\
3.96	-3.06948030581675e-15\\
3.96299999999998	-2.9938941716905e-15\\
3.96599999999995	-2.92070036092197e-15\\
3.966	-2.92070036092077e-15\\
3.96600000000003	-2.92070036092009e-15\\
3.96700000000001	-2.89682844199341e-15\\
3.96700000000003	-2.89682844199273e-15\\
3.96800000000001	-2.87321757195077e-15\\
3.96899999999998	-2.84986659375462e-15\\
3.97099999999993	-2.80393974876799e-15\\
3.97299999999996	-2.75903890519198e-15\\
3.97299999999999	-2.75903890519135e-15\\
3.97699999999989	-2.67228021897267e-15\\
3.97899999999996	-2.63040537128516e-15\\
3.97899999999999	-2.63040537128457e-15\\
3.97999999999997	-2.60980227817279e-15\\
3.98	-2.6098022781722e-15\\
3.98099999999999	-2.58936982915896e-15\\
3.98199999999997	-2.56910702303091e-15\\
3.98399999999994	-2.52908637616805e-15\\
3.98599999999997	-2.48973249338103e-15\\
3.986	-2.48973249338048e-15\\
3.98999999999994	-2.41299429540171e-15\\
3.98999999999997	-2.41299429540103e-15\\
3.99000000000001	-2.41299429540037e-15\\
3.99399999999994	-2.33883226238571e-15\\
3.99599999999998	-2.30269905918614e-15\\
3.99600000000001	-2.30269905918563e-15\\
3.99999999999994	-2.23229286699979e-15\\
3.99999999999997	-2.23229286699922e-15\\
4	-2.23229286699867e-15\\
4.00199999999993	-2.19800607812475e-15\\
4.00199999999999	-2.19800607812378e-15\\
4.00399999999992	-2.16432116057834e-15\\
4.00599999999986	-2.1312315119864e-15\\
4.00599999999993	-2.13123151198521e-15\\
4.006	-2.131231511984e-15\\
4.00999999999987	-2.06681219435629e-15\\
4.01199999999995	-2.03546989890434e-15\\
4.012	-2.03546989890346e-15\\
4.01599999999987	-1.97453381523466e-15\\
4.01999999999973	-1.91591911619815e-15\\
4.01999999999995	-1.9159191161951e-15\\
4.02	-1.91591911619428e-15\\
4.02500000000001	-1.84584534691078e-15\\
4.02500000000006	-1.84584534691e-15\\
4.02599999999995	-1.83224962131657e-15\\
4.026	-1.8322496213158e-15\\
4.02699999999996	-1.81879200145041e-15\\
4.02799999999992	-1.80547182785829e-15\\
4.02999999999985	-1.77924121543208e-15\\
4.03099999999993	-1.76632949127905e-15\\
4.03099999999999	-1.76632949127832e-15\\
4.03499999999984	-1.71602512289913e-15\\
4.03699999999994	-1.69166985123904e-15\\
4.03699999999999	-1.69166985123835e-15\\
4.03799999999995	-1.67966141511966e-15\\
4.038	-1.67966141511898e-15\\
4.03899999999996	-1.66772776201919e-15\\
4.03999999999991	-1.65586830717578e-15\\
4.04	-1.65586830717471e-15\\
4.04199999999991	-1.63236967140167e-15\\
4.04399999999982	-1.60916090199095e-15\\
4.046	-1.5862374499312e-15\\
4.04600000000006	-1.58623744993055e-15\\
4.04999999999988	-1.54122858071832e-15\\
4.05399999999969	-1.49730777448647e-15\\
4.05999999999994	-1.43339158073021e-15\\
4.06	-1.43339158072961e-15\\
};
\pgfplotsset{max space between ticks=50}
\addplot [color=mycolor2,solid,forget plot]
  table[row sep=crcr]{%
0	0.15313\\
3.15544362088405e-30	0.15313\\
0.000656101980281985	0.153131614989962\\
0.00393661188169191	0.153188143215565\\
0.00599999999999994	0.153265080494076\\
0.006	0.153265080494076\\
0.012	0.153670560289007\\
0.0120000000000001	0.153670560289007\\
0.018	0.153071664853654\\
0.0180000000000001	0.153071664853654\\
0.0199999999999998	0.152365329512728\\
0.02	0.152365329512728\\
0.026	0.148724274488254\\
0.0260000000000002	0.148724274488254\\
0.0289999999999998	0.146046083995443\\
0.029	0.146046083995443\\
0.0319999999999996	0.142794627452511\\
0.0349999999999991	0.138968470912041\\
0.035	0.13896847091204\\
0.0399999999999996	0.131789348388764\\
0.04	0.131789348388763\\
0.0449999999999996	0.123959478342115\\
0.0459999999999996	0.122314584640995\\
0.046	0.122314584640994\\
0.047	0.120643199324198\\
0.0470000000000004	0.120643199324197\\
0.0490000000000003	0.117220624955326\\
0.0510000000000002	0.113691084393681\\
0.055	0.106308316385691\\
0.0579999999999996	0.100485154924159\\
0.058	0.100485154924158\\
0.0599999999999996	0.0964653994777478\\
0.06	0.0964653994777469\\
0.0619999999999995	0.092334609798713\\
0.0639999999999991	0.0880919761882657\\
0.0659999999999991	0.0837366670740184\\
0.066	0.0837366670740165\\
0.0699999999999991	0.0746845854767947\\
0.07	0.0746845854767927\\
0.0700000000000009	0.0746845854767906\\
0.074	0.0654820973483735\\
0.076	0.0609387524318376\\
0.0760000000000009	0.0609387524318356\\
0.08	0.0519634081808075\\
0.0800000000000009	0.0519634081808055\\
0.0839999999999999	0.0431306724709095\\
0.086	0.0387656144357706\\
0.0860000000000009	0.0387656144357686\\
0.0869999999999991	0.0365955374773486\\
0.087	0.0365955374773467\\
0.0880000000000004	0.0344336199800889\\
0.0890000000000009	0.03227975600762\\
0.0910000000000017	0.0279957668674395\\
0.0929999999999991	0.0237427303784554\\
0.093	0.0237427303784535\\
0.0970000000000017	0.0154879858559104\\
0.0999999999999991	0.00958444869935146\\
0.1	0.00958444869934975\\
0.104000000000002	0.00209147738315415\\
0.104999999999999	0.000285191714757269\\
0.105	0.000285191714755677\\
0.105999999999999	-0.00149449006990914\\
0.106	-0.00149449006991071\\
0.106999999999999	-0.00324765517394124\\
0.107999999999998	-0.00497438950448962\\
0.109999999999997	-0.00834890299548232\\
0.111999999999999	-0.0116186919605986\\
0.112	-0.0116186919606\\
0.115999999999997	-0.0178466394573827\\
0.115999999999998	-0.0178466394573853\\
0.116	-0.0178466394573879\\
0.119999999999997	-0.0236631150271634\\
0.119999999999998	-0.0236631150271658\\
0.12	-0.0236631150271682\\
0.123999999999997	-0.0290726790847185\\
0.125999999999999	-0.031626207097874\\
0.126	-0.0316262070978751\\
0.127999999999998	-0.03407957300888\\
0.128	-0.0340795730088821\\
0.129999999999998	-0.0364271843737179\\
0.131999999999996	-0.0386634280235806\\
0.135999999999993	-0.0428035436794524\\
0.139999999999998	-0.046503166088059\\
0.14	-0.0465031660880606\\
0.144999999999998	-0.0505126366966201\\
0.145	-0.0505126366966214\\
0.145999999999998	-0.0512329261213502\\
0.146	-0.0512329261213514\\
0.146999999999999	-0.0519260994409184\\
0.147999999999998	-0.0525921906227423\\
0.149999999999997	-0.0538432558021337\\
0.151999999999998	-0.0549863668732067\\
0.152	-0.0549863668732077\\
0.155999999999997	-0.0570007720155776\\
0.157999999999998	-0.0578852528495309\\
0.158	-0.0578852528495317\\
0.16	-0.0586881556806868\\
0.160000000000002	-0.0586881556806875\\
0.162000000000002	-0.0594096378799837\\
0.164000000000002	-0.0600498408609338\\
0.166	-0.0606088901058539\\
0.166000000000002	-0.0606088901058543\\
0.170000000000002	-0.0614839498018867\\
0.174	-0.0620355030603464\\
0.174000000000001	-0.0620355030603465\\
0.175	-0.0621228997391115\\
0.175000000000002	-0.0621228997391117\\
0.176000000000001	-0.0621901098124187\\
0.177	-0.0622371365737359\\
0.178999999999998	-0.0622706483888091\\
0.179999999999998	-0.0622571350847035\\
0.18	-0.0622571350847035\\
0.183999999999997	-0.0620804705828559\\
0.186	-0.0619304203383477\\
0.186000000000002	-0.0619304203383475\\
0.189999999999998	-0.0615067368047821\\
0.192	-0.061233020472304\\
0.192000000000002	-0.0612330204723037\\
0.195999999999998	-0.0605615704193962\\
0.199999999999995	-0.0597243116296848\\
0.199999999999997	-0.0597243116296842\\
0.2	-0.0597243116296836\\
0.202999999999998	-0.0589871655511613\\
0.203	-0.0589871655511608\\
0.205999999999998	-0.0581560603911522\\
0.206	-0.0581560603911517\\
0.208999999999998	-0.0572306296136106\\
0.209999999999998	-0.0569011221000455\\
0.21	-0.0569011221000449\\
0.211999999999998	-0.0562104650884198\\
0.212	-0.0562104650884192\\
0.213999999999998	-0.0554939446732612\\
0.215999999999997	-0.0547678599943136\\
0.217999999999998	-0.0540320687361973\\
0.218	-0.0540320687361967\\
0.219999999999998	-0.0532864266819164\\
0.22	-0.0532864266819157\\
0.221999999999998	-0.0525307876837086\\
0.223999999999996	-0.0517650036334558\\
0.225999999999998	-0.0509889244345494\\
0.226	-0.0509889244345487\\
0.229999999999996	-0.0494052700894414\\
0.231999999999998	-0.0485973845409472\\
0.232	-0.0485973845409465\\
0.235999999999996	-0.0469487049222538\\
0.237999999999998	-0.0461075877045604\\
0.238	-0.0461075877045596\\
0.239999999999998	-0.0452638608140846\\
0.24	-0.0452638608140839\\
0.241999999999998	-0.0444261532252999\\
0.243999999999996	-0.0435943007442838\\
0.245	-0.0431805191730594\\
0.245000000000002	-0.0431805191730586\\
0.245999999999998	-0.042768140324533\\
0.246	-0.0427681403245322\\
0.246999999999999	-0.0423571439924296\\
0.247999999999998	-0.0419475100374713\\
0.249999999999997	-0.0411322490361785\\
0.252	-0.0403221975263589\\
0.252000000000003	-0.0403221975263574\\
0.256	-0.0387170888829923\\
0.259999999999997	-0.0371309256194401\\
0.26	-0.0371309256194387\\
0.260999999999996	-0.0367371989641445\\
0.261	-0.0367371989641431\\
0.261999999999998	-0.0363445593118938\\
0.262999999999996	-0.0359529874189366\\
0.264999999999993	-0.0351729702138744\\
0.265999999999997	-0.0347844866804151\\
0.266	-0.0347844866804137\\
0.269999999999993	-0.0332402760675451\\
0.271999999999997	-0.032473740201349\\
0.272	-0.0324737402013477\\
0.275999999999993	-0.0309768515277924\\
0.279999999999986	-0.0295444303379096\\
0.279999999999993	-0.0295444303379072\\
0.28	-0.0295444303379047\\
0.285999999999996	-0.0275142319192832\\
0.286	-0.0275142319192821\\
0.289999999999996	-0.0262381743224773\\
0.29	-0.0262381743224762\\
0.291999999999996	-0.0256229882303971\\
0.292	-0.0256229882303961\\
0.293999999999996	-0.0250228688456157\\
0.295999999999993	-0.0244376985423862\\
0.297999999999996	-0.0238673626250648\\
0.298	-0.0238673626250638\\
0.299999999999996	-0.0233108682285088\\
0.3	-0.0233108682285079\\
0.301999999999996	-0.0227672252005078\\
0.303999999999993	-0.022236326984904\\
0.305999999999996	-0.0217180695235936\\
0.306	-0.0217180695235927\\
0.309999999999993	-0.0207190730028746\\
0.313999999999986	-0.0197694523735308\\
0.314999999999997	-0.0195396741022794\\
0.315	-0.0195396741022786\\
0.318999999999997	-0.0186507307825328\\
0.319	-0.018650730782532\\
0.319999999999996	-0.0184359829485392\\
0.32	-0.0184359829485385\\
0.320999999999998	-0.0182242090470824\\
0.321999999999996	-0.0180153987009941\\
0.323999999999993	-0.01760662789264\\
0.325999999999996	-0.0172095904028153\\
0.326	-0.0172095904028146\\
0.329999999999993	-0.0164504063821836\\
0.331	-0.0162678248691355\\
0.331000000000004	-0.0162678248691348\\
0.333	-0.0159097955544508\\
0.333000000000004	-0.0159097955544502\\
0.335	-0.0155602458559603\\
0.336999999999996	-0.0152191072605565\\
0.339999999999996	-0.0147230243676384\\
0.34	-0.0147230243676379\\
0.343999999999993	-0.0140904089839253\\
0.345999999999997	-0.0137863164050161\\
0.346	-0.0137863164050155\\
0.347999999999997	-0.0134902872333528\\
0.348	-0.0134902872333523\\
0.349999999999997	-0.0132022634464662\\
0.35	-0.0132022634464657\\
0.351999999999997	-0.0129221885905648\\
0.353999999999993	-0.0126500077700806\\
0.354	-0.0126500077700797\\
0.357999999999993	-0.012125862983203\\
0.359999999999996	-0.0118729829735022\\
0.36	-0.0118729829735018\\
0.363999999999993	-0.011385359886108\\
0.365999999999996	-0.0111505212323883\\
0.366	-0.0111505212323879\\
0.369999999999993	-0.0106985595508986\\
0.373999999999986	-0.0102699253278371\\
0.376999999999997	-0.00996355545253901\\
0.377	-0.00996355545253866\\
0.379999999999997	-0.0096699827491936\\
0.38	-0.00966998274919326\\
0.382999999999996	-0.00938907775174519\\
0.384999999999997	-0.00920878395989238\\
0.385	-0.00920878395989206\\
0.385999999999997	-0.00912071657186842\\
0.386	-0.00912071657186811\\
0.386999999999998	-0.00903402974610353\\
0.387999999999996	-0.00894871923490015\\
0.388999999999997	-0.00886478085802219\\
0.389	-0.0088647808580219\\
0.390999999999997	-0.00870031158854401\\
0.392999999999993	-0.00853989716740149\\
0.394999999999997	-0.00838350615274735\\
0.395	-0.00838350615274707\\
0.398999999999993	-0.00808267251348065\\
0.399999999999997	-0.00800993178341249\\
0.4	-0.00800993178341223\\
0.403999999999993	-0.00772873264866916\\
0.405999999999997	-0.00759394329312424\\
0.406	-0.00759394329312401\\
0.409999999999993	-0.00733585290516443\\
0.411999999999997	-0.00721250128602441\\
0.412	-0.00721250128602419\\
0.415999999999993	-0.00697476061535019\\
0.419999999999986	-0.00674727983018301\\
0.419999999999996	-0.00674727983018241\\
0.42	-0.00674727983018221\\
0.426	-0.0064249079467603\\
0.426000000000004	-0.00642490794676012\\
0.432000000000004	-0.00612465260943283\\
0.432000000000007	-0.00612465260943266\\
0.434999999999997	-0.0059826516063656\\
0.435	-0.00598265160636544\\
0.43799999999999	-0.00584598408830854\\
0.439999999999997	-0.00575780559630316\\
0.44	-0.00575780559630301\\
0.44299999999999	-0.00562989442946651\\
0.445999999999979	-0.00550716117548773\\
0.445999999999995	-0.0055071611754871\\
0.446	-0.00550716117548689\\
0.447	-0.00546739131729642\\
0.447000000000004	-0.00546739131729628\\
0.448000000000004	-0.00542809205493715\\
0.449000000000004	-0.00538916471605226\\
0.451000000000004	-0.00531241819704646\\
0.454999999999997	-0.00516330552264952\\
0.455	-0.00516330552264939\\
0.459	-0.00501993638212905\\
0.459999999999997	-0.00498497825355961\\
0.46	-0.00498497825355948\\
0.463999999999997	-0.00484863097486232\\
0.464	-0.0048486309748622\\
0.465999999999997	-0.00478252515245442\\
0.466	-0.0047825251524543\\
0.466999999999997	-0.00474998348273337\\
0.467	-0.00474998348273326\\
0.467999999999998	-0.00471778050903213\\
0.468999999999997	-0.00468591465339736\\
0.470999999999993	-0.00462318806703604\\
0.472999999999997	-0.00456179142899598\\
0.473	-0.00456179142899587\\
0.476999999999993	-0.00444244961269803\\
0.479999999999997	-0.00435570274098324\\
0.48	-0.00435570274098314\\
0.483999999999993	-0.00424364682919994\\
0.485999999999997	-0.00418913978671694\\
0.486	-0.00418913978671685\\
0.489999999999993	-0.00408311410884927\\
0.49	-0.00408311410884908\\
0.490000000000004	-0.00408311410884899\\
0.492999999999997	-0.00400616888817767\\
0.493	-0.00400616888817757\\
0.495999999999993	-0.00393139219129555\\
0.498999999999986	-0.00385875104045262\\
0.498999999999993	-0.00385875104045245\\
0.499	-0.00385875104045228\\
0.499999999999997	-0.00383494896778564\\
0.5	-0.00383494896778555\\
0.500999999999998	-0.00381126472966637\\
0.501999999999997	-0.00378769716546426\\
0.503999999999993	-0.00374090744521033\\
0.505999999999993	-0.00369457063605833\\
0.506	-0.00369457063605816\\
0.507999999999997	-0.0036486776558763\\
0.508000000000004	-0.00364867765587614\\
0.51	-0.00360321950952306\\
0.511999999999997	-0.00355818728702618\\
0.51599999999999	-0.00346936538951466\\
0.519999999999993	-0.00338214232149069\\
0.52	-0.00338214232149054\\
0.521999999999993	-0.00333910893017963\\
0.522	-0.00333910893017948\\
0.523999999999993	-0.00329644969685139\\
0.524999999999993	-0.00327525777344733\\
0.525	-0.00327525777344718\\
0.526	-0.00325415625951213\\
0.526000000000007	-0.00325415625951198\\
0.527000000000007	-0.00323314412106835\\
0.528000000000007	-0.00321222032851632\\
0.530000000000007	-0.00317063368428566\\
0.532	-0.00312938817567781\\
0.532000000000007	-0.00312938817567766\\
0.536000000000007	-0.00304820674104637\\
0.538	-0.00300833451107474\\
0.538000000000007	-0.0030083345110746\\
0.539999999999993	-0.00296893082144862\\
0.54	-0.00296893082144848\\
0.541999999999986	-0.0029299879489443\\
0.543999999999972	-0.00289149826060799\\
0.546	-0.00285345421231157\\
0.546000000000007	-0.00285345421231143\\
0.549999999999979	-0.00277867329483269\\
0.550999999999993	-0.00276024504295249\\
0.551	-0.00276024504295236\\
0.554999999999972	-0.00268757285469358\\
0.556999999999993	-0.00265184880636015\\
0.557	-0.00265184880636002\\
0.559999999999993	-0.00259888476765015\\
0.56	-0.00259888476765003\\
0.562999999999993	-0.00254654778824355\\
0.565999999999986	-0.00249481478667443\\
0.565999999999993	-0.00249481478667431\\
0.566	-0.00249481478667418\\
0.571999999999986	-0.0023930697129838\\
0.571999999999993	-0.00239306971298368\\
0.572	-0.00239306971298356\\
0.577999999999986	-0.00229347004188118\\
0.579999999999993	-0.00226071643314977\\
0.58	-0.00226071643314965\\
0.585999999999986	-0.00216370456659175\\
0.585999999999993	-0.00216370456659164\\
0.586	-0.00216370456659152\\
0.591999999999986	-0.00206843344302369\\
0.591999999999993	-0.00206843344302358\\
0.592	-0.00206843344302346\\
0.594999999999993	-0.00202174583165955\\
0.595	-0.00202174583165944\\
0.597999999999993	-0.00197612637154641\\
0.599999999999993	-0.00194629686101901\\
0.6	-0.0019462968610189\\
0.602999999999993	-0.00190241317260978\\
0.605999999999986	-0.00185954500765136\\
0.606	-0.00185954500765116\\
0.606999999999993	-0.00184547801078418\\
0.607	-0.00184547801078408\\
0.607999999999999	-0.00183152105639463\\
0.608999999999997	-0.00181767346053036\\
0.609000000000004	-0.00181767346053026\\
0.611	-0.00179030363556277\\
0.612999999999997	-0.0017633631713082\\
0.614999999999997	-0.00173684678733163\\
0.615000000000004	-0.00173684678733154\\
0.618999999999997	-0.00168477807008363\\
0.619999999999993	-0.00167192806435549\\
0.62	-0.0016719280643554\\
0.623999999999993	-0.00162117784545424\\
0.625999999999993	-0.00159618392566223\\
0.626	-0.00159618392566214\\
0.629999999999993	-0.00154693397006005\\
0.63	-0.00154693397005996\\
0.633999999999993	-0.00149863576091037\\
0.635999999999993	-0.0014748316990718\\
0.636	-0.00147483169907172\\
0.637999999999993	-0.00145125142991222\\
0.638	-0.00145125142991213\\
0.639999999999993	-0.00142789033163865\\
0.64	-0.00142789033163857\\
0.641999999999993	-0.00140474382538547\\
0.643999999999986	-0.00138180737436343\\
0.645999999999993	-0.00135907648293958\\
0.646	-0.0013590764829395\\
0.649999999999986	-0.0013142135970691\\
0.65	-0.00131421359706894\\
0.650000000000007	-0.00131421359706887\\
0.653999999999993	-0.00127027370634741\\
0.657999999999979	-0.00122737607303918\\
0.659999999999993	-0.00120630756079553\\
0.66	-0.00120630756079546\\
0.664999999999993	-0.00115471242812662\\
0.665	-0.00115471242812654\\
0.665999999999993	-0.00114457383303655\\
0.666	-0.00114457383303648\\
0.666999999999998	-0.00113449421574781\\
0.667000000000006	-0.00113449421574773\\
0.668000000000004	-0.00112447308233544\\
0.669000000000002	-0.00111450994176199\\
0.670999999999998	-0.00109475568916581\\
0.673000000000005	-0.00107522758604948\\
0.673000000000013	-0.00107522758604941\\
0.677000000000005	-0.00103698988691533\\
0.678	-0.00102761442563734\\
0.678000000000007	-0.00102761442563727\\
0.679999999999993	-0.00100908103090358\\
0.68	-0.00100908103090352\\
0.681999999999986	-0.000990834652012911\\
0.683999999999972	-0.000972871712604208\\
0.686	-0.000955188691871652\\
0.686000000000007	-0.00095518869187159\\
0.689999999999979	-0.000920648596967051\\
0.69399999999995	-0.000887187286082886\\
0.695999999999993	-0.000870852943369295\\
0.696	-0.000870852943369238\\
0.699999999999993	-0.00083896087550719\\
0.7	-0.000838960875507134\\
0.703999999999993	-0.000808083544671208\\
0.705999999999993	-0.000793017809496112\\
0.706	-0.000793017809496059\\
0.707999999999993	-0.000778196741068832\\
0.708	-0.00077819674106878\\
0.709999999999993	-0.000763631499413554\\
0.711999999999986	-0.000749333294677653\\
0.713999999999993	-0.000735299324357685\\
0.714	-0.000735299324357636\\
0.717999999999986	-0.000708013135452678\\
0.719999999999993	-0.000694755568668496\\
0.72	-0.000694755568668449\\
0.723999999999986	-0.000668998497336283\\
0.724999999999993	-0.00066271531406811\\
0.725	-0.000662715314068066\\
0.725999999999993	-0.00065649394429886\\
0.726	-0.000656493944298816\\
0.726999999999999	-0.000650334083185832\\
0.727999999999997	-0.000644235428888822\\
0.729999999999993	-0.000632220548386102\\
0.731999999999993	-0.000620446947827684\\
0.732	-0.000620446947827642\\
0.734999999999993	-0.000603234887845982\\
0.735	-0.000603234887845942\\
0.737999999999993	-0.000586554887958984\\
0.74	-0.00057572683474829\\
0.740000000000007	-0.000575726834748252\\
0.743	-0.000559917369083065\\
0.745999999999993	-0.000544620859781388\\
0.746000000000007	-0.000544620859781318\\
0.746999999999993	-0.00053963484119366\\
0.747	-0.000539634841193625\\
0.747999999999999	-0.000534704822068924\\
0.748999999999997	-0.000529830560814073\\
0.750999999999993	-0.000520248359275258\\
0.753999999999993	-0.000506287358166193\\
0.754	-0.00050628735816616\\
0.757999999999993	-0.000488432797560279\\
0.759999999999993	-0.000479827328689932\\
0.76	-0.000479827328689902\\
0.763999999999993	-0.000463158262339389\\
0.766	-0.000455068069382597\\
0.766000000000007	-0.000455068069382568\\
0.77	-0.000439368435098841\\
0.770000000000007	-0.000439368435098813\\
0.774	-0.00042429959742365\\
0.776	-0.000416998016139368\\
0.776000000000007	-0.000416998016139343\\
0.779999999999993	-0.000402853372750483\\
0.78	-0.000402853372750459\\
0.782999999999993	-0.000392640650914832\\
0.783	-0.000392640650914808\\
0.785999999999993	-0.000382762134321589\\
0.786000000000001	-0.000382762134321566\\
0.788999999999994	-0.000373213466382827\\
0.791999999999987	-0.000363990435981253\\
0.792	-0.000363990435981211\\
0.792000000000008	-0.000363990435981189\\
0.797999999999994	-0.000346452790223192\\
0.799999999999993	-0.000340864097740224\\
0.8	-0.000340864097740204\\
0.804999999999993	-0.000327444554621578\\
0.805000000000001	-0.00032744455462156\\
0.805999999999993	-0.000324854251627215\\
0.806	-0.000324854251627197\\
0.806999999999994	-0.000322294850800829\\
0.807999999999987	-0.000319766226725563\\
0.809999999999973	-0.000314800814717995\\
0.811999999999993	-0.000309957042364469\\
0.812	-0.000309957042364452\\
0.815999999999973	-0.000300630642712753\\
0.817999999999993	-0.000296146187403729\\
0.818000000000001	-0.000296146187403713\\
0.819999999999993	-0.000291755346293803\\
0.82	-0.000291755346293788\\
0.821999999999993	-0.000287432889674865\\
0.823999999999986	-0.000283177970328631\\
0.825999999999993	-0.000278989754274457\\
0.826	-0.000278989754274443\\
0.829999999999986	-0.000270810161354624\\
0.833999999999972	-0.000262887697692615\\
0.839999999999993	-0.000251472585533723\\
0.84	-0.00025147258553371\\
0.840999999999993	-0.00024962353120844\\
0.841000000000001	-0.000249623531208427\\
0.841999999999994	-0.000247789508349448\\
0.842999999999987	-0.000245970427068194\\
0.844999999999973	-0.000242376733425243\\
0.845999999999993	-0.000240601944970145\\
0.846	-0.000240601944970133\\
0.849999999999973	-0.000233647826061438\\
0.851999999999993	-0.000230256597378268\\
0.852	-0.000230256597378256\\
0.855999999999973	-0.000223652582586451\\
0.857999999999993	-0.000220441028094253\\
0.858	-0.000220441028094242\\
0.86	-0.000217288932313668\\
0.860000000000007	-0.000217288932313657\\
0.862000000000007	-0.00021419567742588\\
0.864000000000007	-0.000211160657140891\\
0.866	-0.000208183276583013\\
0.866000000000007	-0.000208183276583002\\
0.87	-0.000202399111537331\\
0.870000000000007	-0.000202399111537321\\
0.874	-0.000196838647205059\\
0.874999999999994	-0.000195482969651698\\
0.875000000000001	-0.000195482969651688\\
0.876	-0.000194140933644273\\
0.876000000000007	-0.000194140933644264\\
0.877000000000007	-0.000192809225187038\\
0.878000000000006	-0.000191484530787782\\
0.879999999999998	-0.00018885592486254\\
0.880000000000006	-0.000188855924862531\\
0.882000000000005	-0.000186254600653333\\
0.884000000000004	-0.000183680048292316\\
0.886000000000005	-0.000181131763157299\\
0.886000000000013	-0.00018113176315729\\
0.888000000000007	-0.000178609245777949\\
0.888000000000014	-0.00017860924577794\\
0.890000000000009	-0.00017611200173442\\
0.892000000000004	-0.000173639541557222\\
0.895999999999993	-0.000168767039124104\\
0.898999999999993	-0.000165174149997176\\
0.899000000000001	-0.000165174149997168\\
0.899999999999993	-0.00016398791812363\\
0.9	-0.000163987918123621\\
0.900999999999994	-0.000162807288071465\\
0.901999999999987	-0.000161632201976358\\
0.903999999999973	-0.000159298431608309\\
0.905999999999993	-0.000156986149591459\\
0.906	-0.000156986149591451\\
0.909999999999973	-0.000152424241883398\\
0.909999999999987	-0.000152424241883383\\
0.910000000000001	-0.000152424241883367\\
0.910999999999993	-0.000151296491998817\\
0.911000000000001	-0.000151296491998809\\
0.911999999999994	-0.000150174064988998\\
0.912999999999987	-0.000149057248801709\\
0.914999999999973	-0.000146840230270077\\
0.916999999999993	-0.000144645001859707\\
0.917000000000001	-0.000144645001859699\\
0.919999999999993	-0.000141392075653412\\
0.92	-0.000141392075653405\\
0.922999999999993	-0.000138185775500394\\
0.925999999999986	-0.000135024687369718\\
0.925999999999993	-0.00013502468736971\\
0.926	-0.000135024687369702\\
0.927999999999993	-0.000132941723821195\\
0.928000000000001	-0.000132941723821187\\
0.929999999999994	-0.000130877826253349\\
0.931999999999987	-0.000128832590125961\\
0.933999999999994	-0.000126805614564855\\
0.934000000000001	-0.000126805614564848\\
0.937999999999987	-0.000122820316751409\\
0.939999999999993	-0.000120865077495109\\
0.940000000000001	-0.000120865077495103\\
0.943999999999987	-0.000117027502047741\\
0.944999999999994	-0.000116083021664212\\
0.945000000000001	-0.000116083021664206\\
0.945999999999993	-0.000115144413676528\\
0.946000000000001	-0.000115144413676522\\
0.946999999999994	-0.000114211632093562\\
0.947999999999987	-0.000113284631207966\\
0.949999999999973	-0.000111447790116981\\
0.952	-0.000109633530375599\\
0.952000000000008	-0.000109633530375592\\
0.95599999999998	-0.000106071336901298\\
0.956999999999994	-0.000105194351508317\\
0.957000000000001	-0.00010519435150831\\
0.96	-0.000102595257840317\\
0.960000000000008	-0.000102595257840311\\
0.963000000000007	-0.000100043069139321\\
0.966000000000007	-9.7536659848261e-05\\
0.966000000000014	-9.75366598482552e-05\\
0.969000000000007	-9.50749245916848e-05\\
0.969000000000014	-9.5074924591679e-05\\
0.972000000000007	-9.26625234900481e-05\\
0.975	-9.03041383956829e-05\\
0.979999999999994	-8.6490717806386e-05\\
0.980000000000001	-8.64907178063807e-05\\
0.985999999999987	-8.21023051300174e-05\\
0.986000000000001	-8.21023051300074e-05\\
0.991999999999987	-7.79113493409311e-05\\
0.992000000000001	-7.79113493409216e-05\\
0.997999999999987	-7.3910456506309e-05\\
0.998000000000001	-7.39104565062999e-05\\
0.999999999999993	-7.26226099703555e-05\\
1	-7.26226099703509e-05\\
1.00199999999999	-7.13643974323865e-05\\
1.00399999999999	-7.01355722724094e-05\\
1.00599999999999	-6.8935893635937e-05\\
1.006	-6.89358936359286e-05\\
1.00999999999999	-6.66230410409537e-05\\
1.01399999999997	-6.44240262516722e-05\\
1.01499999999999	-6.38918557160394e-05\\
1.015	-6.38918557160319e-05\\
1.01999999999999	-6.13352024439096e-05\\
1.02	-6.13352024439025e-05\\
1.02499999999999	-5.89499490360049e-05\\
1.02599999999999	-5.84932055236174e-05\\
1.026	-5.8493205523611e-05\\
1.02699999999999	-5.80431780747783e-05\\
1.027	-5.8043178074772e-05\\
1.02799999999999	-5.75982749071178e-05\\
1.02899999999999	-5.71569044873816e-05\\
1.03099999999997	-5.62846755547533e-05\\
1.03299999999999	-5.54263203166417e-05\\
1.033	-5.54263203166356e-05\\
1.03699999999997	-5.37505606517162e-05\\
1.04	-5.25289275156282e-05\\
1.04000000000001	-5.25289275156224e-05\\
1.04399999999999	-5.09459744840712e-05\\
1.044	-5.09459744840656e-05\\
1.046	-5.01738158365423e-05\\
1.04600000000001	-5.01738158365368e-05\\
1.04800000000001	-4.94143331365673e-05\\
1.05	-4.86673775225938e-05\\
1.05000000000001	-4.86673775225885e-05\\
1.05200000000001	-4.79328025889087e-05\\
1.05200000000002	-4.79328025889035e-05\\
1.05400000000002	-4.72100886209416e-05\\
1.05600000000002	-4.64987182289034e-05\\
1.05800000000001	-4.57985519814528e-05\\
1.05800000000002	-4.57985519814479e-05\\
1.05999999999999	-4.51094526442447e-05\\
1.06	-4.51094526442399e-05\\
1.06199999999996	-4.44312851520494e-05\\
1.06399999999992	-4.37639165814064e-05\\
1.06599999999999	-4.31072161254649e-05\\
1.066	-4.31072161254602e-05\\
1.06999999999992	-4.18253067642421e-05\\
1.07299999999999	-4.08909361827503e-05\\
1.073	-4.0890936182746e-05\\
1.07699999999992	-3.96804071947624e-05\\
1.07899999999999	-3.90900031293295e-05\\
1.079	-3.90900031293254e-05\\
1.07999999999999	-3.87981956482676e-05\\
1.08	-3.87981956482634e-05\\
1.08099999999999	-3.85082726205468e-05\\
1.08199999999999	-3.82202198396166e-05\\
1.08399999999997	-3.76496686504906e-05\\
1.08499999999999	-3.73671422849065e-05\\
1.085	-3.73671422849025e-05\\
1.08599999999999	-3.70864302505011e-05\\
1.086	-3.70864302504971e-05\\
1.08699999999999	-3.68075187925797e-05\\
1.08799999999999	-3.65303942441707e-05\\
1.08999999999997	-3.59814516461387e-05\\
1.09199999999999	-3.54394948615293e-05\\
1.092	-3.54394948615255e-05\\
1.09599999999997	-3.43761151817767e-05\\
1.09999999999995	-3.33394214604027e-05\\
1.09999999999997	-3.33394214603956e-05\\
1.1	-3.33394214603886e-05\\
1.10199999999999	-3.28308270200997e-05\\
1.102	-3.28308270200961e-05\\
1.10399999999999	-3.23286008688398e-05\\
1.10599999999997	-3.18326445609249e-05\\
1.10599999999999	-3.18326445609213e-05\\
1.106	-3.18326445609177e-05\\
1.10999999999997	-3.08591538488759e-05\\
1.11199999999999	-3.03814286371267e-05\\
1.112	-3.03814286371234e-05\\
1.11599999999997	-2.94480294466614e-05\\
1.11999999999994	-2.85464106227169e-05\\
1.12	-2.85464106227042e-05\\
1.12000000000001	-2.85464106227011e-05\\
1.126	-2.7252030869233e-05\\
1.12600000000001	-2.725203086923e-05\\
1.13099999999999	-2.62251417991066e-05\\
1.131	-2.62251417991037e-05\\
1.132	-2.6025288472609e-05\\
1.13200000000001	-2.60252884726062e-05\\
1.13300000000001	-2.58272535828071e-05\\
1.13400000000001	-2.5631027425519e-05\\
1.13600000000001	-2.52439629361489e-05\\
1.138	-2.48640191383652e-05\\
1.13800000000001	-2.48640191383625e-05\\
1.13999999999999	-2.44901822689232e-05\\
1.14	-2.44901822689206e-05\\
1.14199999999997	-2.41214397615318e-05\\
1.14399999999994	-2.3757719341231e-05\\
1.14599999999999	-2.33989497174015e-05\\
1.146	-2.3398949717399e-05\\
1.14999999999994	-2.26959825370229e-05\\
1.15399999999989	-2.20119870578637e-05\\
1.15499999999999	-2.18438893475066e-05\\
1.155	-2.18438893475042e-05\\
1.15999999999999	-2.10203966062128e-05\\
1.16	-2.10203966062105e-05\\
1.16499999999999	-2.02244954409214e-05\\
1.16599999999999	-2.00685398077086e-05\\
1.166	-2.00685398077064e-05\\
1.17099999999999	-1.93044643132627e-05\\
1.17199999999999	-1.9154737332757e-05\\
1.172	-1.91547373327549e-05\\
1.173	-1.90062037346705e-05\\
1.17300000000001	-1.90062037346684e-05\\
1.17400000000001	-1.88590374865917e-05\\
1.17500000000001	-1.87132313770234e-05\\
1.17700000000001	-1.84256710615945e-05\\
1.17999999999999	-1.80043551613652e-05\\
1.18	-1.80043551613633e-05\\
1.184	-1.7461025495754e-05\\
1.18599999999999	-1.71971369393908e-05\\
1.186	-1.71971369393889e-05\\
1.18899999999999	-1.68108786213554e-05\\
1.189	-1.68108786213536e-05\\
1.18999999999999	-1.66846537550029e-05\\
1.19	-1.66846537550012e-05\\
1.19099999999999	-1.65596824819567e-05\\
1.19199999999999	-1.64359586785654e-05\\
1.19399999999997	-1.61922292916064e-05\\
1.19499999999999	-1.60722117651356e-05\\
1.195	-1.60722117651339e-05\\
1.19899999999997	-1.55998671664565e-05\\
1.19999999999999	-1.54834052032631e-05\\
1.2	-1.54834052032615e-05\\
1.20399999999997	-1.50238828899385e-05\\
1.20599999999999	-1.4797838355262e-05\\
1.206	-1.47978383552604e-05\\
1.20699999999999	-1.46857257808375e-05\\
1.207	-1.46857257808359e-05\\
1.20799999999999	-1.45742123338824e-05\\
1.20899999999999	-1.44632925501146e-05\\
1.21099999999997	-1.42432122605927e-05\\
1.21499999999995	-1.38099384121128e-05\\
1.21799999999999	-1.34908419348825e-05\\
1.218	-1.3490841934881e-05\\
1.21999999999999	-1.32808225252269e-05\\
1.22	-1.32808225252255e-05\\
1.22199999999999	-1.30729242920309e-05\\
1.22399999999997	-1.28671064834061e-05\\
1.22499999999999	-1.27649651159534e-05\\
1.225	-1.27649651159519e-05\\
1.22599999999999	-1.26633287582128e-05\\
1.226	-1.26633287582114e-05\\
1.22699999999999	-1.25621924300732e-05\\
1.22799999999999	-1.24615511757429e-05\\
1.22999999999997	-1.22617341867563e-05\\
1.23	-1.22617341867536e-05\\
1.23399999999997	-1.18685858088014e-05\\
1.236	-1.1675367377563e-05\\
1.23600000000001	-1.16753673775617e-05\\
1.23999999999999	-1.12954526665125e-05\\
1.24	-1.12954526665112e-05\\
1.24399999999997	-1.09239866368018e-05\\
1.24599999999999	-1.07413306097238e-05\\
1.246	-1.07413306097225e-05\\
1.247	-1.06507561147432e-05\\
1.24700000000001	-1.06507561147419e-05\\
1.24800000000001	-1.05606780398285e-05\\
1.24900000000001	-1.04710919710537e-05\\
1.25100000000001	-1.02933783168721e-05\\
1.253	-1.01175803196372e-05\\
1.25300000000001	-1.0117580319636e-05\\
1.25700000000001	-9.77290823288604e-06\\
1.25999999999999	-9.5209048441266e-06\\
1.26	-9.52090484412542e-06\\
1.264	-9.19335659052538e-06\\
1.266	-9.03313421889334e-06\\
1.26600000000001	-9.0331342188922e-06\\
1.27000000000001	-8.71963595975306e-06\\
1.272	-8.56629862537267e-06\\
1.27200000000001	-8.56629862537159e-06\\
1.276	-8.26629726846761e-06\\
1.27600000000001	-8.26629726846656e-06\\
1.27999999999999	-7.97499837083127e-06\\
1.28000000000001	-7.97499837083025e-06\\
1.28399999999999	-7.69217353920681e-06\\
1.28599999999999	-7.55386934669907e-06\\
1.28600000000001	-7.5538693466981e-06\\
1.288	-7.4176010242562e-06\\
1.28800000000001	-7.41760102425524e-06\\
1.29	-7.28349907203066e-06\\
1.29199999999999	-7.15169441461058e-06\\
1.29499999999999	-6.95823846590688e-06\\
1.295	-6.95823846590598e-06\\
1.29899999999998	-6.70810033377469e-06\\
1.29999999999999	-6.6469386986188e-06\\
1.3	-6.64693869861793e-06\\
1.30399999999998	-6.40769383803153e-06\\
1.30499999999999	-6.34921838904478e-06\\
1.305	-6.34921838904395e-06\\
1.30599999999999	-6.29127147678454e-06\\
1.306	-6.29127147678373e-06\\
1.30699999999999	-6.23385026189332e-06\\
1.30700000000001	-6.23385026189251e-06\\
1.308	-6.17695193066612e-06\\
1.30899999999999	-6.12057369507411e-06\\
1.31099999999998	-6.00936648596339e-06\\
1.31300000000001	-5.90020683751614e-06\\
1.31300000000002	-5.90020683751537e-06\\
1.31699999999999	-5.68835595601555e-06\\
1.31999999999999	-5.53522588918257e-06\\
1.32	-5.53522588918185e-06\\
1.32399999999997	-5.33860152695866e-06\\
1.32599999999999	-5.24348118236777e-06\\
1.326	-5.2434811823671e-06\\
1.32999999999997	-5.05953094337961e-06\\
1.33	-5.05953094337843e-06\\
1.33399999999997	-4.88384846585816e-06\\
1.334	-4.88384846585701e-06\\
1.33799999999997	-4.71629600574267e-06\\
1.34	-4.63552723297305e-06\\
1.34000000000001	-4.63552723297248e-06\\
1.34399999999999	-4.47887982618637e-06\\
1.346	-4.40270909690946e-06\\
1.34600000000001	-4.40270909690892e-06\\
1.348	-4.32795368402734e-06\\
1.34800000000001	-4.32795368402681e-06\\
1.35	-4.25459893528839e-06\\
1.35199999999999	-4.18263047287656e-06\\
1.35599999999996	-4.04279625178921e-06\\
1.35999999999999	-3.90834138206002e-06\\
1.36	-3.90834138205955e-06\\
1.36299999999999	-3.81096685718704e-06\\
1.363	-3.81096685718659e-06\\
1.36499999999999	-3.74767730526322e-06\\
1.365	-3.74767730526278e-06\\
1.36599999999999	-3.71651589275579e-06\\
1.366	-3.71651589275535e-06\\
1.36699999999999	-3.6856746814089e-06\\
1.36799999999999	-3.65515215999635e-06\\
1.36999999999997	-3.59505722007696e-06\\
1.37199999999999	-3.53621929415617e-06\\
1.372	-3.53621929415575e-06\\
1.37599999999997	-3.42215863432302e-06\\
1.378	-3.36688605426768e-06\\
1.37800000000001	-3.36688605426729e-06\\
1.37999999999999	-3.31277078635789e-06\\
1.38	-3.31277078635751e-06\\
1.38199999999997	-3.25980222386241e-06\\
1.38399999999994	-3.20796998473659e-06\\
1.38599999999999	-3.1572639096593e-06\\
1.386	-3.15726390965894e-06\\
1.38999999999994	-3.05919071640367e-06\\
1.39199999999999	-3.0118043754947e-06\\
1.392	-3.01180437549437e-06\\
1.39599999999994	-2.92028576405832e-06\\
1.39799999999999	-2.87613555552382e-06\\
1.398	-2.87613555552351e-06\\
1.39999999999999	-2.83284513793707e-06\\
1.4	-2.83284513793677e-06\\
1.40199999999999	-2.79020469373484e-06\\
1.40399999999997	-2.74820586522469e-06\\
1.40599999999999	-2.70684042047345e-06\\
1.406	-2.70684042047316e-06\\
1.40999999999997	-2.62597737390653e-06\\
1.412	-2.5864639226192e-06\\
1.41200000000001	-2.58646392261892e-06\\
1.41599999999999	-2.50923443881342e-06\\
1.41999999999996	-2.43435124846182e-06\\
1.41999999999998	-2.43435124846142e-06\\
1.42	-2.43435124846102e-06\\
1.42099999999999	-2.41599013891771e-06\\
1.421	-2.41599013891745e-06\\
1.42199999999999	-2.39777109495273e-06\\
1.42299999999999	-2.37969322362116e-06\\
1.42499999999997	-2.34395746245432e-06\\
1.42599999999999	-2.32629782154287e-06\\
1.426	-2.32629782154262e-06\\
1.42999999999997	-2.25702740992588e-06\\
1.43199999999999	-2.22320123860696e-06\\
1.432	-2.22320123860672e-06\\
1.43499999999999	-2.17355606186817e-06\\
1.435	-2.17355606186794e-06\\
1.43799999999998	-2.12528554065162e-06\\
1.43999999999999	-2.0938583533811e-06\\
1.44	-2.09385835338088e-06\\
1.44299999999999	-2.04783191295891e-06\\
1.44599999999997	-2.00312468166224e-06\\
1.44599999999999	-2.00312468166201e-06\\
1.446	-2.00312468166179e-06\\
1.44699999999999	-1.98851200435102e-06\\
1.447	-1.98851200435081e-06\\
1.44799999999999	-1.97404299441789e-06\\
1.44899999999999	-1.95971694281675e-06\\
1.44999999999999	-1.94553314756733e-06\\
1.45	-1.94553314756713e-06\\
1.45199999999999	-1.91758955302813e-06\\
1.45399999999997	-1.89020673376602e-06\\
1.45599999999999	-1.86337932264217e-06\\
1.456	-1.86337932264198e-06\\
1.45999999999997	-1.8108746123559e-06\\
1.46	-1.81087461235553e-06\\
1.46000000000001	-1.81087461235534e-06\\
1.46399999999999	-1.75953906842417e-06\\
1.466	-1.7342971208543e-06\\
1.46600000000001	-1.73429712085412e-06\\
1.46999999999999	-1.68464013609193e-06\\
1.47	-1.68464013609176e-06\\
1.47399999999997	-1.63605334338901e-06\\
1.47599999999999	-1.6121493325462e-06\\
1.476	-1.61214933254603e-06\\
1.47899999999998	-1.57676685289909e-06\\
1.479	-1.57676685289892e-06\\
1.47999999999999	-1.56509665470064e-06\\
1.48	-1.56509665470047e-06\\
1.48099999999999	-1.553487481758e-06\\
1.48199999999999	-1.54193876521626e-06\\
1.48399999999997	-1.51902044071195e-06\\
1.48599999999999	-1.4963371891074e-06\\
1.486	-1.49633718910724e-06\\
1.48999999999997	-1.45165816591769e-06\\
1.491	-1.44062843834867e-06\\
1.49100000000001	-1.44062843834851e-06\\
1.49499999999999	-1.39709761722863e-06\\
1.49899999999996	-1.35449992138319e-06\\
1.49999999999999	-1.34399236851035e-06\\
1.5	-1.3439923685102e-06\\
1.50499999999999	-1.29228012082052e-06\\
1.505	-1.29228012082037e-06\\
1.50599999999999	-1.28209922717746e-06\\
1.506	-1.28209922717731e-06\\
1.50699999999999	-1.27197101536964e-06\\
1.50799999999999	-1.2618949890911e-06\\
1.508	-1.26189498909096e-06\\
1.50999999999999	-1.24189752074501e-06\\
1.51199999999997	-1.2221029025715e-06\\
1.51399999999999	-1.20250725474878e-06\\
1.514	-1.20250725474864e-06\\
1.51799999999997	-1.1640287229794e-06\\
1.518	-1.16402872297914e-06\\
1.51999999999999	-1.14517108992208e-06\\
1.52	-1.14517108992195e-06\\
1.52199999999999	-1.1265629340214e-06\\
1.52399999999997	-1.10820060800662e-06\\
1.52599999999999	-1.09008051279067e-06\\
1.526	-1.09008051279054e-06\\
1.52999999999997	-1.05455285521203e-06\\
1.53399999999994	-1.01995210576965e-06\\
1.537	-9.94593480424472e-07\\
1.53700000000001	-9.94593480424353e-07\\
1.53999999999999	-9.69729782962207e-07\\
1.54	-9.69729782962091e-07\\
1.54299999999997	-9.45350048435924e-07\\
1.54599999999994	-9.21443524626113e-07\\
1.54599999999997	-9.21443524625894e-07\\
1.546	-9.21443524625673e-07\\
1.549	-8.97999668287964e-07\\
1.54900000000001	-8.97999668287854e-07\\
1.55200000000001	-8.75053552116595e-07\\
1.55500000000001	-8.52640468075433e-07\\
1.55500000000003	-8.52640468075328e-07\\
1.55999999999999	-8.16443101525268e-07\\
1.56	-8.16443101525167e-07\\
1.56499999999996	-7.81654671402262e-07\\
1.56599999999998	-7.7486227737189e-07\\
1.566	-7.74862277371794e-07\\
1.57099999999996	-7.41708460508551e-07\\
1.57199999999998	-7.35237053425477e-07\\
1.572	-7.35237053425385e-07\\
1.57499999999999	-7.16147748981336e-07\\
1.575	-7.16147748981247e-07\\
1.57799999999999	-6.97545267539713e-07\\
1.57999999999999	-6.85410010373826e-07\\
1.58	-6.85410010373741e-07\\
1.58299999999999	-6.67600773846542e-07\\
1.58599999999997	-6.50256950296673e-07\\
1.586	-6.50256950296515e-07\\
1.58699999999999	-6.44577772981549e-07\\
1.587	-6.44577772981469e-07\\
1.58799999999999	-6.38949178422497e-07\\
1.58899999999999	-6.3337089079341e-07\\
1.59099999999997	-6.22364145427712e-07\\
1.59499999999995	-6.00942474538823e-07\\
1.59499999999997	-6.00942474538682e-07\\
1.595	-6.00942474538541e-07\\
1.59999999999999	-5.75253545681064e-07\\
1.6	-5.75253545680993e-07\\
1.60499999999999	-5.50744378397886e-07\\
1.60599999999999	-5.45981448921479e-07\\
1.606	-5.45981448921412e-07\\
1.60699999999998	-5.41264270937806e-07\\
1.607	-5.41264270937739e-07\\
1.60799999999999	-5.36591460388822e-07\\
1.60899999999999	-5.31961635370232e-07\\
1.60999999999998	-5.273745690194e-07\\
1.61	-5.27374569019336e-07\\
1.61199999999999	-5.18327815337533e-07\\
1.61399999999997	-5.09449426150442e-07\\
1.61599999999999	-5.0073766125941e-07\\
1.616	-5.00737661259349e-07\\
1.61999999999997	-4.83807206584525e-07\\
1.61999999999999	-4.83807206584465e-07\\
1.62	-4.83807206584405e-07\\
1.62399999999997	-4.67523176962907e-07\\
1.624	-4.67523176962808e-07\\
1.62599999999999	-4.59619562172117e-07\\
1.626	-4.59619562172062e-07\\
1.62799999999999	-4.51872804875926e-07\\
1.62999999999998	-4.44281386679705e-07\\
1.63199999999999	-4.36843819636052e-07\\
1.632	-4.36843819635999e-07\\
1.63599999999998	-4.22417270529958e-07\\
1.63999999999995	-4.08574679174971e-07\\
1.63999999999998	-4.0857467917489e-07\\
1.64	-4.08574679174809e-07\\
1.645	-3.92076139029517e-07\\
1.64500000000001	-3.92076139029472e-07\\
1.64599999999999	-3.88882094301436e-07\\
1.646	-3.88882094301391e-07\\
1.64699999999999	-3.85722901474031e-07\\
1.64799999999999	-3.82598405739308e-07\\
1.64999999999997	-3.7645289483677e-07\\
1.65199999999999	-3.70444357049822e-07\\
1.652	-3.70444357049779e-07\\
1.65299999999998	-3.67491084004262e-07\\
1.653	-3.6749108400422e-07\\
1.65399999999999	-3.64571614691419e-07\\
1.65499999999997	-3.61685806053969e-07\\
1.65699999999995	-3.56014606826774e-07\\
1.65899999999998	-3.50476374745519e-07\\
1.659	-3.5047637474548e-07\\
1.65999999999999	-3.47751586552093e-07\\
1.66	-3.47751586552055e-07\\
1.66099999999999	-3.45049245653866e-07\\
1.66199999999998	-3.42369219633085e-07\\
1.66399999999995	-3.37075588023497e-07\\
1.66599999999999	-3.31869654150935e-07\\
1.666	-3.31869654150898e-07\\
1.66999999999995	-3.21716815105324e-07\\
1.67399999999991	-3.11902742146659e-07\\
1.67999999999998	-2.97800057981531e-07\\
1.68	-2.97800057981499e-07\\
1.68199999999998	-2.93260375010377e-07\\
1.682	-2.93260375010345e-07\\
1.68399999999998	-2.88799801928544e-07\\
1.68599999999997	-2.84417464378762e-07\\
1.68599999999999	-2.84417464378729e-07\\
1.686	-2.84417464378698e-07\\
1.68999999999997	-2.75884075232543e-07\\
1.69199999999999	-2.7173135106169e-07\\
1.692	-2.71731351061661e-07\\
1.69599999999997	-2.6365531746971e-07\\
1.69799999999999	-2.59731811004002e-07\\
1.698	-2.59731811003975e-07\\
1.69999999999999	-2.55883614430066e-07\\
1.7	-2.55883614430039e-07\\
1.70199999999999	-2.52109973491559e-07\\
1.70399999999998	-2.48410148540172e-07\\
1.70599999999999	-2.44783414395868e-07\\
1.706	-2.44783414395842e-07\\
1.70999999999998	-2.37746389321702e-07\\
1.71099999999998	-2.36031721247582e-07\\
1.711	-2.36031721247558e-07\\
1.71499999999997	-2.29348881058384e-07\\
1.715	-2.29348881058343e-07\\
1.71699999999998	-2.26111817106046e-07\\
1.717	-2.26111817106023e-07\\
1.71899999999998	-2.22928827510215e-07\\
1.71999999999999	-2.21351923118652e-07\\
1.72	-2.21351923118629e-07\\
1.72199999999999	-2.18226908801564e-07\\
1.72399999999997	-2.15139777460543e-07\\
1.72599999999999	-2.12089924005886e-07\\
1.726	-2.12089924005865e-07\\
1.72899999999998	-2.07583734189946e-07\\
1.729	-2.07583734189925e-07\\
1.73199999999998	-2.03158088998718e-07\\
1.73499999999997	-1.98811036650114e-07\\
1.73999999999998	-1.91735408499294e-07\\
1.74	-1.91735408499275e-07\\
1.74599999999997	-1.83513629856202e-07\\
1.74599999999998	-1.8351362985618e-07\\
1.746	-1.83513629856157e-07\\
1.74999999999998	-1.78188734508626e-07\\
1.75	-1.78188734508607e-07\\
1.75199999999998	-1.75571599313201e-07\\
1.752	-1.75571599313182e-07\\
1.75399999999998	-1.72991476358216e-07\\
1.75599999999997	-1.70455349769457e-07\\
1.75799999999998	-1.67962722456277e-07\\
1.758	-1.67962722456259e-07\\
1.75999999999999	-1.65513105857513e-07\\
1.76	-1.65513105857496e-07\\
1.76199999999999	-1.63106019842063e-07\\
1.76399999999998	-1.60740992611713e-07\\
1.76599999999999	-1.58417560611964e-07\\
1.766	-1.58417560611947e-07\\
1.76899999999998	-1.5500940969396e-07\\
1.769	-1.55009409693944e-07\\
1.77199999999998	-1.51692322238845e-07\\
1.77499999999996	-1.48464835356644e-07\\
1.77499999999998	-1.48464835356626e-07\\
1.775	-1.48464835356608e-07\\
1.78	-1.43225225325629e-07\\
1.78000000000002	-1.43225225325614e-07\\
1.78499999999998	-1.38112690231254e-07\\
1.785	-1.3811269023124e-07\\
1.78600000000001	-1.37104878361819e-07\\
1.78600000000003	-1.37104878361805e-07\\
1.78700000000005	-1.36101849095579e-07\\
1.78800000000006	-1.35103553281763e-07\\
1.79000000000009	-1.3312096657413e-07\\
1.79200000000003	-1.31156729644371e-07\\
1.79200000000004	-1.31156729644357e-07\\
1.79600000000011	-1.27281768658706e-07\\
1.79799999999998	-1.25370285095737e-07\\
1.798	-1.25370285095723e-07\\
1.8	-1.23475632153339e-07\\
1.80000000000002	-1.23475632153325e-07\\
1.80200000000002	-1.21597438474833e-07\\
1.80400000000002	-1.19735335927007e-07\\
1.806	-1.17888959530602e-07\\
1.80600000000002	-1.17888959530589e-07\\
1.80999999999998	-1.1424194062674e-07\\
1.81	-1.14241940626727e-07\\
1.81399999999997	-1.10662852743448e-07\\
1.81799999999994	-1.07158220123571e-07\\
1.82	-1.0543296359934e-07\\
1.82000000000001	-1.05432963599328e-07\\
1.826	-1.00361385610016e-07\\
1.82600000000001	-1.00361385610004e-07\\
1.827	-9.95308817704226e-08\\
1.82700000000001	-9.95308817704108e-08\\
1.828	-9.87044856343912e-08\\
1.82899999999999	-9.788215670494e-08\\
1.83099999999996	-9.62495394859337e-08\\
1.83200000000001	-9.54391711970167e-08\\
1.83200000000003	-9.54391711970052e-08\\
1.83599999999998	-9.22363758777221e-08\\
1.83800000000001	-9.06576362891712e-08\\
1.83800000000003	-9.06576362891601e-08\\
1.83999999999999	-8.91008341433525e-08\\
1.84	-8.91008341433416e-08\\
1.84199999999996	-8.75729099284202e-08\\
1.84399999999992	-8.60735641652866e-08\\
1.84599999999999	-8.46025029763092e-08\\
1.846	-8.46025029762989e-08\\
1.84999999999992	-8.17440864827255e-08\\
1.85399999999984	-7.89954193016209e-08\\
1.85499999999998	-7.83251464120328e-08\\
1.855	-7.83251464120233e-08\\
1.85599999999998	-7.76615649058536e-08\\
1.856	-7.76615649058442e-08\\
1.85699999999999	-7.70046422748691e-08\\
1.85799999999997	-7.63543463220908e-08\\
1.85999999999995	-7.50735073157444e-08\\
1.85999999999998	-7.50735073157281e-08\\
1.86	-7.50735073157118e-08\\
1.86399999999995	-7.25899689623508e-08\\
1.86599999999999	-7.13867828330052e-08\\
1.866	-7.13867828329967e-08\\
1.86799999999998	-7.02090026227763e-08\\
1.868	-7.02090026227681e-08\\
1.86999999999998	-6.90524149336459e-08\\
1.87199999999996	-6.79128105198424e-08\\
1.87299999999998	-6.73493071099085e-08\\
1.873	-6.73493071099005e-08\\
1.87699999999996	-6.51366435477561e-08\\
1.87999999999999	-6.35198073114671e-08\\
1.88	-6.35198073114595e-08\\
1.88399999999997	-6.14195518673429e-08\\
1.88499999999998	-6.090423123367e-08\\
1.885	-6.09042312336627e-08\\
1.886	-6.03927571041919e-08\\
1.88600000000002	-6.03927571041846e-08\\
1.88700000000002	-5.98851044170427e-08\\
1.88800000000002	-5.93812482966702e-08\\
1.88999999999998	-5.8384827184874e-08\\
1.89	-5.8384827184867e-08\\
1.892	-5.74032984670194e-08\\
1.89200000000002	-5.74032984670125e-08\\
1.89400000000002	-5.64369952288924e-08\\
1.89600000000003	-5.54862535405005e-08\\
1.898	-5.45508870528139e-08\\
1.89800000000002	-5.45508870528073e-08\\
1.9	-5.36307124316243e-08\\
1.90000000000002	-5.36307124316178e-08\\
1.902	-5.27255493203731e-08\\
1.90399999999998	-5.18352203035764e-08\\
1.906	-5.09595508732905e-08\\
1.90600000000002	-5.09595508732844e-08\\
1.90999999999998	-4.92515070789121e-08\\
1.91399999999995	-4.76000787427945e-08\\
1.91399999999997	-4.7600078742784e-08\\
1.914	-4.76000787427735e-08\\
1.91999999999998	-4.52262697180054e-08\\
1.92	-4.5226269718e-08\\
1.92499999999998	-4.33371521238859e-08\\
1.925	-4.33371521238806e-08\\
1.92599999999998	-4.29684259535666e-08\\
1.926	-4.29684259535613e-08\\
1.92699999999999	-4.26026897307241e-08\\
1.92799999999997	-4.22399255334169e-08\\
1.92999999999995	-4.1523242258079e-08\\
1.93199999999998	-4.08182356548415e-08\\
1.932	-4.08182356548365e-08\\
1.93599999999995	-3.94427019945113e-08\\
1.9399999999999	-3.81122460630739e-08\\
1.93999999999999	-3.81122460630458e-08\\
1.94	-3.81122460630411e-08\\
1.94299999999998	-3.71433575371429e-08\\
1.943	-3.71433575371384e-08\\
1.94599999999998	-3.61988104751906e-08\\
1.946	-3.61988104751841e-08\\
1.94899999999998	-3.5278188302255e-08\\
1.95199999999997	-3.43810850090458e-08\\
1.95199999999998	-3.43810850090406e-08\\
1.952	-3.43810850090353e-08\\
1.95799999999997	-3.26625036054325e-08\\
1.95999999999998	-3.21126215778703e-08\\
1.96	-3.21126215778664e-08\\
1.96599999999996	-3.05304017833844e-08\\
1.96599999999998	-3.05304017833799e-08\\
1.966	-3.05304017833754e-08\\
1.97199999999996	-2.90471733417579e-08\\
1.97199999999998	-2.90471733417538e-08\\
1.972	-2.90471733417496e-08\\
1.97799999999996	-2.76603194533904e-08\\
1.97799999999998	-2.76603194533865e-08\\
1.978	-2.76603194533826e-08\\
1.98	-2.7217379652368e-08\\
1.98000000000002	-2.72173796523649e-08\\
1.98200000000002	-2.67815000429602e-08\\
1.98400000000002	-2.63525951892528e-08\\
1.986	-2.59305810242426e-08\\
1.98600000000002	-2.59305810242396e-08\\
1.99000000000002	-2.51068952315779e-08\\
1.99400000000003	-2.43097968528297e-08\\
1.995	-2.411460285019e-08\\
1.99500000000001	-2.41146028501873e-08\\
1.99999999999999	-2.31626391432532e-08\\
2	-2.31626391432506e-08\\
2.00099999999997	-2.29769823588253e-08\\
2.001	-2.29769823588201e-08\\
2.00199999999999	-2.27928828202309e-08\\
2.00299999999997	-2.26103315061639e-08\\
2.00499999999995	-2.22498378468183e-08\\
2.00599999999997	-2.20718778371068e-08\\
2.006	-2.20718778371018e-08\\
2.00999999999995	-2.13750806589733e-08\\
2.01199999999997	-2.10355885404165e-08\\
2.012	-2.10355885404117e-08\\
2.01299999999998	-2.08681600833547e-08\\
2.01300000000001	-2.086816008335e-08\\
2.014	-2.07024261118696e-08\\
2.01499999999999	-2.05383785048111e-08\\
2.01699999999996	-2.02153103127361e-08\\
2.01999999999997	-1.97431574631391e-08\\
2.02	-1.97431574631347e-08\\
2.02399999999995	-1.91365370702398e-08\\
2.02599999999997	-1.88429142116733e-08\\
2.026	-1.88429142116692e-08\\
2.02999999999995	-1.82747554047517e-08\\
2.03	-1.82747554047456e-08\\
2.03399999999995	-1.77316766630693e-08\\
2.03599999999997	-1.74694084946324e-08\\
2.036	-1.74694084946287e-08\\
2.03999999999995	-1.6957670134545e-08\\
2.04	-1.69576701345384e-08\\
2.04399999999995	-1.64590043940177e-08\\
2.04599999999997	-1.62144511828978e-08\\
2.046	-1.62144511828944e-08\\
2.04799999999997	-1.59730202932478e-08\\
2.048	-1.59730202932443e-08\\
2.04999999999996	-1.57346644039945e-08\\
2.05199999999993	-1.54993367964626e-08\\
2.05599999999987	-1.50375825112089e-08\\
2.05899999999997	-1.46988772047929e-08\\
2.059	-1.46988772047897e-08\\
2.05999999999997	-1.45873953924117e-08\\
2.06	-1.45873953924086e-08\\
2.06099999999999	-1.44766144260293e-08\\
2.06199999999998	-1.43665288761049e-08\\
2.06399999999995	-1.41484224825758e-08\\
2.06499999999997	-1.40403909516116e-08\\
2.065	-1.40403909516085e-08\\
2.06599999999997	-1.39330334620676e-08\\
2.066	-1.39330334620645e-08\\
2.06699999999999	-1.38263447535004e-08\\
2.06799999999998	-1.37203195980469e-08\\
2.06999999999995	-1.35102391977167e-08\\
2.07099999999997	-1.34061736587596e-08\\
2.071	-1.34061736587566e-08\\
2.07499999999995	-1.29961670106311e-08\\
2.07699999999997	-1.27948282383822e-08\\
2.077	-1.27948282383794e-08\\
2.07999999999997	-1.24972896109556e-08\\
2.08	-1.24972896109528e-08\\
2.08299999999998	-1.22049979556526e-08\\
2.08599999999995	-1.19178243670625e-08\\
2.086	-1.1917824367058e-08\\
2.08799999999997	-1.17291560439996e-08\\
2.088	-1.17291560439969e-08\\
2.08999999999997	-1.15426691051607e-08\\
2.09199999999993	-1.13583269976241e-08\\
2.09399999999997	-1.11760935896177e-08\\
2.094	-1.11760935896151e-08\\
2.09799999999993	-1.08186796305074e-08\\
2.09999999999997	-1.06436463204004e-08\\
2.1	-1.0643646320398e-08\\
2.10399999999993	-1.03007554974024e-08\\
2.10599999999997	-1.01328307765728e-08\\
2.106	-1.01328307765704e-08\\
2.10999999999993	-9.80385814135552e-09\\
2.112	-9.64274574704631e-09\\
2.11200000000003	-9.64274574704404e-09\\
2.11599999999996	-9.3271109065708e-09\\
2.11699999999997	-9.24955250837899e-09\\
2.117	-9.24955250837679e-09\\
2.11999999999997	-9.02005714984252e-09\\
2.12	-9.02005714984037e-09\\
2.12299999999998	-8.79525309737646e-09\\
2.12599999999995	-8.57504120870862e-09\\
2.126	-8.5750412087051e-09\\
2.12899999999997	-8.35932436608786e-09\\
2.129	-8.35932436608584e-09\\
2.13199999999996	-8.14831987337215e-09\\
2.13499999999993	-7.94224711040477e-09\\
2.13499999999997	-7.94224711040248e-09\\
2.135	-7.94224711040019e-09\\
2.13999999999997	-7.60950603390769e-09\\
2.14	-7.60950603390584e-09\\
2.14499999999998	-7.28980498755184e-09\\
2.14599999999997	-7.22739483875215e-09\\
2.146	-7.22739483875038e-09\\
2.14699999999997	-7.16548751651524e-09\\
2.14699999999999	-7.16548751651348e-09\\
2.14799999999998	-7.10407998750108e-09\\
2.14899999999997	-7.04316924259715e-09\\
2.15099999999995	-6.9228261907563e-09\\
2.15299999999999	-6.80443477328446e-09\\
2.15300000000002	-6.80443477328279e-09\\
2.15699999999997	-6.57376022468586e-09\\
2.15999999999997	-6.40616154117094e-09\\
2.16	-6.40616154116937e-09\\
2.16399999999995	-6.18976573636064e-09\\
2.16599999999997	-6.08454962395508e-09\\
2.166	-6.0845496239536e-09\\
2.16999999999995	-5.87997786191998e-09\\
2.17	-5.87997786191764e-09\\
2.17399999999995	-5.68308702417037e-09\\
2.17499999999997	-5.63504623800271e-09\\
2.175	-5.63504623800135e-09\\
2.17899999999995	-5.44754016342283e-09\\
2.17999999999997	-5.40181642747196e-09\\
2.18	-5.40181642747067e-09\\
2.18399999999995	-5.22346539767049e-09\\
2.18599999999997	-5.13698572108021e-09\\
2.186	-5.136985721079e-09\\
2.187	-5.09441236909743e-09\\
2.18700000000002	-5.09441236909622e-09\\
2.18800000000002	-5.05224649303867e-09\\
2.18800000000005	-5.05224649303748e-09\\
2.18900000000004	-5.01045196711214e-09\\
2.19000000000004	-4.96902674337888e-09\\
2.19200000000003	-4.88727610110798e-09\\
2.19600000000001	-4.72811832998898e-09\\
2.2	-4.5746483901216e-09\\
2.20000000000003	-4.57464839012053e-09\\
2.20399999999997	-4.42674595504692e-09\\
2.204	-4.42674595504589e-09\\
2.20499999999997	-4.3906265852502e-09\\
2.205	-4.39062658524918e-09\\
2.20599999999999	-4.35484614611123e-09\\
2.20600000000003	-4.35484614610972e-09\\
2.20700000000002	-4.31940288434744e-09\\
2.20800000000001	-4.28429506323391e-09\\
2.20999999999998	-4.2150788781444e-09\\
2.21200000000003	-4.14718402419941e-09\\
2.21200000000006	-4.14718402419846e-09\\
2.21600000000001	-4.01523781093887e-09\\
2.21800000000003	-3.9511437092303e-09\\
2.21800000000006	-3.9511437092294e-09\\
2.21999999999997	-3.88828544574074e-09\\
2.22	-3.88828544573985e-09\\
2.22199999999992	-3.82665070008822e-09\\
2.22399999999983	-3.76622739162316e-09\\
2.226	-3.70700367713569e-09\\
2.22600000000003	-3.70700367713485e-09\\
2.22999999999986	-3.59210883102941e-09\\
2.23299999999997	-3.5090019835504e-09\\
2.233	-3.50900198354963e-09\\
2.23699999999983	-3.40220888502965e-09\\
2.23899999999997	-3.35050988073183e-09\\
2.239	-3.3505098807311e-09\\
2.24	-3.32503858109399e-09\\
2.24000000000003	-3.32503858109327e-09\\
2.24100000000003	-3.2997621294787e-09\\
2.24200000000003	-3.27467928731174e-09\\
2.24400000000004	-3.22508952449019e-09\\
2.246	-3.17625957284491e-09\\
2.24600000000003	-3.17625957284423e-09\\
2.25000000000004	-3.0808409669833e-09\\
2.252	-3.03423361035298e-09\\
2.25200000000003	-3.03423361035232e-09\\
2.25600000000004	-2.94317711206793e-09\\
2.25999999999997	-2.85493897376965e-09\\
2.26	-2.85493897376903e-09\\
2.26199999999997	-2.81185508500588e-09\\
2.262	-2.81185508500528e-09\\
2.26399999999996	-2.76945001220587e-09\\
2.26599999999993	-2.72771544381106e-09\\
2.26599999999997	-2.72771544381033e-09\\
2.266	-2.7277154438096e-09\\
2.26999999999994	-2.64622522961595e-09\\
2.272	-2.60645361144537e-09\\
2.27200000000003	-2.6064536114448e-09\\
2.27499999999997	-2.5480754855516e-09\\
2.275	-2.54807548555105e-09\\
2.27799999999994	-2.49127721465512e-09\\
2.27999999999997	-2.45427699974086e-09\\
2.28	-2.45427699974033e-09\\
2.28299999999994	-2.4000564951968e-09\\
2.28599999999989	-2.34735056686552e-09\\
2.28599999999997	-2.34735056686401e-09\\
2.286	-2.34735056686351e-09\\
2.28699999999997	-2.33011445995248e-09\\
2.287	-2.33011445995199e-09\\
2.28799999999999	-2.31304320668827e-09\\
2.28899999999997	-2.29613597050458e-09\\
2.29099999999995	-2.26281024354173e-09\\
2.291	-2.26281024354096e-09\\
2.29499999999995	-2.19809107593592e-09\\
2.29699999999997	-2.16668495010748e-09\\
2.297	-2.16668495010704e-09\\
2.29999999999997	-2.12047063806669e-09\\
2.3	-2.12047063806626e-09\\
2.30299999999998	-2.07508842244407e-09\\
2.30599999999995	-2.0305182889387e-09\\
2.306	-2.03051828893801e-09\\
2.30999999999997	-1.97232075992021e-09\\
2.31	-1.9723207599198e-09\\
2.31399999999997	-1.91548639869764e-09\\
2.31599999999997	-1.8875664398366e-09\\
2.316	-1.88756643983621e-09\\
2.31999999999997	-1.83269360261039e-09\\
2.32	-1.83269360261001e-09\\
2.32399999999997	-1.7790744582363e-09\\
2.32599999999997	-1.75272184581226e-09\\
2.326	-1.75272184581188e-09\\
2.32999999999997	-1.70090471356611e-09\\
2.33199999999997	-1.67543003738564e-09\\
2.332	-1.67543003738528e-09\\
2.33599999999997	-1.62564104710781e-09\\
2.33999999999994	-1.57757850727103e-09\\
2.34	-1.57757850727027e-09\\
2.34000000000003	-1.57757850726994e-09\\
2.34499999999997	-1.51987087104506e-09\\
2.345	-1.51987087104473e-09\\
2.34600000000003	-1.5086397218698e-09\\
2.34600000000006	-1.50863972186949e-09\\
2.34700000000009	-1.49751073617442e-09\\
2.34800000000012	-1.48648336861221e-09\\
2.34899999999997	-1.47555707884112e-09\\
2.349	-1.47555707884081e-09\\
2.35100000000006	-1.45400559602965e-09\\
2.35300000000011	-1.4328520635825e-09\\
2.35499999999997	-1.41209233532746e-09\\
2.355	-1.41209233532717e-09\\
2.35800000000003	-1.38148800108238e-09\\
2.35800000000006	-1.38148800108209e-09\\
2.35999999999997	-1.36134978915803e-09\\
2.36	-1.36134978915775e-09\\
2.36199999999992	-1.34141870825641e-09\\
2.36399999999983	-1.32169085182004e-09\\
2.36599999999997	-1.30216235311065e-09\\
2.366	-1.30216235311038e-09\\
2.36999999999983	-1.26368815667936e-09\\
2.37399999999966	-1.22596595320688e-09\\
2.37799999999997	-1.188966166152e-09\\
2.378	-1.18896616615174e-09\\
2.37999999999997	-1.17072809182309e-09\\
2.38	-1.17072809182283e-09\\
2.38199999999997	-1.15265978632628e-09\\
2.38399999999995	-1.13475770794173e-09\\
2.38599999999997	-1.1170183477925e-09\\
2.386	-1.11701834779225e-09\\
2.38999999999995	-1.08201390558644e-09\\
2.39	-1.08201390558604e-09\\
2.39399999999995	-1.0477011583532e-09\\
2.396	-1.03082654402201e-09\\
2.39600000000002	-1.03082654402177e-09\\
2.39999999999997	-9.97624295936824e-10\\
2.4	-9.9762429593659e-10\\
2.40399999999995	-9.65129720981008e-10\\
2.40599999999997	-9.49139828099996e-10\\
2.406	-9.4913982809977e-10\\
2.40699999999997	-9.41207853686677e-10\\
2.407	-9.41207853686453e-10\\
2.40799999999998	-9.33317341725171e-10\\
2.40899999999997	-9.25467905571892e-10\\
2.41099999999995	-9.09890724187453e-10\\
2.41299999999997	-8.94473256345259e-10\\
2.413	-8.94473256345041e-10\\
2.41499999999997	-8.79243402562226e-10\\
2.415	-8.79243402562011e-10\\
2.41699999999997	-8.64229100127463e-10\\
2.41899999999995	-8.49427406179605e-10\\
2.41999999999997	-8.42105379150392e-10\\
2.42	-8.42105379150185e-10\\
2.42399999999995	-8.13334395457072e-10\\
2.426	-7.99254252777932e-10\\
2.42600000000003	-7.99254252777733e-10\\
2.427	-7.92289304838051e-10\\
2.42700000000003	-7.92289304837854e-10\\
2.42800000000002	-7.85373983310724e-10\\
2.429	-7.78507949337657e-10\\
2.43099999999998	-7.64922400704173e-10\\
2.43499999999993	-7.38328110650226e-10\\
2.43599999999997	-7.31797802427922e-10\\
2.436	-7.31797802427737e-10\\
2.43999999999997	-7.06140026998827e-10\\
2.44	-7.06140026998647e-10\\
2.44399999999998	-6.81208777546781e-10\\
2.446	-6.69009469560939e-10\\
2.44600000000003	-6.69009469560767e-10\\
2.448	-6.56984506389637e-10\\
2.44800000000002	-6.56984506389467e-10\\
2.44999999999999	-6.45146786155684e-10\\
2.45000000000002	-6.45146786155517e-10\\
2.45199999999998	-6.33509243666346e-10\\
2.45399999999995	-6.220695979262e-10\\
2.45600000000002	-6.10825606719827e-10\\
2.45600000000005	-6.10825606719669e-10\\
2.45999999999998	-5.88915810432843e-10\\
2.46000000000001	-5.8891581043269e-10\\
2.46399999999994	-5.67762676394512e-10\\
2.46499999999997	-5.62590664937169e-10\\
2.465	-5.62590664937023e-10\\
2.46599999999998	-5.57464652036668e-10\\
2.46600000000001	-5.57464652036523e-10\\
2.46699999999999	-5.5238438652194e-10\\
2.46799999999998	-5.47349619454327e-10\\
2.46999999999996	-5.37415596058988e-10\\
2.47199999999998	-5.27660634743381e-10\\
2.47200000000001	-5.27660634743244e-10\\
2.47599999999996	-5.08701837993906e-10\\
2.47999999999991	-4.90479917432525e-10\\
2.47999999999997	-4.90479917432237e-10\\
2.48	-4.9047991743211e-10\\
2.48499999999997	-4.68717037588441e-10\\
2.485	-4.6871703758832e-10\\
2.48599999999997	-4.64497581859411e-10\\
2.486	-4.64497581859292e-10\\
2.48699999999999	-4.60322012063178e-10\\
2.48799999999998	-4.56190123586914e-10\\
2.48999999999995	-4.48056582870958e-10\\
2.49199999999997	-4.40095365505868e-10\\
2.492	-4.40095365505757e-10\\
2.494	-4.32304911073312e-10\\
2.49400000000002	-4.32304911073202e-10\\
2.49600000000002	-4.24683692624427e-10\\
2.49800000000001	-4.17230216370744e-10\\
2.49999999999997	-4.09943021402142e-10\\
2.5	-4.09943021402039e-10\\
2.50399999999999	-3.95801568461802e-10\\
2.50599999999997	-3.88929482963257e-10\\
2.506	-3.8892948296316e-10\\
2.50999999999999	-3.75575858910049e-10\\
2.51399999999998	-3.62734281577958e-10\\
2.51999999999997	-3.44410086208213e-10\\
2.52	-3.44410086208129e-10\\
2.52299999999997	-3.3566138821935e-10\\
2.523	-3.35661388219268e-10\\
2.52599999999996	-3.27183105482373e-10\\
2.526	-3.27183105482271e-10\\
2.52899999999997	-3.18971498804839e-10\\
2.53199999999993	-3.11022946735238e-10\\
2.53199999999997	-3.11022946735153e-10\\
2.532	-3.11022946735068e-10\\
2.53799999999993	-2.9588286600721e-10\\
2.53799999999997	-2.95882866007129e-10\\
2.538	-2.9588286600705e-10\\
2.53999999999997	-2.91054137892289e-10\\
2.54	-2.91054137892221e-10\\
2.54199999999998	-2.86332797406583e-10\\
2.54399999999995	-2.81717919128634e-10\\
2.54599999999997	-2.77208598524297e-10\\
2.546	-2.77208598524234e-10\\
2.54999999999995	-2.68503115503731e-10\\
2.55199999999997	-2.64305246778778e-10\\
2.552	-2.64305246778719e-10\\
2.55499999999997	-2.58199713432034e-10\\
2.555	-2.58199713431977e-10\\
2.55799999999998	-2.52321317798459e-10\\
2.55800000000001	-2.52321317798405e-10\\
2.55999999999997	-2.48507762119094e-10\\
2.56	-2.4850776211904e-10\\
2.56199999999997	-2.44754197708056e-10\\
2.56399999999994	-2.41059888854356e-10\\
2.56599999999997	-2.37424111459042e-10\\
2.566	-2.3742411145899e-10\\
2.56999999999994	-2.30325311892969e-10\\
2.56999999999997	-2.30325311892911e-10\\
2.57	-2.30325311892853e-10\\
2.57399999999994	-2.23452233184687e-10\\
2.57799999999988	-2.16799486488609e-10\\
2.57999999999997	-2.13554100930566e-10\\
2.58	-2.13554100930521e-10\\
2.58099999999997	-2.11951375096382e-10\\
2.581	-2.11951375096337e-10\\
2.58199999999998	-2.10361855643429e-10\\
2.58299999999997	-2.08785464666732e-10\\
2.58499999999995	-2.05671759807589e-10\\
2.58599999999997	-2.04134293351497e-10\\
2.586	-2.04134293351454e-10\\
2.58999999999995	-1.98111916818838e-10\\
2.59	-1.98111916818768e-10\\
2.59199999999997	-1.95176191448851e-10\\
2.592	-1.95176191448809e-10\\
2.59399999999997	-1.92292564224302e-10\\
2.59599999999995	-1.89463029977122e-10\\
2.59799999999997	-1.86687034107568e-10\\
2.598	-1.86687034107528e-10\\
2.59999999999997	-1.83964032513601e-10\\
2.6	-1.83964032513562e-10\\
2.60199999999997	-1.81293491479945e-10\\
2.60399999999995	-1.78674887570187e-10\\
2.60599999999997	-1.76107707527744e-10\\
2.606	-1.76107707527708e-10\\
2.60999999999995	-1.71125616332435e-10\\
2.61	-1.71125616332374e-10\\
2.61399999999994	-1.66343311677438e-10\\
2.61599999999997	-1.64025901522118e-10\\
2.616	-1.64025901522085e-10\\
2.61999999999994	-1.59490406701906e-10\\
2.61999999999997	-1.59490406701871e-10\\
2.62	-1.59490406701836e-10\\
2.62399999999995	-1.55053800476911e-10\\
2.62499999999997	-1.53959690958384e-10\\
2.625	-1.53959690958353e-10\\
2.62599999999997	-1.52871490592379e-10\\
2.626	-1.52871490592348e-10\\
2.62699999999999	-1.51789146057808e-10\\
2.62799999999998	-1.50712604318615e-10\\
2.62999999999995	-1.48576718504981e-10\\
2.63199999999997	-1.46463414509556e-10\\
2.632	-1.46463414509526e-10\\
2.63599999999995	-1.42302899468046e-10\\
2.63899999999997	-1.39238741128887e-10\\
2.639	-1.39238741128858e-10\\
2.63999999999997	-1.38227797134185e-10\\
2.64	-1.38227797134156e-10\\
2.64099999999999	-1.37221991426941e-10\\
2.64199999999998	-1.36221274721581e-10\\
2.64399999999995	-1.34234912422112e-10\\
2.64599999999997	-1.32268320901048e-10\\
2.646	-1.3226832090102e-10\\
2.64999999999995	-1.2839291216746e-10\\
2.65099999999997	-1.27435819024646e-10\\
2.651	-1.27435819024618e-10\\
2.65499999999995	-1.23656305061751e-10\\
2.6589999999999	-1.19954042790006e-10\\
2.65999999999997	-1.1904020632283e-10\\
2.66	-1.19040206322804e-10\\
2.66599999999997	-1.13652272640948e-10\\
2.666	-1.13652272640923e-10\\
2.66799999999997	-1.11891462909331e-10\\
2.668	-1.11891462909306e-10\\
2.66999999999996	-1.10147659215674e-10\\
2.67199999999993	-1.08420519760689e-10\\
2.67199999999996	-1.08420519760658e-10\\
2.672	-1.08420519760627e-10\\
2.67599999999993	-1.0501488266705e-10\\
2.67799999999997	-1.03335717511515e-10\\
2.678	-1.03335717511491e-10\\
2.67999999999997	-1.01678081316763e-10\\
2.68	-1.0167808131674e-10\\
2.68199999999998	-1.00047849066362e-10\\
2.68399999999995	-9.8444701228471e-11\\
2.68599999999997	-9.68683235798675e-11\\
2.686	-9.68683235798453e-11\\
2.68999999999995	-9.37946481422319e-11\\
2.69399999999989	-9.08244128333673e-11\\
2.69499999999997	-9.00977418540617e-11\\
2.695	-9.00977418540411e-11\\
2.69699999999997	-8.86632149799054e-11\\
2.697	-8.86632149798852e-11\\
2.69899999999996	-8.72535411158407e-11\\
2.69999999999997	-8.65579375214488e-11\\
2.7	-8.65579375214291e-11\\
2.70199999999997	-8.51850265828549e-11\\
2.70399999999993	-8.38362868959798e-11\\
2.70599999999997	-8.25114541025741e-11\\
2.706	-8.25114541025554e-11\\
2.70899999999997	-8.05684635779014e-11\\
2.709	-8.05684635778833e-11\\
2.71199999999996	-7.86711452059434e-11\\
2.71299999999997	-7.80472259386162e-11\\
2.713	-7.80472259385985e-11\\
2.71599999999997	-7.62006006868876e-11\\
2.71899999999993	-7.43910429939378e-11\\
2.71999999999997	-7.37959559313848e-11\\
2.72	-7.37959559313679e-11\\
2.72599999999993	-7.03084403914061e-11\\
2.726	-7.03084403913706e-11\\
2.72600000000002	-7.03084403913545e-11\\
2.72999999999997	-6.80604059761885e-11\\
2.73	-6.80604059761728e-11\\
2.73199999999999	-6.69588161304778e-11\\
2.73200000000002	-6.69588161304623e-11\\
2.73400000000002	-6.58723788514433e-11\\
2.73600000000001	-6.48013710442619e-11\\
2.73799999999999	-6.37455827872648e-11\\
2.73800000000002	-6.37455827872499e-11\\
2.73999999999997	-6.27048071432907e-11\\
2.74	-6.2704807143276e-11\\
2.74199999999995	-6.16788401176581e-11\\
2.7439999999999	-6.06674806168605e-11\\
2.74599999999997	-5.96705304104633e-11\\
2.746	-5.96705304104492e-11\\
2.7499999999999	-5.77190790487262e-11\\
2.7539999999998	-5.58229559938976e-11\\
2.75499999999997	-5.53573951187365e-11\\
2.755	-5.53573951187233e-11\\
2.75999999999997	-5.30792714771791e-11\\
2.76	-5.30792714771664e-11\\
2.76499999999998	-5.08819171637348e-11\\
2.76500000000001	-5.08819171637226e-11\\
2.76599999999997	-5.04519000145789e-11\\
2.766	-5.04519000145667e-11\\
2.76699999999999	-5.00249845805003e-11\\
2.76700000000002	-5.00249845804882e-11\\
2.76800000000001	-4.96012923640465e-11\\
2.76899999999999	-4.91809450266339e-11\\
2.77099999999997	-4.8350202763372e-11\\
2.77299999999999	-4.75325949619976e-11\\
2.77300000000002	-4.75325949619861e-11\\
2.77699999999997	-4.5936144275735e-11\\
2.77999999999997	-4.4772110750182e-11\\
2.78	-4.47721107501711e-11\\
2.78399999999995	-4.32634959983698e-11\\
2.784	-4.32634959983543e-11\\
2.786	-4.25274681208067e-11\\
2.78600000000003	-4.25274681207963e-11\\
2.78800000000004	-4.18034334315656e-11\\
2.79000000000004	-4.10912500170369e-11\\
2.792	-4.03907782865305e-11\\
2.79200000000003	-4.03907782865206e-11\\
2.79600000000004	-3.90275324263901e-11\\
2.79999999999997	-3.77157364528605e-11\\
2.8	-3.77157364528514e-11\\
2.80400000000001	-3.64543618506334e-11\\
2.80599999999997	-3.5842271983563e-11\\
2.806	-3.58422719835544e-11\\
2.81000000000001	-3.46546872377297e-11\\
2.81199999999997	-3.4078959587776e-11\\
2.812	-3.4078959587768e-11\\
2.81299999999997	-3.37955621886837e-11\\
2.813	-3.37955621886757e-11\\
2.81399999999998	-3.35151238432256e-11\\
2.81499999999997	-3.32376308096137e-11\\
2.81699999999995	-3.26914264327023e-11\\
2.81899999999997	-3.21568419997632e-11\\
2.819	-3.21568419997557e-11\\
2.81999999999997	-3.18935331083627e-11\\
2.82	-3.18935331083553e-11\\
2.82099999999999	-3.16324074272388e-11\\
2.82199999999998	-3.13734521609446e-11\\
2.82399999999995	-3.08620022231127e-11\\
2.82599999999997	-3.03590830486882e-11\\
2.826	-3.03590830486811e-11\\
2.82999999999995	-2.9378444349466e-11\\
2.83399999999991	-2.84307671919571e-11\\
2.83499999999997	-2.81989105078471e-11\\
2.835	-2.81989105078405e-11\\
2.83999999999997	-2.70694358863701e-11\\
2.84	-2.70694358863639e-11\\
2.84199999999997	-2.66313506422189e-11\\
2.842	-2.66313506422128e-11\\
2.84399999999996	-2.62009669414929e-11\\
2.84599999999992	-2.57782004266896e-11\\
2.846	-2.57782004266735e-11\\
2.84600000000003	-2.57782004266676e-11\\
2.847	-2.55696476610475e-11\\
2.84700000000003	-2.55696476610416e-11\\
2.84800000000002	-2.53629682339943e-11\\
2.84900000000001	-2.51581520179943e-11\\
2.85099999999998	-2.47540691658004e-11\\
2.853	-2.43573199026532e-11\\
2.85300000000003	-2.43573199026476e-11\\
2.85699999999998	-2.35876581370586e-11\\
2.85999999999997	-2.30319114378984e-11\\
2.86	-2.30319114378933e-11\\
2.86399999999995	-2.2319116014759e-11\\
2.86599999999997	-2.19746464756795e-11\\
2.866	-2.19746464756747e-11\\
2.86999999999995	-2.1309226144578e-11\\
2.87	-2.13092261445701e-11\\
2.87099999999997	-2.11477231109902e-11\\
2.871	-2.11477231109856e-11\\
2.87199999999998	-2.09881449276362e-11\\
2.87299999999997	-2.08304837750047e-11\\
2.87499999999995	-2.05208817537397e-11\\
2.87699999999997	-2.02188563639917e-11\\
2.87699999999999	-2.02188563639874e-11\\
2.87999999999997	-1.97759646024369e-11\\
2.88	-1.97759646024328e-11\\
2.88299999999998	-1.93419305511428e-11\\
2.88599999999996	-1.89165627939782e-11\\
2.886	-1.89165627939722e-11\\
2.888	-1.86377061019167e-11\\
2.88800000000003	-1.86377061019127e-11\\
2.89000000000003	-1.83625629885689e-11\\
2.89200000000003	-1.80910795236882e-11\\
2.89600000000002	-1.75588793996783e-11\\
2.89999999999997	-1.70406884520758e-11\\
2.9	-1.70406884520722e-11\\
2.90499999999997	-1.6412032248061e-11\\
2.905	-1.64120322480575e-11\\
2.90599999999997	-1.6288783420559e-11\\
2.90599999999999	-1.62887834205555e-11\\
2.90699999999998	-1.61663478786139e-11\\
2.90799999999997	-1.60447196226005e-11\\
2.90999999999995	-1.58038611683885e-11\\
2.91199999999997	-1.55661608487291e-11\\
2.91199999999999	-1.55661608487257e-11\\
2.91599999999995	-1.51022962082032e-11\\
2.91799999999997	-1.48766027766688e-11\\
2.91799999999999	-1.48766027766656e-11\\
2.91999999999997	-1.46550093514556e-11\\
2.92	-1.46550093514525e-11\\
2.92199999999998	-1.4437472499687e-11\\
2.92399999999996	-1.42239495832958e-11\\
2.926	-1.40143987509569e-11\\
2.92600000000003	-1.40143987509539e-11\\
2.92899999999997	-1.37074305249999e-11\\
2.92899999999999	-1.3707430524997e-11\\
2.93199999999993	-1.3409171877743e-11\\
2.93499999999987	-1.3119491272232e-11\\
2.93499999999997	-1.31194912722226e-11\\
2.935	-1.31194912722199e-11\\
2.93999999999997	-1.26499582301345e-11\\
2.94	-1.26499582301319e-11\\
2.94499999999998	-1.21924264059038e-11\\
2.94599999999997	-1.21023106090409e-11\\
2.946	-1.21023106090383e-11\\
2.95099999999998	-1.165844158684e-11\\
2.95199999999997	-1.15709793252031e-11\\
2.952	-1.15709793252006e-11\\
2.95699999999998	-1.11399900049709e-11\\
2.95799999999999	-1.10550269724475e-11\\
2.95800000000002	-1.1055026972445e-11\\
2.95999999999997	-1.08863023130504e-11\\
2.96	-1.0886302313048e-11\\
2.96199999999995	-1.07191520019713e-11\\
2.9639999999999	-1.05535432768704e-11\\
2.96599999999997	-1.03894436777879e-11\\
2.966	-1.03894436777856e-11\\
2.9699999999999	-1.00656434915668e-11\\
2.96999999999999	-1.00656434915595e-11\\
2.97000000000002	-1.00656434915573e-11\\
2.97399999999992	-9.74819873334464e-12\\
2.97499999999997	-9.66991085063486e-12\\
2.975	-9.66991085063264e-12\\
2.9789999999999	-9.36093745905984e-12\\
2.97999999999997	-9.2847197212227e-12\\
2.98	-9.28471972122053e-12\\
2.9839999999999	-8.98383916524345e-12\\
2.986	-8.8357415905065e-12\\
2.98600000000003	-8.83574159050441e-12\\
2.98699999999997	-8.76226572963944e-12\\
2.98699999999999	-8.76226572963736e-12\\
2.98799999999998	-8.68916701318394e-12\\
2.98899999999997	-8.6164418592264e-12\\
2.99099999999995	-8.4720980027406e-12\\
2.99299999999999	-8.32920586822312e-12\\
2.99300000000002	-8.3292058682211e-12\\
2.99699999999997	-8.04882457266744e-12\\
2.99999999999997	-7.84366547942586e-12\\
3	-7.84366547942394e-12\\
3.00399999999995	-7.57678383474065e-12\\
3.00599999999997	-7.44613997929631e-12\\
3.006	-7.44613997929447e-12\\
3.00999999999995	-7.19031816447464e-12\\
3.01	-7.19031816447159e-12\\
3.01399999999995	-6.94161765599174e-12\\
3.01599999999997	-6.81987674204591e-12\\
3.01599999999999	-6.8198767420442e-12\\
3.01999999999995	-6.58149428235554e-12\\
3.02	-6.58149428235225e-12\\
3.02399999999995	-6.34975577989442e-12\\
3.02599999999997	-6.23632103334458e-12\\
3.026	-6.23632103334298e-12\\
3.02799999999997	-6.12447953965855e-12\\
3.02799999999999	-6.12447953965698e-12\\
3.02999999999996	-6.01435762815523e-12\\
3.03199999999992	-5.90608196505215e-12\\
3.03599999999985	-5.69498485224775e-12\\
3.03999999999997	-5.49102268801804e-12\\
3.04	-5.49102268801662e-12\\
3.04499999999997	-5.24586040784268e-12\\
3.04499999999999	-5.24586040784131e-12\\
3.04599999999997	-5.19810915700904e-12\\
3.046	-5.19810915700769e-12\\
3.04699999999998	-5.15077946638651e-12\\
3.04799999999996	-5.1038690167082e-12\\
3.04999999999992	-5.01129666612474e-12\\
3.05199999999997	-4.9203739607151e-12\\
3.052	-4.92037396071382e-12\\
3.05599999999992	-4.74363105250425e-12\\
3.05799999999997	-4.65783233751927e-12\\
3.058	-4.65783233751806e-12\\
3.05999999999997	-4.57372624787792e-12\\
3.06	-4.57372624787674e-12\\
3.06199999999997	-4.49129629857033e-12\\
3.06399999999995	-4.41052633300613e-12\\
3.066	-4.33140051995588e-12\\
3.06600000000003	-4.33140051995477e-12\\
3.06999999999997	-4.17801963525039e-12\\
3.07399999999992	-4.03103338592601e-12\\
3.07399999999996	-4.03103338592465e-12\\
3.07399999999999	-4.0310333859233e-12\\
3.07999999999997	-3.82229320334724e-12\\
3.07999999999999	-3.82229320334628e-12\\
3.08599999999997	-3.62608712734713e-12\\
3.08599999999999	-3.62608712734623e-12\\
3.09199999999997	-3.44084159370827e-12\\
3.09199999999999	-3.44084159370742e-12\\
3.09799999999997	-3.26622978127088e-12\\
3.09999999999997	-3.21033511783159e-12\\
3.1	-3.21033511783081e-12\\
3.10299999999997	-3.12862011676307e-12\\
3.10299999999999	-3.1286201167623e-12\\
3.10599999999996	-3.0494255584865e-12\\
3.106	-3.04942555848534e-12\\
3.10600000000003	-3.0494255584846e-12\\
3.10899999999999	-2.9727165166237e-12\\
3.11199999999996	-2.89845916124697e-12\\
3.11199999999999	-2.89845916124607e-12\\
3.11200000000003	-2.89845916124518e-12\\
3.11499999999997	-2.82658241870818e-12\\
3.115	-2.82658241870751e-12\\
3.11799999999995	-2.75701626601513e-12\\
3.11999999999997	-2.71190729971945e-12\\
3.12	-2.71190729971882e-12\\
3.12299999999995	-2.64612444940945e-12\\
3.12599999999989	-2.58257260379649e-12\\
3.12599999999995	-2.58257260379534e-12\\
3.126	-2.58257260379417e-12\\
3.127	-2.5618795616458e-12\\
3.12700000000003	-2.56187956164522e-12\\
3.12800000000003	-2.5414302728821e-12\\
3.12900000000003	-2.52122373539655e-12\\
3.13100000000003	-2.48153496560698e-12\\
3.13199999999997	-2.46205078852823e-12\\
3.13199999999999	-2.46205078852768e-12\\
3.13599999999999	-2.38649332495359e-12\\
3.13799999999997	-2.35012922236835e-12\\
3.13799999999999	-2.35012922236784e-12\\
3.13999999999997	-2.314516599372e-12\\
3.14	-2.3145165993715e-12\\
3.14199999999998	-2.27946643400058e-12\\
3.14399999999996	-2.2449718562859e-12\\
3.14599999999997	-2.21102610515666e-12\\
3.146	-2.21102610515619e-12\\
3.14999999999996	-2.14475457513491e-12\\
3.15	-2.14475457513425e-12\\
3.15399999999996	-2.08059988363723e-12\\
3.15599999999997	-2.04930056964218e-12\\
3.156	-2.04930056964174e-12\\
3.15999999999996	-1.98822733021419e-12\\
3.16	-1.98822733021358e-12\\
3.16099999999997	-1.97327243637491e-12\\
3.16099999999999	-1.97327243637449e-12\\
3.16199999999996	-1.9584414341952e-12\\
3.16299999999992	-1.94373359693927e-12\\
3.16499999999985	-1.91468454044941e-12\\
3.16599999999997	-1.90034189779036e-12\\
3.166	-1.90034189778996e-12\\
3.16999999999986	-1.84416756786716e-12\\
3.17199999999997	-1.81678853368838e-12\\
3.172	-1.816788533688e-12\\
3.17599999999986	-1.76351374942234e-12\\
3.17999999999972	-1.71224460211712e-12\\
3.17999999999997	-1.71224460211394e-12\\
3.18	-1.71224460211358e-12\\
3.18499999999997	-1.65091750310505e-12\\
3.185	-1.65091750310471e-12\\
3.18599999999997	-1.63901393987906e-12\\
3.186	-1.63901393987873e-12\\
3.18699999999997	-1.62722962079605e-12\\
3.18799999999994	-1.6155639683963e-12\\
3.18999999999989	-1.59258638295502e-12\\
3.18999999999994	-1.59258638295441e-12\\
3.18999999999999	-1.5925863829538e-12\\
3.19399999999988	-1.54803044718028e-12\\
3.19599999999999	-1.52644336372526e-12\\
3.19600000000002	-1.52644336372496e-12\\
3.198	-1.50520358369057e-12\\
3.19800000000003	-1.50520358369027e-12\\
3.19999999999998	-1.48419932932466e-12\\
3.20000000000001	-1.48419932932436e-12\\
3.20199999999996	-1.463426483712e-12\\
3.20399999999991	-1.44288097530863e-12\\
3.20599999999998	-1.4225587771143e-12\\
3.20600000000001	-1.42255877711401e-12\\
3.20999999999991	-1.38256842155084e-12\\
3.21399999999982	-1.34342406251998e-12\\
3.21899999999997	-1.29563660113993e-12\\
3.21899999999999	-1.29563660113966e-12\\
3.21999999999997	-1.2862268055264e-12\\
3.22	-1.28622680552614e-12\\
3.22099999999998	-1.27686516080441e-12\\
3.22199999999996	-1.26755120814657e-12\\
3.22399999999992	-1.24906455579493e-12\\
3.22599999999997	-1.23076322501534e-12\\
3.226	-1.23076322501508e-12\\
3.22999999999992	-1.19470221537947e-12\\
3.23099999999999	-1.18579722704082e-12\\
3.23100000000002	-1.18579722704057e-12\\
3.23499999999994	-1.15063342051144e-12\\
3.23699999999999	-1.13332340948752e-12\\
3.23700000000002	-1.13332340948728e-12\\
3.23999999999997	-1.10768870340634e-12\\
3.24	-1.1076887034061e-12\\
3.24299999999996	-1.0824403335762e-12\\
3.24599999999991	-1.05756716505222e-12\\
3.24599999999997	-1.05756716505167e-12\\
3.246	-1.05756716505144e-12\\
3.24799999999997	-1.04118807411113e-12\\
3.24799999999999	-1.0411880741109e-12\\
3.24999999999996	-1.02496765040343e-12\\
3.25199999999992	-1.00890271459748e-12\\
3.25399999999997	-9.92990117903129e-13\\
3.25399999999999	-9.92990117902904e-13\\
3.25499999999998	-9.85096867423981e-13\\
3.255	-9.85096867423758e-13\\
3.25599999999998	-9.77254327797006e-13\\
3.25699999999997	-9.69462114729095e-13\\
3.25899999999993	-9.54027143437104e-13\\
3.25999999999997	-9.46383628888793e-13\\
3.26	-9.46383628888577e-13\\
3.26399999999993	-9.16294000041679e-13\\
3.26599999999997	-9.01534691024276e-13\\
3.266	-9.01534691024067e-13\\
3.267	-8.94225141162788e-13\\
3.26700000000003	-8.94225141162581e-13\\
3.26800000000003	-8.86961849420162e-13\\
3.26900000000003	-8.7974445988708e-13\\
3.27100000000003	-8.65445975066691e-13\\
3.27500000000003	-8.37384419757946e-13\\
3.27699999999997	-8.23615848907968e-13\\
3.27699999999999	-8.23615848907774e-13\\
3.27999999999997	-8.03283149260644e-13\\
3.28	-8.03283149260453e-13\\
3.28299999999998	-7.83326679624501e-13\\
3.28599999999996	-7.63737638578494e-13\\
3.286	-7.63737638578238e-13\\
3.28899999999999	-7.44507386972403e-13\\
3.28900000000002	-7.44507386972222e-13\\
3.29	-7.381801444034e-13\\
3.29000000000003	-7.38180144403221e-13\\
3.29100000000001	-7.31900674893263e-13\\
3.29199999999999	-7.25668670745078e-13\\
3.29399999999996	-7.13345839367081e-13\\
3.296	-7.01209234946281e-13\\
3.29600000000003	-7.0120923494611e-13\\
3.29999999999996	-6.77485227800774e-13\\
3.3	-6.77485227800517e-13\\
3.30000000000003	-6.77485227800351e-13\\
3.30399999999996	-6.54478048462734e-13\\
3.30599999999997	-6.43237610522232e-13\\
3.30599999999999	-6.43237610522073e-13\\
3.30999999999992	-6.2127202188811e-13\\
3.31199999999999	-6.1054256585329e-13\\
3.31200000000002	-6.10542565853138e-13\\
3.31599999999995	-5.89604706977556e-13\\
3.31999999999988	-5.69364557178409e-13\\
3.32	-5.6936455717778e-13\\
3.32000000000003	-5.69364557177638e-13\\
3.32499999999998	-5.45021398316581e-13\\
3.325	-5.45021398316446e-13\\
3.326	-5.40277969206347e-13\\
3.32600000000003	-5.40277969206212e-13\\
3.32700000000003	-5.35575727232871e-13\\
3.32800000000003	-5.30914441975173e-13\\
3.33000000000002	-5.21713829986766e-13\\
3.332	-5.1267432988564e-13\\
3.33200000000003	-5.12674329885512e-13\\
3.33499999999999	-4.99413297340533e-13\\
3.33500000000002	-4.99413297340409e-13\\
3.33799999999998	-4.86504939062543e-13\\
3.33999999999997	-4.78092479442034e-13\\
3.34	-4.78092479441915e-13\\
3.34299999999997	-4.65759336617874e-13\\
3.34599999999993	-4.53764026116269e-13\\
3.34599999999997	-4.53764026116122e-13\\
3.346	-4.53764026115977e-13\\
3.34699999999999	-4.49839744525567e-13\\
3.34700000000002	-4.49839744525456e-13\\
3.34800000000001	-4.45951611499564e-13\\
3.349	-4.42098829414964e-13\\
3.35099999999999	-4.34498564644221e-13\\
3.35499999999995	-4.19714054693102e-13\\
3.35999999999997	-4.01998615635722e-13\\
3.36	-4.01998615635623e-13\\
3.36399999999997	-3.88424880544722e-13\\
3.36399999999999	-3.88424880544627e-13\\
3.36599999999997	-3.81833866867395e-13\\
3.366	-3.81833866867303e-13\\
3.36799999999998	-3.75371691408737e-13\\
3.36999999999996	-3.69037087557253e-13\\
3.37199999999997	-3.62828813705956e-13\\
3.372	-3.62828813705869e-13\\
3.37599999999996	-3.5078400380992e-13\\
3.37799999999997	-3.44944504622079e-13\\
3.378	-3.44944504621997e-13\\
3.37999999999997	-3.3922540858859e-13\\
3.38	-3.3922540858851e-13\\
3.38199999999997	-3.33625594752062e-13\\
3.38399999999995	-3.28143965527206e-13\\
3.38599999999997	-3.22779446493302e-13\\
3.386	-3.22779446493227e-13\\
3.38999999999995	-3.123975559136e-13\\
3.39299999999997	-3.04910896893513e-13\\
3.39299999999999	-3.04910896893443e-13\\
3.39499999999998	-3.00060688498007e-13\\
3.395	-3.00060688497939e-13\\
3.39699999999999	-2.95322085888371e-13\\
3.39899999999997	-2.90694160280035e-13\\
3.399	-2.90694160279971e-13\\
3.39999999999997	-2.88417186972103e-13\\
3.4	-2.88417186972038e-13\\
3.40099999999998	-2.86159084097203e-13\\
3.40199999999996	-2.8391974100578e-13\\
3.40399999999991	-2.79496896173988e-13\\
3.40599999999997	-2.75147785581789e-13\\
3.406	-2.75147785581728e-13\\
3.40999999999991	-2.66667371658563e-13\\
3.41099999999997	-2.64592037398338e-13\\
3.411	-2.6459203739828e-13\\
3.41499999999991	-2.56466726368137e-13\\
3.41899999999982	-2.4861830992822e-13\\
3.41999999999997	-2.46698744754804e-13\\
3.42	-2.4669874475475e-13\\
3.42199999999997	-2.42909999816595e-13\\
3.42199999999999	-2.42909999816542e-13\\
3.42399999999996	-2.39187818031333e-13\\
3.42599999999992	-2.35531469780483e-13\\
3.426	-2.35531469780332e-13\\
3.42600000000003	-2.35531469780281e-13\\
3.42999999999996	-2.2841342002098e-13\\
3.43	-2.28413420020894e-13\\
3.432	-2.2495032334948e-13\\
3.43200000000003	-2.24950323349432e-13\\
3.43400000000003	-2.21551472542229e-13\\
3.43600000000003	-2.18217404333269e-13\\
3.438	-2.1494746523235e-13\\
3.43800000000003	-2.14947465232304e-13\\
3.43999999999997	-2.11741014323186e-13\\
3.44	-2.1174101432314e-13\\
3.44199999999995	-2.08597423133236e-13\\
3.44399999999989	-2.05516075506416e-13\\
3.44599999999997	-2.02496367486448e-13\\
3.446	-2.02496367486405e-13\\
3.44999999999989	-1.96639514754116e-13\\
3.45099999999996	-1.95212915986067e-13\\
3.45099999999999	-1.95212915986027e-13\\
3.45499999999988	-1.89654881384628e-13\\
3.45699999999996	-1.86963929681985e-13\\
3.45699999999999	-1.86963929681948e-13\\
3.45999999999997	-1.83008302930096e-13\\
3.46	-1.83008302930059e-13\\
3.46299999999998	-1.79125854378142e-13\\
3.46499999999998	-1.76577363051057e-13\\
3.465	-1.76577363051021e-13\\
3.46599999999997	-1.75314871802041e-13\\
3.466	-1.75314871802005e-13\\
3.46699999999997	-1.74060134219138e-13\\
3.46799999999994	-1.7281308881963e-13\\
3.46999999999988	-1.70341830522718e-13\\
3.47199999999997	-1.67900612534936e-13\\
3.472	-1.67900612534901e-13\\
3.47599999999988	-1.63106389343816e-13\\
3.47999999999976	-1.58426660327924e-13\\
3.47999999999996	-1.58426660327687e-13\\
3.47999999999999	-1.58426660327654e-13\\
3.48599999999996	-1.51613740938768e-13\\
3.48599999999999	-1.51613740938736e-13\\
3.49199999999996	-1.45038178492298e-13\\
3.49199999999999	-1.45038178492266e-13\\
3.49799999999996	-1.38742720757769e-13\\
3.5	-1.36716621523399e-13\\
3.50000000000003	-1.3671662152337e-13\\
3.506	-1.30849923658407e-13\\
3.50600000000003	-1.3084992365838e-13\\
3.50899999999999	-1.28033233235286e-13\\
3.50900000000002	-1.28033233235259e-13\\
3.51199999999998	-1.2529264752307e-13\\
3.51200000000003	-1.25292647523028e-13\\
3.51499999999999	-1.22626957869034e-13\\
3.51799999999995	-1.20034988660896e-13\\
3.51799999999999	-1.20034988660864e-13\\
3.51800000000003	-1.20034988660832e-13\\
3.51999999999997	-1.18342420479173e-13\\
3.52	-1.18342420479149e-13\\
3.52199999999995	-1.16671819119517e-13\\
3.52399999999989	-1.15022857130709e-13\\
3.52599999999997	-1.13395211309709e-13\\
3.526	-1.13395211309686e-13\\
3.52999999999989	-1.10202596197434e-13\\
3.53399999999978	-1.07091470607722e-13\\
3.53499999999998	-1.0632613470975e-13\\
3.535	-1.06326134709728e-13\\
3.53799999999997	-1.04059395217808e-13\\
3.53799999999999	-1.04059395217787e-13\\
3.53999999999997	-1.02572256107124e-13\\
3.54	-1.02572256107103e-13\\
3.54199999999998	-1.01103992807932e-13\\
3.54399999999996	-9.96543175357902e-14\\
3.54599999999997	-9.82229461487186e-14\\
3.546	-9.82229461486984e-14\\
3.54999999999996	-9.54139963537645e-14\\
3.54999999999999	-9.54139963537441e-14\\
3.553	-9.33479021665389e-14\\
3.55300000000003	-9.33479021665194e-14\\
3.55600000000004	-9.13094743862282e-14\\
3.55900000000005	-8.92978140327765e-14\\
3.55999999999997	-8.86330552246655e-14\\
3.56	-8.86330552246467e-14\\
3.56599999999999	-8.47030732310526e-14\\
3.56600000000002	-8.47030732310342e-14\\
3.56699999999996	-8.40575384876842e-14\\
3.56699999999999	-8.40575384876659e-14\\
3.56799999999996	-8.34146224620194e-14\\
3.56899999999992	-8.27742936484327e-14\\
3.56999999999998	-8.21365206705835e-14\\
3.57	-8.21365206705654e-14\\
3.57199999999993	-8.08685173421828e-14\\
3.57299999999996	-8.02382248590833e-14\\
3.57299999999999	-8.02382248590654e-14\\
3.57499999999992	-7.8987313043547e-14\\
3.57699999999985	-7.77507008602865e-14\\
3.57899999999996	-7.65281459284218e-14\\
3.57899999999999	-7.65281459284045e-14\\
3.57999999999997	-7.59220649429326e-14\\
3.58	-7.59220649429154e-14\\
3.58099999999998	-7.53194086245056e-14\\
3.58199999999997	-7.47201474423177e-14\\
3.58399999999993	-7.353169319586e-14\\
3.58599999999997	-7.23564692618849e-14\\
3.586	-7.23564692618683e-14\\
3.58999999999993	-7.00447934929168e-14\\
3.59399999999986	-6.77833076623847e-14\\
3.59599999999996	-6.66708303581155e-14\\
3.59599999999999	-6.66708303580998e-14\\
3.59999999999997	-6.44813167269379e-14\\
3.6	-6.44813167269225e-14\\
3.60399999999998	-6.23376310075543e-14\\
3.60499999999998	-6.18086724501077e-14\\
3.605	-6.18086724500927e-14\\
3.60599999999997	-6.12824470001096e-14\\
3.606	-6.12824470000947e-14\\
3.60699999999997	-6.07589288728904e-14\\
3.60799999999994	-6.02380924154869e-14\\
3.60799999999999	-6.02380924154601e-14\\
3.60999999999993	-5.92064880598895e-14\\
3.61199999999987	-5.81895572396438e-14\\
3.61399999999996	-5.71871006322851e-14\\
3.61399999999999	-5.7187100632271e-14\\
3.61799999999987	-5.52248269196181e-14\\
3.61999999999997	-5.42646252009165e-14\\
3.62	-5.4264625200903e-14\\
3.62399999999988	-5.23851509917096e-14\\
3.62499999999996	-5.19236748143419e-14\\
3.62499999999999	-5.19236748143288e-14\\
3.62599999999997	-5.14655101166031e-14\\
3.626	-5.14655101165901e-14\\
3.62699999999998	-5.10106344487166e-14\\
3.62799999999997	-5.05590255212702e-14\\
3.62999999999993	-4.96655195309347e-14\\
3.63199999999997	-4.87848170149286e-14\\
3.632	-4.87848170149162e-14\\
3.63599999999993	-4.70667888188776e-14\\
3.63999999999985	-4.5409248326602e-14\\
3.63999999999997	-4.54092483265523e-14\\
3.64	-4.54092483265408e-14\\
3.64599999999997	-4.30335218739657e-14\\
3.646	-4.30335218739548e-14\\
3.65199999999997	-4.07867881487847e-14\\
3.652	-4.07867881487744e-14\\
3.65399999999996	-4.0065850428228e-14\\
3.65399999999999	-4.00658504282179e-14\\
3.65599999999995	-3.93586616898057e-14\\
3.65799999999992	-3.86650833187905e-14\\
3.65999999999996	-3.79849793711063e-14\\
3.65999999999999	-3.79849793710968e-14\\
3.66399999999992	-3.66620174405385e-14\\
3.66599999999996	-3.60182385062685e-14\\
3.66599999999999	-3.60182385062594e-14\\
3.66999999999992	-3.47654537632079e-14\\
3.67399999999984	-3.35582184224623e-14\\
3.67499999999998	-3.32634149819908e-14\\
3.67500000000001	-3.32634149819824e-14\\
3.67999999999997	-3.18307095656636e-14\\
3.68	-3.18307095656557e-14\\
3.68299999999996	-3.1003632689695e-14\\
3.68299999999999	-3.10036326896873e-14\\
3.68599999999995	-3.02005175971104e-14\\
3.686	-3.02005175970975e-14\\
3.68699999999998	-2.99380758878437e-14\\
3.68700000000001	-2.99380758878363e-14\\
3.68799999999998	-2.96782442994591e-14\\
3.68899999999995	-2.94210101000822e-14\\
3.6909999999999	-2.89142835768752e-14\\
3.69299999999998	-2.84177969978292e-14\\
3.693	-2.84177969978222e-14\\
3.6969999999999	-2.745692688457e-14\\
3.69999999999997	-2.67647692759721e-14\\
3.7	-2.67647692759657e-14\\
3.7039999999999	-2.58793053955078e-14\\
3.70599999999997	-2.54524117325379e-14\\
3.706	-2.54524117325319e-14\\
3.7099999999999	-2.46298825833841e-14\\
3.70999999999998	-2.46298825833682e-14\\
3.71000000000001	-2.46298825833625e-14\\
3.71199999999996	-2.42340858782358e-14\\
3.71199999999999	-2.42340858782302e-14\\
3.71399999999995	-2.38484969779765e-14\\
3.71599999999991	-2.34730403056608e-14\\
3.71799999999996	-2.31076422703066e-14\\
3.71799999999999	-2.31076422703015e-14\\
3.71999999999997	-2.27501313329533e-14\\
3.72	-2.27501313329483e-14\\
3.72199999999998	-2.23983375004584e-14\\
3.72399999999997	-2.20521918198678e-14\\
3.72599999999997	-2.17116264452839e-14\\
3.726	-2.17116264452791e-14\\
3.728	-2.13765746250376e-14\\
3.72800000000003	-2.13765746250329e-14\\
3.73000000000004	-2.10469706881228e-14\\
3.73200000000004	-2.07227500308906e-14\\
3.73600000000004	-2.00902054050009e-14\\
3.74	-1.94784447931998e-14\\
3.74000000000003	-1.94784447931955e-14\\
3.74099999999996	-1.93286955817515e-14\\
3.74099999999999	-1.93286955817473e-14\\
3.74199999999996	-1.91802079796183e-14\\
3.74299999999992	-1.90329747091763e-14\\
3.74499999999985	-1.87422423666236e-14\\
3.745	-1.87422423666015e-14\\
3.74500000000003	-1.87422423665974e-14\\
3.746	-1.85987290484101e-14\\
3.74600000000003	-1.8598729048406e-14\\
3.747	-1.84564415694561e-14\\
3.74799999999997	-1.83153729574943e-14\\
3.74999999999991	-1.80368647443602e-14\\
3.752	-1.77631498202981e-14\\
3.75200000000003	-1.77631498202943e-14\\
3.75599999999991	-1.72307583624399e-14\\
3.758	-1.69721955143894e-14\\
3.75800000000003	-1.69721955143858e-14\\
3.75999999999997	-1.67186533502142e-14\\
3.76	-1.67186533502106e-14\\
3.76199999999995	-1.6470082174989e-14\\
3.76399999999989	-1.62264332677977e-14\\
3.76599999999997	-1.59876588724943e-14\\
3.766	-1.59876588724909e-14\\
3.76999999999989	-1.55245473625053e-14\\
3.76999999999996	-1.5524547362497e-14\\
3.76999999999999	-1.55245473624938e-14\\
3.77399999999988	-1.50803845416847e-14\\
3.77599999999999	-1.48652994897593e-14\\
3.77600000000002	-1.48652994897563e-14\\
3.77999999999991	-1.44446501590058e-14\\
3.77999999999996	-1.4444650159001e-14\\
3.78	-1.44446501589961e-14\\
3.78399999999989	-1.4033580523772e-14\\
3.78599999999997	-1.38315370867457e-14\\
3.786	-1.38315370867428e-14\\
3.78999999999989	-1.34342349608915e-14\\
3.79199999999997	-1.3238898399286e-14\\
3.792	-1.32388983992832e-14\\
3.79599999999989	-1.28546628388249e-14\\
3.79899999999996	-1.2571970969019e-14\\
3.79899999999999	-1.25719709690163e-14\\
3.79999999999997	-1.24787603422333e-14\\
3.8	-1.24787603422307e-14\\
3.80099999999999	-1.23860520808388e-14\\
3.80199999999997	-1.22938416420288e-14\\
3.80399999999993	-1.21108961830063e-14\\
3.80599999999997	-1.19298881071415e-14\\
3.806	-1.19298881071389e-14\\
3.80999999999993	-1.15735425656467e-14\\
3.81099999999996	-1.14856120598298e-14\\
3.81099999999999	-1.14856120598273e-14\\
3.81499999999992	-1.11386352763876e-14\\
3.81499999999998	-1.11386352763827e-14\\
3.81500000000001	-1.11386352763803e-14\\
3.81899999999993	-1.07991480734361e-14\\
3.81999999999997	-1.07154152011171e-14\\
3.82	-1.07154152011148e-14\\
3.82399999999993	-1.03849163329116e-14\\
3.826	-1.02222698653995e-14\\
3.82600000000003	-1.02222698653972e-14\\
3.82700000000001	-1.01415833708547e-14\\
3.82700000000003	-1.01415833708524e-14\\
3.82799999999998	-1.00613160907547e-14\\
3.82800000000001	-1.00613160907524e-14\\
3.82899999999997	-9.98146409190876e-15\\
3.82999999999993	-9.90202346155654e-15\\
3.83199999999986	-9.7443607558656e-15\\
3.83399999999998	-9.58829707116651e-15\\
3.83400000000001	-9.58829707116431e-15\\
3.83799999999985	-9.28185894385006e-15\\
3.84	-9.13167798247637e-15\\
3.84000000000003	-9.13167798247425e-15\\
3.84399999999988	-8.83724507924813e-15\\
3.846	-8.69293542737535e-15\\
3.84600000000003	-8.69293542737331e-15\\
3.84999999999988	-8.40998829224045e-15\\
3.84999999999998	-8.40998829223347e-15\\
3.85000000000001	-8.40998829223149e-15\\
3.85399999999985	-8.13441979263783e-15\\
3.85599999999998	-7.99933479104834e-15\\
3.85600000000001	-7.99933479104644e-15\\
3.85699999999996	-7.93245546115579e-15\\
3.85699999999999	-7.93245546115389e-15\\
3.85799999999995	-7.86601386859672e-15\\
3.85899999999992	-7.80000675765987e-15\\
3.85999999999997	-7.7344308939727e-15\\
3.86	-7.73443089397084e-15\\
3.86199999999993	-7.60456007644265e-15\\
3.86399999999985	-7.47637596090515e-15\\
3.86599999999997	-7.34985342276511e-15\\
3.866	-7.34985342276333e-15\\
3.86899999999996	-7.1631309138908e-15\\
3.86899999999999	-7.16313091388904e-15\\
3.87199999999995	-6.98036688197598e-15\\
3.87499999999991	-6.80183872246308e-15\\
3.87999999999997	-6.5134918256232e-15\\
3.88	-6.5134918256216e-15\\
3.88499999999998	-6.23634002943552e-15\\
3.88500000000001	-6.23634002943398e-15\\
3.88599999999999	-6.18222296551753e-15\\
3.88600000000002	-6.182222965516e-15\\
3.88700000000001	-6.1285374296198e-15\\
3.88799999999999	-6.07528079102458e-15\\
3.88999999999996	-5.97004378828145e-15\\
3.89199999999999	-5.8664913304252e-15\\
3.89200000000002	-5.86649133042374e-15\\
3.89599999999996	-5.66453058494059e-15\\
3.89799999999999	-5.56612555898365e-15\\
3.89800000000002	-5.56612555898226e-15\\
3.89999999999997	-5.46941160225526e-15\\
3.9	-5.4694116022539e-15\\
3.90199999999996	-5.37436975857132e-15\\
3.90399999999991	-5.28098139936188e-15\\
3.90599999999997	-5.18922822014204e-15\\
3.906	-5.18922822014074e-15\\
3.90999999999991	-5.01055578331232e-15\\
3.91399999999982	-4.83821235967253e-15\\
3.91499999999996	-4.7960994689954e-15\\
3.91499999999999	-4.79609946899421e-15\\
3.91999999999997	-4.59126986449174e-15\\
3.92	-4.59126986449061e-15\\
3.92499999999999	-4.39581603031912e-15\\
3.92599999999997	-4.35782908477845e-15\\
3.926	-4.35782908477737e-15\\
3.92699999999996	-4.32020569459746e-15\\
3.92699999999999	-4.32020569459639e-15\\
3.92799999999995	-4.2829351122064e-15\\
3.92899999999992	-4.24600660716146e-15\\
3.93099999999984	-4.17316860776341e-15\\
3.93299999999996	-4.1016774198662e-15\\
3.93299999999999	-4.1016774198652e-15\\
3.93699999999984	-3.96267969007723e-15\\
3.93999999999997	-3.86186444783726e-15\\
3.94	-3.86186444783632e-15\\
3.94399999999985	-3.73193699856929e-15\\
3.94399999999996	-3.73193699856567e-15\\
3.94399999999999	-3.73193699856476e-15\\
3.94599999999997	-3.66887007118261e-15\\
3.946	-3.66887007118172e-15\\
3.94799999999999	-3.60705111127102e-15\\
3.94999999999997	-3.54646800207454e-15\\
3.95199999999997	-3.48710886906835e-15\\
3.952	-3.48710886906751e-15\\
3.95499999999998	-3.40031014019776e-15\\
3.95500000000001	-3.40031014019695e-15\\
3.95799999999998	-3.31614175764132e-15\\
3.95999999999998	-3.26147242817937e-15\\
3.96	-3.2614724281786e-15\\
3.96299999999998	-3.18160603349842e-15\\
3.96599999999995	-3.10427353305979e-15\\
3.966	-3.10427353305853e-15\\
3.96600000000003	-3.10427353305781e-15\\
3.96700000000001	-3.07905319384385e-15\\
3.96700000000003	-3.07905319384313e-15\\
3.96800000000001	-3.05410936423452e-15\\
3.96899999999998	-3.0294408218713e-15\\
3.97099999999993	-2.98092477725691e-15\\
3.97299999999996	-2.93349555066585e-15\\
3.97299999999999	-2.93349555066519e-15\\
3.97699999999989	-2.84186057780222e-15\\
3.97899999999996	-2.79763687071404e-15\\
3.97899999999999	-2.79763687071342e-15\\
3.97999999999997	-2.7758783808727e-15\\
3.98	-2.77587838087208e-15\\
3.98099999999999	-2.75429903518935e-15\\
3.98199999999997	-2.73289777625201e-15\\
3.98399999999994	-2.69062533262455e-15\\
3.98599999999997	-2.6490527644262e-15\\
3.986	-2.64905276442562e-15\\
3.98999999999994	-2.56797479661868e-15\\
3.98999999999997	-2.56797479661797e-15\\
3.99000000000001	-2.56797479661726e-15\\
3.99399999999994	-2.48960030351875e-15\\
3.99599999999998	-2.45140757548199e-15\\
3.99600000000001	-2.45140757548145e-15\\
3.99999999999994	-2.3769737255901e-15\\
3.99999999999997	-2.37697372558951e-15\\
4	-2.37697372558892e-15\\
4.00199999999993	-2.34071801441014e-15\\
4.00199999999999	-2.34071801440912e-15\\
4.00399999999992	-2.30509359575872e-15\\
4.00599999999986	-2.27009348711197e-15\\
4.00599999999993	-2.27009348711071e-15\\
4.006	-2.27009348710943e-15\\
4.00999999999987	-2.20193888031619e-15\\
4.01199999999995	-2.16877102362353e-15\\
4.012	-2.16877102362259e-15\\
4.01599999999987	-2.10426758795762e-15\\
4.01999999999973	-2.04219583783107e-15\\
4.01999999999995	-2.04219583782784e-15\\
4.02	-2.04219583782697e-15\\
4.02500000000001	-1.96795174407805e-15\\
4.02500000000006	-1.96795174407723e-15\\
4.02599999999995	-1.95354167437605e-15\\
4.026	-1.95354167437523e-15\\
4.02699999999996	-1.93927619014224e-15\\
4.02799999999992	-1.92515459233546e-15\\
4.02999999999985	-1.8973402951726e-15\\
4.03099999999993	-1.88364623289588e-15\\
4.03099999999999	-1.8836462328951e-15\\
4.03499999999984	-1.8302749827595e-15\\
4.03699999999994	-1.80442322775104e-15\\
4.03699999999999	-1.80442322775031e-15\\
4.03799999999995	-1.79167436130337e-15\\
4.038	-1.79167436130265e-15\\
4.03899999999996	-1.77900370508835e-15\\
4.03999999999991	-1.76641063823019e-15\\
4.04	-1.76641063822905e-15\\
4.04199999999991	-1.74145480810164e-15\\
4.04399999999982	-1.71680197949668e-15\\
4.046	-1.69244732036146e-15\\
4.04600000000006	-1.69244732036077e-15\\
4.04999999999988	-1.64461347373902e-15\\
4.05399999999969	-1.59791576402969e-15\\
4.05999999999994	-1.5299196008647e-15\\
4.06	-1.52991960086407e-15\\
};
\end{axis}
\end{tikzpicture}%
}
      \caption{Evolution of the angular displacement of pendulum $P_2$.
        $C_2 = 6$ ms. \texttt{Blue}: RM scheduling, \texttt{Red}: EDF scheduling}
      \label{fig:02.6.6.2}
    \end{figure}
  \end{minipage}
  \hfill
  \begin{minipage}{0.45\linewidth}
    \begin{figure}[H]\centering
      \scalebox{0.7}{% This file was created by matlab2tikz.
%
%The latest updates can be retrieved from
%  http://www.mathworks.com/matlabcentral/fileexchange/22022-matlab2tikz-matlab2tikz
%where you can also make suggestions and rate matlab2tikz.
%
\definecolor{mycolor1}{rgb}{0.00000,0.44700,0.74100}%
%
\begin{tikzpicture}

\begin{axis}[%
width=4.133in,
height=3.26in,
at={(0.693in,0.44in)},
scale only axis,
xmin=0,
xmax=1.22,
xmajorgrids,
ymin=-0.0001,
ymax=0.0014,
ymajorgrids,
axis background/.style={fill=white}
]
\pgfplotsset{max space between ticks=50}
\addplot [color=mycolor1,solid,forget plot]
  table[row sep=crcr]{%
0	0\\
3.15544362088405e-30	0\\
0.000656101980281985	0\\
0.00393661188169191	0\\
0.00599999999999994	0\\
0.006	0\\
0.012	0\\
0.0120000000000001	0\\
0.018	0\\
0.0180000000000001	0\\
0.0199999999999998	0\\
0.02	0\\
0.026	0\\
0.0260000000000002	0\\
0.0289999999999998	0\\
0.029	0\\
0.0319999999999996	0\\
0.0349999999999991	0\\
0.035	0\\
0.0399999999999996	0\\
0.04	0\\
0.0449999999999996	0\\
0.0459999999999996	0\\
0.046	0\\
0.047	0\\
0.0470000000000004	0\\
0.0490000000000003	0\\
0.0510000000000002	0\\
0.055	0\\
0.0579999999999996	0\\
0.058	0\\
0.0599999999999996	0\\
0.06	0\\
0.0619999999999995	0\\
0.0639999999999991	0\\
0.0659999999999991	0\\
0.066	0\\
0.0699999999999991	0\\
0.07	0\\
0.0700000000000009	0\\
0.074	0\\
0.076	0\\
0.0760000000000009	0\\
0.08	0\\
0.0800000000000009	0\\
0.0839999999999999	0\\
0.086	0\\
0.0860000000000009	0\\
0.0869999999999991	0\\
0.087	0\\
0.0880000000000004	0\\
0.0890000000000009	0\\
0.0910000000000017	0\\
0.0929999999999991	0\\
0.093	0\\
0.0970000000000017	0\\
0.0999999999999991	0\\
0.1	0\\
0.104000000000002	0\\
0.104999999999999	0\\
0.105	0\\
0.105999999999999	0\\
0.106	0\\
0.106999999999999	0\\
0.107999999999998	0\\
0.109999999999997	0\\
0.111999999999999	0\\
0.112	0\\
0.115999999999997	0\\
0.115999999999998	0\\
0.116	0\\
0.119999999999997	0\\
0.119999999999998	0\\
0.12	0\\
0.123999999999997	0\\
0.125999999999999	0\\
0.126	0\\
0.127999999999998	0\\
0.128	0\\
0.129999999999998	0\\
0.131999999999996	0\\
0.135999999999993	0\\
0.139999999999998	0\\
0.14	0\\
0.144999999999998	0\\
0.145	0\\
0.145999999999998	0\\
0.146	0\\
0.146999999999999	0\\
0.147999999999998	0\\
0.149999999999997	0\\
0.151999999999998	0\\
0.152	0\\
0.155999999999997	0\\
0.157999999999998	0\\
0.158	0\\
0.16	0\\
0.160000000000002	0\\
0.162000000000002	0\\
0.164000000000002	0\\
0.166	0\\
0.166000000000002	0\\
0.170000000000002	0\\
0.174	0\\
0.174000000000001	0\\
0.175	0\\
0.175000000000002	0\\
0.176000000000001	0\\
0.177	0\\
0.178999999999998	0\\
0.179999999999998	0\\
0.18	0\\
0.183999999999997	0\\
0.186	0\\
0.186000000000002	0\\
0.189999999999998	0\\
0.192	0\\
0.192000000000002	0\\
0.195999999999998	0\\
0.199999999999995	0\\
0.199999999999997	0\\
0.2	0\\
0.202999999999998	0\\
0.203	0\\
0.205999999999998	0\\
0.206	0\\
0.208999999999998	0\\
0.209999999999998	0\\
0.21	0\\
0.211999999999998	0\\
0.212	0\\
0.213999999999998	0\\
0.215999999999997	0\\
0.217999999999998	0\\
0.218	0\\
0.219999999999998	0\\
0.22	0\\
0.221999999999998	0\\
0.223999999999996	0\\
0.225999999999998	0\\
0.226	0\\
0.229999999999996	0\\
0.231999999999998	0\\
0.232	0\\
0.235999999999996	0\\
0.237999999999998	0\\
0.238	0\\
0.239999999999998	0\\
0.24	0\\
0.241999999999998	0\\
0.243999999999996	0\\
0.245	0\\
0.245000000000002	0\\
0.245999999999998	0\\
0.246	0\\
0.246999999999999	0\\
0.247999999999998	0\\
0.249999999999997	0\\
0.252	0\\
0.252000000000003	0\\
0.256	0\\
0.259999999999997	0\\
0.26	0\\
0.260999999999996	0\\
0.261	0\\
0.261999999999998	0\\
0.262999999999996	0\\
0.264999999999993	0\\
0.265999999999997	0\\
0.266	0\\
0.269999999999993	0\\
0.271999999999997	0\\
0.272	0\\
0.275999999999993	0\\
0.279999999999986	0\\
0.279999999999993	0\\
0.28	0\\
0.285999999999996	0\\
0.286	0\\
0.289999999999996	0\\
0.29	0\\
0.291999999999996	0.000600119384363888\\
0.292	0.00118528968800085\\
0.293999999999996	0.000585170303221496\\
0.295999999999993	0.000571785771018154\\
0.297999999999996	1.44985369776585e-06\\
0.298	0.000561413017294635\\
0.299999999999996	4.91862074066679e-06\\
0.3	0.00055316893187423\\
0.301999999999996	0.00054617082136155\\
0.303999999999993	0.000540417314075559\\
0.305999999999996	2.21598527660391e-05\\
0.306	0.00103835812636294\\
0.309999999999993	0.0010107652789132\\
0.313999999999986	0.000297086430918924\\
0.314999999999997	6.73081596684263e-05\\
0.315	0.000983785576868294\\
0.318999999999997	9.48422571232371e-05\\
0.319	0.000317195316184394\\
0.319999999999996	0.000102447482192413\\
0.32	0.000322116013085044\\
0.320999999999998	0.000327336878366592\\
0.321999999999996	0.000544537173940443\\
0.323999999999993	0.000551207851012842\\
0.325999999999996	0.000154170361188841\\
0.326	0.000953669517710882\\
0.329999999999993	0.000387879309204097\\
0.331	0.000205297796156682\\
0.331000000000004	0.000584897921386017\\
0.333	0.000226868606702662\\
0.333000000000004	0.000597295806305806\\
0.335	0.000609072986973724\\
0.336999999999996	0.000793016046696333\\
0.339999999999996	0.000296933153778918\\
0.34	0.000965837100503001\\
0.343999999999993	0.000654448136371524\\
0.345999999999997	0.000350355557462868\\
0.346	0.000662849464216899\\
0.347999999999997	0.000366820292554691\\
0.348	0.000670642936081734\\
0.349999999999997	0.00038261914919617\\
0.35	0.000677830079926191\\
0.351999999999997	0.000684412304246278\\
0.353999999999993	0.00041223148376314\\
0.354	0.000963028692236903\\
0.357999999999993	0.000703862336607106\\
0.359999999999996	0.000450982326906752\\
0.36	0.000960358535959291\\
0.363999999999993	0.000717233066097629\\
0.365999999999996	0.000482394412378318\\
0.366	0.000951248293735418\\
0.369999999999993	0.000941589504487661\\
0.373999999999986	0.000827467758456853\\
0.376999999999997	0.000521097883159126\\
0.377	0.000821009623126234\\
0.379999999999997	0.000527436919781524\\
0.38	0.00081288018423087\\
0.382999999999996	0.000714295083524902\\
0.384999999999997	0.000534001291672408\\
0.385	0.000622782073814989\\
0.385999999999997	0.000534714685791662\\
0.386	0.000621915183329071\\
0.386999999999998	0.000620852843365331\\
0.387999999999996	0.000619595001846525\\
0.388999999999997	0.000535656624968864\\
0.389	0.000699901049685073\\
0.390999999999997	0.000695113802733363\\
0.392999999999993	0.000689850300221151\\
0.394999999999997	0.000533459285567282\\
0.395	0.000830288568753345\\
0.398999999999993	0.000600876759294507\\
0.399999999999997	0.000528136029226594\\
0.4	0.000802786379093192\\
0.403999999999993	0.000652337152178106\\
0.405999999999997	0.000517547796633416\\
0.406	0.000766025219728579\\
0.409999999999993	0.000625711049585941\\
0.411999999999997	0.000502359430446121\\
0.412	0.000728105136722029\\
0.415999999999993	0.000705182917266275\\
0.419999999999986	0.000477702132099661\\
0.419999999999996	0.000477702132099249\\
0.42	0.000779807334332317\\
0.426	0.000457435450910584\\
0.426000000000004	0.000735864307651945\\
0.432000000000004	0.000435608970324818\\
0.432000000000007	0.000566099883977509\\
0.434999999999997	0.000424098880910602\\
0.435	0.000548851700746396\\
0.43799999999999	0.000492192172418767\\
0.439999999999997	0.000404013680413526\\
0.44	0.00051932403496473\\
0.44299999999999	0.000501126285920806\\
0.445999999999979	0.000378393031942585\\
0.445999999999995	0.000378393031942145\\
0.446	0.000413728841231905\\
0.447	0.000373958983041552\\
0.447000000000004	0.000408807247439756\\
0.448000000000004	0.000403997621685272\\
0.449000000000004	0.000432980415134887\\
0.451000000000004	0.000487823819966594\\
0.454999999999997	0.000338711145569766\\
0.455	0.000464746133149328\\
0.459	0.000352029455561498\\
0.459999999999997	0.000317071326992169\\
0.46	0.000436301103932277\\
0.463999999999997	0.000299953825235215\\
0.464	0.000357561059313572\\
0.465999999999997	0.000291455236905893\\
0.466	0.000347739176787858\\
0.466999999999997	0.000370169184624825\\
0.467	0.000423839374519779\\
0.467999999999998	0.000391636400818746\\
0.468999999999997	0.00038617668555706\\
0.470999999999993	0.000323450099195836\\
0.472999999999997	0.000288249769413829\\
0.473	0.000314237529205747\\
0.476999999999993	0.000246250670559048\\
0.479999999999997	0.00023500362818491\\
0.48	0.000235003628184897\\
0.483999999999993	0.000220804875193979\\
0.485999999999997	0.000214044056787704\\
0.486	0.000214044056787693\\
0.489999999999993	0.000201192910203876\\
0.49	0.00020119291020377\\
0.490000000000004	0.000269081512277783\\
0.492999999999997	0.000192136291606537\\
0.493	0.000258350547103222\\
0.495999999999993	0.000248142960713525\\
0.498999999999986	0.000175501809870749\\
0.498999999999993	0.000196710401773757\\
0.499	0.000196710401773663\\
0.499999999999997	0.000194030736780378\\
0.5	0.000215067995312905\\
0.500999999999998	0.000233206946034385\\
0.501999999999997	0.000251134290813701\\
0.503999999999993	0.000204344570559911\\
0.505999999999993	0.000199182523637343\\
0.506	0.000199182523637324\\
0.507999999999997	0.000194152229291636\\
0.508000000000004	0.000234710847978705\\
0.51	0.000269489477989487\\
0.511999999999997	0.000303572414278648\\
0.51599999999999	0.000214750516767262\\
0.519999999999993	0.000166683838879781\\
0.52	0.000166683838879766\\
0.521999999999993	0.000162549627860692\\
0.522	0.000181905149075904\\
0.523999999999993	0.000139245915747946\\
0.524999999999993	0.000137348066523191\\
0.525	0.000137348066523178\\
0.526	0.000135480125144316\\
0.526000000000007	0.000154654138526843\\
0.527000000000007	0.000171815104577598\\
0.528000000000007	0.000188837417150849\\
0.530000000000007	0.00014725077292033\\
0.532	0.000180904222685593\\
0.532000000000007	0.000217786270396433\\
0.536000000000007	0.000136604835765271\\
0.538	0.000133246029059068\\
0.538000000000007	0.000133246029059057\\
0.539999999999993	0.00012999429497748\\
0.54	0.000165791868723951\\
0.541999999999986	0.000162299204631489\\
0.543999999999972	0.000123809516295305\\
0.546	0.00015565151539219\\
0.546000000000007	0.000172916700246198\\
0.549999999999979	9.81357827675795e-05\\
0.550999999999993	0.000147968343240746\\
0.551	0.000181630486500746\\
0.554999999999972	0.000108958298242082\\
0.556999999999993	0.000123239694221101\\
0.557	0.000123239694221092\\
0.559999999999993	0.000119778543678955\\
0.56	0.000168800707287078\\
0.562999999999993	0.00011646372788071\\
0.565999999999986	6.47307263117124e-05\\
0.565999999999993	0.000161419155649866\\
0.566	0.000161419155649857\\
0.571999999999986	5.96740819595936e-05\\
0.571999999999993	0.000154696472324899\\
0.572	0.000186029426203076\\
0.577999999999986	8.64297551008072e-05\\
0.579999999999993	0.00014673746586689\\
0.58	0.000146737465866885\\
0.585999999999986	4.97255993090959e-05\\
0.585999999999993	0.000141507987728295\\
0.586	0.000141507987728289\\
0.591999999999986	4.62368641605663e-05\\
0.591999999999993	9.13396154827669e-05\\
0.592	9.13396154827596e-05\\
0.594999999999993	8.8776372987824e-05\\
0.595	0.000117658700339835\\
0.597999999999993	7.20392402269062e-05\\
0.599999999999993	8.47464956918835e-05\\
0.6	0.000126355835927638\\
0.602999999999993	8.2472147518704e-05\\
0.605999999999986	5.32708513816477e-05\\
0.606	5.32708513815469e-05\\
0.606999999999993	5.27703878841416e-05\\
0.607	6.62372506216337e-05\\
0.607999999999999	5.2280296232272e-05\\
0.608999999999997	6.50700486115605e-05\\
0.609000000000004	9.13165699969434e-05\\
0.611	8.98075839762761e-05\\
0.612999999999997	6.28671197217968e-05\\
0.614999999999997	8.7177620669448e-05\\
0.615000000000004	9.97267760052831e-05\\
0.618999999999997	4.76580587574656e-05\\
0.619999999999993	8.43928588473307e-05\\
0.62	0.000108826665966935\\
0.623999999999993	5.80764470658607e-05\\
0.625999999999993	8.12568963305159e-05\\
0.626	8.12568963305138e-05\\
0.629999999999993	7.92883838195398e-05\\
0.63	0.000102605869402928\\
0.633999999999993	5.4307660253419e-05\\
0.635999999999993	5.36117010605866e-05\\
0.636	5.36117010605842e-05\\
0.637999999999993	5.2934680856281e-05\\
0.638	5.29346808562786e-05\\
0.639999999999993	5.22764669728397e-05\\
0.64	7.47834366356611e-05\\
0.641999999999993	7.39523967711751e-05\\
0.643999999999986	5.10159457492109e-05\\
0.645999999999993	7.23589361390693e-05\\
0.646	7.23589361391432e-05\\
0.649999999999986	2.74960502688235e-05\\
0.65	7.06875158026365e-05\\
0.650000000000007	0.000112862011140281\\
0.653999999999993	8.9638368435593e-05\\
0.657999999999979	4.67407351274324e-05\\
0.659999999999993	7.64132812676895e-05\\
0.66	7.6413281267686e-05\\
0.664999999999993	3.47904193723342e-05\\
0.665	3.47904193723327e-05\\
0.665999999999993	3.45665949144666e-05\\
0.666	3.45665949144651e-05\\
0.666999999999998	3.43447339636309e-05\\
0.667000000000006	4.41459590391333e-05\\
0.668000000000004	5.35596002926516e-05\\
0.669000000000002	6.28108369282031e-05\\
0.670999999999998	4.3056584332088e-05\\
0.673000000000005	6.11517615366102e-05\\
0.673000000000013	7.03758749713238e-05\\
0.677000000000005	3.21381758373064e-05\\
0.678	4.09960697718731e-05\\
0.678000000000007	4.09960697718712e-05\\
0.679999999999993	4.0412507026353e-05\\
0.68	5.80823340269113e-05\\
0.681999999999986	5.722924050397e-05\\
0.683999999999972	3.92663010953283e-05\\
0.686	5.55538091626247e-05\\
0.686000000000007	8.8458389041122e-05\\
0.689999999999979	6.99789477137173e-05\\
0.69399999999995	3.65176368296087e-05\\
0.695999999999993	5.15370682183206e-05\\
0.696	5.15370682183179e-05\\
0.699999999999993	4.99957993956741e-05\\
0.7	6.48025500462055e-05\\
0.703999999999993	3.39252192103313e-05\\
0.705999999999993	3.34243543194837e-05\\
0.706	3.34243543194821e-05\\
0.707999999999993	3.29150075595137e-05\\
0.708	4.69622627604822e-05\\
0.709999999999993	4.61825631563525e-05\\
0.711999999999986	3.18843584204996e-05\\
0.713999999999993	4.46470899802886e-05\\
0.714	5.76635020667413e-05\\
0.717999999999986	3.03773131618297e-05\\
0.719999999999993	4.24014496002091e-05\\
0.72	4.85672029318608e-05\\
0.723999999999986	2.28101315997385e-05\\
0.724999999999993	2.26314414512736e-05\\
0.725	2.26314414512724e-05\\
0.725999999999993	2.24536037056536e-05\\
0.726	2.84364707730695e-05\\
0.726999999999999	3.40615186151242e-05\\
0.727999999999997	3.95085930248093e-05\\
0.729999999999993	2.74937125221308e-05\\
0.731999999999993	3.25950169938531e-05\\
0.732	3.25950169938511e-05\\
0.734999999999993	3.17319041483612e-05\\
0.735	4.23425734292365e-05\\
0.737999999999993	2.56625735423158e-05\\
0.74	3.03227804366416e-05\\
0.740000000000007	4.5303802852458e-05\\
0.743	2.94943371873394e-05\\
0.745999999999993	1.90799308746803e-05\\
0.746000000000007	1.90799308746446e-05\\
0.746999999999993	1.89206306501733e-05\\
0.747	2.36922009184165e-05\\
0.747999999999999	2.8140810586325e-05\\
0.748999999999997	3.69266397941771e-05\\
0.750999999999993	2.73444382553938e-05\\
0.753999999999993	3.08448815440446e-05\\
0.754	3.92571760655941e-05\\
0.757999999999993	2.14026154597419e-05\\
0.759999999999993	2.90868070745727e-05\\
0.76	3.69907853058811e-05\\
0.763999999999993	2.03217189553957e-05\\
0.766	2.75655247457881e-05\\
0.766000000000007	2.75655247457862e-05\\
0.77	2.6577912992005e-05\\
0.770000000000007	3.37043066607518e-05\\
0.774	1.86354689856137e-05\\
0.776	2.51344240804489e-05\\
0.776000000000007	2.51344240804472e-05\\
0.779999999999993	2.09497743765711e-05\\
0.78	2.09497743765698e-05\\
0.782999999999993	2.03673044760569e-05\\
0.783	2.03673044760556e-05\\
0.785999999999993	1.97935451672229e-05\\
0.786000000000001	2.8776911349624e-05\\
0.788999999999994	1.92282434109264e-05\\
0.791999999999987	1.00052130093726e-05\\
0.792	2.70793927030181e-05\\
0.792000000000008	3.25187101821932e-05\\
0.797999999999994	1.49810644242147e-05\\
0.799999999999993	2.24493855983276e-05\\
0.8	2.2449385598326e-05\\
0.804999999999993	1.15494810493777e-05\\
0.805000000000001	1.1549481049377e-05\\
0.805999999999993	1.14485201302574e-05\\
0.806	1.39076876948393e-05\\
0.806999999999994	1.61765789919545e-05\\
0.807999999999987	1.83569758878225e-05\\
0.809999999999973	1.33915638802701e-05\\
0.811999999999993	1.76116887314539e-05\\
0.812	2.1968341524362e-05\\
0.815999999999973	1.26419418726787e-05\\
0.817999999999993	1.24225293201694e-05\\
0.818000000000001	1.24225293201687e-05\\
0.819999999999993	1.22300645553321e-05\\
0.82	1.63625973870302e-05\\
0.821999999999993	1.61076400788605e-05\\
0.823999999999986	1.18527207326406e-05\\
0.825999999999993	1.56075791043193e-05\\
0.826	2.32999235936251e-05\\
0.829999999999986	2.62018694569247e-05\\
0.833999999999972	1.82794057949283e-05\\
0.839999999999993	8.65907121860388e-06\\
0.84	8.65907121860339e-06\\
0.840999999999993	8.59013266268699e-06\\
0.841000000000001	1.035567386531e-05\\
0.841999999999994	1.20093550490219e-05\\
0.842999999999987	1.19126075076611e-05\\
0.844999999999973	8.31891386472223e-06\\
0.845999999999993	1.32919662534445e-05\\
0.846	1.65821450648236e-05\\
0.849999999999973	9.62802615614021e-06\\
0.851999999999993	1.26438654863666e-05\\
0.852	1.57597528759135e-05\\
0.855999999999973	9.15573808411911e-06\\
0.857999999999993	9.00245800438779e-06\\
0.858	9.00245800438722e-06\\
0.86	8.85162308878715e-06\\
0.860000000000007	1.17964586650791e-05\\
0.862000000000007	1.15921912640614e-05\\
0.864000000000007	8.55717097908256e-06\\
0.866	1.11924752841054e-05\\
0.866000000000007	1.11924752841048e-05\\
0.87	1.08042575356521e-05\\
0.870000000000007	1.21198785875505e-05\\
0.874	6.55941425529771e-06\\
0.874999999999994	6.50614135383183e-06\\
0.875000000000001	6.50614135383148e-06\\
0.876	6.4565216418113e-06\\
0.876000000000007	7.74217713721232e-06\\
0.877000000000007	8.96174893215822e-06\\
0.878000000000006	7.63705453291178e-06\\
0.879999999999998	7.53343738032382e-06\\
0.880000000000006	1.00326295793246e-05\\
0.882000000000005	9.90519084679068e-06\\
0.884000000000004	7.3306384857822e-06\\
0.886000000000005	7.2314169928882e-06\\
0.886000000000013	7.23141699288782e-06\\
0.888000000000007	7.13362144386136e-06\\
0.888000000000014	9.53447671722381e-06\\
0.890000000000009	1.1769221265191e-05\\
0.892000000000004	1.27865696659742e-05\\
0.895999999999993	7.91406723286362e-06\\
0.898999999999993	5.47347979030073e-06\\
0.899000000000001	5.47347979030049e-06\\
0.899999999999993	5.43415999724783e-06\\
0.9	6.57573868560946e-06\\
0.900999999999994	7.66248932558677e-06\\
0.901999999999987	8.73411826127427e-06\\
0.903999999999973	6.40034789323364e-06\\
0.905999999999993	8.52125643550171e-06\\
0.906	8.52125643550878e-06\\
0.909999999999973	3.95934872747129e-06\\
0.909999999999987	5.05541361731608e-06\\
0.910000000000001	5.0554136173083e-06\\
0.910999999999993	5.01860428266668e-06\\
0.911000000000001	6.10413493929725e-06\\
0.911999999999994	7.13675206535084e-06\\
0.912999999999987	8.15397534022731e-06\\
0.914999999999973	5.93695680860316e-06\\
0.916999999999993	6.90431770284717e-06\\
0.917000000000001	6.90431770284685e-06\\
0.919999999999993	6.76905521575168e-06\\
0.92	9.84316829033546e-06\\
0.922999999999993	6.63686813733206e-06\\
0.925999999999986	3.47578000666309e-06\\
0.925999999999993	5.50166420023457e-06\\
0.926	5.50166420023419e-06\\
0.927999999999993	5.42622900480768e-06\\
0.928000000000001	7.4157950087879e-06\\
0.929999999999994	7.32389105845312e-06\\
0.931999999999987	5.27865493107205e-06\\
0.933999999999994	7.12929134251959e-06\\
0.934000000000001	9.0318697116763e-06\\
0.937999999999987	5.04657189824455e-06\\
0.939999999999993	6.82589877125538e-06\\
0.940000000000001	7.7451012123486e-06\\
0.943999999999987	3.90752576499376e-06\\
0.944999999999994	3.8765626904167e-06\\
0.945000000000001	3.87656269041647e-06\\
0.945999999999993	3.84583164125193e-06\\
0.946000000000001	4.74811269631584e-06\\
0.946999999999994	5.60328214894675e-06\\
0.947999999999987	6.44237578627702e-06\\
0.949999999999973	4.60553469529822e-06\\
0.952	6.2592748803493e-06\\
0.952000000000008	7.1131503279529e-06\\
0.95599999999998	3.55095685366422e-06\\
0.956999999999994	5.20476939954e-06\\
0.957000000000001	5.20476939953973e-06\\
0.96	5.09109652598814e-06\\
0.960000000000008	7.5322362873391e-06\\
0.963000000000007	4.98004758635464e-06\\
0.966000000000007	4.87157361386953e-06\\
0.966000000000014	4.87157361386928e-06\\
0.969000000000007	4.75994956850339e-06\\
0.969000000000014	7.05759594115836e-06\\
0.972000000000007	8.36075881805611e-06\\
0.975	6.00237372369609e-06\\
0.979999999999994	6.46535919978871e-06\\
0.980000000000001	6.46535919979315e-06\\
0.985999999999987	6.16162364850854e-06\\
0.986000000000001	6.16162364850783e-06\\
0.991999999999987	5.86265825128707e-06\\
0.992000000000001	5.86265825128638e-06\\
0.997999999999987	3.11565877838547e-06\\
0.998000000000001	3.11565877838076e-06\\
0.999999999999993	3.06040035078195e-06\\
1	4.27192480297906e-06\\
1.00199999999999	4.20441052454242e-06\\
1.00399999999999	2.97558536457374e-06\\
1.00599999999999	4.09575415152424e-06\\
1.006	6.33504516668246e-06\\
1.00999999999999	4.56963550918311e-06\\
1.01399999999997	2.37062071990936e-06\\
1.01499999999999	4.50272423279377e-06\\
1.015	4.50272423279365e-06\\
1.01999999999999	4.49114914281846e-06\\
1.02	4.9861381926065e-06\\
1.02499999999999	2.60088478471589e-06\\
1.02599999999999	2.63451191933445e-06\\
1.026	2.63451191933494e-06\\
1.02699999999999	2.67032489128641e-06\\
1.027	3.15172304253371e-06\\
1.02799999999999	3.65638350670151e-06\\
1.02899999999999	4.14708789671912e-06\\
1.03099999999997	3.27485896409739e-06\\
1.03299999999999	4.22891424260321e-06\\
1.033	5.54364814805081e-06\\
1.03699999999997	3.86788848313754e-06\\
1.04	4.34090430975393e-06\\
1.04000000000001	4.34090430975411e-06\\
1.04399999999999	3.58065196154849e-06\\
1.044	3.58065196154875e-06\\
1.046	3.61499420265804e-06\\
1.04600000000001	4.40545337388556e-06\\
1.04800000000001	3.64597067392163e-06\\
1.05	3.67358744632052e-06\\
1.05000000000001	3.6735874463207e-06\\
1.05200000000001	3.6979296879099e-06\\
1.05200000000002	4.44142015736149e-06\\
1.05400000000002	4.44691568004237e-06\\
1.05600000000002	3.73554528800932e-06\\
1.05800000000001	3.74845028326362e-06\\
1.05800000000002	3.74845028326369e-06\\
1.05999999999999	3.75742370461388e-06\\
1.06	4.44063480399614e-06\\
1.06199999999996	4.43095066407326e-06\\
1.06399999999992	3.76358209343498e-06\\
1.06599999999999	4.40018633672305e-06\\
1.066	5.33245827996496e-06\\
1.06999999999992	4.05054891875116e-06\\
1.07299999999999	4.30971385886525e-06\\
1.073	4.88553397527385e-06\\
1.07699999999992	3.6750049872943e-06\\
1.07899999999999	3.36785837232559e-06\\
1.079	3.36785837232545e-06\\
1.07999999999999	3.35693779525193e-06\\
1.08	3.63546791652959e-06\\
1.08099999999999	3.8955892605815e-06\\
1.08199999999999	3.87907768267954e-06\\
1.08399999999997	3.30852649355735e-06\\
1.08499999999999	3.29523845417654e-06\\
1.085	3.29523845417636e-06\\
1.08599999999999	3.28147506164693e-06\\
1.086	3.5461470999703e-06\\
1.08699999999999	3.78980167308644e-06\\
1.08799999999999	4.02629093151004e-06\\
1.08999999999997	3.47734833348161e-06\\
1.09199999999999	3.93616343303695e-06\\
1.092	4.90231202028369e-06\\
1.09599999999997	3.83893234054138e-06\\
1.09999999999995	2.80223861917393e-06\\
1.09999999999997	3.27256694955454e-06\\
1.1	3.2725669495508e-06\\
1.10199999999999	3.22592698628071e-06\\
1.102	3.67959816181977e-06\\
1.10399999999999	3.17737201056682e-06\\
1.10599999999997	2.68141570265513e-06\\
1.10599999999999	3.56426222718888e-06\\
1.106	3.99361061722425e-06\\
1.10999999999997	3.0201199051855e-06\\
1.11199999999999	3.37689433779e-06\\
1.112	4.17936162289995e-06\\
1.11599999999997	3.24596243244927e-06\\
1.11999999999994	2.34434360850756e-06\\
1.12	3.48935344191423e-06\\
1.12000000000001	3.48935344191372e-06\\
1.126	3.09663097943286e-06\\
1.12600000000001	3.09663097943238e-06\\
1.13099999999999	2.92483734517678e-06\\
1.131	3.09036218813349e-06\\
1.132	2.89050886164108e-06\\
1.13200000000001	3.05461040018605e-06\\
1.13300000000001	2.85657551038932e-06\\
1.13400000000001	2.82342460459224e-06\\
1.13600000000001	2.59841707072675e-06\\
1.138	2.21847327294532e-06\\
1.13800000000001	2.37951987328855e-06\\
1.13999999999999	2.16572714034984e-06\\
1.14	2.48283934185465e-06\\
1.14199999999997	2.42729265584546e-06\\
1.14399999999994	2.06357223554682e-06\\
1.14599999999999	2.31968990700629e-06\\
1.146	2.91963857804283e-06\\
1.14999999999994	2.36437978297388e-06\\
1.15399999999989	1.68038430381674e-06\\
1.15499999999999	2.23751716816709e-06\\
1.155	2.23751716816674e-06\\
1.15999999999999	2.11771700990596e-06\\
1.16	2.25594728661599e-06\\
1.16499999999999	1.46004612132885e-06\\
1.16599999999999	1.98307322690271e-06\\
1.166	2.11648262423775e-06\\
1.17099999999999	1.35240712979597e-06\\
1.17199999999999	1.33512582555619e-06\\
1.172	1.33512582555595e-06\\
1.173	1.31789732444925e-06\\
1.17300000000001	1.44806827640066e-06\\
1.17400000000001	1.55786695550822e-06\\
1.17500000000001	1.7891598196812e-06\\
1.17700000000001	1.50159950425408e-06\\
1.17999999999999	1.56775608694028e-06\\
1.18	1.80503170907196e-06\\
1.184	1.26170204346441e-06\\
1.18599999999999	1.34578224404373e-06\\
1.186	1.3457822440435e-06\\
1.18899999999999	1.07341771716696e-06\\
1.189	1.07341771716677e-06\\
1.18999999999999	1.0600480887367e-06\\
1.19	1.17187030338067e-06\\
1.19099999999999	1.26746625669687e-06\\
1.19199999999999	1.25249827296194e-06\\
1.19399999999997	1.0087688860045e-06\\
1.19499999999999	1.317441705159e-06\\
1.195	1.42329033834292e-06\\
1.19899999999997	9.5094573966694e-07\\
1.19999999999999	1.25273810052271e-06\\
1.2	1.45885278169392e-06\\
1.20399999999997	9.99330468372414e-07\\
1.20599999999999	8.75607892607477e-07\\
1.206	8.75607892607331e-07\\
1.20699999999999	8.65333715579316e-07\\
1.207	9.66693541651079e-07\\
1.20799999999999	1.05648387628e-06\\
1.20899999999999	1.34264579117771e-06\\
1.21099999999997	1.41569470142936e-06\\
1.21499999999995	9.82420852950862e-07\\
1.21799999999999	8.5658626166885e-07\\
1.218	8.56586261668716e-07\\
1.21999999999999	8.38146912857071e-07\\
1.22	1.02808270179712e-06\\
1.22199999999999	9.14547407344824e-07\\
1.22399999999997	7.0872959872139e-07\\
1.22499999999999	7.00554037762642e-07\\
1.225	7.00554037762527e-07\\
1.22599999999999	6.9249095852991e-07\\
1.226	7.8567629424254e-07\\
1.22699999999999	8.6976505477004e-07\\
1.22799999999999	7.69123800442178e-07\\
1.22999999999997	9.34378502374486e-07\\
1.23	1.11406596249079e-06\\
1.23399999999997	7.20917584539878e-07\\
1.236	8.81553405856868e-07\\
1.23600000000001	8.81553405856748e-07\\
1.23999999999999	8.48362659207558e-07\\
1.24	1.0191359489061e-06\\
1.24399999999997	6.47669919197916e-07\\
1.24599999999999	5.49768062839157e-07\\
1.246	5.49768062839067e-07\\
1.247	5.43531633226003e-07\\
1.24700000000001	6.27457725898314e-07\\
1.24800000000001	7.04012347849995e-07\\
1.24900000000001	7.79460175415013e-07\\
1.25100000000001	6.01746521234606e-07\\
1.253	7.49935314419906e-07\\
1.25300000000001	9.87110880758123e-07\\
1.25700000000001	6.42438794009305e-07\\
1.25999999999999	6.9911566233399e-07\\
1.26	8.50286618220747e-07\\
1.264	5.22738364621775e-07\\
1.266	6.58667714813849e-07\\
1.26600000000001	8.03706527633127e-07\\
1.27000000000001	4.9020826849501e-07\\
1.272	6.21016609114316e-07\\
1.27200000000001	6.2101660911423e-07\\
1.276	5.97436670415421e-07\\
1.27600000000001	5.9743667041534e-07\\
1.27999999999999	5.75051662741614e-07\\
1.28000000000001	7.06759398308088e-07\\
1.28399999999999	4.23934566685571e-07\\
1.28599999999999	4.1553984276643e-07\\
1.28600000000001	4.15539842766372e-07\\
1.288	4.07222662313658e-07\\
1.28800000000001	5.33055034414643e-07\\
1.29	5.83782785986951e-07\\
1.29199999999999	4.51978128567736e-07\\
1.29499999999999	4.97772854731978e-07\\
1.295	5.56321408313679e-07\\
1.29899999999998	3.06183276183224e-07\\
1.29999999999999	4.74245464896315e-07\\
1.3	5.3032286330385e-07\\
1.30399999999998	2.91078002718243e-07\\
1.30499999999999	2.88194060697486e-07\\
1.305	2.88194060697444e-07\\
1.30599999999999	2.85355487906797e-07\\
1.306	3.39983360248866e-07\\
1.30699999999999	3.9038717236148e-07\\
1.30700000000001	4.96322148677709e-07\\
1.308	4.39423817452062e-07\\
1.30899999999999	4.35291974025842e-07\\
1.31099999999998	3.2408476491587e-07\\
1.31300000000001	2.66669092698324e-07\\
1.31300000000002	3.17913188505074e-07\\
1.31699999999999	2.0706087686628e-07\\
1.31999999999999	2.01741546448564e-07\\
1.32	2.0174154644854e-07\\
1.32399999999997	1.95416444699934e-07\\
1.32599999999999	1.9258134653163e-07\\
1.326	1.92581346531612e-07\\
1.32999999999997	1.87563277065671e-07\\
1.33	1.87563277065116e-07\\
1.33399999999997	1.83411502368097e-07\\
1.334	1.83411502368016e-07\\
1.33799999999997	1.80122767235872e-07\\
1.34	1.78801207339214e-07\\
1.34000000000001	1.78801207339205e-07\\
1.34399999999999	1.76366951975376e-07\\
1.346	1.75144920789974e-07\\
1.34600000000001	1.75144920789966e-07\\
1.348	1.73919301458891e-07\\
1.34800000000001	1.73919301458882e-07\\
1.35	1.72689853757538e-07\\
1.35199999999999	1.71456336709461e-07\\
1.35599999999996	1.68976126634951e-07\\
1.35999999999999	1.6647672659804e-07\\
1.36	1.66476726598031e-07\\
1.36299999999999	1.64588404920615e-07\\
1.363	1.64588404920606e-07\\
1.36499999999999	1.63322503736827e-07\\
1.365	1.63322503736819e-07\\
1.36599999999999	1.6268735418066e-07\\
1.366	1.62687354180652e-07\\
1.36699999999999	1.62050697175813e-07\\
1.36799999999999	1.61412501525953e-07\\
1.36999999999997	1.60131369127385e-07\\
1.37199999999999	1.58843705878062e-07\\
1.372	1.58843705878053e-07\\
1.37599999999997	1.56205472365999e-07\\
1.378	1.54843809619773e-07\\
1.37800000000001	1.54843809619763e-07\\
1.37999999999999	1.53453428885796e-07\\
1.38	1.53453428885785e-07\\
1.38199999999997	1.52034057645829e-07\\
1.38399999999994	1.5058541769775e-07\\
1.38599999999999	1.49107225102525e-07\\
1.386	1.49107225102515e-07\\
1.38999999999994	1.46061017204215e-07\\
1.39199999999999	1.44492404832712e-07\\
1.392	1.44492404832701e-07\\
1.39599999999994	1.41262625923468e-07\\
1.39799999999999	1.39600826336685e-07\\
1.398	1.39600826336673e-07\\
1.39999999999999	1.37918105196819e-07\\
1.4	1.37918105196807e-07\\
1.40199999999999	1.36224916794439e-07\\
1.40399999999997	1.34520929258155e-07\\
1.40599999999999	1.32805808599838e-07\\
1.406	1.32805808599825e-07\\
1.40999999999997	1.29340820998017e-07\\
1.412	1.27590274903634e-07\\
1.41200000000001	1.27590274903622e-07\\
1.41599999999999	1.24051362501979e-07\\
1.41999999999996	1.20459706757878e-07\\
1.41999999999998	1.20459706757858e-07\\
1.42	1.20459706757838e-07\\
1.42099999999999	1.19553222600111e-07\\
1.421	1.19553222600098e-07\\
1.42199999999999	1.18643221785132e-07\\
1.42299999999999	1.17729659713186e-07\\
1.42499999999997	1.15891672562759e-07\\
1.42599999999999	1.149671574217e-07\\
1.426	1.14967157421687e-07\\
1.42999999999997	1.11231227234516e-07\\
1.43199999999999	1.09339898988864e-07\\
1.432	1.09339898988851e-07\\
1.43499999999999	1.06513878130162e-07\\
1.435	1.06513878130148e-07\\
1.43799999999998	1.03732889664946e-07\\
1.43999999999999	1.01903308890049e-07\\
1.44	1.01903308890036e-07\\
1.44299999999999	9.91946585065504e-08\\
1.44599999999997	9.65278126215436e-08\\
1.44599999999999	9.652781262153e-08\\
1.446	9.65278126215167e-08\\
1.44699999999999	9.56479500766553e-08\\
1.447	9.56479500766428e-08\\
1.44799999999999	9.47725585385214e-08\\
1.44899999999999	9.39015951091601e-08\\
1.44999999999999	9.30350171111369e-08\\
1.45	9.30350171111242e-08\\
1.45199999999999	9.1314847772118e-08\\
1.45399999999997	8.96117133627524e-08\\
1.45599999999999	8.79252800621028e-08\\
1.456	8.79252800620907e-08\\
1.45999999999997	8.46118730131451e-08\\
1.46	8.4611873013122e-08\\
1.46000000000001	8.46118730131106e-08\\
1.46399999999999	8.13827039403713e-08\\
1.466	7.9798914920563e-08\\
1.46600000000001	7.97989149205517e-08\\
1.46999999999999	7.66913757189442e-08\\
1.47	7.66913757189334e-08\\
1.47399999999997	7.36618644179432e-08\\
1.47599999999999	7.21756246118953e-08\\
1.476	7.2175624611885e-08\\
1.47899999999998	6.99810890904086e-08\\
1.479	6.99810890903984e-08\\
1.47999999999999	6.92587200962078e-08\\
1.48	6.92587200961974e-08\\
1.48099999999999	6.85408633456968e-08\\
1.48199999999999	6.78274836635585e-08\\
1.48399999999997	6.64140158990444e-08\\
1.48599999999999	6.50180397575091e-08\\
1.486	6.50180397574991e-08\\
1.48999999999997	6.22774712597098e-08\\
1.491	6.16028379484946e-08\\
1.49100000000001	6.16028379484848e-08\\
1.49499999999999	5.89595804166798e-08\\
1.49899999999996	5.64089125770964e-08\\
1.49999999999999	5.5785476819102e-08\\
1.5	5.57854768190931e-08\\
1.50499999999999	5.27521612448986e-08\\
1.505	5.27521612448903e-08\\
1.50599999999999	5.21620626890746e-08\\
1.506	5.21620626890661e-08\\
1.50699999999999	5.15774175561375e-08\\
1.50799999999999	5.09981971970835e-08\\
1.508	5.09981971970753e-08\\
1.50999999999999	4.98559175373639e-08\\
1.51199999999997	4.87349998194532e-08\\
1.51399999999999	4.76352243391307e-08\\
1.514	4.76352243391228e-08\\
1.51799999999997	4.55025644510815e-08\\
1.518	4.55025644510671e-08\\
1.51999999999999	4.4470342603858e-08\\
1.52	4.44703426038508e-08\\
1.52199999999999	4.34605882495281e-08\\
1.52399999999997	4.24731034722956e-08\\
1.52599999999999	4.15076947213094e-08\\
1.526	4.15076947213027e-08\\
1.52999999999997	3.96423526985031e-08\\
1.53399999999994	3.78630996567184e-08\\
1.537	3.65843140484055e-08\\
1.53700000000001	3.65843140483995e-08\\
1.53999999999999	3.53526034645446e-08\\
1.54	3.53526034645386e-08\\
1.54299999999997	3.41674247185145e-08\\
1.54599999999994	3.30282551082246e-08\\
1.54599999999997	3.30282551082142e-08\\
1.546	3.30282551082037e-08\\
1.549	3.19345922374791e-08\\
1.54900000000001	3.19345922374741e-08\\
1.55200000000001	3.08837470453128e-08\\
1.55500000000001	2.98730493419925e-08\\
1.55500000000003	2.98730493419877e-08\\
1.55999999999999	2.82765671069725e-08\\
1.56	2.8276567106968e-08\\
1.56499999999996	2.67884188397115e-08\\
1.56599999999998	2.65036265126799e-08\\
1.566	2.65036265126759e-08\\
1.57099999999996	2.5143083945028e-08\\
1.57199999999998	2.48835659110636e-08\\
1.572	2.488356591106e-08\\
1.57499999999999	2.41256373460087e-08\\
1.575	2.41256373460052e-08\\
1.57799999999999	2.33962264821385e-08\\
1.57999999999999	2.29256363710964e-08\\
1.58	2.2925636371093e-08\\
1.58299999999999	2.22430463051189e-08\\
1.58599999999997	2.15881436862503e-08\\
1.586	2.15881436862444e-08\\
1.58699999999999	2.13759453091681e-08\\
1.587	2.13759453091651e-08\\
1.58799999999999	2.11667807274022e-08\\
1.58899999999999	2.09606396909294e-08\\
1.59099999999997	2.05573879977048e-08\\
1.59499999999995	1.97867325789216e-08\\
1.59499999999997	1.97867325789165e-08\\
1.595	1.97867325789114e-08\\
1.59999999999999	1.88898583610645e-08\\
1.6	1.88898583610621e-08\\
1.60499999999999	1.80657906759169e-08\\
1.60599999999999	1.7909623522465e-08\\
1.606	1.79096235224628e-08\\
1.60699999999998	1.77563203129603e-08\\
1.607	1.77563203129581e-08\\
1.60799999999999	1.76054139023115e-08\\
1.60899999999999	1.74564372621878e-08\\
1.60999999999998	1.73093830926953e-08\\
1.61	1.73093830926934e-08\\
1.61199999999999	1.70210134367509e-08\\
1.61399999999997	1.6740248413099e-08\\
1.61599999999999	1.64670329907037e-08\\
1.616	1.64670329907018e-08\\
1.61999999999997	1.59430382155077e-08\\
1.61999999999999	1.59430382155058e-08\\
1.62	1.5943038215504e-08\\
1.62399999999997	1.54486182722135e-08\\
1.624	1.54486182722105e-08\\
1.62599999999999	1.52123768242093e-08\\
1.626	1.52123768242076e-08\\
1.62799999999999	1.49833855100979e-08\\
1.62999999999998	1.47615994466818e-08\\
1.63199999999999	1.45469751630301e-08\\
1.632	1.45469751630286e-08\\
1.63599999999998	1.41344026578991e-08\\
1.63999999999995	1.3740702111382e-08\\
1.63999999999998	1.37407021113797e-08\\
1.64	1.37407021113774e-08\\
1.645	1.3274646249047e-08\\
1.64500000000001	1.32746462490456e-08\\
1.64599999999999	1.3184864956114e-08\\
1.646	1.31848649561127e-08\\
1.64699999999999	1.30962165610125e-08\\
1.64799999999999	1.30086967197577e-08\\
1.64999999999997	1.28370255999144e-08\\
1.65199999999999	1.26698179483771e-08\\
1.652	1.2669817948376e-08\\
1.65299999999998	1.25878776476157e-08\\
1.653	1.25878776476146e-08\\
1.65399999999999	1.25070409920691e-08\\
1.65499999999997	1.24273040206481e-08\\
1.65699999999995	1.22711135553629e-08\\
1.65899999999998	1.21192756378194e-08\\
1.659	1.21192756378183e-08\\
1.65999999999999	1.20446913961176e-08\\
1.66	1.20446913961166e-08\\
1.66099999999999	1.19706078734447e-08\\
1.66199999999998	1.18970214396251e-08\\
1.66399999999995	1.17513254397235e-08\\
1.66599999999999	1.16075748394302e-08\\
1.666	1.16075748394291e-08\\
1.66999999999995	1.13257975116374e-08\\
1.67399999999991	1.10514685282874e-08\\
1.67999999999998	1.06534717481711e-08\\
1.68	1.06534717481702e-08\\
1.68199999999998	1.05243009110965e-08\\
1.682	1.05243009110956e-08\\
1.68399999999998	1.03968349749696e-08\\
1.68599999999997	1.02710489540839e-08\\
1.68599999999999	1.02710489540829e-08\\
1.686	1.0271048954082e-08\\
1.68999999999997	1.00244183646171e-08\\
1.69199999999999	9.90352545557014e-09\\
1.692	9.90352545556934e-09\\
1.69599999999997	9.66625397779807e-09\\
1.69799999999999	9.5497759181014e-09\\
1.698	9.54977591810066e-09\\
1.69999999999999	9.43470577767886e-09\\
1.7	9.43470577767802e-09\\
1.70199999999999	9.32102100248851e-09\\
1.70399999999998	9.20869930987457e-09\\
1.70599999999999	9.0977186843505e-09\\
1.706	9.09771868434971e-09\\
1.70999999999998	8.8796938833042e-09\\
1.71099999999998	8.82599217369678e-09\\
1.711	8.82599217369604e-09\\
1.71499999999997	8.61432435296978e-09\\
1.715	8.61432435296843e-09\\
1.71699999999998	8.51033760506874e-09\\
1.717	8.510337605068e-09\\
1.71899999999998	8.40727362786104e-09\\
1.71999999999999	8.35597582988597e-09\\
1.72	8.3559758298852e-09\\
1.72199999999999	8.25383604706719e-09\\
1.72399999999997	8.15228734433524e-09\\
1.72599999999999	8.05130981775153e-09\\
1.726	8.05130981775079e-09\\
1.72899999999998	7.9008712181874e-09\\
1.729	7.90087121818668e-09\\
1.73199999999998	7.75160691278751e-09\\
1.73499999999997	7.60345107354696e-09\\
1.73999999999998	7.35881067766093e-09\\
1.74	7.35881067766024e-09\\
1.74599999999997	7.0686454810842e-09\\
1.74599999999998	7.0686454810834e-09\\
1.746	7.06864548108264e-09\\
1.74999999999998	6.87702750160575e-09\\
1.75	6.87702750160506e-09\\
1.75199999999998	6.7817097165433e-09\\
1.752	6.78170971654264e-09\\
1.75399999999998	6.68715326537257e-09\\
1.75599999999997	6.59379840988265e-09\\
1.75799999999998	6.50162685216018e-09\\
1.758	6.50162685215954e-09\\
1.75999999999999	6.41062052634912e-09\\
1.76	6.41062052634846e-09\\
1.76199999999999	6.32076159497691e-09\\
1.76399999999998	6.23203244534518e-09\\
1.76599999999999	6.14441568619717e-09\\
1.766	6.14441568619653e-09\\
1.76899999999998	6.01503877229677e-09\\
1.769	6.01503877229613e-09\\
1.77199999999998	5.88806909051541e-09\\
1.77499999999996	5.76345064514181e-09\\
1.77499999999998	5.76345064514112e-09\\
1.775	5.76345064514044e-09\\
1.78	5.55979915009709e-09\\
1.78000000000002	5.55979915009656e-09\\
1.78499999999998	5.36021663386457e-09\\
1.785	5.36021663386398e-09\\
1.78600000000001	5.32076680644625e-09\\
1.78600000000003	5.32076680644567e-09\\
1.78700000000005	5.28146800967858e-09\\
1.78800000000006	5.24231831782953e-09\\
1.79000000000009	5.16445858272653e-09\\
1.79200000000003	5.0871723403328e-09\\
1.79200000000004	5.08717234033221e-09\\
1.79600000000011	4.9342598499403e-09\\
1.79799999999998	4.85860363051366e-09\\
1.798	4.85860363051316e-09\\
1.8	4.78346095528851e-09\\
1.80000000000002	4.78346095528795e-09\\
1.80200000000002	4.70881709611375e-09\\
1.80400000000002	4.63465742250534e-09\\
1.806	4.5609673988822e-09\\
1.80600000000002	4.56096739888165e-09\\
1.80999999999998	4.41493861712869e-09\\
1.81	4.41493861712818e-09\\
1.81399999999997	4.271308904334e-09\\
1.81799999999994	4.13065829502072e-09\\
1.82	4.06141560261676e-09\\
1.82000000000001	4.06141560261629e-09\\
1.826	3.85785552539493e-09\\
1.82600000000001	3.85785552539448e-09\\
1.827	3.82451920153263e-09\\
1.82700000000001	3.82451920153215e-09\\
1.828	3.79134716758625e-09\\
1.82899999999999	3.75833779798284e-09\\
1.83099999999996	3.69280058983186e-09\\
1.83200000000001	4.43435581446501e-09\\
1.83200000000003	4.43435581446451e-09\\
1.83599999999998	4.27795457880689e-09\\
1.83800000000001	3.44768757162602e-09\\
1.83800000000003	3.44768757162547e-09\\
1.83999999999999	3.37233708604618e-09\\
1.84	3.37233708604567e-09\\
1.84199999999996	3.3054480810642e-09\\
1.84399999999992	3.24700744615463e-09\\
1.84599999999999	3.19700372668849e-09\\
1.846	3.19700372668817e-09\\
1.84999999999992	3.12226948267445e-09\\
1.85399999999984	3.08118675345205e-09\\
1.85499999999998	3.07616976608807e-09\\
1.855	3.07616976608802e-09\\
1.85599999999998	3.07325361076975e-09\\
1.856	3.07325361076972e-09\\
1.85699999999999	3.07243814470836e-09\\
1.85799999999997	3.07372332775864e-09\\
1.85999999999995	3.08259599602651e-09\\
1.85999999999998	3.08259599602655e-09\\
1.86	3.08259599602671e-09\\
1.86399999999995	3.12555879003678e-09\\
1.86599999999999	3.15965733662841e-09\\
1.866	3.15965733662869e-09\\
1.86799999999998	3.20217567785027e-09\\
1.868	3.2021756778506e-09\\
1.86999999999998	3.24741953365884e-09\\
1.87199999999996	3.28969515824819e-09\\
1.87299999999998	3.3097224959613e-09\\
1.873	3.30972249596158e-09\\
1.87699999999996	3.3824511587049e-09\\
1.87999999999999	3.42927237416455e-09\\
1.88	3.42927237416475e-09\\
1.88399999999997	3.48143954533935e-09\\
1.88499999999998	3.49265337593957e-09\\
1.885	3.49265337593972e-09\\
1.886	3.50313717432437e-09\\
1.88600000000002	3.50313717432451e-09\\
1.88700000000002	3.5128914541978e-09\\
1.88800000000002	3.52191669353383e-09\\
1.88999999999998	3.53778178386749e-09\\
1.89	3.53778178387083e-09\\
1.892	2.55625306603359e-09\\
1.89200000000002	3.550735555568e-09\\
1.89400000000002	2.58443231745494e-09\\
1.89600000000003	2.60987574588461e-09\\
1.898	2.63189975784605e-09\\
1.89800000000002	3.5706832925561e-09\\
1.9	2.65050867137974e-09\\
1.90000000000002	3.57086924506189e-09\\
1.902	2.66570613382363e-09\\
1.90399999999998	2.67749512271217e-09\\
1.906	2.68587794875503e-09\\
1.90600000000002	3.55203773338813e-09\\
1.90999999999998	1.8439939390219e-09\\
1.91399999999995	1.87188336076143e-09\\
1.91399999999997	3.48162350726963e-09\\
1.914	3.48162350726985e-09\\
1.91999999999998	1.10781448251242e-09\\
1.92	3.39471290071557e-09\\
1.92499999999998	1.50559530660674e-09\\
1.925	3.30586251165558e-09\\
1.92599999999998	2.93713634134649e-09\\
1.926	3.28707865013733e-09\\
1.92699999999999	2.92134242730498e-09\\
1.92799999999997	2.90519055454099e-09\\
1.92999999999995	2.53180660432971e-09\\
1.93199999999998	2.50352677852472e-09\\
1.932	3.16720132479428e-09\\
1.93599999999995	1.79166766447373e-09\\
1.9399999999999	1.74992192632375e-09\\
1.93999999999999	2.98795140445742e-09\\
1.94	2.98795140447883e-09\\
1.94299999999998	2.01906287858496e-09\\
1.943	2.91491117721122e-09\\
1.94599999999998	1.97036411526761e-09\\
1.946	2.83864638760351e-09\\
1.94899999999998	1.91802421468047e-09\\
1.95199999999997	1.86202010762335e-09\\
1.95199999999998	2.67630714595295e-09\\
1.952	2.67630714595238e-09\\
1.95799999999997	9.57725742354243e-10\\
1.95999999999998	1.95888717480314e-09\\
1.96	2.45352160115854e-09\\
1.96599999999996	8.71301806680005e-10\\
1.96599999999998	2.2894223223048e-09\\
1.966	2.28942232230426e-09\\
1.97199999999996	8.06193880690866e-10\\
1.97199999999998	2.1275637085316e-09\\
1.972	2.12756370853113e-09\\
1.97799999999996	7.40709820175597e-10\\
1.97799999999998	1.96766019927004e-09\\
1.978	1.96766019926962e-09\\
1.98	1.52472039825853e-09\\
1.98000000000002	1.91509193563742e-09\\
1.98200000000002	1.47921232623549e-09\\
1.98400000000002	1.43448895574662e-09\\
1.986	1.39054152245433e-09\\
1.98600000000002	1.76256760520152e-09\\
1.99000000000002	9.38881812542442e-10\\
1.99400000000003	8.68004154682533e-10\\
1.995	1.37572817385067e-09\\
1.99500000000001	1.54788161454586e-09\\
1.99999999999999	5.95917907614214e-10\\
2	1.43564964041988e-09\\
2.00099999999997	1.24999285599701e-09\\
2.001	1.41378971364484e-09\\
2.00199999999999	1.22969017506025e-09\\
2.00299999999997	1.20957144058233e-09\\
2.00499999999995	1.01015404234292e-09\\
2.00599999999997	1.15030910439813e-09\\
2.006	1.30736350290709e-09\\
2.00999999999995	6.10566324783109e-10\\
2.01199999999997	8.86116361747657e-10\\
2.012	1.18583743297117e-09\\
2.01299999999998	1.0184089759184e-09\\
2.01300000000001	1.16628227173227e-09\\
2.014	1.00054830025605e-09\\
2.01499999999999	9.82974704160563e-10\\
2.01699999999996	8.04988415586914e-10\\
2.01999999999997	6.18851418911724e-10\\
2.02	1.03760992007366e-09\\
2.02399999999995	4.30989527182711e-10\\
2.02599999999997	6.76835364684596e-10\\
2.026	9.38599261700332e-10\\
2.02999999999995	3.70440454786515e-10\\
2.03	8.78283888916153e-10\\
2.03399999999995	3.3520514724547e-10\\
2.03599999999997	5.60198042758243e-10\\
2.036	7.96230194936109e-10\\
2.03999999999995	2.84491834855686e-10\\
2.04	7.46331200227175e-10\\
2.04399999999995	2.47665459712483e-10\\
2.04599999999997	4.54779442264438e-10\\
2.046	6.76909444460261e-10\\
2.04799999999997	4.35478554816775e-10\\
2.048	6.55197705089441e-10\\
2.04999999999996	4.16841815842688e-10\\
2.05199999999993	3.98865572183576e-10\\
2.05599999999987	1.52137466108714e-10\\
2.05899999999997	2.36677099761413e-10\\
2.059	5.48317426298905e-10\\
2.05999999999997	4.36835613923843e-10\\
2.06	5.39639475711934e-10\\
2.06099999999999	4.28858509335583e-10\\
2.06199999999998	4.21047114262284e-10\\
2.06399999999995	3.04690181246393e-10\\
2.06499999999997	3.98603405804538e-10\\
2.065	4.98808596238939e-10\\
2.06599999999997	3.91451106700853e-10\\
2.066	4.91151393943687e-10\\
2.06699999999999	3.84462685382334e-10\\
2.06799999999998	3.77637799232135e-10\\
2.06999999999995	2.6626251118599e-10\\
2.07099999999997	3.58141047994714e-10\\
2.071	4.55389612864311e-10\\
2.07499999999995	4.53829647415499e-11\\
2.07699999999997	2.28039370808483e-10\\
2.077	4.17060257125577e-10\\
2.07999999999997	1.19521629704463e-10\\
2.08	3.99436417470022e-10\\
2.08299999999998	1.07144762172449e-10\\
2.08599999999995	9.56557009026815e-11\\
2.086	3.67231549167689e-10\\
2.08799999999997	1.78563226113549e-10\\
2.088	3.57390411301696e-10\\
2.08999999999997	1.70903472468054e-10\\
2.09199999999993	1.63650787428817e-10\\
2.09399999999997	1.56803750153763e-10\\
2.094	3.30521445526298e-10\\
2.09799999999993	-2.68925135789834e-11\\
2.09999999999997	1.39445172975988e-10\\
2.1	3.06866837955493e-10\\
2.10399999999993	-3.60239850376648e-11\\
2.10599999999997	1.24531039270703e-10\\
2.106	2.85653502737106e-10\\
2.10999999999993	-4.33191324755194e-11\\
2.112	1.11736490282505e-10\\
2.11200000000003	2.6684401404659e-10\\
2.11599999999996	-4.87908264244638e-11\\
2.11699999999997	1.78064866752187e-10\\
2.117	2.52981644481953e-10\\
2.11999999999997	2.3486285949807e-11\\
2.12	2.45447259576964e-10\\
2.12299999999998	2.06432071151407e-11\\
2.12599999999995	1.82841497836072e-11\\
2.126	2.32124915494885e-10\\
2.12899999999997	1.64080728810616e-11\\
2.129	2.26331080926042e-10\\
2.13199999999996	1.53265882143227e-11\\
2.13499999999993	1.49037999019737e-11\\
2.13499999999997	2.15923583120889e-10\\
2.135	2.15923583120834e-10\\
2.13999999999997	-1.16817493369428e-10\\
2.14	2.08166042558922e-10\\
2.14499999999998	-1.11535003793261e-10\\
2.14599999999997	1.38820319861379e-10\\
2.146	1.99941224542196e-10\\
2.14699999999997	1.3803390230878e-10\\
2.14699999999999	4.36642208162129e-10\\
2.14799999999998	4.33577581481553e-10\\
2.14899999999997	3.72666836579282e-10\\
2.15099999999995	3.10196850265565e-10\\
2.15299999999999	1.91805432795358e-10\\
2.15300000000002	2.49194523838823e-10\\
2.15699999999997	7.54279040065533e-11\\
2.15999999999997	2.02127217386749e-11\\
2.16	1.85248210059699e-10\\
2.16399999999995	-3.11475947474944e-11\\
2.16599999999997	7.71887887870344e-11\\
2.166	1.8122873278018e-10\\
2.16999999999995	-2.33430292519774e-11\\
2.17	1.79366215299755e-10\\
2.17399999999995	-1.75246224451956e-11\\
2.17499999999997	1.30114840504338e-10\\
2.175	1.77954720946481e-10\\
2.17899999999995	-9.55135363068415e-12\\
2.17999999999997	1.31834032491018e-10\\
2.18	1.77560173597917e-10\\
2.18399999999995	-7.9085620096589e-13\\
2.18599999999997	9.14966943106411e-11\\
2.186	1.78428493905643e-10\\
2.187	1.35855141925297e-10\\
2.18700000000002	1.78715595035344e-10\\
2.18800000000002	1.36549718979004e-10\\
2.18800000000005	1.79015882193029e-10\\
2.18900000000004	1.37221356268893e-10\\
2.19000000000004	1.37876621211917e-10\\
2.19200000000003	9.78228550307603e-11\\
2.19600000000001	2.0916152255669e-11\\
2.2	2.74336497005078e-11\\
2.20000000000003	1.81505531285422e-10\\
2.20399999999997	3.36030962128947e-11\\
2.204	1.81880220466122e-10\\
2.20499999999997	1.45760850671465e-10\\
2.205	1.81938422888126e-10\\
2.20599999999999	1.46157983751193e-10\\
2.20600000000003	1.81982444107895e-10\\
2.20700000000002	1.46539182347118e-10\\
2.20800000000001	1.46904465171523e-10\\
2.20999999999998	1.12811765788905e-10\\
2.21200000000003	1.14121895748784e-10\\
2.21200000000006	1.81948838359108e-10\\
2.21600000000001	5.00026251004727e-11\\
2.21800000000003	1.17505435650526e-10\\
2.21800000000006	1.81307258875926e-10\\
2.21999999999997	1.18448995388166e-10\\
2.22	1.80936490480085e-10\\
2.22199999999992	1.19301744829325e-10\\
2.22399999999983	1.2006385103483e-10\\
2.226	1.20735463375732e-10\\
2.22600000000003	1.79352442083653e-10\\
2.22999999999986	6.44575959790366e-11\\
2.23299999999997	9.47953694754839e-11\\
2.233	1.76606821485741e-10\\
2.23699999999983	6.98137229665255e-11\\
2.23899999999997	1.22902568221008e-10\\
2.239	1.73479343818598e-10\\
2.24	1.48008044182198e-10\\
2.24000000000003	1.72892623393821e-10\\
2.24100000000003	1.47616171779957e-10\\
2.24200000000003	1.47211803534845e-10\\
2.24400000000004	1.22095618618072e-10\\
2.246	1.21602921590793e-10\\
2.24600000000003	1.69134853286832e-10\\
2.25000000000004	7.37162474265767e-11\\
2.252	1.19793894894102e-10\\
2.25200000000003	1.64965133913042e-10\\
2.25600000000004	7.39086356292839e-11\\
2.25999999999997	7.37147194653254e-11\\
2.26	1.58751876948973e-10\\
2.26199999999997	1.15667988186419e-10\\
2.262	1.570798438329e-10\\
2.26399999999996	1.14674771034072e-10\\
2.26599999999993	1.13625113342817e-10\\
2.26599999999997	1.53591053502968e-10\\
2.266	1.53591053502946e-10\\
2.26999999999994	7.21008393099914e-11\\
2.272	1.10135384256393e-10\\
2.27200000000003	1.47990857496655e-10\\
2.27499999999997	8.96127316039831e-11\\
2.275	1.45077118994787e-10\\
2.27799999999994	8.82788480993752e-11\\
2.27999999999997	1.05157941818139e-10\\
2.28	1.40208644203384e-10\\
2.28299999999994	8.59881396603423e-11\\
2.28599999999989	8.4572185257658e-11\\
2.28599999999997	1.3433891972313e-10\\
2.286	1.34338919724067e-10\\
2.28699999999997	1.17102812813495e-10\\
2.287	1.33357039346434e-10\\
2.28799999999999	1.16285786083176e-10\\
2.28899999999997	1.15466768104238e-10\\
2.29099999999995	9.80640394119587e-11\\
2.291	1.29417489012633e-10\\
2.29499999999995	6.46983214083101e-11\\
2.29699999999997	9.40502479466968e-11\\
2.297	1.23466759505334e-10\\
2.29999999999997	7.72524474653994e-11\\
2.3	1.20493871380441e-10\\
2.30299999999998	7.51116557586634e-11\\
2.30599999999995	7.29826853301118e-11\\
2.306	1.1464230481656e-10\\
2.30999999999997	5.64447757993999e-11\\
2.31	1.10806907157462e-10\\
2.31399999999997	5.39725459356819e-11\\
2.31599999999997	7.9100553770274e-11\\
2.316	1.05144762374571e-10\\
2.31999999999997	5.02719251491242e-11\\
2.32	1.01426317543332e-10\\
2.32399999999997	4.78071731699816e-11\\
2.32599999999997	7.13969085520715e-11\\
2.326	9.59258387020117e-11\\
2.32999999999997	4.4108706456588e-11\\
2.33199999999997	6.68314426914932e-11\\
2.332	9.05093718358711e-11\\
2.33599999999997	4.07203815587274e-11\\
2.33999999999994	3.89312914222915e-11\\
2.34	8.36156004173147e-11\\
2.34000000000003	8.36156004176861e-11\\
2.34499999999997	2.59079641931144e-11\\
2.345	7.95818785983689e-11\\
2.34600000000003	6.83507294237374e-11\\
2.34600000000006	7.87999447623384e-11\\
2.34700000000009	6.76709590675662e-11\\
2.34800000000012	6.69988231969443e-11\\
2.34899999999997	6.63342889012932e-11\\
2.349	7.65030709869724e-11\\
2.35100000000006	5.49515881761004e-11\\
2.35300000000011	5.3858688297585e-11\\
2.35499999999997	5.27936195652167e-11\\
2.355	7.21261662689226e-11\\
2.35800000000003	4.1521832024405e-11\\
2.35800000000006	7.00348155197292e-11\\
2.35999999999997	4.98966035959369e-11\\
2.36	6.86689578441161e-11\\
2.36199999999992	4.87378769430418e-11\\
2.36399999999983	4.75976316801226e-11\\
2.36599999999997	4.64756443023912e-11\\
2.366	6.47046635411113e-11\\
2.36999999999983	2.62304671103513e-11\\
2.37399999999966	2.44480834633947e-11\\
2.37799999999997	2.27205841629676e-11\\
2.378	5.73529955010539e-11\\
2.37999999999997	3.91149211726456e-11\\
2.38	5.61996776165716e-11\\
2.38199999999997	3.81313721202619e-11\\
2.38399999999995	3.71642252944666e-11\\
2.38599999999997	3.6213291375773e-11\\
2.386	5.28585028744035e-11\\
2.38999999999995	1.78540606688311e-11\\
2.39	5.07278620424687e-11\\
2.39399999999995	1.64151148100143e-11\\
2.396	3.17990911257491e-11\\
2.39600000000002	4.76751761679876e-11\\
2.39999999999997	1.44729280832623e-11\\
2.4	4.57342369888637e-11\\
2.40399999999995	1.32396620335009e-11\\
2.40599999999997	2.78769687225935e-11\\
2.406	4.29603005744128e-11\\
2.40699999999997	3.50283261615332e-11\\
2.407	4.25137234084186e-11\\
2.40799999999998	3.46232114473487e-11\\
2.40899999999997	3.42221497394716e-11\\
2.41099999999995	2.60566849772969e-11\\
2.41299999999997	2.53541278341574e-11\\
2.413	3.99267423438623e-11\\
2.41499999999997	2.46968885612534e-11\\
2.415	3.90990768342646e-11\\
2.41699999999997	2.40847743999202e-11\\
2.41899999999995	2.34866790380401e-11\\
2.41999999999997	3.01724448680187e-11\\
2.42	3.71037736322287e-11\\
2.42399999999995	8.33278993931196e-12\\
2.426	2.15020754210975e-11\\
2.42600000000003	3.48459226035904e-11\\
2.427	2.7880974664097e-11\\
2.42700000000003	3.44838317619777e-11\\
2.42800000000002	2.75685102350346e-11\\
2.429	2.72597211630656e-11\\
2.43099999999998	2.01861265460275e-11\\
2.43499999999993	6.48114601558361e-12\\
2.43599999999997	2.51998592780447e-11\\
2.436	3.1403589286928e-11\\
2.43999999999997	5.74581385819367e-12\\
2.44	3.01357520730889e-11\\
2.44399999999998	5.20450262139361e-12\\
2.446	1.67294291493615e-11\\
2.44600000000003	2.834774043308e-11\\
2.448	1.63227772621135e-11\\
2.44800000000002	2.77815980952572e-11\\
2.44999999999999	1.59438778616361e-11\\
2.45000000000002	2.72292982121075e-11\\
2.45199999999998	1.55917557230953e-11\\
2.45399999999995	1.52501858465977e-11\\
2.45600000000002	1.49191012752842e-11\\
2.45600000000005	2.56489776539228e-11\\
2.45999999999998	3.73918136724828e-12\\
2.46000000000001	2.46582202152394e-11\\
2.46399999999994	3.50508617720751e-12\\
2.46499999999997	1.85447710083605e-11\\
2.465	2.34890418877972e-11\\
2.46599999999998	1.83630289875823e-11\\
2.46600000000001	2.32643263118447e-11\\
2.46699999999999	1.81840607973999e-11\\
2.46799999999998	1.80078576593243e-11\\
2.46999999999996	1.28899028738161e-11\\
2.47199999999998	1.26405347013392e-11\\
2.47200000000001	2.19789515812212e-11\\
2.47599999999996	3.02015483201453e-12\\
2.47999999999991	2.95495400099363e-12\\
2.47999999999997	2.04127450054944e-11\\
2.48	2.0412745005643e-11\\
2.48499999999997	-1.35013483790494e-12\\
2.485	1.95100367502069e-11\\
2.48599999999997	1.52905810214131e-11\\
2.486	1.93363996075947e-11\\
2.48699999999999	1.51608298115959e-11\\
2.48799999999998	1.50331553767155e-11\\
2.48999999999995	1.08624203733245e-11\\
2.49199999999997	1.07033641404941e-11\\
2.492	1.83421959048218e-11\\
2.494	1.05517414724842e-11\\
2.49400000000002	1.8028741098849e-11\\
2.49600000000002	1.04075226501794e-11\\
2.49800000000001	1.02706793960387e-11\\
2.49999999999997	1.01411848886829e-11\\
2.5	1.71413557483456e-11\\
2.50399999999999	2.99990280820612e-12\\
2.50599999999997	9.71526960973398e-12\\
2.506	1.63187740268703e-11\\
2.50999999999999	2.96514997385158e-12\\
2.51399999999998	2.95660978644663e-12\\
2.51999999999997	-3.02483315976445e-12\\
2.52	1.45910776437576e-11\\
2.52299999999997	5.84237965505944e-12\\
2.523	1.42545747706112e-11\\
2.52599999999996	5.77629203379489e-12\\
2.526	1.39296319195705e-11\\
2.52899999999997	5.71802524223762e-12\\
2.53199999999993	5.6675537109675e-12\\
2.53199999999997	1.33138580831922e-11\\
2.532	1.33138580831878e-11\\
2.53799999999993	-1.82622264458787e-12\\
2.53799999999997	1.27383448091632e-11\\
2.538	1.27383448091612e-11\\
2.53999999999997	7.90961669447599e-12\\
2.54	1.25543154603731e-11\\
2.54199999999998	7.83297497480013e-12\\
2.54399999999995	7.75925128753278e-12\\
2.54599999999997	7.68843120211955e-12\\
2.546	1.20251476125755e-11\\
2.54999999999995	3.31966459213353e-12\\
2.55199999999997	7.4932566053799e-12\\
2.552	1.15296417721135e-11\\
2.55499999999997	5.4241084254852e-12\\
2.555	1.1294236639902e-11\\
2.55799999999998	5.41584100643892e-12\\
2.55800000000001	1.10669237382408e-11\\
2.55999999999997	7.25336805898215e-12\\
2.56	1.09189985953857e-11\\
2.56199999999997	7.16543418445308e-12\\
2.56399999999994	7.07862904255482e-12\\
2.56599999999997	6.99293561704702e-12\\
2.566	1.04862956750709e-11\\
2.56999999999994	3.38749610909913e-12\\
2.56999999999997	1.02067723328367e-11\\
2.57	1.02067723328344e-11\\
2.57399999999994	3.33369362472481e-12\\
2.57799999999988	3.28140201277297e-12\\
2.57999999999997	6.42282549406933e-12\\
2.58	9.53768410020591e-12\\
2.58099999999997	7.93495826611121e-12\\
2.581	9.47301984936868e-12\\
2.58199999999998	7.88350039650485e-12\\
2.58299999999997	7.83235987456153e-12\\
2.58499999999995	6.23116909672436e-12\\
2.58599999999997	7.6808175164179e-12\\
2.586	9.15556468150449e-12\\
2.58999999999995	3.13318814893002e-12\\
2.59	8.90845402937039e-12\\
2.59199999999997	5.97272865952008e-12\\
2.592	8.78711629437925e-12\\
2.59399999999997	5.90348906991115e-12\\
2.59599999999995	5.8378812156064e-12\\
2.59799999999997	5.77352251855725e-12\\
2.598	8.43340196207426e-12\\
2.59999999999997	5.71040036818377e-12\\
2.6	8.31904342564437e-12\\
2.60199999999997	5.64850239206415e-12\\
2.60399999999995	5.58781645449566e-12\\
2.60599999999997	5.52833066082819e-12\\
2.606	7.98629270397231e-12\\
2.60999999999995	3.00420150873451e-12\\
2.61	7.77284639002648e-12\\
2.61399999999994	2.99054173515013e-12\\
2.61599999999997	5.24850391425094e-12\\
2.616	7.46483975900814e-12\\
2.61999999999994	2.92934493886025e-12\\
2.61999999999997	7.26600313316395e-12\\
2.62	7.26600313316242e-12\\
2.62399999999995	2.82939690827023e-12\\
2.62499999999997	5.97642829877657e-12\\
2.625	7.02218023952727e-12\\
2.62599999999997	5.9339798735826e-12\\
2.626	6.97402140838022e-12\\
2.62699999999999	5.89167687386922e-12\\
2.62799999999998	5.8495172249175e-12\\
2.62999999999995	4.74240474281227e-12\\
2.63199999999997	4.67002250234058e-12\\
2.632	6.68911061901619e-12\\
2.63599999999995	2.52859557756468e-12\\
2.63899999999997	3.4386913392182e-12\\
2.639	6.36498163051715e-12\\
2.63999999999997	5.35403763587132e-12\\
2.64	6.31936169615797e-12\\
2.64099999999999	5.31355598896956e-12\\
2.64199999999998	5.27318936485507e-12\\
2.64399999999995	4.24225022578815e-12\\
2.64599999999997	4.17191128757381e-12\\
2.646	6.04901488444755e-12\\
2.64999999999995	2.17360615091363e-12\\
2.65099999999997	4.91474122976348e-12\\
2.651	5.82790060131067e-12\\
2.65499999999995	2.04838663846935e-12\\
2.6589999999999	1.95182081790962e-12\\
2.65999999999997	4.56961005639575e-12\\
2.66	5.44126867274962e-12\\
2.66599999999997	5.33349909176298e-14\\
2.666	5.19208562369205e-12\\
2.66799999999997	3.43127589212463e-12\\
2.668	5.11045546039628e-12\\
2.66999999999996	3.36665176678801e-12\\
2.67199999999993	3.30237448719655e-12\\
2.67199999999996	4.94924520379488e-12\\
2.672	6.58044725471063e-12\\
2.67599999999993	3.17481016115704e-12\\
2.67799999999997	4.7011052255566e-12\\
2.678	4.70110522556133e-12\\
2.67999999999997	3.0434690308621e-12\\
2.68	4.61555218135738e-12\\
2.68199999999998	2.98531993100158e-12\\
2.68399999999995	2.93423610810383e-12\\
2.68599999999997	2.89020754953542e-12\\
2.686	4.40314206101803e-12\\
2.68999999999995	1.32946662342601e-12\\
2.69399999999989	1.32803900926845e-12\\
2.69499999999997	3.49602751628345e-12\\
2.695	4.20837395442965e-12\\
2.69699999999997	2.77384708031414e-12\\
2.697	4.18520294272411e-12\\
2.69899999999996	2.77552907869948e-12\\
2.69999999999997	3.47372879638166e-12\\
2.7	4.16413375890019e-12\\
2.70199999999997	2.79122282034557e-12\\
2.70399999999993	2.8104688229236e-12\\
2.70599999999997	2.83674583120418e-12\\
2.706	4.17124456787906e-12\\
2.70899999999997	2.22825404324392e-12\\
2.709	4.19942769667544e-12\\
2.71199999999996	2.30210932475405e-12\\
2.71299999999997	3.6092405042223e-12\\
2.713	4.24322093554007e-12\\
2.71599999999997	2.39659568384716e-12\\
2.71899999999993	2.46030505393703e-12\\
2.71999999999997	3.69611380948406e-12\\
2.72	4.29713667730324e-12\\
2.72599999999993	8.09621137358423e-13\\
2.726	4.32041287420221e-12\\
2.72600000000002	4.32041287422157e-12\\
2.72999999999997	2.07237845907189e-12\\
2.73	4.32419161596779e-12\\
2.73199999999999	3.22260177028859e-12\\
2.73200000000002	4.32256117099448e-12\\
2.73400000000002	3.23612389199099e-12\\
2.73600000000001	3.24742837510267e-12\\
2.73799999999999	3.25587985722537e-12\\
2.73800000000002	4.30225564024555e-12\\
2.73999999999997	3.26147999630112e-12\\
2.74	4.29019691550679e-12\\
2.74199999999995	3.26422988990341e-12\\
2.7439999999999	3.26413007566456e-12\\
2.74599999999997	3.2611805339983e-12\\
2.746	4.23811700490953e-12\\
2.7499999999999	2.28666564320023e-12\\
2.7539999999998	2.29398556881745e-12\\
2.75499999999997	3.66586591643463e-12\\
2.755	4.11508339937122e-12\\
2.75999999999997	1.83695975783972e-12\\
2.76	4.02328837984448e-12\\
2.76499999999998	1.82593406642512e-12\\
2.76500000000001	3.91461110518074e-12\\
2.76599999999997	3.48459395604864e-12\\
2.766	3.89083842872341e-12\\
2.76699999999999	3.46392299466849e-12\\
2.76700000000002	3.86638403819275e-12\\
2.76800000000001	3.44269182176248e-12\\
2.76899999999999	3.42110316696987e-12\\
2.77099999999997	2.98549811135253e-12\\
2.77299999999999	2.9473775710457e-12\\
2.77300000000002	3.71260898543201e-12\\
2.77699999999997	2.11615829919174e-12\\
2.77999999999997	2.44041724618744e-12\\
2.78	3.52031050467102e-12\\
2.78399999999995	2.01169575287983e-12\\
2.784	3.40402412165699e-12\\
2.786	2.66799624412361e-12\\
2.78600000000003	3.34409848463185e-12\\
2.78800000000004	2.6200637954107e-12\\
2.79000000000004	2.57078555319269e-12\\
2.792	2.5201518587875e-12\\
2.79200000000003	3.15705012941769e-12\\
2.79600000000004	1.79380426929619e-12\\
2.79999999999997	1.71770984272194e-12\\
2.8	2.9035729631229e-12\\
2.80400000000001	1.64219836091311e-12\\
2.80599999999997	2.16706296518443e-12\\
2.806	2.71747950845501e-12\\
2.81000000000001	1.52989476263799e-12\\
2.81199999999997	2.01941825116261e-12\\
2.812	2.53446163969704e-12\\
2.81299999999997	2.25106424062e-12\\
2.813	2.50423493064544e-12\\
2.81399999999998	2.22379658520251e-12\\
2.81499999999997	2.19659015972288e-12\\
2.81699999999995	1.89780057337071e-12\\
2.81899999999997	1.84947862220177e-12\\
2.819	2.32439749637838e-12\\
2.81999999999997	2.0610886049921e-12\\
2.82	2.29475626754094e-12\\
2.82099999999999	2.03363058643109e-12\\
2.82199999999998	2.00640764168662e-12\\
2.82399999999995	1.72476603954414e-12\\
2.82599999999997	1.67573652781119e-12\\
2.826	2.12206416008245e-12\\
2.82999999999995	1.14142546087364e-12\\
2.83399999999991	1.06406593667347e-12\\
2.83499999999997	1.6733095981386e-12\\
2.835	1.87909726269974e-12\\
2.83999999999997	7.49622641235166e-13\\
2.84	1.75213953476127e-12\\
2.84199999999997	1.31405429062192e-12\\
2.842	1.70291404249902e-12\\
2.84399999999996	1.27253034178461e-12\\
2.84599999999992	1.23179752168837e-12\\
2.846	1.60708003985138e-12\\
2.84600000000003	1.60708003985972e-12\\
2.847	1.39852727424494e-12\\
2.84700000000003	2.1291316984734e-12\\
2.84800000000002	2.27791737854351e-12\\
2.84900000000001	2.07310116254856e-12\\
2.85099999999998	1.84429073945523e-12\\
2.853	1.447541476313e-12\\
2.85300000000003	1.62114324759693e-12\\
2.85699999999998	1.02342110216968e-12\\
2.85999999999997	8.06600809220328e-13\\
2.86	1.30273228705534e-12\\
2.86399999999995	5.8993686392571e-13\\
2.86599999999997	8.84432805315686e-13\\
2.866	1.19439544104985e-12\\
2.86999999999995	5.2897510995751e-13\\
2.87	1.13016395181142e-12\\
2.87099999999997	9.68660918238644e-13\\
2.871	1.1150977772954e-12\\
2.87199999999998	9.55519593950099e-13\\
2.87299999999997	9.42765623673094e-13\\
2.87499999999995	7.76548208671545e-13\\
2.87699999999997	7.56752332412467e-13\\
2.87699999999999	1.0329748888484e-12\\
2.87999999999997	5.90083127301743e-13\\
2.88	9.96122285260896e-13\\
2.88299999999998	5.62088233974795e-13\\
2.88599999999996	5.35245104063451e-13\\
2.886	9.26430956497442e-13\\
2.888	6.47574264447307e-13\\
2.88800000000003	9.04373062414411e-13\\
2.89000000000003	6.29229949074257e-13\\
2.89200000000003	6.11410676918619e-13\\
2.89600000000002	3.2978986059029e-13\\
2.89999999999997	3.03698735583314e-13\\
2.9	7.84040717037784e-13\\
2.90499999999997	1.55384513029956e-13\\
2.905	7.39850407822618e-13\\
2.90599999999997	6.16601580327425e-13\\
2.90599999999999	7.31424084333782e-13\\
2.90699999999998	6.08988542395352e-13\\
2.90799999999997	6.01505771885394e-13\\
2.90999999999995	4.74121378065516e-13\\
2.91199999999997	4.61377112235838e-13\\
2.91199999999999	6.83713766254381e-13\\
2.91599999999995	2.19849125735027e-13\\
2.91799999999997	4.28665095048698e-13\\
2.91799999999999	6.40303158548825e-13\\
2.91999999999997	4.18709733341798e-13\\
2.92	6.26658924825172e-13\\
2.92199999999998	4.09122073062609e-13\\
2.92399999999996	3.99900234716221e-13\\
2.926	3.91042410787261e-13\\
2.92600000000003	5.8816668616906e-13\\
2.92899999999997	2.81198460217799e-13\\
2.92899999999999	5.70277827467584e-13\\
2.93199999999993	2.72019180216281e-13\\
2.93499999999987	2.6360255203236e-13\\
2.93499999999997	5.37175181453158e-13\\
2.935	5.3717518145932e-13\\
2.93999999999997	6.76421393765084e-14\\
2.94	5.11652821536587e-13\\
2.94499999999998	5.41209973109715e-14\\
2.94599999999997	3.97172771612231e-13\\
2.946	4.82552039382904e-13\\
2.95099999999998	3.86830171870502e-14\\
2.95199999999997	3.72075974145687e-13\\
2.952	4.55067664670669e-13\\
2.95699999999998	2.40783444433196e-14\\
2.95799999999999	3.4840067569055e-13\\
2.95800000000002	4.29151207835425e-13\\
2.95999999999997	2.60426548443031e-13\\
2.96	4.20852834159081e-13\\
2.96199999999995	2.53702523084694e-13\\
2.9639999999999	2.47113219520926e-13\\
2.96599999999997	2.40657346438439e-13\\
2.966	3.96956275685332e-13\\
2.9699999999999	7.3156089468699e-14\\
2.96999999999999	3.8184183992984e-13\\
2.97000000000002	3.81841839934459e-13\\
2.97399999999992	6.4397081724011e-14\\
2.97499999999997	2.89018089181662e-13\\
2.975	3.63751559376709e-13\\
2.9789999999999	5.47781678060365e-14\\
2.97999999999997	2.73629164277371e-13\\
2.98	3.46448134326046e-13\\
2.9839999999999	4.5567578351027e-14\\
2.986	1.85060078163697e-13\\
2.98600000000003	3.26692891859915e-13\\
2.98699999999997	2.53217030996954e-13\\
2.98699999999999	3.23505073049339e-13\\
2.98799999999998	2.50406356597918e-13\\
2.98899999999997	2.47621599336227e-13\\
2.99099999999995	1.72873921342199e-13\\
2.99299999999999	1.6815511340308e-13\\
2.99300000000002	3.04992273879075e-13\\
2.99699999999997	2.46109783273457e-14\\
2.99999999999997	8.80608316006081e-14\\
3	2.84679629808229e-13\\
3.00399999999995	1.77979851267811e-14\\
3.00599999999997	1.43026842322392e-13\\
3.006	2.68325853150717e-13\\
3.00999999999995	1.25040383326576e-14\\
3.01	2.57950168512608e-13\\
3.01399999999995	9.24966003568556e-15\\
3.01599999999997	1.26245497591249e-13\\
3.01599999999999	2.4315655218777e-13\\
3.01999999999995	4.77409250075929e-15\\
3.02	2.33796138262854e-13\\
3.02399999999995	2.05763580818456e-15\\
3.02599999999997	1.11393967454313e-13\\
3.026	2.20490163945375e-13\\
3.02799999999997	1.08648670262484e-13\\
3.02799999999999	2.16247248476237e-13\\
3.02999999999996	1.06125336976008e-13\\
3.03199999999992	1.03820018603549e-13\\
3.03599999999985	-3.06591716196526e-15\\
3.03999999999997	-3.80186829205952e-15\\
3.04	1.92628378087543e-13\\
3.04499999999997	-5.25339020850249e-14\\
3.04499999999999	1.83681202588335e-13\\
3.04599999999997	1.35929951757376e-13\\
3.046	1.81953245970802e-13\\
3.04699999999998	1.34623555350929e-13\\
3.04799999999996	1.33335132604775e-13\\
3.04999999999992	8.59857772594205e-14\\
3.05199999999997	8.4320811543002e-14\\
3.052	1.72008763623427e-13\\
3.05599999999992	-4.73414458491391e-15\\
3.05799999999997	7.99591653427475e-14\\
3.058	1.62735918190498e-13\\
3.05999999999997	7.86298285515173e-14\\
3.06	1.59782148151465e-13\\
3.06199999999997	7.73521988462021e-14\\
3.06399999999995	7.61260256998329e-14\\
3.066	7.49510687776272e-14\\
3.06600000000003	1.51324267286565e-13\\
3.06999999999997	-2.05661741673345e-15\\
3.07399999999992	-9.69871576820546e-16\\
3.07399999999996	1.40968160983008e-13\\
3.07399999999999	1.40968160982962e-13\\
3.07999999999997	-6.77720215917842e-14\\
3.07999999999999	1.3387411891887e-13\\
3.08599999999997	-6.23319570793591e-14\\
3.08599999999999	1.27261648398531e-13\\
3.09199999999997	-5.79838852385536e-14\\
3.09199999999999	1.21037765433321e-13\\
3.09799999999997	-5.35740470023932e-14\\
3.09999999999997	5.92968260412101e-14\\
3.1	1.13324841926859e-13\\
3.10299999999997	3.16098408598792e-14\\
3.10299999999999	1.10600581740224e-13\\
3.10599999999996	3.14060234651533e-14\\
3.106	1.07966109941724e-13\\
3.10600000000003	1.07966109942101e-13\\
3.10899999999999	3.12570680819153e-14\\
3.11199999999996	3.11629093019462e-14\\
3.11199999999999	1.02961923188577e-13\\
3.11200000000003	1.02961923188557e-13\\
3.11499999999997	3.10851806524063e-14\\
3.115	1.00584321853867e-13\\
3.11799999999995	3.10181691621407e-14\\
3.11999999999997	5.31717663905051e-14\\
3.12	9.67856214200992e-14\\
3.12299999999995	3.1002771111347e-14\\
3.12599999999989	3.10512985513001e-14\\
3.12599999999995	9.24920412683595e-14\\
3.126	9.24920412683111e-14\\
3.127	7.17989991210681e-14\\
3.12700000000003	9.18040394526301e-14\\
3.12800000000003	7.13547506900746e-14\\
3.12900000000003	7.09172932232391e-14\\
3.13100000000003	5.07626119133455e-14\\
3.13199999999997	6.964545440603e-14\\
3.13199999999999	8.84802655053913e-14\\
3.13599999999999	1.29228019318362e-14\\
3.13799999999997	4.95946884489828e-14\\
3.13799999999999	8.47432163230149e-14\\
3.13999999999997	4.91305933276726e-14\\
3.14	8.35509753894499e-14\\
3.14199999999998	4.85008100190169e-14\\
3.14399999999996	4.78808019001834e-14\\
3.14599999999997	4.72704474477139e-14\\
3.146	8.00732050717318e-14\\
3.14999999999996	1.38016750509204e-14\\
3.15	7.78348876728588e-14\\
3.15399999999996	1.36801961764825e-14\\
3.15599999999997	4.4359358467474e-14\\
3.156	7.45933181035096e-14\\
3.15999999999996	1.35200786763838e-14\\
3.16	7.25070717777256e-14\\
3.16099999999997	5.75521779396387e-14\\
3.16099999999999	7.19946238506057e-14\\
3.16199999999996	5.71636216717271e-14\\
3.16299999999992	5.67779336431861e-14\\
3.16499999999985	4.19314314983648e-14\\
3.16599999999997	5.56378907683101e-14\\
3.166	6.94858127784549e-14\\
3.16999999999986	1.33114828560523e-14\\
3.17199999999997	4.01623765656773e-14\\
3.172	6.65895651799141e-14\\
3.17599999999986	1.3314780914633e-14\\
3.17999999999972	1.34613483607391e-14\\
3.17999999999997	6.29329409524434e-14\\
3.18	6.29329409551616e-14\\
3.18499999999997	1.60584194697351e-15\\
3.185	6.07703601532601e-14\\
3.18599999999997	4.88667969279453e-14\\
3.186	6.03488790391128e-14\\
3.18699999999997	4.85645599567635e-14\\
3.18799999999994	4.82653752661408e-14\\
3.18999999999989	3.65392001159605e-14\\
3.18999999999994	5.86990481180193e-14\\
3.18999999999999	5.86990481179973e-14\\
3.19399999999988	1.4143112345066e-14\\
3.19599999999999	3.55189195829539e-14\\
3.19600000000002	5.63303831531229e-14\\
3.198	3.50906031190286e-14\\
3.19800000000003	5.55653742753539e-14\\
3.19999999999998	3.45611199100306e-14\\
3.20000000000001	5.48077059687925e-14\\
3.20199999999996	3.40348603567108e-14\\
3.20399999999991	3.3511721324434e-14\\
3.20599999999998	3.29916002617275e-14\\
3.20600000000001	5.25772664381166e-14\\
3.20999999999991	1.25869108752205e-14\\
3.21399999999982	1.19799650106116e-14\\
3.21899999999997	1.90980197516071e-15\\
3.21899999999999	4.79481309451677e-14\\
3.21999999999997	3.85383353321692e-14\\
3.22	4.76027582812408e-14\\
3.22099999999998	3.82411135597715e-14\\
3.22199999999996	3.79448779139912e-14\\
3.22399999999992	2.84296786084164e-14\\
3.22599999999997	2.79342181895287e-14\\
3.226	4.55602788725218e-14\\
3.22999999999992	9.49926923716262e-15\\
3.23099999999999	3.53207576615828e-14\\
3.23100000000002	4.38953521815673e-14\\
3.23499999999994	8.7315456526822e-15\\
3.23699999999999	2.52791346851193e-14\\
3.23700000000002	4.19454745171419e-14\\
3.23999999999997	1.63107684364355e-14\\
3.24	4.09914025206536e-14\\
3.24299999999996	1.57430326909854e-14\\
3.24599999999991	1.51775101853337e-14\\
3.24599999999997	3.91228882074577e-14\\
3.246	3.91228882077514e-14\\
3.24799999999997	2.27437972676636e-14\\
3.24799999999999	3.85113440875993e-14\\
3.24999999999996	2.2290920380352e-14\\
3.25199999999992	2.18403104706039e-14\\
3.25399999999997	2.13918792798483e-14\\
3.25399999999999	3.67089154272257e-14\\
3.25499999999998	2.88156649485196e-14\\
3.255	3.64133681210355e-14\\
3.25599999999998	2.85708284944988e-14\\
3.25699999999997	2.83274981593916e-14\\
3.25899999999993	2.03929590640773e-14\\
3.25999999999997	2.76064250353738e-14\\
3.26	3.4963701360372e-14\\
3.26399999999993	4.87407251368298e-15\\
3.26599999999997	1.90774943077868e-14\\
3.266	3.32839942629281e-14\\
3.267	2.59744444018498e-14\\
3.26700000000003	3.30102020903522e-14\\
3.26800000000003	2.57469103481326e-14\\
3.26900000000003	2.5520744041298e-14\\
3.27100000000003	1.81692905625529e-14\\
3.27500000000003	3.87058175825533e-15\\
3.27699999999997	1.7112236122813e-14\\
3.27699999999999	3.03649091225135e-14\\
3.27999999999997	1.00322094755706e-14\\
3.28	2.96029865097407e-14\\
3.28299999999998	9.64651687397262e-15\\
3.28599999999996	9.26603363695764e-15\\
3.286	2.81208437892837e-14\\
3.28899999999999	8.89059218369435e-15\\
3.28900000000002	2.73999700192914e-14\\
3.29	2.10727274506433e-14\\
3.29000000000003	2.71627849896613e-14\\
3.29100000000001	2.08833154798753e-14\\
3.29199999999999	2.0695419127073e-14\\
3.29399999999996	1.43710419648931e-14\\
3.296	1.40955587162263e-14\\
3.29600000000003	2.57775638750716e-14\\
3.29999999999996	2.05355672989967e-15\\
3.3	2.48893431499316e-14\\
3.30000000000003	2.48893431500128e-14\\
3.30399999999996	1.88216381255582e-15\\
3.30599999999997	1.27880768163415e-14\\
3.30599999999999	2.36081609430264e-14\\
3.30999999999992	1.64257230921575e-15\\
3.31199999999999	1.20577060786118e-14\\
3.31200000000002	2.2386356178208e-14\\
3.31599999999995	1.44849730277102e-15\\
3.31999999999988	1.36453209467146e-15\\
3.32	2.08497114706355e-14\\
3.32000000000003	2.08497114711003e-14\\
3.32499999999998	-3.49344738982036e-15\\
3.325	1.99426509216585e-14\\
3.326	1.51992218116896e-14\\
3.32600000000003	1.9766036323495e-14\\
3.32700000000003	1.50637943502826e-14\\
3.32800000000003	1.49297152633029e-14\\
3.33000000000002	1.02169229483316e-14\\
3.332	1.0035780399305e-14\\
3.33200000000003	1.87392059313897e-14\\
3.33499999999999	5.47817338653257e-15\\
3.33500000000002	1.82465375637434e-14\\
3.33799999999998	5.33817928599728e-15\\
3.33999999999997	9.35494907275637e-15\\
3.34	1.74554067984483e-14\\
3.34299999999997	5.12226397452151e-15\\
3.34599999999993	5.00306659751457e-15\\
3.34599999999997	1.65543364374888e-14\\
3.346	1.65543364374851e-14\\
3.34699999999999	1.26300548472151e-14\\
3.34700000000002	1.64091753256046e-14\\
3.34800000000001	1.25210422998195e-14\\
3.349	1.24126154006376e-14\\
3.35099999999999	8.5227000622317e-15\\
3.35499999999995	1.05759729062042e-15\\
3.35999999999997	-2.41920745115814e-15\\
3.36	1.46430384698008e-14\\
3.36399999999997	1.06930337899359e-15\\
3.36399999999999	1.41433057543593e-14\\
3.36599999999997	7.55229207721867e-15\\
3.366	1.39009109640102e-14\\
3.36799999999998	7.43873550553405e-15\\
3.36999999999996	7.32882969426321e-15\\
3.37199999999997	7.22255310159383e-15\\
3.372	1.32030455936472e-14\\
3.37599999999996	1.15823569778247e-15\\
3.37799999999997	6.92200843289281e-15\\
3.378	1.25475961394702e-14\\
3.37999999999997	6.82850010614299e-15\\
3.38	1.23382001277386e-14\\
3.38199999999997	6.73838629136822e-15\\
3.38399999999995	6.65164931852709e-15\\
3.38599999999997	6.56827218691242e-15\\
3.386	1.17366988637454e-14\\
3.38999999999995	1.35480828419105e-15\\
3.39299999999997	3.87088714212859e-15\\
3.39299999999999	1.10844387534118e-14\\
3.39499999999998	6.23423035804251e-15\\
3.395	1.09076603744266e-14\\
3.39699999999999	6.16905776492411e-15\\
3.39899999999997	6.10715042407732e-15\\
3.399	1.05666518999658e-14\\
3.39999999999997	8.28967859216063e-15\\
3.4	1.04838244646535e-14\\
3.40099999999998	8.22572158988084e-15\\
3.40199999999996	8.16235146664639e-15\\
3.40399999999991	5.89741332383612e-15\\
3.40599999999997	5.81033934334927e-15\\
3.406	1.00013682853731e-14\\
3.40999999999991	1.52095436226739e-15\\
3.41099999999997	7.61792702576046e-15\\
3.411	9.61789086731213e-15\\
3.41499999999991	1.4925798372258e-15\\
3.41899999999982	1.47450727874512e-15\\
3.41999999999997	7.11864233187447e-15\\
3.42	8.96860382565001e-15\\
3.42199999999997	5.17985888754743e-15\\
3.42199999999999	8.83126293335534e-15\\
3.42399999999996	5.1090811481976e-15\\
3.42599999999992	5.0400400335836e-15\\
3.426	8.56395344062098e-15\\
3.42600000000003	8.56395344071529e-15\\
3.42999999999996	1.44590368146389e-15\\
3.43	8.30629981180811e-15\\
3.432	4.84320314047661e-15\\
3.43200000000003	8.18103052560699e-15\\
3.43400000000003	4.78217971845152e-15\\
3.43600000000003	4.72405010125745e-15\\
3.438	4.66761825055904e-15\\
3.43800000000003	7.81932401897576e-15\\
3.43999999999997	4.61287310990195e-15\\
3.44	7.70339513877405e-15\\
3.44199999999995	4.55980394891349e-15\\
3.44399999999989	4.50840036169445e-15\\
3.44599999999997	4.45865227285611e-15\\
3.446	7.36921021928026e-15\\
3.44999999999989	1.51235748703241e-15\\
3.45099999999996	5.73093903057153e-15\\
3.45099999999999	7.10598530451757e-15\\
3.45499999999988	1.54795070315823e-15\\
3.45699999999996	4.21421028472272e-15\\
3.45699999999999	6.80794797441694e-15\\
3.45999999999997	2.85232122260164e-15\\
3.46	6.66504051810276e-15\\
3.46299999999998	2.78259196622199e-15\\
3.46499999999998	3.97625828694618e-15\\
3.465	6.43264538465484e-15\\
3.46599999999997	5.17015413570892e-15\\
3.466	6.38701339148106e-15\\
3.46699999999997	5.13227580864879e-15\\
3.46799999999994	5.09461309398593e-15\\
3.46999999999988	3.8253201712995e-15\\
3.47199999999997	3.766016225721e-15\\
3.472	6.11896311553151e-15\\
3.47599999999988	1.32473992447978e-15\\
3.47999999999976	1.26584237571701e-15\\
3.47999999999996	5.77626734804578e-15\\
3.47999999999999	5.77626734824155e-15\\
3.48599999999996	-1.03665204061309e-15\\
3.48599999999999	5.52968615896526e-15\\
3.49199999999996	-1.04587628744265e-15\\
3.49199999999999	5.29156927361433e-15\\
3.49799999999996	-1.00388846085292e-15\\
3.5	3.0373556977727e-15\\
3.50000000000003	4.99000463845718e-15\\
3.506	-8.76693226478256e-16\\
3.50600000000003	4.77721564949123e-15\\
3.50899999999999	1.96052522642279e-15\\
3.50900000000002	4.67498888553956e-15\\
3.51199999999998	1.93440317337521e-15\\
3.51200000000003	4.57548009146303e-15\\
3.51499999999999	4.47864538194265e-15\\
3.51799999999995	1.88667617383614e-15\\
3.51799999999999	1.88667617383548e-15\\
3.51800000000003	4.4007413498075e-15\\
3.51999999999997	2.70817316817208e-15\\
3.52	4.36338122081068e-15\\
3.52199999999995	2.69277986120195e-15\\
3.52399999999989	2.68273658199094e-15\\
3.52599999999997	2.6780413610076e-15\\
3.526	4.28534195609759e-15\\
3.52999999999989	1.0927268438685e-15\\
3.53399999999978	1.15050956236315e-15\\
3.53499999999998	3.49522367330704e-15\\
3.535	4.26382631599575e-15\\
3.53799999999997	1.99708682409783e-15\\
3.53799999999999	4.28211437016353e-15\\
3.53999999999997	2.7949752595221e-15\\
3.54	4.30138275026141e-15\\
3.54199999999998	2.83311945111147e-15\\
3.54399999999996	2.87664117748414e-15\\
3.54599999999997	2.92554896822768e-15\\
3.546	4.39320046576178e-15\\
3.54999999999996	1.58425067084857e-15\\
3.54999999999999	4.48281885911035e-15\\
3.553	2.41672467192617e-15\\
3.55300000000003	4.55560062591827e-15\\
3.55600000000004	2.51717284564715e-15\\
3.55900000000005	2.61079526075023e-15\\
3.55999999999997	4.01865449996146e-15\\
3.56	4.70242675075357e-15\\
3.56599999999999	7.72444757178883e-16\\
3.56600000000002	4.80286016953751e-15\\
3.56699999999996	4.15732542620631e-15\\
3.56699999999999	4.81733218842918e-15\\
3.56799999999996	4.17441616280144e-15\\
3.56899999999992	4.19082970914314e-15\\
3.56999999999998	4.20656687211888e-15\\
3.57	4.85687681603393e-15\\
3.57199999999993	3.58887348766978e-15\\
3.57299999999996	4.24972760736075e-15\\
3.57299999999999	4.89062753279433e-15\\
3.57499999999992	3.6397157172941e-15\\
3.57699999999985	3.67298152295673e-15\\
3.57899999999996	3.70279442062435e-15\\
3.57899999999999	4.93789755978171e-15\\
3.57999999999997	4.3318165743272e-15\\
3.58	4.94296966695361e-15\\
3.58099999999998	4.3403133485611e-15\\
3.58199999999997	4.34797960542445e-15\\
3.58399999999993	3.76225688455949e-15\\
3.58599999999997	3.78002619310216e-15\\
3.586	4.95658754479812e-15\\
3.58999999999993	2.64491177586326e-15\\
3.59399999999986	2.68817223583863e-15\\
3.59599999999996	3.8174399583623e-15\\
3.59599999999999	4.91523816954913e-15\\
3.59999999999997	2.72572453840273e-15\\
3.6	4.87624821472985e-15\\
3.60399999999998	2.73256249537674e-15\\
3.60499999999998	4.29543082005213e-15\\
3.605	4.8094093625715e-15\\
3.60599999999997	4.28318391260291e-15\\
3.606	4.79362181665378e-15\\
3.60699999999997	4.27010368946399e-15\\
3.60799999999994	4.25618950876138e-15\\
3.60799999999999	4.75962099783795e-15\\
3.60999999999993	3.72801664229314e-15\\
3.61199999999987	3.70615042818675e-15\\
3.61399999999996	3.68223979074294e-15\\
3.61399999999999	4.64445892327756e-15\\
3.61799999999987	2.68218521063829e-15\\
3.61999999999997	3.59819247029745e-15\\
3.62	4.51255006048605e-15\\
3.62399999999988	2.63307585130556e-15\\
3.62499999999996	3.9537171738006e-15\\
3.62499999999999	4.38966632307416e-15\\
3.62599999999997	3.93150162536076e-15\\
3.626	4.36366168802187e-15\\
3.62699999999998	3.90878602016055e-15\\
3.62799999999997	3.88556924435289e-15\\
3.62999999999993	3.41670848632659e-15\\
3.63199999999997	3.37413872517554e-15\\
3.632	4.19753112821267e-15\\
3.63599999999993	2.47950293218567e-15\\
3.63999999999985	2.42288747724281e-15\\
3.63999999999997	3.96218619874122e-15\\
3.64	3.96218619877573e-15\\
3.64599999999997	1.58645974621142e-15\\
3.646	3.7824855847724e-15\\
3.65199999999997	1.53575185961243e-15\\
3.652	3.59969723736246e-15\\
3.65399999999996	2.87875951682553e-15\\
3.65399999999999	3.53802605433442e-15\\
3.65599999999995	2.8308373159315e-15\\
3.65799999999992	2.78238554900145e-15\\
3.65999999999996	2.73339472205878e-15\\
3.65999999999999	3.35061806008169e-15\\
3.66399999999992	2.02765612953213e-15\\
3.66599999999996	2.58191836206765e-15\\
3.66599999999999	3.16416661023433e-15\\
3.66999999999992	1.91138186719099e-15\\
3.67399999999984	1.835668985625e-15\\
3.67499999999998	2.62915835133197e-15\\
3.67500000000001	2.89457765117864e-15\\
3.67999999999997	1.46187223486732e-15\\
3.68	2.74969399814055e-15\\
3.68299999999996	1.922617122187e-15\\
3.68299999999999	2.66434244856724e-15\\
3.68599999999995	1.86122735599732e-15\\
3.686	2.58012876884922e-15\\
3.68699999999998	2.31768705960693e-15\\
3.68700000000001	3.01408160699714e-15\\
3.68799999999998	3.42844482183266e-15\\
3.68899999999995	3.17121062246651e-15\\
3.6909999999999	2.66448409926584e-15\\
3.69299999999998	2.38788157462711e-15\\
3.693	2.38788157462635e-15\\
3.6969999999999	1.64454088501664e-15\\
3.69999999999997	1.16756872921609e-15\\
3.7	1.59094978445133e-15\\
3.7039999999999	1.32323760361728e-15\\
3.70599999999997	8.96343940653133e-16\\
3.706	1.68810385642796e-15\\
3.7099999999999	1.2479544595566e-15\\
3.70999999999998	1.24795445954603e-15\\
3.71000000000001	1.98608830467557e-15\\
3.71199999999996	1.59029159956326e-15\\
3.71199999999999	1.59029159956283e-15\\
3.71399999999995	1.56060146930297e-15\\
3.71599999999991	1.18514479699224e-15\\
3.71799999999996	1.16696034456838e-15\\
3.71799999999999	1.50555679565091e-15\\
3.71999999999997	1.47809154377625e-15\\
3.72	1.47809154377584e-15\\
3.72199999999998	1.44950608662992e-15\\
3.72399999999997	1.10336040604382e-15\\
3.72599999999997	1.08087821210075e-15\\
3.726	1.39389854357375e-15\\
3.728	1.36686555884289e-15\\
3.72800000000003	1.36686555884251e-15\\
3.73000000000004	1.34033933393741e-15\\
3.73200000000004	1.01611867670951e-15\\
3.73600000000004	6.81770043323361e-16\\
3.74	3.63382152098628e-16\\
3.74000000000003	9.35886851325653e-16\\
3.74099999999996	1.34002720440413e-15\\
3.74099999999999	1.34002720440377e-15\\
3.74199999999996	1.32715458530931e-15\\
3.74299999999992	1.17992131487118e-15\\
3.74499999999985	1.02367436462608e-15\\
3.745	1.15703675753981e-15\\
3.74500000000003	1.42041862972052e-15\\
3.746	1.27690531155325e-15\\
3.74600000000003	1.2769053115529e-15\\
3.747	1.26465021773738e-15\\
3.74799999999997	1.12358160577924e-15\\
3.74999999999991	9.74008590447278e-16\\
3.752	8.28137994827283e-16\\
3.75200000000003	1.08057904807206e-15\\
3.75599999999991	7.96339649910377e-16\\
3.758	5.3777680186339e-16\\
3.75800000000003	1.02057183939606e-15\\
3.75999999999997	1.0015583371676e-15\\
3.76	1.00155833716733e-15\\
3.76199999999995	9.8299817518253e-16\\
3.76399999999989	7.39349267994517e-16\\
3.76599999999997	7.26113319983342e-16\\
3.766	9.47223407742629e-16\\
3.76999999999989	7.00836964435459e-16\\
3.76999999999996	7.00836964430253e-16\\
3.76999999999999	1.12130109779527e-15\\
3.77399999999988	6.77138276993639e-16\\
3.77599999999999	4.62053225071194e-16\\
3.77600000000002	8.65476002950897e-16\\
3.77999999999991	6.40246913095843e-16\\
3.77999999999996	6.40246913093746e-16\\
3.78	1.02262162740224e-15\\
3.78399999999989	6.11551992182503e-16\\
3.78599999999997	4.09508555160649e-16\\
3.786	7.83461252437794e-16\\
3.78999999999989	5.7006909499076e-16\\
3.79199999999997	3.74732533387794e-16\\
3.792	7.36599950959896e-16\\
3.79599999999989	5.30392767110182e-16\\
3.79899999999996	2.47700897306783e-16\\
3.79899999999999	5.98121463139363e-16\\
3.79999999999997	7.62929640437522e-16\\
3.8	7.62929640437304e-16\\
3.80099999999999	7.55336570442777e-16\\
3.80199999999997	6.63126131635149e-16\\
3.80399999999993	5.64857294802647e-16\\
3.80599999999997	4.68091421170463e-16\\
3.806	6.35289056215599e-16\\
3.80999999999993	4.44452731782946e-16\\
3.81099999999996	3.56522225968401e-16\\
3.81099999999999	6.82604842310472e-16\\
3.81499999999992	4.16143748254132e-16\\
3.81499999999998	4.16143748251539e-16\\
3.81500000000001	7.34067587569952e-16\\
3.81899999999993	3.94580384630244e-16\\
3.81999999999997	3.10847512313498e-16\\
3.82	6.22240333378797e-16\\
3.82399999999993	3.68597919764384e-16\\
3.826	2.05951452254424e-16\\
3.82600000000003	5.0952325172312e-16\\
3.82700000000001	5.78362243446684e-16\\
3.82700000000003	5.78362243446512e-16\\
3.82799999999998	5.72305473105286e-16\\
3.82800000000001	5.72305473105114e-16\\
3.82899999999997	5.66300682787014e-16\\
3.82999999999993	4.86860052436883e-16\\
3.83199999999986	4.02684872640091e-16\\
3.83399999999998	3.197526321061e-16\\
3.83400000000001	4.64961591796015e-16\\
3.83799999999985	3.02350875336356e-16\\
3.84	1.52169913964705e-16\\
3.84000000000003	4.34803840231724e-16\\
3.84399999999988	2.78985675491607e-16\\
3.846	1.34676023620792e-16\\
3.84600000000003	4.06636900848177e-16\\
3.84999999999988	2.57089209872254e-16\\
3.84999999999998	2.57089209867163e-16\\
3.85000000000001	5.18856847568344e-16\\
3.85399999999985	2.43288347981143e-16\\
3.85599999999998	1.0820334639349e-16\\
3.85600000000001	3.63433137301463e-16\\
3.85699999999996	4.21779741715785e-16\\
3.85699999999999	4.21779741715657e-16\\
3.85799999999995	4.17364665123351e-16\\
3.85899999999992	3.51357554188257e-16\\
3.85999999999997	3.47421297729209e-16\\
3.86	4.08677016629095e-16\\
3.86199999999993	3.39681031227099e-16\\
3.86399999999985	2.11496915691321e-16\\
3.86599999999997	2.0559329443359e-16\\
3.866	3.24724218870819e-16\\
3.86899999999996	2.55667992125931e-16\\
3.86899999999999	2.55667992125842e-16\\
3.87199999999995	2.46707551662762e-16\\
3.87499999999991	6.81793921514923e-17\\
3.87999999999997	-4.99016838104561e-17\\
3.88	1.16530655720271e-16\\
3.88499999999998	1.08445849441234e-16\\
3.88500000000001	1.08445849441189e-16\\
3.88599999999999	3.13280061547233e-16\\
3.88600000000002	3.13280061547145e-16\\
3.88700000000001	3.10199026763553e-16\\
3.88799999999999	2.56942388169761e-16\\
3.88999999999996	2.01920465821115e-16\\
3.89199999999999	1.48196128299719e-16\\
3.89200000000002	2.46701236299091e-16\\
3.89599999999996	1.41726813472118e-16\\
3.89799999999999	4.33217875165449e-17\\
3.89800000000002	2.3258894282662e-16\\
3.89999999999997	2.28136699065375e-16\\
3.9	2.28136699065311e-16\\
3.90199999999996	2.23799539536835e-16\\
3.90399999999991	1.30411180328672e-16\\
3.90599999999997	1.27823434803865e-16\\
3.906	2.15467094810525e-16\\
3.90999999999991	1.2293372280298e-16\\
3.91399999999982	-4.94097008355987e-17\\
3.91499999999996	7.63089947878469e-17\\
3.91499999999999	2.38321031184713e-16\\
3.91999999999997	7.31001162148516e-17\\
3.92	7.31001162148351e-17\\
3.92499999999999	7.04206385423128e-17\\
3.92599999999997	3.24336930027144e-17\\
3.926	2.16597850512836e-16\\
3.92699999999996	2.14794105622829e-16\\
3.92699999999999	2.14794105622778e-16\\
3.92799999999995	2.13009573616768e-16\\
3.92899999999992	1.76081068572833e-16\\
3.93099999999984	1.38402672525672e-16\\
3.93299999999996	1.01751772975449e-16\\
3.93299999999999	1.70481217818242e-16\\
3.93699999999984	9.89559572630915e-17\\
3.93999999999997	-1.85928497593001e-18\\
3.94	1.29367381894474e-16\\
3.94399999999985	9.46569650553485e-17\\
3.94399999999996	9.46569650526166e-17\\
3.94399999999999	2.17422223665347e-16\\
3.94599999999997	1.54355296286616e-16\\
3.946	1.54355296286584e-16\\
3.94799999999999	1.52150051149754e-16\\
3.94999999999997	9.15669419541199e-17\\
3.95199999999997	9.06579033907314e-17\\
3.952	1.47955833948019e-16\\
3.95499999999998	1.17314102367915e-16\\
3.95500000000001	1.17314102367896e-16\\
3.95799999999998	1.15277264963003e-16\\
3.95999999999998	6.06079355018181e-17\\
3.96	1.40255880012076e-16\\
3.96299999999998	1.12125727680949e-16\\
3.96599999999995	3.47932272430472e-17\\
3.966	1.10379361368035e-16\\
3.96600000000003	1.83573172135842e-16\\
3.96700000000001	1.58352832923071e-16\\
3.96700000000003	1.5835283292304e-16\\
3.96800000000001	1.57280922241111e-16\\
3.96899999999998	1.32612379878567e-16\\
3.97099999999993	1.07707205306134e-16\\
3.97299999999996	8.36289569112375e-17\\
3.97299999999999	1.29555801897199e-16\\
3.97699999999989	8.28216726102391e-17\\
3.97899999999996	3.85979655226873e-17\\
3.97899999999999	1.25356651740749e-16\\
3.97999999999997	1.45473009587544e-16\\
3.98	1.45473009587515e-16\\
3.98099999999999	1.44496757016563e-16\\
3.98199999999997	1.23095498079809e-16\\
3.98399999999994	1.0125550346559e-16\\
3.98599999999997	7.9945741395291e-17\\
3.986	1.1996638825757e-16\\
3.98999999999994	7.82423032376416e-17\\
3.98999999999997	7.82423032374882e-17\\
3.99000000000001	1.54980501215555e-16\\
3.99399999999994	7.66060081177173e-17\\
3.99599999999998	3.8413280081623e-17\\
3.99600000000001	1.12575313095738e-16\\
3.99999999999994	7.42746664039601e-17\\
3.99999999999997	7.42746664038765e-17\\
4	1.44680858589139e-16\\
4.00199999999993	1.08425147410912e-16\\
4.00199999999999	1.08425147410443e-16\\
4.00399999999992	1.07087517633966e-16\\
4.00599999999986	7.20874089881865e-17\\
4.00599999999993	1.05772326532364e-16\\
4.006	1.38861975123025e-16\\
4.00999999999987	7.07073683309806e-17\\
4.01199999999995	3.75395116395242e-17\\
4.012	1.01958829266305e-16\\
4.01599999999987	6.87976890532725e-17\\
4.01999999999973	6.72593892760745e-18\\
4.01999999999995	6.76620225931755e-17\\
4.02	1.26276721628821e-16\\
4.02500000000001	5.20326278829483e-17\\
4.02500000000006	5.20326278829416e-17\\
4.02599999999995	1.07696327465266e-16\\
4.026	1.07696327465227e-16\\
4.02699999999996	1.07026568825674e-16\\
4.02799999999992	9.2904971019663e-17\\
4.02999999999985	7.85482937221865e-17\\
4.03099999999993	7.81744050375871e-17\\
4.03099999999999	1.04405017463024e-16\\
4.03499999999984	6.39454914804535e-17\\
4.03699999999994	3.80937364727245e-17\\
4.03699999999999	8.83981048511845e-17\\
4.03799999999995	1.00004510064336e-16\\
4.038	1.00004510064298e-16\\
4.03899999999996	9.93422899686887e-17\\
4.03999999999991	8.67492231112081e-17\\
4.04	9.86828762098629e-17\\
4.04199999999991	8.55865009258651e-17\\
4.04399999999982	6.09336723219735e-17\\
4.046	6.00776489597844e-17\\
4.04600000000006	8.32864183698184e-17\\
4.04999999999988	5.83760238078212e-17\\
4.05399999999969	1.16783140991385e-17\\
4.05999999999994	-1.13089798536136e-17\\
4.06	3.26118263775953e-17\\
};
\end{axis}
\end{tikzpicture}%
}
      \caption{The evolution of the difference in angular displacement between
        RM and EDF of pendulum $P_2$ for execution time $C_2 = 6$ ms.}
      \label{fig:02.6.6.2_diff}
    \end{figure}
  \end{minipage}
\end{minipage}
}

\noindent\makebox[\textwidth][c]{%
\begin{minipage}{\linewidth}
  \begin{minipage}{0.45\linewidth}
    \begin{figure}[H]\centering
      \scalebox{0.7}{% This file was created by matlab2tikz.
%
%The latest updates can be retrieved from
%  http://www.mathworks.com/matlabcentral/fileexchange/22022-matlab2tikz-matlab2tikz
%where you can also make suggestions and rate matlab2tikz.
%
\definecolor{mycolor1}{rgb}{0.00000,0.44700,0.74100}%
\definecolor{mycolor2}{rgb}{0.85000,0.32500,0.09800}%
%
\begin{tikzpicture}

\begin{axis}[%
width=4.133in,
height=3.26in,
at={(0.693in,0.44in)},
scale only axis,
xmin=0,
xmax=1,
xmajorgrids,
ymin=-0.05,
ymax=0.12,
ymajorgrids,
axis background/.style={fill=white}
]
\pgfplotsset{max space between ticks=50}
\addplot [color=mycolor1,solid,forget plot]
  table[row sep=crcr]{%
0	0.10153\\
3.15544362088405e-30	0.10153\\
0.000656101980281985	0.101530709989553\\
0.00393661188169191	0.101555560666546\\
0.00599999999999994	0.10158938182706\\
0.006	0.10158938182706\\
0.012	0.101767596768679\\
0.0120000000000001	0.101767596768679\\
0.018	0.102064853328018\\
0.0180000000000001	0.102064853328018\\
0.0199999999999998	0.10213229397037\\
0.02	0.10213229397037\\
0.026	0.101716366820929\\
0.0260000000000002	0.101716366820929\\
0.0289999999999998	0.10116051629751\\
0.029	0.10116051629751\\
0.0319999999999996	0.100372547032607\\
0.0349999999999991	0.0993522286136822\\
0.035	0.0993522286136819\\
0.0399999999999996	0.0971345253230206\\
0.04	0.0971345253230203\\
0.0449999999999996	0.0942687495276435\\
0.0459999999999996	0.0936176301011948\\
0.046	0.0936176301011944\\
0.047	0.0929404751960583\\
0.0470000000000004	0.092940475196058\\
0.0490000000000003	0.0915346663918733\\
0.0510000000000002	0.0900778367053162\\
0.055	0.087010350712579\\
0.0579999999999996	0.0845743497767112\\
0.058	0.0845743497767108\\
0.0599999999999996	0.082885486841699\\
0.06	0.0828854868416986\\
0.0619999999999995	0.0811444789878217\\
0.0639999999999991	0.0793510999448183\\
0.0659999999999991	0.0775051166450086\\
0.066	0.0775051166450078\\
0.0699999999999991	0.0736543707951944\\
0.07	0.0736543707951936\\
0.0700000000000009	0.0736543707951927\\
0.074	0.0695902396041448\\
0.076	0.067477498629349\\
0.0760000000000009	0.0674774986293481\\
0.08	0.063214520109214\\
0.0800000000000009	0.0632145201092131\\
0.0839999999999999	0.0589831831846656\\
0.086	0.0568786923069218\\
0.0860000000000009	0.0568786923069209\\
0.0869999999999991	0.0558291214523434\\
0.087	0.0558291214523424\\
0.0880000000000004	0.0547812881682954\\
0.0890000000000009	0.053735158410586\\
0.0910000000000017	0.0516478735747052\\
0.0929999999999991	0.0495669956809782\\
0.093	0.0495669956809773\\
0.0970000000000017	0.0454233797984614\\
0.0999999999999991	0.0423304804899824\\
0.1	0.0423304804899815\\
0.104000000000002	0.0382246453687673\\
0.104999999999999	0.037201183877005\\
0.105	0.037201183877004\\
0.105999999999999	0.0361788547394758\\
0.106	0.0361788547394749\\
0.106999999999999	0.0351576247414335\\
0.107999999999998	0.0341374607030283\\
0.109999999999997	0.032100197959437\\
0.111999999999999	0.0300668017458478\\
0.112	0.0300668017458469\\
0.115999999999997	0.0260810521578851\\
0.115999999999998	0.0260810521578834\\
0.116	0.0260810521578817\\
0.119999999999997	0.0222486397395016\\
0.119999999999998	0.0222486397395\\
0.12	0.0222486397394984\\
0.123999999999997	0.0185675721876527\\
0.125999999999999	0.0167831916823141\\
0.126	0.0167831916823133\\
0.127999999999998	0.0150359358757995\\
0.128	0.015035935875798\\
0.129999999999998	0.0133255776955832\\
0.131999999999996	0.0116518948633202\\
0.135999999999993	0.00841368993471907\\
0.139999999999998	0.00531963767383442\\
0.14	0.00531963767383307\\
0.144999999999998	0.00165234374314154\\
0.145	0.00165234374314027\\
0.145999999999998	0.000945347349630184\\
0.146	0.000945347349628936\\
0.146999999999999	0.000247117489500776\\
0.147999999999998	-0.000442368523203164\\
0.149999999999997	-0.00179519832772279\\
0.151999999999998	-0.00311331787782769\\
0.152	-0.00311331787782884\\
0.155999999999997	-0.00564610693435692\\
0.157999999999998	-0.00686110560234216\\
0.158	-0.00686110560234323\\
0.16	-0.00803453236788899\\
0.160000000000002	-0.00803453236789001\\
0.162000000000002	-0.00915901971699857\\
0.164000000000002	-0.0102347137882067\\
0.166	-0.01126175437883\\
0.166000000000002	-0.0112617543788309\\
0.170000000000002	-0.013170402707014\\
0.174	-0.0148859569244241\\
0.174000000000001	-0.0148859569244248\\
0.175	-0.0152847812356085\\
0.175000000000002	-0.0152847812356092\\
0.176000000000001	-0.0156716063502979\\
0.177	-0.0160464448364982\\
0.178999999999998	-0.0167602102479289\\
0.179999999999998	-0.0170991603633261\\
0.18	-0.0170991603633267\\
0.183999999999997	-0.0183356626541074\\
0.186	-0.0188824741501191\\
0.186000000000002	-0.0188824741501196\\
0.189999999999998	-0.0198335731693998\\
0.192	-0.0202379842976751\\
0.192000000000002	-0.0202379842976754\\
0.195999999999998	-0.0209395293867196\\
0.199999999999995	-0.0215214927991707\\
0.199999999999997	-0.021521492799171\\
0.2	-0.0215214927991714\\
0.202999999999998	-0.0218796742141849\\
0.203	-0.0218796742141851\\
0.205999999999998	-0.022170867118344\\
0.206	-0.0221708671183442\\
0.208999999999998	-0.0223951566640238\\
0.209999999999998	-0.0224550643318199\\
0.21	-0.02245506433182\\
0.211999999999998	-0.0225526084365377\\
0.212	-0.0225526084365378\\
0.213999999999998	-0.0226204677288109\\
0.215999999999997	-0.0226586510276281\\
0.217999999999998	-0.022667163295306\\
0.218	-0.022667163295306\\
0.219999999999998	-0.0226554236175242\\
0.22	-0.0226554236175242\\
0.221999999999998	-0.0226328484480922\\
0.223999999999996	-0.0225994348531525\\
0.225999999999998	-0.0225551784902755\\
0.226	-0.0225551784902754\\
0.229999999999996	-0.022434113044661\\
0.231999999999998	-0.0223572882282681\\
0.232	-0.022357288228268\\
0.235999999999996	-0.0221710044863672\\
0.237999999999998	-0.0220615213514281\\
0.238	-0.022061521351428\\
0.239999999999998	-0.0219411255414157\\
0.24	-0.0219411255414156\\
0.241999999999998	-0.0218098014097432\\
0.243999999999996	-0.0216675318895284\\
0.245	-0.0215922868864084\\
0.245000000000002	-0.0215922868864082\\
0.245999999999998	-0.0215142984913975\\
0.246	-0.0215142984913974\\
0.246999999999999	-0.0214335641707048\\
0.247999999999998	-0.0213500813012717\\
0.249999999999997	-0.0211748589774669\\
0.252	-0.0209886087480779\\
0.252000000000003	-0.0209886087480776\\
0.256	-0.0206014286709987\\
0.259999999999997	-0.0202068421468002\\
0.26	-0.0202068421467999\\
0.260999999999996	-0.0201070143055804\\
0.261	-0.02010701430558\\
0.261999999999998	-0.0200067075173039\\
0.262999999999996	-0.0199059185225771\\
0.264999999999993	-0.0197028807995861\\
0.265999999999997	-0.0196006254746493\\
0.266	-0.019600625474649\\
0.269999999999993	-0.019186616711248\\
0.271999999999997	-0.0189765727153971\\
0.272	-0.0189765727153967\\
0.275999999999993	-0.018550269005692\\
0.279999999999986	-0.018115492540964\\
0.279999999999993	-0.0181154925409633\\
0.28	-0.0181154925409625\\
0.285999999999996	-0.0174469457727508\\
0.286	-0.0174469457727504\\
0.289999999999996	-0.0169899886525383\\
0.29	-0.0169899886525379\\
0.293999999999996	-0.0165237476563155\\
0.295999999999997	-0.0162870701532782\\
0.296	-0.0162870701532778\\
0.297999999999997	-0.0160479804054144\\
0.298	-0.0160479804054139\\
0.299999999999997	-0.0158087498176743\\
0.3	-0.0158087498176738\\
0.301999999999997	-0.0155716497765734\\
0.303999999999993	-0.0153366494686365\\
0.305999999999997	-0.0151037183532005\\
0.306	-0.0151037183532001\\
0.309999999999993	-0.0146439428784181\\
0.313999999999986	-0.0141920843350631\\
0.314999999999997	-0.0140803291030243\\
0.315	-0.0140803291030239\\
0.318999999999997	-0.0136380368624003\\
0.319	-0.0136380368623999\\
0.319999999999996	-0.0135286280122194\\
0.32	-0.013528628012219\\
0.320999999999998	-0.0134196777139701\\
0.321999999999996	-0.0133111824278474\\
0.323999999999993	-0.013095542806738\\
0.325999999999996	-0.012881681124345\\
0.326	-0.0128816811243446\\
0.329999999999993	-0.0124591806297764\\
0.331	-0.0123546235651481\\
0.331000000000004	-0.0123546235651477\\
0.333	-0.0121467672790984\\
0.333000000000004	-0.012146767279098\\
0.335	-0.0119408489384267\\
0.336999999999996	-0.0117371250911896\\
0.339999999999996	-0.0114355961565763\\
0.34	-0.0114355961565759\\
0.343999999999993	-0.0110409935488434\\
0.345999999999997	-0.0108468213198131\\
0.346	-0.0108468213198128\\
0.347999999999997	-0.0106547014046\\
0.348	-0.0106547014045997\\
0.349999999999997	-0.0104646088353912\\
0.35	-0.0104646088353909\\
0.351999999999997	-0.0102765189077371\\
0.353999999999993	-0.0100904071775072\\
0.354	-0.0100904071775066\\
0.357999999999993	-0.0097240218150832\\
0.359999999999996	-0.00954370056746534\\
0.36	-0.00954370056746502\\
0.363999999999993	-0.00919077679500085\\
0.365999999999996	-0.00901865164489114\\
0.366	-0.00901865164489083\\
0.369999999999993	-0.00868296296745967\\
0.373999999999986	-0.00835854497898147\\
0.376999999999997	-0.00812252620401951\\
0.377	-0.00812252620401923\\
0.379999999999997	-0.00789268319222999\\
0.38	-0.00789268319222972\\
0.382999999999996	-0.0076689487358352\\
0.384999999999997	-0.00752315369204404\\
0.385	-0.00752315369204379\\
0.385999999999997	-0.00745125741033368\\
0.386	-0.00745125741033343\\
0.386999999999998	-0.00738002550035561\\
0.387999999999996	-0.0073094556477985\\
0.388999999999997	-0.00723954555986822\\
0.389	-0.00723954555986797\\
0.390999999999997	-0.00710169561382885\\
0.392999999999993	-0.00696645774728413\\
0.394999999999997	-0.00683381438470716\\
0.395	-0.00683381438470692\\
0.398999999999993	-0.00657602231243384\\
0.399999999999997	-0.00651310049586287\\
0.4	-0.00651310049586265\\
0.403999999999993	-0.00626745670468034\\
0.405999999999997	-0.00614823302933968\\
0.406	-0.00614823302933947\\
0.409999999999993	-0.0059169046458613\\
0.411999999999997	-0.00580476987426106\\
0.412	-0.00580476987426086\\
0.415999999999993	-0.00558748630451453\\
0.419999999999986	-0.00537942373935587\\
0.419999999999996	-0.00537942373935532\\
0.42	-0.00537942373935513\\
0.426	-0.00508438396598435\\
0.426000000000004	-0.00508438396598418\\
0.432000000000004	-0.00480950407811492\\
0.432000000000007	-0.00480950407811476\\
0.434999999999997	-0.00467913021761094\\
0.435	-0.00467913021761079\\
0.43799999999999	-0.00455289249453435\\
0.439999999999997	-0.00447101358809983\\
0.44	-0.00447101358809969\\
0.44299999999999	-0.00435158798916604\\
0.445999999999979	-0.00423620274967594\\
0.445999999999995	-0.00423620274967535\\
0.446	-0.00423620274967515\\
0.447	-0.00419863299527993\\
0.447000000000004	-0.0041986329952798\\
0.448000000000004	-0.00416150719059865\\
0.449000000000004	-0.00412482412935519\\
0.451000000000004	-0.00405278148424236\\
0.454999999999997	-0.00391395763418054\\
0.455	-0.00391395763418042\\
0.459	-0.00378208927666521\\
0.459999999999997	-0.00375020087054584\\
0.46	-0.00375020087054573\\
0.463999999999997	-0.00362693089697287\\
0.464	-0.00362693089697276\\
0.465999999999997	-0.00356785213171983\\
0.466	-0.00356785213171973\\
0.467999999999996	-0.00351046720332325\\
0.469999999999993	-0.00345476865403072\\
0.471999999999997	-0.00340074924525286\\
0.472	-0.00340074924525277\\
0.473	-0.00337436701620138\\
0.473000000000004	-0.00337436701620128\\
0.474000000000004	-0.00334830620674449\\
0.475000000000004	-0.00332247022024791\\
0.477000000000004	-0.00327146936579996\\
0.479999999999997	-0.00319663350718657\\
0.48	-0.00319663350718648\\
0.484	-0.00309992726254386\\
0.485999999999997	-0.00305287778361639\\
0.486	-0.00305287778361631\\
0.489999999999997	-0.00296135553910436\\
0.49	-0.00296135553910428\\
0.492999999999997	-0.00289494511762741\\
0.493	-0.00289494511762733\\
0.495999999999997	-0.00283042557771053\\
0.498999999999993	-0.00276777805296202\\
0.499	-0.00276777805296187\\
0.499999999999997	-0.00274730836613627\\
0.5	-0.00274730836613619\\
0.500999999999998	-0.00272704397607654\\
0.501999999999997	-0.00270698422436106\\
0.503999999999993	-0.00266747603564985\\
0.505999999999993	-0.00262877866551443\\
0.506	-0.0026287786655143\\
0.507999999999993	-0.00259088708486403\\
0.508	-0.00259088708486389\\
0.509999999999993	-0.0025536816380878\\
0.511999999999986	-0.00251704275872247\\
0.515999999999972	-0.00244544572890958\\
0.519999999999993	-0.00237605877493396\\
0.52	-0.00237605877493384\\
0.521999999999993	-0.00234218275983508\\
0.522	-0.00234218275983496\\
0.523999999999993	-0.00230884582602377\\
0.524999999999993	-0.00229237815789886\\
0.525	-0.00229237815789875\\
0.525999999999993	-0.00227604364080765\\
0.526	-0.00227604364080754\\
0.526999999999998	-0.00225984174404697\\
0.527999999999997	-0.00224377194122217\\
0.529999999999993	-0.00221202653324303\\
0.531999999999993	-0.00218080329123382\\
0.532	-0.00218080329123371\\
0.535999999999993	-0.00211990714143867\\
0.538	-0.00209022631958258\\
0.538000000000007	-0.00209022631958248\\
0.539999999999993	-0.00206106707883049\\
0.54	-0.00206106707883039\\
0.541999999999986	-0.00203244087391804\\
0.543999999999972	-0.00200434398458051\\
0.545999999999993	-0.00197677275934289\\
0.546	-0.0019767727593428\\
0.549999999999972	-0.00192319303645576\\
0.550999999999993	-0.00191012112917138\\
0.551	-0.00191012112917129\\
0.554999999999972	-0.00185911285239129\\
0.556999999999993	-0.00183437053952553\\
0.557	-0.00183437053952544\\
0.559999999999993	-0.00179820028528159\\
0.56	-0.0017982002852815\\
0.562999999999993	-0.00176315251209034\\
0.565999999999986	-0.00172921697150584\\
0.565999999999993	-0.00172921697150576\\
0.566	-0.00172921697150568\\
0.571999999999986	-0.00166464321764433\\
0.571999999999993	-0.00166464321764425\\
0.572	-0.00166464321764418\\
0.577999999999986	-0.00160367441896799\\
0.579999999999993	-0.00158397814979059\\
0.58	-0.00158397814979052\\
0.585999999999986	-0.00152673349519535\\
0.585999999999993	-0.00152673349519529\\
0.586	-0.00152673349519522\\
0.591999999999986	-0.00147220340707188\\
0.591999999999993	-0.00147220340707182\\
0.592	-0.00147220340707176\\
0.594999999999993	-0.00144593627029927\\
0.595	-0.00144593627029921\\
0.597999999999993	-0.00142032409937663\\
0.599999999999993	-0.00140360948239768\\
0.6	-0.00140360948239762\\
0.602999999999993	-0.001379072371753\\
0.605999999999986	-0.00135517067500935\\
0.606	-0.00135517067500924\\
0.606999999999993	-0.00134734343251699\\
0.607	-0.00134734343251693\\
0.607999999999999	-0.00133956609669239\\
0.608999999999997	-0.00133181875233916\\
0.609000000000004	-0.00133181875233911\\
0.611	-0.00131641303217761\\
0.612999999999997	-0.00130112426990796\\
0.614999999999997	-0.00128595047860117\\
0.615000000000004	-0.00128595047860112\\
0.618999999999997	-0.00125593993565148\\
0.619999999999993	-0.00124850609326256\\
0.62	-0.00124850609326251\\
0.623999999999993	-0.00121903865121622\\
0.625999999999993	-0.00120446233883791\\
0.626	-0.00120446233883786\\
0.629999999999993	-0.00117561505946791\\
0.63	-0.00117561505946785\\
0.633999999999993	-0.0011471624321774\\
0.635999999999993	-0.00113307948298211\\
0.636	-0.00113307948298206\\
0.637999999999993	-0.00111909504850738\\
0.638	-0.00111909504850734\\
0.639999999999993	-0.00110521269416423\\
0.64	-0.00110521269416418\\
0.641999999999993	-0.00109143061580051\\
0.643999999999986	-0.00107774702230003\\
0.645999999999993	-0.00106416013534157\\
0.646	-0.00106416013534152\\
0.649999999999986	-0.00103726943041231\\
0.65	-0.00103726943041222\\
0.650000000000007	-0.00103726943041217\\
0.653999999999993	-0.00101074452171113\\
0.657999999999979	-0.000984571620102137\\
0.659999999999993	-0.00097161291299919\\
0.66	-0.000971612912999144\\
0.664999999999993	-0.000939575233859847\\
0.665	-0.000939575233859801\\
0.665999999999993	-0.00093322758932793\\
0.666	-0.000933227589327885\\
0.666999999999998	-0.000926899425850729\\
0.667000000000006	-0.000926899425850684\\
0.668000000000004	-0.000920590537822319\\
0.669000000000002	-0.000914300720268944\\
0.670999999999998	-0.000901777479783274\\
0.673000000000005	-0.000889328076872133\\
0.673000000000013	-0.000889328076872089\\
0.677000000000005	-0.00086464432148919\\
0.678	-0.000858517014438211\\
0.678000000000007	-0.000858517014438167\\
0.679999999999993	-0.000846339711696203\\
0.68	-0.000846339711696159\\
0.681999999999986	-0.000834281739258305\\
0.683999999999972	-0.000822341530069241\\
0.686	-0.000810517532378022\\
0.686000000000007	-0.00081051753237798\\
0.689999999999979	-0.000787212039832741\\
0.69399999999995	-0.000764353146115868\\
0.695999999999993	-0.000753087451302199\\
0.696	-0.000753087451302159\\
0.699999999999993	-0.000730876245329672\\
0.7	-0.000730876245329633\\
0.703999999999993	-0.00070908235178475\\
0.705999999999993	-0.000698338348289211\\
0.706	-0.000698338348289173\\
0.707999999999993	-0.00068769444086561\\
0.708	-0.000687694440865572\\
0.709999999999993	-0.00067714924623682\\
0.711999999999986	-0.000666701393948323\\
0.713999999999993	-0.000656349526196159\\
0.714	-0.000656349526196122\\
0.717999999999986	-0.000636078617611166\\
0.719999999999993	-0.00062619450172427\\
0.72	-0.000626194501724236\\
0.723999999999986	-0.000606922523917738\\
0.724999999999993	-0.000602207000631983\\
0.725	-0.000602207000631949\\
0.725999999999993	-0.000597532157408669\\
0.726	-0.000597532157408636\\
0.726999999999999	-0.000592897842365431\\
0.727999999999997	-0.000588303904932954\\
0.729999999999993	-0.000579236567184816\\
0.731999999999993	-0.000570328965772121\\
0.732	-0.00057032896577209\\
0.734999999999993	-0.000557264542613193\\
0.735	-0.000557264542613163\\
0.737999999999993	-0.000544553106164312\\
0.74	-0.000536273079156958\\
0.740000000000007	-0.000536273079156929\\
0.743	-0.000524141744377641\\
0.745999999999993	-0.000512353710753088\\
0.746000000000007	-0.000512353710753034\\
0.746999999999993	-0.000508500056703298\\
0.747	-0.000508500056703271\\
0.747999999999999	-0.000504676960932998\\
0.748999999999997	-0.000500877221943231\\
0.750999999999993	-0.000493347321250716\\
0.753999999999993	-0.000482224584205322\\
0.754	-0.000482224584205296\\
0.757999999999993	-0.000467710503789883\\
0.759999999999993	-0.000460586867822856\\
0.76	-0.00046058686782283\\
0.763999999999993	-0.000446601775326444\\
0.766	-0.000439738501293001\\
0.766000000000007	-0.000439738501292977\\
0.77	-0.000426266037799022\\
0.770000000000007	-0.000426266037798998\\
0.774	-0.000413126535572189\\
0.776	-0.000406679503690046\\
0.776000000000007	-0.000406679503690023\\
0.779999999999993	-0.000394023129236104\\
0.78	-0.000394023129236082\\
0.782999999999993	-0.000384734035915887\\
0.783	-0.000384734035915866\\
0.785999999999993	-0.000375616080568731\\
0.786000000000001	-0.000375616080568709\\
0.788999999999994	-0.000366666596980394\\
0.791999999999987	-0.000357882968200854\\
0.792	-0.000357882968200814\\
0.792000000000008	-0.000357882968200794\\
0.797999999999994	-0.000340803049008005\\
0.799999999999993	-0.000335251421910655\\
0.8	-0.000335251421910636\\
0.804999999999993	-0.000321675427423215\\
0.805000000000001	-0.000321675427423196\\
0.805999999999993	-0.000319011477089507\\
0.806	-0.000319011477089488\\
0.806999999999994	-0.000316364406371278\\
0.807999999999987	-0.000313734129263531\\
0.809999999999973	-0.000308523614594737\\
0.811999999999993	-0.00030337925592404\\
0.812	-0.000303379255924022\\
0.815999999999973	-0.00029330361602072\\
0.817999999999993	-0.000288375344007574\\
0.818000000000001	-0.000288375344007556\\
0.819999999999993	-0.000283519246830481\\
0.82	-0.000283519246830464\\
0.821999999999993	-0.000278734693393331\\
0.823999999999986	-0.000274021061895062\\
0.825999999999993	-0.000269377739751625\\
0.826	-0.000269377739751609\\
0.829999999999986	-0.000260299618814199\\
0.833999999999972	-0.000251495611262912\\
0.839999999999993	-0.000238793589107709\\
0.84	-0.000238793589107694\\
0.840999999999993	-0.000236734496236461\\
0.841000000000001	-0.000236734496236446\\
0.841999999999994	-0.000234691769142919\\
0.842999999999987	-0.000232665341449229\\
0.844999999999973	-0.000228661121436768\\
0.845999999999993	-0.000226683199021279\\
0.846	-0.000226683199021265\\
0.849999999999973	-0.000218931266585115\\
0.851999999999993	-0.000215150275003348\\
0.852	-0.000215150275003334\\
0.855999999999973	-0.00020777578416863\\
0.857999999999993	-0.000204181326525959\\
0.858	-0.000204181326525947\\
0.86	-0.000200659695034456\\
0.860000000000007	-0.000200659695034444\\
0.862000000000007	-0.000197222025178053\\
0.864000000000007	-0.000193867870196777\\
0.866	-0.000190596794184314\\
0.866000000000007	-0.000190596794184303\\
0.87	-0.000184302189378774\\
0.870000000000007	-0.000184302189378763\\
0.874	-0.000178334938471567\\
0.874999999999994	-0.000176893907546205\\
0.875000000000001	-0.000176893907546195\\
0.876	-0.000175473094715837\\
0.876000000000007	-0.000175473094715827\\
0.877000000000007	-0.000174072453819297\\
0.878000000000006	-0.000172691939349612\\
0.879999999999998	-0.00016999111093329\\
0.880000000000006	-0.000169991110933281\\
0.882000000000005	-0.000167370258467693\\
0.884000000000004	-0.000164829041347366\\
0.886000000000005	-0.000162367129315463\\
0.886000000000013	-0.000162367129315454\\
0.888000000000007	-0.000159984202423165\\
0.888000000000014	-0.000159984202423156\\
0.890000000000009	-0.000157658439687576\\
0.892000000000004	-0.000155368027553403\\
0.895999999999993	-0.000150892068998686\\
0.898999999999993	-0.000147625714564065\\
0.899000000000001	-0.000147625714564057\\
0.899999999999993	-0.00014655399987869\\
0.9	-0.000146553999878682\\
0.900999999999994	-0.000145490761997462\\
0.901999999999987	-0.000144435966370759\\
0.903999999999973	-0.000142351565074158\\
0.905999999999993	-0.000140300525099852\\
0.906	-0.000140300525099845\\
0.909999999999973	-0.000136297467211274\\
0.909999999999987	-0.000136297467211261\\
0.910000000000001	-0.000136297467211247\\
0.910999999999993	-0.000135317142276678\\
0.911000000000001	-0.000135317142276671\\
0.911999999999994	-0.000134344929059292\\
0.912999999999987	-0.000133380795971798\\
0.914999999999973	-0.000131476645150429\\
0.916999999999993	-0.000129604442348912\\
0.917000000000001	-0.000129604442348906\\
0.919999999999993	-0.000126844444577203\\
0.92	-0.000126844444577197\\
0.922999999999993	-0.000124132846321049\\
0.925999999999986	-0.000121468854672222\\
0.925999999999993	-0.000121468854672216\\
0.926	-0.000121468854672209\\
0.927999999999993	-0.000119718923100263\\
0.928000000000001	-0.000119718923100257\\
0.929999999999994	-0.000117989576105028\\
0.931999999999987	-0.0001162805889373\\
0.933999999999994	-0.000114591739496929\\
0.934000000000001	-0.000114591739496923\\
0.937999999999987	-0.000111273578459155\\
0.939999999999993	-0.000109643835633193\\
0.940000000000001	-0.000109643835633187\\
0.943999999999987	-0.000106441966336215\\
0.944999999999994	-0.000105653350485271\\
0.945000000000001	-0.000105653350485266\\
0.945999999999993	-0.000104869423749749\\
0.946000000000001	-0.000104869423749743\\
0.946999999999994	-0.000104090160660422\\
0.947999999999987	-0.000103315535898963\\
0.949999999999973	-0.000101780100840244\\
0.952	-0.000100262919028272\\
0.952000000000008	-0.000100262919028267\\
0.95599999999998	-9.72822451165736e-05\\
0.956999999999994	-9.65480910081088e-05\\
0.957000000000001	-9.65480910081036e-05\\
0.96	-9.43716809550119e-05\\
0.960000000000008	-9.43716809550068e-05\\
0.963000000000007	-9.22338558011013e-05\\
0.966000000000007	-9.01339904172115e-05\\
0.966000000000014	-9.01339904172065e-05\\
0.969000000000007	-8.80714707712307e-05\\
0.969000000000014	-8.80714707712259e-05\\
0.972000000000007	-8.60456937607083e-05\\
0.975	-8.40560670179295e-05\\
0.979999999999994	-8.08187809069555e-05\\
0.980000000000001	-8.0818780906951e-05\\
0.985999999999987	-7.70607639647571e-05\\
0.986000000000001	-7.70607639647485e-05\\
0.991999999999987	-7.34368486360907e-05\\
0.992000000000001	-7.34368486360824e-05\\
0.997999999999987	-6.99427958836705e-05\\
0.998000000000001	-6.99427958836625e-05\\
0.999999999999993	-6.88108070655468e-05\\
1	-6.88108070655428e-05\\
1.00199999999999	-6.77017405470094e-05\\
1.00399999999999	-6.66154521915754e-05\\
1.00599999999999	-6.55518008250965e-05\\
1.006	-6.55518008250891e-05\\
1.00999999999999	-6.34918590568378e-05\\
1.01399999999997	-6.15208443685755e-05\\
1.01499999999999	-6.10418644667751e-05\\
1.015	-6.10418644667683e-05\\
1.01999999999999	-5.87288317020414e-05\\
1.02	-5.8728831702035e-05\\
1.02499999999999	-5.6550881909985e-05\\
1.02599999999999	-5.61313443081439e-05\\
1.026	-5.61313443081379e-05\\
1.02699999999999	-5.57171252822709e-05\\
1.027	-5.57171252822651e-05\\
1.02799999999999	-5.53082113749104e-05\\
1.02899999999999	-5.49045893001819e-05\\
1.03099999999997	-5.41131683658863e-05\\
1.03299999999999	-5.33427596262248e-05\\
1.033	-5.33427596262194e-05\\
1.03699999999997	-5.18538819492714e-05\\
1.04	-5.07776103968344e-05\\
1.04000000000001	-5.07776103968294e-05\\
1.04399999999999	-4.93958322415679e-05\\
1.044	-4.93958322415631e-05\\
1.046	-4.87275622638067e-05\\
1.04600000000001	-4.8727562263802e-05\\
1.04800000000001	-4.80742552366946e-05\\
1.05	-4.74358262563825e-05\\
1.05000000000001	-4.74358262563781e-05\\
1.05200000000001	-4.6812192352745e-05\\
1.05200000000002	-4.68121923527406e-05\\
1.05400000000002	-4.62032724784491e-05\\
1.05600000000002	-4.56089874983894e-05\\
1.05800000000001	-4.502926017923e-05\\
1.05800000000002	-4.50292601792259e-05\\
1.05999999999999	-4.44593232113491e-05\\
1.06	-4.44593232113451e-05\\
1.06199999999996	-4.38944105574388e-05\\
1.06399999999992	-4.33344488013981e-05\\
1.06599999999999	-4.27793651705134e-05\\
1.066	-4.27793651705095e-05\\
1.06999999999992	-4.16835443554885e-05\\
1.07299999999999	-4.08739518103376e-05\\
1.073	-4.08739518103337e-05\\
1.07699999999992	-3.98104090742664e-05\\
1.07899999999999	-3.92853023273898e-05\\
1.079	-3.92853023273861e-05\\
1.07999999999999	-3.9024385159051e-05\\
1.08	-3.90243851590473e-05\\
1.08099999999999	-3.87645474796231e-05\\
1.08199999999999	-3.85057808469614e-05\\
1.08399999999997	-3.79914271274617e-05\\
1.08499999999999	-3.77358233293241e-05\\
1.085	-3.77358233293205e-05\\
1.08599999999999	-3.74812571550265e-05\\
1.086	-3.74812571550229e-05\\
1.08699999999999	-3.722772033389e-05\\
1.08799999999999	-3.69752046284852e-05\\
1.08999999999997	-3.64732037811429e-05\\
1.09199999999999	-3.5975189372913e-05\\
1.092	-3.59751893729095e-05\\
1.09599999999997	-3.49902089688725e-05\\
1.09999999999995	-3.40190988402216e-05\\
1.09999999999997	-3.40190988402149e-05\\
1.1	-3.40190988402083e-05\\
1.10199999999999	-3.35385870898403e-05\\
1.102	-3.35385870898369e-05\\
1.10399999999999	-3.30613541462998e-05\\
1.10599999999997	-3.25873379853504e-05\\
1.10599999999999	-3.25873379853469e-05\\
1.106	-3.25873379853435e-05\\
1.10999999999997	-3.16487100090595e-05\\
1.11199999999999	-3.11839762096732e-05\\
1.112	-3.11839762096699e-05\\
1.11599999999997	-3.02633669975464e-05\\
1.11999999999994	-2.93541707830853e-05\\
1.12	-2.93541707830723e-05\\
1.12000000000001	-2.93541707830691e-05\\
1.126	-2.80107434675148e-05\\
1.12600000000001	-2.80107434675116e-05\\
1.13099999999999	-2.69088908905314e-05\\
1.131	-2.69088908905283e-05\\
1.13599999999997	-2.58221467330313e-05\\
1.13699999999999	-2.56065330211548e-05\\
1.137	-2.56065330211518e-05\\
1.138	-2.53914812935822e-05\\
1.13800000000001	-2.53914812935792e-05\\
1.13900000000001	-2.51774034896481e-05\\
1.14	-2.49647115799376e-05\\
1.14000000000001	-2.49647115799345e-05\\
1.14100000000001	-2.47533986541448e-05\\
1.14200000000001	-2.45434578467879e-05\\
1.14400000000001	-2.41276653480854e-05\\
1.146	-2.37172800473974e-05\\
1.14600000000001	-2.37172800473945e-05\\
1.15000000000001	-2.29125184020574e-05\\
1.15400000000001	-2.21287545503868e-05\\
1.15499999999999	-2.19360465075884e-05\\
1.155	-2.19360465075857e-05\\
1.15999999999999	-2.09915912801461e-05\\
1.16	-2.09915912801435e-05\\
1.16499999999999	-2.00783855482156e-05\\
1.16599999999999	-1.98994286231405e-05\\
1.166	-1.98994286231379e-05\\
1.17099999999999	-1.90227453074682e-05\\
1.17199999999999	-1.88509890297659e-05\\
1.172	-1.88509890297635e-05\\
1.173	-1.86804131030493e-05\\
1.17300000000001	-1.86804131030469e-05\\
1.17400000000001	-1.85109488079252e-05\\
1.17500000000001	-1.8342527460796e-05\\
1.17700000000001	-1.8008791756394e-05\\
1.17999999999999	-1.75158746025563e-05\\
1.18	-1.7515874602554e-05\\
1.184	-1.68727761656201e-05\\
1.18599999999999	-1.65571861052058e-05\\
1.186	-1.65571861052036e-05\\
1.18899999999999	-1.60911344282276e-05\\
1.189	-1.60911344282255e-05\\
1.18999999999999	-1.59377192253401e-05\\
1.19	-1.59377192253379e-05\\
1.19099999999999	-1.5785263367037e-05\\
1.19199999999999	-1.5633761900059e-05\\
1.19399999999997	-1.53336024819783e-05\\
1.19499999999999	-1.51849347787266e-05\\
1.195	-1.51849347787245e-05\\
1.19899999999997	-1.45995649873224e-05\\
1.19999999999999	-1.44555240217226e-05\\
1.2	-1.44555240217205e-05\\
1.20399999999997	-1.38884256893322e-05\\
1.20599999999999	-1.3610251456297e-05\\
1.206	-1.3610251456295e-05\\
1.20699999999999	-1.34724921807964e-05\\
1.207	-1.34724921807945e-05\\
1.20799999999999	-1.33357962417154e-05\\
1.20899999999999	-1.32003432840482e-05\\
1.21099999999997	-1.29331487498932e-05\\
1.21499999999995	-1.24134845105127e-05\\
1.21799999999999	-1.20364879048475e-05\\
1.218	-1.20364879048457e-05\\
1.21999999999999	-1.1791167475368e-05\\
1.22	-1.17911674753663e-05\\
1.22199999999999	-1.15506182739558e-05\\
1.22399999999997	-1.1314809037192e-05\\
1.22499999999999	-1.1198672304761e-05\\
1.225	-1.11986723047593e-05\\
1.22599999999999	-1.10837091192543e-05\\
1.226	-1.10837091192527e-05\\
1.22699999999999	-1.09699157455985e-05\\
1.22799999999999	-1.08572884866372e-05\\
1.22999999999997	-1.06355177136626e-05\\
1.23	-1.06355177136596e-05\\
1.23399999999997	-1.02058110620362e-05\\
1.236	-9.99781933872632e-06\\
1.23600000000001	-9.99781933872486e-06\\
1.23999999999999	-9.59459531980094e-06\\
1.24	-9.59459531979954e-06\\
1.24399999999997	-9.20765768247975e-06\\
1.24599999999999	-9.02023306905809e-06\\
1.246	-9.02023306905678e-06\\
1.247	-8.92802107153153e-06\\
1.24700000000001	-8.92802107153022e-06\\
1.24800000000001	-8.83680527476635e-06\\
1.24900000000001	-8.74658271515275e-06\\
1.25100000000001	-8.56910561433993e-06\\
1.253	-8.39556670415108e-06\\
1.25300000000001	-8.39556670414986e-06\\
1.25700000000001	-8.06021375244364e-06\\
1.25999999999999	-7.81887274414232e-06\\
1.26	-7.81887274414119e-06\\
1.264	-7.5105154988457e-06\\
1.266	-7.3620476953007e-06\\
1.26600000000001	-7.36204769529966e-06\\
1.27000000000001	-7.07643724985542e-06\\
1.272	-6.93925748999008e-06\\
1.27200000000001	-6.93925748998912e-06\\
1.276	-6.67444280931904e-06\\
1.27600000000001	-6.67444280931812e-06\\
1.27999999999999	-6.42116256064517e-06\\
1.28000000000001	-6.42116256064429e-06\\
1.28399999999999	-6.17928507468131e-06\\
1.28599999999999	-6.0625828437179e-06\\
1.28600000000001	-6.06258284371708e-06\\
1.288	-5.94868460986684e-06\\
1.28800000000001	-5.94868460986604e-06\\
1.29	-5.83757557096877e-06\\
1.29199999999999	-5.72924128728039e-06\\
1.29499999999999	-5.57191182786585e-06\\
1.295	-5.57191182786512e-06\\
1.29899999999998	-5.37172245716849e-06\\
1.29999999999999	-5.32337549353129e-06\\
1.3	-5.32337549353061e-06\\
1.30399999999998	-5.13674203058674e-06\\
1.30499999999999	-5.0917646831574e-06\\
1.305	-5.09176468315676e-06\\
1.30599999999999	-5.04745678773358e-06\\
1.306	-5.04745678773295e-06\\
1.30699999999999	-5.00381690478335e-06\\
1.30799999999999	-4.96084361644268e-06\\
1.30999999999997	-4.87689126044531e-06\\
1.31199999999999	-4.79558880928305e-06\\
1.312	-4.79558880928248e-06\\
1.31299999999999	-4.75592798271572e-06\\
1.313	-4.75592798271516e-06\\
1.31399999999999	-4.71677321990754e-06\\
1.31499999999999	-4.67797077168384e-06\\
1.31699999999997	-4.6014177876761e-06\\
1.32	-4.4891995238802e-06\\
1.32000000000001	-4.48919952387968e-06\\
1.32399999999999	-4.34439862710617e-06\\
1.326	-4.27404415283193e-06\\
1.32600000000001	-4.27404415283144e-06\\
1.32999999999999	-4.13738143513845e-06\\
1.33	-4.13738143513797e-06\\
1.33399999999997	-4.00605477935771e-06\\
1.334	-4.00605477935689e-06\\
1.33799999999997	-3.87999591452373e-06\\
1.34	-3.81892131876401e-06\\
1.34000000000001	-3.81892131876358e-06\\
1.34399999999999	-3.70064211339162e-06\\
1.346	-3.6434221321988e-06\\
1.34600000000001	-3.6434221321984e-06\\
1.348	-3.58747192828866e-06\\
1.34800000000002	-3.58747192828826e-06\\
1.35	-3.53260538914318e-06\\
1.35199999999999	-3.47863654304474e-06\\
1.35599999999997	-3.37336399061715e-06\\
1.35999999999999	-3.27159954392725e-06\\
1.36	-3.27159954392689e-06\\
1.36299999999999	-3.19754651137043e-06\\
1.363	-3.19754651137008e-06\\
1.36499999999999	-3.14924683732566e-06\\
1.365	-3.14924683732532e-06\\
1.36599999999999	-3.12541534024489e-06\\
1.366	-3.12541534024456e-06\\
1.36699999999999	-3.10179503512559e-06\\
1.36799999999999	-3.07838515455147e-06\\
1.36999999999997	-3.0321936315308e-06\\
1.37199999999999	-2.98683476812717e-06\\
1.372	-2.98683476812685e-06\\
1.37599999999997	-2.89859154837075e-06\\
1.378	-2.85569572391848e-06\\
1.37800000000001	-2.85569572391818e-06\\
1.37999999999999	-2.81362054340756e-06\\
1.38	-2.81362054340727e-06\\
1.38199999999997	-2.77237146071002e-06\\
1.38399999999994	-2.73194311509065e-06\\
1.38599999999999	-2.69233025247608e-06\\
1.386	-2.6923302524758e-06\\
1.38999999999994	-2.61553048931817e-06\\
1.39199999999999	-2.57833360786682e-06\\
1.392	-2.57833360786656e-06\\
1.39599999999994	-2.50632167414293e-06\\
1.39799999999999	-2.47149726318928e-06\\
1.398	-2.47149726318903e-06\\
1.39999999999999	-2.43745448776194e-06\\
1.4	-2.43745448776171e-06\\
1.40199999999999	-2.40418892367754e-06\\
1.40399999999997	-2.37169624773965e-06\\
1.40599999999999	-2.33997223719625e-06\\
1.406	-2.33997223719603e-06\\
1.40999999999997	-2.2788138203174e-06\\
1.412	-2.24937146582634e-06\\
1.41200000000001	-2.24937146582613e-06\\
1.41599999999999	-2.19220882895316e-06\\
1.41999999999996	-2.13696263853035e-06\\
1.41999999999998	-2.13696263853005e-06\\
1.42	-2.13696263852976e-06\\
1.42099999999999	-2.1234471483059e-06\\
1.421	-2.12344714830571e-06\\
1.42199999999999	-2.11004919714298e-06\\
1.42299999999999	-2.0967683496783e-06\\
1.42499999999997	-2.07055624367132e-06\\
1.42599999999999	-2.05762413350032e-06\\
1.426	-2.05762413350014e-06\\
1.42999999999997	-2.00704555163605e-06\\
1.43199999999999	-1.98244043749658e-06\\
1.432	-1.9824404374964e-06\\
1.43499999999999	-1.94637897294012e-06\\
1.435	-1.94637897293995e-06\\
1.43799999999998	-1.91132360510995e-06\\
1.43999999999999	-1.88850722844079e-06\\
1.44	-1.88850722844063e-06\\
1.44299999999999	-1.85510605315271e-06\\
1.44599999999997	-1.82268428515175e-06\\
1.44599999999999	-1.82268428515158e-06\\
1.446	-1.82268428515142e-06\\
1.44699999999999	-1.81209302838325e-06\\
1.447	-1.8120930283831e-06\\
1.44799999999999	-1.80157806527552e-06\\
1.44899999999999	-1.79110792462691e-06\\
1.44999999999999	-1.78068226626543e-06\\
1.45	-1.78068226626528e-06\\
1.45199999999999	-1.75996304293358e-06\\
1.45399999999997	-1.73941770261373e-06\\
1.45599999999999	-1.71904357523116e-06\\
1.456	-1.71904357523102e-06\\
1.45999999999997	-1.67879838994445e-06\\
1.46	-1.67879838994416e-06\\
1.46000000000001	-1.67879838994402e-06\\
1.46399999999999	-1.63920656536118e-06\\
1.466	-1.61964921846246e-06\\
1.46600000000001	-1.61964921846232e-06\\
1.46999999999999	-1.58099894679845e-06\\
1.47	-1.58099894679831e-06\\
1.47399999999997	-1.54295119419918e-06\\
1.47599999999999	-1.52414706954113e-06\\
1.476	-1.52414706954099e-06\\
1.47899999999998	-1.49620584971478e-06\\
1.479	-1.49620584971465e-06\\
1.47999999999999	-1.48696105081122e-06\\
1.48	-1.48696105081109e-06\\
1.48099999999999	-1.47775022089775e-06\\
1.48199999999999	-1.46857306071638e-06\\
1.48399999999997	-1.45031855798153e-06\\
1.48599999999999	-1.43219517024947e-06\\
1.486	-1.43219517024935e-06\\
1.48999999999997	-1.39633233530651e-06\\
1.491	-1.3874449134606e-06\\
1.49100000000001	-1.38744491346047e-06\\
1.49499999999999	-1.35219971092675e-06\\
1.49899999999996	-1.31742797778249e-06\\
1.49999999999999	-1.30880690045681e-06\\
1.5	-1.30880690045669e-06\\
1.50499999999999	-1.26611865609272e-06\\
1.505	-1.2661186560926e-06\\
1.50599999999999	-1.25766249398597e-06\\
1.506	-1.25766249398585e-06\\
1.50699999999999	-1.24923285103744e-06\\
1.50799999999999	-1.24082945336297e-06\\
1.508	-1.24082945336285e-06\\
1.50999999999999	-1.22410030258449e-06\\
1.51199999999997	-1.20747286753343e-06\\
1.51399999999999	-1.19094498730647e-06\\
1.514	-1.19094498730635e-06\\
1.51799999999997	-1.158179312168e-06\\
1.518	-1.15817931216778e-06\\
1.51999999999999	-1.1419696900432e-06\\
1.52	-1.14196969004309e-06\\
1.52199999999999	-1.12591597200658e-06\\
1.52399999999997	-1.1100160717153e-06\\
1.52599999999999	-1.09426792281653e-06\\
1.526	-1.09426792281642e-06\\
1.52999999999997	-1.06321871217032e-06\\
1.53399999999994	-1.03275219894112e-06\\
1.537	-1.01027517023457e-06\\
1.53700000000001	-1.01027517023446e-06\\
1.53999999999999	-9.88110421209936e-07\\
1.54	-9.88110421209832e-07\\
1.54299999999997	-9.66251470729093e-07\\
1.54599999999994	-9.44691926800628e-07\\
1.54599999999997	-9.44691926800429e-07\\
1.546	-9.4469192680023e-07\\
1.549	-9.23425485090297e-07\\
1.54900000000001	-9.23425485090197e-07\\
1.55200000000001	-9.02445927066862e-07\\
1.55500000000001	-8.81747117991856e-07\\
1.55500000000003	-8.81747117991759e-07\\
1.55999999999999	-8.48194456003589e-07\\
1.56	-8.48194456003495e-07\\
1.56499999999996	-8.16053356535617e-07\\
1.56599999999998	-8.09792206690529e-07\\
1.566	-8.0979220669044e-07\\
1.57099999999996	-7.79310619006414e-07\\
1.57199999999998	-7.73377747302594e-07\\
1.572	-7.7337774730251e-07\\
1.57499999999999	-7.55902538303565e-07\\
1.575	-7.55902538303483e-07\\
1.57799999999999	-7.38908482400922e-07\\
1.57999999999999	-7.27843959352312e-07\\
1.58	-7.27843959352234e-07\\
1.58299999999999	-7.11640851211603e-07\\
1.58599999999997	-6.95905953442358e-07\\
1.586	-6.95905953442214e-07\\
1.58699999999999	-6.90764235089701e-07\\
1.587	-6.90764235089629e-07\\
1.58799999999999	-6.85663181686217e-07\\
1.58899999999999	-6.80591948457276e-07\\
1.59099999999997	-6.70538284435167e-07\\
1.59499999999995	-6.50781613179119e-07\\
1.59499999999997	-6.50781613178988e-07\\
1.595	-6.50781613178857e-07\\
1.59999999999999	-6.26730074329425e-07\\
1.6	-6.26730074329358e-07\\
1.60499999999999	-6.03376217291001e-07\\
1.60599999999999	-5.9878748850891e-07\\
1.606	-5.98787488508845e-07\\
1.60699999999998	-5.9422575733718e-07\\
1.607	-5.94225757337116e-07\\
1.60799999999999	-5.89690875570795e-07\\
1.60899999999999	-5.85182695868667e-07\\
1.60999999999998	-5.80701071761272e-07\\
1.61	-5.80701071761208e-07\\
1.61199999999999	-5.7181690876294e-07\\
1.61399999999997	-5.63037231991701e-07\\
1.61599999999999	-5.54360900439764e-07\\
1.616	-5.54360900439702e-07\\
1.61999999999997	-5.37312505757157e-07\\
1.61999999999999	-5.37312505757096e-07\\
1.62	-5.37312505757035e-07\\
1.62399999999997	-5.20661591763625e-07\\
1.624	-5.20661591763523e-07\\
1.62599999999999	-5.12482477050178e-07\\
1.626	-5.1248247705012e-07\\
1.62799999999999	-5.04399502377279e-07\\
1.62999999999998	-4.96411617280712e-07\\
1.63199999999999	-4.88517783654041e-07\\
1.632	-4.88517783653985e-07\\
1.63599999999998	-4.73008179387464e-07\\
1.63999999999995	-4.57862626812516e-07\\
1.63999999999998	-4.57862626812426e-07\\
1.64	-4.57862626812337e-07\\
1.645	-4.39430656620511e-07\\
1.64500000000001	-4.39430656620459e-07\\
1.64599999999999	-4.35809722255028e-07\\
1.646	-4.35809722254977e-07\\
1.64699999999999	-4.32210331633317e-07\\
1.64799999999999	-4.28632367809007e-07\\
1.64999999999997	-4.21540256257095e-07\\
1.65199999999999	-4.1453246590766e-07\\
1.652	-4.1453246590761e-07\\
1.65299999999998	-4.11061800670064e-07\\
1.653	-4.11061800670015e-07\\
1.65399999999999	-4.07615664150092e-07\\
1.65499999999997	-4.04193944382489e-07\\
1.65699999999995	-3.97423311211494e-07\\
1.65899999999998	-3.9074902124471e-07\\
1.659	-3.90749021244663e-07\\
1.65999999999999	-3.87447733418197e-07\\
1.66	-3.8744773341815e-07\\
1.66099999999999	-3.84170207095578e-07\\
1.66199999999998	-3.80916335789742e-07\\
1.66399999999995	-3.74479136124336e-07\\
1.66599999999999	-3.68135297842673e-07\\
1.666	-3.68135297842629e-07\\
1.66999999999995	-3.55724419690842e-07\\
1.67399999999991	-3.43677249461267e-07\\
1.67999999999998	-3.26274797479858e-07\\
1.68	-3.26274797479818e-07\\
1.68199999999998	-3.20649167346845e-07\\
1.682	-3.20649167346805e-07\\
1.68399999999998	-3.15109902905386e-07\\
1.68599999999997	-3.09656284235622e-07\\
1.68599999999999	-3.09656284235581e-07\\
1.686	-3.09656284235543e-07\\
1.68999999999997	-2.99003160244483e-07\\
1.69199999999999	-2.93802270443621e-07\\
1.692	-2.93802270443585e-07\\
1.69599999999997	-2.83648455616658e-07\\
1.69799999999999	-2.78694210999605e-07\\
1.698	-2.7869421099957e-07\\
1.69999999999999	-2.73838042644009e-07\\
1.7	-2.73838042643975e-07\\
1.70199999999999	-2.69096482510627e-07\\
1.70399999999998	-2.64468914385773e-07\\
1.70599999999999	-2.59954736870151e-07\\
1.706	-2.5995473687012e-07\\
1.70999999999998	-2.51264221687087e-07\\
1.71099999999998	-2.49161563097237e-07\\
1.711	-2.49161563097207e-07\\
1.71499999999997	-2.41028760067188e-07\\
1.715	-2.41028760067138e-07\\
1.71699999999998	-2.3712813568234e-07\\
1.717	-2.37128135682313e-07\\
1.71899999999998	-2.33337348707015e-07\\
1.71999999999999	-2.31482989738485e-07\\
1.72	-2.31482989738459e-07\\
1.72199999999999	-2.27856039600167e-07\\
1.72399999999997	-2.2433772187035e-07\\
1.72599999999999	-2.2092757930803e-07\\
1.726	-2.20927579308006e-07\\
1.72899999999998	-2.16014228277282e-07\\
1.729	-2.16014228277259e-07\\
1.73199999999998	-2.11279373075563e-07\\
1.73499999999997	-2.06659161473222e-07\\
1.73999999999998	-1.99209917947137e-07\\
1.74	-1.99209917947116e-07\\
1.74599999999997	-1.90677747786089e-07\\
1.74599999999998	-1.90677747786066e-07\\
1.746	-1.90677747786043e-07\\
1.74999999999998	-1.85231623478077e-07\\
1.75	-1.85231623478058e-07\\
1.75199999999998	-1.82580087003719e-07\\
1.752	-1.825800870037e-07\\
1.75399999999998	-1.7997577279753e-07\\
1.75599999999997	-1.77418342402563e-07\\
1.75799999999998	-1.74907463454862e-07\\
1.758	-1.74907463454845e-07\\
1.75999999999999	-1.72435201047358e-07\\
1.76	-1.72435201047341e-07\\
1.76199999999999	-1.69993625291426e-07\\
1.76399999999998	-1.67582418879642e-07\\
1.76599999999999	-1.65201268451353e-07\\
1.766	-1.65201268451336e-07\\
1.76899999999998	-1.61685221907955e-07\\
1.769	-1.61685221907939e-07\\
1.77199999999998	-1.58235077824646e-07\\
1.77499999999996	-1.54849827332354e-07\\
1.77499999999998	-1.54849827332335e-07\\
1.775	-1.54849827332316e-07\\
1.78	-1.4934927102657e-07\\
1.78000000000002	-1.49349271026555e-07\\
1.78499999999998	-1.44021764609497e-07\\
1.785	-1.44021764609482e-07\\
1.78600000000001	-1.42976646056969e-07\\
1.78600000000003	-1.42976646056954e-07\\
1.78700000000005	-1.41938242006569e-07\\
1.78800000000006	-1.40906518719901e-07\\
1.79000000000009	-1.3886298057221e-07\\
1.79200000000003	-1.3684576603557e-07\\
1.79200000000004	-1.36845766035556e-07\\
1.79600000000011	-1.3288269702957e-07\\
1.79799999999998	-1.30934686190948e-07\\
1.798	-1.30934686190934e-07\\
1.8	-1.29008685946777e-07\\
1.80000000000002	-1.29008685946763e-07\\
1.80200000000002	-1.27104445995123e-07\\
1.80400000000002	-1.25221718860781e-07\\
1.806	-1.23360259864361e-07\\
1.80600000000002	-1.23360259864348e-07\\
1.80999999999998	-1.19700181361071e-07\\
1.81	-1.19700181361058e-07\\
1.81399999999997	-1.1612230777049e-07\\
1.81799999999994	-1.1262477911315e-07\\
1.82	-1.10905574326329e-07\\
1.82000000000001	-1.10905574326317e-07\\
1.826	-1.05863524556283e-07\\
1.82600000000001	-1.05863524556271e-07\\
1.827	-1.05039749011402e-07\\
1.82700000000001	-1.0503974901139e-07\\
1.828	-1.04220634753536e-07\\
1.82899999999999	-1.03406155168435e-07\\
1.83099999999996	-1.01790994317464e-07\\
1.833	-1.00194056551976e-07\\
1.83300000000001	-1.00194056551964e-07\\
1.83699999999996	-9.70540224703924e-08\\
1.838	-9.62800819292947e-08\\
1.83800000000001	-9.62800819292838e-08\\
1.83999999999999	-9.47486524127589e-08\\
1.84	-9.47486524127481e-08\\
1.84199999999998	-9.32412237778587e-08\\
1.84399999999995	-9.1757600119015e-08\\
1.84599999999999	-9.02975886242995e-08\\
1.846	-9.02975886242892e-08\\
1.84999999999995	-8.74476462023507e-08\\
1.8539999999999	-8.46899149515299e-08\\
1.85499999999998	-8.40147210855867e-08\\
1.855	-8.40147210855771e-08\\
1.85599999999998	-8.33451786492371e-08\\
1.856	-8.33451786492276e-08\\
1.85699999999999	-8.26812658925018e-08\\
1.85799999999997	-8.20229612417127e-08\\
1.85999999999995	-8.07230908870048e-08\\
1.85999999999998	-8.07230908869884e-08\\
1.86	-8.07230908869718e-08\\
1.86399999999995	-7.81897184946446e-08\\
1.86599999999999	-7.69558872200321e-08\\
1.866	-7.69558872200234e-08\\
1.86799999999998	-7.57437444822985e-08\\
1.868	-7.57437444822899e-08\\
1.86999999999998	-7.45531327519333e-08\\
1.87199999999996	-7.33838972969116e-08\\
1.87299999999998	-7.28072480789191e-08\\
1.873	-7.2807248078911e-08\\
1.87699999999996	-7.05406503064172e-08\\
1.87999999999999	-6.88788564376297e-08\\
1.88	-6.88788564376219e-08\\
1.88399999999997	-6.67130800099638e-08\\
1.88499999999998	-6.61804376585938e-08\\
1.885	-6.61804376585863e-08\\
1.88599999999999	-6.56512812277083e-08\\
1.886	-6.56512812277008e-08\\
1.88699999999999	-6.51255935253856e-08\\
1.88799999999998	-6.4603357471988e-08\\
1.88999999999996	-6.35691725543917e-08\\
1.88999999999998	-6.35691725543811e-08\\
1.89	-6.35691725543705e-08\\
1.89199999999999	-6.25485920721163e-08\\
1.892	-6.25485920721091e-08\\
1.89399999999999	-6.15414833913563e-08\\
1.89599999999998	-6.05477156281541e-08\\
1.89799999999999	-5.95671596323403e-08\\
1.898	-5.95671596323334e-08\\
1.89999999999999	-5.86002450878112e-08\\
1.9	-5.86002450878043e-08\\
1.90199999999999	-5.76474034512706e-08\\
1.90399999999997	-5.67085108913239e-08\\
1.90599999999999	-5.57834453893846e-08\\
1.906	-5.57834453893781e-08\\
1.90999999999997	-5.39743164569504e-08\\
1.91399999999994	-5.22190761669391e-08\\
1.91399999999997	-5.22190761669274e-08\\
1.914	-5.22190761669157e-08\\
1.91999999999998	-4.96852684427392e-08\\
1.92	-4.96852684427334e-08\\
1.92499999999998	-4.76626773163142e-08\\
1.925	-4.76626773163085e-08\\
1.92599999999998	-4.72676996631771e-08\\
1.926	-4.72676996631716e-08\\
1.92699999999999	-4.68758720178999e-08\\
1.92799999999997	-4.6487181649755e-08\\
1.92999999999995	-4.5719162332649e-08\\
1.93199999999998	-4.49635418999681e-08\\
1.932	-4.49635418999628e-08\\
1.93599999999995	-4.34869515001132e-08\\
1.9399999999999	-4.20544878489514e-08\\
1.93999999999999	-4.20544878489211e-08\\
1.94	-4.20544878489161e-08\\
1.94299999999998	-4.10086494627108e-08\\
1.943	-4.10086494627059e-08\\
1.94599999999998	-3.99869072199537e-08\\
1.946	-3.99869072199466e-08\\
1.94899999999998	-3.89889623426333e-08\\
1.95199999999997	-3.80145230181793e-08\\
1.95199999999998	-3.80145230181737e-08\\
1.952	-3.8014523018168e-08\\
1.95799999999997	-3.61350280609427e-08\\
1.95999999999998	-3.55287887239109e-08\\
1.96	-3.55287887239067e-08\\
1.96599999999996	-3.37697445805102e-08\\
1.96599999999998	-3.37697445805052e-08\\
1.966	-3.37697445805001e-08\\
1.97199999999996	-3.20986243961502e-08\\
1.97199999999998	-3.20986243961455e-08\\
1.972	-3.20986243961408e-08\\
1.97799999999996	-3.05169658525116e-08\\
1.97799999999998	-3.05169658525071e-08\\
1.978	-3.05169658525026e-08\\
1.98	-3.00100799377869e-08\\
1.98000000000002	-3.00100799377833e-08\\
1.98200000000002	-2.9513249935627e-08\\
1.98400000000002	-2.90264112770205e-08\\
1.986	-2.85495006923701e-08\\
1.98600000000002	-2.85495006923668e-08\\
1.99000000000002	-2.7625217110984e-08\\
1.99400000000003	-2.67399186968558e-08\\
1.995	-2.65246311498279e-08\\
1.99500000000001	-2.65246311498249e-08\\
1.99999999999999	-2.54840669942834e-08\\
2	-2.54840669942805e-08\\
2.00099999999997	-2.52830815487958e-08\\
2.001	-2.52830815487901e-08\\
2.00199999999999	-2.50844564822445e-08\\
2.00299999999997	-2.48881853411806e-08\\
2.00499999999995	-2.45026794046165e-08\\
2.00599999999997	-2.43134320840638e-08\\
2.006	-2.43134320840584e-08\\
2.00999999999995	-2.35796712015113e-08\\
2.01199999999997	-2.32266446240845e-08\\
2.012	-2.32266446240795e-08\\
2.01299999999998	-2.30535745327399e-08\\
2.01300000000001	-2.3053574532735e-08\\
2.014	-2.28823247086397e-08\\
2.01499999999999	-2.27124219122754e-08\\
2.01699999999996	-2.23766353659699e-08\\
2.01999999999997	-2.18829216650166e-08\\
2.02	-2.1882921665012e-08\\
2.02399999999995	-2.12430164797365e-08\\
2.02599999999997	-2.09308471737697e-08\\
2.026	-2.09308471737653e-08\\
2.02999999999995	-2.0321872282986e-08\\
2.03	-2.03218722829794e-08\\
2.03399999999995	-1.97331196443368e-08\\
2.03599999999997	-1.94462306173805e-08\\
2.036	-1.94462306173764e-08\\
2.03999999999995	-1.88872407235026e-08\\
2.04	-1.88872407234954e-08\\
2.04399999999995	-1.83477272797632e-08\\
2.04599999999997	-1.80851861884563e-08\\
2.046	-1.80851861884526e-08\\
2.04799999999997	-1.78274098166458e-08\\
2.048	-1.78274098166422e-08\\
2.04999999999996	-1.75743188434333e-08\\
2.05199999999993	-1.7325834556689e-08\\
2.05599999999987	-1.68425574652024e-08\\
2.05899999999997	-1.64919563301564e-08\\
2.059	-1.64919563301531e-08\\
2.05999999999997	-1.63773273048035e-08\\
2.06	-1.63773273048003e-08\\
2.06099999999999	-1.62638110524097e-08\\
2.06199999999998	-1.61514038842918e-08\\
2.06399999999995	-1.5929902228494e-08\\
2.06499999999997	-1.58208005442463e-08\\
2.065	-1.58208005442433e-08\\
2.06599999999997	-1.57127935510008e-08\\
2.066	-1.57127935509977e-08\\
2.06699999999999	-1.56058777396845e-08\\
2.06799999999998	-1.55000496365958e-08\\
2.06999999999995	-1.52916428370319e-08\\
2.07099999999997	-1.51890573694397e-08\\
2.071	-1.51890573694368e-08\\
2.07499999999995	-1.47894243510093e-08\\
2.07699999999997	-1.45959878302132e-08\\
2.077	-1.45959878302105e-08\\
2.07999999999997	-1.43123892227219e-08\\
2.08	-1.43123892227192e-08\\
2.08299999999998	-1.40355061400603e-08\\
2.08599999999995	-1.37652576178489e-08\\
2.086	-1.37652576178447e-08\\
2.08799999999997	-1.35887386994294e-08\\
2.088	-1.35887386994269e-08\\
2.08999999999997	-1.3415110373187e-08\\
2.09199999999993	-1.32443500740477e-08\\
2.09399999999997	-1.30764356099788e-08\\
2.094	-1.30764356099764e-08\\
2.09799999999993	-1.27490572658089e-08\\
2.09999999999997	-1.25895508395748e-08\\
2.1	-1.25895508395725e-08\\
2.10399999999993	-1.22787998292002e-08\\
2.10599999999997	-1.21275148598169e-08\\
2.106	-1.21275148598148e-08\\
2.10999999999993	-1.1833027685473e-08\\
2.112	-1.16897872089179e-08\\
2.11200000000003	-1.16897872089158e-08\\
2.11599999999996	-1.14094938955127e-08\\
2.11699999999997	-1.13405166233441e-08\\
2.117	-1.13405166233421e-08\\
2.11999999999997	-1.11361794276316e-08\\
2.12	-1.11361794276296e-08\\
2.12299999999998	-1.09356878373254e-08\\
2.12599999999995	-1.07389832259578e-08\\
2.126	-1.07389832259547e-08\\
2.12899999999997	-1.05460080741477e-08\\
2.129	-1.05460080741459e-08\\
2.13199999999996	-1.03567059539355e-08\\
2.13499999999993	-1.01710215105282e-08\\
2.13499999999997	-1.01710215105261e-08\\
2.135	-1.0171021510524e-08\\
2.13999999999997	-9.86943971969575e-09\\
2.14	-9.86943971969406e-09\\
2.14499999999998	-9.57751166256484e-09\\
2.14599999999997	-9.52026349571015e-09\\
2.146	-9.52026349570853e-09\\
2.15099999999997	-9.23960758155095e-09\\
2.15199999999997	-9.18458061692662e-09\\
2.152	-9.18458061692506e-09\\
2.15299999999997	-9.1299175256994e-09\\
2.15299999999999	-9.12991752569786e-09\\
2.15399999999998	-9.07552288675105e-09\\
2.15499999999997	-9.02130128758601e-09\\
2.15699999999995	-8.91337016759465e-09\\
2.15999999999997	-8.75272742454187e-09\\
2.16	-8.75272742454036e-09\\
2.16399999999995	-8.54080482429254e-09\\
2.16599999999997	-8.43578439815359e-09\\
2.166	-8.43578439815211e-09\\
2.16999999999995	-8.22755705002855e-09\\
2.17	-8.22755705002614e-09\\
2.17399999999995	-8.02165763062327e-09\\
2.17499999999997	-7.97053394529519e-09\\
2.175	-7.97053394529374e-09\\
2.17899999999995	-7.76739405101364e-09\\
2.17999999999997	-7.71693953924465e-09\\
2.18	-7.71693953924321e-09\\
2.18399999999995	-7.51639416056964e-09\\
2.18599999999997	-7.41686228160451e-09\\
2.186	-7.4168622816031e-09\\
2.187	-7.36727587519915e-09\\
2.18700000000002	-7.36727587519774e-09\\
2.18800000000002	-7.31780700818303e-09\\
2.18800000000005	-7.31780700818163e-09\\
2.18900000000004	-7.26848568060184e-09\\
2.19000000000004	-7.21934189731601e-09\\
2.19200000000003	-7.12158058267797e-09\\
2.19600000000001	-6.92811861173238e-09\\
2.2	-6.73732052206635e-09\\
2.20000000000003	-6.737320522065e-09\\
2.20399999999997	-6.54908712704114e-09\\
2.204	-6.54908712703981e-09\\
2.20499999999997	-6.50241801615624e-09\\
2.205	-6.50241801615491e-09\\
2.20599999999999	-6.45590156028398e-09\\
2.20600000000003	-6.45590156028202e-09\\
2.20700000000002	-6.40953624810054e-09\\
2.20800000000001	-6.36332057321397e-09\\
2.20999999999998	-6.27133213400704e-09\\
2.21200000000003	-6.17992428783198e-09\\
2.21200000000006	-6.17992428783068e-09\\
2.21600000000001	-5.99880293124934e-09\\
2.21800000000003	-5.90906588228906e-09\\
2.21800000000006	-5.90906588228778e-09\\
2.21999999999997	-5.82023309870345e-09\\
2.22	-5.8202330987022e-09\\
2.22199999999992	-5.73266378818647e-09\\
2.22399999999983	-5.64634657022235e-09\\
2.226	-5.56127022700545e-09\\
2.22600000000003	-5.56127022700425e-09\\
2.22999999999986	-5.39479609879661e-09\\
2.23299999999997	-5.27311672483081e-09\\
2.233	-5.27311672482967e-09\\
2.23699999999983	-5.11504487416896e-09\\
2.23899999999997	-5.03777178652507e-09\\
2.239	-5.03777178652398e-09\\
2.24	-4.99957153363265e-09\\
2.24000000000003	-4.99957153363156e-09\\
2.24100000000003	-4.96166048341273e-09\\
2.24200000000003	-4.92403740413255e-09\\
2.24400000000004	-4.84965027825611e-09\\
2.246	-4.77640048864578e-09\\
2.24600000000003	-4.77640048864475e-09\\
2.25000000000004	-4.63327498677951e-09\\
2.252	-4.56338067392239e-09\\
2.25200000000003	-4.56338067392141e-09\\
2.25600000000004	-4.42679668826937e-09\\
2.25999999999997	-4.29436873708854e-09\\
2.26	-4.29436873708762e-09\\
2.26199999999997	-4.22969167740638e-09\\
2.262	-4.22969167740547e-09\\
2.26399999999996	-4.16602797687513e-09\\
2.26599999999993	-4.10336936175421e-09\\
2.26599999999997	-4.10336936175311e-09\\
2.266	-4.10336936175202e-09\\
2.26999999999994	-3.98103494489507e-09\\
2.272	-3.92134324458105e-09\\
2.27200000000003	-3.92134324458021e-09\\
2.27499999999997	-3.83362821620044e-09\\
2.275	-3.83362821619962e-09\\
2.27799999999994	-3.7480774624267e-09\\
2.27999999999997	-3.69223362185149e-09\\
2.28	-3.6922336218507e-09\\
2.28299999999994	-3.61023471006843e-09\\
2.28599999999989	-3.53033474960294e-09\\
2.28599999999997	-3.53033474960064e-09\\
2.286	-3.53033474959989e-09\\
2.28699999999997	-3.50416380728212e-09\\
2.287	-3.50416380728138e-09\\
2.28799999999999	-3.47818097569758e-09\\
2.28899999999997	-3.45234375576077e-09\\
2.29099999999995	-3.40110279775884e-09\\
2.291	-3.40110279775767e-09\\
2.29499999999995	-3.30033159964338e-09\\
2.29699999999997	-3.25078826330992e-09\\
2.297	-3.25078826330921e-09\\
2.29999999999997	-3.17750795280653e-09\\
2.3	-3.17750795280584e-09\\
2.30299999999998	-3.10545025465897e-09\\
2.30599999999995	-3.03459409821629e-09\\
2.306	-3.0345940982152e-09\\
2.30999999999997	-2.94195252647808e-09\\
2.31	-2.94195252647743e-09\\
2.31399999999997	-2.85136207413993e-09\\
2.31599999999997	-2.80682125985258e-09\\
2.316	-2.80682125985195e-09\\
2.31999999999997	-2.71928150585529e-09\\
2.32	-2.71928150585467e-09\\
2.32399999999997	-2.63380110247954e-09\\
2.32599999999997	-2.59181922935678e-09\\
2.326	-2.59181922935619e-09\\
2.32999999999997	-2.50934485980113e-09\\
2.33199999999997	-2.46884164499077e-09\\
2.332	-2.4688416449902e-09\\
2.33599999999997	-2.38927683571862e-09\\
2.33999999999994	-2.31159982212138e-09\\
2.34	-2.31159982212015e-09\\
2.34000000000003	-2.3115998221196e-09\\
2.34499999999997	-2.2170968415172e-09\\
2.345	-2.21709684151667e-09\\
2.34600000000003	-2.19853584833344e-09\\
2.34600000000006	-2.19853584833292e-09\\
2.34700000000009	-2.18008664049011e-09\\
2.34800000000012	-2.16174861856122e-09\\
2.34899999999997	-2.14352118675251e-09\\
2.349	-2.143521186752e-09\\
2.35100000000006	-2.10739572824019e-09\\
2.35300000000011	-2.07170557009945e-09\\
2.35499999999997	-2.03644607403705e-09\\
2.355	-2.03644607403655e-09\\
2.35800000000003	-1.98435431269727e-09\\
2.35800000000006	-1.98435431269678e-09\\
2.35999999999997	-1.95019309718556e-09\\
2.36	-1.95019309718508e-09\\
2.36199999999992	-1.91652985602559e-09\\
2.36399999999983	-1.8833602143479e-09\\
2.36599999999997	-1.85067986141861e-09\\
2.366	-1.85067986141815e-09\\
2.36999999999983	-1.78677009637755e-09\\
2.37399999999966	-1.72476733700718e-09\\
2.37799999999997	-1.66463935049509e-09\\
2.378	-1.66463935049467e-09\\
2.37999999999997	-1.63526859141116e-09\\
2.38	-1.63526859141074e-09\\
2.38199999999997	-1.60635487934867e-09\\
2.38399999999995	-1.5778944564882e-09\\
2.38599999999997	-1.54988362410941e-09\\
2.386	-1.54988362410901e-09\\
2.38999999999995	-1.49519622766717e-09\\
2.39	-1.49519622766654e-09\\
2.39399999999995	-1.44226426043518e-09\\
2.396	-1.41644792844721e-09\\
2.39600000000002	-1.41644792844684e-09\\
2.39999999999997	-1.36638704217793e-09\\
2.4	-1.36638704217758e-09\\
2.40399999999995	-1.31859312343718e-09\\
2.40599999999997	-1.29553846684214e-09\\
2.406	-1.29553846684181e-09\\
2.40699999999997	-1.28422039147085e-09\\
2.407	-1.28422039147053e-09\\
2.40799999999998	-1.27304132625174e-09\\
2.40899999999997	-1.26200090797633e-09\\
2.41099999999995	-1.24033458194954e-09\\
2.41299999999997	-1.21921859763182e-09\\
2.413	-1.21921859763152e-09\\
2.41499999999997	-1.1986502108012e-09\\
2.415	-1.19865021080091e-09\\
2.41699999999997	-1.17862674839992e-09\\
2.41899999999995	-1.15914560817661e-09\\
2.41999999999997	-1.14960761495658e-09\\
2.42	-1.14960761495631e-09\\
2.42399999999995	-1.11279897499065e-09\\
2.426	-1.09519648431347e-09\\
2.42600000000003	-1.09519648431322e-09\\
2.427	-1.08659468409751e-09\\
2.42700000000003	-1.08659468409726e-09\\
2.42800000000002	-1.07809204550985e-09\\
2.429	-1.06965486457384e-09\\
2.43099999999998	-1.05297578128836e-09\\
2.43499999999993	-1.02039118659298e-09\\
2.43599999999997	-1.01240465175987e-09\\
2.436	-1.01240465175965e-09\\
2.43999999999997	-9.81089183331464e-10\\
2.44	-9.81089183331245e-10\\
2.44399999999998	-9.50770652485436e-10\\
2.446	-9.35980293175643e-10\\
2.44600000000003	-9.35980293175435e-10\\
2.448	-9.21433297782577e-10\\
2.44800000000002	-9.21433297782372e-10\\
2.44999999999999	-9.07127775785499e-10\\
2.45000000000002	-9.07127775785297e-10\\
2.45199999999998	-8.93061868037195e-10\\
2.45399999999995	-8.79233746535199e-10\\
2.45600000000002	-8.65641614174632e-10\\
2.45600000000005	-8.65641614174441e-10\\
2.45999999999998	-8.39039565810524e-10\\
2.46000000000001	-8.39039565810337e-10\\
2.46399999999994	-8.13123177896004e-10\\
2.46499999999997	-8.06749628785623e-10\\
2.465	-8.06749628785443e-10\\
2.46599999999998	-8.00417882623186e-10\\
2.46600000000001	-8.00417882623007e-10\\
2.46699999999999	-7.94127733694624e-10\\
2.46799999999998	-7.87878977632594e-10\\
2.46999999999996	-7.75504833365304e-10\\
2.47199999999998	-7.63293841675914e-10\\
2.47200000000001	-7.63293841675741e-10\\
2.47599999999996	-7.39354989294415e-10\\
2.47999999999991	-7.16049975725573e-10\\
2.47999999999997	-7.16049975725203e-10\\
2.48	-7.1604997572504e-10\\
2.48499999999997	-6.87791618540178e-10\\
2.485	-6.8779161854002e-10\\
2.48599999999997	-6.82254489484068e-10\\
2.486	-6.82254489483912e-10\\
2.48699999999999	-6.76755118572682e-10\\
2.48799999999998	-6.71293327128166e-10\\
2.48999999999995	-6.60481774046852e-10\\
2.49199999999997	-6.4981842516928e-10\\
2.492	-6.49818425169129e-10\\
2.494	-6.39302796777584e-10\\
2.49400000000002	-6.39302796777435e-10\\
2.49600000000002	-6.28934424351249e-10\\
2.49800000000001	-6.18711960415632e-10\\
2.49999999999997	-6.08634076458371e-10\\
2.5	-6.08634076458229e-10\\
2.50399999999999	-5.88906828232688e-10\\
2.50599999999997	-5.79254900209413e-10\\
2.506	-5.79254900209277e-10\\
2.50999999999999	-5.60368164376377e-10\\
2.51399999999998	-5.42029436878167e-10\\
2.51999999999997	-5.15528083367621e-10\\
2.52	-5.15528083367498e-10\\
2.52299999999997	-5.02722027533226e-10\\
2.523	-5.02722027533106e-10\\
2.52599999999996	-4.90207364117709e-10\\
2.526	-4.90207364117557e-10\\
2.52899999999997	-4.7798043358146e-10\\
2.53199999999993	-4.66037660604797e-10\\
2.53199999999997	-4.6603766060467e-10\\
2.532	-4.6603766060454e-10\\
2.53799999999993	-4.42990700486908e-10\\
2.53799999999997	-4.42990700486784e-10\\
2.538	-4.42990700486663e-10\\
2.53999999999997	-4.35577306179189e-10\\
2.54	-4.35577306179085e-10\\
2.54199999999998	-4.2833293973238e-10\\
2.54399999999995	-4.21256659654032e-10\\
2.54599999999997	-4.14347546309813e-10\\
2.546	-4.14347546309716e-10\\
2.54999999999995	-4.01027249815082e-10\\
2.55199999999997	-3.94614335557106e-10\\
2.552	-3.94614335557016e-10\\
2.55499999999997	-3.85301655120551e-10\\
2.555	-3.85301655120465e-10\\
2.55799999999998	-3.76354591203759e-10\\
2.55800000000001	-3.76354591203676e-10\\
2.55999999999997	-3.70591705865999e-10\\
2.56	-3.70591705865918e-10\\
2.56199999999997	-3.64989402857238e-10\\
2.56399999999994	-3.5954695410306e-10\\
2.56599999999997	-3.54263652302022e-10\\
2.566	-3.54263652301948e-10\\
2.56999999999994	-3.44171763694225e-10\\
2.56999999999997	-3.44171763694143e-10\\
2.57	-3.44171763694062e-10\\
2.57399999999994	-3.34549874985756e-10\\
2.57799999999988	-3.25234368463067e-10\\
2.57999999999997	-3.20689988987326e-10\\
2.58	-3.20689988987262e-10\\
2.58099999999997	-3.18445882721159e-10\\
2.581	-3.18445882721096e-10\\
2.58199999999998	-3.16220401361461e-10\\
2.58299999999997	-3.14013472591894e-10\\
2.58499999999995	-3.09654986613691e-10\\
2.58599999999997	-3.07503287798248e-10\\
2.586	-3.07503287798187e-10\\
2.58999999999995	-2.99078496087753e-10\\
2.59	-2.99078496087655e-10\\
2.59199999999997	-2.94974346402438e-10\\
2.592	-2.9497434640238e-10\\
2.59399999999997	-2.90941646548214e-10\\
2.59599999999995	-2.86979872434978e-10\\
2.59799999999997	-2.83088509190025e-10\\
2.598	-2.8308850918997e-10\\
2.59999999999997	-2.7925418880031e-10\\
2.6	-2.79254188800256e-10\\
2.60199999999997	-2.75463550665734e-10\\
2.60399999999995	-2.71716102154625e-10\\
2.60599999999997	-2.68011356248067e-10\\
2.606	-2.68011356248014e-10\\
2.60999999999995	-2.6072805186926e-10\\
2.61	-2.60728051869169e-10\\
2.61399999999994	-2.53609851257762e-10\\
2.61599999999997	-2.50111505176311e-10\\
2.616	-2.50111505176262e-10\\
2.61999999999994	-2.43234048185356e-10\\
2.61999999999997	-2.43234048185303e-10\\
2.62	-2.4323404818525e-10\\
2.62399999999995	-2.36512600596826e-10\\
2.62499999999997	-2.34856203937317e-10\\
2.625	-2.3485620393727e-10\\
2.62599999999997	-2.33209285275986e-10\\
2.626	-2.3320928527594e-10\\
2.62699999999999	-2.31571791105556e-10\\
2.62799999999998	-2.29943668223753e-10\\
2.62999999999995	-2.26715325039822e-10\\
2.63199999999997	-2.23523836168297e-10\\
2.632	-2.23523836168252e-10\\
2.63599999999995	-2.17242902389087e-10\\
2.63899999999997	-2.126168823754e-10\\
2.639	-2.12616882375357e-10\\
2.63999999999997	-2.11090737049337e-10\\
2.64	-2.11090737049294e-10\\
2.64099999999999	-2.09572439542638e-10\\
2.64199999999998	-2.08061940525732e-10\\
2.64399999999995	-2.05064141910377e-10\\
2.64599999999997	-2.02096951608978e-10\\
2.646	-2.02096951608937e-10\\
2.64999999999995	-1.96252857418454e-10\\
2.65099999999997	-1.94810366286848e-10\\
2.651	-1.94810366286807e-10\\
2.65499999999995	-1.89113125405707e-10\\
2.6589999999999	-1.83530029528424e-10\\
2.65999999999997	-1.82151749301171e-10\\
2.66	-1.82151749301132e-10\\
2.66599999999997	-1.74025575902437e-10\\
2.666	-1.74025575902399e-10\\
2.66799999999997	-1.71370466379966e-10\\
2.668	-1.71370466379928e-10\\
2.66999999999996	-1.6874158635423e-10\\
2.67199999999993	-1.66138594171082e-10\\
2.674	-1.63561151545251e-10\\
2.67400000000002	-1.63561151545215e-10\\
2.67799999999996	-1.58481578390978e-10\\
2.67799999999999	-1.58481578390933e-10\\
2.67800000000003	-1.58481578390887e-10\\
2.67999999999997	-1.55985198750437e-10\\
2.68	-1.55985198750402e-10\\
2.68199999999995	-1.53525871195349e-10\\
2.68399999999989	-1.51103276111219e-10\\
2.68599999999997	-1.48717098657164e-10\\
2.686	-1.48717098657131e-10\\
2.68999999999989	-1.44052760907447e-10\\
2.69399999999978	-1.39530433160849e-10\\
2.69499999999997	-1.38421763173827e-10\\
2.695	-1.38421763173795e-10\\
2.69699999999997	-1.36230464404868e-10\\
2.697	-1.36230464404838e-10\\
2.69899999999996	-1.34073650413986e-10\\
2.69999999999997	-1.33008087437165e-10\\
2.7	-1.33008087437134e-10\\
2.70199999999997	-1.30902476420729e-10\\
2.70399999999993	-1.28830657743219e-10\\
2.70599999999997	-1.26792362150778e-10\\
2.706	-1.2679236215075e-10\\
2.70899999999997	-1.23797196270298e-10\\
2.709	-1.2379719627027e-10\\
2.71199999999996	-1.20875986501856e-10\\
2.71299999999997	-1.19918536345893e-10\\
2.713	-1.19918536345866e-10\\
2.71599999999997	-1.17084834504579e-10\\
2.71899999999993	-1.14303575657476e-10\\
2.71999999999997	-1.13388002368431e-10\\
2.72	-1.13388002368406e-10\\
2.72599999999993	-1.08013376484171e-10\\
2.726	-1.08013376484116e-10\\
2.72600000000002	-1.08013376484091e-10\\
2.72999999999997	-1.04541324344143e-10\\
2.73	-1.04541324344119e-10\\
2.73199999999999	-1.02837926664758e-10\\
2.73200000000002	-1.02837926664733e-10\\
2.73400000000002	-1.01155985142364e-10\\
2.73600000000001	-9.94952811916888e-11\\
2.73799999999999	-9.78555989874745e-11\\
2.73800000000002	-9.78555989874514e-11\\
2.73999999999997	-9.62381565827216e-11\\
2.74	-9.62381565826988e-11\\
2.74199999999995	-9.46441749206911e-11\\
2.7439999999999	-9.30734468473773e-11\\
2.74599999999997	-9.1525768230696e-11\\
2.746	-9.15257682306741e-11\\
2.7499999999999	-8.84987577952527e-11\\
2.7539999999998	-8.556157000394e-11\\
2.75499999999997	-8.48411278906161e-11\\
2.755	-8.48411278905957e-11\\
2.75999999999997	-8.13208782060909e-11\\
2.76	-8.13208782060712e-11\\
2.76499999999998	-7.79351492782917e-11\\
2.76500000000001	-7.79351492782728e-11\\
2.76599999999997	-7.72739020653635e-11\\
2.766	-7.72739020653448e-11\\
2.76699999999999	-7.66179038482606e-11\\
2.76700000000002	-7.6617903848242e-11\\
2.76800000000001	-7.59671333131907e-11\\
2.76899999999999	-7.53215693168406e-11\\
2.77099999999997	-7.40459772120303e-11\\
2.77299999999999	-7.27909617577121e-11\\
2.77300000000002	-7.27909617576944e-11\\
2.77699999999997	-7.03407365051236e-11\\
2.77999999999997	-6.85542142579713e-11\\
2.78	-6.85542142579546e-11\\
2.78399999999995	-6.6239386957971e-11\\
2.784	-6.62393869579473e-11\\
2.786	-6.51104351813983e-11\\
2.78600000000003	-6.51104351813824e-11\\
2.78800000000004	-6.40002615434013e-11\\
2.79000000000004	-6.29087217656994e-11\\
2.792	-6.18356739916879e-11\\
2.79200000000003	-6.18356739916728e-11\\
2.79600000000004	-5.97444990289505e-11\\
2.79999999999997	-5.77256495448526e-11\\
2.8	-5.77256495448385e-11\\
2.80400000000001	-5.57780760280262e-11\\
2.80599999999997	-5.48306999311095e-11\\
2.806	-5.48306999310962e-11\\
2.81000000000001	-5.29881534412131e-11\\
2.81199999999997	-5.20927435907496e-11\\
2.812	-5.2092743590737e-11\\
2.81299999999997	-5.16514638628747e-11\\
2.813	-5.16514638628622e-11\\
2.81399999999998	-5.12144626054046e-11\\
2.81499999999997	-5.07817256201448e-11\\
2.81699999999995	-4.99289883663481e-11\\
2.81899999999997	-4.90931412796375e-11\\
2.819	-4.90931412796258e-11\\
2.81999999999997	-4.86815175178523e-11\\
2.82	-4.86815175178406e-11\\
2.82099999999999	-4.82740757338359e-11\\
2.82199999999998	-4.78708026897769e-11\\
2.82399999999995	-4.70767105476504e-11\\
2.82599999999997	-4.6299137891152e-11\\
2.826	-4.62991378911411e-11\\
2.82999999999995	-4.47931489572452e-11\\
2.83399999999991	-4.33520529902527e-11\\
2.83499999999997	-4.30018299946815e-11\\
2.835	-4.30018299946716e-11\\
2.83999999999997	-4.13104536808421e-11\\
2.84	-4.13104536808328e-11\\
2.84199999999997	-4.06615766448527e-11\\
2.842	-4.06615766448436e-11\\
2.84399999999996	-4.0028385385788e-11\\
2.84599999999992	-3.94107976136267e-11\\
2.846	-3.94107976136033e-11\\
2.84600000000003	-3.94107976135946e-11\\
2.84999999999996	-3.82221135010045e-11\\
2.852	-3.76508626792191e-11\\
2.85200000000003	-3.76508626792111e-11\\
2.853	-3.73709772697155e-11\\
2.85300000000003	-3.73709772697075e-11\\
2.85400000000002	-3.7094097908257e-11\\
2.85500000000001	-3.68194071452676e-11\\
2.85699999999998	-3.627655578702e-11\\
2.85999999999997	-3.54784724181292e-11\\
2.86	-3.54784724181217e-11\\
2.86399999999995	-3.44442278849423e-11\\
2.86599999999997	-3.39397540778684e-11\\
2.866	-3.39397540778613e-11\\
2.86999999999995	-3.29557755682758e-11\\
2.87	-3.29557755682641e-11\\
2.87099999999997	-3.27149361175745e-11\\
2.871	-3.27149361175677e-11\\
2.87199999999998	-3.24761429878326e-11\\
2.87299999999997	-3.22393884206441e-11\\
2.87499999999995	-3.17719642715613e-11\\
2.87699999999997	-3.13126029238222e-11\\
2.87699999999999	-3.13126029238158e-11\\
2.87999999999997	-3.06385483492713e-11\\
2.88	-3.0638548349265e-11\\
2.88299999999998	-2.99823038975676e-11\\
2.88599999999996	-2.93436776739063e-11\\
2.886	-2.93436776738974e-11\\
2.888	-2.89276223326308e-11\\
2.88800000000003	-2.89276223326249e-11\\
2.89000000000003	-2.85183345761542e-11\\
2.89200000000003	-2.81148356451979e-11\\
2.89600000000002	-2.73249952507561e-11\\
2.89999999999997	-2.65576905422465e-11\\
2.9	-2.65576905422411e-11\\
2.90499999999997	-2.56296438897278e-11\\
2.905	-2.56296438897227e-11\\
2.90599999999997	-2.54481184849383e-11\\
2.90599999999999	-2.54481184849332e-11\\
2.90699999999998	-2.52679405245781e-11\\
2.90799999999997	-2.50891041545584e-11\\
2.90999999999995	-2.47354329875725e-11\\
2.91199999999997	-2.43870590208296e-11\\
2.91199999999999	-2.43870590208247e-11\\
2.91599999999995	-2.37060222727038e-11\\
2.91799999999997	-2.33732709837042e-11\\
2.91799999999999	-2.33732709836995e-11\\
2.91999999999997	-2.30459910151176e-11\\
2.92	-2.3045991015113e-11\\
2.92199999999998	-2.27244909801635e-11\\
2.92399999999996	-2.24087290966658e-11\\
2.926	-2.20986643281659e-11\\
2.92600000000003	-2.20986643281615e-11\\
2.92899999999997	-2.16441613200959e-11\\
2.92899999999999	-2.16441613200916e-11\\
2.93199999999993	-2.1202253424606e-11\\
2.93499999999987	-2.07728114218158e-11\\
2.93499999999997	-2.07728114218019e-11\\
2.935	-2.07728114217979e-11\\
2.93999999999997	-2.00844373017157e-11\\
2.94	-2.00844373017118e-11\\
2.94499999999998	-1.94297841651129e-11\\
2.94599999999997	-1.93028527826365e-11\\
2.946	-1.93028527826329e-11\\
2.95099999999998	-1.86879652675215e-11\\
2.95199999999997	-1.85689136753922e-11\\
2.952	-1.85689136753888e-11\\
2.95699999999998	-1.79881149287777e-11\\
2.95799999999999	-1.78746223133473e-11\\
2.95800000000002	-1.78746223133441e-11\\
2.95999999999997	-1.76502745448238e-11\\
2.96	-1.76502745448207e-11\\
2.96199999999995	-1.74294191436585e-11\\
2.9639999999999	-1.72120274073223e-11\\
2.96599999999997	-1.69980710835486e-11\\
2.966	-1.69980710835456e-11\\
2.9699999999999	-1.65803538940267e-11\\
2.96999999999999	-1.65803538940174e-11\\
2.97000000000002	-1.65803538940145e-11\\
2.97399999999992	-1.61760504221711e-11\\
2.97499999999997	-1.60770456654789e-11\\
2.975	-1.60770456654761e-11\\
2.9789999999999	-1.56892145823358e-11\\
2.97999999999997	-1.55942880467743e-11\\
2.98	-1.55942880467716e-11\\
2.9839999999999	-1.52226143190927e-11\\
2.986	-1.50415547301976e-11\\
2.98600000000003	-1.5041554730195e-11\\
2.98699999999997	-1.49522088922689e-11\\
2.98699999999999	-1.49522088922664e-11\\
2.98799999999998	-1.48634373462487e-11\\
2.98899999999997	-1.47750260744385e-11\\
2.99099999999995	-1.4599272875234e-11\\
2.99299999999999	-1.44249264537444e-11\\
2.99300000000002	-1.4424926453742e-11\\
2.99699999999997	-1.40803634919614e-11\\
2.99999999999997	-1.38254668436162e-11\\
3	-1.38254668436138e-11\\
3.00399999999995	-1.34901636986179e-11\\
3.00599999999997	-1.3324416935558e-11\\
3.006	-1.33244169355556e-11\\
3.00999999999995	-1.29966253236314e-11\\
3.01	-1.29966253236275e-11\\
3.01399999999995	-1.26736278594765e-11\\
3.01599999999997	-1.25138743652434e-11\\
3.01599999999999	-1.25138743652412e-11\\
3.01999999999995	-1.21980109016831e-11\\
3.02	-1.21980109016787e-11\\
3.02399999999995	-1.18870401400841e-11\\
3.02599999999997	-1.17333389066601e-11\\
3.026	-1.17333389066579e-11\\
3.02799999999997	-1.15808004203072e-11\\
3.02799999999999	-1.1580800420305e-11\\
3.02999999999996	-1.14294048571943e-11\\
3.03199999999992	-1.12791325419413e-11\\
3.03599999999985	-1.09818796810031e-11\\
3.03999999999997	-1.06888873073914e-11\\
3.04	-1.06888873073893e-11\\
3.04499999999997	-1.03284063171219e-11\\
3.04499999999999	-1.03284063171199e-11\\
3.04599999999997	-1.0257054563797e-11\\
3.046	-1.0257054563795e-11\\
3.04699999999998	-1.01859455309269e-11\\
3.04799999999996	-1.01150769081369e-11\\
3.04999999999992	-9.97405169052941e-12\\
3.05199999999997	-9.83396058331543e-12\\
3.052	-9.83396058331345e-12\\
3.05599999999992	-9.55650799447882e-12\\
3.05799999999997	-9.41911045507976e-12\\
3.058	-9.41911045507781e-12\\
3.05999999999997	-9.2830981891478e-12\\
3.06	-9.28309818914588e-12\\
3.06199999999997	-9.14897680367051e-12\\
3.06399999999995	-9.01672886822019e-12\\
3.066	-8.88633719583541e-12\\
3.06600000000003	-8.88633719583357e-12\\
3.06999999999997	-8.63105509680975e-12\\
3.07399999999992	-8.38299779659073e-12\\
3.07399999999996	-8.38299779658842e-12\\
3.07399999999999	-8.38299779658611e-12\\
3.07999999999997	-8.02417726687836e-12\\
3.07999999999999	-8.02417726687669e-12\\
3.08599999999997	-7.68090352620886e-12\\
3.08599999999999	-7.68090352620727e-12\\
3.09199999999997	-7.3527750331781e-12\\
3.09199999999999	-7.35277503317659e-12\\
3.09799999999997	-7.03902334688692e-12\\
3.09999999999997	-6.93748493038314e-12\\
3.1	-6.9374849303817e-12\\
3.10299999999997	-6.78798682689855e-12\\
3.10299999999999	-6.78798682689715e-12\\
3.10599999999996	-6.64182136391659e-12\\
3.106	-6.64182136391446e-12\\
3.10600000000003	-6.64182136391309e-12\\
3.10899999999999	-6.49894580044732e-12\\
3.11199999999996	-6.35931835774908e-12\\
3.11199999999999	-6.35931835774738e-12\\
3.11200000000003	-6.35931835774571e-12\\
3.11499999999997	-6.22289820665782e-12\\
3.115	-6.22289820665654e-12\\
3.11799999999995	-6.08964545649625e-12\\
3.11999999999997	-6.00255067708292e-12\\
3.12	-6.00255067708169e-12\\
3.12299999999995	-5.87449079164882e-12\\
3.12599999999989	-5.74949642789299e-12\\
3.12599999999995	-5.7494964278907e-12\\
3.126	-5.7494964278884e-12\\
3.127	-5.7085065240266e-12\\
3.12700000000003	-5.70850652402544e-12\\
3.12800000000003	-5.66779026053319e-12\\
3.12900000000003	-5.62728474140518e-12\\
3.13100000000003	-5.54690067899946e-12\\
3.13199999999997	-5.50701952405509e-12\\
3.13199999999999	-5.50701952405396e-12\\
3.13599999999999	-5.34953725670882e-12\\
3.13799999999997	-5.27200364953797e-12\\
3.13799999999999	-5.27200364953687e-12\\
3.13999999999997	-5.19526159031841e-12\\
3.14	-5.19526159031732e-12\\
3.14199999999998	-5.11930110569022e-12\\
3.14399999999996	-5.04411232382214e-12\\
3.14599999999997	-4.96968547317081e-12\\
3.146	-4.96968547316976e-12\\
3.14999999999996	-4.82307897339808e-12\\
3.15	-4.8230789733966e-12\\
3.15399999999996	-4.67940539206325e-12\\
3.15599999999997	-4.60864504680089e-12\\
3.156	-4.60864504679989e-12\\
3.15999999999996	-4.46939663194856e-12\\
3.16	-4.46939663194717e-12\\
3.16099999999997	-4.43506853709184e-12\\
3.16099999999999	-4.43506853709086e-12\\
3.16199999999996	-4.40093180577717e-12\\
3.16299999999992	-4.36698532889948e-12\\
3.16499999999985	-4.29965873295585e-12\\
3.16599999999997	-4.26627642645114e-12\\
3.166	-4.26627642645019e-12\\
3.16999999999986	-4.13459524032445e-12\\
3.17199999999997	-4.06984851798964e-12\\
3.172	-4.06984851798873e-12\\
3.17599999999986	-3.94250074142325e-12\\
3.17999999999972	-3.81795884126759e-12\\
3.17999999999997	-3.81795884125979e-12\\
3.18	-3.81795884125891e-12\\
3.18499999999997	-3.66612873223205e-12\\
3.185	-3.6661287322312e-12\\
3.18599999999997	-3.6362657721565e-12\\
3.186	-3.63626577215566e-12\\
3.18699999999997	-3.60656822278426e-12\\
3.18799999999994	-3.57703511922393e-12\\
3.18999999999989	-3.51845841676283e-12\\
3.18999999999994	-3.51845841676127e-12\\
3.18999999999999	-3.5184584167597e-12\\
3.19399999999988	-3.4032364967797e-12\\
3.19599999999999	-3.34657630502077e-12\\
3.19600000000002	-3.34657630501997e-12\\
3.198	-3.29054011335281e-12\\
3.19800000000003	-3.29054011335202e-12\\
3.19999999999998	-3.23520143083435e-12\\
3.20000000000001	-3.23520143083356e-12\\
3.20199999999996	-3.18063385714113e-12\\
3.20399999999991	-3.12683030068256e-12\\
3.20599999999998	-3.07378376913948e-12\\
3.20600000000001	-3.07378376913873e-12\\
3.20999999999991	-2.96993430271128e-12\\
3.21399999999982	-2.86903147105809e-12\\
3.21899999999997	-2.74696686806502e-12\\
3.21899999999999	-2.74696686806434e-12\\
3.21999999999997	-2.72308783021175e-12\\
3.22	-2.72308783021107e-12\\
3.22099999999998	-2.69938492970696e-12\\
3.22199999999996	-2.67585739633242e-12\\
3.22399999999992	-2.629325379036e-12\\
3.22599999999997	-2.58348573101433e-12\\
3.226	-2.58348573101368e-12\\
3.22999999999992	-2.49385980283126e-12\\
3.23099999999999	-2.47187685111626e-12\\
3.23100000000002	-2.47187685111564e-12\\
3.23499999999994	-2.38561773846254e-12\\
3.23699999999999	-2.34348200866763e-12\\
3.23700000000002	-2.34348200866704e-12\\
3.23999999999997	-2.28182552764559e-12\\
3.24	-2.28182552764501e-12\\
3.24299999999996	-2.22226580885877e-12\\
3.24599999999991	-2.16478543623504e-12\\
3.24599999999997	-2.16478543623379e-12\\
3.246	-2.16478543623325e-12\\
3.24799999999997	-2.12761205491333e-12\\
3.24799999999999	-2.12761205491281e-12\\
3.24999999999996	-2.09135051382925e-12\\
3.25199999999992	-2.05599610035871e-12\\
3.25399999999997	-2.0215442198367e-12\\
3.25399999999999	-2.02154421983622e-12\\
3.25499999999998	-2.00465532569367e-12\\
3.255	-2.00465532569319e-12\\
3.25599999999998	-1.98799039491937e-12\\
3.25699999999997	-1.97154888606879e-12\\
3.25899999999993	-1.93933400466095e-12\\
3.25999999999997	-1.92355958544463e-12\\
3.26	-1.92355958544419e-12\\
3.26399999999993	-1.86267017717907e-12\\
3.26599999999997	-1.8335435352374e-12\\
3.266	-1.833543535237e-12\\
3.267	-1.81930805505565e-12\\
3.26700000000003	-1.81930805505525e-12\\
3.26800000000003	-1.80523349032254e-12\\
3.26900000000003	-1.79126235771018e-12\\
3.27100000000003	-1.76362857652125e-12\\
3.27500000000003	-1.70958245055422e-12\\
3.27699999999997	-1.68316308177643e-12\\
3.27699999999999	-1.68316308177606e-12\\
3.27999999999997	-1.64427897156743e-12\\
3.28	-1.64427897156707e-12\\
3.28299999999998	-1.60627870557744e-12\\
3.28599999999996	-1.56915117176388e-12\\
3.286	-1.5691511717634e-12\\
3.28899999999999	-1.53288551347812e-12\\
3.28900000000002	-1.53288551347778e-12\\
3.29	-1.52098664855219e-12\\
3.29000000000003	-1.52098664855186e-12\\
3.29100000000001	-1.50918198065693e-12\\
3.29199999999999	-1.49747112626117e-12\\
3.29399999999996	-1.47432933907474e-12\\
3.296	-1.45155827948278e-12\\
3.29600000000003	-1.45155827948246e-12\\
3.29999999999996	-1.40694400199491e-12\\
3.3	-1.40694400199443e-12\\
3.30000000000003	-1.40694400199412e-12\\
3.30399999999996	-1.36343254915404e-12\\
3.30599999999997	-1.34208329124401e-12\\
3.30599999999999	-1.3420832912437e-12\\
3.30999999999992	-1.30018383938286e-12\\
3.31199999999999	-1.27962820017344e-12\\
3.31200000000002	-1.27962820017315e-12\\
3.31599999999995	-1.23929173784323e-12\\
3.31999999999988	-1.19997094510532e-12\\
3.32	-1.1999709451041e-12\\
3.32000000000003	-1.19997094510382e-12\\
3.32499999999998	-1.15221714799285e-12\\
3.325	-1.15221714799258e-12\\
3.326	-1.14284956306104e-12\\
3.32600000000003	-1.14284956306077e-12\\
3.32700000000003	-1.13354232153346e-12\\
3.32800000000003	-1.12429512101216e-12\\
3.33000000000002	-1.10597964317388e-12\\
3.332	-1.08790074926491e-12\\
3.33200000000003	-1.08790074926465e-12\\
3.33499999999999	-1.0612377280357e-12\\
3.33500000000002	-1.06123772803545e-12\\
3.33799999999998	-1.03512765675746e-12\\
3.33999999999997	-1.01802436006121e-12\\
3.34	-1.01802436006097e-12\\
3.34299999999997	-9.92818984336899e-13\\
3.34599999999993	-9.68146551995513e-13\\
3.34599999999997	-9.68146551995209e-13\\
3.346	-9.6814655199491e-13\\
3.34699999999999	-9.60039588320935e-13\\
3.34700000000002	-9.60039588320705e-13\\
3.34800000000001	-9.51990774318063e-13\\
3.349	-9.43999848468442e-13\\
3.35099999999999	-9.28190624609852e-13\\
3.35499999999995	-8.97255536309801e-13\\
3.35999999999997	-8.59847358065254e-13\\
3.36	-8.59847358065046e-13\\
3.36399999999997	-8.30910400487484e-13\\
3.36399999999999	-8.30910400487282e-13\\
3.36599999999997	-8.16766583220285e-13\\
3.366	-8.16766583220086e-13\\
3.36799999999998	-8.02836746740302e-13\\
3.36999999999996	-7.89119080724671e-13\\
3.37199999999997	-7.75611802423819e-13\\
3.372	-7.75611802423628e-13\\
3.37599999999996	-7.49221414487827e-13\\
3.37799999999997	-7.3633487515492e-13\\
3.378	-7.36334875154738e-13\\
3.37999999999997	-7.2369583401453e-13\\
3.38	-7.23695834014353e-13\\
3.38199999999997	-7.11346618801662e-13\\
3.38399999999995	-6.99285624611014e-13\\
3.38599999999997	-6.87511283994095e-13\\
3.386	-6.8751128399393e-13\\
3.38999999999995	-6.64816479791137e-13\\
3.39299999999997	-6.48536730472498e-13\\
3.39299999999999	-6.48536730472346e-13\\
3.39499999999998	-6.3803392807861e-13\\
3.395	-6.38033928078463e-13\\
3.39699999999999	-6.27809819006834e-13\\
3.39899999999997	-6.17863074530837e-13\\
3.399	-6.17863074530698e-13\\
3.39999999999997	-6.1299330864911e-13\\
3.4	-6.12993308648973e-13\\
3.40099999999998	-6.08192401971333e-13\\
3.40199999999996	-6.03460198515628e-13\\
3.40399999999991	-5.94201288505919e-13\\
3.40599999999997	-5.85215375330503e-13\\
3.406	-5.85215375330377e-13\\
3.40999999999991	-5.68057903595257e-13\\
3.41099999999997	-5.63937376725839e-13\\
3.411	-5.63937376725723e-13\\
3.41499999999991	-5.47870874309481e-13\\
3.41899999999982	-5.32360823860031e-13\\
3.41999999999997	-5.28569305670017e-13\\
3.42	-5.2856930566991e-13\\
3.42199999999997	-5.21088596404067e-13\\
3.42199999999999	-5.21088596403962e-13\\
3.42399999999996	-5.13743515041848e-13\\
3.42599999999992	-5.06533106966768e-13\\
3.426	-5.06533106966469e-13\\
3.42600000000003	-5.06533106966368e-13\\
3.42999999999996	-4.92512579804714e-13\\
3.43	-4.92512579804546e-13\\
3.432	-4.85700638610793e-13\\
3.43200000000003	-4.85700638610697e-13\\
3.43400000000003	-4.79019726259208e-13\\
3.43600000000003	-4.72468974497852e-13\\
3.438	-4.66047531990366e-13\\
3.43800000000003	-4.66047531990276e-13\\
3.43999999999997	-4.59727621110515e-13\\
3.44	-4.59727621110425e-13\\
3.44199999999995	-4.53481477426894e-13\\
3.44399999999989	-4.47308289190111e-13\\
3.44599999999997	-4.41207254131539e-13\\
3.446	-4.41207254131453e-13\\
3.44999999999989	-4.29218481276533e-13\\
3.45099999999996	-4.2626515611457e-13\\
3.45099999999999	-4.26265156114486e-13\\
3.45499999999988	-4.14624449023134e-13\\
3.45699999999996	-4.08906329842964e-13\\
3.45699999999999	-4.08906329842883e-13\\
3.45999999999997	-4.00454850867052e-13\\
3.46	-4.00454850866973e-13\\
3.46299999999998	-3.92152018166391e-13\\
3.46499999999998	-3.86698176918659e-13\\
3.465	-3.86698176918582e-13\\
3.46599999999997	-3.83995403879164e-13\\
3.466	-3.83995403879087e-13\\
3.46699999999997	-3.81308611972707e-13\\
3.46799999999994	-3.78637713905992e-13\\
3.46999999999988	-3.73343252698551e-13\\
3.47199999999997	-3.6811133218769e-13\\
3.472	-3.68111332187616e-13\\
3.47599999999988	-3.57815189995811e-13\\
3.47999999999976	-3.47726723264663e-13\\
3.47999999999996	-3.47726723264151e-13\\
3.47999999999999	-3.4772672326408e-13\\
3.48599999999996	-3.32971971029132e-13\\
3.48599999999999	-3.32971971029064e-13\\
3.49199999999996	-3.18655453806414e-13\\
3.49199999999999	-3.18655453806346e-13\\
3.49799999999996	-3.04760424978466e-13\\
3.5	-3.00219589688184e-13\\
3.50000000000003	-3.0021958968812e-13\\
3.506	-2.8686134452885e-13\\
3.50600000000003	-2.86861344528788e-13\\
3.50899999999999	-2.80327292360738e-13\\
3.50900000000002	-2.80327292360677e-13\\
3.51199999999998	-2.7388739692581e-13\\
3.51499999999995	-2.67539775101815e-13\\
3.51499999999998	-2.67539775101735e-13\\
3.51500000000002	-2.67539775101657e-13\\
3.518	-2.61282570748797e-13\\
3.51800000000003	-2.61282570748739e-13\\
3.51999999999997	-2.57171025804487e-13\\
3.52	-2.57171025804429e-13\\
3.52199999999995	-2.53119512957877e-13\\
3.52399999999989	-2.49127505675007e-13\\
3.52599999999997	-2.45194485153904e-13\\
3.526	-2.45194485153848e-13\\
3.52999999999989	-2.37503367465211e-13\\
3.53399999999978	-2.30042161615837e-13\\
3.53499999999998	-2.28212326730322e-13\\
3.535	-2.28212326730271e-13\\
3.53799999999997	-2.22806988808608e-13\\
3.53799999999999	-2.22806988808557e-13\\
3.53999999999997	-2.19272984907912e-13\\
3.54	-2.19272984907863e-13\\
3.54199999999998	-2.15794087882797e-13\\
3.54399999999996	-2.1236984561636e-13\\
3.54599999999997	-2.08999813093652e-13\\
3.546	-2.08999813093604e-13\\
3.54999999999996	-2.02420632397654e-13\\
3.54999999999999	-2.02420632397606e-13\\
3.553	-1.97625340738993e-13\\
3.55300000000003	-1.97625340738948e-13\\
3.55600000000004	-1.92932456040873e-13\\
3.55900000000005	-1.88325350915856e-13\\
3.55999999999997	-1.86808477771384e-13\\
3.56	-1.86808477771342e-13\\
3.56599999999999	-1.77901504872211e-13\\
3.56600000000002	-1.77901504872169e-13\\
3.56699999999996	-1.76448936895814e-13\\
3.56699999999999	-1.76448936895772e-13\\
3.56799999999996	-1.75005364234871e-13\\
3.56899999999992	-1.73570739985586e-13\\
3.56999999999998	-1.7214501753727e-13\\
3.57	-1.7214501753723e-13\\
3.57199999999993	-1.69320093046872e-13\\
3.57299999999996	-1.67920799223645e-13\\
3.57299999999999	-1.67920799223606e-13\\
3.57499999999992	-1.65148321106948e-13\\
3.57699999999985	-1.62410355908058e-13\\
3.57899999999996	-1.59706547800477e-13\\
3.57899999999999	-1.59706547800439e-13\\
3.57999999999997	-1.58368205184141e-13\\
3.58	-1.58368205184103e-13\\
3.58099999999998	-1.57039995460471e-13\\
3.58199999999997	-1.55721875475837e-13\\
3.58399999999993	-1.53115733748754e-13\\
3.58599999999997	-1.50549441308324e-13\\
3.586	-1.50549441308288e-13\\
3.58999999999993	-1.45535075367669e-13\\
3.59399999999986	-1.40676170904232e-13\\
3.59599999999996	-1.38304224236145e-13\\
3.59599999999999	-1.38304224236112e-13\\
3.59999999999997	-1.33673800740663e-13\\
3.6	-1.33673800740631e-13\\
3.60399999999998	-1.29192672589379e-13\\
3.60499999999998	-1.28095443263953e-13\\
3.605	-1.28095443263922e-13\\
3.60599999999997	-1.27007363278057e-13\\
3.606	-1.27007363278026e-13\\
3.60699999999997	-1.25928397280757e-13\\
3.60799999999994	-1.24858510216369e-13\\
3.60799999999999	-1.24858510216314e-13\\
3.60999999999993	-1.22745834138834e-13\\
3.61199999999987	-1.20669060482555e-13\\
3.61399999999996	-1.18627919349615e-13\\
3.61399999999999	-1.18627919349586e-13\\
3.61799999999987	-1.14647415442449e-13\\
3.61999999999997	-1.12706519707675e-13\\
3.62	-1.12706519707648e-13\\
3.62399999999988	-1.08922179453689e-13\\
3.62499999999996	-1.07996217037526e-13\\
3.62499999999999	-1.079962170375e-13\\
3.62599999999997	-1.07078243120731e-13\\
3.626	-1.07078243120705e-13\\
3.62699999999998	-1.06168227878919e-13\\
3.62799999999997	-1.05266141745627e-13\\
3.62999999999993	-1.03485639827199e-13\\
3.63199999999997	-1.01736505971147e-13\\
3.632	-1.01736505971123e-13\\
3.63599999999993	-9.83314372277627e-14\\
3.63999999999985	-9.50491653690815e-14\\
3.63999999999997	-9.50491653689831e-14\\
3.64	-9.50491653689602e-14\\
3.64599999999997	-9.03522864183967e-14\\
3.646	-9.03522864183751e-14\\
3.65199999999997	-8.59223818978032e-14\\
3.652	-8.59223818977828e-14\\
3.65399999999996	-8.45031802009861e-14\\
3.65399999999999	-8.45031802009661e-14\\
3.65599999999995	-8.31108996138365e-14\\
3.65799999999992	-8.17453591928472e-14\\
3.65999999999996	-8.04063814721768e-14\\
3.65999999999999	-8.04063814721579e-14\\
3.66399999999992	-7.78074215103782e-14\\
3.66599999999996	-7.65471015080389e-14\\
3.66599999999999	-7.65471015080212e-14\\
3.66999999999992	-7.41039624861691e-14\\
3.67399999999984	-7.17631052926432e-14\\
3.67499999999998	-7.11937287612782e-14\\
3.67500000000001	-7.11937287612621e-14\\
3.67999999999997	-6.8440950313698e-14\\
3.68	-6.84409503136827e-14\\
3.68299999999996	-6.68639263156114e-14\\
3.68299999999999	-6.68639263155967e-14\\
3.68599999999995	-6.53423176719222e-14\\
3.686	-6.53423176718979e-14\\
3.68899999999996	-6.3875679442839e-14\\
3.69199999999993	-6.246358276354e-14\\
3.69199999999997	-6.24635827635174e-14\\
3.692	-6.24635827635043e-14\\
3.69299999999998	-6.20049323886349e-14\\
3.69300000000001	-6.2004932388622e-14\\
3.69399999999998	-6.15509802566563e-14\\
3.69499999999995	-6.11004105990522e-14\\
3.6969999999999	-6.02093602608786e-14\\
3.69999999999997	-5.88977908857261e-14\\
3.7	-5.88977908857139e-14\\
3.7039999999999	-5.71951155056758e-14\\
3.70599999999997	-5.63632785904321e-14\\
3.706	-5.63632785904204e-14\\
3.7099999999999	-5.47380657698098e-14\\
3.70999999999998	-5.47380657697778e-14\\
3.71000000000001	-5.47380657697665e-14\\
3.71199999999996	-5.3944478651514e-14\\
3.71199999999999	-5.39444786515028e-14\\
3.71399999999995	-5.31634330013443e-14\\
3.71599999999991	-5.23948273145019e-14\\
3.71799999999996	-5.16385617028711e-14\\
3.71799999999999	-5.16385617028605e-14\\
3.71999999999997	-5.08945378826005e-14\\
3.72	-5.089453788259e-14\\
3.72199999999998	-5.0162659160695e-14\\
3.72399999999997	-4.94428304221257e-14\\
3.72599999999997	-4.87349581178618e-14\\
3.726	-4.87349581178518e-14\\
3.728	-4.80389502531701e-14\\
3.72800000000003	-4.80389502531603e-14\\
3.73000000000004	-4.73534773447235e-14\\
3.73200000000004	-4.66772112779088e-14\\
3.73600000000004	-4.53519493046146e-14\\
3.74	-4.40624753815468e-14\\
3.74000000000003	-4.40624753815378e-14\\
3.74099999999996	-4.37456198252768e-14\\
3.74099999999999	-4.37456198252679e-14\\
3.74199999999996	-4.34309487441167e-14\\
3.74299999999992	-4.31184519129454e-14\\
3.74499999999985	-4.24999404591548e-14\\
3.745	-4.24999404591078e-14\\
3.74500000000003	-4.2499940459099e-14\\
3.746	-4.21939057411107e-14\\
3.74600000000003	-4.2193905741102e-14\\
3.747	-4.18900050820773e-14\\
3.74799999999997	-4.15882286082537e-14\\
3.74999999999991	-4.09910090664114e-14\\
3.752	-4.04021695008206e-14\\
3.75200000000003	-4.04021695008123e-14\\
3.75599999999991	-3.92493252766915e-14\\
3.758	-3.86851707943161e-14\\
3.75800000000003	-3.86851707943082e-14\\
3.75999999999997	-3.81299339324799e-14\\
3.76	-3.81299339324721e-14\\
3.76199999999995	-3.75843798431627e-14\\
3.76399999999989	-3.7048437626102e-14\\
3.76599999999997	-3.65220376301517e-14\\
3.766	-3.65220376301443e-14\\
3.76999999999989	-3.54975918903857e-14\\
3.76999999999996	-3.54975918903673e-14\\
3.76999999999999	-3.54975918903602e-14\\
3.77399999999988	-3.45105100592673e-14\\
3.77599999999999	-3.40308195014905e-14\\
3.77600000000002	-3.40308195014838e-14\\
3.77999999999991	-3.30988273916411e-14\\
3.77999999999996	-3.30988273916304e-14\\
3.78	-3.30988273916196e-14\\
3.78399999999989	-3.22029521780392e-14\\
3.78599999999997	-3.17684121410113e-14\\
3.786	-3.17684121410051e-14\\
3.78999999999989	-3.09258448360814e-14\\
3.79199999999997	-3.05177080679881e-14\\
3.792	-3.05177080679824e-14\\
3.79599999999989	-2.9722652282824e-14\\
3.79899999999996	-2.91426182927212e-14\\
3.79899999999999	-2.91426182927158e-14\\
3.79999999999997	-2.8952336659388e-14\\
3.8	-2.89523366593826e-14\\
3.80099999999999	-2.87635761961415e-14\\
3.80199999999997	-2.85763307701404e-14\\
3.80399999999993	-2.82063607447016e-14\\
3.80599999999997	-2.78423785017122e-14\\
3.806	-2.7842378501707e-14\\
3.80999999999993	-2.71321889234608e-14\\
3.81099999999996	-2.6958310915273e-14\\
3.81099999999999	-2.69583109152681e-14\\
3.81499999999992	-2.6277306946432e-14\\
3.81499999999998	-2.62773069464225e-14\\
3.81500000000001	-2.62773069464177e-14\\
3.81899999999993	-2.56192520556124e-14\\
3.81999999999997	-2.54582837775804e-14\\
3.82	-2.54582837775758e-14\\
3.82399999999993	-2.48284355458696e-14\\
3.826	-2.45218548941431e-14\\
3.82600000000003	-2.45218548941388e-14\\
3.82700000000001	-2.43706328847773e-14\\
3.82700000000003	-2.4370632884773e-14\\
3.82799999999998	-2.42204318690089e-14\\
3.82800000000001	-2.42204318690047e-14\\
3.82899999999997	-2.40708956484329e-14\\
3.82999999999993	-2.39220193646707e-14\\
3.83199999999986	-2.36262272810829e-14\\
3.83399999999998	-2.33330171770454e-14\\
3.83400000000001	-2.33330171770413e-14\\
3.83799999999985	-2.2754190816675e-14\\
3.84	-2.24684993359545e-14\\
3.84000000000003	-2.24684993359504e-14\\
3.84399999999988	-2.19043741270197e-14\\
3.846	-2.16258670849825e-14\\
3.84600000000003	-2.16258670849785e-14\\
3.84999999999988	-2.10757831478834e-14\\
3.84999999999998	-2.10757831478698e-14\\
3.85000000000001	-2.10757831478659e-14\\
3.85399999999985	-2.05347016009909e-14\\
3.85599999999998	-2.02674486435633e-14\\
3.85600000000001	-2.02674486435595e-14\\
3.85699999999996	-2.01346482484026e-14\\
3.85699999999999	-2.01346482483988e-14\\
3.85799999999995	-2.00024186450615e-14\\
3.85899999999992	-1.98707555373921e-14\\
3.85999999999997	-1.97396546476822e-14\\
3.86	-1.97396546476785e-14\\
3.86199999999993	-1.94791225026325e-14\\
3.86399999999985	-1.92207883511973e-14\\
3.86599999999997	-1.89646186202171e-14\\
3.866	-1.89646186202135e-14\\
3.86899999999996	-1.8584349562497e-14\\
3.86899999999999	-1.85843495624934e-14\\
3.87199999999995	-1.82087644136719e-14\\
3.87499999999991	-1.78377533475368e-14\\
3.87999999999997	-1.72292721302248e-14\\
3.88	-1.72292721302214e-14\\
3.88499999999998	-1.66327016119151e-14\\
3.88500000000001	-1.66327016119118e-14\\
3.88599999999999	-1.65147739821933e-14\\
3.88600000000002	-1.65147739821899e-14\\
3.88700000000001	-1.63972995357626e-14\\
3.88799999999999	-1.62802744558082e-14\\
3.88999999999996	-1.60475572013611e-14\\
3.89199999999999	-1.58165919748862e-14\\
3.89200000000002	-1.58165919748829e-14\\
3.89599999999996	-1.53597977652232e-14\\
3.89799999999999	-1.51339094170052e-14\\
3.89800000000002	-1.5133909417002e-14\\
3.89999999999997	-1.49104772550524e-14\\
3.9	-1.49104772550492e-14\\
3.90199999999996	-1.4690295137929e-14\\
3.90399999999991	-1.44733344507466e-14\\
3.90599999999997	-1.42595669972591e-14\\
3.906	-1.42595669972561e-14\\
3.90999999999991	-1.3841501078411e-14\\
3.91399999999982	-1.34358800471876e-14\\
3.91499999999996	-1.33363944512966e-14\\
3.91499999999999	-1.33363944512938e-14\\
3.91999999999997	-1.28503231383985e-14\\
3.92	-1.28503231383958e-14\\
3.92499999999999	-1.23828923088844e-14\\
3.92599999999997	-1.22916093248726e-14\\
3.926	-1.229160932487e-14\\
3.92699999999996	-1.22010537592516e-14\\
3.92699999999999	-1.22010537592491e-14\\
3.92799999999995	-1.21112226699843e-14\\
3.92899999999992	-1.20221131384028e-14\\
3.93099999999984	-1.18460471910753e-14\\
3.93299999999996	-1.16728330357022e-14\\
3.93299999999999	-1.16728330356997e-14\\
3.93699999999984	-1.13346120685687e-14\\
3.93999999999997	-1.10879283752368e-14\\
3.94	-1.10879283752345e-14\\
3.94399999999985	-1.07681890150445e-14\\
3.94399999999996	-1.07681890150356e-14\\
3.94399999999999	-1.07681890150334e-14\\
3.94599999999997	-1.06122034480178e-14\\
3.946	-1.06122034480156e-14\\
3.94799999999999	-1.04587801487854e-14\\
3.94999999999997	-1.03078991784317e-14\\
3.95199999999997	-1.01595409284475e-14\\
3.952	-1.01595409284454e-14\\
3.95499999999998	-9.94169156699364e-15\\
3.95500000000001	-9.94169156699161e-15\\
3.95799999999998	-9.72941131927256e-15\\
3.95999999999998	-9.59095438608065e-15\\
3.96	-9.5909543860787e-15\\
3.96299999999998	-9.38781884914114e-15\\
3.96599999999995	-9.19009046604382e-15\\
3.966	-9.19009046604058e-15\\
3.96600000000003	-9.19009046603874e-15\\
3.96700000000001	-9.12537255609827e-15\\
3.96700000000003	-9.12537255609644e-15\\
3.96800000000001	-9.0611190426479e-15\\
3.96899999999998	-8.99719997930679e-15\\
3.97099999999993	-8.87035690690385e-15\\
3.97299999999996	-8.74482685434484e-15\\
3.97299999999999	-8.74482685434306e-15\\
3.97699999999989	-8.49764072203956e-15\\
3.97899999999996	-8.37595251794778e-15\\
3.97899999999999	-8.37595251794606e-15\\
3.97999999999997	-8.31557768563936e-15\\
3.98	-8.31557768563765e-15\\
3.98099999999999	-8.2555130810568e-15\\
3.98199999999997	-8.19575675269707e-15\\
3.98399999999994	-8.07716116873582e-15\\
3.98599999999997	-7.9597755210565e-15\\
3.986	-7.95977552105484e-15\\
3.98999999999994	-7.72857316832978e-15\\
3.98999999999997	-7.72857316832774e-15\\
3.99000000000001	-7.72857316832571e-15\\
3.99399999999994	-7.50202950251901e-15\\
3.99599999999998	-7.39046778109309e-15\\
3.99600000000001	-7.39046778109151e-15\\
3.99999999999994	-7.17085471048034e-15\\
3.99999999999997	-7.17085471047858e-15\\
4	-7.17085471047684e-15\\
4.00199999999993	-7.06281547996256e-15\\
4.00199999999999	-7.06281547995951e-15\\
4.00399999999992	-6.9559356808458e-15\\
4.00599999999986	-6.85020142302297e-15\\
4.00599999999993	-6.85020142301914e-15\\
4.006	-6.85020142301526e-15\\
4.00999999999987	-6.64211471351613e-15\\
4.01199999999995	-6.53973521889363e-15\\
4.012	-6.53973521889074e-15\\
4.01599999999987	-6.33823742222355e-15\\
4.01999999999973	-6.14100082537459e-15\\
4.01999999999995	-6.14100082536423e-15\\
4.02	-6.14100082536146e-15\\
4.02500000000001	-5.90029139229422e-15\\
4.02500000000006	-5.90029139229152e-15\\
4.02599999999995	-5.85291197702569e-15\\
4.026	-5.852911977023e-15\\
4.02699999999996	-5.80578310789862e-15\\
4.02799999999992	-5.75890325367266e-15\\
4.02999999999985	-5.66588450550949e-15\\
4.03099999999993	-5.61974258939777e-15\\
4.03099999999999	-5.61974258939515e-15\\
4.03499999999984	-5.43758975680228e-15\\
4.03699999999994	-5.34794155343909e-15\\
4.03699999999999	-5.34794155343656e-15\\
4.03799999999995	-5.3034693876551e-15\\
4.038	-5.30346938765258e-15\\
4.03899999999996	-5.25926354677536e-15\\
4.03999999999991	-5.21535622455422e-15\\
4.04	-5.21535622455025e-15\\
4.04199999999991	-5.12843143958787e-15\\
4.04399999999982	-5.04268373603311e-15\\
4.046	-4.95810197009478e-15\\
4.04600000000006	-4.95810197009239e-15\\
4.04999999999988	-4.79239243242353e-15\\
4.05399999999969	-4.63121668142419e-15\\
4.05999999999994	-4.39777125195078e-15\\
4.06	-4.39777125194861e-15\\
};
\pgfplotsset{max space between ticks=50}
\addplot [color=mycolor2,solid,forget plot]
  table[row sep=crcr]{%
0	0.10153\\
3.15544362088405e-30	0.10153\\
0.000656101980281985	0.101530709989553\\
0.00393661188169191	0.101555560666546\\
0.00599999999999994	0.10158938182706\\
0.006	0.10158938182706\\
0.012	0.101767596768679\\
0.0120000000000001	0.101767596768679\\
0.018	0.102064853328018\\
0.0180000000000001	0.102064853328018\\
0.0199999999999998	0.10213229397037\\
0.02	0.10213229397037\\
0.026	0.101716366820929\\
0.0260000000000002	0.101716366820929\\
0.0289999999999998	0.10116051629751\\
0.029	0.10116051629751\\
0.0319999999999996	0.100372547032607\\
0.0349999999999991	0.0993522286136822\\
0.035	0.0993522286136819\\
0.0399999999999996	0.0971345253230206\\
0.04	0.0971345253230203\\
0.0449999999999996	0.0942687495276435\\
0.0459999999999996	0.0936176301011948\\
0.046	0.0936176301011944\\
0.047	0.0929404751960583\\
0.0470000000000004	0.092940475196058\\
0.0490000000000003	0.0915346663918733\\
0.0510000000000002	0.0900778367053162\\
0.055	0.087010350712579\\
0.0579999999999996	0.0845743497767112\\
0.058	0.0845743497767108\\
0.0599999999999996	0.082885486841699\\
0.06	0.0828854868416986\\
0.0619999999999995	0.0811444789878217\\
0.0639999999999991	0.0793510999448183\\
0.0659999999999991	0.0775051166450086\\
0.066	0.0775051166450078\\
0.0699999999999991	0.0736543707951944\\
0.07	0.0736543707951936\\
0.0700000000000009	0.0736543707951927\\
0.074	0.0695902396041448\\
0.076	0.067477498629349\\
0.0760000000000009	0.0674774986293481\\
0.08	0.063214520109214\\
0.0800000000000009	0.0632145201092131\\
0.0839999999999999	0.0589831831846656\\
0.086	0.0568786923069218\\
0.0860000000000009	0.0568786923069209\\
0.0869999999999991	0.0558291214523434\\
0.087	0.0558291214523424\\
0.0880000000000004	0.0547812881682954\\
0.0890000000000009	0.053735158410586\\
0.0910000000000017	0.0516478735747052\\
0.0929999999999991	0.0495669956809782\\
0.093	0.0495669956809773\\
0.0970000000000017	0.0454233797984614\\
0.0999999999999991	0.0423304804899824\\
0.1	0.0423304804899815\\
0.104000000000002	0.0382246453687673\\
0.104999999999999	0.037201183877005\\
0.105	0.037201183877004\\
0.105999999999999	0.0361788547394758\\
0.106	0.0361788547394749\\
0.106999999999999	0.0351576247414335\\
0.107999999999998	0.0341374607030283\\
0.109999999999997	0.032100197959437\\
0.111999999999999	0.0300668017458478\\
0.112	0.0300668017458469\\
0.115999999999997	0.0260810521578851\\
0.115999999999998	0.0260810521578834\\
0.116	0.0260810521578817\\
0.119999999999997	0.0222486397395016\\
0.119999999999998	0.0222486397395\\
0.12	0.0222486397394984\\
0.123999999999997	0.0185675721876527\\
0.125999999999999	0.0167831916823141\\
0.126	0.0167831916823133\\
0.127999999999998	0.0150359358757995\\
0.128	0.015035935875798\\
0.129999999999998	0.0133255776955832\\
0.131999999999996	0.0116518948633202\\
0.135999999999993	0.00841368993471907\\
0.139999999999998	0.00531963767383442\\
0.14	0.00531963767383307\\
0.144999999999998	0.00165234374314154\\
0.145	0.00165234374314027\\
0.145999999999998	0.000945347349630184\\
0.146	0.000945347349628936\\
0.146999999999999	0.000247117489500776\\
0.147999999999998	-0.000442368523203164\\
0.149999999999997	-0.00179519832772279\\
0.151999999999998	-0.00311331787782769\\
0.152	-0.00311331787782884\\
0.155999999999997	-0.00564610693435692\\
0.157999999999998	-0.00686110560234216\\
0.158	-0.00686110560234323\\
0.16	-0.00803453236788899\\
0.160000000000002	-0.00803453236789001\\
0.162000000000002	-0.00915901971699857\\
0.164000000000002	-0.0102347137882067\\
0.166	-0.01126175437883\\
0.166000000000002	-0.0112617543788309\\
0.170000000000002	-0.013170402707014\\
0.174	-0.0148859569244241\\
0.174000000000001	-0.0148859569244248\\
0.175	-0.0152847812356085\\
0.175000000000002	-0.0152847812356092\\
0.176000000000001	-0.0156716063502979\\
0.177	-0.0160464448364982\\
0.178999999999998	-0.0167602102479289\\
0.179999999999998	-0.0170991603633261\\
0.18	-0.0170991603633267\\
0.183999999999997	-0.0183356626541074\\
0.186	-0.0188824741501191\\
0.186000000000002	-0.0188824741501196\\
0.189999999999998	-0.0198335731693998\\
0.192	-0.0202379842976751\\
0.192000000000002	-0.0202379842976754\\
0.195999999999998	-0.0209395293867196\\
0.199999999999995	-0.0215214927991707\\
0.199999999999997	-0.021521492799171\\
0.2	-0.0215214927991714\\
0.202999999999998	-0.0218796742141849\\
0.203	-0.0218796742141851\\
0.205999999999998	-0.022170867118344\\
0.206	-0.0221708671183442\\
0.208999999999998	-0.0223951566640238\\
0.209999999999998	-0.0224550643318199\\
0.21	-0.02245506433182\\
0.211999999999998	-0.0225526084365377\\
0.212	-0.0225526084365378\\
0.213999999999998	-0.0226204677288109\\
0.215999999999997	-0.0226586510276281\\
0.217999999999998	-0.022667163295306\\
0.218	-0.022667163295306\\
0.219999999999998	-0.0226554236175242\\
0.22	-0.0226554236175242\\
0.221999999999998	-0.0226328484480922\\
0.223999999999996	-0.0225994348531525\\
0.225999999999998	-0.0225551784902755\\
0.226	-0.0225551784902754\\
0.229999999999996	-0.022434113044661\\
0.231999999999998	-0.0223572882282681\\
0.232	-0.022357288228268\\
0.235999999999996	-0.0221710044863672\\
0.237999999999998	-0.0220615213514281\\
0.238	-0.022061521351428\\
0.239999999999998	-0.0219411255414157\\
0.24	-0.0219411255414156\\
0.241999999999998	-0.0218098014097432\\
0.243999999999996	-0.0216675318895284\\
0.245	-0.0215922868864084\\
0.245000000000002	-0.0215922868864082\\
0.245999999999998	-0.0215142984913975\\
0.246	-0.0215142984913974\\
0.246999999999999	-0.0214335641707048\\
0.247999999999998	-0.0213500813012717\\
0.249999999999997	-0.0211748589774669\\
0.252	-0.0209886087480779\\
0.252000000000003	-0.0209886087480776\\
0.256	-0.0206014286709987\\
0.259999999999997	-0.0202068421468002\\
0.26	-0.0202068421467999\\
0.260999999999996	-0.0201070143055804\\
0.261	-0.02010701430558\\
0.261999999999998	-0.0200067075173039\\
0.262999999999996	-0.0199059185225771\\
0.264999999999993	-0.0197028807995861\\
0.265999999999997	-0.0196006254746493\\
0.266	-0.019600625474649\\
0.269999999999993	-0.019186616711248\\
0.271999999999997	-0.0189765727153971\\
0.272	-0.0189765727153967\\
0.275999999999993	-0.018550269005692\\
0.279999999999986	-0.018115492540964\\
0.279999999999993	-0.0181154925409633\\
0.28	-0.0181154925409625\\
0.285999999999996	-0.0174469457727508\\
0.286	-0.0174469457727504\\
0.289999999999996	-0.0169899886525383\\
0.29	-0.0169899886525379\\
0.291999999999996	-0.0167580436729622\\
0.292	-0.0167580436729618\\
0.293999999999996	-0.0165260501331934\\
0.295999999999993	-0.0162962803602752\\
0.297999999999996	-0.0160687044933052\\
0.298	-0.0160687044933048\\
0.299999999999996	-0.0158432929566358\\
0.3	-0.0158432929566354\\
0.301999999999996	-0.0156200164558907\\
0.303999999999993	-0.015398845974034\\
0.305999999999996	-0.015179752767728\\
0.306	-0.0151797527677276\\
0.309999999999993	-0.0147476845551074\\
0.313999999999986	-0.0143235872046597\\
0.314999999999997	-0.0142187823880662\\
0.315	-0.0142187823880658\\
0.318999999999997	-0.013804339052661\\
0.319	-0.0138043390526606\\
0.319999999999996	-0.0137019053706168\\
0.32	-0.0137019053706164\\
0.320999999999998	-0.0135999358702287\\
0.321999999999996	-0.0134984272384985\\
0.323999999999993	-0.0132967794039424\\
0.325999999999996	-0.013096935660776\\
0.326	-0.0130969356607756\\
0.329999999999993	-0.0127025567933712\\
0.331	-0.0126050493319295\\
0.331000000000004	-0.0126050493319291\\
0.333	-0.01241131689032\\
0.333000000000004	-0.0124113168903197\\
0.335	-0.0122194776563815\\
0.336999999999996	-0.0120297108893724\\
0.339999999999996	-0.0117488927389454\\
0.34	-0.0117488927389451\\
0.343999999999993	-0.0113814942163562\\
0.345999999999997	-0.0112007521713454\\
0.346	-0.0112007521713451\\
0.347999999999997	-0.0110219501998974\\
0.348	-0.0110219501998971\\
0.349999999999997	-0.0108450650649932\\
0.35	-0.0108450650649929\\
0.351999999999997	-0.0106700737786224\\
0.353999999999993	-0.0104969535989516\\
0.354	-0.010496953598951\\
0.357999999999993	-0.0101562368052926\\
0.359999999999996	-0.0099885959117665\\
0.36	-0.00998859591176621\\
0.363999999999993	-0.00966025817714279\\
0.365999999999996	-0.00949992314743398\\
0.366	-0.00949992314743369\\
0.369999999999993	-0.00918681657515862\\
0.373999999999986	-0.00888365955520996\\
0.376999999999997	-0.00866272618535034\\
0.377	-0.00866272618535008\\
0.379999999999997	-0.00844723608155697\\
0.38	-0.00844723608155672\\
0.382999999999996	-0.00823712623294379\\
0.384999999999997	-0.00810001162597915\\
0.385	-0.0081000116259789\\
0.385999999999997	-0.00803233519913159\\
0.386	-0.00803233519913136\\
0.386999999999998	-0.00796524308531946\\
0.387999999999996	-0.00789873310473264\\
0.388999999999997	-0.00783280309648106\\
0.389	-0.00783280309648082\\
0.390999999999997	-0.00770267444758192\\
0.392999999999993	-0.00757484022712702\\
0.394999999999997	-0.00744928382176812\\
0.395	-0.0074492838217679\\
0.398999999999993	-0.00720465294344195\\
0.399999999999997	-0.00714480416393434\\
0.4	-0.00714480416393413\\
0.403999999999993	-0.00691058647064839\\
0.405999999999997	-0.00679655750351996\\
0.406	-0.00679655750351975\\
0.409999999999993	-0.00657458523059078\\
0.411999999999997	-0.00646661307724917\\
0.412	-0.00646661307724898\\
0.415999999999993	-0.00625662657491753\\
0.419999999999986	-0.00605449327202308\\
0.419999999999996	-0.00605449327202254\\
0.42	-0.00605449327202236\\
0.426	-0.00576578930604481\\
0.426000000000004	-0.00576578930604465\\
0.432000000000004	-0.00549418184424848\\
0.432000000000007	-0.00549418184424832\\
0.434999999999997	-0.00536439922956594\\
0.435	-0.00536439922956578\\
0.43799999999999	-0.00523819302221601\\
0.439999999999997	-0.00515602417639017\\
0.44	-0.00515602417639002\\
0.44299999999999	-0.00503569714935346\\
0.445999999999979	-0.00491885041251289\\
0.445999999999995	-0.00491885041251229\\
0.446	-0.00491885041251209\\
0.447	-0.00488066896335491\\
0.447000000000004	-0.00488066896335477\\
0.448000000000004	-0.00484286916798622\\
0.449000000000004	-0.00480544979823171\\
0.451000000000004	-0.00473174748491064\\
0.454999999999997	-0.00458885529620634\\
0.455	-0.00458885529620622\\
0.459	-0.00445191811386207\\
0.459999999999997	-0.00441860587821894\\
0.46	-0.00441860587821882\\
0.463999999999997	-0.00428901270545446\\
0.464	-0.00428901270545435\\
0.465999999999997	-0.00422639488576227\\
0.466	-0.00422639488576216\\
0.466999999999997	-0.00419562707643343\\
0.467	-0.00419562707643332\\
0.467999999999998	-0.00416513880253127\\
0.468999999999997	-0.00413484921200659\\
0.470999999999993	-0.00407486215111517\\
0.472999999999997	-0.00401565809783476\\
0.473	-0.00401565809783466\\
0.476999999999993	-0.00389956833836478\\
0.479999999999997	-0.00381449991875875\\
0.48	-0.00381449991875865\\
0.483999999999993	-0.00370369326502329\\
0.485999999999997	-0.00364939567002454\\
0.486	-0.00364939567002445\\
0.489999999999993	-0.00354297666038756\\
0.49	-0.00354297666038737\\
0.490000000000004	-0.00354297666038727\\
0.492999999999997	-0.00346503937862841\\
0.493	-0.00346503937862832\\
0.495999999999993	-0.00338868556746301\\
0.498999999999986	-0.00331389289999764\\
0.498999999999993	-0.00331389289999746\\
0.499	-0.00331389289999729\\
0.499999999999997	-0.00328930513852606\\
0.5	-0.00328930513852597\\
0.500999999999998	-0.00326488760491461\\
0.501999999999997	-0.00324063950580012\\
0.503999999999993	-0.00319264846528835\\
0.505999999999993	-0.00314532578007042\\
0.506	-0.00314532578007025\\
0.507999999999997	-0.00309866530011343\\
0.508000000000004	-0.00309866530011327\\
0.51	-0.00305261406543181\\
0.511999999999997	-0.0030071191951903\\
0.51599999999999	-0.00291777496954556\\
0.519999999999993	-0.00283058617662023\\
0.52	-0.00283058617662008\\
0.521999999999993	-0.00278778586013303\\
0.522	-0.00278778586013288\\
0.523999999999993	-0.00274550749134963\\
0.524999999999993	-0.00272456231758125\\
0.525	-0.0027245623175811\\
0.526	-0.00270374557549404\\
0.526000000000007	-0.0027037455754939\\
0.527000000000007	-0.00268305658875907\\
0.528000000000007	-0.00266249468519739\\
0.530000000000007	-0.00262174945950128\\
0.532	-0.00258150460315236\\
0.532000000000007	-0.00258150460315222\\
0.536000000000007	-0.00250249514204711\\
0.538	-0.00246372026921363\\
0.538000000000007	-0.00246372026921349\\
0.539999999999993	-0.00242551753421773\\
0.54	-0.00242551753421759\\
0.541999999999986	-0.00238797427817236\\
0.543999999999972	-0.00235108562195193\\
0.546	-0.00231484677150225\\
0.546000000000007	-0.00231484677150212\\
0.549999999999979	-0.00224429973343307\\
0.550999999999993	-0.0022270618451542\\
0.551	-0.00222706184515408\\
0.554999999999972	-0.0021596889749526\\
0.556999999999993	-0.00212694210620408\\
0.557	-0.00212694210620396\\
0.559999999999993	-0.00207898426298481\\
0.56	-0.0020789842629847\\
0.562999999999993	-0.00203240893645441\\
0.565999999999986	-0.00198720250735346\\
0.565999999999993	-0.00198720250735335\\
0.566	-0.00198720250735324\\
0.571999999999986	-0.00190084386197493\\
0.571999999999993	-0.00190084386197483\\
0.572	-0.00190084386197473\\
0.577999999999986	-0.00181923712544194\\
0.579999999999993	-0.00179294751424575\\
0.58	-0.00179294751424566\\
0.585999999999986	-0.00171676874852746\\
0.585999999999993	-0.00171676874852737\\
0.586	-0.00171676874852728\\
0.591999999999986	-0.00164455638742728\\
0.591999999999993	-0.0016445563874272\\
0.592	-0.00164455638742711\\
0.594999999999993	-0.00160991103328873\\
0.595	-0.00160991103328864\\
0.597999999999993	-0.00157622596123363\\
0.599999999999993	-0.00155429786089437\\
0.6	-0.00155429786089429\\
0.602999999999993	-0.00152219150742449\\
0.605999999999986	-0.00149101978560816\\
0.606	-0.00149101978560801\\
0.606999999999993	-0.00148083532431449\\
0.607	-0.00148083532431442\\
0.607999999999999	-0.00147072972894193\\
0.608999999999997	-0.00146067903498334\\
0.609000000000004	-0.00146067903498327\\
0.611	-0.00144074104690527\\
0.612999999999997	-0.00142101876894327\\
0.614999999999997	-0.00140150963798814\\
0.615000000000004	-0.00140150963798808\\
0.618999999999997	-0.00136312070288643\\
0.619999999999993	-0.00135365275748734\\
0.62	-0.00135365275748728\\
0.623999999999993	-0.00131628890083929\\
0.625999999999993	-0.00129790748440702\\
0.626	-0.00129790748440696\\
0.629999999999993	-0.00126173373081034\\
0.63	-0.00126173373081028\\
0.633999999999993	-0.0012263297876626\\
0.635999999999993	-0.0012089107280837\\
0.636	-0.00120891072808364\\
0.637999999999993	-0.00119167546743098\\
0.638	-0.00119167546743092\\
0.639999999999993	-0.00117461998325188\\
0.64	-0.00117461998325182\\
0.641999999999993	-0.00115774205901393\\
0.643999999999986	-0.00114103950126541\\
0.645999999999993	-0.00112451013934017\\
0.646	-0.00112451013934012\\
0.649999999999986	-0.00109196243260266\\
0.65	-0.00109196243260255\\
0.650000000000007	-0.00109196243260249\\
0.653999999999993	-0.00106008201867179\\
0.657999999999979	-0.00102885232431956\\
0.659999999999993	-0.00101347641048094\\
0.66	-0.00101347641048088\\
0.664999999999993	-0.000975717507553919\\
0.665	-0.000975717507553866\\
0.665999999999993	-0.000968280484185491\\
0.666	-0.000968280484185439\\
0.666999999999998	-0.000960881146542803\\
0.667000000000006	-0.000960881146542751\\
0.668000000000004	-0.000953519254216742\\
0.669000000000002	-0.000946194568021774\\
0.670999999999998	-0.000931655863318971\\
0.673000000000005	-0.000917263142982852\\
0.673000000000013	-0.000917263142982801\\
0.677000000000005	-0.000888908192260236\\
0.678	-0.000881907794811557\\
0.678000000000007	-0.000881907794811507\\
0.679999999999993	-0.000868030488427701\\
0.68	-0.000868030488427652\\
0.681999999999986	-0.000854329050333715\\
0.683999999999972	-0.000840801699889314\\
0.686	-0.000827446679078431\\
0.686000000000007	-0.000827446679078384\\
0.689999999999979	-0.000801246706084755\\
0.69399999999995	-0.000775715511132179\\
0.695999999999993	-0.000763196544275791\\
0.696	-0.000763196544275747\\
0.699999999999993	-0.000738643736996597\\
0.7	-0.000738643736996554\\
0.703999999999993	-0.000714727163342571\\
0.705999999999993	-0.000703003565966427\\
0.706	-0.000703003565966385\\
0.707999999999993	-0.000691434390009101\\
0.708	-0.000691434390009061\\
0.709999999999993	-0.00068001813194603\\
0.711999999999986	-0.000668753308119149\\
0.713999999999993	-0.000657638454550774\\
0.714	-0.000657638454550734\\
0.717999999999986	-0.000635973497491851\\
0.719999999999993	-0.000625450726921634\\
0.72	-0.000625450726921597\\
0.723999999999986	-0.000605017763848642\\
0.724999999999993	-0.000600036172966322\\
0.725	-0.000600036172966287\\
0.725999999999993	-0.000595104915874549\\
0.726	-0.000595104915874515\\
0.726999999999999	-0.00059022383236026\\
0.727999999999997	-0.000585392763836592\\
0.729999999999993	-0.000575880045541254\\
0.731999999999993	-0.000566565524714689\\
0.732	-0.000566565524714657\\
0.734999999999993	-0.000552962722743504\\
0.735	-0.000552962722743472\\
0.737999999999993	-0.00053979916978624\\
0.74	-0.000531265588952171\\
0.740000000000007	-0.000531265588952141\\
0.743	-0.000518825626820687\\
0.745999999999993	-0.00050681493155342\\
0.746000000000007	-0.000506814931553365\\
0.746999999999993	-0.000502906148469695\\
0.747	-0.000502906148469668\\
0.747999999999999	-0.00049903455798491\\
0.748999999999997	-0.000495190048830314\\
0.750999999999993	-0.000487581775759804\\
0.753999999999993	-0.000476369381836014\\
0.754	-0.000476369381835987\\
0.757999999999993	-0.000461787803823608\\
0.759999999999993	-0.000454652713606544\\
0.76	-0.000454652713606519\\
0.763999999999993	-0.000440689294263871\\
0.766	-0.000433859150450819\\
0.766000000000007	-0.000433859150450795\\
0.77	-0.000420497556213229\\
0.770000000000007	-0.000420497556213206\\
0.774	-0.000407528452914783\\
0.776	-0.000401188970596003\\
0.776000000000007	-0.000401188970595981\\
0.779999999999993	-0.000388767641396413\\
0.78	-0.000388767641396391\\
0.782999999999993	-0.000379661485049944\\
0.783	-0.000379661485049923\\
0.785999999999993	-0.000370732228558406\\
0.786000000000001	-0.000370732228558385\\
0.788999999999994	-0.00036197726088567\\
0.791999999999987	-0.000353394021960795\\
0.792	-0.000353394021960756\\
0.792000000000008	-0.000353394021960736\\
0.797999999999994	-0.000336732740416052\\
0.799999999999993	-0.000331326016780796\\
0.8	-0.000331326016780777\\
0.804999999999993	-0.000318124178754444\\
0.805000000000001	-0.000318124178754425\\
0.805999999999993	-0.000315537118742536\\
0.806	-0.000315537118742517\\
0.806999999999994	-0.000312967630466049\\
0.807999999999987	-0.00031041563044068\\
0.809999999999973	-0.000305363764053996\\
0.811999999999993	-0.000300380863041349\\
0.812	-0.000300380863041332\\
0.815999999999973	-0.00029062570780146\\
0.817999999999993	-0.000285853768762768\\
0.818000000000001	-0.000285853768762752\\
0.819999999999993	-0.000281151425523863\\
0.82	-0.000281151425523846\\
0.821999999999993	-0.000276518066970423\\
0.823999999999986	-0.000271953090950688\\
0.825999999999993	-0.000267455904199894\\
0.826	-0.000267455904199878\\
0.829999999999986	-0.000258662569431718\\
0.833999999999972	-0.000250133491395636\\
0.839999999999993	-0.000237825693550472\\
0.84	-0.000237825693550457\\
0.840999999999993	-0.000235830212485902\\
0.841000000000001	-0.000235830212485888\\
0.841999999999994	-0.000233850504990185\\
0.842999999999987	-0.000231886506733272\\
0.844999999999973	-0.00022800538320537\\
0.845999999999993	-0.000226088131837057\\
0.846	-0.000226088131837043\\
0.849999999999973	-0.000218573081396153\\
0.851999999999993	-0.000214907076367157\\
0.852	-0.000214907076367144\\
0.855999999999973	-0.00020775572446821\\
0.857999999999993	-0.000204269448207704\\
0.858	-0.000204269448207691\\
0.86	-0.000200852785067654\\
0.860000000000007	-0.000200852785067642\\
0.862000000000007	-0.000197515911023471\\
0.864000000000007	-0.00019425839241462\\
0.866	-0.00019107980589366\\
0.866000000000007	-0.000191079805893649\\
0.87	-0.000184957786970514\\
0.870000000000007	-0.000184957786970504\\
0.874	-0.000179146671684028\\
0.874999999999994	-0.000177742114009642\\
0.875000000000001	-0.000177742114009632\\
0.876	-0.00017635675258845\\
0.876000000000007	-0.00017635675258844\\
0.877000000000007	-0.000174990542411077\\
0.878000000000006	-0.000173643439089208\\
0.879999999999998	-0.00017100637856345\\
0.880000000000006	-0.000171006378563441\\
0.882000000000005	-0.00016844522829926\\
0.884000000000004	-0.000165959655450069\\
0.886000000000005	-0.000163549336990568\\
0.886000000000013	-0.000163549336990559\\
0.888000000000007	-0.000161213959677009\\
0.888000000000014	-0.000161213959677\\
0.890000000000009	-0.000158932560524497\\
0.892000000000004	-0.000156684183561721\\
0.895999999999993	-0.000152285331664397\\
0.898999999999993	-0.000149070733578774\\
0.899000000000001	-0.000149070733578767\\
0.899999999999993	-0.000148015117189559\\
0.9	-0.000148015117189551\\
0.900999999999994	-0.00014696740158166\\
0.901999999999987	-0.000145927552709866\\
0.903999999999973	-0.000143871320291598\\
0.905999999999993	-0.000141846152706551\\
0.906	-0.000141846152706543\\
0.909999999999973	-0.000137887963280688\\
0.909999999999987	-0.000137887963280675\\
0.910000000000001	-0.000137887963280661\\
0.910999999999993	-0.000136917425049025\\
0.911000000000001	-0.000136917425049018\\
0.911999999999994	-0.000135954427033275\\
0.912999999999987	-0.000134998937945522\\
0.914999999999973	-0.000133110362622773\\
0.916999999999993	-0.000131251453641309\\
0.917000000000001	-0.000131251453641302\\
0.919999999999993	-0.00012850837738974\\
0.92	-0.000128508377389734\\
0.922999999999993	-0.000125811086121122\\
0.925999999999986	-0.000123158791110783\\
0.925999999999993	-0.000123158791110777\\
0.926	-0.00012315879111077\\
0.927999999999993	-0.000121415208824316\\
0.928000000000001	-0.00012141520882431\\
0.929999999999994	-0.000119691053320369\\
0.931999999999987	-0.000117986100524433\\
0.933999999999994	-0.000116300128860666\\
0.934000000000001	-0.00011630012886066\\
0.937999999999987	-0.000112984254936241\\
0.939999999999993	-0.000111353921744261\\
0.940000000000001	-0.000111353921744255\\
0.943999999999987	-0.000108147403486229\\
0.944999999999994	-0.000107356902640734\\
0.945000000000001	-0.000107356902640728\\
0.945999999999993	-0.000106570801700549\\
0.946000000000001	-0.000106570801700544\\
0.946999999999994	-0.000105789075125811\\
0.947999999999987	-0.000105011697518153\\
0.949999999999973	-0.000103469888317514\\
0.952	-0.000101945173724927\\
0.952000000000008	-0.000101945173724922\\
0.95599999999998	-9.89474188185791e-05\\
0.956999999999994	-9.82087263275e-05\\
0.957000000000001	-9.82087263274948e-05\\
0.96	-9.60180552597849e-05\\
0.960000000000008	-9.60180552597798e-05\\
0.963000000000007	-9.38649970109269e-05\\
0.966000000000007	-9.17489219973963e-05\\
0.966000000000014	-9.17489219973914e-05\\
0.969000000000007	-8.96692114471455e-05\\
0.969000000000014	-8.96692114471407e-05\\
0.972000000000007	-8.76252572309117e-05\\
0.975	-8.56164616656996e-05\\
0.979999999999994	-8.23450085686041e-05\\
0.980000000000001	-8.23450085685995e-05\\
0.985999999999987	-7.85422953672995e-05\\
0.986000000000001	-7.85422953672908e-05\\
0.991999999999987	-7.48696023506846e-05\\
0.992000000000001	-7.48696023506762e-05\\
0.997999999999987	-7.1347030512513e-05\\
0.998000000000001	-7.1347030512505e-05\\
0.999999999999993	-7.02109037638929e-05\\
1	-7.02109037638889e-05\\
1.00199999999999	-6.90935608126913e-05\\
1.00399999999999	-6.79948564468458e-05\\
1.00599999999999	-6.69146478786238e-05\\
1.006	-6.69146478786162e-05\\
1.00999999999999	-6.48091589873392e-05\\
1.01399999999997	-6.27759995872311e-05\\
1.01499999999999	-6.22788865380725e-05\\
1.015	-6.22788865380655e-05\\
1.01999999999999	-5.98595763402566e-05\\
1.02	-5.98595763402498e-05\\
1.02499999999999	-5.75492632127948e-05\\
1.02599999999999	-5.71001135478326e-05\\
1.026	-5.71001135478262e-05\\
1.02699999999999	-5.66552338280099e-05\\
1.027	-5.66552338280036e-05\\
1.02799999999999	-5.62146095997392e-05\\
1.02899999999999	-5.57782265468593e-05\\
1.03099999999997	-5.491812739279e-05\\
1.03299999999999	-5.40748245872153e-05\\
1.033	-5.40748245872094e-05\\
1.03699999999997	-5.24403503118484e-05\\
1.04	-5.12606455202036e-05\\
1.04000000000001	-5.12606455201981e-05\\
1.04399999999999	-4.97485844878559e-05\\
1.044	-4.97485844878507e-05\\
1.046	-4.90184231333625e-05\\
1.04600000000001	-4.90184231333574e-05\\
1.04800000000001	-4.83053806191598e-05\\
1.05	-4.76093642781677e-05\\
1.05000000000001	-4.76093642781628e-05\\
1.05200000000001	-4.69302836562168e-05\\
1.05200000000002	-4.6930283656212e-05\\
1.05400000000002	-4.62680505001301e-05\\
1.05600000000002	-4.56225787462189e-05\\
1.05800000000001	-4.49937845089083e-05\\
1.05800000000002	-4.49937845089039e-05\\
1.05999999999999	-4.43768842540106e-05\\
1.06	-4.43768842540062e-05\\
1.06199999999996	-4.37670959930363e-05\\
1.06399999999992	-4.31643404778443e-05\\
1.06599999999999	-4.2568539374237e-05\\
1.066	-4.25685393742328e-05\\
1.06999999999992	-4.13974915757357e-05\\
1.07299999999999	-4.05368916442088e-05\\
1.073	-4.05368916442048e-05\\
1.07699999999992	-3.94125278367813e-05\\
1.07899999999999	-3.88600824913482e-05\\
1.079	-3.88600824913443e-05\\
1.07999999999999	-3.8586262409986e-05\\
1.08	-3.85862624099822e-05\\
1.08099999999999	-3.83140321772306e-05\\
1.08199999999999	-3.8043382948301e-05\\
1.08399999999997	-3.75067923797528e-05\\
1.08499999999999	-3.72408336063612e-05\\
1.085	-3.72408336063575e-05\\
1.08599999999999	-3.69764209689098e-05\\
1.086	-3.6976420968906e-05\\
1.08699999999999	-3.67135458768152e-05\\
1.08799999999999	-3.64521997892459e-05\\
1.08999999999997	-3.59340607128617e-05\\
1.09199999999999	-3.54219364023437e-05\\
1.092	-3.54219364023401e-05\\
1.09599999999997	-3.44129821696437e-05\\
1.09999999999995	-3.34223281191999e-05\\
1.09999999999997	-3.34223281191932e-05\\
1.1	-3.34223281191864e-05\\
1.10199999999999	-3.29337023581072e-05\\
1.102	-3.29337023581037e-05\\
1.10399999999999	-3.24494592502548e-05\\
1.10599999999997	-3.19695358602703e-05\\
1.10599999999999	-3.19695358602668e-05\\
1.106	-3.19695358602634e-05\\
1.10999999999997	-3.10223993041679e-05\\
1.11199999999999	-3.055506304823e-05\\
1.112	-3.05550630482267e-05\\
1.11599999999997	-2.96325508985978e-05\\
1.11999999999994	-2.87258537947723e-05\\
1.12	-2.87258537947595e-05\\
1.12000000000001	-2.87258537947563e-05\\
1.126	-2.73944324532531e-05\\
1.12600000000001	-2.739443245325e-05\\
1.13099999999999	-2.63101633009274e-05\\
1.131	-2.63101633009243e-05\\
1.132	-2.60959788772883e-05\\
1.13200000000001	-2.60959788772853e-05\\
1.13300000000001	-2.58829886877625e-05\\
1.13400000000001	-2.56715065479812e-05\\
1.13600000000001	-2.52530389824609e-05\\
1.138	-2.4840521796643e-05\\
1.13800000000001	-2.48405217966401e-05\\
1.13999999999999	-2.4433901379994e-05\\
1.14	-2.44339013799912e-05\\
1.14199999999997	-2.40331248883197e-05\\
1.14399999999994	-2.36381402366669e-05\\
1.14599999999999	-2.32488960927767e-05\\
1.146	-2.3248896092774e-05\\
1.14999999999994	-2.24874277240065e-05\\
1.15399999999989	-2.17483239278908e-05\\
1.15499999999999	-2.156699716316e-05\\
1.155	-2.15669971631574e-05\\
1.15999999999999	-2.06807644357507e-05\\
1.16	-2.06807644357482e-05\\
1.16499999999999	-1.98280096638531e-05\\
1.16599999999999	-1.96614146348624e-05\\
1.166	-1.966141463486e-05\\
1.17099999999999	-1.88479214107754e-05\\
1.17199999999999	-1.86890821478633e-05\\
1.172	-1.8689082147861e-05\\
1.173	-1.85315172069141e-05\\
1.17300000000001	-1.85315172069118e-05\\
1.17400000000001	-1.83750889396583e-05\\
1.17500000000001	-1.82196597344588e-05\\
1.17700000000001	-1.79117783431312e-05\\
1.17999999999999	-1.7457324021528e-05\\
1.18	-1.74573240215259e-05\\
1.184	-1.68649320379695e-05\\
1.18599999999999	-1.65744551996525e-05\\
1.186	-1.65744551996505e-05\\
1.18899999999999	-1.6145782536612e-05\\
1.189	-1.614578253661e-05\\
1.18999999999999	-1.60047510737459e-05\\
1.19	-1.60047510737439e-05\\
1.19099999999999	-1.5864641661413e-05\\
1.19199999999999	-1.57254497474873e-05\\
1.19399999999997	-1.54498003553065e-05\\
1.19499999999999	-1.53133339212306e-05\\
1.195	-1.53133339212287e-05\\
1.19899999999997	-1.47764201510806e-05\\
1.19999999999999	-1.4644407894837e-05\\
1.2	-1.46444078948351e-05\\
1.20399999999997	-1.41250949406154e-05\\
1.20599999999999	-1.38706211513074e-05\\
1.206	-1.38706211513056e-05\\
1.20699999999999	-1.37446653865764e-05\\
1.207	-1.37446653865746e-05\\
1.20799999999999	-1.36196871553865e-05\\
1.20899999999999	-1.34958113096386e-05\\
1.21099999999997	-1.32513507113985e-05\\
1.21499999999995	-1.2775483437645e-05\\
1.21799999999999	-1.24298836838874e-05\\
1.218	-1.24298836838858e-05\\
1.21999999999999	-1.22048087313558e-05\\
1.22	-1.22048087313542e-05\\
1.22199999999999	-1.19839594806224e-05\\
1.22399999999997	-1.1767307228614e-05\\
1.22499999999999	-1.16605461571029e-05\\
1.225	-1.16605461571014e-05\\
1.22599999999999	-1.15548238191545e-05\\
1.226	-1.1554823819153e-05\\
1.22699999999999	-1.14501367799238e-05\\
1.22799999999999	-1.13464816381201e-05\\
1.22999999999997	-1.11422536093129e-05\\
1.23	-1.11422536093102e-05\\
1.23399999999997	-1.07460343738581e-05\\
1.236	-1.05539916745757e-05\\
1.23600000000001	-1.05539916745743e-05\\
1.23999999999999	-1.0180960373095e-05\\
1.24	-1.01809603730937e-05\\
1.24399999999997	-9.82187041654044e-06\\
1.24599999999999	-9.64749482615837e-06\\
1.246	-9.64749482615714e-06\\
1.247	-9.56158939016521e-06\\
1.24700000000001	-9.561589390164e-06\\
1.24800000000001	-9.4765351297098e-06\\
1.24900000000001	-9.3923292813714e-06\\
1.25100000000001	-9.22645190523081e-06\\
1.253	-9.06393570430299e-06\\
1.25300000000001	-9.06393570430184e-06\\
1.25700000000001	-8.74890278010037e-06\\
1.25999999999999	-8.52129723765548e-06\\
1.26	-8.52129723765442e-06\\
1.264	-8.22925551011215e-06\\
1.266	-8.08809139882927e-06\\
1.26600000000001	-8.08809139882828e-06\\
1.27000000000001	-7.81538495072435e-06\\
1.272	-7.68380717294294e-06\\
1.27200000000001	-7.68380717294202e-06\\
1.276	-7.42876844391719e-06\\
1.27600000000001	-7.4287684439163e-06\\
1.27999999999999	-7.18354318634862e-06\\
1.28000000000001	-7.18354318634777e-06\\
1.28399999999999	-6.94800391838649e-06\\
1.28599999999999	-6.8338280709989e-06\\
1.28600000000001	-6.8338280709981e-06\\
1.288	-6.72202819343346e-06\\
1.28800000000001	-6.72202819343268e-06\\
1.29	-6.61258975623202e-06\\
1.29199999999999	-6.50549853676335e-06\\
1.29499999999999	-6.34923237910887e-06\\
1.295	-6.34923237910814e-06\\
1.29899999999998	-6.14896540500326e-06\\
1.29999999999999	-6.10033202304176e-06\\
1.3	-6.10033202304107e-06\\
1.30399999999998	-5.91148453773299e-06\\
1.30499999999999	-5.86568650752996e-06\\
1.305	-5.86568650752932e-06\\
1.30599999999999	-5.82045101034084e-06\\
1.306	-5.8204510103402e-06\\
1.30699999999999	-5.77577657649645e-06\\
1.30700000000001	-5.77577657649581e-06\\
1.308	-5.7315345471217e-06\\
1.30899999999999	-5.68759627740947e-06\\
1.31099999999998	-5.60062531663063e-06\\
1.31300000000001	-5.5148523914033e-06\\
1.31300000000002	-5.5148523914027e-06\\
1.31699999999999	-5.34685621385352e-06\\
1.31999999999999	-5.22392213606576e-06\\
1.32	-5.22392213606519e-06\\
1.32399999999997	-5.06402576030405e-06\\
1.32599999999999	-4.98577517923626e-06\\
1.326	-4.98577517923571e-06\\
1.32999999999997	-4.83261838234848e-06\\
1.33	-4.8326183823475e-06\\
1.33399999999997	-4.68385463036508e-06\\
1.334	-4.68385463036409e-06\\
1.33799999999997	-4.53940658753969e-06\\
1.34	-4.4687774041641e-06\\
1.34000000000001	-4.4687774041636e-06\\
1.34399999999999	-4.33066281760675e-06\\
1.346	-4.26315946503717e-06\\
1.34600000000001	-4.26315946503669e-06\\
1.348	-4.19668033124229e-06\\
1.34800000000001	-4.19668033124182e-06\\
1.35	-4.13113710381037e-06\\
1.35199999999999	-4.06644159192198e-06\\
1.35599999999996	-3.93956019258578e-06\\
1.35999999999999	-3.81597017256711e-06\\
1.36	-3.81597017256667e-06\\
1.36299999999999	-3.72539891231839e-06\\
1.363	-3.72539891231796e-06\\
1.36499999999999	-3.66601343156786e-06\\
1.365	-3.66601343156744e-06\\
1.36599999999999	-3.63661639510304e-06\\
1.366	-3.63661639510263e-06\\
1.36699999999999	-3.60741521897946e-06\\
1.36799999999999	-3.57840895446008e-06\\
1.36999999999997	-3.52097739691673e-06\\
1.37199999999999	-3.46431425866118e-06\\
1.372	-3.46431425866078e-06\\
1.37599999999997	-3.35326388328442e-06\\
1.378	-3.29886221404358e-06\\
1.37800000000001	-3.2988622140432e-06\\
1.37999999999999	-3.24532091911658e-06\\
1.38	-3.2453209191162e-06\\
1.38199999999997	-3.19275386135538e-06\\
1.38399999999994	-3.14115420913963e-06\\
1.38599999999999	-3.09051525657142e-06\\
1.386	-3.09051525657107e-06\\
1.38999999999994	-2.9920932503445e-06\\
1.39199999999999	-2.94429740574817e-06\\
1.392	-2.94429740574783e-06\\
1.39599999999994	-2.85150497515212e-06\\
1.39799999999999	-2.80649632983515e-06\\
1.398	-2.80649632983484e-06\\
1.39999999999999	-2.76240489211035e-06\\
1.4	-2.76240489211004e-06\\
1.40199999999999	-2.71922493187476e-06\\
1.40399999999997	-2.67695083745608e-06\\
1.40599999999999	-2.63557711490801e-06\\
1.406	-2.63557711490772e-06\\
1.40999999999997	-2.55550939411761e-06\\
1.412	-2.51680499026436e-06\\
1.41200000000001	-2.51680499026408e-06\\
1.41599999999999	-2.4416216329719e-06\\
1.41999999999996	-2.36910091893579e-06\\
1.41999999999998	-2.36910091893541e-06\\
1.42	-2.36910091893502e-06\\
1.42099999999999	-2.35138233020653e-06\\
1.421	-2.35138233020628e-06\\
1.42199999999999	-2.33382721816548e-06\\
1.42299999999999	-2.316435012363e-06\\
1.42499999999997	-2.28213706447805e-06\\
1.42599999999999	-2.26523020805867e-06\\
1.426	-2.26523020805843e-06\\
1.42999999999997	-2.19920414305382e-06\\
1.43199999999999	-2.16714443696985e-06\\
1.432	-2.16714443696963e-06\\
1.43499999999999	-2.12023478857781e-06\\
1.435	-2.12023478857759e-06\\
1.43799999999998	-2.07472887077619e-06\\
1.43999999999999	-2.04516485465885e-06\\
1.44	-2.04516485465864e-06\\
1.44299999999999	-2.00196911866891e-06\\
1.44599999999997	-1.96014253078412e-06\\
1.44599999999999	-1.96014253078391e-06\\
1.446	-1.9601425307837e-06\\
1.44699999999999	-1.94650246452656e-06\\
1.447	-1.94650246452636e-06\\
1.44799999999999	-1.93297720768573e-06\\
1.44899999999999	-1.91953080801798e-06\\
1.44999999999999	-1.9061628286535e-06\\
1.45	-1.90616282865331e-06\\
1.45199999999999	-1.87966039608351e-06\\
1.45399999999997	-1.85346646572479e-06\\
1.45599999999999	-1.82757763341117e-06\\
1.456	-1.82757763341098e-06\\
1.45999999999997	-1.77670184414105e-06\\
1.46	-1.77670184414069e-06\\
1.46000000000001	-1.77670184414052e-06\\
1.46399999999999	-1.72700658017452e-06\\
1.466	-1.70259354832369e-06\\
1.46600000000001	-1.70259354832352e-06\\
1.46999999999999	-1.65462082093921e-06\\
1.47	-1.65462082093904e-06\\
1.47399999999997	-1.6077651542917e-06\\
1.47599999999999	-1.58474858446257e-06\\
1.476	-1.58474858446241e-06\\
1.47899999999998	-1.55070498138273e-06\\
1.479	-1.55070498138257e-06\\
1.47999999999999	-1.53947963257197e-06\\
1.48	-1.53947963257181e-06\\
1.48099999999999	-1.52831493370332e-06\\
1.48199999999999	-1.517210522038e-06\\
1.48399999999997	-1.49518111914843e-06\\
1.48599999999999	-1.47338856095956e-06\\
1.486	-1.4733885609594e-06\\
1.48999999999997	-1.43050268050243e-06\\
1.491	-1.41992484981844e-06\\
1.49100000000001	-1.41992484981829e-06\\
1.49499999999999	-1.37817777442642e-06\\
1.49899999999996	-1.33731728373921e-06\\
1.49999999999999	-1.32723819066221e-06\\
1.5	-1.32723819066207e-06\\
1.50499999999999	-1.27764255119209e-06\\
1.505	-1.27764255119195e-06\\
1.50599999999999	-1.26788113260993e-06\\
1.506	-1.26788113260979e-06\\
1.50699999999999	-1.25817153982713e-06\\
1.50799999999999	-1.24851345737308e-06\\
1.508	-1.24851345737295e-06\\
1.50999999999999	-1.22935056996169e-06\\
1.51199999999997	-1.21038997997074e-06\\
1.51399999999999	-1.19162922327978e-06\\
1.514	-1.19162922327965e-06\\
1.51799999999997	-1.1546974829024e-06\\
1.518	-1.15469748290216e-06\\
1.51999999999999	-1.13654290464229e-06\\
1.52	-1.13654290464216e-06\\
1.52199999999999	-1.1186209727003e-06\\
1.52399999999997	-1.10092935794029e-06\\
1.52599999999999	-1.08346576115795e-06\\
1.526	-1.08346576115783e-06\\
1.52999999999997	-1.04921357262852e-06\\
1.53399999999994	-1.01584660089692e-06\\
1.537	-9.91391843703725e-07\\
1.53700000000001	-9.9139184370361e-07\\
1.53999999999999	-9.6741810367338e-07\\
1.54	-9.67418103673268e-07\\
1.54299999999997	-9.4391837070677e-07\\
1.54599999999994	-9.20885773010966e-07\\
1.54599999999997	-9.20885773010755e-07\\
1.546	-9.20885773010543e-07\\
1.549	-8.98313575508408e-07\\
1.54900000000001	-8.98313575508302e-07\\
1.55200000000001	-8.76195177852366e-07\\
1.55500000000001	-8.54524112289572e-07\\
1.55500000000003	-8.54524112289471e-07\\
1.55999999999999	-8.1964574175582e-07\\
1.56	-8.19645741755723e-07\\
1.56499999999996	-7.86490386777966e-07\\
1.56599999999998	-7.80063683249524e-07\\
1.566	-7.80063683249433e-07\\
1.57099999999996	-7.48940518397185e-07\\
1.57199999999998	-7.42916540326751e-07\\
1.572	-7.42916540326666e-07\\
1.57499999999999	-7.25242373455314e-07\\
1.575	-7.25242373455232e-07\\
1.57799999999999	-7.08160860499542e-07\\
1.57999999999999	-6.97099963677889e-07\\
1.58	-6.97099963677812e-07\\
1.58299999999999	-6.80995192449376e-07\\
1.58599999999997	-6.65470136653115e-07\\
1.586	-6.65470136652974e-07\\
1.58699999999999	-6.60423154193241e-07\\
1.587	-6.6042315419317e-07\\
1.58799999999999	-6.5542521801101e-07\\
1.58899999999999	-6.50461468932103e-07\\
1.59099999999997	-6.40635888106903e-07\\
1.59499999999995	-6.21387949563937e-07\\
1.59499999999997	-6.21387949563809e-07\\
1.595	-6.21387949563682e-07\\
1.59999999999999	-5.98071238189109e-07\\
1.6	-5.98071238189044e-07\\
1.60499999999999	-5.75562671655537e-07\\
1.60599999999999	-5.71156315376054e-07\\
1.606	-5.71156315375992e-07\\
1.60699999999998	-5.66781408329532e-07\\
1.607	-5.6678140832947e-07\\
1.60799999999999	-5.62437808380658e-07\\
1.60899999999999	-5.58125374403174e-07\\
1.60999999999998	-5.53843966287283e-07\\
1.61	-5.53843966287222e-07\\
1.61199999999999	-5.45373672237784e-07\\
1.61399999999997	-5.37025825434391e-07\\
1.61599999999999	-5.28799340989976e-07\\
1.616	-5.28799340989918e-07\\
1.61999999999997	-5.12668493258539e-07\\
1.61999999999999	-5.12668493258481e-07\\
1.62	-5.12668493258424e-07\\
1.62399999999997	-4.969350382049e-07\\
1.624	-4.96935038204804e-07\\
1.62599999999999	-4.89214768998135e-07\\
1.626	-4.8921476899808e-07\\
1.62799999999999	-4.81590796693942e-07\\
1.62999999999998	-4.74062130480093e-07\\
1.63199999999999	-4.66627791930287e-07\\
1.632	-4.66627791930234e-07\\
1.63599999999998	-4.52038245306433e-07\\
1.63999999999995	-4.37814572355073e-07\\
1.63999999999998	-4.37814572354989e-07\\
1.64	-4.37814572354905e-07\\
1.645	-4.20538244066133e-07\\
1.64500000000001	-4.20538244066084e-07\\
1.64599999999999	-4.17148952118714e-07\\
1.646	-4.17148952118666e-07\\
1.64699999999999	-4.13781392808056e-07\\
1.64799999999999	-4.10435456720119e-07\\
1.64999999999997	-4.03808020077581e-07\\
1.65199999999999	-3.97265780888672e-07\\
1.652	-3.97265780888626e-07\\
1.65299999999998	-3.94027198333299e-07\\
1.653	-3.94027198333253e-07\\
1.65399999999999	-3.90811305432929e-07\\
1.65499999999997	-3.87617997702943e-07\\
1.65699999999995	-3.81298723486112e-07\\
1.65899999999998	-3.75068554429104e-07\\
1.659	-3.7506855442906e-07\\
1.65999999999999	-3.71986630864172e-07\\
1.66	-3.71986630864128e-07\\
1.66099999999999	-3.68926680863035e-07\\
1.66199999999998	-3.65888605007657e-07\\
1.66399999999995	-3.59877681616973e-07\\
1.66599999999999	-3.53953079511738e-07\\
1.666	-3.53953079511696e-07\\
1.66999999999995	-3.42359770444758e-07\\
1.67399999999991	-3.31102650952164e-07\\
1.67999999999998	-3.14834572750618e-07\\
1.68	-3.1483457275058e-07\\
1.68199999999998	-3.09573744030886e-07\\
1.682	-3.09573744030849e-07\\
1.68399999999998	-3.0439269914411e-07\\
1.68599999999997	-2.99290764727075e-07\\
1.68599999999999	-2.99290764727037e-07\\
1.686	-2.99290764727001e-07\\
1.68999999999997	-2.89321585311426e-07\\
1.69199999999999	-2.84453044718731e-07\\
1.692	-2.84453044718697e-07\\
1.69599999999997	-2.7494489811434e-07\\
1.69799999999999	-2.70304056422813e-07\\
1.698	-2.70304056422781e-07\\
1.69999999999999	-2.65753192396858e-07\\
1.7	-2.65753192396826e-07\\
1.70199999999999	-2.6130701195764e-07\\
1.70399999999998	-2.56964937279055e-07\\
1.70599999999999	-2.52726404064569e-07\\
1.706	-2.52726404064539e-07\\
1.70999999999998	-2.44557772063317e-07\\
1.71099999999998	-2.42579482982904e-07\\
1.711	-2.42579482982877e-07\\
1.71499999999997	-2.34919874071443e-07\\
1.715	-2.34919874071395e-07\\
1.71699999999998	-2.3124132992782e-07\\
1.717	-2.31241329927795e-07\\
1.71899999999998	-2.27662984215705e-07\\
1.71999999999999	-2.25911239938646e-07\\
1.72	-2.25911239938621e-07\\
1.72199999999999	-2.22482323972695e-07\\
1.72399999999997	-2.19152468118394e-07\\
1.72599999999999	-2.15921239627272e-07\\
1.726	-2.15921239627249e-07\\
1.72899999999998	-2.11258408094454e-07\\
1.729	-2.11258408094432e-07\\
1.73199999999998	-2.06757624049903e-07\\
1.73499999999997	-2.02360012647313e-07\\
1.73999999999998	-1.95256453718612e-07\\
1.74	-1.95256453718592e-07\\
1.74599999999997	-1.87097675318713e-07\\
1.74599999999998	-1.87097675318691e-07\\
1.746	-1.87097675318669e-07\\
1.74999999999998	-1.81875573125122e-07\\
1.75	-1.81875573125104e-07\\
1.75199999999998	-1.79328620630159e-07\\
1.752	-1.79328620630141e-07\\
1.75399999999998	-1.76823957354738e-07\\
1.75599999999997	-1.74361257792563e-07\\
1.75799999999998	-1.71940201890909e-07\\
1.758	-1.71940201890892e-07\\
1.75999999999999	-1.69554019960508e-07\\
1.76	-1.69554019960491e-07\\
1.76199999999999	-1.67195946844166e-07\\
1.76399999999998	-1.64865676086473e-07\\
1.76599999999999	-1.62562904845182e-07\\
1.766	-1.62562904845166e-07\\
1.76899999999998	-1.5915965589808e-07\\
1.769	-1.59159655898064e-07\\
1.77199999999998	-1.55816613184853e-07\\
1.77499999999996	-1.52532799154329e-07\\
1.77499999999998	-1.5253279915431e-07\\
1.775	-1.52532799154292e-07\\
1.78	-1.47188794748675e-07\\
1.78000000000002	-1.4718879474866e-07\\
1.78499999999998	-1.42002313256241e-07\\
1.785	-1.42002313256226e-07\\
1.78600000000001	-1.40983546753248e-07\\
1.78600000000003	-1.40983546753234e-07\\
1.78700000000005	-1.39970879151721e-07\\
1.78800000000006	-1.38964277549456e-07\\
1.79000000000009	-1.36969141722007e-07\\
1.79200000000003	-1.3499787998294e-07\\
1.79200000000004	-1.34997879982927e-07\\
1.79600000000011	-1.31121924558128e-07\\
1.79799999999998	-1.29215719049733e-07\\
1.798	-1.29215719049719e-07\\
1.8	-1.27330363790059e-07\\
1.80000000000002	-1.27330363790045e-07\\
1.80200000000002	-1.25465613759372e-07\\
1.80400000000002	-1.2362122661458e-07\\
1.806	-1.21796962658964e-07\\
1.80600000000002	-1.21796962658951e-07\\
1.80999999999998	-1.18207858578257e-07\\
1.81	-1.18207858578244e-07\\
1.81399999999997	-1.14696435698988e-07\\
1.81799999999994	-1.11260868586497e-07\\
1.82	-1.09570971291549e-07\\
1.82000000000001	-1.09570971291537e-07\\
1.826	-1.04610196147631e-07\\
1.82600000000001	-1.04610196147619e-07\\
1.827	-1.03799001540376e-07\\
1.82700000000001	-1.03799001540364e-07\\
1.828	-1.02992193673242e-07\\
1.82899999999999	-1.02189746331835e-07\\
1.83099999999996	-1.00597829081938e-07\\
1.83200000000001	-9.98083074521864e-08\\
1.83200000000003	-9.98083074521752e-08\\
1.83599999999998	-9.67068784143788e-08\\
1.83800000000001	-9.51919574659811e-08\\
1.83800000000003	-9.51919574659704e-08\\
1.83999999999999	-9.37006353623608e-08\\
1.84	-9.37006353623503e-08\\
1.84199999999996	-9.22327182919678e-08\\
1.84399999999992	-9.078801548415e-08\\
1.84599999999999	-8.93663391851083e-08\\
1.846	-8.93663391850983e-08\\
1.84999999999992	-8.6591330041212e-08\\
1.85399999999984	-8.39062482543018e-08\\
1.85499999999998	-8.32488643600815e-08\\
1.855	-8.32488643600722e-08\\
1.85599999999998	-8.25969921019009e-08\\
1.856	-8.25969921018916e-08\\
1.85699999999999	-8.19506103037564e-08\\
1.85799999999997	-8.13096979616436e-08\\
1.85999999999995	-8.00441985303418e-08\\
1.85999999999998	-8.00441985303257e-08\\
1.86	-8.00441985303095e-08\\
1.86399999999995	-7.75779287493652e-08\\
1.86599999999999	-7.63768378833824e-08\\
1.866	-7.63768378833739e-08\\
1.86799999999998	-7.51969006528353e-08\\
1.868	-7.5196900652827e-08\\
1.86999999999998	-7.40379637136263e-08\\
1.87199999999996	-7.28998764501895e-08\\
1.87299999999998	-7.23386051208194e-08\\
1.873	-7.23386051208115e-08\\
1.87699999999996	-7.01324887615064e-08\\
1.87999999999999	-6.8515057916358e-08\\
1.88	-6.85150579163504e-08\\
1.88399999999997	-6.64071261357416e-08\\
1.88499999999998	-6.58887148650397e-08\\
1.885	-6.58887148650323e-08\\
1.886	-6.53736984115734e-08\\
1.88600000000002	-6.53736984115661e-08\\
1.88700000000002	-6.4862060042826e-08\\
1.88800000000002	-6.43537831356236e-08\\
1.88999999999998	-6.33472477594973e-08\\
1.89	-6.33472477594902e-08\\
1.892	-6.2353961473994e-08\\
1.89200000000002	-6.2353961473987e-08\\
1.89400000000002	-6.1373795191848e-08\\
1.89600000000003	-6.04066215308447e-08\\
1.898	-5.94523147970117e-08\\
1.89800000000002	-5.9452314797005e-08\\
1.9	-5.85111993133228e-08\\
1.90000000000002	-5.85111993133162e-08\\
1.902	-5.75836011171071e-08\\
1.90399999999998	-5.66693996576102e-08\\
1.906	-5.57684761251e-08\\
1.90600000000002	-5.57684761250937e-08\\
1.90999999999998	-5.40059962174294e-08\\
1.91399999999995	-5.2295245157851e-08\\
1.91399999999997	-5.22952451578401e-08\\
1.914	-5.22952451578292e-08\\
1.91999999999998	-4.98241719534748e-08\\
1.92	-4.98241719534691e-08\\
1.92499999999998	-4.7850240108676e-08\\
1.925	-4.78502401086705e-08\\
1.92599999999998	-4.74646033171379e-08\\
1.926	-4.74646033171325e-08\\
1.92699999999999	-4.70819869198595e-08\\
1.92799999999997	-4.6702378485393e-08\\
1.92999999999995	-4.59521362687676e-08\\
1.93199999999998	-4.5213779165685e-08\\
1.932	-4.52137791656798e-08\\
1.93599999999995	-4.37701446388508e-08\\
1.9399999999999	-4.23685310547912e-08\\
1.93999999999999	-4.23685310547615e-08\\
1.94	-4.23685310547566e-08\\
1.94299999999998	-4.13444579724297e-08\\
1.943	-4.13444579724249e-08\\
1.94599999999998	-4.03433119254613e-08\\
1.946	-4.03433119254543e-08\\
1.94899999999998	-3.93648001586018e-08\\
1.95199999999997	-3.8408636541785e-08\\
1.95199999999998	-3.84086365417795e-08\\
1.952	-3.84086365417739e-08\\
1.95799999999997	-3.65622418307297e-08\\
1.95999999999998	-3.59660186285265e-08\\
1.96	-3.59660186285223e-08\\
1.96599999999996	-3.42339892726863e-08\\
1.96599999999998	-3.42339892726814e-08\\
1.966	-3.42339892726764e-08\\
1.97199999999996	-3.25853573391773e-08\\
1.97199999999998	-3.25853573391726e-08\\
1.972	-3.2585357339168e-08\\
1.97799999999996	-3.10220369080688e-08\\
1.97799999999998	-3.10220369080643e-08\\
1.978	-3.10220369080599e-08\\
1.98	-3.05204230781025e-08\\
1.98000000000002	-3.0520423078099e-08\\
1.98200000000002	-3.00284460383846e-08\\
1.98400000000002	-2.95460418506071e-08\\
1.986	-2.90731478214773e-08\\
1.98600000000002	-2.90731478214739e-08\\
1.99000000000002	-2.81556456389185e-08\\
1.99400000000003	-2.72754625214488e-08\\
1.995	-2.70611938007662e-08\\
1.99500000000001	-2.70611938007632e-08\\
1.99999999999999	-2.60241654028837e-08\\
2	-2.60241654028809e-08\\
2.00099999999997	-2.58235755949382e-08\\
2.001	-2.58235755949325e-08\\
2.00199999999999	-2.5625242365797e-08\\
2.00299999999997	-2.54291592714937e-08\\
2.00499999999995	-2.50437180775691e-08\\
2.00599999999997	-2.48543474549988e-08\\
2.006	-2.48543474549935e-08\\
2.00999999999995	-2.41190554093091e-08\\
2.01199999999997	-2.37646402913546e-08\\
2.012	-2.37646402913497e-08\\
2.01299999999998	-2.35907201199138e-08\\
2.01300000000001	-2.35907201199089e-08\\
2.014	-2.34185407505901e-08\\
2.01499999999999	-2.32476533574679e-08\\
2.01699999999996	-2.29097323332971e-08\\
2.01999999999997	-2.24124031711204e-08\\
2.02	-2.24124031711157e-08\\
2.02399999999995	-2.17669033657222e-08\\
2.02599999999997	-2.14516042233281e-08\\
2.026	-2.14516042233236e-08\\
2.02999999999995	-2.08357025731386e-08\\
2.03	-2.08357025731319e-08\\
2.03399999999995	-2.02391306773921e-08\\
2.03599999999997	-1.99479960831541e-08\\
2.036	-1.994799608315e-08\\
2.03999999999995	-1.93798404184137e-08\\
2.04	-1.93798404184064e-08\\
2.04399999999995	-1.88302576692654e-08\\
2.04599999999997	-1.85623414824093e-08\\
2.046	-1.85623414824055e-08\\
2.04799999999997	-1.82989621316002e-08\\
2.048	-1.82989621315965e-08\\
2.04999999999996	-1.80401239766215e-08\\
2.05199999999993	-1.77858319672819e-08\\
2.05599999999987	-1.72907547810689e-08\\
2.05899999999997	-1.69311389834356e-08\\
2.059	-1.69311389834322e-08\\
2.05999999999997	-1.68134732011613e-08\\
2.06	-1.68134732011579e-08\\
2.06099999999999	-1.66969040931563e-08\\
2.06199999999998	-1.65814278715372e-08\\
2.06399999999995	-1.63537391156869e-08\\
2.06499999999997	-1.62415191838698e-08\\
2.065	-1.62415191838666e-08\\
2.06599999999997	-1.61303773431243e-08\\
2.066	-1.61303773431212e-08\\
2.06699999999999	-1.60203099825285e-08\\
2.06799999999998	-1.5911313525987e-08\\
2.06999999999995	-1.56965191946919e-08\\
2.07099999999997	-1.55907143412913e-08\\
2.071	-1.55907143412883e-08\\
2.07499999999995	-1.51780305328708e-08\\
2.07699999999997	-1.49779631829772e-08\\
2.077	-1.49779631829744e-08\\
2.07999999999997	-1.46844178551145e-08\\
2.08	-1.46844178551118e-08\\
2.08299999999998	-1.43976918845526e-08\\
2.08599999999995	-1.41177014287175e-08\\
2.086	-1.41177014287131e-08\\
2.08799999999997	-1.39347425656365e-08\\
2.088	-1.39347425656339e-08\\
2.08999999999997	-1.3754717059712e-08\\
2.09199999999993	-1.35776015144807e-08\\
2.09399999999997	-1.34033729119811e-08\\
2.094	-1.34033729119786e-08\\
2.09799999999993	-1.30634863368428e-08\\
2.09999999999997	-1.28977841924998e-08\\
2.1	-1.28977841924975e-08\\
2.10399999999993	-1.25747545145896e-08\\
2.10599999999997	-1.24173850000424e-08\\
2.106	-1.24173850000402e-08\\
2.10999999999993	-1.21108343587518e-08\\
2.112	-1.19616133926442e-08\\
2.11200000000003	-1.19616133926421e-08\\
2.11599999999996	-1.16695716309546e-08\\
2.11699999999997	-1.15977128334947e-08\\
2.117	-1.15977128334927e-08\\
2.11999999999997	-1.13848629693614e-08\\
2.12	-1.13848629693594e-08\\
2.12299999999998	-1.11760546432405e-08\\
2.12599999999995	-1.09712267967374e-08\\
2.126	-1.09712267967341e-08\\
2.12899999999997	-1.0770319535115e-08\\
2.129	-1.07703195351131e-08\\
2.13199999999996	-1.05732741109821e-08\\
2.13499999999993	-1.03800329052864e-08\\
2.13499999999997	-1.03800329052843e-08\\
2.135	-1.03800329052821e-08\\
2.13999999999997	-1.00662650703644e-08\\
2.14	-1.00662650703627e-08\\
2.14499999999998	-9.76265312581216e-09\\
2.14599999999997	-9.7031275979927e-09\\
2.146	-9.70312759799101e-09\\
2.14699999999997	-9.64399648116707e-09\\
2.14699999999999	-9.64399648116539e-09\\
2.14799999999998	-9.5851635222707e-09\\
2.14899999999997	-9.5265324778186e-09\\
2.15099999999995	-9.40986851912303e-09\\
2.15299999999999	-9.29398944341576e-09\\
2.15300000000002	-9.29398944341412e-09\\
2.15699999999997	-9.06452580270846e-09\\
2.15999999999997	-8.89437710658337e-09\\
2.16	-8.89437710658177e-09\\
2.16399999999995	-8.67001629585317e-09\\
2.16599999999997	-8.55887625503718e-09\\
2.166	-8.5588762550356e-09\\
2.16999999999995	-8.33860468206925e-09\\
2.17	-8.3386046820667e-09\\
2.17399999999995	-8.12091584014499e-09\\
2.17499999999997	-8.06688389096518e-09\\
2.175	-8.06688389096364e-09\\
2.17899999999995	-7.85226447365822e-09\\
2.17999999999997	-7.79897799801965e-09\\
2.18	-7.79897799801814e-09\\
2.18399999999995	-7.58725367105386e-09\\
2.18599999999997	-7.48222039272814e-09\\
2.186	-7.48222039272665e-09\\
2.187	-7.42990499129917e-09\\
2.18700000000002	-7.42990499129768e-09\\
2.18800000000002	-7.37772148065659e-09\\
2.18800000000005	-7.37772148065511e-09\\
2.18900000000004	-7.32570323052513e-09\\
2.19000000000004	-7.27388361602122e-09\\
2.19200000000003	-7.17083356587807e-09\\
2.19600000000001	-6.96704344024927e-09\\
2.2	-6.76624516128715e-09\\
2.20000000000003	-6.76624516128573e-09\\
2.20399999999997	-6.56833434374509e-09\\
2.204	-6.56833434374369e-09\\
2.20499999999997	-6.51929571634699e-09\\
2.205	-6.5192957163456e-09\\
2.20599999999999	-6.47042952466547e-09\\
2.20600000000003	-6.47042952466341e-09\\
2.20700000000002	-6.42173418103352e-09\\
2.20800000000001	-6.37320810335734e-09\\
2.20999999999998	-6.2766574449258e-09\\
2.21200000000003	-6.18076500163295e-09\\
2.21200000000006	-6.18076500163159e-09\\
2.21600000000001	-5.99090499588505e-09\\
2.21800000000003	-5.89691275920133e-09\\
2.21800000000006	-5.8969127592e-09\\
2.21999999999997	-5.80392343288967e-09\\
2.22	-5.80392343288836e-09\\
2.22199999999992	-5.71231897884445e-09\\
2.22399999999983	-5.62208749214222e-09\\
2.226	-5.53321724627805e-09\\
2.22600000000003	-5.53321724627679e-09\\
2.22999999999986	-5.35951445441455e-09\\
2.23299999999997	-5.23272598974714e-09\\
2.233	-5.23272598974595e-09\\
2.23699999999983	-5.06825572044294e-09\\
2.23899999999997	-4.98795978466435e-09\\
2.239	-4.98795978466321e-09\\
2.24	-4.94829202362755e-09\\
2.24000000000003	-4.94829202362643e-09\\
2.24100000000003	-4.90894267865719e-09\\
2.24200000000003	-4.86991047129017e-09\\
2.24400000000004	-4.79279240704828e-09\\
2.246	-4.71692780863133e-09\\
2.24600000000003	-4.71692780863027e-09\\
2.25000000000004	-4.56891973359997e-09\\
2.252	-4.49675702184465e-09\\
2.25200000000003	-4.49675702184363e-09\\
2.25600000000004	-4.35594226052024e-09\\
2.25999999999997	-4.21966423239372e-09\\
2.26	-4.21966423239277e-09\\
2.26199999999997	-4.15320426309415e-09\\
2.262	-4.15320426309322e-09\\
2.26399999999996	-4.08785209247352e-09\\
2.26599999999993	-4.02359922735743e-09\\
2.26599999999997	-4.0235992273563e-09\\
2.266	-4.02359922735518e-09\\
2.26999999999994	-3.89835815425466e-09\\
2.272	-3.83735366994323e-09\\
2.27200000000003	-3.83735366994237e-09\\
2.27499999999997	-3.74784466160139e-09\\
2.275	-3.74784466160055e-09\\
2.27799999999994	-3.66070970136271e-09\\
2.27999999999997	-3.60392603269303e-09\\
2.28	-3.60392603269223e-09\\
2.28299999999994	-3.52069153773857e-09\\
2.28599999999989	-3.43976466823471e-09\\
2.28599999999997	-3.43976466823239e-09\\
2.286	-3.43976466823163e-09\\
2.28699999999997	-3.41329774127334e-09\\
2.287	-3.4132977412726e-09\\
2.28799999999999	-3.3870371982741e-09\\
2.28899999999997	-3.36093566181777e-09\\
2.29099999999995	-3.30920622154707e-09\\
2.291	-3.30920622154588e-09\\
2.29499999999995	-3.20761844890866e-09\\
2.29699999999997	-3.15774691419902e-09\\
2.297	-3.15774691419832e-09\\
2.29999999999997	-3.08407429483783e-09\\
2.3	-3.08407429483713e-09\\
2.30299999999998	-3.01174417490839e-09\\
2.30599999999995	-2.94073540409989e-09\\
2.306	-2.9407354040988e-09\\
2.30999999999997	-2.84807664154388e-09\\
2.31	-2.84807664154323e-09\\
2.31399999999997	-2.75768190511268e-09\\
2.31599999999997	-2.71331881627093e-09\\
2.316	-2.7133188162703e-09\\
2.31999999999997	-2.62624045260451e-09\\
2.32	-2.6262404526039e-09\\
2.32399999999997	-2.54132695820948e-09\\
2.32599999999997	-2.49966819902131e-09\\
2.326	-2.49966819902072e-09\\
2.32999999999997	-2.41791958618761e-09\\
2.33199999999997	-2.37781910848398e-09\\
2.332	-2.37781910848341e-09\\
2.33599999999997	-2.2991397526214e-09\\
2.33999999999994	-2.22245522073022e-09\\
2.34	-2.22245522072901e-09\\
2.34000000000003	-2.22245522072847e-09\\
2.34499999999997	-2.12934414213662e-09\\
2.345	-2.1293441421361e-09\\
2.34600000000003	-2.11108179147541e-09\\
2.34600000000006	-2.1110817914749e-09\\
2.34700000000009	-2.09293800232195e-09\\
2.34800000000012	-2.07491218517344e-09\\
2.34899999999997	-2.057003754379e-09\\
2.349	-2.0570037543785e-09\\
2.35100000000006	-2.02153672827532e-09\\
2.35300000000011	-1.98653231472689e-09\\
2.35499999999997	-1.95198596455982e-09\\
2.355	-1.95198596455933e-09\\
2.35800000000003	-1.90101550322076e-09\\
2.35800000000006	-1.90101550322028e-09\\
2.35999999999997	-1.8676265459832e-09\\
2.36	-1.86762654598273e-09\\
2.36199999999992	-1.83474369481848e-09\\
2.36399999999983	-1.80236267627651e-09\\
2.36599999999997	-1.77047928211305e-09\\
2.366	-1.7704792821126e-09\\
2.36999999999983	-1.70818885687958e-09\\
2.37399999999966	-1.64784003704684e-09\\
2.37799999999997	-1.58940144962044e-09\\
2.378	-1.58940144962004e-09\\
2.37999999999997	-1.56088896112764e-09\\
2.38	-1.56088896112723e-09\\
2.38199999999997	-1.53284271534291e-09\\
2.38399999999995	-1.50525906718773e-09\\
2.38599999999997	-1.47813443188754e-09\\
2.386	-1.47813443188716e-09\\
2.38999999999995	-1.42524815864059e-09\\
2.39	-1.42524815863998e-09\\
2.39399999999995	-1.37415640234045e-09\\
2.396	-1.34927513186415e-09\\
2.39600000000002	-1.3492751318638e-09\\
2.39999999999997	-1.3010793662642e-09\\
2.4	-1.30107936626387e-09\\
2.40399999999995	-1.25512102414162e-09\\
2.40599999999997	-1.23297337562175e-09\\
2.406	-1.23297337562144e-09\\
2.40699999999997	-1.22210616095164e-09\\
2.407	-1.22210616095133e-09\\
2.40799999999998	-1.21137621375939e-09\\
2.40899999999997	-1.20078318542834e-09\\
2.41099999999995	-1.18000651312949e-09\\
2.41299999999997	-1.15977344391593e-09\\
2.413	-1.15977344391564e-09\\
2.41499999999997	-1.14008134830888e-09\\
2.415	-1.14008134830861e-09\\
2.41699999999997	-1.12092766713299e-09\\
2.41899999999995	-1.10230991117372e-09\\
2.41999999999997	-1.09320124632161e-09\\
2.42	-1.09320124632136e-09\\
2.42399999999995	-1.05809448628276e-09\\
2.426	-1.04133386879716e-09\\
2.42600000000003	-1.04133386879693e-09\\
2.427	-1.03315078647841e-09\\
2.42700000000003	-1.03315078647818e-09\\
2.42800000000002	-1.02506530123807e-09\\
2.429	-1.01704361908082e-09\\
2.43099999999998	-1.00119062359158e-09\\
2.43499999999993	-9.70238940092997e-10\\
2.43599999999997	-9.6265669708941e-10\\
2.436	-9.62656697089195e-10\\
2.43999999999997	-9.32943046162917e-10\\
2.44	-9.3294304616271e-10\\
2.44399999999998	-9.04202395852785e-10\\
2.446	-8.90192257260786e-10\\
2.44600000000003	-8.90192257260589e-10\\
2.448	-8.76419804998703e-10\\
2.44800000000002	-8.76419804998509e-10\\
2.44999999999999	-8.62883249205061e-10\\
2.45000000000002	-8.6288324920487e-10\\
2.45199999999998	-8.49580830667571e-10\\
2.45399999999995	-8.36510820606665e-10\\
2.45600000000002	-8.23671520441748e-10\\
2.45600000000005	-8.23671520441567e-10\\
2.45999999999998	-7.98552096371417e-10\\
2.46000000000001	-7.9855209637124e-10\\
2.46399999999994	-7.74083191121005e-10\\
2.46499999999997	-7.68066110374526e-10\\
2.465	-7.68066110374355e-10\\
2.46599999999998	-7.6208869492313e-10\\
2.46600000000001	-7.62088694922961e-10\\
2.46699999999999	-7.56150750564709e-10\\
2.46799999999998	-7.50252084375068e-10\\
2.46999999999996	-7.38571821188239e-10\\
2.47199999999998	-7.27046387394115e-10\\
2.47200000000001	-7.27046387393953e-10\\
2.47599999999996	-7.04454036556288e-10\\
2.47999999999991	-6.82463287086331e-10\\
2.47999999999997	-6.82463287085982e-10\\
2.48	-6.82463287085828e-10\\
2.48499999999997	-6.55803465411233e-10\\
2.485	-6.55803465411084e-10\\
2.48599999999997	-6.50580241856073e-10\\
2.486	-6.50580241855926e-10\\
2.48699999999999	-6.45392866521513e-10\\
2.48799999999998	-6.40241170866621e-10\\
2.48999999999995	-6.30044150241243e-10\\
2.49199999999997	-6.19987854773809e-10\\
2.492	-6.19987854773667e-10\\
2.494	-6.10070654587919e-10\\
2.49400000000002	-6.1007065458778e-10\\
2.49600000000002	-6.00290937883608e-10\\
2.49800000000001	-5.90647433687987e-10\\
2.49999999999997	-5.81138888730354e-10\\
2.5	-5.8113888873022e-10\\
2.50399999999999	-5.62521751006095e-10\\
2.50599999999997	-5.53410738754673e-10\\
2.506	-5.53410738754545e-10\\
2.50999999999999	-5.35577907002691e-10\\
2.51399999999998	-5.18256301522083e-10\\
2.51999999999997	-4.93212816166493e-10\\
2.52	-4.93212816166377e-10\\
2.52299999999997	-4.81105640615888e-10\\
2.523	-4.81105640615775e-10\\
2.52599999999996	-4.69270096043635e-10\\
2.526	-4.69270096043491e-10\\
2.52899999999997	-4.57702721498261e-10\\
2.53199999999993	-4.46400134523169e-10\\
2.53199999999997	-4.46400134523049e-10\\
2.532	-4.46400134522926e-10\\
2.53799999999993	-4.24576179627388e-10\\
2.53799999999997	-4.24576179627271e-10\\
2.538	-4.24576179627156e-10\\
2.53999999999997	-4.17552371054245e-10\\
2.54	-4.17552371054146e-10\\
2.54199999999998	-4.10686714594308e-10\\
2.54399999999995	-4.0397831797306e-10\\
2.54599999999997	-3.97426309366355e-10\\
2.546	-3.97426309366263e-10\\
2.54999999999995	-3.84788070429591e-10\\
2.55199999999997	-3.7870019763236e-10\\
2.552	-3.78700197632274e-10\\
2.55499999999997	-3.69855214533069e-10\\
2.555	-3.69855214532987e-10\\
2.55799999999998	-3.61352131178514e-10\\
2.55800000000001	-3.61352131178435e-10\\
2.55999999999997	-3.55872121291957e-10\\
2.56	-3.5587212129188e-10\\
2.56199999999997	-3.50542247550265e-10\\
2.56399999999994	-3.45361817283884e-10\\
2.56599999999997	-3.40330157243336e-10\\
2.566	-3.40330157243266e-10\\
2.56999999999994	-3.30710551443107e-10\\
2.56999999999997	-3.3071055144303e-10\\
2.57	-3.30710551442952e-10\\
2.57399999999994	-3.21530185498483e-10\\
2.57799999999988	-3.12636043101998e-10\\
2.57999999999997	-3.08294855060593e-10\\
2.58	-3.08294855060532e-10\\
2.58099999999997	-3.06150483512964e-10\\
2.581	-3.06150483512903e-10\\
2.58199999999998	-3.04023500539571e-10\\
2.58299999999997	-3.01913837024768e-10\\
2.58499999999995	-2.97746194761989e-10\\
2.58599999999997	-2.95688080607698e-10\\
2.586	-2.9568808060764e-10\\
2.58999999999995	-2.87625450048361e-10\\
2.59	-2.87625450048267e-10\\
2.59199999999997	-2.83695115487463e-10\\
2.592	-2.83695115487407e-10\\
2.59399999999997	-2.79831417451586e-10\\
2.59599999999995	-2.76033853814125e-10\\
2.59799999999997	-2.7230193104319e-10\\
2.598	-2.72301931043137e-10\\
2.59999999999997	-2.686234491143e-10\\
2.6	-2.68623449114248e-10\\
2.60199999999997	-2.6498621494782e-10\\
2.60399999999995	-2.61389755848473e-10\\
2.60599999999997	-2.57833604419946e-10\\
2.606	-2.57833604419896e-10\\
2.60999999999995	-2.50840381132278e-10\\
2.61	-2.50840381132191e-10\\
2.61399999999994	-2.44002909613744e-10\\
2.61599999999997	-2.40641466873175e-10\\
2.616	-2.40641466873127e-10\\
2.61999999999994	-2.3403098311458e-10\\
2.61999999999997	-2.3403098311453e-10\\
2.62	-2.34030983114479e-10\\
2.62399999999995	-2.27567512656683e-10\\
2.62499999999997	-2.25974220456848e-10\\
2.625	-2.25974220456803e-10\\
2.62599999999997	-2.24389854453506e-10\\
2.626	-2.24389854453461e-10\\
2.62699999999999	-2.22814363171672e-10\\
2.62799999999998	-2.21247695423551e-10\\
2.62999999999995	-2.18140627211147e-10\\
2.63199999999997	-2.15068246021307e-10\\
2.632	-2.15068246021263e-10\\
2.63599999999995	-2.09020651778157e-10\\
2.63899999999997	-2.04566114088466e-10\\
2.639	-2.04566114088424e-10\\
2.63999999999997	-2.03096468565584e-10\\
2.64	-2.03096468565542e-10\\
2.64099999999999	-2.01634343444653e-10\\
2.64199999999998	-2.00179691221145e-10\\
2.64399999999995	-1.97292616662743e-10\\
2.64599999999997	-1.94434869685796e-10\\
2.646	-1.94434869685756e-10\\
2.64999999999995	-1.8880587667484e-10\\
2.65099999999997	-1.87416382405251e-10\\
2.651	-1.87416382405212e-10\\
2.65499999999995	-1.81928066897559e-10\\
2.6589999999999	-1.76549078175463e-10\\
2.65999999999997	-1.75221084379237e-10\\
2.66	-1.75221084379199e-10\\
2.66599999999997	-1.67390535998993e-10\\
2.666	-1.67390535998956e-10\\
2.66799999999997	-1.64831685153533e-10\\
2.668	-1.64831685153497e-10\\
2.66999999999996	-1.6229794327925e-10\\
2.67199999999993	-1.5978898108633e-10\\
2.67199999999996	-1.59788981086285e-10\\
2.672	-1.5978898108624e-10\\
2.67599999999993	-1.54869252357506e-10\\
2.67799999999997	-1.52464135672581e-10\\
2.678	-1.52464135672547e-10\\
2.67999999999997	-1.50095099106683e-10\\
2.68	-1.5009509910665e-10\\
2.68199999999998	-1.47761834780912e-10\\
2.68399999999995	-1.45464039464016e-10\\
2.68599999999997	-1.43201414534206e-10\\
2.686	-1.43201414534174e-10\\
2.68999999999995	-1.38780504170762e-10\\
2.69399999999989	-1.34496805452586e-10\\
2.69499999999997	-1.33447057571004e-10\\
2.695	-1.33447057570974e-10\\
2.69699999999997	-1.31372734264259e-10\\
2.697	-1.3137273426423e-10\\
2.69899999999996	-1.29331750056121e-10\\
2.69999999999997	-1.28323677051056e-10\\
2.7	-1.28323677051027e-10\\
2.70199999999997	-1.26332205478112e-10\\
2.70399999999993	-1.24373417923541e-10\\
2.70599999999997	-1.22447059823027e-10\\
2.706	-1.22447059822999e-10\\
2.70899999999997	-1.19617781406167e-10\\
2.709	-1.1961778140614e-10\\
2.71199999999996	-1.16860079637936e-10\\
2.71299999999997	-1.15956611662643e-10\\
2.713	-1.15956611662617e-10\\
2.71599999999997	-1.13283232221075e-10\\
2.71899999999993	-1.10659852310813e-10\\
2.71999999999997	-1.0979637034201e-10\\
2.72	-1.09796370341985e-10\\
2.72599999999993	-1.04728792984089e-10\\
2.726	-1.04728792984037e-10\\
2.72600000000002	-1.04728792984013e-10\\
2.72999999999997	-1.01456323809824e-10\\
2.73	-1.01456323809801e-10\\
2.73199999999999	-9.98512219024294e-11\\
2.73200000000002	-9.98512219024067e-11\\
2.73400000000002	-9.8266596087112e-11\\
2.73600000000001	-9.67022404257558e-11\\
2.73799999999999	-9.51579516145365e-11\\
2.73800000000002	-9.51579516145147e-11\\
2.73999999999997	-9.36345920164655e-11\\
2.74	-9.3634592016444e-11\\
2.74199999999995	-9.21330267144909e-11\\
2.7439999999999	-9.0653060565046e-11\\
2.74599999999997	-8.91945012314817e-11\\
2.746	-8.91945012314611e-11\\
2.7499999999999	-8.63408475555862e-11\\
2.7539999999998	-8.35705821969859e-11\\
2.75499999999997	-8.2890875570351e-11\\
2.755	-8.28908755703318e-11\\
2.75999999999997	-7.95683986044432e-11\\
2.76	-7.95683986044247e-11\\
2.76499999999998	-7.63707178432406e-11\\
2.76500000000001	-7.63707178432228e-11\\
2.76599999999997	-7.57459268971584e-11\\
2.766	-7.57459268971407e-11\\
2.76699999999999	-7.51260032841652e-11\\
2.76700000000002	-7.51260032841477e-11\\
2.76800000000001	-7.45109268625562e-11\\
2.76899999999999	-7.39006776487074e-11\\
2.77099999999997	-7.26945816932022e-11\\
2.77299999999999	-7.15075586732612e-11\\
2.77300000000002	-7.15075586732445e-11\\
2.77699999999997	-6.91885673252657e-11\\
2.77999999999997	-6.74962265911433e-11\\
2.78	-6.74962265911275e-11\\
2.78399999999995	-6.53013684486913e-11\\
2.784	-6.53013684486687e-11\\
2.786	-6.42300161952202e-11\\
2.78600000000003	-6.42300161952051e-11\\
2.78800000000004	-6.31758620927788e-11\\
2.79000000000004	-6.21387691433918e-11\\
2.792	-6.11186025663693e-11\\
2.79200000000003	-6.1118602566355e-11\\
2.79600000000004	-5.91285203907031e-11\\
2.79999999999997	-5.72045810091602e-11\\
2.8	-5.72045810091468e-11\\
2.80400000000001	-5.53457842499754e-11\\
2.80599999999997	-5.44405110736612e-11\\
2.806	-5.44405110736484e-11\\
2.81000000000001	-5.2677626867633e-11\\
2.81199999999997	-5.18197867333533e-11\\
2.812	-5.18197867333412e-11\\
2.81299999999997	-5.13967496507354e-11\\
2.813	-5.13967496507234e-11\\
2.81399999999998	-5.09776583245847e-11\\
2.81499999999997	-5.05624991386017e-11\\
2.81699999999995	-4.97439233608487e-11\\
2.81899999999997	-4.89409159352597e-11\\
2.819	-4.89409159352484e-11\\
2.81999999999997	-4.85452176638304e-11\\
2.82	-4.85452176638192e-11\\
2.82099999999999	-4.81533725034787e-11\\
2.82199999999998	-4.77653677231287e-11\\
2.82399999999995	-4.70008290021468e-11\\
2.82599999999997	-4.62515021413308e-11\\
2.826	-4.62515021413202e-11\\
2.82999999999995	-4.47980964361652e-11\\
2.83399999999991	-4.34043950455514e-11\\
2.83499999999997	-4.30652129135926e-11\\
2.835	-4.3065212913583e-11\\
2.83999999999997	-4.14242120398079e-11\\
2.84	-4.14242120397988e-11\\
2.84199999999997	-4.0793237764968e-11\\
2.842	-4.07932377649591e-11\\
2.84399999999996	-4.01766686137662e-11\\
2.84599999999992	-3.95744244564068e-11\\
2.846	-3.9574424456384e-11\\
2.84600000000003	-3.95744244563755e-11\\
2.847	-3.92786497192185e-11\\
2.84700000000003	-3.92786497192102e-11\\
2.84800000000002	-3.89856909568888e-11\\
2.84900000000001	-3.86948025828463e-11\\
2.85099999999998	-3.81191992630671e-11\\
2.853	-3.75517649543981e-11\\
2.85300000000003	-3.75517649543901e-11\\
2.85699999999998	-3.64411094487245e-11\\
2.85999999999997	-3.5629020632153e-11\\
2.86	-3.56290206321454e-11\\
2.86399999999995	-3.45736544750587e-11\\
2.86599999999997	-3.40575677819514e-11\\
2.866	-3.40575677819441e-11\\
2.86999999999995	-3.30482518059052e-11\\
2.87	-3.30482518058932e-11\\
2.87099999999997	-3.28006368600938e-11\\
2.871	-3.28006368600867e-11\\
2.87199999999998	-3.25548913521842e-11\\
2.87299999999997	-3.2311007297888e-11\\
2.87499999999995	-3.18287919154679e-11\\
2.87699999999997	-3.13539280440591e-11\\
2.87699999999999	-3.13539280440524e-11\\
2.87999999999997	-3.06552815821347e-11\\
2.88	-3.06552815821281e-11\\
2.88299999999998	-2.99728330940074e-11\\
2.88599999999996	-2.93063830224564e-11\\
2.886	-2.93063830224472e-11\\
2.888	-2.88708745557116e-11\\
2.88800000000003	-2.88708745557054e-11\\
2.89000000000003	-2.84416967299812e-11\\
2.89200000000003	-2.8018157307795e-11\\
2.89600000000002	-2.7187774227244e-11\\
2.89999999999997	-2.63792936304578e-11\\
2.9	-2.63792936304521e-11\\
2.90499999999997	-2.53988540351364e-11\\
2.905	-2.53988540351309e-11\\
2.90599999999997	-2.52067235697621e-11\\
2.90599999999999	-2.52067235697567e-11\\
2.90699999999998	-2.5015897590732e-11\\
2.90799999999997	-2.48263698979974e-11\\
2.90999999999995	-2.44511847825973e-11\\
2.91199999999997	-2.40811194644551e-11\\
2.91199999999999	-2.40811194644499e-11\\
2.91599999999995	-2.33561565051707e-11\\
2.91799999999997	-2.30011646477571e-11\\
2.91799999999999	-2.30011646477521e-11\\
2.91999999999997	-2.26517650986021e-11\\
2.92	-2.26517650985972e-11\\
2.92199999999998	-2.23085734051255e-11\\
2.92399999999996	-2.19715449661014e-11\\
2.926	-2.16406359812663e-11\\
2.92600000000003	-2.16406359812616e-11\\
2.92899999999997	-2.11556526238781e-11\\
2.92899999999999	-2.11556526238736e-11\\
2.93199999999993	-2.06841996471354e-11\\
2.93499999999987	-2.02261391917741e-11\\
2.93499999999997	-2.02261391917593e-11\\
2.935	-2.0226139191755e-11\\
2.93999999999997	-1.94921042181311e-11\\
2.94	-1.9492104218127e-11\\
2.94499999999998	-1.87943039713519e-11\\
2.94599999999997	-1.86590416961837e-11\\
2.946	-1.86590416961799e-11\\
2.95099999999998	-1.80039774769861e-11\\
2.95199999999997	-1.78771842684517e-11\\
2.952	-1.78771842684481e-11\\
2.95699999999998	-1.72596680839597e-11\\
2.95799999999999	-1.71392498725293e-11\\
2.95800000000002	-1.71392498725259e-11\\
2.95999999999997	-1.69014669763732e-11\\
2.96	-1.69014669763699e-11\\
2.96199999999995	-1.66677297525617e-11\\
2.9639999999999	-1.64380078244331e-11\\
2.96599999999997	-1.62122713372871e-11\\
2.966	-1.6212271337284e-11\\
2.9699999999999	-1.57726378537941e-11\\
2.96999999999999	-1.57726378537844e-11\\
2.97000000000002	-1.57726378537813e-11\\
2.97399999999992	-1.53486007558365e-11\\
2.97499999999997	-1.52450024074328e-11\\
2.975	-1.52450024074299e-11\\
2.9789999999999	-1.48401517392571e-11\\
2.97999999999997	-1.47413083116871e-11\\
2.98	-1.47413083116843e-11\\
2.9839999999999	-1.43553152151184e-11\\
2.986	-1.41679032482607e-11\\
2.98600000000003	-1.41679032482581e-11\\
2.98699999999997	-1.40755826753259e-11\\
2.98699999999999	-1.40755826753233e-11\\
2.98799999999998	-1.39839560690341e-11\\
2.98899999999997	-1.38927948213271e-11\\
2.99099999999995	-1.37118565694808e-11\\
2.99299999999999	-1.35327444050262e-11\\
2.99300000000002	-1.35327444050237e-11\\
2.99699999999997	-1.31799054633445e-11\\
2.99999999999997	-1.29198983070989e-11\\
3	-1.29198983070965e-11\\
3.00399999999995	-1.25792402797173e-11\\
3.00599999999997	-1.24114406179577e-11\\
3.006	-1.24114406179553e-11\\
3.00999999999995	-1.20807909596136e-11\\
3.01	-1.20807909596096e-11\\
3.01399999999995	-1.17565979689463e-11\\
3.01599999999997	-1.15968699284531e-11\\
3.01599999999999	-1.15968699284508e-11\\
3.01999999999995	-1.12820156488957e-11\\
3.02	-1.12820156488914e-11\\
3.02399999999995	-1.09731401595441e-11\\
3.02599999999997	-1.08208941636547e-11\\
3.026	-1.08208941636525e-11\\
3.02799999999997	-1.06700828894996e-11\\
3.02799999999999	-1.06700828894975e-11\\
3.02999999999996	-1.05206867377158e-11\\
3.03199999999992	-1.03726862927663e-11\\
3.03599999999985	-1.0080795765813e-11\\
3.03999999999997	-9.79425956619819e-12\\
3.04	-9.79425956619617e-12\\
3.04499999999997	-9.44339217328454e-12\\
3.04499999999999	-9.44339217328256e-12\\
3.04599999999997	-9.37416953136604e-12\\
3.046	-9.37416953136408e-12\\
3.04699999999998	-9.30525856340003e-12\\
3.04799999999996	-9.23665703042779e-12\\
3.04999999999992	-9.10037336407018e-12\\
3.05199999999997	-8.96530082084684e-12\\
3.052	-8.96530082084492e-12\\
3.05599999999992	-8.69871904313831e-12\\
3.05799999999997	-8.56717516364819e-12\\
3.058	-8.56717516364633e-12\\
3.05999999999997	-8.43721175756052e-12\\
3.06	-8.43721175755869e-12\\
3.06199999999997	-8.30925057944518e-12\\
3.06399999999995	-8.18327499945535e-12\\
3.066	-8.05926864578576e-12\\
3.06600000000003	-8.05926864578401e-12\\
3.06999999999997	-7.81709940807259e-12\\
3.07399999999992	-7.58261697310577e-12\\
3.07399999999996	-7.58261697310359e-12\\
3.07399999999999	-7.58261697310142e-12\\
3.07999999999997	-7.24504050239408e-12\\
3.07999999999999	-7.24504050239252e-12\\
3.08599999999997	-6.92409111493065e-12\\
3.08599999999999	-6.92409111492917e-12\\
3.09199999999997	-6.61939338301488e-12\\
3.09199999999999	-6.61939338301347e-12\\
3.09799999999997	-6.32963046213013e-12\\
3.09999999999997	-6.23608919357034e-12\\
3.1	-6.23608919356902e-12\\
3.10299999999997	-6.09859122149346e-12\\
3.10299999999999	-6.09859122149217e-12\\
3.10599999999996	-5.96443430053832e-12\\
3.106	-5.96443430053636e-12\\
3.10600000000003	-5.96443430053511e-12\\
3.10899999999999	-5.83357920119237e-12\\
3.11199999999996	-5.70598765965977e-12\\
3.11199999999999	-5.70598765965822e-12\\
3.11200000000003	-5.7059876596567e-12\\
3.11499999999997	-5.58162236625439e-12\\
3.115	-5.58162236625322e-12\\
3.11799999999995	-5.46044695526774e-12\\
3.11999999999997	-5.38141796871775e-12\\
3.12	-5.38141796871664e-12\\
3.12299999999995	-5.26548074329917e-12\\
3.12599999999989	-5.15264095601907e-12\\
3.12599999999995	-5.15264095601701e-12\\
3.126	-5.15264095601493e-12\\
3.127	-5.11571028420988e-12\\
3.12700000000003	-5.11571028420883e-12\\
3.12800000000003	-5.07905060955641e-12\\
3.12900000000003	-5.04259245166776e-12\\
3.13100000000003	-4.97027595463383e-12\\
3.13199999999997	-4.93441526593593e-12\\
3.13199999999999	-4.93441526593492e-12\\
3.13599999999999	-4.79292913192514e-12\\
3.13799999999997	-4.72334397046571e-12\\
3.13799999999999	-4.72334397046473e-12\\
3.13999999999997	-4.65451865504145e-12\\
3.14	-4.65451865504048e-12\\
3.14199999999998	-4.58644424114856e-12\\
3.14399999999996	-4.51911188183016e-12\\
3.14599999999997	-4.45251282656451e-12\\
3.146	-4.45251282656357e-12\\
3.14999999999996	-4.32148010166643e-12\\
3.15	-4.32148010166511e-12\\
3.15399999999996	-4.19327794827004e-12\\
3.15599999999997	-4.13021745225328e-12\\
3.156	-4.13021745225239e-12\\
3.15999999999996	-4.00619030931576e-12\\
3.16	-4.00619030931452e-12\\
3.16099999999997	-3.97561944725066e-12\\
3.16099999999999	-3.9756194472498e-12\\
3.16199999999996	-3.94522095972702e-12\\
3.16299999999992	-3.91499385909558e-12\\
3.16499999999985	-3.85504989576685e-12\\
3.16599999999997	-3.82533108549228e-12\\
3.166	-3.82533108549143e-12\\
3.16999999999986	-3.70812118741818e-12\\
3.17199999999997	-3.65050213452436e-12\\
3.172	-3.65050213452355e-12\\
3.17599999999986	-3.53719837933177e-12\\
3.17999999999972	-3.42642481273379e-12\\
3.17999999999997	-3.42642481272685e-12\\
3.18	-3.42642481272607e-12\\
3.18499999999997	-3.29142832544425e-12\\
3.185	-3.29142832544349e-12\\
3.18599999999997	-3.2648829482323e-12\\
3.186	-3.26488294823155e-12\\
3.18699999999997	-3.23848685396699e-12\\
3.18799999999994	-3.21223918502321e-12\\
3.18999999999989	-3.16018571677825e-12\\
3.18999999999994	-3.16018571677686e-12\\
3.18999999999999	-3.16018571677547e-12\\
3.19399999999988	-3.05782268157658e-12\\
3.19599999999999	-3.00749981152321e-12\\
3.19600000000002	-3.0074998115225e-12\\
3.198	-2.95774062854675e-12\\
3.19800000000003	-2.95774062854604e-12\\
3.19999999999998	-2.90860270417661e-12\\
3.20000000000001	-2.90860270417592e-12\\
3.20199999999996	-2.86014369065341e-12\\
3.20399999999991	-2.81235729025376e-12\\
3.20599999999998	-2.76523729264911e-12\\
3.20600000000001	-2.76523729264844e-12\\
3.20999999999991	-2.67297209689359e-12\\
3.21399999999982	-2.58330013874443e-12\\
3.21899999999997	-2.4747860401545e-12\\
3.21899999999999	-2.4747860401539e-12\\
3.21999999999997	-2.4535529106153e-12\\
3.22	-2.4535529106147e-12\\
3.22099999999998	-2.43247472308926e-12\\
3.22199999999996	-2.41155079264742e-12\\
3.22399999999992	-2.37016298875636e-12\\
3.22599999999997	-2.32938412017674e-12\\
3.226	-2.32938412017617e-12\\
3.22999999999992	-2.24963206901337e-12\\
3.23099999999999	-2.23006645497329e-12\\
3.23100000000002	-2.23006645497274e-12\\
3.23499999999994	-2.15327452291503e-12\\
3.23699999999999	-2.11575215562313e-12\\
3.23700000000002	-2.1157521556226e-12\\
3.23999999999997	-2.06083010559489e-12\\
3.24	-2.06083010559438e-12\\
3.24299999999996	-2.00775421352683e-12\\
3.24599999999991	-1.95650895930596e-12\\
3.24599999999997	-1.95650895930485e-12\\
3.246	-1.95650895930437e-12\\
3.24799999999997	-1.92335505564688e-12\\
3.24799999999999	-1.92335505564641e-12\\
3.24999999999996	-1.89100378936893e-12\\
3.25199999999992	-1.85945095603547e-12\\
3.25399999999997	-1.82869245503542e-12\\
3.25399999999999	-1.82869245503499e-12\\
3.25499999999998	-1.81360982596709e-12\\
3.255	-1.81360982596667e-12\\
3.25599999999998	-1.7987242890141e-12\\
3.25699999999997	-1.78403536054407e-12\\
3.25899999999993	-1.75524542646907e-12\\
3.25999999999997	-1.74114348548134e-12\\
3.26	-1.74114348548094e-12\\
3.26399999999993	-1.68667861169982e-12\\
3.26599999999997	-1.66060573749454e-12\\
3.266	-1.66060573749418e-12\\
3.267	-1.64785769218243e-12\\
3.26700000000003	-1.64785769218207e-12\\
3.26800000000003	-1.63525105221471e-12\\
3.26900000000003	-1.62273510646564e-12\\
3.27100000000003	-1.59797367399911e-12\\
3.27500000000003	-1.54952144336905e-12\\
3.27699999999997	-1.52582434820566e-12\\
3.27699999999999	-1.52582434820533e-12\\
3.27999999999997	-1.49093130312942e-12\\
3.28	-1.49093130312909e-12\\
3.28299999999998	-1.45681232287604e-12\\
3.28599999999996	-1.42345743036958e-12\\
3.286	-1.42345743036915e-12\\
3.28899999999999	-1.39085687213843e-12\\
3.28900000000002	-1.39085687213813e-12\\
3.29	-1.38015599530777e-12\\
3.29000000000003	-1.38015599530747e-12\\
3.29100000000001	-1.36953752473796e-12\\
3.29199999999999	-1.35900111543714e-12\\
3.29399999999996	-1.33817311399801e-12\\
3.296	-1.3176692841802e-12\\
3.29600000000003	-1.31766928417991e-12\\
3.29999999999996	-1.27747873705213e-12\\
3.3	-1.27747873705169e-12\\
3.30000000000003	-1.27747873705141e-12\\
3.30399999999996	-1.23826379526547e-12\\
3.30599999999997	-1.21901578618486e-12\\
3.30599999999999	-1.21901578618459e-12\\
3.30999999999992	-1.1812261840096e-12\\
3.31199999999999	-1.16267967977326e-12\\
3.31200000000002	-1.162679679773e-12\\
3.31599999999995	-1.1262712132146e-12\\
3.31999999999988	-1.09075974584147e-12\\
3.32	-1.09075974584036e-12\\
3.32000000000003	-1.09075974584011e-12\\
3.32499999999998	-1.04760373626409e-12\\
3.325	-1.04760373626384e-12\\
3.326	-1.03913416014032e-12\\
3.32600000000003	-1.03913416014008e-12\\
3.32700000000003	-1.03071781314122e-12\\
3.32800000000003	-1.02235442181453e-12\\
3.33000000000002	-1.00578542099349e-12\\
3.332	-9.89425004368199e-13\\
3.33200000000003	-9.89425004367968e-13\\
3.33499999999999	-9.65289783590867e-13\\
3.33500000000002	-9.6528978359064e-13\\
3.33799999999998	-9.41649947129182e-13\\
3.33999999999997	-9.2616185012817e-13\\
3.34	-9.26161850127951e-13\\
3.34299999999997	-9.0333236345637e-13\\
3.34599999999993	-8.80980143910773e-13\\
3.34599999999997	-8.80980143910498e-13\\
3.346	-8.80980143910226e-13\\
3.34699999999999	-8.73634329731536e-13\\
3.34700000000002	-8.73634329731328e-13\\
3.34800000000001	-8.66340579218316e-13\\
3.349	-8.59098655385848e-13\\
3.35099999999999	-8.44769348287355e-13\\
3.35499999999995	-8.16722413472853e-13\\
3.35999999999997	-7.82791950518157e-13\\
3.36	-7.82791950517967e-13\\
3.36399999999997	-7.56532946631132e-13\\
3.36399999999999	-7.56532946630948e-13\\
3.36599999999997	-7.43693848316785e-13\\
3.366	-7.43693848316604e-13\\
3.36799999999998	-7.31046119266725e-13\\
3.36999999999996	-7.18588115780895e-13\\
3.37199999999997	-7.06318218815622e-13\\
3.372	-7.06318218815449e-13\\
3.37599999999996	-6.82336390330809e-13\\
3.37799999999997	-6.70621342130165e-13\\
3.378	-6.7062134213e-13\\
3.37999999999997	-6.59129005982098e-13\\
3.38	-6.59129005981936e-13\\
3.38199999999997	-6.47898727628992e-13\\
3.38399999999995	-6.36929047582701e-13\\
3.38599999999997	-6.26218540222072e-13\\
3.386	-6.26218540221921e-13\\
3.38999999999995	-6.05569509334561e-13\\
3.39299999999997	-5.9075295514787e-13\\
3.39299999999999	-5.90752955147732e-13\\
3.39499999999998	-5.81191981778849e-13\\
3.395	-5.81191981778715e-13\\
3.39699999999999	-5.7188293752358e-13\\
3.39899999999997	-5.62824612577657e-13\\
3.399	-5.6282461257753e-13\\
3.39999999999997	-5.58389100694485e-13\\
3.4	-5.5838910069436e-13\\
3.40099999999998	-5.54015829720831e-13\\
3.40199999999996	-5.49704657568737e-13\\
3.40399999999991	-5.41268051467328e-13\\
3.40599999999997	-5.33078185967531e-13\\
3.406	-5.33078185967417e-13\\
3.40999999999991	-5.17434451302054e-13\\
3.41099999999997	-5.1367610608931e-13\\
3.411	-5.13676106089204e-13\\
3.41499999999991	-4.99019454729664e-13\\
3.41899999999982	-4.84867912648881e-13\\
3.41999999999997	-4.81408082600115e-13\\
3.42	-4.81408082600017e-13\\
3.42199999999997	-4.74581298670417e-13\\
3.42199999999999	-4.74581298670321e-13\\
3.42399999999996	-4.67877612026575e-13\\
3.42599999999992	-4.61296151412028e-13\\
3.426	-4.61296151411756e-13\\
3.42600000000003	-4.61296151411663e-13\\
3.42999999999996	-4.48496502742112e-13\\
3.43	-4.48496502741959e-13\\
3.432	-4.42276651245099e-13\\
3.43200000000003	-4.42276651245011e-13\\
3.43400000000003	-4.36175698683289e-13\\
3.43600000000003	-4.30192852176216e-13\\
3.438	-4.24327334192544e-13\\
3.43800000000003	-4.24327334192462e-13\\
3.43999999999997	-4.18554197259641e-13\\
3.44	-4.18554197259559e-13\\
3.44199999999995	-4.12848505910518e-13\\
3.44399999999989	-4.07209518633014e-13\\
3.44599999999997	-4.01636502583354e-13\\
3.446	-4.01636502583275e-13\\
3.44999999999989	-3.90685495584634e-13\\
3.45099999999996	-3.87987854428571e-13\\
3.45099999999999	-3.87987854428495e-13\\
3.45499999999988	-3.77355102747119e-13\\
3.45699999999996	-3.7213220735603e-13\\
3.45699999999999	-3.72132207355956e-13\\
3.45999999999997	-3.64412802946371e-13\\
3.46	-3.64412802946298e-13\\
3.46299999999998	-3.56829321475419e-13\\
3.46499999999998	-3.51848083220725e-13\\
3.465	-3.51848083220654e-13\\
3.46599999999997	-3.49379545429496e-13\\
3.466	-3.49379545429426e-13\\
3.46699999999997	-3.46925621434294e-13\\
3.46799999999994	-3.44486231507651e-13\\
3.46999999999988	-3.39650737309398e-13\\
3.47199999999997	-3.34872434413023e-13\\
3.472	-3.34872434412955e-13\\
3.47599999999988	-3.25469660448457e-13\\
3.47999999999976	-3.16257756033951e-13\\
3.47999999999996	-3.16257756033484e-13\\
3.47999999999999	-3.16257756033419e-13\\
3.48599999999996	-3.0278733067587e-13\\
3.48599999999999	-3.02787330675808e-13\\
3.49199999999996	-2.89719852659886e-13\\
3.49199999999999	-2.89719852659824e-13\\
3.49799999999996	-2.77040036420389e-13\\
3.5	-2.72897001763409e-13\\
3.50000000000003	-2.72897001763351e-13\\
3.506	-2.60711071630858e-13\\
3.50600000000003	-2.60711071630801e-13\\
3.50899999999999	-2.54751629494733e-13\\
3.50900000000002	-2.54751629494677e-13\\
3.51199999999998	-2.48878870545927e-13\\
3.51200000000003	-2.48878870545837e-13\\
3.51499999999999	-2.43112755320934e-13\\
3.51799999999995	-2.37473275549401e-13\\
3.51799999999999	-2.37473275549331e-13\\
3.51800000000003	-2.37473275549262e-13\\
3.51999999999997	-2.33783159530413e-13\\
3.52	-2.33783159530361e-13\\
3.52199999999995	-2.3014811270856e-13\\
3.52399999999989	-2.26567662666115e-13\\
3.52599999999997	-2.23041344087154e-13\\
3.526	-2.23041344087105e-13\\
3.52999999999989	-2.16149275182676e-13\\
3.53399999999978	-2.09468323109701e-13\\
3.53499999999998	-2.07830662694518e-13\\
3.535	-2.07830662694472e-13\\
3.53799999999997	-2.02995014695301e-13\\
3.53799999999999	-2.02995014695256e-13\\
3.53999999999997	-1.99835171104662e-13\\
3.54	-1.99835171104618e-13\\
3.54199999999998	-1.96725984829474e-13\\
3.54399999999996	-1.93667051800383e-13\\
3.54599999999997	-1.90657974478074e-13\\
3.546	-1.90657974478032e-13\\
3.54999999999996	-1.84787829150406e-13\\
3.54999999999999	-1.84787829150363e-13\\
3.553	-1.80513229506385e-13\\
3.55300000000003	-1.80513229506345e-13\\
3.55600000000004	-1.76332123334914e-13\\
3.55900000000005	-1.72228450990501e-13\\
3.55999999999997	-1.70877559676222e-13\\
3.56	-1.70877559676183e-13\\
3.56599999999999	-1.62947648196046e-13\\
3.56600000000002	-1.6294764819601e-13\\
3.56699999999996	-1.61654834909373e-13\\
3.56699999999999	-1.61654834909337e-13\\
3.56799999999996	-1.60370148336711e-13\\
3.56899999999992	-1.59093546736702e-13\\
3.56999999999998	-1.57824988632788e-13\\
3.57	-1.57824988632752e-13\\
3.57199999999993	-1.5531183831365e-13\\
3.57299999999996	-1.54067164446752e-13\\
3.57299999999999	-1.54067164446717e-13\\
3.57499999999992	-1.51601417102002e-13\\
3.57699999999985	-1.49166870327812e-13\\
3.57899999999996	-1.46763207730002e-13\\
3.57899999999999	-1.46763207729968e-13\\
3.57999999999997	-1.45573505076217e-13\\
3.58	-1.45573505076184e-13\\
3.58099999999998	-1.44392696363181e-13\\
3.58199999999997	-1.43220743226331e-13\\
3.58399999999993	-1.40903251662004e-13\\
3.58599999999997	-1.38620729201687e-13\\
3.586	-1.38620729201655e-13\\
3.58999999999993	-1.34159409558185e-13\\
3.59399999999986	-1.2983446505078e-13\\
3.59599999999996	-1.27722428112649e-13\\
3.59599999999999	-1.27722428112619e-13\\
3.59999999999997	-1.23597852469784e-13\\
3.6	-1.23597852469755e-13\\
3.60399999999998	-1.19604161493547e-13\\
3.60499999999998	-1.18625944697342e-13\\
3.605	-1.18625944697314e-13\\
3.60599999999997	-1.1765574632357e-13\\
3.606	-1.17655746323542e-13\\
3.60699999999997	-1.16693534851186e-13\\
3.60799999999994	-1.1573927901787e-13\\
3.60799999999999	-1.15739279017821e-13\\
3.60999999999993	-1.13854510511877e-13\\
3.61199999999987	-1.12001195861509e-13\\
3.61399999999996	-1.10179094209628e-13\\
3.61399999999999	-1.10179094209602e-13\\
3.61799999999987	-1.06623354111207e-13\\
3.61999999999997	-1.04888195439617e-13\\
3.62	-1.04888195439593e-13\\
3.62399999999988	-1.0150217332053e-13\\
3.62499999999996	-1.0067306853105e-13\\
3.62499999999999	-1.00673068531026e-13\\
3.62599999999997	-9.98508698248068e-14\\
3.626	-9.98508698247835e-14\\
3.62699999999998	-9.90355504890917e-14\\
3.62799999999997	-9.82270840341185e-14\\
3.62999999999993	-9.66306049208317e-14\\
3.63199999999997	-9.50612250062793e-14\\
3.632	-9.50612250062572e-14\\
3.63599999999993	-9.20029504460641e-14\\
3.63999999999985	-8.90506704902203e-14\\
3.63999999999997	-8.90506704901318e-14\\
3.64	-8.90506704901112e-14\\
3.64599999999997	-8.48176480144035e-14\\
3.646	-8.4817648014384e-14\\
3.65199999999997	-8.08147213859459e-14\\
3.652	-8.08147213859274e-14\\
3.65399999999996	-7.9529974896962e-14\\
3.65399999999999	-7.9529974896944e-14\\
3.65599999999995	-7.82685990709824e-14\\
3.65799999999992	-7.70304299771485e-14\\
3.65999999999996	-7.58153067028163e-14\\
3.65999999999999	-7.58153067027992e-14\\
3.66399999999992	-7.34535689204675e-14\\
3.66599999999996	-7.23066474806705e-14\\
3.66599999999999	-7.23066474806544e-14\\
3.66999999999992	-7.00799542175549e-14\\
3.67399999999984	-6.79418339817725e-14\\
3.67499999999998	-6.74210121966049e-14\\
3.67500000000001	-6.74210121965902e-14\\
3.67999999999997	-6.48983090529137e-14\\
3.68	-6.48983090528997e-14\\
3.68299999999996	-6.34492248225999e-14\\
3.68299999999999	-6.34492248225864e-14\\
3.68599999999995	-6.20480230617279e-14\\
3.686	-6.20480230617055e-14\\
3.68699999999998	-6.15915251380261e-14\\
3.68700000000001	-6.15915251380132e-14\\
3.68799999999998	-6.11391712039542e-14\\
3.68899999999995	-6.06898307491993e-14\\
3.6909999999999	-5.98001319793103e-14\\
3.69299999999998	-5.89223132029539e-14\\
3.693	-5.89223132029415e-14\\
3.6969999999999	-5.72018608368551e-14\\
3.69999999999997	-5.5941869149571e-14\\
3.7	-5.59418691495592e-14\\
3.7039999999999	-5.43016431837418e-14\\
3.70599999999997	-5.34983312421547e-14\\
3.706	-5.34983312421434e-14\\
3.7099999999999	-5.19247874378742e-14\\
3.70999999999998	-5.19247874378434e-14\\
3.71000000000001	-5.19247874378324e-14\\
3.71199999999996	-5.11543510771801e-14\\
3.71199999999999	-5.11543510771692e-14\\
3.71399999999995	-5.03946711065675e-14\\
3.71599999999991	-4.96456487979261e-14\\
3.71799999999996	-4.89071868081999e-14\\
3.71799999999999	-4.89071868081895e-14\\
3.71999999999997	-4.8179189167284e-14\\
3.72	-4.81791891672737e-14\\
3.72199999999998	-4.74615612649377e-14\\
3.72399999999997	-4.67542098381721e-14\\
3.72599999999997	-4.60570429595098e-14\\
3.726	-4.60570429595e-14\\
3.728	-4.53699700254876e-14\\
3.72800000000003	-4.53699700254779e-14\\
3.73000000000004	-4.46921711810974e-14\\
3.73200000000004	-4.4022827776191e-14\\
3.73600000000004	-4.27091604214052e-14\\
3.74	-4.14282850369943e-14\\
3.74000000000003	-4.14282850369853e-14\\
3.74099999999996	-4.11131115359076e-14\\
3.74099999999999	-4.11131115358987e-14\\
3.74199999999996	-4.07999355965395e-14\\
3.74299999999992	-4.04887470423596e-14\\
3.74499999999985	-3.98722917121025e-14\\
3.745	-3.98722917120556e-14\\
3.74500000000003	-3.98722917120468e-14\\
3.746	-3.95670049074038e-14\\
3.74600000000003	-3.95670049073951e-14\\
3.747	-3.92636654305224e-14\\
3.74799999999997	-3.89622634258905e-14\\
3.74999999999991	-3.83652327261511e-14\\
3.752	-3.77758352179644e-14\\
3.75200000000003	-3.77758352179561e-14\\
3.75599999999991	-3.66196343674435e-14\\
3.758	-3.60526807650458e-14\\
3.75800000000003	-3.60526807650378e-14\\
3.75999999999997	-3.54943227294292e-14\\
3.76	-3.54943227294214e-14\\
3.76199999999995	-3.49457506103997e-14\\
3.76399999999989	-3.44068931154712e-14\\
3.76599999999997	-3.38776802146275e-14\\
3.766	-3.38776802146201e-14\\
3.76999999999989	-3.28479143350018e-14\\
3.76999999999996	-3.28479143349834e-14\\
3.76999999999999	-3.28479143349762e-14\\
3.77399999999988	-3.18559176412169e-14\\
3.77599999999999	-3.13739208244165e-14\\
3.77600000000002	-3.13739208244097e-14\\
3.77999999999991	-3.04376170424985e-14\\
3.77999999999996	-3.04376170424878e-14\\
3.78	-3.0437617042477e-14\\
3.78399999999989	-2.95378294386278e-14\\
3.78599999999997	-2.91014822934737e-14\\
3.786	-2.91014822934676e-14\\
3.78999999999989	-2.82555977720232e-14\\
3.79199999999997	-2.78459504644292e-14\\
3.792	-2.78459504644235e-14\\
3.79599999999989	-2.70489670685353e-14\\
3.79899999999996	-2.64687929440349e-14\\
3.79899999999999	-2.64687929440295e-14\\
3.79999999999997	-2.62787132556788e-14\\
3.8	-2.62787132556734e-14\\
3.80099999999999	-2.60902790732e-14\\
3.80199999999997	-2.59034842743579e-14\\
3.80399999999993	-2.55347886050143e-14\\
3.80599999999997	-2.51725783319041e-14\\
3.806	-2.5172578331899e-14\\
3.80999999999993	-2.4467426520601e-14\\
3.81099999999996	-2.42951194009421e-14\\
3.81099999999999	-2.42951194009372e-14\\
3.81499999999992	-2.36216463045562e-14\\
3.81499999999998	-2.36216463045468e-14\\
3.81500000000001	-2.36216463045421e-14\\
3.81899999999993	-2.29731210776856e-14\\
3.81999999999997	-2.28148480844999e-14\\
3.82	-2.28148480844955e-14\\
3.82399999999993	-2.21970350947492e-14\\
3.826	-2.18972259084023e-14\\
3.82600000000003	-2.1897225908398e-14\\
3.82700000000001	-2.17495784671099e-14\\
3.82700000000003	-2.17495784671057e-14\\
3.82799999999998	-2.16030760573187e-14\\
3.82800000000001	-2.16030760573145e-14\\
3.82899999999997	-2.14573605949561e-14\\
3.82999999999993	-2.13124273457786e-14\\
3.83199999999986	-2.10248886768832e-14\\
3.83399999999998	-2.07404226820477e-14\\
3.83400000000001	-2.07404226820437e-14\\
3.83799999999985	-2.01805612330449e-14\\
3.84	-1.99050930191724e-14\\
3.84000000000003	-1.99050930191685e-14\\
3.84399999999988	-1.93629026094901e-14\\
3.846	-1.90961099505069e-14\\
3.84600000000003	-1.90961099505031e-14\\
3.84999999999988	-1.85709563585486e-14\\
3.84999999999998	-1.85709563585356e-14\\
3.85000000000001	-1.85709563585319e-14\\
3.85399999999985	-1.8056818170852e-14\\
3.85599999999998	-1.78037961093084e-14\\
3.85600000000001	-1.78037961093049e-14\\
3.85699999999996	-1.76782740755903e-14\\
3.85699999999999	-1.76782740755868e-14\\
3.85799999999995	-1.75533948383861e-14\\
3.85899999999992	-1.74291543403613e-14\\
3.85999999999997	-1.73055485449632e-14\\
3.86	-1.73055485449597e-14\\
3.86199999999993	-1.70602250189506e-14\\
3.86399999999985	-1.68173923781427e-14\\
3.86599999999997	-1.65770190639593e-14\\
3.866	-1.65770190639559e-14\\
3.86899999999996	-1.62210020817481e-14\\
3.86899999999999	-1.62210020817447e-14\\
3.87199999999995	-1.5870344267002e-14\\
3.87499999999991	-1.55249430826071e-14\\
3.87999999999997	-1.49606822050036e-14\\
3.88	-1.49606822050004e-14\\
3.88499999999998	-1.44102847970004e-14\\
3.88500000000001	-1.44102847969973e-14\\
3.88599999999999	-1.43018293941066e-14\\
3.88600000000002	-1.43018293941035e-14\\
3.88700000000001	-1.41939070881438e-14\\
3.88799999999999	-1.40865143726512e-14\\
3.88999999999996	-1.38733037735374e-14\\
3.89199999999999	-1.36621698878863e-14\\
3.89200000000002	-1.36621698878833e-14\\
3.89599999999996	-1.32460227679542e-14\\
3.89799999999999	-1.3040955451143e-14\\
3.89800000000002	-1.30409554511401e-14\\
3.89999999999997	-1.28385087942447e-14\\
3.9	-1.28385087942419e-14\\
3.90199999999996	-1.2639308605591e-14\\
3.90399999999991	-1.24433289971063e-14\\
3.90599999999997	-1.22505444992488e-14\\
3.906	-1.22505444992461e-14\\
3.90999999999991	-1.18744610307849e-14\\
3.91399999999982	-1.15108626908422e-14\\
3.91499999999996	-1.14218916148904e-14\\
3.91499999999999	-1.14218916148879e-14\\
3.91999999999997	-1.09884630669536e-14\\
3.92	-1.09884630669512e-14\\
3.92499999999999	-1.05738233173555e-14\\
3.92599999999997	-1.04931201182458e-14\\
3.926	-1.04931201182435e-14\\
3.92699999999996	-1.04131523286452e-14\\
3.92699999999999	-1.04131523286429e-14\\
3.92799999999995	-1.03339173505001e-14\\
3.92899999999992	-1.02554126094167e-14\\
3.93099999999984	-1.0100583659692e-14\\
3.93299999999996	-9.9486453578684e-15\\
3.93299999999999	-9.94864535786626e-15\\
3.93699999999984	-9.65271788783832e-15\\
3.93999999999997	-9.43737039441022e-15\\
3.94	-9.4373703944082e-15\\
3.94399999999985	-9.15891815838489e-15\\
3.94399999999996	-9.15891815837712e-15\\
3.94399999999999	-9.15891815837517e-15\\
3.94599999999997	-9.02337027679908e-15\\
3.946	-9.02337027679718e-15\\
3.94799999999999	-8.89025095924434e-15\\
3.94999999999997	-8.75954290551334e-15\\
3.95199999999997	-8.63122912877139e-15\\
3.952	-8.63122912876959e-15\\
3.95499999999998	-8.44321133972912e-15\\
3.95500000000001	-8.44321133972736e-15\\
3.95799999999998	-8.26048824731601e-15\\
3.95999999999998	-8.14158790171556e-15\\
3.96	-8.14158790171389e-15\\
3.96299999999998	-7.96757132623824e-15\\
3.96599999999995	-7.79871036380253e-15\\
3.966	-7.79871036379977e-15\\
3.96600000000003	-7.7987103637982e-15\\
3.96700000000001	-7.74356049052123e-15\\
3.96700000000003	-7.74356049051967e-15\\
3.96800000000001	-7.68884599908964e-15\\
3.96899999999998	-7.63443493804758e-15\\
3.97099999999993	-7.52651604573469e-15\\
3.97299999999996	-7.41978978842718e-15\\
3.97299999999999	-7.41978978842567e-15\\
3.97699999999989	-7.20985985170138e-15\\
3.97899999999996	-7.10662888975801e-15\\
3.97899999999999	-7.10662888975655e-15\\
3.97999999999997	-7.05544101644666e-15\\
3.98	-7.05544101644521e-15\\
3.98099999999999	-7.00453599441383e-15\\
3.98199999999997	-6.9539121697519e-15\\
3.98399999999994	-6.85350154260989e-15\\
3.98599999999997	-6.75419608564106e-15\\
3.986	-6.75419608563966e-15\\
3.98999999999994	-6.55884920146257e-15\\
3.98999999999997	-6.55884920146085e-15\\
3.99000000000001	-6.55884920145913e-15\\
3.99399999999994	-6.3677699649093e-15\\
3.99599999999998	-6.27379958743651e-15\\
3.99600000000001	-6.27379958743518e-15\\
3.99999999999994	-6.08891953184325e-15\\
3.99999999999997	-6.08891953184177e-15\\
4	-6.0889195318403e-15\\
4.00199999999993	-5.99798164709298e-15\\
4.00199999999999	-5.99798164709041e-15\\
4.00399999999992	-5.9080293901827e-15\\
4.00599999999986	-5.81905107091003e-15\\
4.00599999999993	-5.81905107090681e-15\\
4.006	-5.81905107090355e-15\\
4.00999999999987	-5.64397011591273e-15\\
4.01199999999995	-5.55784472667584e-15\\
4.012	-5.55784472667341e-15\\
4.01599999999987	-5.38836815919313e-15\\
4.01999999999973	-5.22251731975485e-15\\
4.01999999999995	-5.22251731974614e-15\\
4.02	-5.22251731974381e-15\\
4.02500000000001	-5.02017127712649e-15\\
4.02500000000006	-5.02017127712422e-15\\
4.02599999999995	-4.9803512012435e-15\\
4.026	-4.98035120124124e-15\\
4.02699999999996	-4.94074446785103e-15\\
4.02799999999992	-4.90134979010663e-15\\
4.02999999999985	-4.82319148872011e-15\\
4.03099999999993	-4.78442532571864e-15\\
4.03099999999999	-4.78442532571644e-15\\
4.03499999999984	-4.63141794826962e-15\\
4.03699999999994	-4.55613124833273e-15\\
4.03699999999999	-4.5561312483306e-15\\
4.03799999999995	-4.51878784776469e-15\\
4.038	-4.51878784776257e-15\\
4.03899999999996	-4.48166803011022e-15\\
4.03999999999991	-4.44479582583654e-15\\
4.04	-4.44479582583321e-15\\
4.04199999999991	-4.37178947356219e-15\\
4.04399999999982	-4.29975930305465e-15\\
4.046	-4.22869595325316e-15\\
4.04600000000006	-4.22869595325116e-15\\
4.04999999999988	-4.08943289884979e-15\\
4.05399999999969	-3.95392791357355e-15\\
4.05999999999994	-3.75756324615279e-15\\
4.06	-3.75756324615097e-15\\
};
\end{axis}
\end{tikzpicture}%
}
      \caption{Evolution of the angular displacement of pendulum $P_3$.
        $C_3 = 6$ ms. \texttt{Blue}: RM scheduling, \texttt{Red}: EDF scheduling}
      \label{fig:02.6.6.3}
    \end{figure}
  \end{minipage}
  \hfill
  \begin{minipage}{0.45\linewidth}
    \begin{figure}[H]\centering
      \scalebox{0.7}{% This file was created by matlab2tikz.
%
%The latest updates can be retrieved from
%  http://www.mathworks.com/matlabcentral/fileexchange/22022-matlab2tikz-matlab2tikz
%where you can also make suggestions and rate matlab2tikz.
%
\definecolor{mycolor1}{rgb}{0.00000,0.44700,0.74100}%
%
\begin{tikzpicture}

\begin{axis}[%
width=4.133in,
height=3.26in,
at={(0.693in,0.44in)},
scale only axis,
xmin=0,
xmax=1.22,
xmajorgrids,
ymin=-0.0001,
ymax=0.001,
ymajorgrids,
axis background/.style={fill=white}
]
\pgfplotsset{max space between ticks=50}
\addplot [color=mycolor1,solid,forget plot]
  table[row sep=crcr]{%
0	0\\
3.15544362088405e-30	0\\
0.000656101980281985	0\\
0.00393661188169191	0\\
0.00599999999999994	0\\
0.006	0\\
0.012	0\\
0.0120000000000001	0\\
0.018	0\\
0.0180000000000001	0\\
0.0199999999999998	0\\
0.02	0\\
0.026	0\\
0.0260000000000002	0\\
0.0289999999999998	0\\
0.029	0\\
0.0319999999999996	0\\
0.0349999999999991	0\\
0.035	0\\
0.0399999999999996	0\\
0.04	0\\
0.0449999999999996	0\\
0.0459999999999996	0\\
0.046	0\\
0.047	0\\
0.0470000000000004	0\\
0.0490000000000003	0\\
0.0510000000000002	0\\
0.055	0\\
0.0579999999999996	0\\
0.058	0\\
0.0599999999999996	0\\
0.06	0\\
0.0619999999999995	0\\
0.0639999999999991	0\\
0.0659999999999991	0\\
0.066	0\\
0.0699999999999991	0\\
0.07	0\\
0.0700000000000009	0\\
0.074	0\\
0.076	0\\
0.0760000000000009	0\\
0.08	0\\
0.0800000000000009	0\\
0.0839999999999999	0\\
0.086	0\\
0.0860000000000009	0\\
0.0869999999999991	0\\
0.087	0\\
0.0880000000000004	0\\
0.0890000000000009	0\\
0.0910000000000017	0\\
0.0929999999999991	0\\
0.093	0\\
0.0970000000000017	0\\
0.0999999999999991	0\\
0.1	0\\
0.104000000000002	0\\
0.104999999999999	0\\
0.105	0\\
0.105999999999999	0\\
0.106	0\\
0.106999999999999	0\\
0.107999999999998	0\\
0.109999999999997	0\\
0.111999999999999	0\\
0.112	0\\
0.115999999999997	0\\
0.115999999999998	0\\
0.116	0\\
0.119999999999997	0\\
0.119999999999998	0\\
0.12	0\\
0.123999999999997	0\\
0.125999999999999	0\\
0.126	0\\
0.127999999999998	0\\
0.128	0\\
0.129999999999998	0\\
0.131999999999996	0\\
0.135999999999993	0\\
0.139999999999998	0\\
0.14	0\\
0.144999999999998	0\\
0.145	0\\
0.145999999999998	0\\
0.146	0\\
0.146999999999999	0\\
0.147999999999998	0\\
0.149999999999997	0\\
0.151999999999998	0\\
0.152	0\\
0.155999999999997	0\\
0.157999999999998	0\\
0.158	0\\
0.16	0\\
0.160000000000002	0\\
0.162000000000002	0\\
0.164000000000002	0\\
0.166	0\\
0.166000000000002	0\\
0.170000000000002	0\\
0.174	0\\
0.174000000000001	0\\
0.175	0\\
0.175000000000002	0\\
0.176000000000001	0\\
0.177	0\\
0.178999999999998	0\\
0.179999999999998	0\\
0.18	0\\
0.183999999999997	0\\
0.186	0\\
0.186000000000002	0\\
0.189999999999998	0\\
0.192	0\\
0.192000000000002	0\\
0.195999999999998	0\\
0.199999999999995	0\\
0.199999999999997	0\\
0.2	0\\
0.202999999999998	0\\
0.203	0\\
0.205999999999998	0\\
0.206	0\\
0.208999999999998	0\\
0.209999999999998	0\\
0.21	0\\
0.211999999999998	0\\
0.212	0\\
0.213999999999998	0\\
0.215999999999997	0\\
0.217999999999998	0\\
0.218	0\\
0.219999999999998	0\\
0.22	0\\
0.221999999999998	0\\
0.223999999999996	0\\
0.225999999999998	0\\
0.226	0\\
0.229999999999996	0\\
0.231999999999998	0\\
0.232	0\\
0.235999999999996	0\\
0.237999999999998	0\\
0.238	0\\
0.239999999999998	0\\
0.24	0\\
0.241999999999998	0\\
0.243999999999996	0\\
0.245	0\\
0.245000000000002	0\\
0.245999999999998	0\\
0.246	0\\
0.246999999999999	0\\
0.247999999999998	0\\
0.249999999999997	0\\
0.252	0\\
0.252000000000003	0\\
0.256	0\\
0.259999999999997	0\\
0.26	0\\
0.260999999999996	0\\
0.261	0\\
0.261999999999998	0\\
0.262999999999996	0\\
0.264999999999993	0\\
0.265999999999997	0\\
0.266	0\\
0.269999999999993	0\\
0.271999999999997	0\\
0.272	0\\
0.275999999999993	0\\
0.279999999999986	0\\
0.279999999999993	0\\
0.28	0\\
0.285999999999996	0\\
0.286	0\\
0.289999999999996	0\\
0.29	0\\
0.291999999999996	0.000234296016646662\\
0.292	0.000470973519683549\\
0.293999999999996	0.000238979979915628\\
0.295999999999993	0.000248299954860809\\
0.297999999999996	2.07240878912843e-05\\
0.298	0.000259954675630548\\
0.299999999999996	3.45431389619059e-05\\
0.3	0.000271643180061957\\
0.301999999999996	0.000283366987254263\\
0.303999999999993	0.000295127620833457\\
0.305999999999996	7.60344145278678e-05\\
0.306	0.000535809889309546\\
0.309999999999993	0.000555600220044345\\
0.313999999999986	0.000243258101635394\\
0.314999999999997	0.000138453285042266\\
0.315	0.000580745525665538\\
0.318999999999997	0.00016630219026109\\
0.319	0.000275711040441207\\
0.319999999999996	0.000173277358397769\\
0.32	0.000282227656646379\\
0.320999999999998	0.000288753442381235\\
0.321999999999996	0.000402884431760505\\
0.323999999999993	0.000415098279597377\\
0.325999999999996	0.00021525453643134\\
0.326	0.000637755030999242\\
0.329999999999993	0.000347933228223171\\
0.331	0.000250425766781791\\
0.331000000000004	0.000458282052830744\\
0.333	0.000264549611222027\\
0.333000000000004	0.000470467951892982\\
0.335	0.000482352565191873\\
0.336999999999996	0.00059411473279611\\
0.339999999999996	0.000313296582369496\\
0.34	0.000707899190101741\\
0.343999999999993	0.000534672896543038\\
0.345999999999997	0.000353930851532605\\
0.346	0.000546050766745092\\
0.347999999999997	0.000367248795297766\\
0.348	0.000557341364505862\\
0.349999999999997	0.000380456229602286\\
0.35	0.000568546157255751\\
0.351999999999997	0.000579666601115239\\
0.353999999999993	0.000406546421445055\\
0.354	0.0007729317838678\\
0.357999999999993	0.000612536237827217\\
0.359999999999996	0.000444895344301482\\
0.36	0.000797819116765358\\
0.363999999999993	0.000641606532251655\\
0.365999999999996	0.000481271502543143\\
0.366	0.000816960179974024\\
0.369999999999993	0.000828271596177151\\
0.373999999999986	0.000761133351190452\\
0.376999999999997	0.000540199981331105\\
0.377	0.00077004299312009\\
0.379999999999997	0.000554552889327252\\
0.38	0.000778287345721518\\
0.382999999999996	0.000713972540899747\\
0.384999999999997	0.000576857933935361\\
0.385	0.00064875421564522\\
0.385999999999997	0.000581077788798166\\
0.386	0.000652309698775744\\
0.386999999999998	0.000655787437520961\\
0.387999999999996	0.000659187544864422\\
0.388999999999997	0.000593257536613087\\
0.389	0.000731107482651973\\
0.390999999999997	0.000736216700297787\\
0.392999999999993	0.000741025842419868\\
0.394999999999997	0.0006154694370612\\
0.395	0.000873261509334066\\
0.398999999999993	0.000691552447579083\\
0.399999999999997	0.000631703668071694\\
0.4	0.000877347459253788\\
0.403999999999993	0.000762353441308718\\
0.405999999999997	0.00064832447418049\\
0.406	0.000879652857658457\\
0.409999999999993	0.000769815356329722\\
0.411999999999997	0.000661843202988305\\
0.412	0.000879126772734448\\
0.415999999999993	0.00087720283556166\\
0.419999999999986	0.00067506953266776\\
0.419999999999996	0.000675069532667408\\
0.42	0.000970109306038019\\
0.426	0.000681405340060636\\
0.426000000000004	0.000956285227929731\\
0.432000000000004	0.000684677766133717\\
0.432000000000007	0.000815051626637382\\
0.434999999999997	0.000685269011955151\\
0.435	0.00081150673503143\\
0.43799999999999	0.000767179434116177\\
0.439999999999997	0.000685010588290481\\
0.44	0.000804436187223989\\
0.44299999999999	0.000799494399677521\\
0.445999999999979	0.000682647662837542\\
0.445999999999995	0.000682647662837137\\
0.446	0.00072021741723216\\
0.447	0.000682035968075108\\
0.447000000000004	0.000719161772756118\\
0.448000000000004	0.000718045038631026\\
0.449000000000004	0.000752668313989352\\
0.451000000000004	0.000817789850730099\\
0.454999999999997	0.000674897662025925\\
0.455	0.000806766019541006\\
0.459	0.000701717243316225\\
0.459999999999997	0.00066840500767321\\
0.46	0.000791674981245953\\
0.463999999999997	0.000662081808481699\\
0.464	0.000721160573734521\\
0.465999999999997	0.000658542754042548\\
0.466	0.000715927682438913\\
0.466999999999997	0.000740858422402709\\
0.467	0.000794877831180461\\
0.467999999999998	0.0007643895572785\\
0.468999999999997	0.000760482195805211\\
0.470999999999993	0.000700495134913882\\
0.472999999999997	0.000667351891090271\\
0.473	0.000693187877586751\\
0.476999999999993	0.000628098972564827\\
0.479999999999997	0.000617866411572181\\
0.48	0.000617866411572169\\
0.483999999999993	0.000603766002479431\\
0.485999999999997	0.000596517886408152\\
0.486	0.000596517886408138\\
0.489999999999993	0.000581621121283194\\
0.49	0.000581621121283084\\
0.490000000000004	0.000648031542759864\\
0.492999999999997	0.000570094261001083\\
0.493	0.000634613800917791\\
0.495999999999993	0.000620907514500992\\
0.498999999999986	0.000546114847035763\\
0.498999999999993	0.000566584533861194\\
0.499	0.000566584533861092\\
0.499999999999997	0.000562261162449524\\
0.5	0.000582320914164909\\
0.500999999999998	0.000597411569264762\\
0.501999999999997	0.000611860840285686\\
0.503999999999993	0.000563869799774052\\
0.505999999999993	0.000554438695206393\\
0.506	0.000554438695206359\\
0.507999999999997	0.000544983662025632\\
0.508000000000004	0.000581622541390804\\
0.51	0.000607168336522238\\
0.511999999999997	0.000631060420256334\\
0.51599999999999	0.000541716194611719\\
0.519999999999993	0.000488403416785153\\
0.52	0.000488403416785119\\
0.521999999999993	0.000478940034109259\\
0.522	0.000495407702234015\\
0.523999999999993	0.000453129333450881\\
0.524999999999993	0.000448518676773597\\
0.525	0.000448518676773564\\
0.526	0.00044390383144707\\
0.526000000000007	0.00045997363427173\\
0.527000000000007	0.000471030055516045\\
0.528000000000007	0.000481691393963569\\
0.530000000000007	0.000440946168267575\\
0.532	0.000461597461713693\\
0.532000000000007	0.000491278283569636\\
0.536000000000007	0.000412268822464634\\
0.538	0.000402653190383132\\
0.538000000000007	0.000402653190383097\\
0.539999999999993	0.000393076660299689\\
0.54	0.000421173549637081\\
0.541999999999986	0.000411201518829467\\
0.543999999999972	0.000374312862609136\\
0.546	0.000391653735046484\\
0.546000000000007	0.000404725642330742\\
0.549999999999979	0.00033417860426178\\
0.550999999999993	0.000367948992762918\\
0.551	0.000392691305628551\\
0.554999999999972	0.000325318435427154\\
0.556999999999993	0.000328741820922491\\
0.557	0.00032874182092246\\
0.559999999999993	0.000315831750894472\\
0.56	0.000349767291478855\\
0.562999999999993	0.000303191964948646\\
0.565999999999986	0.000257985535847775\\
0.565999999999993	0.000322559289709024\\
0.566	0.000322559289708991\\
0.571999999999986	0.000236200644330756\\
0.571999999999993	0.000297169443006837\\
0.572	0.000316865712184143\\
0.577999999999986	0.000235258975651425\\
0.579999999999993	0.000266214019050394\\
0.58	0.000266214019050369\\
0.585999999999986	0.00019003525333224\\
0.585999999999993	0.000244565341455488\\
0.586	0.000244565341455462\\
0.591999999999986	0.000172352980355525\\
0.591999999999993	0.000198620117127928\\
0.592	0.000198620117127905\\
0.594999999999993	0.000189586933912094\\
0.595	0.000206301550890968\\
0.597999999999993	0.000172616478836014\\
0.599999999999993	0.000175225489141367\\
0.6	0.000199127185884945\\
0.602999999999993	0.000167020832415251\\
0.605999999999986	0.000143676353091168\\
0.606	0.00014367635309108\\
0.606999999999993	0.000141269227622094\\
0.607	0.000149016571975252\\
0.607999999999999	0.000138910976602817\\
0.608999999999997	0.000144266002805727\\
0.609000000000004	0.000159554765075309\\
0.611	0.0001547905683041\\
0.612999999999997	0.000135068290342152\\
0.614999999999997	0.000145569702336667\\
0.615000000000004	0.000153003544725513\\
0.618999999999997	0.000114614609623918\\
0.619999999999993	0.00013461410627112\\
0.62	0.000149190418649361\\
0.623999999999993	0.000111826562001422\\
0.625999999999993	0.000122292424939115\\
0.626	0.000122292424939102\\
0.629999999999993	0.000114571298632939\\
0.63	0.000128654247828167\\
0.633999999999993	9.32503046805427e-05\\
0.635999999999993	8.98156795763137e-05\\
0.636	8.98156795763017e-05\\
0.637999999999993	8.64627732667486e-05\\
0.638	8.64627732667367e-05\\
0.639999999999993	8.3189367451366e-05\\
0.64	9.68729609517902e-05\\
0.641999999999993	9.35819236723638e-05\\
0.643999999999986	7.6879365923889e-05\\
0.645999999999993	8.7240708927861e-05\\
0.646	8.72407089278966e-05\\
0.649999999999986	5.46930021904923e-05\\
0.65	8.12179108914237e-05\\
0.650000000000007	0.000107390812500356\\
0.653999999999993	8.84691056725974e-05\\
0.657999999999979	5.72394113204119e-05\\
0.659999999999993	7.39011766210889e-05\\
0.66	7.39011766210795e-05\\
0.664999999999993	4.24899182259892e-05\\
0.665	4.24899182259813e-05\\
0.665999999999993	4.13810583347626e-05\\
0.666	4.13810583347549e-05\\
0.666999999999998	4.02906087204841e-05\\
0.667000000000006	4.65804262738074e-05\\
0.668000000000004	5.17417744334676e-05\\
0.669000000000002	5.68664911496407e-05\\
0.670999999999998	4.23277864468817e-05\\
0.673000000000005	5.26188214936618e-05\\
0.673000000000013	5.87461285445905e-05\\
0.677000000000005	3.03911778220689e-05\\
0.678	3.55680831153542e-05\\
0.678000000000007	3.55680831153475e-05\\
0.679999999999993	3.37487491693954e-05\\
0.68	4.56889583584109e-05\\
0.681999999999986	4.38115179556934e-05\\
0.683999999999972	3.02841675113337e-05\\
0.686	4.02346392456894e-05\\
0.686000000000007	6.30935329625153e-05\\
0.689999999999979	4.81592547825565e-05\\
0.69399999999995	2.26280598300196e-05\\
0.695999999999993	3.23202989461188e-05\\
0.696	3.23202989461138e-05\\
0.699999999999993	2.95613852118468e-05\\
0.7	4.03053887073426e-05\\
0.703999999999993	1.63888150533972e-05\\
0.705999999999993	1.53091251008173e-05\\
0.706	1.53091251008135e-05\\
0.707999999999993	1.42851437722816e-05\\
0.708	2.47329960607378e-05\\
0.709999999999993	2.36686057498714e-05\\
0.711999999999986	1.24037819230264e-05\\
0.713999999999993	2.15598369396081e-05\\
0.714	3.14439528264639e-05\\
0.717999999999986	9.77899576761485e-06\\
0.719999999999993	1.8528203003896e-05\\
0.72	2.32437262896142e-05\\
0.723999999999986	2.81076321669298e-06\\
0.724999999999993	2.50401555765311e-06\\
0.725	2.50401555765105e-06\\
0.725999999999993	2.2070735091179e-06\\
0.726	6.80101094156023e-06\\
0.726999999999999	1.09872651754442e-05\\
0.727999999999997	1.50637980644712e-05\\
0.729999999999993	5.5510797691646e-06\\
0.731999999999993	9.30098210149589e-06\\
0.732	9.30098210149373e-06\\
0.734999999999993	8.40961657919137e-06\\
0.735	1.66896435865137e-05\\
0.737999999999993	3.52609062931103e-06\\
0.74	7.12384457452972e-06\\
0.740000000000007	1.89118781990529e-05\\
0.743	6.47191606765335e-06\\
0.745999999999993	-1.68512514987752e-06\\
0.746000000000007	-1.68512514990516e-06\\
0.746999999999993	-1.77081246330313e-06\\
0.747	2.0289265264369e-06\\
0.747999999999999	5.68723673419365e-06\\
0.748999999999997	1.29654646249917e-05\\
0.750999999999993	5.35719155450825e-06\\
0.753999999999993	8.65887804613074e-06\\
0.754	1.57825140131317e-05\\
0.757999999999993	1.20093600077713e-06\\
0.759999999999993	8.05093828010027e-06\\
0.76	1.49142123135181e-05\\
0.763999999999993	9.5079297089413e-07\\
0.766	7.5931126517972e-06\\
0.766000000000007	7.59311265179671e-06\\
0.77	7.37102064103965e-06\\
0.770000000000007	1.38180525231595e-05\\
0.774	8.48949224759791e-07\\
0.776	7.1658413598994e-06\\
0.776000000000007	7.16584135989929e-06\\
0.779999999999993	4.03360548052528e-06\\
0.78	4.03360548052533e-06\\
0.782999999999993	4.04540448121371e-06\\
0.783	4.04540448121376e-06\\
0.785999999999993	4.06563157801193e-06\\
0.786000000000001	1.28492603575308e-05\\
0.788999999999994	4.09429268485563e-06\\
0.791999999999987	-4.48894623999839e-06\\
0.792	1.25909729527512e-05\\
0.792000000000008	1.81426000500806e-05\\
0.797999999999994	1.48131850541682e-06\\
0.799999999999993	9.65058935758085e-06\\
0.8	9.65058935758074e-06\\
0.804999999999993	-8.87298335063231e-07\\
0.805000000000001	-8.87298335062851e-07\\
0.805999999999993	-8.2728762874272e-07\\
0.806	1.80298947898618e-06\\
0.806999999999994	4.44401587131228e-06\\
0.807999999999987	7.03637451663947e-06\\
0.809999999999973	1.98450812997387e-06\\
0.811999999999993	7.07724702062902e-06\\
0.812	1.20055190337581e-05\\
0.815999999999973	2.25036379390377e-06\\
0.817999999999993	2.3345219322878e-06\\
0.818000000000001	2.33452193228813e-06\\
0.819999999999993	2.41673213053202e-06\\
0.82	7.13036362878485e-06\\
0.821999999999993	7.14032721879762e-06\\
0.823999999999986	2.57535119907925e-06\\
0.825999999999993	7.15628538569481e-06\\
0.826	1.59602929369657e-05\\
0.829999999999986	1.98689803240099e-05\\
0.833999999999972	1.13399022879423e-05\\
0.839999999999993	1.09119731401118e-06\\
0.84	1.09119731401153e-06\\
0.840999999999993	1.1384433429825e-06\\
0.841000000000001	3.16487103665852e-06\\
0.841999999999994	5.18938355341663e-06\\
0.842999999999987	5.20330771199302e-06\\
0.844999999999973	1.32218418410568e-06\\
0.845999999999993	7.15686525194206e-06\\
0.846	1.09378568336954e-05\\
0.849999999999973	3.42280639281859e-06\\
0.851999999999993	7.13129219852748e-06\\
0.852	1.07257498411849e-05\\
0.855999999999973	3.57439794226343e-06\\
0.857999999999993	3.60975317324744e-06\\
0.858	3.60975317324752e-06\\
0.86	3.63075988960021e-06\\
0.860000000000007	6.98491487086423e-06\\
0.862000000000007	6.91911683915649e-06\\
0.864000000000007	3.661598230317e-06\\
0.866	6.77761651488644e-06\\
0.866000000000007	6.77761651488617e-06\\
0.87	6.62284849894739e-06\\
0.870000000000007	8.06387942429846e-06\\
0.874	2.2527641378332e-06\\
0.874999999999994	2.26901929380524e-06\\
0.875000000000001	2.26901929380535e-06\\
0.876	2.28429876915251e-06\\
0.876000000000007	3.6648132388281e-06\\
0.877000000000007	4.99943147778658e-06\\
0.878000000000006	3.65232815592725e-06\\
0.879999999999998	3.63612009575724e-06\\
0.880000000000006	6.17733721607492e-06\\
0.882000000000005	6.07809898379668e-06\\
0.884000000000004	3.59252613461471e-06\\
0.886000000000005	3.56513456740328e-06\\
0.886000000000013	3.56513456740317e-06\\
0.888000000000007	3.55551998943297e-06\\
0.888000000000014	5.84593212359739e-06\\
0.890000000000009	8.04049152581081e-06\\
0.892000000000004	9.05846899765581e-06\\
0.895999999999993	4.65961710033975e-06\\
0.898999999999993	2.5167337000841e-06\\
0.899000000000001	2.51673370008413e-06\\
0.899999999999993	2.52435519209707e-06\\
0.9	3.57915081879224e-06\\
0.900999999999994	4.61583650750258e-06\\
0.901999999999987	5.62702761001409e-06\\
0.903999999999973	3.57079519175255e-06\\
0.905999999999993	5.54868549527618e-06\\
0.906	5.54868549528266e-06\\
0.909999999999973	1.59049606944105e-06\\
0.909999999999987	2.57082100399658e-06\\
0.910000000000001	2.57082100399008e-06\\
0.910999999999993	2.57249598973274e-06\\
0.911000000000001	3.53662907722079e-06\\
0.911999999999994	4.47778188284587e-06\\
0.912999999999987	5.3944955966102e-06\\
0.914999999999973	3.50592027386709e-06\\
0.916999999999993	4.40700906410556e-06\\
0.917000000000001	4.40700906410553e-06\\
0.919999999999993	4.37553106869129e-06\\
0.92	7.03952271751215e-06\\
0.922999999999993	4.34223144890615e-06\\
0.925999999999986	1.68993643857384e-06\\
0.925999999999993	3.4398680105132e-06\\
0.926	3.43986801051305e-06\\
0.927999999999993	3.42563271928853e-06\\
0.928000000000001	5.13461988701036e-06\\
0.929999999999994	5.09931382343985e-06\\
0.931999999999987	3.39436102750966e-06\\
0.933999999999994	5.02655040151026e-06\\
0.934000000000001	6.65629322746709e-06\\
0.937999999999987	3.34041930305437e-06\\
0.939999999999993	4.91195540804587e-06\\
0.940000000000001	5.70057125898376e-06\\
0.943999999999987	2.4940530009635e-06\\
0.944999999999994	2.48747889098505e-06\\
0.945000000000001	2.48747889098501e-06\\
0.945999999999993	2.48064104012753e-06\\
0.946000000000001	3.25526580158114e-06\\
0.946999999999994	4.00897428556712e-06\\
0.947999999999987	4.74877848988108e-06\\
0.949999999999973	3.20696928924713e-06\\
0.952	4.66292860835346e-06\\
0.952000000000008	5.39708271681284e-06\\
0.95599999999998	2.39932781047542e-06\\
0.956999999999994	3.83704537248813e-06\\
0.957000000000001	3.83704537248799e-06\\
0.96	3.78419945868366e-06\\
0.960000000000008	5.88406484256835e-06\\
0.963000000000007	3.73100659372032e-06\\
0.966000000000007	3.67745122616563e-06\\
0.966000000000014	3.67745122616548e-06\\
0.969000000000007	3.62351768643726e-06\\
0.969000000000014	5.61314442921116e-06\\
0.972000000000007	6.80647632395619e-06\\
0.975	4.79768075874868e-06\\
0.979999999999994	5.28424460384698e-06\\
0.980000000000001	5.28424460385101e-06\\
0.985999999999987	5.10544673120875e-06\\
0.986000000000001	5.10544673120833e-06\\
0.991999999999987	4.92680646701418e-06\\
0.992000000000001	4.92680646701376e-06\\
0.997999999999987	2.53622344696623e-06\\
0.998000000000001	2.5362234469622e-06\\
0.999999999999993	2.50916321688349e-06\\
1	3.59545157231347e-06\\
1.00199999999999	3.54175998759473e-06\\
1.00399999999999	2.44305562175672e-06\\
1.00599999999999	3.42278882178604e-06\\
1.006	5.39380351004075e-06\\
1.00999999999999	3.76729452056417e-06\\
1.01399999999997	1.73413512046277e-06\\
1.01499999999999	3.55005483603113e-06\\
1.015	3.55005483603046e-06\\
1.01999999999999	3.30869443027153e-06\\
1.02	3.72823203210597e-06\\
1.02499999999999	1.41791890465683e-06\\
1.02599999999999	1.38298826556167e-06\\
1.026	1.38298826556116e-06\\
1.02699999999999	1.34702245309942e-06\\
1.027	1.75064452782167e-06\\
1.02799999999999	2.10144123385298e-06\\
1.02899999999999	2.43546692063455e-06\\
1.03099999999997	1.57536776657063e-06\\
1.03299999999999	2.22094263794392e-06\\
1.033	3.29721419037498e-06\\
1.03699999999997	1.66273991501902e-06\\
1.04	1.86481327863571e-06\\
1.04000000000001	1.86481327863501e-06\\
1.04399999999999	1.02102222404919e-06\\
1.044	1.02102222404863e-06\\
1.046	9.44167896667878e-07\\
1.04600000000001	1.58259687697484e-06\\
1.04800000000001	8.69554362781782e-07\\
1.05	7.97171925422764e-07\\
1.05000000000001	7.97171925422249e-07\\
1.05200000000001	7.27011177767697e-07\\
1.05200000000002	1.32129615782258e-06\\
1.05400000000002	1.23879032090017e-06\\
1.05600000000002	5.93318566992968e-07\\
1.05800000000001	5.34461297559205e-07\\
1.05800000000002	5.34461297558832e-07\\
1.05999999999999	4.82473696571741e-07\\
1.06	1.04243545260809e-06\\
1.06199999999996	9.8773082252289e-07\\
1.06399999999992	3.84975307334774e-07\\
1.06599999999999	8.84995018748555e-07\\
1.066	1.69458756389528e-06\\
1.06999999999992	5.23539765402e-07\\
1.07299999999999	7.2648256994242e-07\\
1.073	1.25158931681496e-06\\
1.07699999999992	1.27225509395195e-07\\
1.07899999999999	-1.6430266770275e-07\\
1.079	-1.6430266770296e-07\\
1.07999999999999	-1.78285069637089e-07\\
1.08	8.04815630207147e-08\\
1.08099999999999	3.22605049768874e-07\\
1.08199999999999	3.07559618976927e-07\\
1.08399999999997	-2.29030949567698e-07\\
1.08499999999999	-2.40423548665315e-07\\
1.085	-2.40423548665478e-07\\
1.08599999999999	-2.51299364980222e-07\\
1.086	1.21634042089455e-09\\
1.08699999999999	2.40342095672317e-07\\
1.08799999999999	4.77010416332872e-07\\
1.08999999999997	-4.11286600478003e-08\\
1.09199999999999	4.31727433471217e-07\\
1.092	1.40283756211852e-06\\
1.09599999999997	3.93883329428796e-07\\
1.09999999999995	-5.96770721008369e-07\\
1.09999999999997	-1.1625897064717e-07\\
1.1	-1.16258970650517e-07\\
1.10199999999999	-1.27651788192592e-07\\
1.102	3.46364372753351e-07\\
1.10399999999999	-1.37878735092113e-07\\
1.10599999999997	-6.17802125073161e-07\\
1.10599999999999	3.20825851207377e-07\\
1.106	7.85559650590234e-07\\
1.10999999999997	-1.61576905501954e-07\\
1.11199999999999	2.9169605068357e-07\\
1.112	1.20089226514139e-06\\
1.11599999999997	2.78380115525451e-07\\
1.11999999999994	-6.28316988296802e-07\\
1.12	7.15110327244671e-07\\
1.12000000000001	7.15110327244633e-07\\
1.126	4.85541562721744e-07\\
1.12600000000001	4.85541562721737e-07\\
1.13099999999999	4.88016567896079e-07\\
1.131	7.03630279769507e-07\\
1.132	4.89445856136592e-07\\
1.13200000000001	7.0449758370307e-07\\
1.13300000000001	4.91507394183305e-07\\
1.13400000000001	4.9410305833311e-07\\
1.13600000000001	2.88327402523398e-07\\
1.138	-1.24189783291517e-07\\
1.13800000000001	8.71231424952693e-08\\
1.13999999999999	-1.09556466793907e-07\\
1.14	3.06236031905775e-07\\
1.14199999999997	3.15844840922318e-07\\
1.14399999999994	-7.91398107275268e-08\\
1.14599999999999	3.36377690719325e-07\\
1.146	1.12014154238723e-06\\
1.14999999999994	5.51381216418091e-07\\
1.15399999999989	-1.87722579694843e-07\\
1.15499999999999	5.75405883013875e-07\\
1.155	5.7540588301395e-07\\
1.15999999999999	6.02378887535108e-07\\
1.16	7.81335812607788e-07\\
1.16499999999999	-7.14189592847993e-08\\
1.16599999999999	6.38669327394168e-07\\
1.166	8.1042560509409e-07\\
1.17099999999999	-3.06761898808265e-09\\
1.17199999999999	8.66904481393552e-09\\
1.172	8.66904481410153e-09\\
1.173	2.0568398988829e-08\\
1.17300000000001	1.8898974611585e-07\\
1.17400000000001	3.66297183264352e-07\\
1.17500000000001	7.03785131902455e-07\\
1.17700000000001	3.95903740577185e-07\\
1.17999999999999	5.84547855907944e-07\\
1.18	9.00137916320049e-07\\
1.184	3.07745932765926e-07\\
1.18599999999999	4.83320771424866e-07\\
1.186	4.83320771425001e-07\\
1.18899999999999	2.08063311271921e-07\\
1.189	2.08063311272087e-07\\
1.18999999999999	2.19487706708954e-07\\
1.19	3.70989173684929e-07\\
1.19099999999999	5.3103917943471e-07\\
1.19199999999999	5.40514968760729e-07\\
1.19399999999997	2.64865576582006e-07\\
1.19499999999999	7.1376893390816e-07\\
1.195	8.57809899506071e-07\\
1.19899999999997	3.20896129360017e-07\\
1.19999999999999	7.55982205504777e-07\\
1.2	1.03415643853812e-06\\
1.20399999999997	5.14843484320349e-07\\
1.20599999999999	3.98128970511015e-07\\
1.206	3.98128970511169e-07\\
1.20699999999999	4.08869144860968e-07\\
1.207	5.44322102526385e-07\\
1.20799999999999	6.86538405493291e-07\\
1.20899999999999	1.08232679912591e-06\\
1.21099999999997	1.21486280655097e-06\\
1.21499999999995	7.38995532799239e-07\\
1.21799999999999	6.38716208519364e-07\\
1.218	6.38716208519472e-07\\
1.21999999999999	6.54190457399976e-07\\
1.22	8.89999694162264e-07\\
1.22199999999999	7.8528717586145e-07\\
1.22399999999997	5.68634923854634e-07\\
1.22499999999999	5.76837037848597e-07\\
1.225	5.76837037848712e-07\\
1.22599999999999	5.84908073555952e-07\\
1.226	6.97535332515843e-07\\
1.22699999999999	8.14619066261249e-07\\
1.22799999999999	7.10963924460486e-07\\
1.22999999999997	9.36442547276734e-07\\
1.23	1.14443427058386e-06\\
1.23399999999997	7.48215035133206e-07\\
1.236	9.59396354774714e-07\\
1.23600000000001	9.59396354774766e-07\\
1.23999999999999	9.7330269061528e-07\\
1.24	1.16072730403563e-06\\
1.24399999999997	8.01637347483664e-07\\
1.24599999999999	7.1947375462684e-07\\
1.246	7.19473754626914e-07\\
1.247	7.24784115398866e-07\\
1.24700000000001	8.15006675011245e-07\\
1.24800000000001	9.07429515369875e-07\\
1.24900000000001	9.96762577220326e-07\\
1.25100000000001	8.30885201080953e-07\\
1.253	1.00372195185935e-06\\
1.25300000000001	1.24506296015953e-06\\
1.25700000000001	9.30030035959173e-07\\
1.25999999999999	1.01078173880978e-06\\
1.26	1.15924954235372e-06\\
1.264	8.67207814812499e-07\\
1.266	1.01165414897386e-06\\
1.26600000000001	1.1488339088382e-06\\
1.27000000000001	8.76127460735228e-07\\
1.272	1.0093643636239e-06\\
1.27200000000001	1.0093643636239e-06\\
1.276	1.00760588327202e-06\\
1.27600000000001	1.00760588327201e-06\\
1.27999999999999	1.00425811166731e-06\\
1.28000000000001	1.12096034262987e-06\\
1.28399999999999	8.85421074669409e-07\\
1.28599999999999	8.8514346113206e-07\\
1.28600000000001	8.85143461132057e-07\\
1.288	8.84452622464696e-07\\
1.28800000000001	9.92786906152293e-07\\
1.29	1.04067792836617e-06\\
1.29199999999999	9.33586708898227e-07\\
1.29499999999999	9.77509921940374e-07\\
1.295	1.02585688557685e-06\\
1.29899999999998	8.25589911472648e-07\\
1.29999999999999	9.63589992455019e-07\\
1.3	1.00856733988367e-06\\
1.30399999999998	8.19719854576227e-07\\
1.30499999999999	8.18229719796386e-07\\
1.305	8.18229719796364e-07\\
1.30599999999999	8.16634105557493e-07\\
1.306	8.59607393897528e-07\\
1.30699999999999	8.9888531605114e-07\\
1.30700000000001	9.80187767212763e-07\\
1.308	9.35945737839219e-07\\
1.30899999999999	9.31668294693746e-07\\
1.31099999999998	8.44697333915465e-07\\
1.31300000000001	7.98079171495763e-07\\
1.31300000000002	8.36881619718858e-07\\
1.31699999999999	7.45438426177419e-07\\
1.31999999999999	7.34722612185557e-07\\
1.32	7.34722612185505e-07\\
1.32399999999997	7.19627133197879e-07\\
1.32599999999999	7.1173102640433e-07\\
1.326	7.11731026404273e-07\\
1.32999999999997	6.95236947210035e-07\\
1.33	6.95236947209521e-07\\
1.33399999999997	6.77799851007369e-07\\
1.334	6.77799851007205e-07\\
1.33799999999997	6.59410673015958e-07\\
1.34	6.49856085400092e-07\\
1.34000000000001	6.49856085400022e-07\\
1.34399999999999	6.30020704215121e-07\\
1.346	6.19737332838361e-07\\
1.34600000000001	6.19737332838287e-07\\
1.348	6.0920840295363e-07\\
1.34800000000001	6.09208402953555e-07\\
1.35	5.98531714667192e-07\\
1.35199999999999	5.87805048877239e-07\\
1.35599999999996	5.66196201968632e-07\\
1.35999999999999	5.44370628639861e-07\\
1.36	5.44370628639782e-07\\
1.36299999999999	5.27852400947959e-07\\
1.363	5.2785240094788e-07\\
1.36499999999999	5.16766594242193e-07\\
1.365	5.16766594242113e-07\\
1.36599999999999	5.1120105485815e-07\\
1.366	5.11201054858071e-07\\
1.36699999999999	5.05620183853869e-07\\
1.36799999999999	5.00023799908609e-07\\
1.36999999999997	4.88783765385926e-07\\
1.37199999999999	4.77479490534006e-07\\
1.372	4.77479490533926e-07\\
1.37599999999997	4.5467233491367e-07\\
1.378	4.43166490125096e-07\\
1.37800000000001	4.43166490125014e-07\\
1.37999999999999	4.31700375709015e-07\\
1.38	4.31700375708934e-07\\
1.38199999999997	4.20382400645359e-07\\
1.38399999999994	4.09211094048977e-07\\
1.38599999999999	3.98185004095345e-07\\
1.386	3.98185004095267e-07\\
1.38999999999994	3.76562761026324e-07\\
1.39199999999999	3.65963797881347e-07\\
1.392	3.65963797881272e-07\\
1.39599999999994	3.45183301009194e-07\\
1.39799999999999	3.34999066645877e-07\\
1.398	3.34999066645805e-07\\
1.39999999999999	3.24950404348404e-07\\
1.4	3.24950404348333e-07\\
1.40199999999999	3.15036008197225e-07\\
1.40399999999997	3.05254589716433e-07\\
1.40599999999999	2.95604877711759e-07\\
1.406	2.95604877711691e-07\\
1.40999999999997	2.76695573800218e-07\\
1.412	2.6743352443802e-07\\
1.41200000000001	2.67433524437955e-07\\
1.41599999999999	2.49412804018744e-07\\
1.41999999999996	2.32138280405445e-07\\
1.41999999999998	2.32138280405354e-07\\
1.42	2.32138280405263e-07\\
1.42099999999999	2.27935181900635e-07\\
1.421	2.27935181900576e-07\\
1.42199999999999	2.23778021022498e-07\\
1.42299999999999	2.19666662684701e-07\\
1.42499999999997	2.11580820806729e-07\\
1.42599999999999	2.0760607455835e-07\\
1.426	2.07606074558293e-07\\
1.42999999999997	1.92158591417776e-07\\
1.43199999999999	1.84703999473277e-07\\
1.432	1.84703999473225e-07\\
1.43499999999999	1.73855815637692e-07\\
1.435	1.73855815637642e-07\\
1.43799999999998	1.63405265666246e-07\\
1.43999999999999	1.5665762621806e-07\\
1.44	1.56657626218013e-07\\
1.44299999999999	1.46863065516208e-07\\
1.44599999999997	1.37458245632374e-07\\
1.44599999999999	1.37458245632327e-07\\
1.446	1.37458245632279e-07\\
1.44699999999999	1.34409436143308e-07\\
1.447	1.34409436143265e-07\\
1.44799999999999	1.31399142410203e-07\\
1.44899999999999	1.2842288339107e-07\\
1.44999999999999	1.25480562388068e-07\\
1.45	1.25480562388026e-07\\
1.45199999999999	1.19697353149936e-07\\
1.45399999999997	1.14048763111058e-07\\
1.45599999999999	1.08534058180005e-07\\
1.456	1.08534058179966e-07\\
1.45999999999997	9.79034541966046e-08\\
1.46	9.79034541965309e-08\\
1.46000000000001	9.7903454196494e-08\\
1.46399999999999	8.78000148133387e-08\\
1.466	8.29443298612375e-08\\
1.46600000000001	8.29443298612034e-08\\
1.46999999999999	7.36218741407667e-08\\
1.47	7.36218741407347e-08\\
1.47399999999997	6.48139600925225e-08\\
1.47599999999999	6.06015149214459e-08\\
1.476	6.06015149214165e-08\\
1.47899999999998	5.44991316679566e-08\\
1.479	5.44991316679287e-08\\
1.47999999999999	5.25185817607437e-08\\
1.48	5.25185817607157e-08\\
1.48099999999999	5.05647128055699e-08\\
1.48199999999999	4.86374613216242e-08\\
1.48399999999997	4.4862561166896e-08\\
1.48599999999999	4.11933907100814e-08\\
1.486	4.11933907100557e-08\\
1.48999999999997	3.41703451959195e-08\\
1.491	3.24799363578419e-08\\
1.49100000000001	3.24799363578182e-08\\
1.49499999999999	2.597806349967e-08\\
1.49899999999996	1.98893059567227e-08\\
1.49999999999999	1.84312902054035e-08\\
1.5	1.84312902053827e-08\\
1.50499999999999	1.15238950993743e-08\\
1.505	1.15238950993557e-08\\
1.50599999999999	1.02186386239624e-08\\
1.506	1.02186386239439e-08\\
1.50699999999999	8.93868878969557e-09\\
1.50799999999999	7.68400401011627e-09\\
1.508	7.68400401009848e-09\\
1.50999999999999	5.25026737720864e-09\\
1.51199999999997	2.91711243731171e-09\\
1.51399999999999	6.84235973314694e-10\\
1.514	6.84235973299024e-10\\
1.51799999999997	-3.48182926559766e-09\\
1.518	-3.48182926562455e-09\\
1.51999999999999	-5.42678540091599e-09\\
1.52	-5.42678540092955e-09\\
1.52199999999999	-7.29499930627471e-09\\
1.52399999999997	-9.08671377501484e-09\\
1.52599999999999	-1.08021616585752e-08\\
1.526	-1.08021616585872e-08\\
1.52999999999997	-1.4005139541798e-08\\
1.53399999999994	-1.69055980441994e-08\\
1.537	-1.88833265308404e-08\\
1.53700000000001	-1.88833265308495e-08\\
1.53999999999999	-2.06923175365558e-08\\
1.54	-2.06923175365637e-08\\
1.54299999999997	-2.23331000223239e-08\\
1.54599999999994	-2.38061537896615e-08\\
1.54599999999997	-2.38061537896743e-08\\
1.546	-2.38061537896873e-08\\
1.549	-2.51119095818889e-08\\
1.54900000000001	-2.51119095818947e-08\\
1.55200000000001	-2.62507492144964e-08\\
1.55500000000001	-2.72230057022837e-08\\
1.55500000000003	-2.72230057022879e-08\\
1.55999999999999	-2.85487142477687e-08\\
1.56	-2.85487142477721e-08\\
1.56499999999996	-2.95629697576511e-08\\
1.56599999999998	-2.9728523441005e-08\\
1.566	-2.97285234410072e-08\\
1.57099999999996	-3.03701006092284e-08\\
1.57199999999998	-3.0461206975843e-08\\
1.572	-3.04612069758443e-08\\
1.57499999999999	-3.06601648482511e-08\\
1.575	-3.06601648482516e-08\\
1.57799999999999	-3.07476219013799e-08\\
1.57999999999999	-3.07439956744225e-08\\
1.58	-3.07439956744223e-08\\
1.58299999999999	-3.06456587622276e-08\\
1.58599999999997	-3.04358167892429e-08\\
1.586	-3.04358167892405e-08\\
1.58699999999999	-3.034108089646e-08\\
1.587	-3.03410808964584e-08\\
1.58799999999999	-3.02379636752068e-08\\
1.58899999999999	-3.01304795251737e-08\\
1.59099999999997	-2.99023963282638e-08\\
1.59499999999995	-2.93936636151827e-08\\
1.59499999999997	-2.9393663615179e-08\\
1.595	-2.93936636151753e-08\\
1.59999999999999	-2.8658836140316e-08\\
1.6	-2.86588361403137e-08\\
1.60499999999999	-2.78135456354639e-08\\
1.60599999999999	-2.76311731328563e-08\\
1.606	-2.76311731328537e-08\\
1.60699999999998	-2.7444349007648e-08\\
1.607	-2.74443490076452e-08\\
1.60799999999999	-2.72530671901373e-08\\
1.60899999999999	-2.70573214654928e-08\\
1.60999999999998	-2.68571054739893e-08\\
1.61	-2.68571054739862e-08\\
1.61199999999999	-2.64432365251557e-08\\
1.61399999999997	-2.60114065573095e-08\\
1.61599999999999	-2.5561559449788e-08\\
1.616	-2.55615594497846e-08\\
1.61999999999997	-2.46440124986181e-08\\
1.61999999999999	-2.46440124986147e-08\\
1.62	-2.46440124986113e-08\\
1.62399999999997	-2.37265535587252e-08\\
1.624	-2.37265535587194e-08\\
1.62599999999999	-2.32677080520431e-08\\
1.626	-2.32677080520397e-08\\
1.62799999999999	-2.28087056833368e-08\\
1.62999999999998	-2.23494868006189e-08\\
1.63199999999999	-2.18899917237542e-08\\
1.632	-2.1889991723751e-08\\
1.63599999999998	-2.09699340810306e-08\\
1.63999999999995	-2.00480544574427e-08\\
1.63999999999998	-2.00480544574371e-08\\
1.64	-2.00480544574316e-08\\
1.645	-1.88924125543781e-08\\
1.64500000000001	-1.88924125543748e-08\\
1.64599999999999	-1.8660770136314e-08\\
1.646	-1.86607701363107e-08\\
1.64699999999999	-1.84289388252609e-08\\
1.64799999999999	-1.81969110888884e-08\\
1.64999999999997	-1.77322361795136e-08\\
1.65199999999999	-1.72666850189878e-08\\
1.652	-1.72666850189844e-08\\
1.65299999999998	-1.70346023367651e-08\\
1.653	-1.70346023367618e-08\\
1.65399999999999	-1.68043587171624e-08\\
1.65499999999997	-1.65759466795454e-08\\
1.65699999999995	-1.61245877253821e-08\\
1.65899999999998	-1.56804668156062e-08\\
1.659	-1.56804668156031e-08\\
1.65999999999999	-1.5461102554025e-08\\
1.66	-1.54611025540219e-08\\
1.66099999999999	-1.52435262325427e-08\\
1.66199999999998	-1.50277307820857e-08\\
1.66399999999995	-1.46014545073626e-08\\
1.66599999999999	-1.41822183309354e-08\\
1.666	-1.41822183309325e-08\\
1.66999999999995	-1.33646492460835e-08\\
1.67399999999991	-1.25745985091028e-08\\
1.67999999999998	-1.14402247292403e-08\\
1.68	-1.14402247292376e-08\\
1.68199999999998	-1.10754233159587e-08\\
1.682	-1.10754233159562e-08\\
1.68399999999998	-1.07172037612758e-08\\
1.68599999999997	-1.03655195085469e-08\\
1.68599999999999	-1.03655195085442e-08\\
1.686	-1.03655195085418e-08\\
1.68999999999997	-9.68157493305656e-09\\
1.69199999999999	-9.34922572489028e-09\\
1.692	-9.34922572488779e-09\\
1.69599999999997	-8.70355750231733e-09\\
1.69799999999999	-8.39015457679186e-09\\
1.698	-8.39015457678969e-09\\
1.69999999999999	-8.08485024715074e-09\\
1.7	-8.08485024714856e-09\\
1.70199999999999	-7.78947055298681e-09\\
1.70399999999998	-7.50397710671848e-09\\
1.70599999999999	-7.22833280558284e-09\\
1.706	-7.22833280558088e-09\\
1.70999999999998	-6.70644962377001e-09\\
1.71099999999998	-6.58208011433253e-09\\
1.711	-6.58208011433073e-09\\
1.71499999999997	-6.10888599574543e-09\\
1.715	-6.10888599574255e-09\\
1.71699999999998	-5.8868057545197e-09\\
1.717	-5.88680575451817e-09\\
1.71899999999998	-5.67436449130979e-09\\
1.71999999999999	-5.57174979983954e-09\\
1.72	-5.57174979983811e-09\\
1.72199999999999	-5.37371562747254e-09\\
1.72399999999997	-5.18525375195545e-09\\
1.72599999999999	-5.00633968075842e-09\\
1.726	-5.0063396807572e-09\\
1.72899999999998	-4.75582018282778e-09\\
1.729	-4.75582018282661e-09\\
1.73199999999998	-4.52174902565971e-09\\
1.73499999999997	-4.29914882590936e-09\\
1.73999999999998	-3.95346422852462e-09\\
1.74	-3.95346422852366e-09\\
1.74599999999997	-3.58007246737595e-09\\
1.74599999999998	-3.58007246737497e-09\\
1.746	-3.58007246737399e-09\\
1.74999999999998	-3.35605035295534e-09\\
1.75	-3.35605035295454e-09\\
1.75199999999998	-3.25146637356025e-09\\
1.752	-3.25146637355948e-09\\
1.75399999999998	-3.15181544279141e-09\\
1.75599999999997	-3.05708460999952e-09\\
1.75799999999998	-2.96726156395316e-09\\
1.758	-2.96726156395255e-09\\
1.75999999999999	-2.88118108685037e-09\\
1.76	-2.88118108684974e-09\\
1.76199999999999	-2.79767844726015e-09\\
1.76399999999998	-2.71674279316946e-09\\
1.76599999999999	-2.63836360617113e-09\\
1.766	-2.63836360617055e-09\\
1.76899999999998	-2.52556600987515e-09\\
1.769	-2.52556600987465e-09\\
1.77199999999998	-2.41846463979303e-09\\
1.77499999999996	-2.31702817802576e-09\\
1.77499999999998	-2.31702817802518e-09\\
1.775	-2.31702817802462e-09\\
1.78	-2.16047627789581e-09\\
1.78000000000002	-2.16047627789538e-09\\
1.78499999999998	-2.01945135325626e-09\\
1.785	-2.01945135325589e-09\\
1.78600000000001	-1.99309930372023e-09\\
1.78600000000003	-1.99309930371983e-09\\
1.78700000000005	-1.96736285484812e-09\\
1.78800000000006	-1.94224117044413e-09\\
1.79000000000009	-1.89383885020371e-09\\
1.79200000000003	-1.84788605262988e-09\\
1.79200000000004	-1.84788605262956e-09\\
1.79600000000011	-1.76077247144203e-09\\
1.79799999999998	-1.7189671412149e-09\\
1.798	-1.71896714121464e-09\\
1.8	-1.6783221567179e-09\\
1.80000000000002	-1.67832215671761e-09\\
1.80200000000002	-1.63883223575151e-09\\
1.80400000000002	-1.6004922462009e-09\\
1.806	-1.5632972053973e-09\\
1.80600000000002	-1.56329720539706e-09\\
1.80999999999998	-1.49232278281447e-09\\
1.81	-1.49232278281421e-09\\
1.81399999999997	-1.4258720715028e-09\\
1.81799999999994	-1.36391052665366e-09\\
1.82	-1.33460303478069e-09\\
1.82000000000001	-1.33460303478048e-09\\
1.826	-1.2533284086518e-09\\
1.82600000000001	-1.25332840865164e-09\\
1.827	-1.2407474710258e-09\\
1.82700000000001	-1.24074747102562e-09\\
1.828	-1.22844108029368e-09\\
1.82899999999999	-1.2164088366e-09\\
1.83099999999996	-1.19316523552629e-09\\
1.83200000000001	-3.85749099789352e-10\\
1.83200000000003	-3.8574909978926e-10\\
1.83599999999998	-3.47144056013616e-10\\
1.83800000000001	-1.08812446331363e-09\\
1.83800000000003	-1.08812446331334e-09\\
1.83999999999999	-1.04801705039806e-09\\
1.84	-1.04801705039778e-09\\
1.84199999999996	-1.00850548589083e-09\\
1.84399999999992	-9.69584634865012e-10\\
1.84599999999999	-9.31249439191229e-10\\
1.846	-9.31249439190951e-10\\
1.84999999999992	-8.5631616113866e-10\\
1.85399999999984	-7.83666697228085e-10\\
1.85499999999998	-7.65856725505162e-10\\
1.855	-7.65856725504897e-10\\
1.85599999999998	-7.48186547336254e-10\\
1.856	-7.48186547336002e-10\\
1.85699999999999	-7.30655588745471e-10\\
1.85799999999997	-7.13263280069196e-10\\
1.85999999999995	-6.7889235666299e-10\\
1.85999999999998	-6.78892356662685e-10\\
1.86	-6.78892356662249e-10\\
1.86399999999995	-6.11789745279456e-10\\
1.86599999999999	-5.79049336649712e-10\\
1.866	-5.79049336649487e-10\\
1.86799999999998	-5.46843829463126e-10\\
1.868	-5.46843829462914e-10\\
1.86999999999998	-5.15169038306978e-10\\
1.87199999999996	-4.84020846722138e-10\\
1.87299999999998	-4.6864295809968e-10\\
1.873	-4.68642958099468e-10\\
1.87699999999996	-4.08161544910868e-10\\
1.87999999999999	-3.63798521271657e-10\\
1.88	-3.63798521271445e-10\\
1.88399999999997	-3.0595387422218e-10\\
1.88499999999998	-2.9172279355417e-10\\
1.885	-2.91722793553958e-10\\
1.886	-2.77582816134922e-10\\
1.88600000000002	-2.77582816134711e-10\\
1.88700000000002	-2.63533482559668e-10\\
1.88800000000002	-2.49574336364389e-10\\
1.88999999999998	-2.21924794894446e-10\\
1.89	-2.21924794890938e-10\\
1.892	-1.21521108037646e-09\\
1.89200000000002	-1.94630598129302e-10\\
1.89400000000002	-1.17479688026112e-09\\
1.89600000000003	-1.13486186051156e-09\\
1.898	-1.09540083114236e-09\\
1.89800000000002	-1.14844835335351e-10\\
1.9	-1.05596031901061e-09\\
1.90000000000002	-8.90457744950171e-11\\
1.902	-1.01664397069721e-09\\
1.90399999999998	-9.78003793660476e-10\\
1.906	-9.40034766223858e-10\\
1.90600000000002	-1.49692642909329e-11\\
1.90999999999998	-1.77744917194863e-09\\
1.91399999999995	-1.6790712990994e-09\\
1.91399999999997	7.61689909009846e-11\\
1.914	7.61689909018382e-11\\
1.91999999999998	-2.39490421344086e-09\\
1.92	1.38903510729869e-10\\
1.92499999999998	-1.83502833405737e-09\\
1.925	1.87562792356349e-10\\
1.92599999999998	-1.98073999170612e-10\\
1.926	1.96903653955318e-10\\
1.92699999999999	-1.85712743312003e-10\\
1.92799999999997	-1.73493532506948e-10\\
1.92999999999995	-5.35045380987388e-10\\
1.93199999999998	-5.05383166964008e-10\\
1.932	2.5023726571168e-10\\
1.93599999999995	-1.19339726111206e-09\\
1.9399999999999	-1.11842044532203e-09\\
1.93999999999999	3.14043205810103e-10\\
1.94	3.14043205835507e-10\\
1.94299999999998	-7.10029876486411e-10\\
1.943	3.35808509714089e-10\\
1.94599999999998	-6.65337537244654e-10\\
1.946	3.56404705500627e-10\\
1.94899999999998	-6.22107061344808e-10\\
1.95199999999997	-5.80325800848307e-10\\
1.95199999999998	3.94113523600189e-10\\
1.952	3.94113523600222e-10\\
1.95799999999997	-1.45228118743827e-09\\
1.95999999999998	-1.69009432416202e-10\\
1.96	4.37229904611378e-10\\
1.96599999999996	-1.29479945122035e-09\\
1.96599999999998	4.64244692171179e-10\\
1.966	4.64244692171186e-10\\
1.97199999999996	-1.18438724132286e-09\\
1.97199999999998	4.86732943022404e-10\\
1.972	4.8673294302247e-10\\
1.97799999999996	-1.07658748807198e-09\\
1.97799999999998	5.05071055552705e-10\\
1.978	5.05071055552811e-10\\
1.98	3.45722559993326e-12\\
1.98000000000002	5.10343140312142e-10\\
1.98200000000002	1.83661006012807e-11\\
1.98400000000002	3.27919149800416e-11\\
1.986	4.67365444567731e-11\\
1.98600000000002	5.23647129103827e-10\\
1.99000000000002	-3.93855053448252e-10\\
1.99400000000003	-3.4975458953519e-10\\
1.995	3.21275103910424e-10\\
1.99500000000001	5.36562650935227e-10\\
1.99999999999999	-5.00465746941175e-10\\
2	5.40098408597436e-10\\
2.00099999999997	3.39508600657629e-10\\
2.001	5.40494046136709e-10\\
2.00199999999999	3.42160817006872e-10\\
2.00299999999997	3.44702789249248e-10\\
2.00499999999995	1.55532736388552e-10\\
2.00599999999997	3.51668050382352e-10\\
2.006	5.40915370929688e-10\\
2.00999999999995	-1.9437667474932e-10\\
2.01199999999997	1.84969089843383e-10\\
2.012	5.37995667265196e-10\\
2.01299999999998	3.64075495834327e-10\\
2.01300000000001	5.37145587169076e-10\\
2.014	3.64966217855087e-10\\
2.01499999999999	3.65328648828226e-10\\
2.01699999999996	1.97310421021629e-10\\
2.01999999999997	3.57678051504797e-11\\
2.02	5.29481506099131e-10\\
2.02399999999995	-1.16018299289763e-10\\
2.02599999999997	2.08587743591559e-10\\
2.026	5.20757049553936e-10\\
2.02999999999995	-9.51446006267001e-11\\
2.03	5.13830290145818e-10\\
2.03399999999995	-8.27416055873091e-11\\
2.03599999999997	2.14876438817345e-10\\
2.036	5.01765465769542e-10\\
2.03999999999995	-6.63901989626997e-11\\
2.04	4.92599694903783e-10\\
2.04399999999995	-5.69830542299904e-11\\
2.04599999999997	2.14614202646169e-10\\
2.046	4.7715529394921e-10\\
2.04799999999997	2.13775943147601e-10\\
2.048	4.71552314950691e-10\\
2.04999999999996	2.12714159979309e-10\\
2.05199999999993	2.11513123848558e-10\\
2.05599999999987	-3.50797756200316e-11\\
2.05899999999997	8.85815182331342e-11\\
2.059	4.39182653275765e-10\\
2.05999999999997	3.21516871008118e-10\\
2.06	4.36145896354436e-10\\
2.06099999999999	3.19576788355997e-10\\
2.06199999999998	3.1761681912746e-10\\
2.06399999999995	2.02335231395043e-10\\
2.06499999999997	3.1161695537579e-10\\
2.065	4.20718639620258e-10\\
2.06599999999997	3.09576798881048e-10\\
2.066	4.17583792120367e-10\\
2.06699999999999	3.07516431530785e-10\\
2.06799999999998	3.05435786302493e-10\\
2.06999999999995	1.96469558096097e-10\\
2.07099999999997	2.9907150425946e-10\\
2.071	4.01656971848667e-10\\
2.07499999999995	-1.10268365659701e-11\\
2.07699999999997	1.88538831967901e-10\\
2.077	3.81975352761256e-10\\
2.07999999999997	8.84300249040841e-11\\
2.08	3.72028632389937e-10\\
2.08299999999998	8.53026618333945e-11\\
2.08599999999995	8.21952886571367e-11\\
2.086	3.52443810864193e-10\\
2.08799999999997	1.69484947791874e-10\\
2.088	3.46003866204566e-10\\
2.08999999999997	1.65978360285081e-10\\
2.09199999999993	1.62491141293726e-10\\
2.09399999999997	1.59022837933317e-10\\
2.094	3.26937301999805e-10\\
2.09799999999993	-1.29492731336157e-11\\
2.09999999999997	1.48726926690962e-10\\
2.1	3.08233352922731e-10\\
2.10399999999993	-1.47963249829159e-11\\
2.10599999999997	1.38585170842255e-10\\
2.106	2.89870140223303e-10\\
2.10999999999993	-1.6680501062952e-11\\
2.112	1.28585707171195e-10\\
2.11200000000003	2.71826183724235e-10\\
2.11599999999996	-2.02155779612636e-11\\
2.11699999999997	1.88218937982024e-10\\
2.117	2.57196210148585e-10\\
2.11999999999997	4.43463460193015e-11\\
2.12	2.48683541727873e-10\\
2.12299999999998	3.98752156108658e-11\\
2.12599999999995	3.55389594119329e-11\\
2.126	2.32243570776265e-10\\
2.12899999999997	3.13363091603024e-11\\
2.129	2.24311460965424e-10\\
2.13199999999996	2.7266036836187e-11\\
2.13499999999993	2.33269513509138e-11\\
2.13499999999997	2.09011394756084e-10\\
2.135	2.09011394756002e-10\\
2.13999999999997	-1.04756440159599e-10\\
2.14	1.96825350666942e-10\\
2.14499999999998	-1.06786593881901e-10\\
2.14599999999997	1.2561593542786e-10\\
2.146	1.82864102280858e-10\\
2.14699999999997	1.23732985458535e-10\\
2.14699999999999	4.04388899614441e-10\\
2.14799999999998	4.00582905344086e-10\\
2.14899999999997	3.4195186089354e-10\\
2.15099999999995	2.7995099342362e-10\\
2.15299999999999	1.64071917717908e-10\\
2.15300000000002	2.1846655666307e-10\\
2.15699999999997	4.3224515122446e-11\\
2.15999999999997	-1.8993061011283e-11\\
2.16	1.41649682039894e-10\\
2.16399999999995	-8.27111286871898e-11\\
2.16599999999997	1.80714307446359e-11\\
2.166	1.23091856882006e-10\\
2.16999999999995	-9.71797160828563e-11\\
2.17	1.11047632038145e-10\\
2.17399999999995	-1.06641209881153e-10\\
2.17499999999997	4.52262603419082e-11\\
2.175	9.63499456684534e-11\\
2.17899999999995	-1.18269471635513e-10\\
2.17999999999997	3.15839470060186e-11\\
2.18	8.2038458773493e-11\\
2.18399999999995	-1.29685868189358e-10\\
2.18599999999997	-3.41737678415005e-11\\
2.186	6.53581111221341e-11\\
2.187	1.30427096960641e-11\\
2.18700000000002	6.26291160985324e-11\\
2.18800000000002	1.04456054588456e-11\\
2.18800000000005	5.9914472472073e-11\\
2.18900000000004	7.89622234350324e-12\\
2.19000000000004	5.39793541937603e-12\\
2.19200000000003	-4.85083314379419e-11\\
2.19600000000001	-1.54537142428699e-10\\
2.2	-1.61873450445226e-10\\
2.20000000000003	2.8924639219383e-11\\
2.20399999999997	-1.68986178319919e-10\\
2.204	1.92472167025462e-11\\
2.20499999999997	-2.97914106928221e-11\\
2.205	1.68777001893658e-11\\
2.20599999999999	-3.19884914894426e-11\\
2.20600000000003	1.45279643794287e-11\\
2.20700000000002	-3.41673792485044e-11\\
2.20800000000001	-3.63281447432017e-11\\
2.20999999999998	-8.66631282881661e-11\\
2.21200000000003	-9.05671323740847e-11\\
2.21200000000006	8.40713799615916e-13\\
2.21600000000001	-1.89019291945635e-10\\
2.21800000000003	-1.01890172048006e-10\\
2.21800000000006	-1.21531230890564e-11\\
2.21999999999997	-1.05142449398113e-10\\
2.22	-1.63096658150947e-11\\
2.22199999999992	-1.07914119857747e-10\\
2.22399999999983	-1.10576296044245e-10\\
2.226	-1.131293239443e-10\\
2.22600000000003	-2.80529807286603e-11\\
2.22999999999986	-2.01755772589701e-10\\
2.23299999999997	-1.62070109049466e-10\\
2.233	-4.03907350848519e-11\\
2.23699999999983	-2.0486100438673e-10\\
2.23899999999997	-1.27085089504609e-10\\
2.239	-4.98120018618538e-11\\
2.24	-8.94797628964281e-11\\
2.24000000000003	-5.12795100062191e-11\\
2.24100000000003	-9.06288549743703e-11\\
2.24200000000003	-9.1750012122565e-11\\
2.24400000000004	-1.31244997084271e-10\\
2.246	-1.32722469624772e-10\\
2.24600000000003	-5.94726800155166e-11\\
2.25000000000004	-2.07480755044784e-10\\
2.252	-1.36517964934858e-10\\
2.25200000000003	-6.66236520787635e-11\\
2.25600000000004	-2.0743841340117e-10\\
2.25999999999997	-2.07132455875647e-10\\
2.26	-7.47045046957744e-11\\
2.26199999999997	-1.41164473993463e-10\\
2.262	-7.64874143131603e-11\\
2.26399999999996	-1.41839584931945e-10\\
2.26599999999993	-1.42428749517692e-10\\
2.26599999999997	-7.97701343979113e-11\\
2.266	-7.97701343979253e-11\\
2.26999999999994	-2.05011207497364e-10\\
2.272	-1.43681274951837e-10\\
2.27200000000003	-8.39895746386798e-11\\
2.27499999999997	-1.73498582978824e-10\\
2.275	-8.57835545998889e-11\\
2.27799999999994	-1.72918514836905e-10\\
2.27999999999997	-1.44151429733679e-10\\
2.28	-8.83075891592626e-11\\
2.28299999999994	-1.71542084112135e-10\\
2.28599999999989	-1.70470041833716e-10\\
2.28599999999997	-9.05700813705535e-11\\
2.286	-9.05700813690095e-11\\
2.28699999999997	-1.17037008326549e-10\\
2.287	-9.08660660095258e-11\\
2.28799999999999	-1.17126609007285e-10\\
2.28899999999997	-1.17245313879811e-10\\
2.29099999999995	-1.43137534213696e-10\\
2.291	-9.18965762129584e-11\\
2.29499999999995	-1.93484348849005e-10\\
2.29699999999997	-1.42584685444351e-10\\
2.297	-9.30413491115951e-11\\
2.29999999999997	-1.66713968471389e-10\\
2.3	-9.34336579693919e-11\\
2.30299999999998	-1.65763777897453e-10\\
2.30599999999995	-1.64714850559077e-10\\
2.306	-9.38586941174942e-11\\
2.30999999999997	-1.8651745667132e-10\\
2.31	-9.38758849348565e-11\\
2.31399999999997	-1.84270621364753e-10\\
2.31599999999997	-1.38043257869003e-10\\
2.316	-9.35024435822763e-11\\
2.31999999999997	-1.80580807247435e-10\\
2.32	-9.30410532513854e-11\\
2.32399999999997	-1.77954547645194e-10\\
2.32599999999997	-1.34132903458232e-10\\
2.326	-9.21510303360605e-11\\
2.32999999999997	-1.73899643168575e-10\\
2.33199999999997	-1.31525751317153e-10\\
2.332	-9.10225365073615e-11\\
2.33599999999997	-1.69701892368802e-10\\
2.33999999999994	-1.66821614988395e-10\\
2.34	-8.91446013923776e-11\\
2.34000000000003	-8.91446013916803e-11\\
2.34499999999997	-1.82255679982978e-10\\
2.345	-8.77526993810979e-11\\
2.34600000000003	-1.06015050041258e-10\\
2.34600000000006	-8.74540568585447e-11\\
2.34700000000009	-1.05597846010962e-10\\
2.34800000000012	-1.05174455316677e-10\\
2.34899999999997	-1.04744864182216e-10\\
2.349	-8.65174323740147e-11\\
2.35100000000006	-1.21984458476678e-10\\
2.35300000000011	-1.20863413513297e-10\\
2.35499999999997	-1.19719605539628e-10\\
2.355	-8.44601094777188e-11\\
2.35800000000003	-1.35430570815797e-10\\
2.35800000000006	-8.33388094769916e-11\\
2.35999999999997	-1.16727766713584e-10\\
2.36	-8.25665512028379e-11\\
2.36199999999992	-1.15449402366604e-10\\
2.36399999999983	-1.14167179749074e-10\\
2.36599999999997	-1.12880932234852e-10\\
2.366	-8.02005793060146e-11\\
2.36999999999983	-1.42491004538568e-10\\
2.37399999999966	-1.38930059330709e-10\\
2.37799999999997	-1.35365887386737e-10\\
2.378	-7.52379008750512e-11\\
2.37999999999997	-1.0375038936703e-10\\
2.38	-7.43796302839214e-11\\
2.38199999999997	-1.02425876067836e-10\\
2.38399999999995	-1.01095812160939e-10\\
2.38599999999997	-9.97600246006563e-11\\
2.386	-7.17491922222482e-11\\
2.38999999999995	-1.24635465468421e-10\\
2.39	-6.99480690271815e-11\\
2.39399999999995	-1.21039825326088e-10\\
2.396	-9.29891285710328e-11\\
2.39600000000002	-6.71727965834103e-11\\
2.39999999999997	-1.15368562182641e-10\\
2.4	-6.53076759140618e-11\\
2.40399999999995	-1.11266018035961e-10\\
2.40599999999997	-8.56197478154351e-11\\
2.406	-6.25650912207025e-11\\
2.40699999999997	-7.34323058901758e-11\\
2.407	-6.21142305195132e-11\\
2.40799999999998	-7.28441777111389e-11\\
2.40899999999997	-7.22581408234028e-11\\
2.41099999999995	-8.19943948468389e-11\\
2.41299999999997	-8.05611380336102e-11\\
2.413	-5.94451537161758e-11\\
2.41499999999997	-7.91372493226381e-11\\
2.415	-5.85688624925946e-11\\
2.41699999999997	-7.77225436679252e-11\\
2.41899999999995	-7.63168372261993e-11\\
2.41999999999997	-6.59443618549979e-11\\
2.42	-5.6406368635221e-11\\
2.42399999999995	-9.1513128673548e-11\\
2.426	-7.14651061934934e-11\\
2.42600000000003	-5.38626155165423e-11\\
2.427	-6.20456978348089e-11\\
2.42700000000003	-5.34438976193235e-11\\
2.42800000000002	-6.15293828591961e-11\\
2.429	-6.10484264290317e-11\\
2.43099999999998	-6.84642409822615e-11\\
2.43499999999993	-8.27368411953592e-11\\
2.43599999999997	-5.7734489503566e-11\\
2.436	-4.97479546706794e-11\\
2.43999999999997	-7.94616055967315e-11\\
2.44	-4.81461371687544e-11\\
2.44399999999998	-7.68867874784603e-11\\
2.446	-6.05783952246499e-11\\
2.44600000000003	-4.57880359150537e-11\\
2.448	-5.95604881767319e-11\\
2.44800000000002	-4.50134927840682e-11\\
2.44999999999999	-5.85500485773112e-11\\
2.45000000000002	-4.42445265806287e-11\\
2.45199999999998	-5.75469451177265e-11\\
2.45399999999995	-5.65510474305304e-11\\
2.45600000000002	-5.5562226093451e-11\\
2.45600000000005	-4.19700937330655e-11\\
2.45999999999998	-6.70895178030241e-11\\
2.46000000000001	-4.04874694392835e-11\\
2.46399999999994	-6.49563746893326e-11\\
2.46499999999997	-4.50570675214783e-11\\
2.465	-3.86835184112678e-11\\
2.46599999999998	-4.46609338623125e-11\\
2.46600000000001	-3.83291877002256e-11\\
2.46699999999999	-4.42671320582985e-11\\
2.46799999999998	-4.38756493195555e-11\\
2.46999999999996	-4.93071564443558e-11\\
2.47199999999998	-4.84584459711884e-11\\
2.47200000000001	-3.62474542819612e-11\\
2.47599999999996	-5.88398051194533e-11\\
2.47999999999991	-5.68917022080835e-11\\
2.47999999999997	-3.35866886395911e-11\\
2.48	-3.3586688639375e-11\\
2.48499999999997	-6.02465103138065e-11\\
2.485	-3.19881531290934e-11\\
2.48599999999997	-3.72113766839462e-11\\
2.486	-3.16742476281429e-11\\
2.48699999999999	-3.68616229623987e-11\\
2.48799999999998	-3.65139477060615e-11\\
2.48999999999995	-4.12491768869228e-11\\
2.49199999999997	-4.04939192730431e-11\\
2.492	-2.98305703956125e-11\\
2.494	-3.97477705812097e-11\\
2.49400000000002	-2.92321421898044e-11\\
2.49600000000002	-3.9011858893827e-11\\
2.49800000000001	-3.82869906632623e-11\\
2.49999999999997	-3.75730716852785e-11\\
2.5	-2.74951877281511e-11\\
2.50399999999999	-4.61123254521337e-11\\
2.50599999999997	-3.54960894780145e-11\\
2.506	-2.5844161454868e-11\\
2.50999999999999	-4.36769932065858e-11\\
2.51399999999998	-4.21118628542937e-11\\
2.51999999999997	-4.88166207116741e-11\\
2.52	-2.23152672012432e-11\\
2.52299999999997	-3.44224427516095e-11\\
2.523	-2.16163869174509e-11\\
2.52599999999996	-3.34519314894706e-11\\
2.526	-2.09372680742176e-11\\
2.52899999999997	-3.25046426192957e-11\\
2.53199999999993	-3.15802990582901e-11\\
2.53199999999997	-1.96375260817487e-11\\
2.532	-1.96375260817435e-11\\
2.53799999999993	-4.14614809771528e-11\\
2.53799999999997	-1.84145208596366e-11\\
2.538	-1.84145208596285e-11\\
2.53999999999997	-2.54383294324178e-11\\
2.54	-1.80249351250432e-11\\
2.54199999999998	-2.48905915847775e-11\\
2.54399999999995	-2.43546217593206e-11\\
2.54599999999997	-2.38303502876764e-11\\
2.546	-1.69212369435493e-11\\
2.54999999999995	-2.95594758801248e-11\\
2.55199999999997	-2.23270521827221e-11\\
2.552	-1.5914137924832e-11\\
2.55499999999997	-2.47591210239473e-11\\
2.555	-1.54464405875644e-11\\
2.55799999999998	-2.39495239419503e-11\\
2.55800000000001	-1.50024600253239e-11\\
2.55999999999997	-2.04824699117192e-11\\
2.56	-1.47195845741181e-11\\
2.56199999999997	-2.00494583156526e-11\\
2.56399999999994	-1.9627585573354e-11\\
2.56599999999997	-1.92167968597239e-11\\
2.566	-1.39334950587567e-11\\
2.56999999999994	-2.35531008588411e-11\\
2.56999999999997	-1.34612122511949e-11\\
2.57	-1.34612122511911e-11\\
2.57399999999994	-2.26415781955789e-11\\
2.57799999999988	-2.1913831883758e-11\\
2.57999999999997	-1.69395134024735e-11\\
2.58	-1.23951339267938e-11\\
2.58099999999997	-1.45395054742978e-11\\
2.581	-1.22953992082558e-11\\
2.58199999999998	-1.44223821815241e-11\\
2.58299999999997	-1.43065643366924e-11\\
2.58499999999995	-1.62672778299048e-11\\
2.58599999999997	-1.39669060059933e-11\\
2.586	-1.18152071906077e-11\\
2.58999999999995	-1.98778377498261e-11\\
2.59	-1.14530460394855e-11\\
2.59199999999997	-1.53833806001928e-11\\
2.592	-1.12792309150309e-11\\
2.59399999999997	-1.51429289507939e-11\\
2.59599999999995	-1.49077927340898e-11\\
2.59799999999997	-1.46779413917884e-11\\
2.598	-1.07865781468876e-11\\
2.59999999999997	-1.44650600756696e-11\\
2.6	-1.06307396860619e-11\\
2.60199999999997	-1.42679738524358e-11\\
2.60399999999995	-1.40737948172612e-11\\
2.60599999999997	-1.38824977346783e-11\\
2.606	-1.01777518281706e-11\\
2.60999999999995	-1.71709751157364e-11\\
2.61	-9.88767073706923e-12\\
2.61399999999994	-1.67251422554245e-11\\
2.61599999999997	-1.29683843845873e-11\\
2.616	-9.47003830318398e-12\\
2.61999999999994	-1.60805220616816e-11\\
2.61999999999997	-9.20306507082608e-12\\
2.62	-9.20306507082396e-12\\
2.62399999999995	-1.56665355285672e-11\\
2.62499999999997	-1.05383801399785e-11\\
2.625	-8.88198348051396e-12\\
2.62599999999997	-1.04663494837633e-11\\
2.626	-8.81943082252474e-12\\
2.62699999999999	-1.0394922104267e-11\\
2.62799999999998	-1.03240956820044e-11\\
2.62999999999995	-1.18030410126054e-11\\
2.63199999999997	-1.16470790185156e-11\\
2.632	-8.45559014703359e-12\\
2.63599999999995	-1.45031843900947e-11\\
2.63899999999997	-1.26767883006215e-11\\
2.639	-8.0507682869765e-12\\
2.63999999999997	-9.52041380977261e-12\\
2.64	-7.99426848379452e-12\\
2.64099999999999	-9.45639360464056e-12\\
2.64199999999998	-9.3927483214927e-12\\
2.64399999999995	-1.07693238629885e-11\\
2.64599999999997	-1.06292722245813e-11\\
2.646	-7.66208192322283e-12\\
2.64999999999995	-1.32910749340964e-11\\
2.65099999999997	-8.83647501320274e-12\\
2.651	-7.39398388163566e-12\\
2.65499999999995	-1.28822993892474e-11\\
2.6589999999999	-1.25640472302437e-11\\
2.65999999999997	-8.3089451491874e-12\\
2.66	-6.93066492197222e-12\\
2.66599999999997	-1.47612133021394e-11\\
2.666	-6.63503990348053e-12\\
2.66799999999997	-9.19389074886598e-12\\
2.668	-6.5387812264691e-12\\
2.66999999999996	-9.07252310067829e-12\\
2.67199999999993	-8.95260526789974e-12\\
2.67199999999996	-6.34961308479686e-12\\
2.672	-3.77217045901077e-12\\
2.67599999999993	-8.69189918770914e-12\\
2.67799999999997	-6.01744271839698e-12\\
2.678	-6.01744271838564e-12\\
2.67999999999997	-8.3864792842045e-12\\
2.68	-5.89009964378744e-12\\
2.68199999999998	-8.22336396948991e-12\\
2.68399999999995	-8.06183173133289e-12\\
2.68599999999997	-7.90186157701292e-12\\
2.686	-5.51568412299009e-12\\
2.68999999999995	-9.93659448636856e-12\\
2.69399999999989	-9.55595545486177e-12\\
2.69499999999997	-6.08337558984459e-12\\
2.695	-4.97470560285249e-12\\
2.69699999999997	-7.04902890953663e-12\\
2.697	-4.85773014063892e-12\\
2.69899999999996	-6.89871434871643e-12\\
2.69999999999997	-5.74997336293021e-12\\
2.7	-4.68441038613745e-12\\
2.70199999999997	-6.67588195902228e-12\\
2.70399999999993	-6.52905849718775e-12\\
2.70599999999997	-6.38359792019207e-12\\
2.706	-4.3453023277788e-12\\
2.70899999999997	-7.17458074458264e-12\\
2.709	-4.17941486415757e-12\\
2.71199999999996	-6.93711663233346e-12\\
2.71299999999997	-4.91937483921335e-12\\
2.713	-3.96192468327634e-12\\
2.71599999999997	-6.63530412479107e-12\\
2.71899999999993	-6.42498219376597e-12\\
2.71999999999997	-4.50720531546594e-12\\
2.72	-3.59163202644608e-12\\
2.72599999999993	-8.65920938431703e-12\\
2.726	-3.28458350013358e-12\\
2.72600000000002	-3.28458350010247e-12\\
2.72999999999997	-6.5570526742674e-12\\
2.73	-3.08500053434255e-12\\
2.73199999999999	-4.69010244168941e-12\\
2.73200000000002	-2.98670476235079e-12\\
2.73400000000002	-4.5713305776215e-12\\
2.73600000000001	-4.45374471660808e-12\\
2.73799999999999	-4.33732957715232e-12\\
2.73800000000002	-2.69764737295985e-12\\
2.73999999999997	-4.22100697098587e-12\\
2.74	-2.60356456627761e-12\\
2.74199999999995	-4.10512986820794e-12\\
2.7439999999999	-3.99111435564515e-12\\
2.74599999999997	-3.8789456158956e-12\\
2.746	-2.33126699923485e-12\\
2.7499999999999	-5.18492067508797e-12\\
2.7539999999998	-4.92817559826678e-12\\
2.75499999999997	-2.67069443358902e-12\\
2.755	-1.95025232028437e-12\\
2.75999999999997	-5.27272928615249e-12\\
2.76	-1.75247960166617e-12\\
2.76499999999998	-4.95016036283062e-12\\
2.76500000000001	-1.56443143506893e-12\\
2.76599999999997	-2.18922238111448e-12\\
2.766	-1.52797516822288e-12\\
2.76699999999999	-2.14789878117956e-12\\
2.76700000000002	-1.49190056411285e-12\\
2.76800000000001	-2.10697698568582e-12\\
2.76899999999999	-2.06645566448333e-12\\
2.77099999999997	-2.62698762363837e-12\\
2.77299999999999	-2.53841853876912e-12\\
2.77300000000002	-1.28340308446768e-12\\
2.77699999999997	-3.60239443242874e-12\\
2.77999999999997	-2.84450991398023e-12\\
2.78	-1.0579876668438e-12\\
2.78399999999995	-3.25284580926331e-12\\
2.784	-9.3801850930232e-13\\
2.786	-2.00937076272705e-12\\
2.78600000000003	-8.80418986193183e-13\\
2.78800000000004	-1.93457308860364e-12\\
2.79000000000004	-1.86149240000944e-12\\
2.792	-1.79011919933012e-12\\
2.79200000000003	-7.17071425332931e-13\\
2.79600000000004	-2.70715360096971e-12\\
2.79999999999997	-2.53991801979029e-12\\
2.8	-5.21068535705867e-13\\
2.80400000000001	-2.37986529486313e-12\\
2.80599999999997	-1.33756495436505e-12\\
2.806	-3.9018885746112e-13\\
2.81000000000001	-2.15307306346319e-12\\
2.81199999999997	-1.16836670785979e-12\\
2.812	-2.72956857408361e-13\\
2.81299999999997	-6.95993940001575e-13\\
2.813	-2.54714212151244e-13\\
2.81399999999998	-6.73805538277459e-13\\
2.81499999999997	-6.51963466802959e-13\\
2.81699999999995	-1.03780225929613e-12\\
2.81899999999997	-9.88072431088368e-13\\
2.819	-1.52225344389151e-13\\
2.81999999999997	-5.47923615795367e-13\\
2.82	-1.36299854033029e-13\\
2.82099999999999	-5.28145014361968e-13\\
2.82199999999998	-5.08708010707198e-13\\
2.82399999999995	-8.6997368763018e-13\\
2.82599999999997	-8.25208406319658e-13\\
2.826	-4.76357498317879e-14\\
2.82999999999995	-1.50104145497587e-12\\
2.83399999999991	-1.38875391169382e-12\\
2.83499999999997	-2.86840076660118e-13\\
2.835	6.33829189015703e-14\\
2.83999999999997	-1.57761795486365e-12\\
2.84	1.13758358956712e-13\\
2.84199999999997	-5.17215915864814e-13\\
2.842	1.31661120106373e-13\\
2.84399999999996	-4.84908031077393e-13\\
2.84599999999992	-4.53960929381201e-13\\
2.846	1.63626842757266e-13\\
2.84600000000003	1.63626842772232e-13\\
2.847	-1.32147894376083e-13\\
2.84700000000003	1.05653621820562e-12\\
2.84800000000002	1.33482827766973e-12\\
2.84900000000001	1.04393990363525e-12\\
2.85099999999998	7.48221993351625e-13\\
2.853	1.80787684690579e-13\\
2.85300000000003	4.57667046133079e-13\\
2.85699999999998	-3.78297696543108e-13\\
2.85999999999997	-6.47535154866975e-13\\
2.86	1.50548214016214e-13\\
2.86399999999995	-9.04817943062994e-13\\
2.86599999999997	-3.86660102990942e-13\\
2.866	1.1781370407569e-13\\
2.86999999999995	-8.91502271956084e-13\\
2.87	9.24762376173545e-14\\
2.87099999999997	-1.55138708170371e-13\\
2.871	8.57007425121975e-14\\
2.87199999999998	-1.60044765383521e-13\\
2.87299999999997	-1.65135689944571e-13\\
2.87499999999995	-4.10596505176173e-13\\
2.87699999999997	-4.1803622750216e-13\\
2.87699999999999	4.13251202302009e-14\\
2.87999999999997	-6.57321341681084e-13\\
2.88	1.67332328568077e-14\\
2.88299999999998	-6.65715255257634e-13\\
2.88599999999996	-6.75920875111207e-13\\
2.886	-3.7294651459122e-14\\
2.888	-4.72803118185875e-13\\
2.88800000000003	-5.67477769253518e-14\\
2.89000000000003	-4.85925602643766e-13\\
2.89200000000003	-5.00177268359231e-13\\
2.89600000000002	-9.27061417953836e-13\\
2.89999999999997	-9.45701620298344e-13\\
2.9	-1.78396911794343e-13\\
2.90499999999997	-1.15883650710468e-12\\
2.905	-2.30789854596921e-13\\
2.90599999999997	-4.22920319960549e-13\\
2.90599999999999	-2.4139491518167e-13\\
2.90699999999998	-4.32220894201229e-13\\
2.90799999999997	-4.41570626580679e-13\\
2.90999999999995	-6.3791937196108e-13\\
2.91199999999997	-6.54313523117376e-13\\
2.91199999999999	-3.05939556379715e-13\\
2.91599999999995	-1.03090251565401e-12\\
2.91799999999997	-7.0485762494665e-13\\
2.91799999999999	-3.7210633595205e-13\\
2.91999999999997	-7.21505885097358e-13\\
2.92	-3.94225916520446e-13\\
2.92199999999998	-7.37417609987499e-13\\
2.92399999999996	-7.52946014062153e-13\\
2.926	-7.68093115399539e-13\\
2.92600000000003	-4.58028346904246e-13\\
2.92899999999997	-9.43011704283389e-13\\
2.92899999999999	-4.88508696222303e-13\\
2.93199999999993	-9.59961672956247e-13\\
2.93499999999987	-9.76114232831854e-13\\
2.93499999999997	-5.46672230056534e-13\\
2.935	-5.46672230046899e-13\\
2.93999999999997	-1.28070720366679e-12\\
2.94	-5.92333083588621e-13\\
2.94499999999998	-1.29013333035998e-12\\
2.94599999999997	-7.70742468929157e-13\\
2.946	-6.43811086456556e-13\\
2.95099999999998	-1.2988753056468e-12\\
2.95199999999997	-8.10780999069755e-13\\
2.952	-6.91729406944058e-13\\
2.95699999999998	-1.30924559142912e-12\\
2.95799999999999	-8.48865056248382e-13\\
2.95800000000002	-7.35372440821434e-13\\
2.95999999999997	-9.73155336970867e-13\\
2.96	-7.48807568453943e-13\\
2.96199999999995	-9.82544792258951e-13\\
2.9639999999999	-9.91411319225363e-13\\
2.96599999999997	-9.99756070035138e-13\\
2.966	-7.8579974626468e-13\\
2.9699999999999	-1.2254332297515e-12\\
2.96999999999999	-8.07716040242296e-13\\
2.97000000000002	-8.07716040236105e-13\\
2.97399999999992	-1.23175313817803e-12\\
2.97499999999997	-9.3104801473833e-13\\
2.975	-8.32043258049043e-13\\
2.9789999999999	-1.23689392621906e-12\\
2.97999999999997	-9.47906270648711e-13\\
2.98	-8.52979735089961e-13\\
2.9839999999999	-1.23897283165318e-12\\
2.986	-1.05471107083197e-12\\
2.98600000000003	-8.73651481939518e-13\\
2.98699999999997	-9.65972054869099e-13\\
2.98699999999999	-8.76626216945616e-13\\
2.98799999999998	-9.68252823232331e-13\\
2.98899999999997	-9.70642524921671e-13\\
2.99099999999995	-1.0631695049577e-12\\
2.99299999999999	-1.06652847020774e-12\\
2.99300000000002	-8.92182048720759e-13\\
2.99699999999997	-1.24502099039743e-12\\
2.99999999999997	-1.16046518486251e-12\\
3	-9.05568536519689e-13\\
3.00399999999995	-1.24622656389645e-12\\
3.00599999999997	-1.07872308066029e-12\\
3.006	-9.1297631760269e-13\\
3.00999999999995	-1.24362597594206e-12\\
3.01	-9.15834364021822e-13\\
3.01399999999995	-1.2400273546812e-12\\
3.01599999999997	-1.07675793102344e-12\\
3.01599999999999	-9.17004436792576e-13\\
3.01999999999995	-1.23185871634544e-12\\
3.02	-9.15995252791709e-13\\
3.02399999999995	-1.22487074213455e-12\\
3.02599999999997	-1.06614597642944e-12\\
3.026	-9.12444743007586e-13\\
3.02799999999997	-1.06325601715827e-12\\
3.02799999999999	-9.10717530809682e-13\\
3.02999999999996	-1.06011368258923e-12\\
3.03199999999992	-1.05671856442804e-12\\
3.03599999999985	-1.1983367761283e-12\\
3.03999999999997	-1.18762011480494e-12\\
3.04	-8.94627741195185e-13\\
3.04499999999997	-1.24549513410475e-12\\
3.04499999999999	-8.85014143839358e-13\\
3.04599999999997	-9.54236785753853e-13\\
3.046	-8.82885032432956e-13\\
3.04699999999998	-9.51796000394979e-13\\
3.04799999999996	-9.49288500499154e-13\\
3.04999999999992	-1.01470354406676e-12\\
3.05199999999997	-1.00875086968258e-12\\
3.052	-8.68659762470507e-13\\
3.05599999999992	-1.13524154017514e-12\\
3.05799999999997	-9.89332830830625e-13\\
3.058	-8.51935291433423e-13\\
3.05999999999997	-9.81898697517292e-13\\
3.06	-8.45886431589109e-13\\
3.06199999999997	-9.73847609700701e-13\\
3.06399999999995	-9.65701804215157e-13\\
3.066	-9.57460222434431e-13\\
3.06600000000003	-8.27068550051395e-13\\
3.06999999999997	-1.06923778776098e-12\\
3.07399999999992	-1.04843812370398e-12\\
3.07399999999996	-8.00380823487133e-13\\
3.07399999999999	-8.00380823487e-13\\
3.07999999999997	-1.13795729419203e-12\\
3.07999999999999	-7.79136764485835e-13\\
3.08599999999997	-1.10008615194605e-12\\
3.08599999999999	-7.56812411279693e-13\\
3.09199999999997	-1.06151014319239e-12\\
3.09199999999999	-7.33381650164635e-13\\
3.09799999999997	-1.02314457104646e-12\\
3.09999999999997	-8.0293415331658e-13\\
3.1	-7.01395736814115e-13\\
3.10299999999997	-8.38893708888249e-13\\
3.10299999999999	-6.89395605406384e-13\\
3.10599999999996	-8.23552526358827e-13\\
3.106	-6.77387063380233e-13\\
3.10600000000003	-6.77387063379349e-13\\
3.10899999999999	-8.08242162720716e-13\\
3.11199999999996	-7.92958140787544e-13\\
3.11199999999999	-6.53330698090853e-13\\
3.11200000000003	-6.53330698090685e-13\\
3.11499999999997	-7.77695991491321e-13\\
3.115	-6.41275840404596e-13\\
3.11799999999995	-7.62451251388804e-13\\
3.11999999999997	-7.08227487778496e-13\\
3.12	-6.21132708366282e-13\\
3.12299999999995	-7.37069933782527e-13\\
3.12599999999989	-7.21849835629748e-13\\
3.12599999999995	-5.96855471875978e-13\\
3.126	-5.96855471875769e-13\\
3.127	-6.33786143678522e-13\\
3.12700000000003	-5.92796239817764e-13\\
3.12800000000003	-6.29455914469028e-13\\
3.12900000000003	-6.25197808865428e-13\\
3.13100000000003	-6.57008786771354e-13\\
3.13199999999997	-6.12485413063521e-13\\
3.13199999999999	-5.72604258120173e-13\\
3.13599999999999	-7.14090392128816e-13\\
3.13799999999997	-6.26193286243108e-13\\
3.13799999999999	-5.48659679073243e-13\\
3.13999999999997	-6.17484994495426e-13\\
3.14	-5.40742935277933e-13\\
3.14199999999998	-6.08817349168769e-13\\
3.14399999999996	-6.00189223860064e-13\\
3.14599999999997	-5.91599497257634e-13\\
3.146	-5.17172646607247e-13\\
3.14999999999996	-6.48205371503328e-13\\
3.15	-5.01598871732971e-13\\
3.15399999999996	-6.29801025126557e-13\\
3.15599999999997	-5.49187939809966e-13\\
3.156	-4.78427594548503e-13\\
3.15999999999996	-6.02454737484131e-13\\
3.16	-4.63206322634036e-13\\
3.16099999999997	-4.93777184696504e-13\\
3.16099999999999	-4.59449089842038e-13\\
3.16199999999996	-4.89847577363845e-13\\
3.16299999999992	-4.8593794668159e-13\\
3.16499999999985	-5.11935433132631e-13\\
3.16599999999997	-4.74327647463571e-13\\
3.166	-4.40945340959703e-13\\
3.16999999999986	-5.58155239032014e-13\\
3.17199999999997	-4.8409310580009e-13\\
3.172	-4.19346383466096e-13\\
3.17599999999986	-5.32650138656959e-13\\
3.17999999999972	-5.16075928689464e-13\\
3.17999999999997	-3.91534028540748e-13\\
3.18	-3.91534028533718e-13\\
3.18499999999997	-5.26530515814661e-13\\
3.185	-3.74700406788554e-13\\
3.18599999999997	-4.01245783998896e-13\\
3.186	-3.71382823924957e-13\\
3.18699999999997	-3.97778918188668e-13\\
3.18799999999994	-3.94329037761053e-13\\
3.18999999999989	-4.16849402445677e-13\\
3.18999999999994	-3.5827269998597e-13\\
3.18999999999999	-3.58272699985795e-13\\
3.19399999999988	-4.60635735183121e-13\\
3.19599999999999	-3.95736685256487e-13\\
3.19600000000002	-3.39076493498274e-13\\
3.198	-3.88835676473228e-13\\
3.19800000000003	-3.32799484806771e-13\\
3.19999999999998	-3.81937409175411e-13\\
3.20000000000001	-3.26598726658428e-13\\
3.20199999999996	-3.75057740180151e-13\\
3.20399999999991	-3.6827656688737e-13\\
3.20599999999998	-3.61593008033451e-13\\
3.20600000000001	-3.08546476491039e-13\\
3.20999999999991	-4.00811672245146e-13\\
3.21399999999982	-3.86634163966843e-13\\
3.21899999999997	-3.94245430903587e-13\\
3.21899999999999	-2.72180827911123e-13\\
3.21999999999997	-2.93413957449039e-13\\
3.22	-2.69534919597046e-13\\
3.22099999999998	-2.90613107121813e-13\\
3.22199999999996	-2.87834137059535e-13\\
3.22399999999992	-3.05694407576058e-13\\
3.22599999999997	-2.99941258859252e-13\\
3.226	-2.5410161083816e-13\\
3.22999999999992	-3.33853662000318e-13\\
3.23099999999999	-2.63793347857972e-13\\
3.23100000000002	-2.41810396143522e-13\\
3.23499999999994	-3.18602328200606e-13\\
3.23699999999999	-2.69865582839416e-13\\
3.23700000000002	-2.27729853045035e-13\\
3.23999999999997	-2.8265190307215e-13\\
3.24	-2.20995422051209e-13\\
3.24299999999996	-2.74071314118183e-13\\
3.24599999999991	-2.65756849552806e-13\\
3.24599999999997	-2.08276476930196e-13\\
3.246	-2.08276476929419e-13\\
3.24799999999997	-2.41430380586379e-13\\
3.24799999999999	-2.04256999266919e-13\\
3.24999999999996	-2.36608265543875e-13\\
3.25199999999992	-2.31899557793779e-13\\
3.25399999999997	-2.27303645323289e-13\\
3.25399999999999	-1.92851764801714e-13\\
3.25499999999998	-2.07934393869126e-13\\
3.255	-1.91045499726998e-13\\
3.25599999999998	-2.05931036679093e-13\\
3.25699999999997	-2.03955034375305e-13\\
3.25899999999993	-2.16303459599724e-13\\
3.25999999999997	-1.98190519179611e-13\\
3.26	-1.82416099963694e-13\\
3.26399999999993	-2.3688097374437e-13\\
3.26599999999997	-2.02064439684531e-13\\
3.266	-1.72937797743228e-13\\
3.267	-1.85685843054568e-13\\
3.26700000000003	-1.71450362873578e-13\\
3.26800000000003	-1.84057002840535e-13\\
3.26900000000003	-1.82498383856896e-13\\
3.27100000000003	-1.93288683711063e-13\\
3.27500000000003	-2.14107133152198e-13\\
3.27699999999997	-1.83758102348559e-13\\
3.27699999999999	-1.57338733571104e-13\\
3.27999999999997	-1.92231778646644e-13\\
3.28	-1.53347668438341e-13\\
3.28299999999998	-1.87466648691027e-13\\
3.28599999999996	-1.82821275207857e-13\\
3.286	-1.45693741394734e-13\\
3.28899999999999	-1.78294299624965e-13\\
3.28900000000002	-1.42028641339994e-13\\
3.29	-1.52729518170012e-13\\
3.29000000000003	-1.40830653244727e-13\\
3.29100000000001	-1.51449123813899e-13\\
3.29199999999999	-1.50180865219785e-13\\
3.29399999999996	-1.59298012263155e-13\\
3.296	-1.5666005489454e-13\\
3.29600000000003	-1.33888995302874e-13\\
3.29999999999996	-1.74079542430332e-13\\
3.3	-1.29465264943221e-13\\
3.30000000000003	-1.29465264943019e-13\\
3.30399999999996	-1.6868020672865e-13\\
3.30599999999997	-1.44416762969177e-13\\
3.30599999999999	-1.23067505059416e-13\\
3.30999999999992	-1.608571072341e-13\\
3.31199999999999	-1.37504159609607e-13\\
3.31200000000002	-1.16948520400448e-13\\
3.31599999999995	-1.53356986958555e-13\\
3.31999999999988	-1.48531992001759e-13\\
3.32	-1.09211199264964e-13\\
3.32000000000003	-1.09211199263985e-13\\
3.32499999999998	-1.52367208839734e-13\\
3.325	-1.04613411729004e-13\\
3.326	-1.13082987852265e-13\\
3.32600000000003	-1.03715402920959e-13\\
3.32700000000003	-1.12131749919555e-13\\
3.32800000000003	-1.11187899718928e-13\\
3.33000000000002	-1.1850970001867e-13\\
3.332	-1.16554638805686e-13\\
3.33200000000003	-9.84757448969402e-14\\
3.33499999999999	-1.22610965673786e-13\\
3.33500000000002	-9.59479444450622e-14\\
3.33799999999998	-1.19587780906271e-13\\
3.33999999999997	-1.08965806629287e-13\\
3.34	-9.1862509933257e-14\\
3.34299999999997	-1.14691996604597e-13\\
3.34599999999993	-1.11838840426127e-13\\
3.34599999999997	-8.71664080850153e-14\\
3.346	-8.71664080849836e-14\\
3.34699999999999	-9.45122222633732e-14\\
3.34700000000002	-8.64052585896066e-14\\
3.34800000000001	-9.36990091023892e-14\\
3.349	-9.28921189322143e-14\\
3.35099999999999	-9.92305001810872e-14\\
3.35499999999995	-1.11468211136999e-13\\
3.35999999999997	-1.14463585791645e-13\\
3.36	-7.70554075472866e-14\\
3.36399999999997	-1.03314411433914e-13\\
3.36399999999999	-7.43774538565361e-14\\
3.36599999999997	-8.72165521704969e-14\\
3.366	-7.30727349036817e-14\\
3.36799999999998	-8.57204639533612e-14\\
3.36999999999996	-8.42486309594066e-14\\
3.37199999999997	-8.28008619090491e-14\\
3.372	-6.92935836083699e-14\\
3.37599999999996	-9.32754120928191e-14\\
3.37799999999997	-7.86000723576616e-14\\
3.378	-6.57135330249202e-14\\
3.37999999999997	-7.72058691726405e-14\\
3.38	-6.4566828032594e-14\\
3.38199999999997	-7.57971063853606e-14\\
3.38399999999995	-7.44175712189603e-14\\
3.38599999999997	-7.30670843889427e-14\\
3.386	-6.12927437721738e-14\\
3.38999999999995	-8.19417746593684e-14\\
3.39299999999997	-7.40635246432677e-14\\
3.39299999999999	-5.77837753247655e-14\\
3.39499999999998	-6.73447486934976e-14\\
3.395	-5.68419462998957e-14\\
3.39699999999999	-6.61509905548832e-14\\
3.39899999999997	-6.49852064291771e-14\\
3.399	-5.5038461953307e-14\\
3.39999999999997	-5.94739738362124e-14\\
3.4	-5.460420795475e-14\\
3.40099999999998	-5.89774789281423e-14\\
3.40199999999996	-5.84877444025963e-14\\
3.40399999999991	-6.21921470483002e-14\\
3.40599999999997	-6.11231025383876e-14\\
3.406	-5.21371893630858e-14\\
3.40999999999991	-6.77809240283227e-14\\
3.41099999999997	-5.43817975059468e-14\\
3.411	-5.02612706366352e-14\\
3.41499999999991	-6.49179219960591e-14\\
3.41899999999982	-6.30029616606005e-14\\
3.41999999999997	-5.09527412599162e-14\\
3.42	-4.71612230700003e-14\\
3.42199999999997	-5.39880069994929e-14\\
3.42199999999999	-4.65072977337463e-14\\
3.42399999999996	-5.32109843773868e-14\\
3.42599999999992	-5.244736362982e-14\\
3.426	-4.52369555550114e-14\\
3.42600000000003	-4.5236955554806e-14\\
3.42999999999996	-5.80366042242556e-14\\
3.43	-4.40160770627545e-14\\
3.432	-5.02359285594466e-14\\
3.43200000000003	-4.34239873657817e-14\\
3.43400000000003	-4.95249399274085e-14\\
3.43600000000003	-4.88268740829919e-14\\
3.438	-4.81416403053081e-14\\
3.43800000000003	-4.17201977979044e-14\\
3.43999999999997	-4.74933347306347e-14\\
3.44	-4.11734238509551e-14\\
3.44199999999995	-4.68791151999077e-14\\
3.44399999999989	-4.62719587938804e-14\\
3.44599999999997	-4.56717866067572e-14\\
3.446	-3.95707515482639e-14\\
3.44999999999989	-5.05217585468183e-14\\
3.45099999999996	-4.12306268479612e-14\\
3.45099999999999	-3.82773016860747e-14\\
3.45499999999988	-4.89100533673665e-14\\
3.45699999999996	-4.24922416671042e-14\\
3.45699999999999	-3.67741224870076e-14\\
3.45999999999997	-4.44935268965124e-14\\
3.46	-3.60420479207536e-14\\
3.46299999999998	-4.36255293915531e-14\\
3.46499999999998	-4.0303934945666e-14\\
3.465	-3.48500936980052e-14\\
3.46599999999997	-3.73186314890864e-14\\
3.466	-3.46158584497376e-14\\
3.46699999999997	-3.70697824447931e-14\\
3.46799999999994	-3.68223804650551e-14\\
3.46999999999988	-3.89869765965944e-14\\
3.47199999999997	-3.84708182855281e-14\\
3.472	-3.32388977747347e-14\\
3.47599999999988	-4.26416717391587e-14\\
3.47999999999976	-4.15574339618595e-14\\
3.47999999999996	-3.14689672311791e-14\\
3.47999999999999	-3.14689672307317e-14\\
3.48599999999996	-4.49393925882093e-14\\
3.48599999999999	-3.01846403533248e-14\\
3.49199999999996	-4.32521183691779e-14\\
3.49199999999999	-2.89356011465898e-14\\
3.49799999999996	-4.16154173859564e-14\\
3.5	-3.1863423215057e-14\\
3.50000000000003	-2.73225879248332e-14\\
3.506	-3.9508518057262e-14\\
3.50600000000003	-2.61502728980495e-14\\
3.50899999999999	-3.21097150340545e-14\\
3.50900000000002	-2.55756628660609e-14\\
3.51199999999998	-3.14484218147496e-14\\
3.51200000000003	-2.50085263799733e-14\\
3.51499999999999	-2.44270197808805e-14\\
3.51799999999995	-3.00664995523341e-14\\
3.51799999999999	-3.00664995523259e-14\\
3.51800000000003	-2.38092951995358e-14\\
3.51999999999997	-2.74994112183259e-14\\
3.52	-2.3387866274126e-14\\
3.52199999999995	-2.70229130958682e-14\\
3.52399999999989	-2.65518502917624e-14\\
3.52599999999997	-2.6086161587853e-14\\
3.526	-2.21531410667991e-14\\
3.52999999999989	-2.90452099711726e-14\\
3.53399999999978	-2.80350443555098e-14\\
3.53499999999998	-2.22114989213182e-14\\
3.535	-2.03816640358505e-14\\
3.53799999999997	-2.52173120349695e-14\\
3.53799999999999	-1.98119741133519e-14\\
3.53999999999997	-2.29718177038948e-14\\
3.54	-1.94378138032946e-14\\
3.54199999999998	-2.25470000783887e-14\\
3.54399999999996	-2.21270360824139e-14\\
3.54599999999997	-2.1711871138286e-14\\
3.546	-1.83418386156201e-14\\
3.54999999999996	-2.42119839431989e-14\\
3.54999999999999	-1.76328032472904e-14\\
3.553	-2.1907402891221e-14\\
3.55300000000003	-1.71121112326474e-14\\
3.55600000000004	-2.12932174040339e-14\\
3.55900000000005	-2.07040050503725e-14\\
3.55999999999997	-1.74477912396346e-14\\
3.56	-1.59309180952013e-14\\
3.56599999999999	-2.38608295752951e-14\\
3.56600000000002	-1.49538566762011e-14\\
3.56699999999996	-1.62466699627958e-14\\
3.56699999999999	-1.47941019864768e-14\\
3.56799999999996	-1.60787885590618e-14\\
3.56899999999992	-1.59118174981691e-14\\
3.56999999999998	-1.57457513527976e-14\\
3.57	-1.43200289045181e-14\\
3.57199999999993	-1.68331792235802e-14\\
3.57299999999996	-1.52529286001201e-14\\
3.57299999999999	-1.38536347769287e-14\\
3.57499999999992	-1.63193821216039e-14\\
3.57699999999985	-1.59814507791357e-14\\
3.57899999999996	-1.56471481780559e-14\\
3.57899999999999	-1.29433400705087e-14\\
3.57999999999997	-1.41330427242212e-14\\
3.58	-1.27947001079575e-14\\
3.58099999999998	-1.39755088209228e-14\\
3.58199999999997	-1.38192522341404e-14\\
3.58399999999993	-1.4818623813833e-14\\
3.58599999999997	-1.44950045470672e-14\\
3.586	-1.19287121066698e-14\\
3.58999999999993	-1.63900317501035e-14\\
3.59399999999986	-1.57006103168892e-14\\
3.59599999999996	-1.29537427915835e-14\\
3.59599999999999	-1.05817961235263e-14\\
3.59999999999997	-1.47063717663278e-14\\
3.6	-1.00759482709081e-14\\
3.60399999999998	-1.40696392470834e-14\\
3.60499999999998	-1.05667278920375e-14\\
3.605	-9.46949856663885e-15\\
3.60599999999997	-1.0439696940352e-14\\
3.606	-9.35161695451429e-15\\
3.60699999999997	-1.03138284268394e-14\\
3.60799999999994	-1.01891182628875e-14\\
3.60799999999999	-9.11923119854815e-15\\
3.60999999999993	-1.10039997044369e-14\\
3.61199999999987	-1.07446382773249e-14\\
3.61399999999996	-1.04899662729269e-14\\
3.61399999999999	-8.44882514001266e-15\\
3.61799999999987	-1.20045652383793e-14\\
3.61999999999997	-9.75922000283225e-15\\
3.62	-7.81832426808279e-15\\
3.62399999999988	-1.1204346387118e-14\\
3.62499999999996	-8.24911092263936e-15\\
3.62499999999999	-7.32314850649961e-15\\
3.62599999999997	-8.14534721269286e-15\\
3.626	-7.22737329594706e-15\\
3.62699999999998	-8.04269263161294e-15\\
3.62799999999997	-7.94114384480016e-15\\
3.62999999999993	-8.63553682479498e-15\\
3.63199999999997	-8.42441482091926e-15\\
3.632	-6.67528096489026e-15\\
3.63599999999993	-9.73355552505876e-15\\
3.63999999999985	-9.28076673754237e-15\\
3.63999999999997	-5.99849487894968e-15\\
3.64	-5.99849487887193e-15\\
3.64599999999997	-1.02315173545567e-14\\
3.646	-5.53463840401274e-15\\
3.65199999999997	-9.53756503242924e-15\\
3.652	-5.10766051187573e-15\\
3.65399999999996	-6.39240700082074e-15\\
3.65399999999999	-4.97320530404213e-15\\
3.65599999999995	-6.2345811299837e-15\\
3.65799999999992	-6.08046963668799e-15\\
3.65999999999996	-5.93005249003094e-15\\
3.65999999999999	-4.5910747693776e-15\\
3.66399999999992	-6.95281255169045e-15\\
3.66599999999996	-5.50077402970768e-15\\
3.66599999999999	-4.24045402738453e-15\\
3.66999999999992	-6.46714729046627e-15\\
3.67399999999984	-6.16212850439661e-15\\
3.67499999999998	-4.34209309603827e-15\\
3.67500000000001	-3.772716564688e-15\\
3.67999999999997	-6.29541970834847e-15\\
3.68	-3.54264126079824e-15\\
3.68299999999996	-4.99172549108282e-15\\
3.68299999999999	-3.41470149302498e-15\\
3.68599999999995	-4.81590325386884e-15\\
3.686	-3.29429461021677e-15\\
3.68699999999998	-3.75079253387177e-15\\
3.68700000000001	-2.28415430482572e-15\\
3.68799999999998	-1.32441155958577e-15\\
3.68899999999995	-1.77375201431809e-15\\
3.6909999999999	-2.66345078419396e-15\\
3.69299999999998	-3.08261918568102e-15\\
3.693	-3.08261918568046e-15\\
3.6969999999999	-4.34911941980114e-15\\
3.69999999999997	-5.15854144948126e-15\\
3.7	-4.26749111131944e-15\\
3.7039999999999	-4.59614770198433e-15\\
3.70599999999997	-5.39945964355917e-15\\
3.706	-3.69678426353248e-15\\
3.7099999999999	-4.43849115255787e-15\\
3.70999999999998	-4.43849115257697e-15\\
3.71000000000001	-2.81327833197743e-15\\
3.71199999999996	-3.58371469259771e-15\\
3.71199999999999	-3.58371469259721e-15\\
3.71399999999995	-3.54980754494649e-15\\
3.71599999999991	-4.2988298535767e-15\\
3.71799999999996	-4.25624619314447e-15\\
3.71799999999999	-3.48764050631241e-15\\
3.71999999999997	-3.45937253558716e-15\\
3.72	-3.45937253558677e-15\\
3.72199999999998	-3.43297661766276e-15\\
3.72399999999997	-4.14032804441785e-15\\
3.72599999999997	-4.10561620118512e-15\\
3.726	-3.3857874626257e-15\\
3.728	-3.36498809237413e-15\\
3.72800000000003	-3.36498809237385e-15\\
3.73000000000004	-3.3467790720727e-15\\
3.73200000000004	-4.01612247696932e-15\\
3.73600000000004	-4.64431692331833e-15\\
3.74	-5.2489262409145e-15\\
3.74000000000003	-3.92366426762925e-15\\
3.74099999999996	-2.94936384563924e-15\\
3.74099999999999	-2.94936384563914e-15\\
3.74199999999996	-2.94568422873729e-15\\
3.74299999999992	-3.25687278290822e-15\\
3.74499999999985	-3.55865703201426e-15\\
3.745	-3.24616020088984e-15\\
3.74500000000003	-2.62764874710798e-15\\
3.746	-2.93293555170398e-15\\
3.74600000000003	-2.9329355517039e-15\\
3.747	-2.93024031058827e-15\\
3.74799999999997	-3.23164231521145e-15\\
3.74999999999991	-3.52477235592626e-15\\
3.752	-3.81239339028933e-15\\
3.75200000000003	-3.21517384845534e-15\\
3.75599999999991	-3.78253513337711e-15\\
3.758	-4.3494887357665e-15\\
3.75800000000003	-3.19664451165378e-15\\
3.75999999999997	-3.19084806488687e-15\\
3.76	-3.19084806488677e-15\\
3.76199999999995	-3.18418332208016e-15\\
3.76399999999989	-3.72304081700088e-15\\
3.76599999999997	-3.70669962853515e-15\\
3.766	-3.17075741148189e-15\\
3.76999999999989	-3.6741232951499e-15\\
3.76999999999996	-3.67412329516089e-15\\
3.76999999999999	-2.64967755540944e-15\\
3.77399999999988	-3.6416742491504e-15\\
3.77599999999999	-4.12367106594374e-15\\
3.77600000000002	-3.13658923485763e-15\\
3.77999999999991	-3.59320245899206e-15\\
3.77999999999996	-3.59320245899602e-15\\
3.78	-2.6612103491641e-15\\
3.78399999999989	-3.56099795300258e-15\\
3.78599999999997	-3.99734509814592e-15\\
3.786	-3.10146988457161e-15\\
3.78999999999989	-3.51281436898806e-15\\
3.79199999999997	-3.92246167657591e-15\\
3.792	-3.07989437165794e-15\\
3.79599999999989	-3.46874099945275e-15\\
3.79899999999996	-4.04891512394749e-15\\
3.79899999999999	-3.25385933879454e-15\\
3.79999999999997	-2.86390503704243e-15\\
3.8	-2.86390503704238e-15\\
3.80099999999999	-2.86205758618803e-15\\
3.80199999999997	-3.04885238502477e-15\\
3.80399999999993	-3.2287875911272e-15\\
3.80599999999997	-3.40375243823635e-15\\
3.806	-3.03378241280264e-15\\
3.80999999999993	-3.37495198111122e-15\\
3.81099999999996	-3.54725910076495e-15\\
3.81099999999999	-2.83706952252359e-15\\
3.81499999999992	-3.33666461071676e-15\\
3.81499999999998	-3.33666461072124e-15\\
3.81500000000001	-2.65566064188987e-15\\
3.81899999999993	-3.30418586873693e-15\\
3.81999999999997	-3.4624588619178e-15\\
3.82	-2.8044039711169e-15\\
3.82399999999993	-3.26124868283114e-15\\
3.826	-3.56105786917354e-15\\
3.82600000000003	-2.93120963747157e-15\\
3.82700000000001	-2.77227642703314e-15\\
3.82700000000003	-2.77227642703301e-15\\
3.82799999999998	-2.76755682745862e-15\\
3.82800000000001	-2.7675568274585e-15\\
3.82899999999997	-2.76307127405283e-15\\
3.82999999999993	-2.90800452322607e-15\\
3.83199999999986	-3.0460069715497e-15\\
3.83399999999998	-3.18159668262298e-15\\
3.83400000000001	-2.88580459903922e-15\\
3.83799999999985	-3.15245594400055e-15\\
3.84	-3.42792415786886e-15\\
3.84000000000003	-2.8490977975065e-15\\
3.84399999999988	-3.10559672646431e-15\\
3.846	-3.3723893854435e-15\\
3.84600000000003	-2.8082641765166e-15\\
3.84999999999988	-3.05491072643386e-15\\
3.84999999999998	-3.05491072644289e-15\\
3.85000000000001	-2.50482678935146e-15\\
3.85399999999985	-3.01896497701782e-15\\
3.85599999999998	-3.27198703855747e-15\\
3.85600000000001	-2.730905491686e-15\\
3.85699999999996	-2.58917456797298e-15\\
3.85699999999999	-2.58917456797275e-15\\
3.85799999999995	-2.58125341001651e-15\\
3.85899999999992	-2.70549390803755e-15\\
3.85999999999997	-2.6968701000983e-15\\
3.86	-2.56520699243235e-15\\
3.86199999999993	-2.67942962873161e-15\\
3.86399999999985	-2.9222622695358e-15\\
3.86599999999997	-2.90210343867315e-15\\
3.866	-2.64376928724138e-15\\
3.86899999999996	-2.74361653846906e-15\\
3.86899999999999	-2.74361653846879e-15\\
3.87199999999995	-2.71400529549503e-15\\
3.87499999999991	-3.05940647988632e-15\\
3.87999999999997	-3.24808220866833e-15\\
3.88	-2.87707114253638e-15\\
3.88499999999998	-2.81898733322442e-15\\
3.88500000000001	-2.81898733322409e-15\\
3.88599999999999	-2.33087221780855e-15\\
3.88600000000002	-2.33087221780827e-15\\
3.88700000000001	-2.32086689404951e-15\\
3.88799999999999	-2.42825960953871e-15\\
3.88999999999996	-2.52399576222524e-15\\
3.89199999999999	-2.61810456792188e-15\\
3.89200000000002	-2.38538731347783e-15\\
3.89599999999996	-2.57056920693193e-15\\
3.89799999999999	-2.77563652373993e-15\\
3.89800000000002	-2.31884231408313e-15\\
3.89999999999997	-2.29540062276054e-15\\
3.9	-2.29540062276019e-15\\
3.90199999999996	-2.27116864946141e-15\\
3.90399999999991	-2.46714825794296e-15\\
3.90599999999997	-2.4397506386802e-15\\
3.906	-2.2227899515005e-15\\
3.90999999999991	-2.38510596647418e-15\\
3.91399999999982	-2.74870430641389e-15\\
3.91499999999996	-2.41960946352056e-15\\
3.91499999999999	-2.01398843229969e-15\\
3.91999999999997	-2.34793138434307e-15\\
3.92	-2.34793138434266e-15\\
3.92499999999999	-2.27649982104309e-15\\
3.92599999999997	-2.35720302015002e-15\\
3.926	-1.88977219064085e-15\\
3.92699999999996	-1.87845699622743e-15\\
3.92699999999999	-1.87845699622711e-15\\
3.92799999999995	-1.8671364087515e-15\\
3.92899999999992	-1.94564114983236e-15\\
3.93099999999984	-2.01063901029227e-15\\
3.93299999999996	-2.07346778053436e-15\\
3.93299999999999	-1.89740183320907e-15\\
3.93699999999984	-2.02011514786384e-15\\
3.93999999999997	-2.2354626412895e-15\\
3.94	-1.89724167416051e-15\\
3.94399999999985	-1.92901021685189e-15\\
3.94399999999996	-1.92901021685736e-15\\
3.94399999999999	-1.60927085666936e-15\\
3.94599999999997	-1.7448187382365e-15\\
3.946	-1.74481873823617e-15\\
3.94799999999999	-1.72195248877346e-15\\
3.94999999999997	-1.85266054250226e-15\\
3.95199999999997	-1.82755102001402e-15\\
3.952	-1.67667004966216e-15\\
3.95499999999998	-1.71632958871835e-15\\
3.95500000000001	-1.71632958871802e-15\\
3.95799999999998	-1.68120331967763e-15\\
3.95999999999998	-1.80010366527604e-15\\
3.96	-1.58782341755867e-15\\
3.96299999999998	-1.62338305984241e-15\\
3.96599999999995	-1.79224402227617e-15\\
3.966	-1.58910848534137e-15\\
3.96600000000003	-1.39138010224562e-15\\
3.96700000000001	-1.44652997551935e-15\\
3.96700000000003	-1.44652997551906e-15\\
3.96800000000001	-1.43652655700863e-15\\
3.96899999999998	-1.49093761804886e-15\\
3.97099999999993	-1.53460299691321e-15\\
3.97299999999996	-1.57741019087961e-15\\
3.97299999999999	-1.45056711847818e-15\\
3.97699999999989	-1.53496700264346e-15\\
3.97899999999996	-1.63819796458505e-15\\
3.97899999999999	-1.39101183228301e-15\\
3.97999999999997	-1.32051150150112e-15\\
3.98	-1.32051150150085e-15\\
3.98099999999999	-1.31104169122553e-15\\
3.98199999999997	-1.36166551588575e-15\\
3.98399999999994	-1.40201153844691e-15\\
3.98599999999997	-1.441560667056e-15\\
3.986	-1.32296508309616e-15\\
3.98999999999994	-1.40092631959393e-15\\
3.98999999999997	-1.40092631959399e-15\\
3.99000000000001	-1.16972396687064e-15\\
3.99399999999994	-1.36080320341844e-15\\
3.99599999999998	-1.4547735808892e-15\\
3.99600000000001	-1.22822991508383e-15\\
3.99999999999994	-1.30154824924984e-15\\
3.99999999999997	-1.30154824924974e-15\\
4	-1.08193517864004e-15\\
4.00199999999993	-1.1728730633856e-15\\
4.00199999999999	-1.17287306338643e-15\\
4.00399999999992	-1.15478608977986e-15\\
4.00599999999986	-1.24376440904948e-15\\
4.00599999999993	-1.13688460993899e-15\\
4.006	-1.03115035211942e-15\\
4.00999999999987	-1.2062313071064e-15\\
4.01199999999995	-1.29235669633942e-15\\
4.012	-1.08426998684272e-15\\
4.01599999999987	-1.15136705970051e-15\\
4.01999999999973	-1.31721789913589e-15\\
4.01999999999995	-1.1157201024774e-15\\
4.02	-9.18483505630774e-16\\
4.02500000000001	-1.12082954823774e-15\\
4.02500000000006	-1.12082954823723e-15\\
4.02599999999995	-9.19940191050725e-16\\
4.026	-9.1994019105028e-16\\
4.02699999999996	-9.12167509174657e-16\\
4.02799999999992	-9.5156218691637e-16\\
4.02999999999985	-9.82591619178519e-16\\
4.03099999999993	-9.74477927954021e-16\\
4.03099999999999	-8.81459179793052e-16\\
4.03499999999984	-9.88324641128153e-16\\
4.03699999999994	-1.06361134106243e-15\\
4.03699999999999	-8.81458508471683e-16\\
4.03799999999995	-8.29153705674407e-16\\
4.038	-8.29153705673989e-16\\
4.03899999999996	-8.21801357544877e-16\\
4.03999999999991	-8.58673561816039e-16\\
4.04	-8.14467720942153e-16\\
4.04199999999991	-8.43566750992027e-16\\
4.04399999999982	-9.15596921495601e-16\\
4.046	-8.99735486334712e-16\\
4.04600000000006	-8.1398778278195e-16\\
4.04999999999988	-8.68669071244992e-16\\
4.05399999999969	-1.00417405651884e-15\\
4.05999999999994	-1.03482918627074e-15\\
4.06	-8.73653435273223e-16\\
};
\end{axis}
\end{tikzpicture}%
}
      \caption{The evolution of the difference in angular displacement between
        RM and EDF of pendulum $P_3$ for execution time $C_3 = 6$ ms.}
      \label{fig:02.6.6.3_diff}
    \end{figure}
  \end{minipage}
\end{minipage}
}

\noindent\makebox[\textwidth][c]{%
\begin{minipage}{\linewidth}
  \begin{minipage}{0.45\linewidth}
    \begin{figure}[H]\centering
      \scalebox{0.7}{% This file was created by matlab2tikz.
%
%The latest updates can be retrieved from
%  http://www.mathworks.com/matlabcentral/fileexchange/22022-matlab2tikz-matlab2tikz
%where you can also make suggestions and rate matlab2tikz.
%
\definecolor{mycolor1}{rgb}{0.00000,0.44700,0.74100}%
\definecolor{mycolor2}{rgb}{0.85000,0.32500,0.09800}%
%
\begin{tikzpicture}

\begin{axis}[%
width=4.133in,
height=3.26in,
at={(0.693in,0.44in)},
scale only axis,
xmin=0,
xmax=1,
xmajorgrids,
ymin=-0.15,
ymax=0.2,
ymajorgrids,
axis background/.style={fill=white}
]
\pgfplotsset{max space between ticks=50}
\addplot [color=mycolor1,solid,forget plot]
  table[row sep=crcr]{%
0	0.15314\\
3.15544362088405e-30	0.15314\\
0.000656101980281985	0.153143230512962\\
0.00393661188169191	0.153256312778436\\
0.00999999999999994	0.153891071773171\\
0.01	0.153891071773171\\
0.0199999999999999	0.150048203824684\\
0.02	0.150048203824684\\
0.0289999999999998	0.137414337712804\\
0.029	0.137414337712803\\
0.03	0.135470213386942\\
0.0300000000000002	0.135470213386942\\
0.0349999999999996	0.124799943342144\\
0.035	0.124799943342143\\
0.0399999999999993	0.112755527823867\\
0.04	0.112755527823865\\
0.0449999999999993	0.0993074487540181\\
0.0499999999999987	0.0844227483826206\\
0.05	0.0844227483826165\\
0.0500000000000004	0.0844227483826151\\
0.0579999999999996	0.0596508269858389\\
0.058	0.0596508269858376\\
0.0599999999999996	0.0535139519248594\\
0.06	0.0535139519248581\\
0.0619999999999995	0.0473946951543089\\
0.0639999999999991	0.0412906572523715\\
0.0679999999999982	0.0291186707010022\\
0.0699999999999991	0.0230459496574581\\
0.07	0.0230459496574554\\
0.0779999999999982	0.000115509555903549\\
0.0779999999999991	0.000115509555901112\\
0.078	0.000115509555898676\\
0.0799999999999991	-0.00520529938069268\\
0.08	-0.00520529938069501\\
0.0819999999999991	-0.0103656555449064\\
0.0839999999999982	-0.0153675824104684\\
0.0869999999999991	-0.0225776906978477\\
0.087	-0.0225776906978498\\
0.0899999999999991	-0.0294420913674593\\
0.09	-0.0294420913674613\\
0.0929999999999991	-0.0358989900454877\\
0.0959999999999982	-0.0418862328905422\\
0.0999999999999991	-0.0491477173641583\\
0.1	-0.0491477173641598\\
0.104999999999999	-0.0570807990847102\\
0.105	-0.0570807990847115\\
0.109999999999999	-0.0637610218754624\\
0.11	-0.0637610218754635\\
0.114999999999999	-0.0692474781164248\\
0.116	-0.0702078763842551\\
0.116000000000001	-0.0702078763842559\\
0.12	-0.0735963345728153\\
0.120000000000001	-0.073596334572816\\
0.124	-0.0762636029644891\\
0.127999999999999	-0.0782138647982409\\
0.129999999999999	-0.0789211179635345\\
0.130000000000001	-0.0789211179635351\\
0.135999999999999	-0.08020719761647\\
0.136000000000001	-0.0802071976164702\\
0.139999999999998	-0.0804343672206592\\
0.14	-0.0804343672206592\\
0.143999999999997	-0.0801578942840472\\
0.144999999999998	-0.0800100484516686\\
0.145	-0.0800100484516683\\
0.148999999999997	-0.0791033199779474\\
0.149999999999998	-0.0787976904879804\\
0.15	-0.0787976904879798\\
0.153999999999997	-0.0773811971506685\\
0.157999999999995	-0.0757016984773037\\
0.159999999999998	-0.0747625151777411\\
0.16	-0.0747625151777403\\
0.167999999999995	-0.0703339909543648\\
0.17	-0.0690567418759476\\
0.170000000000002	-0.0690567418759465\\
0.173999999999998	-0.0664014575579397\\
0.174	-0.0664014575579385\\
0.174999999999998	-0.0657273809182214\\
0.175	-0.0657273809182202\\
0.176	-0.065049070523484\\
0.177	-0.0643664598898951\\
0.179000000000001	-0.0629880698663664\\
0.179999999999998	-0.0622921553774208\\
0.18	-0.0622921553774196\\
0.184000000000001	-0.0594621041509566\\
0.188000000000002	-0.0565544783067479\\
0.189999999999998	-0.0550701547058472\\
0.19	-0.0550701547058459\\
0.193999999999998	-0.0521127014356976\\
0.194	-0.0521127014356963\\
0.197999999999998	-0.0492163971314518\\
0.199999999999998	-0.0477897527246364\\
0.2	-0.0477897527246351\\
0.202999999999998	-0.0456750957196064\\
0.203	-0.0456750957196052\\
0.205999999999998	-0.0435891540275711\\
0.208999999999997	-0.0415300875308923\\
0.209999999999998	-0.0408493896873756\\
0.21	-0.0408493896873744\\
0.215999999999997	-0.0369141020807761\\
0.22	-0.034442653538239\\
0.220000000000002	-0.0344426535382379\\
0.225999999999998	-0.0309539039491991\\
0.23	-0.0287683666824412\\
0.230000000000002	-0.0287683666824403\\
0.231999999999998	-0.0277201050108415\\
0.232	-0.0277201050108406\\
0.233999999999996	-0.0267059225018899\\
0.235999999999993	-0.0257254215228329\\
0.239999999999986	-0.0238639395822936\\
0.24	-0.0238639395822873\\
0.240000000000002	-0.0238639395822865\\
0.244999999999998	-0.0217199736534009\\
0.245	-0.0217199736534001\\
0.249999999999997	-0.0197743341989149\\
0.249999999999999	-0.0197743341989139\\
0.250000000000002	-0.019774334198913\\
0.251999999999996	-0.0190498245766449\\
0.252	-0.0190498245766436\\
0.253999999999995	-0.0183545927475019\\
0.255999999999989	-0.0176883661383964\\
0.259999999999979	-0.0164418950697132\\
0.259999999999995	-0.0164418950697086\\
0.26	-0.016441895069707\\
0.260999999999997	-0.0161480110554556\\
0.261	-0.0161480110554546\\
0.262	-0.0158611619153846\\
0.263	-0.0155813195204978\\
0.265	-0.0150425459404598\\
0.269	-0.0140479186003353\\
0.269999999999996	-0.0138163935677859\\
0.27	-0.0138163935677851\\
0.278	-0.0121608623944767\\
0.279999999999996	-0.0117992539565872\\
0.28	-0.0117992539565866\\
0.288	-0.0105576632685836\\
0.289999999999996	-0.0102978678801549\\
0.29	-0.0102978678801545\\
0.298	-0.0094028571170321\\
0.299999999999996	-0.00921126131825964\\
0.3	-0.00921126131825931\\
0.308	-0.00857125155207012\\
0.309999999999996	-0.00844252880878156\\
0.31	-0.00844252880878134\\
0.314999999999997	-0.00815634985129781\\
0.315	-0.00815634985129762\\
0.319	-0.00795606106076335\\
0.319000000000004	-0.00795606106076318\\
0.319999999999996	-0.00790992644388172\\
0.32	-0.00790992644388155\\
0.321	-0.0078653575850625\\
0.322	-0.0078223501159026\\
0.324	-0.00774100263846268\\
0.328	-0.0075968690952334\\
0.329999999999996	-0.00753402651682357\\
0.33	-0.00753402651682346\\
0.338	-0.00731091603026741\\
0.339000000000004	-0.00728524225051462\\
0.339000000000007	-0.00728524225051453\\
0.339999999999996	-0.00726005203286719\\
0.34	-0.0072600520328671\\
0.340999999999996	-0.00723534290961056\\
0.341999999999993	-0.00721111245776153\\
0.343999999999986	-0.00716407811562298\\
0.347999999999972	-0.00707565211048863\\
0.348	-0.00707565211048802\\
0.348000000000004	-0.00707565211048795\\
0.349999999999996	-0.00703422577719816\\
0.35	-0.00703422577719809\\
0.351999999999993	-0.00699355247184833\\
0.353999999999986	-0.00695253338627603\\
0.357999999999972	-0.00686939340626633\\
0.359999999999997	-0.00682723991434748\\
0.36	-0.00682723991434741\\
0.367999999999972	-0.00665434840384323\\
0.369999999999997	-0.00660997141115594\\
0.37	-0.00660997141115586\\
0.376999999999997	-0.006446757644875\\
0.377	-0.00644675764487491\\
0.379999999999997	-0.00637238566943073\\
0.38	-0.00637238566943065\\
0.382999999999997	-0.00629526613701222\\
0.384999999999997	-0.00624229319973811\\
0.385	-0.00624229319973802\\
0.387999999999997	-0.00616044199835192\\
0.39	-0.00610424880187564\\
0.390000000000004	-0.00610424880187554\\
0.393	-0.00601786612427311\\
0.395999999999997	-0.00592922472555357\\
0.397	-0.00589916219766626\\
0.397000000000004	-0.00589916219766615\\
0.399999999999997	-0.0058073871785191\\
0.4	-0.00580738717851899\\
0.402999999999993	-0.00571316776508011\\
0.405999999999986	-0.00561642084048825\\
0.405999999999997	-0.0056164208404879\\
0.406	-0.00561642084048778\\
0.409999999999997	-0.00548334541510819\\
0.41	-0.00548334541510807\\
0.413999999999997	-0.00534707708841722\\
0.417999999999993	-0.00520906329388098\\
0.419999999999997	-0.00513933431621019\\
0.42	-0.00513933431621006\\
0.427999999999993	-0.00485496423801058\\
0.429999999999997	-0.00478236881457504\\
0.43	-0.00478236881457491\\
0.434999999999997	-0.00460081789937476\\
0.435	-0.00460081789937463\\
0.439999999999997	-0.00442046671457951\\
0.44	-0.00442046671457939\\
0.444999999999997	-0.00424087326153496\\
0.449999999999993	-0.00406159740331977\\
0.45	-0.00406159740331952\\
0.450000000000004	-0.00406159740331939\\
0.454999999999997	-0.00388464242696522\\
0.455	-0.0038846424269651\\
0.459999999999993	-0.00371201736453094\\
0.46	-0.00371201736453071\\
0.463999999999997	-0.00357675021271568\\
0.464	-0.00357675021271557\\
0.467999999999997	-0.00344377110445797\\
0.469999999999997	-0.00337807426057455\\
0.47	-0.00337807426057443\\
0.473999999999997	-0.00324923018247093\\
0.477999999999993	-0.00312434644135401\\
0.479999999999997	-0.00306332825226448\\
0.48	-0.00306332825226437\\
0.487999999999993	-0.00282819266800841\\
0.489999999999997	-0.00277152783070031\\
0.49	-0.00277152783070021\\
0.492999999999997	-0.00268837657874228\\
0.493	-0.00268837657874219\\
0.495999999999997	-0.00260765371783775\\
0.498999999999993	-0.00252928803319682\\
0.499999999999997	-0.00250367794765325\\
0.5	-0.00250367794765316\\
0.505999999999993	-0.00235521701213284\\
0.509999999999993	-0.00226101849347193\\
0.51	-0.00226101849347176\\
0.512999999999993	-0.00219290696937763\\
0.513	-0.00219290696937747\\
0.515999999999993	-0.00212705890550972\\
0.518999999999986	-0.00206341621292921\\
0.519999999999993	-0.00204268230550604\\
0.52	-0.00204268230550589\\
0.521999999999993	-0.00200192274803122\\
0.522	-0.00200192274803108\\
0.523999999999993	-0.00196209425656451\\
0.524999999999993	-0.00194252426656321\\
0.525	-0.00194252426656308\\
0.526999999999993	-0.00190406320753698\\
0.528999999999986	-0.0018664948477686\\
0.529999999999993	-0.00184804081379888\\
0.53	-0.00184804081379874\\
0.533999999999986	-0.00177638202987842\\
0.537999999999972	-0.00170809081087881\\
0.539999999999993	-0.00167517436414704\\
0.54	-0.00167517436414693\\
0.547999999999972	-0.0015514055254998\\
0.549999999999993	-0.00152237689661286\\
0.55	-0.00152237689661275\\
0.550999999999993	-0.00150813397221752\\
0.551	-0.00150813397221742\\
0.551999999999993	-0.0014940599149228\\
0.552999999999987	-0.00148015334465862\\
0.554999999999973	-0.00145283722954938\\
0.558999999999945	-0.00140015595574187\\
0.559999999999993	-0.00138738453305394\\
0.56	-0.00138738453305385\\
0.567999999999945	-0.001290764355199\\
0.569999999999993	-0.0012681080442107\\
0.57	-0.00126810804421062\\
0.570999999999993	-0.00125698896066818\\
0.571	-0.0012569889606681\\
0.571999999999994	-0.00124599514542769\\
0.572999999999987	-0.00123512552046293\\
0.574999999999973	-0.00121375459208277\\
0.578999999999945	-0.00117245694661062\\
0.579999999999993	-0.00116242749231629\\
0.58	-0.00116242749231622\\
0.587999999999945	-0.00108628764628732\\
0.589999999999993	-0.00106835630058816\\
0.59	-0.0010683563005881\\
0.594999999999993	-0.00102518298860799\\
0.595	-0.00102518298860793\\
0.599999999999993	-0.000984176539447219\\
0.6	-0.000984176539447162\\
0.604999999999993	-0.000945236455672845\\
0.608999999999993	-0.000915507877246312\\
0.609	-0.00091550787724626\\
0.61	-0.000908267305150007\\
0.610000000000007	-0.000908267305149956\\
0.611000000000007	-0.000901096956710468\\
0.612000000000007	-0.000893991154167617\\
0.614000000000006	-0.000879970407259534\\
0.618000000000005	-0.000852673236240157\\
0.619999999999993	-0.000839386109424082\\
0.62	-0.000839386109424035\\
0.627999999999998	-0.000788526838264542\\
0.629999999999993	-0.000776359404581159\\
0.63	-0.000776359404581116\\
0.637999999999998	-0.000729611558851254\\
0.638000000000005	-0.000729611558851214\\
0.639999999999993	-0.000718374382523774\\
0.64	-0.000718374382523734\\
0.641999999999989	-0.000707308205907471\\
0.643999999999977	-0.000696408689799411\\
0.647999999999954	-0.000675092610098676\\
0.649999999999993	-0.00066466768887766\\
0.65	-0.000664667688877623\\
0.657999999999954	-0.000624453414342308\\
0.657999999999992	-0.00062445341434212\\
0.658000000000005	-0.000624453414342058\\
0.659999999999993	-0.000614753112835123\\
0.66	-0.000614753112835088\\
0.661999999999989	-0.000605186431259547\\
0.663999999999977	-0.000595749618389984\\
0.664999999999993	-0.000591078754592939\\
0.665	-0.000591078754592906\\
0.666999999999993	-0.000581829823308378\\
0.667	-0.000581829823308345\\
0.668999999999993	-0.000572701603768233\\
0.669999999999993	-0.000568181640729161\\
0.67	-0.000568181640729129\\
0.671999999999993	-0.000559235115436408\\
0.673999999999986	-0.000550415088342362\\
0.677999999999971	-0.000533140745379623\\
0.679999999999993	-0.000524679656571602\\
0.68	-0.000524679656571572\\
0.687999999999971	-0.000491932295087052\\
0.689999999999993	-0.000484003654470002\\
0.69	-0.000484003654469974\\
0.695999999999993	-0.00046087418664401\\
0.696	-0.000460874186643984\\
0.699999999999993	-0.000445995537281835\\
0.7	-0.000445995537281809\\
0.703999999999993	-0.000431522362865119\\
0.707999999999986	-0.000417431964163371\\
0.709999999999993	-0.000410523376917925\\
0.71	-0.000410523376917901\\
0.716	-0.000390396901469173\\
0.716000000000007	-0.000390396901469149\\
0.719999999999993	-0.000377476123497505\\
0.72	-0.000377476123497483\\
0.723999999999986	-0.000364929111947227\\
0.724999999999993	-0.000361848441560947\\
0.725000000000001	-0.000361848441560925\\
0.728999999999986	-0.000349741031181947\\
0.729999999999993	-0.000346766513055465\\
0.73	-0.000346766513055444\\
0.733999999999986	-0.000335100644214133\\
0.734999999999993	-0.000332242774229825\\
0.735	-0.000332242774229805\\
0.738999999999986	-0.00032103727951184\\
0.739999999999993	-0.000318291029310797\\
0.74	-0.000318291029310777\\
0.743999999999986	-0.000307518446958102\\
0.747999999999972	-0.000297073137188031\\
0.749999999999993	-0.000291968070496662\\
0.75	-0.000291968070496644\\
0.753999999999993	-0.000282006412962819\\
0.754	-0.000282006412962801\\
0.757999999999993	-0.000272378629803339\\
0.759999999999993	-0.000267685202435417\\
0.76	-0.000267685202435401\\
0.763999999999993	-0.000258530075133411\\
0.767999999999986	-0.000249672002432226\\
0.769999999999993	-0.000245350002246816\\
0.77	-0.000245350002246801\\
0.773999999999993	-0.000236927001864428\\
0.774000000000001	-0.000236927001864413\\
0.777999999999994	-0.000228797974828517\\
0.779999999999993	-0.000224839699566304\\
0.78	-0.00022483969956629\\
0.782999999999993	-0.000219030710109956\\
0.783	-0.000219030710109942\\
0.785999999999993	-0.000213371283756889\\
0.788999999999986	-0.000207856428032637\\
0.79	-0.000206049425123396\\
0.790000000000007	-0.000206049425123384\\
0.795999999999993	-0.000195544194518791\\
0.8	-0.000188852233065154\\
0.800000000000007	-0.000188852233065143\\
0.804999999999993	-0.000180819255107827\\
0.805	-0.000180819255107816\\
0.809999999999987	-0.000173136804447917\\
0.809999999999997	-0.000173136804447902\\
0.810000000000007	-0.000173136804447886\\
0.811999999999993	-0.000170158974392289\\
0.812	-0.000170158974392278\\
0.813999999999987	-0.000167235535321981\\
0.815999999999973	-0.000164365341073914\\
0.819999999999945	-0.00015878020642566\\
0.819999999999987	-0.000158780206425602\\
0.82	-0.000158780206425583\\
0.827999999999944	-0.000148200708436175\\
0.829999999999993	-0.000145672834812856\\
0.830000000000001	-0.000145672834812847\\
0.831999999999994	-0.000143190413239027\\
0.832000000000001	-0.000143190413239018\\
0.833999999999994	-0.000140753131532892\\
0.835999999999987	-0.000138360033999753\\
0.839999999999973	-0.000133702655540068\\
0.839999999999987	-0.000133702655540053\\
0.84	-0.000133702655540037\\
0.840999999999993	-0.000132564480459671\\
0.841000000000001	-0.000132564480459663\\
0.841999999999994	-0.000131436548631194\\
0.842999999999987	-0.0001303187494533\\
0.844999999999973	-0.000128113111807827\\
0.848999999999945	-0.000123818673087672\\
0.849999999999993	-0.000122768789122539\\
0.85	-0.000122768789122532\\
0.857999999999944	-0.000114702030170695\\
0.859999999999993	-0.000112774262092235\\
0.86	-0.000112774262092228\\
0.867999999999944	-0.000105396195871439\\
0.869999999999993	-0.000103631315024469\\
0.87	-0.000103631315024463\\
0.874999999999993	-9.93531334797321e-05\\
0.875000000000001	-9.93531334797261e-05\\
0.879999999999994	-9.5259379362353e-05\\
0.880000000000001	-9.52593793623474e-05\\
0.884999999999994	-9.13400198201427e-05\\
0.889999999999987	-8.75854495205424e-05\\
0.890000000000001	-8.7585449520532e-05\\
0.898999999999994	-8.12216795789393e-05\\
0.899000000000001	-8.12216795789345e-05\\
0.899999999999993	-8.05445188309035e-05\\
0.9	-8.05445188308987e-05\\
0.900999999999993	-7.98730955065059e-05\\
0.901999999999986	-7.92073437530437e-05\\
0.903999999999972	-7.7892594506006e-05\\
0.907999999999944	-7.53283202983093e-05\\
0.909999999999993	-7.40777899343446e-05\\
0.91	-7.40777899343402e-05\\
0.917999999999944	-6.92826039200973e-05\\
0.918999999999994	-6.87057273213319e-05\\
0.919000000000001	-6.87057273213278e-05\\
0.919999999999993	-6.81336642506684e-05\\
0.92	-6.81336642506644e-05\\
0.920999999999993	-6.75663586656561e-05\\
0.921999999999986	-6.70037549382955e-05\\
0.923999999999972	-6.58924329478781e-05\\
0.927999999999944	-6.37238152687967e-05\\
0.927999999999987	-6.37238152687741e-05\\
0.928000000000001	-6.37238152687666e-05\\
0.929999999999993	-6.26656693062966e-05\\
0.93	-6.26656693062929e-05\\
0.931999999999993	-6.16249210571799e-05\\
0.933999999999986	-6.06016735066236e-05\\
0.937999999999972	-5.86060825433679e-05\\
0.939999999999993	-5.76329566972932e-05\\
0.940000000000001	-5.76329566972898e-05\\
0.945000000000001	-5.52690413858537e-05\\
0.945000000000008	-5.52690413858504e-05\\
0.949999999999993	-5.29989476589604e-05\\
0.950000000000001	-5.29989476589573e-05\\
0.954999999999986	-5.08203058712376e-05\\
0.956999999999994	-4.9974146473985e-05\\
0.957000000000001	-4.9974146473982e-05\\
0.96	-4.87309705573583e-05\\
0.960000000000008	-4.87309705573554e-05\\
0.963000000000007	-4.75180976709576e-05\\
0.966000000000007	-4.63344578716468e-05\\
0.97	-4.47999440496539e-05\\
0.970000000000008	-4.47999440496513e-05\\
0.976000000000007	-4.25911829131623e-05\\
0.976999999999994	-4.22337744034521e-05\\
0.977000000000001	-4.22337744034496e-05\\
0.979999999999993	-4.11792181100863e-05\\
0.980000000000001	-4.11792181100839e-05\\
0.982999999999993	-4.01504451491753e-05\\
0.985999999999986	-3.91465479724907e-05\\
0.985999999999993	-3.91465479724883e-05\\
0.986000000000001	-3.91465479724859e-05\\
0.989999999999993	-3.78451864740688e-05\\
0.990000000000001	-3.78451864740665e-05\\
0.993999999999993	-3.65861076652143e-05\\
0.997999999999986	-3.53690088574584e-05\\
0.999999999999993	-3.4775602913599e-05\\
1	-3.47756029135969e-05\\
1.00799999999999	-3.24975514457578e-05\\
1.00999999999999	-3.19508152372063e-05\\
1.01	-3.19508152372025e-05\\
1.01499999999999	-3.0623662744294e-05\\
1.015	-3.06236627442903e-05\\
1.01999999999999	-2.93518723693235e-05\\
1.02	-2.93518723693199e-05\\
1.02499999999999	-2.81323272460508e-05\\
1.02999999999997	-2.69620385860408e-05\\
1.03	-2.69620385860343e-05\\
1.03499999999999	-2.58399073637635e-05\\
1.035	-2.58399073637604e-05\\
1.03999999999999	-2.47649529537662e-05\\
1.04	-2.47649529537632e-05\\
1.04399999999999	-2.39371825777321e-05\\
1.044	-2.39371825777292e-05\\
1.04799999999999	-2.31366189054387e-05\\
1.04999999999999	-2.27461456033682e-05\\
1.05	-2.27461456033655e-05\\
1.05399999999999	-2.19849465538573e-05\\
1.05799999999997	-2.12496843408677e-05\\
1.05999999999999	-2.08914176261406e-05\\
1.06	-2.08914176261381e-05\\
1.06799999999997	-1.95175300952296e-05\\
1.06999999999999	-1.91881797350223e-05\\
1.07	-1.918817973502e-05\\
1.07299999999999	-1.87046113248536e-05\\
1.073	-1.87046113248513e-05\\
1.07599999999999	-1.82335300099459e-05\\
1.07899999999997	-1.77745201964858e-05\\
1.07999999999999	-1.76241291933546e-05\\
1.08	-1.76241291933525e-05\\
1.08499999999999	-1.68911056977018e-05\\
1.085	-1.68911056976997e-05\\
1.08999999999999	-1.61883286004625e-05\\
1.09	-1.61883286004605e-05\\
1.09299999999999	-1.57807749244403e-05\\
1.093	-1.57807749244384e-05\\
1.09599999999999	-1.53837565637336e-05\\
1.09899999999997	-1.49969232861895e-05\\
1.09999999999999	-1.48701829763446e-05\\
1.1	-1.48701829763428e-05\\
1.10199999999999	-1.4619933839549e-05\\
1.102	-1.46199338395473e-05\\
1.10399999999999	-1.43739117687079e-05\\
1.10599999999997	-1.41320203057708e-05\\
1.10999999999994	-1.36602514504841e-05\\
1.10999999999999	-1.36602514504791e-05\\
1.11	-1.36602514504774e-05\\
1.11799999999994	-1.27640155396916e-05\\
1.11999999999999	-1.25494841871147e-05\\
1.12	-1.25494841871132e-05\\
1.12799999999994	-1.1726945276252e-05\\
1.12999999999999	-1.15298040666984e-05\\
1.13	-1.1529804066697e-05\\
1.13099999999999	-1.14324769036611e-05\\
1.131	-1.14324769036597e-05\\
1.13199999999999	-1.13359970733076e-05\\
1.13299999999999	-1.12403551151518e-05\\
1.13499999999997	-1.10515474010043e-05\\
1.13899999999994	-1.0683579376896e-05\\
1.13999999999999	-1.05935445346036e-05\\
1.14	-1.05935445346023e-05\\
1.14799999999994	-9.90008386585161e-06\\
1.14999999999999	-9.73385729902974e-06\\
1.15	-9.73385729902857e-06\\
1.15099999999999	-9.65178914865432e-06\\
1.151	-9.65178914865316e-06\\
1.15199999999999	-9.57043341782454e-06\\
1.15299999999999	-9.48978212907315e-06\\
1.15499999999997	-9.33056132740966e-06\\
1.155	-9.33056132740744e-06\\
1.15899999999997	-9.0202297548739e-06\\
1.15999999999999	-8.94429199936458e-06\\
1.16	-8.94429199936351e-06\\
1.16399999999997	-8.64689822972458e-06\\
1.16799999999994	-8.35932847993222e-06\\
1.16999999999999	-8.21908608014879e-06\\
1.17	-8.2190860801478e-06\\
1.17799999999994	-7.68155531810598e-06\\
1.17999999999999	-7.55283730330737e-06\\
1.18	-7.55283730330646e-06\\
1.18799999999994	-7.05910955566715e-06\\
1.18899999999999	-6.99967230567464e-06\\
1.189	-6.9996723056738e-06\\
1.18999999999999	-6.94072173435739e-06\\
1.19	-6.94072173435655e-06\\
1.19099999999999	-6.88226694406534e-06\\
1.19199999999999	-6.82431708040442e-06\\
1.19399999999997	-6.70990946314751e-06\\
1.19799999999994	-6.48690668950198e-06\\
1.19999999999999	-6.37822409794813e-06\\
1.2	-6.37822409794737e-06\\
1.20799999999994	-5.96129973912617e-06\\
1.20999999999999	-5.86131582659923e-06\\
1.21	-5.86131582659853e-06\\
1.21799999999994	-5.47802118299907e-06\\
1.21799999999997	-5.47802118299778e-06\\
1.218	-5.4780211829965e-06\\
1.21999999999999	-5.38622169400624e-06\\
1.22	-5.3862216940056e-06\\
1.22199999999999	-5.29595896804445e-06\\
1.22399999999997	-5.20719761178701e-06\\
1.22499999999999	-5.16336904982906e-06\\
1.225	-5.16336904982844e-06\\
1.22899999999997	-4.99163575668567e-06\\
1.22999999999999	-4.94957663239371e-06\\
1.23	-4.94957663239311e-06\\
1.23399999999997	-4.78489230637588e-06\\
1.23799999999995	-4.62577061555543e-06\\
1.23799999999997	-4.62577061555438e-06\\
1.238	-4.62577061555334e-06\\
1.23999999999999	-4.54821743614228e-06\\
1.24	-4.54821743614173e-06\\
1.24199999999999	-4.47196199539983e-06\\
1.24399999999997	-4.39697439246177e-06\\
1.24699999999999	-4.28680600665764e-06\\
1.247	-4.28680600665712e-06\\
1.24999999999999	-4.17932703682658e-06\\
1.25	-4.17932703682608e-06\\
1.25299999999999	-4.07452898997656e-06\\
1.25599999999997	-3.9724057379344e-06\\
1.25999999999999	-3.84024681488454e-06\\
1.26	-3.84024681488407e-06\\
1.26599999999997	-3.65017309488042e-06\\
1.26999999999999	-3.52861056446903e-06\\
1.27	-3.5286105644686e-06\\
1.27599999999997	-3.35381326662095e-06\\
1.276	-3.35381326662018e-06\\
1.28	-3.24217652117616e-06\\
1.28000000000001	-3.24217652117577e-06\\
1.28400000000002	-3.13425045416107e-06\\
1.28800000000002	-3.02986579801481e-06\\
1.29	-2.97895012584763e-06\\
1.29000000000001	-2.97895012584728e-06\\
1.29499999999999	-2.85539628586095e-06\\
1.295	-2.85539628586061e-06\\
1.29599999999999	-2.83131820547539e-06\\
1.296	-2.83131820547505e-06\\
1.29699999999999	-2.80744546673632e-06\\
1.29799999999999	-2.78377572965809e-06\\
1.29999999999997	-2.73703600063049e-06\\
1.3	-2.73703600062983e-06\\
1.30399999999997	-2.64589179300956e-06\\
1.30499999999999	-2.62357919660366e-06\\
1.305	-2.62357919660335e-06\\
1.30899999999997	-2.53615702186708e-06\\
1.30999999999999	-2.51474782049202e-06\\
1.31	-2.51474782049172e-06\\
1.31399999999997	-2.43092538025907e-06\\
1.31799999999994	-2.34994474738521e-06\\
1.31999999999999	-2.31048024809993e-06\\
1.32	-2.31048024809965e-06\\
1.32799999999994	-2.15910301963108e-06\\
1.32999999999999	-2.12280471831712e-06\\
1.33	-2.12280471831687e-06\\
1.33399999999999	-2.05203965944856e-06\\
1.334	-2.05203965944831e-06\\
1.33799999999999	-1.98367576781429e-06\\
1.33999999999999	-1.95036060996255e-06\\
1.34	-1.95036060996232e-06\\
1.34399999999999	-1.88539855919184e-06\\
1.34799999999997	-1.82257631940081e-06\\
1.35	-1.79193670998032e-06\\
1.35000000000001	-1.79193670998011e-06\\
1.354	-1.73220457150146e-06\\
1.35400000000001	-1.73220457150125e-06\\
1.358	-1.67450022506174e-06\\
1.35999999999999	-1.64638007083684e-06\\
1.36	-1.64638007083664e-06\\
1.36299999999999	-1.60508378908813e-06\\
1.363	-1.60508378908793e-06\\
1.36499999999999	-1.57812649371562e-06\\
1.365	-1.57812649371543e-06\\
1.36699999999999	-1.55161568949674e-06\\
1.36899999999997	-1.52554098282752e-06\\
1.36999999999999	-1.51266396488851e-06\\
1.37	-1.51266396488833e-06\\
1.37399999999997	-1.46224911178351e-06\\
1.37799999999995	-1.41354595085914e-06\\
1.37999999999999	-1.38981228194375e-06\\
1.38	-1.38981228194358e-06\\
1.38799999999994	-1.29878171355054e-06\\
1.38999999999999	-1.27695548636117e-06\\
1.39	-1.27695548636102e-06\\
1.39199999999999	-1.25549575148692e-06\\
1.392	-1.25549575148677e-06\\
1.39399999999998	-1.23440535266576e-06\\
1.39599999999997	-1.21367602004296e-06\\
1.39999999999994	-1.17326818392361e-06\\
1.39999999999999	-1.17326818392313e-06\\
1.4	-1.17326818392299e-06\\
1.40799999999994	-1.09643681678172e-06\\
1.41	-1.07801492397606e-06\\
1.41000000000001	-1.07801492397593e-06\\
1.41200000000001	-1.05990226635675e-06\\
1.41200000000003	-1.05990226635663e-06\\
1.41400000000003	-1.04210122593865e-06\\
1.41600000000003	-1.02460482267274e-06\\
1.41999999999999	-9.90498606398203e-07\\
1.42	-9.90498606398084e-07\\
1.42099999999999	-9.82151873289738e-07\\
1.421	-9.8215187328962e-07\\
1.42199999999999	-9.73875421974345e-07\\
1.42299999999999	-9.65668441242658e-07\\
1.42499999999997	-9.49459680782197e-07\\
1.42899999999994	-9.17840861158975e-07\\
1.42999999999999	-9.10098023260279e-07\\
1.43	-9.10098023260169e-07\\
1.43499999999999	-8.72367259820766e-07\\
1.435	-8.72367259820661e-07\\
1.43999999999998	-8.36225015465839e-07\\
1.44	-8.3622501546572e-07\\
1.44499999999999	-8.01582713850155e-07\\
1.44999999999997	-7.68355455687567e-07\\
1.44999999999998	-7.6835545568747e-07\\
1.45	-7.68355455687373e-07\\
1.45999999999997	-7.05993422014959e-07\\
1.45999999999999	-7.05993422014856e-07\\
1.46	-7.05993422014771e-07\\
1.46999999999997	-6.48697016418089e-07\\
1.46999999999998	-6.48697016418003e-07\\
1.47	-6.48697016417917e-07\\
1.47899999999998	-6.01112519549728e-07\\
1.479	-6.01112519549656e-07\\
1.47999999999999	-5.9604819425395e-07\\
1.48	-5.96048194253878e-07\\
1.48099999999999	-5.91026586032959e-07\\
1.48199999999999	-5.86047202371303e-07\\
1.48399999999997	-5.76213160690856e-07\\
1.48799999999994	-5.57030570086589e-07\\
1.49	-5.47674500032706e-07\\
1.49000000000001	-5.47674500032641e-07\\
1.49799999999996	-5.11811586780121e-07\\
1.49900000000001	-5.07499306501934e-07\\
1.49900000000003	-5.07499306501873e-07\\
1.49999999999999	-5.03223503581482e-07\\
1.5	-5.03223503581421e-07\\
1.50099999999999	-4.98983759137381e-07\\
1.50199999999999	-4.94779657435136e-07\\
1.50399999999997	-4.86476737526247e-07\\
1.50499999999999	-4.82377105531803e-07\\
1.505	-4.82377105531745e-07\\
1.50799999999998	-4.70280709691595e-07\\
1.508	-4.70280709691539e-07\\
1.50999999999999	-4.62381252114664e-07\\
1.51	-4.62381252114608e-07\\
1.51199999999999	-4.5461415182675e-07\\
1.51399999999998	-4.46980496605778e-07\\
1.51799999999995	-4.32101601373823e-07\\
1.51999999999999	-4.2485052762974e-07\\
1.52	-4.24850527629689e-07\\
1.52799999999995	-3.97036278163996e-07\\
1.52999999999999	-3.90366596075838e-07\\
1.53	-3.90366596075791e-07\\
1.53699999999998	-3.67901300154341e-07\\
1.537	-3.67901300154297e-07\\
1.53999999999999	-3.58678686899469e-07\\
1.54	-3.58678686899426e-07\\
1.54299999999999	-3.49687442931377e-07\\
1.54599999999997	-3.40919636321779e-07\\
1.54999999999999	-3.2956345586528e-07\\
1.55	-3.29563455865241e-07\\
1.55599999999997	-3.132360262e-07\\
1.55699999999998	-3.10595875481158e-07\\
1.557	-3.10595875481121e-07\\
1.55999999999999	-3.02809221447742e-07\\
1.56	-3.02809221447706e-07\\
1.56299999999999	-2.95217944971359e-07\\
1.56599999999998	-2.8781534928794e-07\\
1.566	-2.87815349287884e-07\\
1.56999999999999	-2.78227461571781e-07\\
1.57	-2.78227461571747e-07\\
1.57399999999999	-2.68958350615977e-07\\
1.57499999999999	-2.66690725288649e-07\\
1.575	-2.66690725288617e-07\\
1.57899999999999	-2.57812165579742e-07\\
1.57999999999999	-2.55639423307626e-07\\
1.58	-2.55639423307595e-07\\
1.58399999999999	-2.47129655801081e-07\\
1.58799999999998	-2.38899864641678e-07\\
1.58999999999999	-2.34885909832709e-07\\
1.59	-2.3488590983268e-07\\
1.59499999999998	-2.25145975673529e-07\\
1.595	-2.25145975673502e-07\\
1.59999999999998	-2.15815938359409e-07\\
1.6	-2.15815938359376e-07\\
1.60499999999998	-2.06872932110389e-07\\
1.60999999999997	-1.98295039940271e-07\\
1.60999999999998	-1.98295039940245e-07\\
1.61	-1.9829503994022e-07\\
1.61499999999998	-1.90072312831009e-07\\
1.615	-1.90072312830986e-07\\
1.61999999999998	-1.82195674957295e-07\\
1.62	-1.82195674957267e-07\\
1.624	-1.76130548918672e-07\\
1.62400000000001	-1.7613054891865e-07\\
1.62800000000001	-1.70265037423448e-07\\
1.62999999999999	-1.67404250529958e-07\\
1.63	-1.67404250529938e-07\\
1.634	-1.61826979512615e-07\\
1.638	-1.56438836596244e-07\\
1.63999999999999	-1.53813036000193e-07\\
1.64	-1.53813036000175e-07\\
1.64499999999998	-1.47439418189617e-07\\
1.645	-1.47439418189599e-07\\
1.64999999999998	-1.41326074765387e-07\\
1.65	-1.41326074765368e-07\\
1.65299999999998	-1.37779575105541e-07\\
1.653	-1.37779575105524e-07\\
1.65599999999998	-1.34323906131562e-07\\
1.65899999999997	-1.30956019404134e-07\\
1.65999999999999	-1.29852381958795e-07\\
1.66	-1.29852381958779e-07\\
1.66599999999997	-1.23422443326495e-07\\
1.67	-1.19310911551141e-07\\
1.67000000000002	-1.19310911551127e-07\\
1.673	-1.16316963685919e-07\\
1.67300000000001	-1.16316963685905e-07\\
1.676	-1.13399692615786e-07\\
1.67899999999998	-1.10556524824044e-07\\
1.67999999999998	-1.09624834266694e-07\\
1.68	-1.0962483426668e-07\\
1.68199999999998	-1.07784952156713e-07\\
1.682	-1.077849521567e-07\\
1.68399999999998	-1.05975802261697e-07\\
1.68599999999997	-1.04196675266974e-07\\
1.68999999999994	-1.00725711439961e-07\\
1.68999999999998	-1.00725711439922e-07\\
1.69	-1.0072571143991e-07\\
1.69799999999994	-9.41284725520489e-08\\
1.69999999999998	-9.25486786618495e-08\\
1.7	-9.25486786618384e-08\\
1.70799999999994	-8.64889798603042e-08\\
1.70999999999998	-8.50359546360451e-08\\
1.71	-8.50359546360348e-08\\
1.711	-8.43185188015419e-08\\
1.71100000000001	-8.43185188015318e-08\\
1.71200000000001	-8.36072925139183e-08\\
1.71300000000001	-8.29022060327357e-08\\
1.715	-8.15101766614325e-08\\
1.71500000000001	-8.15101766614227e-08\\
1.71700000000001	-8.0141884940795e-08\\
1.71900000000001	-7.87967944304307e-08\\
1.71999999999999	-7.81327844818945e-08\\
1.72	-7.81327844818851e-08\\
1.724	-7.55321345757059e-08\\
1.72799999999999	-7.30170688653024e-08\\
1.72999999999999	-7.17903922718982e-08\\
1.73	-7.17903922718896e-08\\
1.731	-7.11847161837202e-08\\
1.73100000000001	-7.11847161837116e-08\\
1.73200000000001	-7.05842820355433e-08\\
1.73300000000001	-6.99890309508308e-08\\
1.735	-6.88138451108618e-08\\
1.73899999999999	-6.65231365337793e-08\\
1.73999999999998	-6.59625610588673e-08\\
1.74	-6.59625610588593e-08\\
1.74799999999998	-6.16437214631233e-08\\
1.74999999999998	-6.06081211027463e-08\\
1.75	-6.0608121102739e-08\\
1.75799999999998	-5.66386191163846e-08\\
1.75999999999998	-5.56880631940834e-08\\
1.76	-5.56880631940767e-08\\
1.76799999999998	-5.20419343353018e-08\\
1.76899999999998	-5.16029902846923e-08\\
1.769	-5.16029902846861e-08\\
1.76999999999998	-5.11676389615332e-08\\
1.77	-5.11676389615271e-08\\
1.77099999999998	-5.07359517695034e-08\\
1.77199999999997	-5.03080004319789e-08\\
1.77399999999994	-4.94631379077416e-08\\
1.77799999999989	-4.78164216208364e-08\\
1.77999999999998	-4.70139222118604e-08\\
1.78	-4.70139222118547e-08\\
1.78499999999998	-4.50659934491632e-08\\
1.785	-4.50659934491578e-08\\
1.78999999999998	-4.31975838182109e-08\\
1.79	-4.31975838182057e-08\\
1.79499999999998	-4.14065230918965e-08\\
1.798	-4.0368336550432e-08\\
1.79800000000001	-4.03683365504272e-08\\
1.79999999999999	-3.96908306507905e-08\\
1.8	-3.96908306507858e-08\\
1.80199999999998	-3.90247089767893e-08\\
1.80399999999995	-3.83697103689099e-08\\
1.8079999999999	-3.70920594396854e-08\\
1.80999999999999	-3.64689061756224e-08\\
1.81	-3.6468906175618e-08\\
1.8179999999999	-3.4080328940984e-08\\
1.818	-3.40803289409557e-08\\
1.81800000000001	-3.40803289409516e-08\\
1.81999999999998	-3.35083484174997e-08\\
1.82	-3.35083484174956e-08\\
1.82199999999997	-3.29459791288564e-08\\
1.82399999999994	-3.23930005618548e-08\\
1.82699999999998	-3.15806670584833e-08\\
1.827	-3.15806670584795e-08\\
1.82999999999998	-3.07882591299664e-08\\
1.83	-3.07882591299627e-08\\
1.83299999999999	-3.00156959651519e-08\\
1.83599999999997	-2.92629142577932e-08\\
1.83999999999998	-2.82888311334205e-08\\
1.84	-2.8288831133417e-08\\
1.84599999999997	-2.68881068500505e-08\\
1.84999999999998	-2.59924217546587e-08\\
1.85	-2.59924217546556e-08\\
1.85499999999998	-2.49146880676918e-08\\
1.855	-2.49146880676888e-08\\
1.85599999999998	-2.4704666343567e-08\\
1.856	-2.4704666343564e-08\\
1.85699999999998	-2.44964378451197e-08\\
1.85799999999997	-2.42899821619672e-08\\
1.85999999999994	-2.38823084740925e-08\\
1.85999999999999	-2.38823084740829e-08\\
1.86	-2.388230847408e-08\\
1.86399999999994	-2.30873559494791e-08\\
1.86799999999988	-2.23185619991498e-08\\
1.86999999999999	-2.19435961207827e-08\\
1.87	-2.19435961207801e-08\\
1.876	-2.08564283110494e-08\\
1.87600000000001	-2.08564283110469e-08\\
1.87999999999999	-2.0162165981429e-08\\
1.88	-2.01621659814266e-08\\
1.88399999999998	-1.94910407780969e-08\\
1.88499999999998	-1.9326750530886e-08\\
1.885	-1.93267505308836e-08\\
1.88899999999997	-1.86830706542753e-08\\
1.89	-1.85254421035873e-08\\
1.89000000000001	-1.85254421035851e-08\\
1.89399999999999	-1.79082910584822e-08\\
1.89799999999996	-1.73120642448229e-08\\
1.89999999999999	-1.70215039475907e-08\\
1.9	-1.70215039475887e-08\\
1.90799999999995	-1.59069809316604e-08\\
1.90999999999999	-1.56397335198152e-08\\
1.91	-1.56397335198133e-08\\
1.914	-1.51187170011752e-08\\
1.91400000000001	-1.51187170011734e-08\\
1.91800000000001	-1.46153654942793e-08\\
1.91999999999999	-1.43700663638507e-08\\
1.92	-1.4370066363849e-08\\
1.924	-1.38917404648783e-08\\
1.92499999999998	-1.37746472188813e-08\\
1.925	-1.37746472188796e-08\\
1.92899999999999	-1.33158828323277e-08\\
1.92999999999999	-1.32035376801399e-08\\
1.93	-1.32035376801383e-08\\
1.934	-1.27636814088443e-08\\
1.93400000000001	-1.27636814088427e-08\\
1.93800000000001	-1.23387384155624e-08\\
1.93999999999999	-1.21316502431209e-08\\
1.94	-1.21316502431195e-08\\
1.94299999999998	-1.18275182977718e-08\\
1.943	-1.18275182977704e-08\\
1.94599999999998	-1.15309487545113e-08\\
1.94899999999996	-1.12416799939151e-08\\
1.94999999999999	-1.11468351821317e-08\\
1.95	-1.11468351821304e-08\\
1.95599999999997	-1.05945842456311e-08\\
1.95999999999998	-1.02419186767822e-08\\
1.96	-1.0241918676781e-08\\
1.96599999999996	-9.73478995943231e-09\\
1.97	-9.41051062892252e-09\\
1.97000000000001	-9.41051062892139e-09\\
1.97199999999998	-9.25240449106222e-09\\
1.972	-9.25240449106111e-09\\
1.97399999999997	-9.0970162275374e-09\\
1.97599999999994	-8.94428491515557e-09\\
1.97999999999988	-8.64655464802799e-09\\
1.98	-8.6465546480193e-09\\
1.98000000000002	-8.64655464801826e-09\\
1.9879999999999	-8.08040981111903e-09\\
1.99	-7.94465597519255e-09\\
1.99000000000002	-7.9446559751916e-09\\
1.99199999999998	-7.81117813049015e-09\\
1.992	-7.81117813048921e-09\\
1.99399999999997	-7.67999477897195e-09\\
1.995	-7.61524742837693e-09\\
1.99500000000001	-7.61524742837601e-09\\
1.99699999999998	-7.48740964737727e-09\\
1.99899999999995	-7.36173941741294e-09\\
1.99999999999999	-7.29970170455632e-09\\
2	-7.29970170455544e-09\\
2.00099999999997	-7.23818746885773e-09\\
2.001	-7.23818746885599e-09\\
2.00199999999997	-7.17719068150393e-09\\
2.00299999999995	-7.11670536394806e-09\\
2.00499999999989	-6.99724547501246e-09\\
2.00899999999979	-6.76420578193909e-09\\
2.00999999999997	-6.70713749567627e-09\\
2.01	-6.70713749567465e-09\\
2.01799999999979	-6.26784243609973e-09\\
2.01999999999997	-6.16264698220194e-09\\
2.02	-6.16264698220046e-09\\
2.02799999999979	-5.75914076666231e-09\\
2.02999999999997	-5.66238536401211e-09\\
2.03	-5.66238536401075e-09\\
2.03799999999979	-5.2915181987276e-09\\
2.03999999999997	-5.20270875243848e-09\\
2.04	-5.20270875243723e-09\\
2.04799999999979	-4.86205548788048e-09\\
2.05	-4.78037135584427e-09\\
2.05000000000003	-4.78037135584312e-09\\
2.05799999999981	-4.46727289198506e-09\\
2.05899999999997	-4.42962561834725e-09\\
2.059	-4.42962561834619e-09\\
2.05999999999997	-4.39229697761058e-09\\
2.06	-4.39229697760952e-09\\
2.06099999999998	-4.35528331285286e-09\\
2.06199999999995	-4.31858099464291e-09\\
2.0639999999999	-4.24609603919023e-09\\
2.06499999999997	-4.2103062975319e-09\\
2.065	-4.21030629753089e-09\\
2.0689999999999	-4.07008406148995e-09\\
2.06999999999997	-4.03574550597972e-09\\
2.07	-4.03574550597875e-09\\
2.0739999999999	-3.90130229423547e-09\\
2.0779999999998	-3.77141717299101e-09\\
2.07899999999997	-3.73963406855528e-09\\
2.079	-3.73963406855438e-09\\
2.07999999999997	-3.70811996473959e-09\\
2.08	-3.7081199647387e-09\\
2.08099999999998	-3.67687177424599e-09\\
2.08199999999995	-3.64588643298565e-09\\
2.08399999999991	-3.58469217608363e-09\\
2.08799999999981	-3.46532591974908e-09\\
2.088	-3.46532591974355e-09\\
2.08800000000003	-3.46532591974272e-09\\
2.08999999999997	-3.40710711906654e-09\\
2.09	-3.40710711906572e-09\\
2.09199999999995	-3.34986436741638e-09\\
2.0939999999999	-3.29360560336496e-09\\
2.09799999999979	-3.18395218991917e-09\\
2.09999999999997	-3.13051454749265e-09\\
2.1	-3.1305145474919e-09\\
2.10799999999979	-2.92553963329846e-09\\
2.10999999999997	-2.87638936530826e-09\\
2.11	-2.87638936530756e-09\\
2.11699999999997	-2.71084122853655e-09\\
2.117	-2.7108412285359e-09\\
2.11999999999997	-2.64288066085341e-09\\
2.12	-2.64288066085277e-09\\
2.12299999999998	-2.57662592839169e-09\\
2.12599999999995	-2.51201858231086e-09\\
2.13	-2.42833975048969e-09\\
2.13000000000003	-2.42833975048911e-09\\
2.13499999999997	-2.32765326384858e-09\\
2.135	-2.32765326384802e-09\\
2.137	-2.28857856085498e-09\\
2.13700000000002	-2.28857856085443e-09\\
2.13900000000002	-2.25016638196336e-09\\
2.14	-2.23120402283894e-09\\
2.14000000000003	-2.2312040228384e-09\\
2.14200000000003	-2.19375747251854e-09\\
2.14400000000002	-2.15693627124857e-09\\
2.14599999999997	-2.12072598313648e-09\\
2.146	-2.12072598313597e-09\\
2.14999999999999	-2.05008159566304e-09\\
2.15000000000003	-2.0500815956624e-09\\
2.15400000000002	-1.98178664874272e-09\\
2.15800000000002	-1.91580716588603e-09\\
2.15999999999997	-1.88365324679536e-09\\
2.16	-1.88365324679491e-09\\
2.16799999999999	-1.76031802689795e-09\\
2.16999999999997	-1.73074389239667e-09\\
2.17	-1.73074389239625e-09\\
2.17499999999997	-1.65898183031058e-09\\
2.175	-1.65898183031018e-09\\
2.17999999999997	-1.59023979053652e-09\\
2.18	-1.59023979053606e-09\\
2.18499999999997	-1.52434930211317e-09\\
2.18999999999994	-1.46114888456256e-09\\
2.18999999999997	-1.46114888456219e-09\\
2.19	-1.46114888456182e-09\\
2.19499999999997	-1.4005650889729e-09\\
2.195	-1.40056508897256e-09\\
2.19999999999996	-1.34253089954285e-09\\
2.2	-1.34253089954247e-09\\
2.20399999999997	-1.29784348199091e-09\\
2.204	-1.2978434819906e-09\\
2.20499999999997	-1.28690408242578e-09\\
2.205	-1.28690408242547e-09\\
2.20599999999997	-1.27605550691793e-09\\
2.20699999999995	-1.26529669215351e-09\\
2.20899999999989	-1.2440441356452e-09\\
2.20999999999997	-1.23354831088973e-09\\
2.21	-1.23354831088943e-09\\
2.21399999999989	-1.19245479764171e-09\\
2.21799999999979	-1.15275451918504e-09\\
2.21999999999997	-1.13340730156993e-09\\
2.22	-1.13340730156965e-09\\
2.22799999999979	-1.05919572869513e-09\\
2.22999999999997	-1.04140079054987e-09\\
2.23	-1.04140079054962e-09\\
2.23299999999997	-1.01526910386878e-09\\
2.233	-1.01526910386853e-09\\
2.23599999999997	-9.89806551713638e-10\\
2.23899999999994	-9.64990670704878e-10\\
2.24	-9.56858615180182e-10\\
2.24000000000003	-9.56858615179952e-10\\
2.24599999999997	-9.09479840111344e-10\\
2.25	-8.79183832895614e-10\\
2.25000000000003	-8.79183832895403e-10\\
2.25299999999997	-8.57122642802336e-10\\
2.253	-8.5712264280213e-10\\
2.25599999999994	-8.35626355366197e-10\\
2.25899999999988	-8.14676007298208e-10\\
2.26	-8.07810669816583e-10\\
2.26000000000003	-8.07810669816389e-10\\
2.26199999999997	-7.94253116853531e-10\\
2.262	-7.9425311685334e-10\\
2.26399999999994	-7.80921973532945e-10\\
2.26599999999988	-7.67812013104202e-10\\
2.26999999999977	-7.42235166518943e-10\\
2.27	-7.42235166517494e-10\\
2.27000000000003	-7.42235166517315e-10\\
2.27499999999997	-7.11459853571599e-10\\
2.275	-7.11459853571428e-10\\
2.27999999999994	-6.81979680510342e-10\\
2.28	-6.81979680510013e-10\\
2.28499999999995	-6.53722398195919e-10\\
2.28999999999989	-6.26618755366946e-10\\
2.29	-6.26618755366368e-10\\
2.29000000000003	-6.26618755366217e-10\\
2.29099999999997	-6.21331966632248e-10\\
2.291	-6.21331966632098e-10\\
2.29199999999997	-6.1609093398736e-10\\
2.29299999999994	-6.10895143241247e-10\\
2.29499999999989	-6.00637254863038e-10\\
2.29899999999979	-5.80642265249891e-10\\
2.29999999999997	-5.7574915088708e-10\\
2.3	-5.75749150886941e-10\\
2.30799999999979	-5.38051158993202e-10\\
2.31	-5.29011684608115e-10\\
2.31000000000003	-5.29011684607988e-10\\
2.31099999999999	-5.24548408170377e-10\\
2.31100000000002	-5.24548408170251e-10\\
2.31199999999999	-5.2012376022836e-10\\
2.31299999999996	-5.15737306916155e-10\\
2.31499999999989	-5.07077268209963e-10\\
2.31899999999976	-4.90196855620298e-10\\
2.31999999999997	-4.86065931192779e-10\\
2.32	-4.86065931192662e-10\\
2.32799999999974	-4.54240073416098e-10\\
2.32999999999997	-4.46608657907617e-10\\
2.33	-4.46608657907509e-10\\
2.33799999999974	-4.17357156437085e-10\\
2.33999999999997	-4.10352468159431e-10\\
2.34	-4.10352468159332e-10\\
2.345	-3.93349840247225e-10\\
2.34500000000003	-3.93349840247131e-10\\
2.34899999999997	-3.80249469645932e-10\\
2.349	-3.8024946964584e-10\\
2.34999999999997	-3.7704136446504e-10\\
2.35	-3.77041364464949e-10\\
2.35099999999997	-3.73860262028317e-10\\
2.35199999999994	-3.70706691050423e-10\\
2.35399999999988	-3.64480909807908e-10\\
2.35799999999976	-3.5234630227334e-10\\
2.35999999999997	-3.464327182308e-10\\
2.36	-3.46432718230716e-10\\
2.36799999999976	-3.23749533652638e-10\\
2.36999999999997	-3.18310408917856e-10\\
2.37	-3.1831040891778e-10\\
2.37799999999976	-2.97462040261316e-10\\
2.378	-2.97462040260732e-10\\
2.37800000000002	-2.97462040260661e-10\\
2.37999999999997	-2.92469601797779e-10\\
2.38	-2.92469601797709e-10\\
2.38199999999995	-2.87561059709314e-10\\
2.38399999999989	-2.82734489284681e-10\\
2.38799999999979	-2.7331972577108e-10\\
2.38999999999997	-2.68727841331465e-10\\
2.39	-2.687278413314e-10\\
2.39799999999979	-2.51126976812513e-10\\
2.398	-2.51126976812071e-10\\
2.39800000000002	-2.5112697681201e-10\\
2.39999999999997	-2.46912199514164e-10\\
2.4	-2.46912199514105e-10\\
2.40199999999994	-2.42768250278816e-10\\
2.40399999999988	-2.38693504203335e-10\\
2.40699999999997	-2.32707653723075e-10\\
2.407	-2.32707653723019e-10\\
2.40999999999997	-2.26868641743776e-10\\
2.41	-2.26868641743721e-10\\
2.41299999999997	-2.2117586934921e-10\\
2.41499999999997	-2.17461975551994e-10\\
2.415	-2.17461975551942e-10\\
2.41799999999997	-2.12009426343423e-10\\
2.41999999999997	-2.08451176224186e-10\\
2.42	-2.08451176224136e-10\\
2.42299999999997	-2.03225487419266e-10\\
2.42599999999994	-1.98129733328019e-10\\
2.42999999999997	-1.91529761647093e-10\\
2.42999999999999	-1.91529761647047e-10\\
2.43599999999994	-1.82040783276192e-10\\
2.43599999999997	-1.82040783276144e-10\\
2.436	-1.82040783276098e-10\\
2.43999999999997	-1.75981147096403e-10\\
2.44	-1.75981147096361e-10\\
2.44399999999998	-1.70123454746378e-10\\
2.44799999999996	-1.64458519221283e-10\\
2.44999999999997	-1.61695547221186e-10\\
2.45	-1.61695547221147e-10\\
2.45599999999999	-1.53684648158516e-10\\
2.45600000000002	-1.53684648158479e-10\\
2.45999999999997	-1.48568909606648e-10\\
2.46	-1.48568909606612e-10\\
2.46399999999995	-1.43623658550312e-10\\
2.46499999999997	-1.42413069849782e-10\\
2.465	-1.42413069849748e-10\\
2.46899999999995	-1.37670051515918e-10\\
2.46999999999997	-1.36508549964847e-10\\
2.47	-1.36508549964814e-10\\
2.47399999999995	-1.31961009909585e-10\\
2.4779999999999	-1.27567649097385e-10\\
2.47999999999997	-1.25426623389579e-10\\
2.48	-1.25426623389549e-10\\
2.48499999999997	-1.20229666858741e-10\\
2.485	-1.20229666858712e-10\\
2.48999999999997	-1.15244882777139e-10\\
2.49	-1.15244882777109e-10\\
2.49399999999997	-1.11405704374608e-10\\
2.494	-1.11405704374581e-10\\
2.49799999999996	-1.07696689012177e-10\\
2.49999999999997	-1.05889166908551e-10\\
2.5	-1.05889166908526e-10\\
2.50399999999997	-1.02364550392215e-10\\
2.50799999999994	-9.89559171402942e-11\\
2.50999999999997	-9.72934166501973e-11\\
2.51	-9.72934166501739e-11\\
2.51399999999997	-9.40522596062534e-11\\
2.514	-9.40522596062307e-11\\
2.51799999999996	-9.09209904855813e-11\\
2.51999999999997	-8.93950226438499e-11\\
2.52	-8.93950226438284e-11\\
2.523	-8.71539680151707e-11\\
2.52300000000002	-8.71539680151497e-11\\
2.52600000000002	-8.496863639077e-11\\
2.52900000000001	-8.28370999759311e-11\\
2.52999999999997	-8.21382159854936e-11\\
2.53	-8.21382159854738e-11\\
2.53599999999999	-7.80688352701193e-11\\
2.53999999999997	-7.54701396619871e-11\\
2.54	-7.54701396619689e-11\\
2.54599999999999	-7.17332429944116e-11\\
2.54999999999997	-6.93437116456493e-11\\
2.55	-6.93437116456326e-11\\
2.55199999999997	-6.81786671831394e-11\\
2.552	-6.8178667183123e-11\\
2.55399999999996	-6.70336497334305e-11\\
2.55499999999997	-6.64685105450542e-11\\
2.555	-6.64685105450382e-11\\
2.55699999999997	-6.5352694288635e-11\\
2.55899999999994	-6.42557970965652e-11\\
2.55999999999997	-6.37143084750512e-11\\
2.56	-6.37143084750358e-11\\
2.56399999999994	-6.15935206818848e-11\\
2.56799999999987	-5.95425208312899e-11\\
2.56999999999997	-5.85421819671317e-11\\
2.57	-5.85421819671176e-11\\
2.57199999999997	-5.75586141181745e-11\\
2.572	-5.75586141181607e-11\\
2.57399999999996	-5.65919537533894e-11\\
2.57599999999993	-5.56418218310819e-11\\
2.57999999999986	-5.3789659646388e-11\\
2.57999999999997	-5.37896596463375e-11\\
2.58	-5.37896596463245e-11\\
2.58099999999997	-5.33363749589487e-11\\
2.581	-5.33363749589359e-11\\
2.58199999999996	-5.28869032210876e-11\\
2.58299999999993	-5.24412003591238e-11\\
2.58499999999987	-5.15609268938935e-11\\
2.58899999999974	-4.98437086248207e-11\\
2.58999999999997	-4.94231848872425e-11\\
2.59	-4.94231848872306e-11\\
2.59799999999975	-4.61861155931048e-11\\
2.59999999999997	-4.54109533416286e-11\\
2.6	-4.54109533416177e-11\\
2.60799999999975	-4.24376046971412e-11\\
2.60999999999997	-4.17246354590366e-11\\
2.61	-4.17246354590266e-11\\
2.61799999999974	-3.89917978391211e-11\\
2.61999999999997	-3.83373810272982e-11\\
2.62	-3.8337381027289e-11\\
2.62499999999997	-3.67489013493981e-11\\
2.625	-3.67489013493893e-11\\
2.62999999999998	-3.52252733311733e-11\\
2.63000000000001	-3.52252733311648e-11\\
2.63499999999998	-3.37647263982788e-11\\
2.63899999999997	-3.26407104420996e-11\\
2.639	-3.26407104420918e-11\\
2.63999999999997	-3.23656446136411e-11\\
2.64	-3.23656446136333e-11\\
2.64099999999998	-3.20928997872129e-11\\
2.64199999999996	-3.18224492286012e-11\\
2.64399999999991	-3.12883251091983e-11\\
2.64799999999982	-3.02464565255819e-11\\
2.64999999999997	-2.97383035644194e-11\\
2.65	-2.97383035644122e-11\\
2.65799999999982	-2.77905343580327e-11\\
2.65899999999997	-2.75563327403653e-11\\
2.659	-2.75563327403587e-11\\
2.66	-2.73241133451885e-11\\
2.66000000000003	-2.73241133451819e-11\\
2.66100000000003	-2.70938534230022e-11\\
2.66200000000004	-2.68655303953196e-11\\
2.66400000000005	-2.64146056989637e-11\\
2.66799999999997	-2.55350268576424e-11\\
2.668	-2.55350268576363e-11\\
2.67	-2.51060278715391e-11\\
2.67000000000003	-2.5106027871533e-11\\
2.67200000000003	-2.46842211959289e-11\\
2.67400000000004	-2.42696653307534e-11\\
2.67800000000005	-2.34616587021671e-11\\
2.67999999999997	-2.30678911346804e-11\\
2.68	-2.30678911346749e-11\\
2.68800000000002	-2.15574870140616e-11\\
2.68999999999997	-2.11953123082948e-11\\
2.69	-2.11953123082896e-11\\
2.69499999999997	-2.03164901140331e-11\\
2.695	-2.03164901140283e-11\\
2.69699999999997	-1.99754342890612e-11\\
2.697	-1.99754342890564e-11\\
2.69899999999996	-1.96401611896043e-11\\
2.69999999999997	-1.94746519973704e-11\\
2.7	-1.94746519973657e-11\\
2.70199999999997	-1.91478072183097e-11\\
2.70399999999993	-1.88264206896965e-11\\
2.70799999999987	-1.81995204744218e-11\\
2.71	-1.78937609920568e-11\\
2.71000000000003	-1.78937609920525e-11\\
2.71699999999999	-1.68639009344399e-11\\
2.71700000000002	-1.68639009344359e-11\\
2.72	-1.64411244794381e-11\\
2.72000000000003	-1.64411244794341e-11\\
2.72300000000001	-1.60289599435184e-11\\
2.72599999999999	-1.56270437218736e-11\\
2.72600000000002	-1.56270437218699e-11\\
2.73	-1.5106485703964e-11\\
2.73000000000003	-1.51064857039604e-11\\
2.73400000000002	-1.46032400575701e-11\\
2.738	-1.41170563805699e-11\\
2.74	-1.38801234693045e-11\\
2.74000000000003	-1.38801234693011e-11\\
2.748	-1.29713019420419e-11\\
2.75	-1.27533786979597e-11\\
2.75000000000003	-1.27533786979567e-11\\
2.75499999999997	-1.22245847774068e-11\\
2.755	-1.22245847774039e-11\\
2.75999999999993	-1.17180444590995e-11\\
2.75999999999997	-1.17180444590961e-11\\
2.76	-1.17180444590926e-11\\
2.76499999999994	-1.1232516328979e-11\\
2.76499999999997	-1.1232516328976e-11\\
2.765	-1.12325163289731e-11\\
2.76999999999994	-1.07668104836721e-11\\
2.76999999999997	-1.07668104836692e-11\\
2.77	-1.07668104836664e-11\\
2.77499999999994	-1.03203857493651e-11\\
2.775	-1.03203857493602e-11\\
2.77999999999993	-9.8927482052384e-12\\
2.77999999999997	-9.89274820523547e-12\\
2.78	-9.89274820523257e-12\\
2.78399999999997	-9.5634591903642e-12\\
2.784	-9.5634591903619e-12\\
2.78799999999996	-9.2450059616068e-12\\
2.78999999999997	-9.08968604644116e-12\\
2.79	-9.08968604643898e-12\\
2.79399999999997	-8.78687936778501e-12\\
2.79799999999993	-8.4943390037188e-12\\
2.79999999999997	-8.3517746942569e-12\\
2.8	-8.35177469425489e-12\\
2.80799999999993	-7.80492996523655e-12\\
2.80999999999997	-7.67380391464512e-12\\
2.81	-7.67380391464327e-12\\
2.81299999999997	-7.481246617235e-12\\
2.813	-7.48124661723319e-12\\
2.81599999999996	-7.29361997272925e-12\\
2.81899999999993	-7.1107584545731e-12\\
2.81999999999997	-7.05083553181084e-12\\
2.82	-7.05083553180914e-12\\
2.82599999999993	-6.7017140793621e-12\\
2.83	-6.47847090626359e-12\\
2.83000000000003	-6.47847090626203e-12\\
2.83299999999999	-6.31590787980596e-12\\
2.83300000000002	-6.31590787980444e-12\\
2.835	-6.20985378851212e-12\\
2.83500000000003	-6.20985378851063e-12\\
2.83700000000001	-6.10560808987649e-12\\
2.83899999999999	-6.0031299142037e-12\\
2.84	-5.95254106543608e-12\\
2.84000000000003	-5.95254106543464e-12\\
2.84199999999997	-5.85263905487972e-12\\
2.842	-5.85263905487832e-12\\
2.84399999999993	-5.75440539073998e-12\\
2.84599999999987	-5.65780156012048e-12\\
2.84999999999974	-5.46933252897814e-12\\
2.85	-5.46933252896612e-12\\
2.85000000000003	-5.4693325289648e-12\\
2.85799999999978	-5.11110773673359e-12\\
2.85999999999997	-5.0253257194054e-12\\
2.86	-5.02532571940419e-12\\
2.86799999999975	-4.69628513175226e-12\\
2.86999999999997	-4.61738560030165e-12\\
2.87	-4.61738560030053e-12\\
2.87099999999997	-4.57842864316575e-12\\
2.871	-4.57842864316465e-12\\
2.87199999999996	-4.53980884876566e-12\\
2.87299999999993	-4.50152243017593e-12\\
2.87499999999986	-4.42593474286191e-12\\
2.87899999999973	-4.2785969135963e-12\\
2.87999999999997	-4.24254083875994e-12\\
2.88	-4.24254083875892e-12\\
2.88799999999974	-3.96475424075987e-12\\
2.88999999999997	-3.89814473036139e-12\\
2.89	-3.89814473036046e-12\\
2.89099999999997	-3.86525601994511e-12\\
2.89099999999999	-3.86525601994418e-12\\
2.89199999999996	-3.83265195310576e-12\\
2.89299999999993	-3.80032933279964e-12\\
2.89499999999986	-3.73651578724671e-12\\
2.89899999999973	-3.61212847480816e-12\\
2.89999999999997	-3.58168878211108e-12\\
2.9	-3.58168878211022e-12\\
2.90499999999997	-3.43328427141137e-12\\
2.905	-3.43328427141054e-12\\
2.90999999999998	-3.29093856112438e-12\\
2.91000000000001	-3.29093856112359e-12\\
2.91499999999999	-3.15448623421408e-12\\
2.91999999999997	-3.02377632106414e-12\\
2.92	-3.02377632106326e-12\\
2.92899999999997	-2.80195534926816e-12\\
2.92899999999999	-2.80195534926749e-12\\
2.92999999999997	-2.778315677716e-12\\
2.93	-2.77831567771533e-12\\
2.93099999999998	-2.75487498408676e-12\\
2.93199999999996	-2.73163716274709e-12\\
2.93399999999992	-2.68576104640216e-12\\
2.93799999999983	-2.59634437396695e-12\\
2.93999999999997	-2.55276875928941e-12\\
2.94	-2.5527687592888e-12\\
2.94799999999983	-2.38562247029478e-12\\
2.94999999999997	-2.34554302606487e-12\\
2.95	-2.34554302606431e-12\\
2.95799999999983	-2.19191702647578e-12\\
2.95799999999997	-2.19191702647326e-12\\
2.95799999999999	-2.19191702647273e-12\\
2.95999999999997	-2.15512909768254e-12\\
2.96	-2.15512909768203e-12\\
2.96199999999998	-2.11895937954043e-12\\
2.96399999999996	-2.08339368937501e-12\\
2.96799999999992	-2.01401885003045e-12\\
2.96999999999997	-1.98018250029204e-12\\
2.97	-1.98018250029157e-12\\
2.97499999999997	-1.89807810543891e-12\\
2.975	-1.89807810543846e-12\\
2.97799999999997	-1.85048651150019e-12\\
2.97799999999999	-1.85048651149974e-12\\
2.97999999999997	-1.8194289664213e-12\\
2.98	-1.81942896642086e-12\\
2.98199999999998	-1.78889333264634e-12\\
2.98399999999996	-1.75886763863053e-12\\
2.98699999999999	-1.71475953775343e-12\\
2.98700000000002	-1.71475953775302e-12\\
2.99	-1.67173345097212e-12\\
2.99000000000003	-1.67173345097172e-12\\
2.99300000000001	-1.62978496457475e-12\\
2.99599999999999	-1.5889106153921e-12\\
2.99999999999997	-1.53602017403538e-12\\
3	-1.53602017403501e-12\\
3.00599999999996	-1.45996428026922e-12\\
3.00999999999997	-1.41133089229158e-12\\
3.01	-1.41133089229124e-12\\
3.01599999999996	-1.34140920359236e-12\\
3.01599999999999	-1.34140920359196e-12\\
3.01600000000002	-1.34140920359164e-12\\
3.01999999999997	-1.29675740008657e-12\\
3.02	-1.29675740008626e-12\\
3.02399999999995	-1.25359366521829e-12\\
3.0279999999999	-1.21185030245838e-12\\
3.03	-1.19149071626197e-12\\
3.03000000000003	-1.19149071626168e-12\\
3.03600000000002	-1.13246058773013e-12\\
3.03600000000005	-1.13246058772986e-12\\
3.03999999999997	-1.09476410590214e-12\\
3.04	-1.09476410590188e-12\\
3.04399999999993	-1.0583238992227e-12\\
3.04499999999997	-1.04940340105033e-12\\
3.04499999999999	-1.04940340105008e-12\\
3.04899999999992	-1.0144533912607e-12\\
3.04999999999997	-1.00589460268304e-12\\
3.05	-1.0058946026828e-12\\
3.05399999999993	-9.72385018150591e-13\\
3.05799999999985	-9.40011537745488e-13\\
3.05999999999997	-9.24234901584549e-13\\
3.06	-9.24234901584326e-13\\
3.06799999999985	-8.63719262134748e-13\\
3.06999999999997	-8.49208421299411e-13\\
3.07	-8.49208421299207e-13\\
3.07399999999997	-8.20918556112193e-13\\
3.07399999999999	-8.20918556111995e-13\\
3.07799999999996	-7.93587827700889e-13\\
3.08	-7.80268687235237e-13\\
3.08000000000003	-7.8026868723505e-13\\
3.08399999999999	-7.54296742957097e-13\\
3.08799999999996	-7.29179447261439e-13\\
3.09	-7.16928930938055e-13\\
3.09000000000003	-7.16928930937882e-13\\
3.09399999999997	-6.93045721199628e-13\\
3.09399999999999	-6.93045721199461e-13\\
3.09799999999993	-6.69972245445211e-13\\
3.09999999999997	-6.58727800737794e-13\\
3.1	-6.58727800737636e-13\\
3.10299999999997	-6.4221407451344e-13\\
3.10299999999999	-6.42214074513286e-13\\
3.10599999999996	-6.26110955809025e-13\\
3.10899999999992	-6.10404239225784e-13\\
3.10999999999997	-6.05254351656362e-13\\
3.11	-6.05254351656217e-13\\
3.11499999999997	-5.80158664331374e-13\\
3.115	-5.80158664331235e-13\\
3.11999999999998	-5.56119095550072e-13\\
3.12	-5.56119095549938e-13\\
3.12499999999998	-5.33076730527318e-13\\
3.12999999999995	-5.10975098409589e-13\\
3.13	-5.10975098409365e-13\\
3.13000000000003	-5.10975098409241e-13\\
3.13199999999997	-5.02390198468968e-13\\
3.13199999999999	-5.02390198468847e-13\\
3.13399999999993	-4.93952873261101e-13\\
3.13599999999986	-4.85659813672119e-13\\
3.13999999999973	-4.69493541419547e-13\\
3.13999999999997	-4.69493541418566e-13\\
3.14	-4.69493541418453e-13\\
3.14799999999974	-4.38752761490381e-13\\
3.14999999999997	-4.313815336812e-13\\
3.15	-4.31381533681096e-13\\
3.15199999999997	-4.24133886964634e-13\\
3.15199999999999	-4.24133886964531e-13\\
3.15399999999996	-4.17010826957679e-13\\
3.15599999999992	-4.10009560604129e-13\\
3.15999999999985	-3.96361476110761e-13\\
3.15999999999997	-3.96361476110335e-13\\
3.16	-3.96361476110239e-13\\
3.16099999999997	-3.93021343290523e-13\\
3.16099999999999	-3.93021343290429e-13\\
3.16199999999996	-3.89709307071888e-13\\
3.16299999999993	-3.86425042687728e-13\\
3.16499999999986	-3.7993854454466e-13\\
3.16899999999973	-3.67284825531157e-13\\
3.16999999999997	-3.64186099354391e-13\\
3.17	-3.64186099354304e-13\\
3.17799999999974	-3.40333007624824e-13\\
3.17999999999997	-3.34621045832994e-13\\
3.18	-3.34621045832914e-13\\
3.18499999999997	-3.20756281072412e-13\\
3.185	-3.20756281072335e-13\\
3.18999999999998	-3.07457562758315e-13\\
3.19	-3.07457562758241e-13\\
3.19499999999998	-2.94709436714089e-13\\
3.19999999999995	-2.82497798520488e-13\\
3.2	-2.82497798520359e-13\\
3.20999999999995	-2.59565516971698e-13\\
3.21	-2.59565516971576e-13\\
3.21899999999997	-2.40520570733581e-13\\
3.21899999999999	-2.40520570733523e-13\\
3.21999999999997	-2.38493684801275e-13\\
3.22	-2.38493684801218e-13\\
3.22099999999998	-2.36483901831054e-13\\
3.22199999999996	-2.34491024704972e-13\\
3.22399999999992	-2.30555208332117e-13\\
3.22799999999984	-2.22877960191707e-13\\
3.22999999999997	-2.19133518320529e-13\\
3.23	-2.19133518320476e-13\\
3.23799999999984	-2.04780933421672e-13\\
3.23899999999997	-2.03055165004676e-13\\
3.23899999999999	-2.03055165004627e-13\\
3.23999999999997	-2.01344003051921e-13\\
3.24	-2.01344003051872e-13\\
3.24099999999998	-1.99647279928481e-13\\
3.24199999999996	-1.97964829259593e-13\\
3.24399999999992	-1.94642087077357e-13\\
3.24799999999984	-1.88160708416176e-13\\
3.24799999999997	-1.88160708415979e-13\\
3.24799999999999	-1.88160708415933e-13\\
3.24999999999997	-1.84999530711385e-13\\
3.25	-1.84999530711341e-13\\
3.25199999999998	-1.81891351393273e-13\\
3.25399999999996	-1.78836601340866e-13\\
3.25499999999998	-1.77328887222537e-13\\
3.255	-1.77328887222494e-13\\
3.25899999999996	-1.71425670150773e-13\\
3.25999999999997	-1.69981052508343e-13\\
3.26	-1.69981052508302e-13\\
3.26399999999996	-1.64323080482902e-13\\
3.26799999999992	-1.58851293119495e-13\\
3.26999999999997	-1.56182525698635e-13\\
3.27	-1.56182525698597e-13\\
3.27699999999999	-1.47193574801774e-13\\
3.27700000000002	-1.47193574801739e-13\\
3.28	-1.43503445473457e-13\\
3.28000000000003	-1.43503445473422e-13\\
3.28300000000001	-1.39905940394269e-13\\
3.28599999999999	-1.36397885904114e-13\\
3.28999999999998	-1.31854287519466e-13\\
3.29	-1.31854287519435e-13\\
3.29599999999996	-1.25321819078155e-13\\
3.29699999999999	-1.24265527414602e-13\\
3.29700000000002	-1.24265527414572e-13\\
3.29999999999997	-1.21150202160501e-13\\
3.3	-1.21150202160472e-13\\
3.30299999999995	-1.18113073209958e-13\\
3.30599999999991	-1.15151461311862e-13\\
3.30599999999995	-1.15151461311819e-13\\
3.30599999999999	-1.15151461311776e-13\\
3.31	-1.11315610111475e-13\\
3.31000000000003	-1.11315610111449e-13\\
3.31400000000004	-1.0760732908583e-13\\
3.31800000000005	-1.04024773087118e-13\\
3.31999999999997	-1.02278878552791e-13\\
3.32	-1.02278878552767e-13\\
3.32499999999998	-9.80410320991686e-14\\
3.325	-9.8041032099145e-14\\
3.32999999999998	-9.39762010521324e-14\\
3.33000000000001	-9.39762010521097e-14\\
3.33499999999998	-9.00796617532878e-14\\
3.33500000000001	-9.00796617532661e-14\\
3.33999999999998	-8.63471031054533e-14\\
3.34000000000001	-8.63471031054325e-14\\
3.34499999999998	-8.27693775765538e-14\\
3.34999999999996	-7.9337717090305e-14\\
3.35000000000001	-7.93377170902691e-14\\
3.35499999999998	-7.60481335211251e-14\\
3.35500000000001	-7.60481335211069e-14\\
3.35999999999998	-7.28969880474132e-14\\
3.36000000000001	-7.28969880473957e-14\\
3.36399999999999	-7.0470546293806e-14\\
3.36400000000002	-7.0470546293789e-14\\
3.36800000000001	-6.81239505114443e-14\\
3.36999999999998	-6.69794400279155e-14\\
3.37000000000001	-6.69794400278994e-14\\
3.37399999999999	-6.47481394369558e-14\\
3.37799999999998	-6.25924885446115e-14\\
3.37999999999997	-6.15419706722581e-14\\
3.38	-6.15419706722433e-14\\
3.38799999999998	-5.7512419635159e-14\\
3.38999999999997	-5.65461872022817e-14\\
3.39	-5.65461872022681e-14\\
3.39299999999997	-5.51272845125411e-14\\
3.39299999999999	-5.51272845125279e-14\\
3.395	-5.42016101341338e-14\\
3.39500000000003	-5.42016101341208e-14\\
3.39700000000004	-5.32917199916527e-14\\
3.39900000000005	-5.23972573611904e-14\\
3.39999999999997	-5.19557015428834e-14\\
3.4	-5.19557015428709e-14\\
3.40400000000002	-5.02263093558626e-14\\
3.40800000000004	-4.85538256023109e-14\\
3.40999999999997	-4.77381012608193e-14\\
3.41	-4.77381012608078e-14\\
3.41299999999996	-4.65402182015537e-14\\
3.41299999999999	-4.65402182015425e-14\\
3.41599999999996	-4.53730083275055e-14\\
3.41899999999992	-4.42354419116353e-14\\
3.41999999999997	-4.38626663536662e-14\\
3.42	-4.38626663536556e-14\\
3.42199999999997	-4.31265154282892e-14\\
3.42199999999999	-4.31265154282788e-14\\
3.42399999999996	-4.24026581090596e-14\\
3.42599999999992	-4.16908105927357e-14\\
3.42999999999985	-4.03020332451345e-14\\
3.42999999999997	-4.03020332450905e-14\\
3.43	-4.03020332450808e-14\\
3.43799999999985	-3.76623715822922e-14\\
3.43999999999997	-3.70302670850876e-14\\
3.44	-3.70302670850787e-14\\
3.44799999999985	-3.46056559439588e-14\\
3.44999999999997	-3.40242666265185e-14\\
3.45	-3.40242666265104e-14\\
3.45099999999999	-3.37372033411582e-14\\
3.45100000000002	-3.37372033411501e-14\\
3.45200000000001	-3.34526245168832e-14\\
3.453	-3.31705022488628e-14\\
3.45499999999999	-3.2613517015954e-14\\
3.45899999999995	-3.1527824384867e-14\\
3.46	-3.12621369173011e-14\\
3.46000000000003	-3.12621369172935e-14\\
3.46499999999998	-2.99668144247102e-14\\
3.465	-2.9966814424703e-14\\
3.46999999999995	-2.8724375099316e-14\\
3.47	-2.87243750993032e-14\\
3.47000000000003	-2.87243750992963e-14\\
3.47099999999999	-2.84820270931444e-14\\
3.47100000000002	-2.84820270931376e-14\\
3.47199999999999	-2.82417765395321e-14\\
3.47299999999996	-2.80035998893117e-14\\
3.47499999999989	-2.75333751248471e-14\\
3.47899999999976	-2.66167986487759e-14\\
3.47999999999996	-2.63924967829074e-14\\
3.47999999999999	-2.63924967829011e-14\\
3.48799999999973	-2.46644092818712e-14\\
3.48999999999997	-2.42500370147055e-14\\
3.48999999999999	-2.42500370146997e-14\\
3.49799999999973	-2.26617327084925e-14\\
3.49999999999999	-2.22813906700042e-14\\
3.50000000000002	-2.22813906699988e-14\\
3.50799999999976	-2.08224838741551e-14\\
3.50899999999999	-2.06468518904536e-14\\
3.50900000000002	-2.06468518904486e-14\\
3.50999999999999	-2.04726575448324e-14\\
3.51000000000002	-2.04726575448274e-14\\
3.51099999999999	-2.02999294097766e-14\\
3.51199999999996	-2.01286961857422e-14\\
3.51399999999991	-1.97906474850479e-14\\
3.51799999999979	-1.91317601953462e-14\\
3.51999999999997	-1.88106632690775e-14\\
3.52	-1.8810663269073e-14\\
3.52799999999977	-1.75790074425977e-14\\
3.52999999999997	-1.72836728504596e-14\\
3.53	-1.72836728504554e-14\\
3.53499999999998	-1.65670391654776e-14\\
3.535	-1.65670391654736e-14\\
3.53799999999997	-1.61516443567838e-14\\
3.53799999999999	-1.61516443567799e-14\\
3.53999999999997	-1.58805640683865e-14\\
3.54	-1.58805640683826e-14\\
3.54199999999998	-1.56140391902497e-14\\
3.54399999999996	-1.53519652308624e-14\\
3.54799999999993	-1.48407607873938e-14\\
3.54999999999997	-1.4591429869743e-14\\
3.55	-1.45914298697394e-14\\
3.55799999999993	-1.36357351933835e-14\\
3.55799999999996	-1.36357351933796e-14\\
3.55799999999999	-1.36357351933757e-14\\
3.56	-1.34068805215365e-14\\
3.56000000000003	-1.34068805215333e-14\\
3.56200000000004	-1.31818716871058e-14\\
3.56400000000005	-1.29606204608543e-14\\
3.56699999999996	-1.26355997833042e-14\\
3.56699999999999	-1.26355997833011e-14\\
3.56999999999998	-1.23185521749813e-14\\
3.57	-1.23185521749783e-14\\
3.57299999999999	-1.20094451124958e-14\\
3.57599999999997	-1.17082530767362e-14\\
3.57999999999997	-1.13185177039216e-14\\
3.58	-1.13185177039189e-14\\
3.58599999999997	-1.07580823702821e-14\\
3.58999999999997	-1.0399716072761e-14\\
3.59	-1.03997160727585e-14\\
3.59599999999997	-9.88448203782458e-15\\
3.596	-9.88448203782221e-15\\
3.59999999999997	-9.5554549600671e-15\\
3.6	-9.5554549600648e-15\\
3.60399999999998	-9.2373930579034e-15\\
3.60499999999998	-9.1595320665675e-15\\
3.605	-9.1595320665653e-15\\
3.60899999999998	-8.85447708457431e-15\\
3.60999999999998	-8.77977321150763e-15\\
3.61	-8.77977321150552e-15\\
3.61399999999998	-8.48729072669616e-15\\
3.61599999999997	-8.34479614443246e-15\\
3.616	-8.34479614443045e-15\\
3.61999999999997	-8.0670209539666e-15\\
3.62	-8.06702095396466e-15\\
3.62399999999998	-7.79850291370831e-15\\
3.62499999999996	-7.73277017216211e-15\\
3.62499999999999	-7.73277017216025e-15\\
3.62899999999997	-7.47523299019562e-15\\
3.62999999999997	-7.41216558893283e-15\\
3.63	-7.41216558893104e-15\\
3.63399999999998	-7.16524251149534e-15\\
3.63799999999996	-6.92669107835007e-15\\
3.63999999999997	-6.81043730763848e-15\\
3.64	-6.81043730763684e-15\\
3.64799999999996	-6.36451390492855e-15\\
3.65	-6.25758743787592e-15\\
3.65000000000003	-6.25758743787441e-15\\
3.65399999999999	-6.04912706171834e-15\\
3.65400000000002	-6.04912706171689e-15\\
3.65799999999998	-5.84773430538666e-15\\
3.65999999999997	-5.74958915644303e-15\\
3.66	-5.74958915644164e-15\\
3.66399999999997	-5.55820891335941e-15\\
3.66799999999993	-5.37312634482376e-15\\
3.66999999999997	-5.28285559852668e-15\\
3.67	-5.2828555985254e-15\\
3.67399999999999	-5.10686667611191e-15\\
3.67400000000002	-5.10686667611069e-15\\
3.67499999999998	-5.06381231762075e-15\\
3.67500000000001	-5.06381231761953e-15\\
3.67599999999997	-5.02112661276939e-15\\
3.67699999999994	-4.97880537775523e-15\\
3.67899999999988	-4.89523976086546e-15\\
3.67999999999997	-4.85398718853167e-15\\
3.68	-4.8539871885305e-15\\
3.68299999999996	-4.7323019999841e-15\\
3.68299999999999	-4.73230199998296e-15\\
3.68599999999995	-4.61364246739825e-15\\
3.68899999999992	-4.49790391502724e-15\\
3.69	-4.45995578443842e-15\\
3.69000000000003	-4.45995578443735e-15\\
3.69599999999995	-4.23899580476093e-15\\
3.69999999999997	-4.09789135267859e-15\\
3.7	-4.0978913526776e-15\\
3.70599999999993	-3.89498464729549e-15\\
3.70999999999997	-3.76523743084398e-15\\
3.71	-3.76523743084307e-15\\
3.71199999999999	-3.70197763004387e-15\\
3.71200000000002	-3.70197763004298e-15\\
3.71400000000001	-3.63980525962263e-15\\
3.716	-3.57869594361301e-15\\
3.71999999999998	-3.45957105028311e-15\\
3.72000000000001	-3.45957105028228e-15\\
3.72799999999997	-3.23305054724439e-15\\
3.73000000000001	-3.17873396086767e-15\\
3.73000000000004	-3.17873396086691e-15\\
3.73200000000002	-3.12532801019286e-15\\
3.73200000000005	-3.12532801019211e-15\\
3.73400000000003	-3.07284010532886e-15\\
3.73600000000002	-3.02124966499903e-15\\
3.73999999999999	-2.92068061748659e-15\\
3.74000000000002	-2.92068061748588e-15\\
3.74099999999999	-2.89606808109949e-15\\
3.74100000000002	-2.89606808109879e-15\\
3.742	-2.87166258097748e-15\\
3.74299999999997	-2.84746172400236e-15\\
3.74499999999993	-2.7996644715727e-15\\
3.745	-2.79966447157087e-15\\
3.74500000000003	-2.79966447157019e-15\\
3.74899999999994	-2.70642263469493e-15\\
3.75000000000001	-2.68358895960908e-15\\
3.75000000000004	-2.68358895960844e-15\\
3.75399999999995	-2.59418998005298e-15\\
3.75799999999986	-2.50782197068299e-15\\
3.76	-2.46573206740449e-15\\
3.76000000000003	-2.4657320674039e-15\\
3.76799999999985	-2.30428463445726e-15\\
3.76999999999996	-2.26557169972573e-15\\
3.76999999999999	-2.26557169972519e-15\\
3.77799999999981	-2.11718358190526e-15\\
3.77999999999996	-2.08164993834561e-15\\
3.77999999999999	-2.0816499383451e-15\\
3.78799999999981	-1.94535084780852e-15\\
3.78999999999996	-1.91266814916523e-15\\
3.78999999999999	-1.91266814916477e-15\\
3.79799999999981	-1.78739415041422e-15\\
3.79899999999996	-1.77233108604529e-15\\
3.79899999999999	-1.77233108604487e-15\\
3.79999999999997	-1.75739551162165e-15\\
3.8	-1.75739551162123e-15\\
3.80099999999999	-1.74258596397104e-15\\
3.80199999999997	-1.72790099092074e-15\\
3.80399999999993	-1.69889902356805e-15\\
3.80799999999986	-1.64232745653036e-15\\
3.80999999999997	-1.61473567620545e-15\\
3.81	-1.61473567620506e-15\\
3.81499999999998	-1.54778381986355e-15\\
3.81500000000001	-1.54778381986318e-15\\
3.81899999999996	-1.4962586341466e-15\\
3.81899999999999	-1.49625863414624e-15\\
3.81999999999996	-1.4836495446832e-15\\
3.81999999999999	-1.48364954468284e-15\\
3.82099999999996	-1.47114685052527e-15\\
3.82199999999993	-1.4587493262319e-15\\
3.82399999999988	-1.43426493708114e-15\\
3.82799999999976	-1.38650540940688e-15\\
3.82799999999996	-1.38650540940449e-15\\
3.82799999999999	-1.38650540940415e-15\\
3.82999999999996	-1.3632115452759e-15\\
3.82999999999999	-1.36321154527557e-15\\
3.83199999999996	-1.34030821194754e-15\\
3.83399999999993	-1.31779858444858e-15\\
3.83799999999988	-1.27392529812536e-15\\
3.83999999999996	-1.25254443741548e-15\\
3.83999999999999	-1.25254443741518e-15\\
3.84799999999988	-1.17053224902326e-15\\
3.84999999999996	-1.15086682332075e-15\\
3.84999999999999	-1.15086682332047e-15\\
3.85699999999996	-1.08462967412263e-15\\
3.85699999999999	-1.08462967412237e-15\\
3.85999999999997	-1.05743810867256e-15\\
3.86	-1.05743810867231e-15\\
3.86299999999999	-1.0309290662916e-15\\
3.86599999999997	-1.00507916113647e-15\\
3.86999999999997	-9.71598613003661e-16\\
3.87	-9.71598613003426e-16\\
3.87599999999997	-9.23462619136342e-16\\
3.87699999999996	-9.1567909143967e-16\\
3.87699999999999	-9.1567909143945e-16\\
3.87999999999996	-8.92723101879098e-16\\
3.87999999999999	-8.92723101878884e-16\\
3.88299999999996	-8.70343319737414e-16\\
3.88499999999998	-8.55733810323189e-16\\
3.88500000000001	-8.55733810322983e-16\\
3.88599999999996	-8.48520002322886e-16\\
3.88599999999999	-8.48520002322682e-16\\
3.88699999999996	-8.41365880794771e-16\\
3.88799999999993	-8.34270744552441e-16\\
3.88999999999986	-8.20254652048968e-16\\
3.88999999999996	-8.20254652048277e-16\\
3.88999999999999	-8.2025465204808e-16\\
3.89399999999986	-7.92929331691809e-16\\
3.89799999999973	-7.66530445048093e-16\\
3.89999999999996	-7.53665420200354e-16\\
3.89999999999999	-7.53665420200173e-16\\
3.90799999999973	-7.0431806898189e-16\\
3.90999999999996	-6.92485234259031e-16\\
3.90999999999999	-6.92485234258864e-16\\
3.91499999999996	-6.63772688641427e-16\\
3.91499999999999	-6.63772688641268e-16\\
3.91999999999997	-6.36268472895701e-16\\
3.92	-6.36268472895503e-16\\
3.92499999999998	-6.09905180502007e-16\\
3.92999999999995	-5.84618201876229e-16\\
3.93	-5.84618201875969e-16\\
3.93499999999996	-5.60378149612068e-16\\
3.93499999999999	-5.60378149611933e-16\\
3.93999999999995	-5.37158210229685e-16\\
3.93999999999999	-5.37158210229515e-16\\
3.94399999999996	-5.19278416684128e-16\\
3.94399999999999	-5.19278416684003e-16\\
3.94799999999997	-5.01986986502629e-16\\
3.94999999999996	-4.93553398031523e-16\\
3.94999999999999	-4.93553398031404e-16\\
3.95399999999996	-4.77111546931145e-16\\
3.95499999999998	-4.73089172258844e-16\\
3.95500000000001	-4.7308917225873e-16\\
3.95899999999998	-4.57340198022226e-16\\
3.95999999999999	-4.53486156111381e-16\\
3.96000000000002	-4.53486156111272e-16\\
3.96399999999999	-4.38391462091876e-16\\
3.96799999999997	-4.23793483269411e-16\\
3.97000000000002	-4.16673577535707e-16\\
3.97000000000005	-4.16673577535606e-16\\
3.97299999999996	-4.06218066978328e-16\\
3.97299999999999	-4.0621806697823e-16\\
3.97599999999991	-3.96030281891238e-16\\
3.97899999999983	-3.86101234483971e-16\\
3.97999999999997	-3.82847528909042e-16\\
3.98	-3.82847528908949e-16\\
3.98599999999984	-3.63890869853576e-16\\
3.98999999999997	-3.51769172113384e-16\\
3.99	-3.517691721133e-16\\
3.99299999999996	-3.42942295825928e-16\\
3.99299999999999	-3.42942295825845e-16\\
3.99599999999995	-3.34341441549184e-16\\
3.99899999999991	-3.25959021903905e-16\\
3.99999999999997	-3.23212139508281e-16\\
4	-3.23212139508203e-16\\
4.00199999999993	-3.17787642180681e-16\\
4.00199999999999	-3.17787642180529e-16\\
4.00399999999992	-3.12453733113924e-16\\
4.00599999999986	-3.07208321038727e-16\\
4.00999999999972	-2.96974795871342e-16\\
4.00999999999995	-2.96974795870776e-16\\
4.01	-2.96974795870633e-16\\
4.01799999999973	-2.77523842018259e-16\\
4.01999999999995	-2.72866034695425e-16\\
4.02	-2.72866034695294e-16\\
4.02500000000001	-2.61560041197087e-16\\
4.02500000000006	-2.61560041196961e-16\\
4.02999999999995	-2.50715629080336e-16\\
4.03	-2.50715629080215e-16\\
4.03099999999999	-2.48600337330702e-16\\
4.03100000000005	-2.48600337330582e-16\\
4.03200000000004	-2.46503352825188e-16\\
4.03300000000003	-2.4442447001921e-16\\
4.035	-2.40320196251141e-16\\
4.03899999999996	-2.32320020440631e-16\\
4.03999999999994	-2.30362241239899e-16\\
4.04	-2.30362241239788e-16\\
4.04799999999991	-2.15278934853653e-16\\
4.04999999999994	-2.1166215974488e-16\\
4.05	-2.11662159744778e-16\\
4.05099999999999	-2.09876362664372e-16\\
4.05100000000005	-2.09876362664271e-16\\
4.05200000000004	-2.0810602121176e-16\\
4.05300000000002	-2.0635096179344e-16\\
4.055	-2.02886002484005e-16\\
4.05899999999996	-1.96131997947154e-16\\
4.05999999999994	-1.94479178144627e-16\\
4.06	-1.94479178144534e-16\\
};
\pgfplotsset{max space between ticks=50}
\addplot [color=mycolor2,solid,forget plot]
  table[row sep=crcr]{%
0	0.15314\\
3.15544362088405e-30	0.15314\\
0.000656101980281985	0.153143230512962\\
0.00393661188169191	0.153256312778436\\
0.00999999999999994	0.153891071773171\\
0.01	0.153891071773171\\
0.0199999999999999	0.150048203824684\\
0.02	0.150048203824684\\
0.0289999999999998	0.137414337712804\\
0.029	0.137414337712803\\
0.03	0.135470213386942\\
0.0300000000000002	0.135470213386942\\
0.0349999999999996	0.124115011067004\\
0.035	0.124115011067003\\
0.0399999999999993	0.110014119663841\\
0.04	0.110014119663839\\
0.0449999999999993	0.0939630779639858\\
0.0499999999999987	0.0767526455719492\\
0.05	0.0767526455719445\\
0.0500000000000004	0.0767526455719429\\
0.0579999999999996	0.0466980443355424\\
0.058	0.0466980443355407\\
0.0599999999999996	0.0386819498575326\\
0.06	0.0386819498575308\\
0.0619999999999995	0.0306155753047838\\
0.0639999999999991	0.0226526210813104\\
0.0679999999999982	0.00702452540129683\\
0.0699999999999991	-0.000646743531092999\\
0.07	-0.000646743531096385\\
0.0779999999999982	-0.0304497245417899\\
0.0799999999999991	-0.0376945518218964\\
0.08	-0.0376945518218996\\
0.087	-0.0613629561565828\\
0.0870000000000009	-0.0613629561565856\\
0.09	-0.0705722498255595\\
0.0900000000000009	-0.0705722498255621\\
0.0929999999999999	-0.0792329294508874\\
0.095999999999999	-0.0873526353855747\\
0.0999999999999991	-0.0973497076333374\\
0.1	-0.0973497076333395\\
0.104999999999999	-0.108387871323025\\
0.105	-0.108387871323027\\
0.109999999999999	-0.117699946526397\\
0.11	-0.117699946526398\\
0.114999999999999	-0.125308754745586\\
0.115999999999999	-0.126627838237935\\
0.116	-0.126627838237936\\
0.119999999999999	-0.131232943213937\\
0.12	-0.131232943213938\\
0.123999999999999	-0.13476926202664\\
0.127999999999998	-0.137242340899987\\
0.129999999999998	-0.138081442501912\\
0.13	-0.138081442501912\\
0.137999999999998	-0.138771927810782\\
0.139999999999998	-0.138276983003393\\
0.14	-0.138276983003392\\
0.144999999999998	-0.135869686909084\\
0.145	-0.135869686909083\\
0.149999999999998	-0.131785845077204\\
0.15	-0.131785845077202\\
0.154999999999998	-0.126461814037968\\
0.159999999999996	-0.120330910865585\\
0.16	-0.12033091086558\\
0.169999999999996	-0.105586372774747\\
0.17	-0.105586372774741\\
0.173999999999998	-0.0990022340863097\\
0.174	-0.0990022340863068\\
0.174999999999998	-0.0973523335757812\\
0.175	-0.0973523335757782\\
0.176	-0.0957005634817941\\
0.177	-0.0940467619077872\\
0.179000000000001	-0.0907324157456936\\
0.179999999999998	-0.0890715463112649\\
0.18	-0.0890715463112619\\
0.184000000000001	-0.0823996248083219\\
0.188000000000002	-0.0756743350020158\\
0.189999999999998	-0.0722883843683618\\
0.19	-0.0722883843683588\\
0.198000000000002	-0.058558154107107\\
0.199999999999998	-0.0550722607951608\\
0.2	-0.0550722607951577\\
0.202999999999998	-0.0499542634358041\\
0.203	-0.0499542634358012\\
0.205999999999998	-0.045090375635058\\
0.208999999999996	-0.0404763063780602\\
0.209999999999998	-0.0389930887103154\\
0.21	-0.0389930887103128\\
0.215999999999996	-0.0305025648988234\\
0.219999999999998	-0.0251966015169791\\
0.22	-0.0251966015169768\\
0.225999999999996	-0.0177484283157274\\
0.229999999999998	-0.0131121370239185\\
0.23	-0.0131121370239165\\
0.231999999999998	-0.0108899737028716\\
0.232	-0.0108899737028697\\
0.233999999999998	-0.00873062873450069\\
0.235999999999997	-0.00663325550056961\\
0.239999999999993	-0.00262115909906458\\
0.239999999999996	-0.0026211590990612\\
0.24	-0.00262115909905783\\
0.244999999999998	0.00188687770292411\\
0.245	0.00188687770292559\\
0.249999999999998	0.00568827569695383\\
0.25	0.00568827569695508\\
0.254999999999999	0.00879235129348301\\
0.259999999999997	0.0112067117758103\\
0.26	0.0112067117758117\\
0.260999999999996	0.0116103534305937\\
0.261	0.0116103534305951\\
0.262	0.0119926581405718\\
0.263	0.0123536634279823\\
0.265	0.0130119151800953\\
0.269	0.0140742429758504\\
0.269999999999997	0.0142870265310786\\
0.27	0.0142870265310794\\
0.278	0.015231929494095\\
0.279999999999996	0.0152581815612772\\
0.28	0.0152581815612772\\
0.288	0.0148381646770985\\
0.289999999999996	0.0146213487868583\\
0.29	0.0146213487868579\\
0.298	0.0133042875626112\\
0.299999999999996	0.0128619274084787\\
0.3	0.0128619274084778\\
0.308	0.01063490283314\\
0.309999999999996	0.00996265931864272\\
0.31	0.00996265931864148\\
0.314999999999997	0.00820076427699913\\
0.315	0.00820076427699786\\
0.319	0.00675620801608072\\
0.319000000000004	0.00675620801607942\\
0.319999999999996	0.00638990890522511\\
0.32	0.0063899089052238\\
0.321	0.00602147428209788\\
0.321999999999999	0.00565086803499177\\
0.323999999999998	0.00490299515937\\
0.327999999999996	0.00337957699652842\\
0.329999999999996	0.00260343440859324\\
0.33	0.00260343440859186\\
0.337999999999996	-0.000386934584574294\\
0.339999999999996	-0.00109385045098048\\
0.34	-0.00109385045098172\\
0.347999999999996	-0.00376712537690133\\
0.348	-0.0037671253769025\\
0.349999999999996	-0.00439815708383284\\
0.35	-0.00439815708383395\\
0.351999999999996	-0.00501477739830985\\
0.353999999999993	-0.00561722810735759\\
0.357999999999985	-0.00678056001263201\\
0.359999999999996	-0.00734189732970837\\
0.36	-0.00734189732970936\\
0.367999999999985	-0.00927534065693867\\
0.369999999999996	-0.00967033707065848\\
0.37	-0.00967033707065915\\
0.377	-0.0108624199510937\\
0.377000000000004	-0.0108624199510943\\
0.379999999999997	-0.011295123023204\\
0.38	-0.0112951230232045\\
0.382999999999993	-0.0116814835302473\\
0.384999999999997	-0.0119134813252346\\
0.385	-0.011913481325235\\
0.387999999999993	-0.0122233415981324\\
0.389999999999997	-0.0124046089407074\\
0.39	-0.0124046089407077\\
0.392999999999993	-0.0126387283665326\\
0.395999999999986	-0.0128276905077501\\
0.399999999999997	-0.0130096777594896\\
0.4	-0.0130096777594898\\
0.405999999999986	-0.0131344066698936\\
0.406	-0.0131344066698937\\
0.406000000000004	-0.0131344066698937\\
0.41	-0.0131194192205885\\
0.410000000000004	-0.0131194192205884\\
0.414	-0.013025910252926\\
0.417999999999996	-0.0128537331127516\\
0.419999999999997	-0.0127380644190282\\
0.42	-0.012738064419028\\
0.427999999999993	-0.0121511608627946\\
0.429999999999997	-0.0119777204409489\\
0.43	-0.0119777204409485\\
0.435	-0.0114966423981403\\
0.435000000000004	-0.01149664239814\\
0.439999999999997	-0.0109467768969577\\
0.44	-0.0109467768969573\\
0.444999999999993	-0.010355451915085\\
0.449999999999986	-0.00974989383042669\\
0.449999999999993	-0.00974989383042581\\
0.45	-0.00974989383042494\\
0.454999999999997	-0.00912861851892244\\
0.455	-0.00912861851892199\\
0.459999999999997	-0.00849010351173955\\
0.46	-0.00849010351173909\\
0.463999999999997	-0.00797875498394609\\
0.464	-0.00797875498394564\\
0.467999999999997	-0.00748042273240351\\
0.469999999999997	-0.00723589270700806\\
0.47	-0.00723589270700762\\
0.473999999999997	-0.00675562548054035\\
0.477999999999993	-0.00628645622411489\\
0.479999999999997	-0.00605580272715413\\
0.48	-0.00605580272715372\\
0.487999999999993	-0.00515729754952254\\
0.489999999999997	-0.00493825812254068\\
0.49	-0.0049382581225403\\
0.492999999999997	-0.00462164187662262\\
0.493	-0.00462164187662226\\
0.495999999999997	-0.00432559297658477\\
0.498999999999993	-0.00404985024110665\\
0.499999999999997	-0.00396240592092304\\
0.5	-0.00396240592092273\\
0.505999999999993	-0.00348408566948026\\
0.509999999999993	-0.00320878304145653\\
0.51	-0.00320878304145607\\
0.515999999999993	-0.00285520465084399\\
0.519999999999993	-0.00265634789090958\\
0.52	-0.00265634789090925\\
0.521999999999993	-0.00256785689554054\\
0.522	-0.00256785689554024\\
0.523999999999993	-0.00248661028412352\\
0.524999999999993	-0.00244869355699522\\
0.525	-0.00244869355699495\\
0.526999999999993	-0.00237825467839496\\
0.528999999999986	-0.00231498584786251\\
0.529999999999993	-0.00228603233951768\\
0.53	-0.00228603233951748\\
0.533999999999986	-0.00218245323015712\\
0.537999999999972	-0.00209610281135348\\
0.539999999999993	-0.00205934498713197\\
0.54	-0.00205934498713185\\
0.547999999999972	-0.00195477946619444\\
0.549999999999993	-0.00193920328278263\\
0.55	-0.00193920328278258\\
0.550999999999993	-0.00193299477772194\\
0.551	-0.0019329947777219\\
0.551999999999997	-0.00192783848496902\\
0.552999999999993	-0.00192373389857262\\
0.554999999999986	-0.00191867833871959\\
0.558999999999972	-0.00192117689668293\\
0.559999999999993	-0.00192442875915677\\
0.56	-0.0019244287591568\\
0.567999999999972	-0.00197140268441475\\
0.57	-0.001988401838639\\
0.570000000000007	-0.00198840183863906\\
0.577999999999979	-0.0020592072895557\\
0.579999999999993	-0.00207649525762915\\
0.58	-0.00207649525762921\\
0.587999999999972	-0.00214419694775721\\
0.589999999999993	-0.00216079300364885\\
0.59	-0.00216079300364891\\
0.594999999999993	-0.00220177812842115\\
0.595	-0.00220177812842121\\
0.599999999999993	-0.00224212195069527\\
0.6	-0.00224212195069533\\
0.604999999999993	-0.00227749864227557\\
0.608999999999993	-0.00229909690584826\\
0.609	-0.0022990969058483\\
0.609999999999993	-0.00230357020084521\\
0.61	-0.00230357020084524\\
0.610999999999997	-0.00230767391869955\\
0.611999999999993	-0.00231140846167129\\
0.613999999999986	-0.00231777145091609\\
0.617999999999972	-0.00232608101572473\\
0.619999999999993	-0.00232803084947865\\
0.62	-0.00232803084947866\\
0.627999999999972	-0.00232051978163345\\
0.629999999999993	-0.00231477188658906\\
0.63	-0.00231477188658903\\
0.637999999999972	-0.00227623216538157\\
0.637999999999993	-0.00227623216538144\\
0.638	-0.00227623216538139\\
0.639999999999993	-0.00226269131337631\\
0.64	-0.00226269131337626\\
0.641999999999993	-0.00224783108334466\\
0.643999999999986	-0.00223189833618672\\
0.647999999999971	-0.00219678988174738\\
0.649999999999993	-0.0021776004092714\\
0.65	-0.00217760040927134\\
0.657999999999971	-0.00208976029042206\\
0.659999999999993	-0.00206498665594992\\
0.66	-0.00206498665594983\\
0.664999999999993	-0.00199802433862666\\
0.665	-0.00199802433862656\\
0.666999999999993	-0.00196920055372759\\
0.667	-0.00196920055372749\\
0.668999999999993	-0.00193919501446106\\
0.669999999999993	-0.0019237454289288\\
0.67	-0.00192374542892869\\
0.671999999999993	-0.00189246372396009\\
0.673999999999986	-0.00186100751792911\\
0.677999999999972	-0.00179752219965694\\
0.679999999999993	-0.00176546819611356\\
0.68	-0.00176546819611344\\
0.687999999999972	-0.00163488227406933\\
0.689999999999993	-0.00160157931998955\\
0.69	-0.00160157931998944\\
0.695999999999993	-0.00150276358338633\\
0.696	-0.00150276358338622\\
0.699999999999993	-0.00143846754788163\\
0.7	-0.00143846754788152\\
0.703999999999993	-0.00137531848145053\\
0.707999999999986	-0.00131321734327865\\
0.709999999999993	-0.00128252924894784\\
0.71	-0.00128252924894773\\
0.717999999999986	-0.00116191364707716\\
0.719999999999993	-0.00113223482981509\\
0.72	-0.00113223482981499\\
0.724999999999993	-0.00106143079390376\\
0.725	-0.00106143079390366\\
0.729999999999993	-0.000996873375159433\\
0.730000000000001	-0.000996873375159346\\
0.734999999999994	-0.000936717601452401\\
0.735000000000001	-0.000936717601452317\\
0.739999999999994	-0.000879129290255161\\
0.740000000000001	-0.000879129290255081\\
0.744999999999994	-0.000823967307569958\\
0.749999999999987	-0.000771096465700289\\
0.750000000000001	-0.000771096465700144\\
0.753999999999993	-0.00073036215460301\\
0.754	-0.000730362154602938\\
0.757999999999993	-0.000690947265248271\\
0.759999999999993	-0.000671715205610819\\
0.76	-0.000671715205610751\\
0.763999999999993	-0.000635005393658975\\
0.767999999999986	-0.00060114792352203\\
0.77	-0.000585272049199614\\
0.770000000000007	-0.000585272049199559\\
0.777999999999993	-0.000528645554914799\\
0.779999999999993	-0.000516180430556586\\
0.78	-0.000516180430556543\\
0.782999999999993	-0.000498598482522057\\
0.783	-0.000498598482522017\\
0.785999999999993	-0.000482235829801109\\
0.788999999999986	-0.000467078036854374\\
0.79	-0.000462290844231463\\
0.790000000000007	-0.000462290844231429\\
0.795999999999993	-0.000436322274410771\\
0.8	-0.000421602903029317\\
0.800000000000007	-0.000421602903029293\\
0.804999999999993	-0.000405445605427315\\
0.805	-0.000405445605427293\\
0.809999999999987	-0.00039117882537358\\
0.809999999999997	-0.000391178825373552\\
0.810000000000007	-0.000391178825373524\\
0.811999999999993	-0.00038599347258466\\
0.812	-0.000385993472584642\\
0.813999999999987	-0.00038110292454074\\
0.815999999999973	-0.000376505263846724\\
0.819999999999945	-0.000368181508555571\\
0.819999999999987	-0.00036818150855549\\
0.82	-0.000368181508555463\\
0.827999999999944	-0.00035497694563609\\
0.829999999999993	-0.000352385372545459\\
0.830000000000001	-0.00035238537254545\\
0.837999999999945	-0.000344115707971929\\
0.839999999999993	-0.000342523970614159\\
0.84	-0.000342523970614154\\
0.840999999999993	-0.000341798980015613\\
0.841000000000001	-0.000341798980015608\\
0.841999999999997	-0.000341121145212625\\
0.842999999999993	-0.000340490399724983\\
0.844999999999986	-0.000339369934026441\\
0.848999999999972	-0.000337691662864691\\
0.849999999999993	-0.000337389069357229\\
0.85	-0.000337389069357227\\
0.857999999999972	-0.000335259240190865\\
0.859999999999993	-0.000334758414659414\\
0.86	-0.000334758414659412\\
0.867999999999972	-0.000332875747277424\\
0.869999999999993	-0.000332434316465353\\
0.87	-0.000332434316465352\\
0.874999999999994	-0.000331068242922347\\
0.875000000000001	-0.000331068242922344\\
0.879999999999994	-0.000329146222069652\\
0.880000000000001	-0.000329146222069649\\
0.884999999999994	-0.000326663543542365\\
0.889999999999987	-0.000323614122940048\\
0.890000000000001	-0.000323614122940039\\
0.890000000000008	-0.000323614122940034\\
0.899	-0.000316486733966596\\
0.899000000000008	-0.000316486733966589\\
0.9	-0.000315554720433605\\
0.900000000000007	-0.000315554720433599\\
0.901000000000004	-0.000314594375941841\\
0.902	-0.000313605606321269\\
0.903999999999993	-0.0003115424012594\\
0.907999999999979	-0.00030707043771199\\
0.909999999999993	-0.00030465992587264\\
0.910000000000001	-0.000304659925872631\\
0.917999999999972	-0.000293831753429642\\
0.919999999999993	-0.000290822872510042\\
0.920000000000001	-0.000290822872510032\\
0.927999999999972	-0.000278102617137219\\
0.927999999999994	-0.000278102617137184\\
0.928000000000001	-0.000278102617137172\\
0.929999999999994	-0.000274779995718834\\
0.930000000000001	-0.000274779995718822\\
0.931999999999994	-0.000271402704015578\\
0.933999999999987	-0.000267974341572245\\
0.937999999999972	-0.000260959007541334\\
0.939999999999993	-0.000257369285387744\\
0.940000000000001	-0.000257369285387731\\
0.944999999999994	-0.000248144342019191\\
0.945000000000001	-0.000248144342019178\\
0.949999999999994	-0.000238543151774327\\
0.950000000000001	-0.000238543151774313\\
0.954999999999994	-0.000228542184423157\\
0.956999999999994	-0.000224424403949383\\
0.957000000000001	-0.000224424403949368\\
0.959999999999993	-0.000218116930134359\\
0.960000000000001	-0.000218116930134344\\
0.962999999999993	-0.000211787464606918\\
0.965999999999986	-0.000205570352027769\\
0.969999999999993	-0.000197446174189792\\
0.970000000000001	-0.000197446174189778\\
0.975999999999986	-0.000185588360519031\\
0.979999999999993	-0.000177884039190569\\
0.980000000000001	-0.000177884039190555\\
0.985999999999986	-0.000166598677044058\\
0.985999999999993	-0.000166598677044044\\
0.986000000000001	-0.000166598677044031\\
0.989999999999993	-0.000159238640787061\\
0.990000000000001	-0.000159238640787048\\
0.993999999999993	-0.000152203250136343\\
0.997999999999986	-0.000145688781985193\\
0.999999999999993	-0.000142623669357667\\
1	-0.000142623669357657\\
1.00799999999999	-0.000131616565662225\\
1.00999999999999	-0.00012917273208024\\
1.01	-0.000129172732080223\\
1.01499999999999	-0.000123518085292801\\
1.015	-0.000123518085292785\\
1.01999999999999	-0.000118460570614788\\
1.02	-0.000118460570614774\\
1.02499999999999	-0.000113987793199313\\
1.02999999999997	-0.000110088791430314\\
1.03	-0.000110088791430294\\
1.03999999999997	-0.000103479707608394\\
1.04	-0.000103479707608377\\
1.04399999999999	-0.000101175945377768\\
1.044	-0.000101175945377761\\
1.04799999999999	-9.90608062925625e-05\\
1.04999999999999	-9.80729222172685e-05\\
1.05	-9.80729222172616e-05\\
1.05399999999999	-9.6177602974403e-05\\
1.05799999999997	-9.43490965384546e-05\\
1.05999999999999	-9.34589979262636e-05\\
1.06	-9.34589979262574e-05\\
1.06799999999997	-9.00515625303967e-05\\
1.06999999999999	-8.92362735415734e-05\\
1.07	-8.92362735415677e-05\\
1.07299999999999	-8.80394545716419e-05\\
1.073	-8.80394545716363e-05\\
1.07599999999999	-8.68730366506243e-05\\
1.07899999999997	-8.57359907502116e-05\\
1.07999999999999	-8.53633284642103e-05\\
1.08	-8.5363328464205e-05\\
1.08499999999999	-8.35461100915477e-05\\
1.085	-8.35461100915426e-05\\
1.08999999999999	-8.18025023960743e-05\\
1.09	-8.18025023960694e-05\\
1.09499999999999	-8.00618123239189e-05\\
1.09999999999997	-7.82533539751721e-05\\
1.09999999999999	-7.82533539751668e-05\\
1.1	-7.82533539751615e-05\\
1.10199999999999	-7.7510008469497e-05\\
1.102	-7.75100084694916e-05\\
1.10399999999999	-7.675482155282e-05\\
1.10599999999997	-7.59874970617088e-05\\
1.10999999999994	-7.44152271576691e-05\\
1.10999999999999	-7.44152271576516e-05\\
1.11	-7.44152271576459e-05\\
1.11799999999994	-7.11114664205761e-05\\
1.11999999999999	-7.0250474023457e-05\\
1.12	-7.02504740234508e-05\\
1.12799999999994	-6.67858740899437e-05\\
1.12999999999999	-6.59209714122933e-05\\
1.13	-6.59209714122871e-05\\
1.13099999999999	-6.54884739975316e-05\\
1.131	-6.54884739975254e-05\\
1.132	-6.50558893623983e-05\\
1.13299999999999	-6.46231750904298e-05\\
1.13499999999999	-6.37571879771102e-05\\
1.13899999999997	-6.20217906837041e-05\\
1.13999999999999	-6.15869802056827e-05\\
1.14	-6.15869802056765e-05\\
1.14799999999997	-5.80879995338669e-05\\
1.14999999999999	-5.72059883990769e-05\\
1.15	-5.72059883990706e-05\\
1.15499999999999	-5.50513794777781e-05\\
1.155	-5.50513794777721e-05\\
1.15999999999999	-5.30028342394061e-05\\
1.16	-5.30028342394004e-05\\
1.16499999999999	-5.10129318808779e-05\\
1.16999999999997	-4.90343953812653e-05\\
1.16999999999999	-4.90343953812597e-05\\
1.17	-4.9034395381254e-05\\
1.17999999999997	-4.50920404146876e-05\\
1.17999999999999	-4.50920404146819e-05\\
1.18	-4.50920404146763e-05\\
1.18899999999999	-4.15342623684143e-05\\
1.189	-4.15342623684087e-05\\
1.18999999999999	-4.11370988638729e-05\\
1.19	-4.11370988638673e-05\\
1.191	-4.07417576403654e-05\\
1.19199999999999	-4.03505355699611e-05\\
1.19399999999999	-3.9580295913081e-05\\
1.19799999999997	-3.80875858982017e-05\\
1.19999999999999	-3.73645302765287e-05\\
1.2	-3.73645302765236e-05\\
1.20799999999997	-3.4621091388731e-05\\
1.20999999999999	-3.39710830709112e-05\\
1.21	-3.39710830709066e-05\\
1.21799999999997	-3.15228970327698e-05\\
1.218	-3.15228970327618e-05\\
1.21999999999999	-3.09486176065481e-05\\
1.22	-3.0948617606544e-05\\
1.22199999999999	-3.03889859706495e-05\\
1.22399999999997	-2.98437826837512e-05\\
1.22499999999999	-2.9576524655378e-05\\
1.225	-2.95765246553742e-05\\
1.22899999999997	-2.85425094680746e-05\\
1.22999999999999	-2.82926332217404e-05\\
1.23	-2.82926332217369e-05\\
1.23399999999997	-2.7323487331463e-05\\
1.23799999999994	-2.6400420021507e-05\\
1.23999999999999	-2.59557106396736e-05\\
1.24	-2.59557106396705e-05\\
1.24699999999999	-2.44846078433585e-05\\
1.247	-2.44846078433556e-05\\
1.24999999999999	-2.38935603471624e-05\\
1.25	-2.38935603471597e-05\\
1.25299999999999	-2.33254048757642e-05\\
1.25599999999997	-2.27796402106095e-05\\
1.25999999999999	-2.20859547896303e-05\\
1.26	-2.20859547896279e-05\\
1.26599999999997	-2.11368468557044e-05\\
1.26999999999999	-2.05728278320262e-05\\
1.27	-2.05728278320243e-05\\
1.27599999999997	-1.98276669456435e-05\\
1.276	-1.98276669456403e-05\\
1.27999999999999	-1.93970018886283e-05\\
1.28	-1.93970018886269e-05\\
1.28399999999999	-1.90065384414176e-05\\
1.28799999999997	-1.86437787838281e-05\\
1.28999999999999	-1.8472608399611e-05\\
1.29	-1.84726083996098e-05\\
1.295	-1.8073924372266e-05\\
1.29500000000001	-1.80739243722649e-05\\
1.3	-1.77162451166257e-05\\
1.30000000000001	-1.77162451166248e-05\\
1.305	-1.7398694040283e-05\\
1.30500000000001	-1.73986940402822e-05\\
1.31	-1.71204929080031e-05\\
1.31000000000001	-1.71204929080023e-05\\
1.315	-1.68726622586677e-05\\
1.31999999999998	-1.66462970613828e-05\\
1.32	-1.66462970613821e-05\\
1.32999999999997	-1.6182407913925e-05\\
1.33	-1.61824079139237e-05\\
1.33399999999999	-1.59781921743013e-05\\
1.334	-1.59781921743006e-05\\
1.33799999999999	-1.5762842652407e-05\\
1.33999999999999	-1.56508878659748e-05\\
1.34	-1.5650887865974e-05\\
1.34399999999999	-1.54181987659343e-05\\
1.34799999999997	-1.51734975921986e-05\\
1.34999999999999	-1.50465232921688e-05\\
1.35	-1.50465232921679e-05\\
1.35799999999997	-1.45157703747024e-05\\
1.35999999999999	-1.43776793979001e-05\\
1.36	-1.43776793978991e-05\\
1.36299999999999	-1.41662731147417e-05\\
1.363	-1.41662731147407e-05\\
1.36499999999999	-1.40224084739941e-05\\
1.365	-1.40224084739931e-05\\
1.36699999999999	-1.3876136464917e-05\\
1.36899999999997	-1.37273997413389e-05\\
1.36999999999999	-1.36520889270863e-05\\
1.37	-1.36520889270852e-05\\
1.37399999999997	-1.33472884960282e-05\\
1.37799999999995	-1.30375964274968e-05\\
1.37999999999999	-1.28807645570106e-05\\
1.38	-1.28807645570095e-05\\
1.38799999999995	-1.22387569278706e-05\\
1.38999999999999	-1.20742703535102e-05\\
1.39	-1.2074270353509e-05\\
1.39199999999999	-1.19089782731835e-05\\
1.392	-1.19089782731824e-05\\
1.39399999999998	-1.17437322663343e-05\\
1.39599999999997	-1.15784675384878e-05\\
1.39999999999994	-1.12476227181129e-05\\
1.39999999999999	-1.12476227181091e-05\\
1.4	-1.12476227181079e-05\\
1.40799999999994	-1.05828539493579e-05\\
1.41	-1.04156403682309e-05\\
1.41000000000001	-1.04156403682297e-05\\
1.41799999999995	-9.74073318180252e-06\\
1.41999999999999	-9.57016241575112e-06\\
1.42	-9.5701624157499e-06\\
1.42099999999999	-9.48510940411088e-06\\
1.421	-9.48510940410967e-06\\
1.422	-9.40093985731432e-06\\
1.42299999999999	-9.31764552193155e-06\\
1.42499999999998	-9.15364991553894e-06\\
1.42899999999997	-8.83580768595152e-06\\
1.42999999999999	-8.75841594688794e-06\\
1.43	-8.75841594688685e-06\\
1.43499999999999	-8.38349188679117e-06\\
1.435	-8.38349188679013e-06\\
1.43999999999998	-8.02795891160401e-06\\
1.44	-8.02795891160285e-06\\
1.44499999999999	-7.6918743754733e-06\\
1.44999999999997	-7.37534330672069e-06\\
1.44999999999998	-7.37534330671979e-06\\
1.45	-7.37534330671889e-06\\
1.45999999999997	-6.79788465090023e-06\\
1.45999999999998	-6.79788465089938e-06\\
1.46	-6.79788465089855e-06\\
1.46999999999997	-6.28420796862364e-06\\
1.46999999999998	-6.2842079686229e-06\\
1.47	-6.28420796862214e-06\\
1.47899999999998	-5.86737449984043e-06\\
1.479	-5.8673744998398e-06\\
1.47999999999999	-5.82356391729689e-06\\
1.48	-5.82356391729627e-06\\
1.481	-5.78018912457834e-06\\
1.48199999999999	-5.73719690438202e-06\\
1.48399999999999	-5.65234336389853e-06\\
1.48799999999997	-5.48704421281186e-06\\
1.48999999999999	-5.40653379153875e-06\\
1.49	-5.40653379153819e-06\\
1.49799999999997	-5.09802290556219e-06\\
1.49999999999999	-5.02412669001387e-06\\
1.5	-5.02412669001335e-06\\
1.50499999999999	-4.84472233789819e-06\\
1.505	-4.84472233789769e-06\\
1.50799999999998	-4.74059844340479e-06\\
1.508	-4.7405984434043e-06\\
1.50999999999999	-4.67259159220506e-06\\
1.51	-4.67259159220458e-06\\
1.51199999999999	-4.60620380003686e-06\\
1.51399999999997	-4.54193200582875e-06\\
1.51799999999994	-4.41963643965947e-06\\
1.51999999999999	-4.36156471780063e-06\\
1.52	-4.36156471780023e-06\\
1.52799999999994	-4.1354138083054e-06\\
1.52999999999999	-4.07943410167076e-06\\
1.53	-4.07943410167036e-06\\
1.53699999999998	-3.88477286306704e-06\\
1.537	-3.88477286306665e-06\\
1.53999999999999	-3.80178680749949e-06\\
1.54	-3.8017868074991e-06\\
1.54299999999999	-3.71895902663506e-06\\
1.54599999999997	-3.63621645142227e-06\\
1.54999999999999	-3.52589940392205e-06\\
1.55	-3.52589940392166e-06\\
1.55599999999997	-3.36465081405609e-06\\
1.56	-3.26175233456832e-06\\
1.56000000000001	-3.26175233456796e-06\\
1.56599999999999	-3.1139019785524e-06\\
1.566	-3.11390197855206e-06\\
1.57	-3.01944120594047e-06\\
1.57000000000001	-3.01944120594014e-06\\
1.57400000000001	-2.92809175786984e-06\\
1.57499999999999	-2.90572364684587e-06\\
1.575	-2.90572364684556e-06\\
1.579	-2.81806243236287e-06\\
1.57999999999999	-2.79658919644889e-06\\
1.58	-2.79658919644859e-06\\
1.584	-2.71408272039318e-06\\
1.588	-2.63757133604752e-06\\
1.59	-2.60152593969683e-06\\
1.59000000000001	-2.60152593969658e-06\\
1.59499999999998	-2.51412913041771e-06\\
1.595	-2.51412913041746e-06\\
1.59999999999997	-2.42837582822453e-06\\
1.6	-2.42837582822399e-06\\
1.60000000000001	-2.42837582822375e-06\\
1.60499999999998	-2.34405587167826e-06\\
1.60999999999995	-2.26096261446063e-06\\
1.60999999999998	-2.26096261446026e-06\\
1.61	-2.26096261445989e-06\\
1.61999999999994	-2.09764414540558e-06\\
1.62	-2.0976441454046e-06\\
1.62000000000001	-2.09764414540437e-06\\
1.62399999999998	-2.03418756527293e-06\\
1.624	-2.03418756527271e-06\\
1.62799999999997	-1.9731982826703e-06\\
1.63000000000001	-1.94359883578264e-06\\
1.63000000000003	-1.94359883578244e-06\\
1.634	-1.886132304118e-06\\
1.63799999999997	-1.83090085696096e-06\\
1.63999999999999	-1.80409608563959e-06\\
1.64	-1.8040960856394e-06\\
1.64499999999999	-1.7392881786609e-06\\
1.645	-1.73928817866072e-06\\
1.64999999999998	-1.67745862571263e-06\\
1.65	-1.67745862571242e-06\\
1.653	-1.64172717467939e-06\\
1.65300000000001	-1.64172717467923e-06\\
1.65600000000001	-1.60698172692253e-06\\
1.65900000000001	-1.57319163153351e-06\\
1.66	-1.56213539098119e-06\\
1.66000000000002	-1.56213539098104e-06\\
1.66600000000001	-1.49789742039325e-06\\
1.67	-1.45699517814464e-06\\
1.67000000000002	-1.4569951781445e-06\\
1.67600000000001	-1.39968997688897e-06\\
1.67999999999998	-1.36468623029255e-06\\
1.68	-1.36468623029243e-06\\
1.68199999999998	-1.34784840090347e-06\\
1.682	-1.34784840090336e-06\\
1.68399999999998	-1.33107786948138e-06\\
1.68599999999997	-1.31436806081161e-06\\
1.68999999999994	-1.28110442856421e-06\\
1.68999999999998	-1.28110442856383e-06\\
1.69	-1.28110442856372e-06\\
1.69799999999994	-1.21501886846055e-06\\
1.69999999999998	-1.19855169779309e-06\\
1.7	-1.19855169779298e-06\\
1.70799999999994	-1.13270615264987e-06\\
1.70999999999998	-1.11621828157094e-06\\
1.71	-1.11621828157083e-06\\
1.711	-1.10800176251657e-06\\
1.71100000000001	-1.10800176251645e-06\\
1.71200000000001	-1.09985021937416e-06\\
1.71300000000001	-1.09176285283314e-06\\
1.715	-1.0757774851832e-06\\
1.71500000000001	-1.07577748518309e-06\\
1.71700000000001	-1.06003939248359e-06\\
1.71900000000001	-1.04454240459638e-06\\
1.71999999999998	-1.03688242267318e-06\\
1.72	-1.03688242267307e-06\\
1.724	-1.00681512337756e-06\\
1.72799999999999	-9.77628863119831e-07\\
1.72999999999998	-9.63351765227398e-07\\
1.73	-9.63351765227298e-07\\
1.73799999999999	-9.08220568807584e-07\\
1.74	-8.94905050029216e-07\\
1.74000000000001	-8.94905050029122e-07\\
1.74800000000001	-8.45423432646866e-07\\
1.74999999999998	-8.34103079362706e-07\\
1.75	-8.34103079362627e-07\\
1.75799999999999	-7.90179089452211e-07\\
1.75999999999998	-7.79346654904122e-07\\
1.76	-7.79346654904045e-07\\
1.76799999999999	-7.36483722537377e-07\\
1.76899999999998	-7.31169702581359e-07\\
1.769	-7.31169702581284e-07\\
1.76999999999998	-7.25863685036019e-07\\
1.77	-7.25863685035943e-07\\
1.771	-7.20565150087193e-07\\
1.77199999999999	-7.15273578191753e-07\\
1.77399999999998	-7.04709249676107e-07\\
1.77799999999997	-6.83641369082132e-07\\
1.77999999999998	-6.73129556706323e-07\\
1.78	-6.73129556706249e-07\\
1.78499999999998	-6.47467912446302e-07\\
1.785	-6.47467912446231e-07\\
1.78999999999998	-6.22974034316735e-07\\
1.79	-6.22974034316664e-07\\
1.79499999999998	-5.99587893434757e-07\\
1.798	-5.86063971681208e-07\\
1.79800000000001	-5.86063971681145e-07\\
1.8	-5.77252176305366e-07\\
1.80000000000001	-5.77252176305304e-07\\
1.802	-5.68601568826153e-07\\
1.80399999999999	-5.60110663255775e-07\\
1.80799999999996	-5.43594703668865e-07\\
1.81	-5.35563174055897e-07\\
1.81000000000001	-5.35563174055841e-07\\
1.81799999999996	-5.04873917348146e-07\\
1.81999999999998	-4.97545778528657e-07\\
1.82	-4.97545778528606e-07\\
1.82699999999998	-4.72812389832366e-07\\
1.827	-4.72812389832317e-07\\
1.82999999999998	-4.6261313542562e-07\\
1.83	-4.62613135425572e-07\\
1.83299999999999	-4.52641246370177e-07\\
1.83599999999997	-4.42887925632578e-07\\
1.84	-4.3020864893363e-07\\
1.84000000000001	-4.30208648933585e-07\\
1.84599999999999	-4.11822465360447e-07\\
1.85	-3.9994820017622e-07\\
1.85000000000001	-3.99948200176179e-07\\
1.85499999999998	-3.85503668253401e-07\\
1.855	-3.85503668253361e-07\\
1.85599999999998	-3.8266502424376e-07\\
1.856	-3.8266502424372e-07\\
1.85699999999999	-3.79842481433457e-07\\
1.85799999999999	-3.77035763159663e-07\\
1.85999999999998	-3.71468701395167e-07\\
1.86	-3.71468701395117e-07\\
1.86399999999999	-3.60512469061441e-07\\
1.86799999999997	-3.49779143941944e-07\\
1.86999999999999	-3.44490797809161e-07\\
1.87	-3.44490797809124e-07\\
1.87799999999997	-3.24598744015289e-07\\
1.87999999999999	-3.19980501534169e-07\\
1.88	-3.19980501534137e-07\\
1.88499999999998	-3.09035713196465e-07\\
1.885	-3.09035713196435e-07\\
1.88999999999998	-2.98928028982588e-07\\
1.89	-2.98928028982553e-07\\
1.89499999999998	-2.89519334852739e-07\\
1.89999999999996	-2.80673230156768e-07\\
1.89999999999998	-2.80673230156738e-07\\
1.9	-2.80673230156709e-07\\
1.90999999999996	-2.64583396732426e-07\\
1.91	-2.64583396732374e-07\\
1.91000000000001	-2.64583396732352e-07\\
1.91399999999998	-2.58640594949897e-07\\
1.914	-2.58640594949876e-07\\
1.91799999999997	-2.5285533944778e-07\\
1.91999999999999	-2.50018946547335e-07\\
1.92	-2.50018946547315e-07\\
1.92399999999997	-2.44453069985938e-07\\
1.92499999999998	-2.43083076799473e-07\\
1.925	-2.43083076799454e-07\\
1.92899999999997	-2.37684979347253e-07\\
1.92999999999999	-2.36355262466908e-07\\
1.93	-2.36355262466889e-07\\
1.93399999999997	-2.31111714788622e-07\\
1.93799999999994	-2.25982537400944e-07\\
1.93999999999999	-2.23458316883346e-07\\
1.94	-2.23458316883328e-07\\
1.94299999999998	-2.19750345120494e-07\\
1.943	-2.19750345120476e-07\\
1.94599999999998	-2.16158052084892e-07\\
1.94899999999996	-2.12678268605521e-07\\
1.95	-2.11542809512277e-07\\
1.95000000000002	-2.11542809512261e-07\\
1.95599999999998	-2.04711314393199e-07\\
1.95999999999998	-2.0008841789635e-07\\
1.96	-2.00088417896333e-07\\
1.96599999999996	-1.93033095225397e-07\\
1.97	-1.88238136792647e-07\\
1.97000000000001	-1.8823813679263e-07\\
1.97199999999998	-1.85810425800462e-07\\
1.972	-1.85810425800445e-07\\
1.97399999999997	-1.83361293977331e-07\\
1.97599999999994	-1.8088978109603e-07\\
1.97999999999988	-1.75875727165959e-07\\
1.98	-1.75875727165809e-07\\
1.98000000000002	-1.75875727165791e-07\\
1.9879999999999	-1.65941906997073e-07\\
1.99	-1.63514690440074e-07\\
1.99000000000002	-1.63514690440057e-07\\
1.99499999999998	-1.57534622188661e-07\\
1.995	-1.57534622188644e-07\\
1.99999999999997	-1.51668541679637e-07\\
1.99999999999998	-1.51668541679617e-07\\
2	-1.51668541679597e-07\\
2.00099999999997	-1.50509532132948e-07\\
2.001	-1.50509532132915e-07\\
2.00199999999999	-1.49357988038231e-07\\
2.00299999999999	-1.48213796522265e-07\\
2.00499999999998	-1.45947023365456e-07\\
2.00899999999997	-1.4149682065517e-07\\
2.00999999999997	-1.4040099612552e-07\\
2.01	-1.40400996125489e-07\\
2.01799999999997	-1.31858598355349e-07\\
2.01999999999997	-1.29781444248783e-07\\
2.02	-1.29781444248754e-07\\
2.02799999999997	-1.2168217103391e-07\\
2.02999999999997	-1.19705719280349e-07\\
2.03	-1.19705719280321e-07\\
2.03799999999997	-1.12298491979534e-07\\
2.03999999999997	-1.10588212849529e-07\\
2.04	-1.10588212849505e-07\\
2.04799999999997	-1.04293085417701e-07\\
2.04999999999997	-1.02852715724958e-07\\
2.05	-1.02852715724938e-07\\
2.05799999999997	-9.75006450775584e-08\\
2.05899999999997	-9.68735927366414e-08\\
2.059	-9.68735927366237e-08\\
2.05999999999997	-9.62556539200199e-08\\
2.06	-9.62556539200025e-08\\
2.06099999999999	-9.56467680918408e-08\\
2.06199999999999	-9.5046875545577e-08\\
2.06399999999998	-9.38738359915828e-08\\
2.06499999999997	-9.33005740116015e-08\\
2.065	-9.33005740115854e-08\\
2.06899999999998	-9.1094610399996e-08\\
2.06999999999997	-9.05646203133432e-08\\
2.07	-9.05646203133283e-08\\
2.07399999999998	-8.8497237322697e-08\\
2.07799999999997	-8.64988165408296e-08\\
2.07999999999997	-8.55244833791364e-08\\
2.08	-8.55244833791227e-08\\
2.08799999999997	-8.17841586434837e-08\\
2.088	-8.178415864347e-08\\
2.08999999999997	-8.08864963937365e-08\\
2.09	-8.08864963937238e-08\\
2.09199999999997	-8.00030892620572e-08\\
2.09399999999994	-7.91335908519449e-08\\
2.09799999999988	-7.74349619640981e-08\\
2.09999999999997	-7.66051654862678e-08\\
2.1	-7.66051654862561e-08\\
2.10799999999988	-7.34621867475569e-08\\
2.10999999999997	-7.27223032497461e-08\\
2.11	-7.27223032497357e-08\\
2.11699999999997	-7.01359118043542e-08\\
2.117	-7.01359118043436e-08\\
2.11999999999997	-6.90084786995951e-08\\
2.12	-6.90084786995844e-08\\
2.12299999999997	-6.78682262649285e-08\\
2.12599999999994	-6.67141485939064e-08\\
2.12999999999997	-6.5152110935404e-08\\
2.13	-6.51521109353928e-08\\
2.13499999999997	-6.31587170592184e-08\\
2.135	-6.3158717059207e-08\\
2.13999999999997	-6.11153729487431e-08\\
2.14	-6.11153729487314e-08\\
2.14499999999998	-5.90655764122158e-08\\
2.14599999999997	-5.8660219728273e-08\\
2.146	-5.86602197282615e-08\\
2.14999999999997	-5.70528094421023e-08\\
2.15	-5.7052809442091e-08\\
2.15399999999998	-5.54659396014975e-08\\
2.15799999999995	-5.38971214237435e-08\\
2.15999999999997	-5.31187117659023e-08\\
2.16	-5.31187117658912e-08\\
2.16799999999995	-5.00767085310798e-08\\
2.16999999999997	-4.93350460865037e-08\\
2.17	-4.93350460864932e-08\\
2.17499999999997	-4.75106647076705e-08\\
2.175	-4.75106647076602e-08\\
2.17999999999997	-4.57252541013196e-08\\
2.18	-4.57252541013083e-08\\
2.18499999999997	-4.39838928097029e-08\\
2.18999999999994	-4.22917673781084e-08\\
2.18999999999997	-4.22917673780976e-08\\
2.19	-4.22917673780881e-08\\
2.19999999999994	-3.90387467611347e-08\\
2.19999999999997	-3.90387467611244e-08\\
2.2	-3.90387467611154e-08\\
2.20399999999997	-3.77808690538949e-08\\
2.204	-3.77808690538861e-08\\
2.20499999999997	-3.74698790930857e-08\\
2.205	-3.74698790930769e-08\\
2.206	-3.71602197407196e-08\\
2.20699999999999	-3.68518606458321e-08\\
2.20899999999998	-3.62389224633498e-08\\
2.20999999999997	-3.59342833003281e-08\\
2.21	-3.59342833003195e-08\\
2.21399999999998	-3.47532303962502e-08\\
2.21799999999997	-3.36412082875508e-08\\
2.21999999999997	-3.31105345788058e-08\\
2.22	-3.31105345787984e-08\\
2.22799999999997	-3.11519764607153e-08\\
2.22999999999997	-3.07024112634375e-08\\
2.23	-3.07024112634312e-08\\
2.23299999999997	-3.0053460407616e-08\\
2.233	-3.005346040761e-08\\
2.23599999999997	-2.94313374040768e-08\\
2.23899999999994	-2.88354934054765e-08\\
2.23999999999997	-2.86426270073689e-08\\
2.24	-2.86426270073634e-08\\
2.24599999999994	-2.75444767703541e-08\\
2.24999999999997	-2.68673123794119e-08\\
2.25	-2.68673123794073e-08\\
2.25599999999994	-2.59107138025959e-08\\
2.25999999999997	-2.53018006409581e-08\\
2.26	-2.53018006409538e-08\\
2.26199999999997	-2.50056283860713e-08\\
2.262	-2.50056283860671e-08\\
2.26399999999997	-2.47148234869504e-08\\
2.26599999999994	-2.44292719274784e-08\\
2.26999999999988	-2.38734830398353e-08\\
2.26999999999997	-2.38734830398229e-08\\
2.27	-2.3873483039819e-08\\
2.27499999999997	-2.31986194585733e-08\\
2.275	-2.31986194585695e-08\\
2.27999999999997	-2.25374278449599e-08\\
2.28	-2.25374278449562e-08\\
2.28499999999997	-2.18882877717095e-08\\
2.28999999999995	-2.12496083650575e-08\\
2.29	-2.12496083650507e-08\\
2.29099999999997	-2.11229896032781e-08\\
2.291	-2.11229896032745e-08\\
2.29199999999999	-2.0996714190757e-08\\
2.29299999999999	-2.087076973941e-08\\
2.29499999999998	-2.06198243759904e-08\\
2.29899999999997	-2.01213640778561e-08\\
2.29999999999997	-1.99973923854767e-08\\
2.3	-1.99973923854732e-08\\
2.30799999999997	-1.9043034047624e-08\\
2.30999999999997	-1.8815452475335e-08\\
2.31	-1.88154524753318e-08\\
2.31799999999997	-1.794648069634e-08\\
2.31999999999997	-1.77391505165955e-08\\
2.32	-1.77391505165925e-08\\
2.32799999999997	-1.69383337235426e-08\\
2.32999999999997	-1.6744319071297e-08\\
2.33	-1.67443190712943e-08\\
2.33799999999997	-1.5990736025936e-08\\
2.33999999999997	-1.58075898429967e-08\\
2.34	-1.58075898429941e-08\\
2.34499999999997	-1.53557189594839e-08\\
2.345	-1.53557189594813e-08\\
2.34899999999997	-1.49988156417685e-08\\
2.349	-1.4998815641766e-08\\
2.34999999999997	-1.49101547050569e-08\\
2.35	-1.49101547050544e-08\\
2.35099999999999	-1.4821702314004e-08\\
2.35199999999999	-1.47334497990372e-08\\
2.35399999999998	-1.4557509817439e-08\\
2.35799999999996	-1.42076489904189e-08\\
2.35999999999997	-1.40335909714773e-08\\
2.36	-1.40335909714748e-08\\
2.36799999999996	-1.33434189214e-08\\
2.36999999999997	-1.31721672622281e-08\\
2.37	-1.31721672622257e-08\\
2.37799999999996	-1.2490311225873e-08\\
2.378	-1.24903112258702e-08\\
2.37999999999997	-1.23203007069544e-08\\
2.38	-1.23203007069519e-08\\
2.38199999999997	-1.21503382091269e-08\\
2.38399999999994	-1.19803570885372e-08\\
2.38799999999988	-1.16400723804213e-08\\
2.38999999999997	-1.14696353740644e-08\\
2.39	-1.14696353740619e-08\\
2.39799999999988	-1.08132943012622e-08\\
2.39999999999997	-1.06570464716145e-08\\
2.4	-1.06570464716123e-08\\
2.40699999999997	-1.01334965078332e-08\\
2.407	-1.01334965078311e-08\\
2.40999999999997	-9.91978269868879e-09\\
2.41	-9.9197826986868e-09\\
2.41299999999997	-9.71205926303422e-09\\
2.41499999999997	-9.57674009994162e-09\\
2.415	-9.57674009993971e-09\\
2.41799999999997	-9.37837341183913e-09\\
2.41999999999997	-9.24912804760087e-09\\
2.42	-9.24912804759905e-09\\
2.42299999999997	-9.05963196180421e-09\\
2.42599999999994	-8.87523397927092e-09\\
2.42999999999997	-8.63701986088292e-09\\
2.42999999999999	-8.63701986088126e-09\\
2.43599999999994	-8.29333449110274e-09\\
2.43599999999997	-8.29333449110102e-09\\
2.436	-8.29333449109933e-09\\
2.43999999999997	-8.07188734077937e-09\\
2.43999999999999	-8.07188734077782e-09\\
2.44399999999996	-7.85617476063091e-09\\
2.44799999999993	-7.64585843445041e-09\\
2.44999999999999	-7.54262046545015e-09\\
2.45000000000002	-7.5426204654487e-09\\
2.45799999999996	-7.14153100452718e-09\\
2.45999999999997	-7.04402767899343e-09\\
2.46	-7.04402767899205e-09\\
2.46499999999997	-6.80958804106677e-09\\
2.465	-6.80958804106549e-09\\
2.46999999999996	-6.59080123893297e-09\\
2.47	-6.59080123893132e-09\\
2.47499999999997	-6.38017963605352e-09\\
2.47999999999993	-6.17025561394541e-09\\
2.48	-6.1702556139426e-09\\
2.48000000000003	-6.17025561394141e-09\\
2.48499999999997	-5.96051468229027e-09\\
2.485	-5.96051468228908e-09\\
2.48999999999994	-5.75044286275796e-09\\
2.48999999999999	-5.75044286275615e-09\\
2.49000000000003	-5.75044286275433e-09\\
2.49399999999997	-5.58180124510572e-09\\
2.494	-5.58180124510452e-09\\
2.49799999999993	-5.41235390295718e-09\\
2.49999999999997	-5.32724509746643e-09\\
2.5	-5.32724509746521e-09\\
2.50399999999994	-5.1595007673463e-09\\
2.50799999999988	-4.99710918787815e-09\\
2.50999999999997	-4.91784076780849e-09\\
2.51	-4.91784076780737e-09\\
2.51799999999988	-4.61289670873426e-09\\
2.51999999999997	-4.53954365351169e-09\\
2.52	-4.53954365351066e-09\\
2.52299999999997	-4.43181696154967e-09\\
2.523	-4.43181696154866e-09\\
2.52599999999996	-4.32697636218877e-09\\
2.52899999999993	-4.224929363717e-09\\
2.52999999999997	-4.19151886857787e-09\\
2.53	-4.19151886857693e-09\\
2.53599999999993	-3.99718248669507e-09\\
2.53999999999997	-3.8732284782851e-09\\
2.54	-3.87322847828423e-09\\
2.54599999999993	-3.69510654296947e-09\\
2.54999999999997	-3.58124405766897e-09\\
2.55	-3.58124405766818e-09\\
2.55199999999997	-3.52571092877424e-09\\
2.552	-3.52571092877346e-09\\
2.55399999999996	-3.47108072725826e-09\\
2.55499999999997	-3.4440975171927e-09\\
2.555	-3.44409751719194e-09\\
2.55699999999997	-3.39078165503003e-09\\
2.55899999999994	-3.33831581999899e-09\\
2.55999999999997	-3.31239522185945e-09\\
2.56	-3.31239522185872e-09\\
2.56399999999994	-3.21073591335066e-09\\
2.56799999999987	-3.11219467034416e-09\\
2.56999999999997	-3.06404483987443e-09\\
2.57	-3.06404483987375e-09\\
2.57799999999987	-2.88524407180878e-09\\
2.57999999999997	-2.84432842347087e-09\\
2.58	-2.8443284234703e-09\\
2.58099999999997	-2.82442689914827e-09\\
2.581	-2.82442689914771e-09\\
2.58199999999999	-2.80489362923989e-09\\
2.58299999999999	-2.78572669831305e-09\\
2.58499999999998	-2.74848437488904e-09\\
2.58899999999996	-2.67831552252334e-09\\
2.58999999999997	-2.66166235193637e-09\\
2.59	-2.6616623519359e-09\\
2.59799999999997	-2.53710187026083e-09\\
2.59999999999997	-2.50817797495435e-09\\
2.6	-2.50817797495394e-09\\
2.60799999999997	-2.40100700001944e-09\\
2.60999999999997	-2.37629288570651e-09\\
2.61	-2.37629288570616e-09\\
2.61799999999996	-2.28546018087545e-09\\
2.62	-2.26471342760547e-09\\
2.62000000000002	-2.26471342760518e-09\\
2.62499999999997	-2.21602389595769e-09\\
2.625	-2.21602389595742e-09\\
2.62999999999995	-2.17168505826279e-09\\
2.62999999999999	-2.17168505826248e-09\\
2.63000000000002	-2.17168505826217e-09\\
2.63499999999997	-2.12898815044444e-09\\
2.63899999999997	-2.09407051001531e-09\\
2.639	-2.09407051001506e-09\\
2.63999999999997	-2.08522843386654e-09\\
2.64	-2.08522843386628e-09\\
2.641	-2.0763395648682e-09\\
2.64199999999999	-2.06740303174738e-09\\
2.64399999999999	-2.04938346488736e-09\\
2.64799999999997	-2.01273349967945e-09\\
2.64999999999997	-1.99408873163685e-09\\
2.65	-1.99408873163658e-09\\
2.65799999999997	-1.91720616116652e-09\\
2.65999999999997	-1.89737196597236e-09\\
2.66	-1.89737196597208e-09\\
2.66799999999997	-1.8173013964392e-09\\
2.668	-1.81730139643891e-09\\
2.66999999999997	-1.79718311763324e-09\\
2.67	-1.79718311763295e-09\\
2.67199999999998	-1.77700883752361e-09\\
2.67399999999995	-1.75677064560987e-09\\
2.6779999999999	-1.7160707605713e-09\\
2.67999999999997	-1.69559310984288e-09\\
2.68	-1.69559310984259e-09\\
2.6879999999999	-1.61365437524818e-09\\
2.68999999999997	-1.59318566162748e-09\\
2.69	-1.59318566162719e-09\\
2.69499999999997	-1.54195404365513e-09\\
2.695	-1.54195404365484e-09\\
2.69699999999997	-1.52141637299945e-09\\
2.697	-1.52141637299916e-09\\
2.69899999999996	-1.50084090326274e-09\\
2.69999999999997	-1.49053647351744e-09\\
2.7	-1.49053647351715e-09\\
2.70199999999997	-1.469962942423e-09\\
2.70399999999993	-1.44947897560284e-09\\
2.70799999999986	-1.40874764369428e-09\\
2.70999999999997	-1.3884843086538e-09\\
2.71	-1.38848430865351e-09\\
2.71799999999986	-1.30792805561576e-09\\
2.71999999999997	-1.28787378560283e-09\\
2.72	-1.28787378560255e-09\\
2.72599999999997	-1.22778861772034e-09\\
2.726	-1.22778861772005e-09\\
2.72999999999997	-1.18771801916427e-09\\
2.73	-1.18771801916399e-09\\
2.73399999999998	-1.14851586634797e-09\\
2.73799999999995	-1.11107349048925e-09\\
2.73999999999997	-1.09299376156537e-09\\
2.74	-1.09299376156512e-09\\
2.74799999999995	-1.02478774765229e-09\\
2.74999999999997	-1.00873104341549e-09\\
2.75	-1.00873104341526e-09\\
2.75499999999997	-9.70215503026923e-10\\
2.755	-9.7021550302671e-10\\
2.75999999999996	-9.33922475639151e-10\\
2.76	-9.33922475638894e-10\\
2.76499999999997	-8.99763015306507e-10\\
2.765	-8.99763015306281e-10\\
2.76500000000003	-8.99763015306092e-10\\
2.77	-8.67653406061231e-10\\
2.77000000000003	-8.67653406061054e-10\\
2.77499999999999	-8.37269809356801e-10\\
2.77999999999996	-8.08292616656451e-10\\
2.78	-8.08292616656215e-10\\
2.78399999999997	-7.86075636475899e-10\\
2.784	-7.86075636475744e-10\\
2.78799999999996	-7.64678401112527e-10\\
2.78999999999997	-7.54276650686385e-10\\
2.79	-7.54276650686239e-10\\
2.79399999999997	-7.33950032010138e-10\\
2.79799999999993	-7.14168481770912e-10\\
2.79999999999997	-7.0447237949933e-10\\
2.8	-7.04472379499193e-10\\
2.80799999999993	-6.66897653688194e-10\\
2.80999999999997	-6.57787978600437e-10\\
2.81	-6.57787978600308e-10\\
2.81299999999997	-6.44321850768275e-10\\
2.813	-6.44321850768148e-10\\
2.81599999999996	-6.31083241571144e-10\\
2.81899999999993	-6.18060471774781e-10\\
2.81999999999997	-6.13765521871101e-10\\
2.82	-6.13765521870979e-10\\
2.82599999999993	-5.89079510097158e-10\\
2.82999999999997	-5.73725002916078e-10\\
2.83	-5.73725002915972e-10\\
2.83499999999997	-5.55730073334501e-10\\
2.835	-5.55730073334402e-10\\
2.83999999999997	-5.39025472383856e-10\\
2.84	-5.39025472383765e-10\\
2.84199999999997	-5.32662449737619e-10\\
2.842	-5.3266244973753e-10\\
2.84399999999996	-5.26430400524361e-10\\
2.84599999999993	-5.20326881407357e-10\\
2.84999999999985	-5.08495911346259e-10\\
2.85	-5.08495911345832e-10\\
2.85000000000003	-5.0849591134575e-10\\
2.85799999999989	-4.86274307529679e-10\\
2.85999999999997	-4.81006203849977e-10\\
2.86	-4.81006203849903e-10\\
2.86799999999986	-4.60544455576838e-10\\
2.86999999999997	-4.55541887995795e-10\\
2.87	-4.55541887995724e-10\\
2.87099999999997	-4.53056176012854e-10\\
2.871	-4.53056176012784e-10\\
2.87199999999999	-4.50580520147854e-10\\
2.87299999999999	-4.4811467765042e-10\\
2.87499999999998	-4.43211466983318e-10\\
2.87899999999996	-4.33512233753437e-10\\
2.87999999999997	-4.31108368620924e-10\\
2.88	-4.31108368620856e-10\\
2.88799999999997	-4.12142390804831e-10\\
2.88999999999997	-4.07465977989865e-10\\
2.89	-4.07465977989799e-10\\
2.89799999999997	-3.89446722652551e-10\\
2.89999999999997	-3.85134730299006e-10\\
2.9	-3.85134730298945e-10\\
2.90499999999997	-3.74372072640642e-10\\
2.905	-3.7437207264058e-10\\
2.90999999999998	-3.63440760263061e-10\\
2.91000000000001	-3.63440760262998e-10\\
2.91499999999999	-3.52314003096606e-10\\
2.91999999999997	-3.4096453234862e-10\\
2.92	-3.40964532348542e-10\\
2.92899999999997	-3.19885072259964e-10\\
2.92899999999999	-3.19885072259896e-10\\
2.92999999999997	-3.17485577454242e-10\\
2.93	-3.17485577454173e-10\\
2.93099999999999	-3.15086616052151e-10\\
2.93199999999999	-3.1270080638304e-10\\
2.93399999999997	-3.07967708172994e-10\\
2.93799999999994	-2.98649127399725e-10\\
2.94	-2.94059991201173e-10\\
2.94000000000003	-2.94059991201108e-10\\
2.94799999999997	-2.76129359545762e-10\\
2.95	-2.71744392561114e-10\\
2.95000000000003	-2.71744392561052e-10\\
2.95799999999997	-2.54771485580687e-10\\
2.958	-2.54771485580628e-10\\
2.96	-2.50675910802522e-10\\
2.96000000000003	-2.50675910802464e-10\\
2.96200000000003	-2.46636151661437e-10\\
2.96400000000003	-2.42650624113409e-10\\
2.96800000000003	-2.34836034321906e-10\\
2.97	-2.31003908127785e-10\\
2.97000000000003	-2.31003908127731e-10\\
2.97499999999997	-2.2163080318579e-10\\
2.975	-2.21630803185738e-10\\
2.97999999999995	-2.12535422483586e-10\\
2.98	-2.12535422483485e-10\\
2.98499999999995	-2.03927993474582e-10\\
2.98699999999997	-2.00681970088336e-10\\
2.98699999999999	-2.00681970088291e-10\\
2.98999999999997	-1.96019939664457e-10\\
2.99	-1.96019939664414e-10\\
2.99299999999998	-1.91602675355719e-10\\
2.99599999999996	-1.87426280443344e-10\\
2.99999999999997	-1.8222610200507e-10\\
3	-1.82226102005035e-10\\
3.00599999999996	-1.75068142506096e-10\\
3.00999999999997	-1.70655384825394e-10\\
3.01	-1.70655384825364e-10\\
3.01599999999996	-1.64557767045681e-10\\
3.01599999999999	-1.64557767045647e-10\\
3.01600000000002	-1.6455776704562e-10\\
3.01999999999997	-1.60831069957029e-10\\
3.02	-1.60831069957003e-10\\
3.02399999999995	-1.57301711999836e-10\\
3.0279999999999	-1.5389767224842e-10\\
3.02999999999997	-1.52240970790692e-10\\
3.03	-1.52240970790669e-10\\
3.0379999999999	-1.45901265434991e-10\\
3.03999999999997	-1.4438500727574e-10\\
3.04	-1.44385007275719e-10\\
3.04499999999997	-1.40707985246066e-10\\
3.04499999999999	-1.40707985246045e-10\\
3.04999999999996	-1.3718612049681e-10\\
3.05	-1.37186120496782e-10\\
3.05499999999997	-1.33834606884127e-10\\
3.05999999999993	-1.30669055856601e-10\\
3.05999999999997	-1.3066905585658e-10\\
3.06	-1.30669055856558e-10\\
3.06999999999993	-1.24381496211569e-10\\
3.06999999999997	-1.24381496211547e-10\\
3.07	-1.24381496211524e-10\\
3.07399999999997	-1.21781565723582e-10\\
3.07399999999999	-1.21781565723563e-10\\
3.07799999999996	-1.19127045421511e-10\\
3.07999999999997	-1.17778016481e-10\\
3.08	-1.17778016480981e-10\\
3.08399999999997	-1.15033776384914e-10\\
3.08799999999993	-1.12224363213874e-10\\
3.08999999999997	-1.10793843899565e-10\\
3.09	-1.10793843899544e-10\\
3.09799999999993	-1.05097859121997e-10\\
3.09999999999997	-1.03690797307052e-10\\
3.1	-1.03690797307032e-10\\
3.10299999999997	-1.01590657914021e-10\\
3.10299999999999	-1.01590657914002e-10\\
3.10599999999996	-9.95014166311225e-11\\
3.10899999999993	-9.74212303151314e-11\\
3.10999999999997	-9.67295297139426e-11\\
3.11	-9.6729529713923e-11\\
3.11499999999997	-9.33106721281124e-11\\
3.115	-9.33106721280931e-11\\
3.11999999999998	-8.99617653087373e-11\\
3.12000000000001	-8.99617653087184e-11\\
3.12499999999998	-8.66746019490183e-11\\
3.12999999999996	-8.34411260651008e-11\\
3.13000000000001	-8.34411260650683e-11\\
3.13199999999999	-8.21643803651155e-11\\
3.13200000000002	-8.21643803650975e-11\\
3.13400000000001	-8.09012378939449e-11\\
3.136	-7.96512032548095e-11\\
3.13999999999997	-7.718850210907e-11\\
3.14	-7.71885021090527e-11\\
3.14000000000003	-7.71885021090353e-11\\
3.14799999999998	-7.23991512266233e-11\\
3.14999999999998	-7.12274129791639e-11\\
3.15	-7.12274129791473e-11\\
3.15799999999995	-6.66761896726878e-11\\
3.15999999999997	-6.55730292283979e-11\\
3.16	-6.55730292283823e-11\\
3.16099999999997	-6.50263452937866e-11\\
3.16099999999999	-6.50263452937711e-11\\
3.16199999999999	-6.4482854022313e-11\\
3.16299999999998	-6.39425021216175e-11\\
3.16499999999997	-6.28710048945904e-11\\
3.16899999999994	-6.07633726243679e-11\\
3.16999999999997	-6.02435299932207e-11\\
3.17	-6.0243529993206e-11\\
3.17799999999994	-5.61786235759564e-11\\
3.17999999999997	-5.51866383593599e-11\\
3.18	-5.51866383593459e-11\\
3.185	-5.28127408562041e-11\\
3.18500000000003	-5.28127408561911e-11\\
3.18999999999997	-5.06245674515098e-11\\
3.18999999999999	-5.06245674514979e-11\\
3.19499999999993	-4.86167554250972e-11\\
3.19999999999986	-4.67843841510351e-11\\
3.19999999999999	-4.678438415099e-11\\
3.20000000000002	-4.67843841509801e-11\\
3.20999999999989	-4.35511899275112e-11\\
3.21000000000002	-4.35511899274727e-11\\
3.21000000000005	-4.35511899274642e-11\\
3.21899999999997	-4.10679448270629e-11\\
3.21899999999999	-4.10679448270556e-11\\
3.22	-4.08160401971188e-11\\
3.22000000000003	-4.08160401971117e-11\\
3.22100000000004	-4.05678091728686e-11\\
3.22200000000005	-4.03221926468718e-11\\
3.22400000000006	-3.98387070516099e-11\\
3.22800000000009	-3.89020675097027e-11\\
3.23000000000003	-3.84485463242212e-11\\
3.23000000000006	-3.84485463242149e-11\\
3.23800000000012	-3.67290137691329e-11\\
3.23999999999997	-3.63219259741442e-11\\
3.24	-3.63219259741384e-11\\
3.24799999999997	-3.47799667099646e-11\\
3.24799999999999	-3.47799667099594e-11\\
3.24999999999997	-3.44153191943706e-11\\
3.25	-3.44153191943655e-11\\
3.25199999999998	-3.40596910884091e-11\\
3.25399999999996	-3.37139161727704e-11\\
3.25499999999998	-3.35446811482818e-11\\
3.255	-3.3544681148277e-11\\
3.25899999999996	-3.28917063315147e-11\\
3.25999999999997	-3.27343739492348e-11\\
3.26	-3.27343739492304e-11\\
3.26399999999996	-3.21282270655415e-11\\
3.26799999999992	-3.15584675779899e-11\\
3.26999999999997	-3.12869516539733e-11\\
3.27	-3.12869516539695e-11\\
3.27699999999999	-3.03700649485645e-11\\
3.27700000000002	-3.03700649485608e-11\\
3.27999999999997	-2.99875051523858e-11\\
3.28	-2.99875051523822e-11\\
3.28299999999995	-2.96106903056755e-11\\
3.2859999999999	-2.92392879890677e-11\\
3.28999999999997	-2.8751938760565e-11\\
3.29	-2.87519387605616e-11\\
3.2959999999999	-2.80215938385712e-11\\
3.29999999999997	-2.75275469030488e-11\\
3.3	-2.75275469030452e-11\\
3.3059999999999	-2.67738136550483e-11\\
3.30599999999995	-2.67738136550426e-11\\
3.30599999999999	-2.67738136550369e-11\\
3.30999999999997	-2.62617336256493e-11\\
3.31	-2.62617336256456e-11\\
3.31399999999998	-2.57412339406962e-11\\
3.31799999999996	-2.52116867395691e-11\\
3.31999999999997	-2.49432613577354e-11\\
3.32	-2.49432613577315e-11\\
3.32499999999998	-2.42607017722203e-11\\
3.325	-2.42607017722164e-11\\
3.32999999999998	-2.35603760196132e-11\\
3.33000000000001	-2.35603760196091e-11\\
3.33499999999998	-2.284056776732e-11\\
3.33500000000001	-2.28405677673159e-11\\
3.33999999999998	-2.20995129522146e-11\\
3.34000000000001	-2.20995129522104e-11\\
3.34499999999998	-2.13561550583863e-11\\
3.34999999999996	-2.06294319255422e-11\\
3.35000000000001	-2.06294319255347e-11\\
3.35999999999996	-1.92188023084681e-11\\
3.36	-1.92188023084619e-11\\
3.36399999999997	-1.86711101522306e-11\\
3.36399999999999	-1.86711101522267e-11\\
3.36799999999996	-1.81359109882585e-11\\
3.36999999999997	-1.78727332124254e-11\\
3.37	-1.78727332124217e-11\\
3.37399999999997	-1.7354705339603e-11\\
3.37799999999993	-1.68471057882617e-11\\
3.37999999999997	-1.65969672329849e-11\\
3.38	-1.65969672329814e-11\\
3.38799999999993	-1.5624490524753e-11\\
3.38999999999997	-1.53882856596439e-11\\
3.39	-1.53882856596406e-11\\
3.39299999999997	-1.50387851933971e-11\\
3.39299999999999	-1.50387851933938e-11\\
3.39499999999998	-1.48088562328646e-11\\
3.395	-1.48088562328613e-11\\
3.39699999999999	-1.45812790349981e-11\\
3.39899999999997	-1.43559643777905e-11\\
3.39999999999997	-1.42441278631695e-11\\
3.4	-1.42441278631663e-11\\
3.40399999999997	-1.3801998741322e-11\\
3.40799999999993	-1.33676984782305e-11\\
3.40999999999997	-1.31532708498908e-11\\
3.41	-1.31532708498877e-11\\
3.41799999999993	-1.23468274070594e-11\\
3.41999999999997	-1.21598289578157e-11\\
3.42	-1.21598289578131e-11\\
3.42199999999997	-1.19763027025746e-11\\
3.42199999999999	-1.1976302702572e-11\\
3.42399999999996	-1.17939539581156e-11\\
3.42599999999992	-1.16127112234657e-11\\
3.42999999999985	-1.12532599636107e-11\\
3.42999999999997	-1.12532599635996e-11\\
3.43	-1.12532599635971e-11\\
3.43799999999986	-1.05445287171985e-11\\
3.43999999999997	-1.03690592521473e-11\\
3.44	-1.03690592521449e-11\\
3.44799999999986	-9.67197034212084e-12\\
3.44999999999997	-9.49855372689938e-12\\
3.45	-9.49855372689692e-12\\
3.45099999999996	-9.41237932611694e-12\\
3.45099999999999	-9.41237932611451e-12\\
3.45199999999999	-9.32715275430196e-12\\
3.45299999999998	-9.24286565436004e-12\\
3.45499999999996	-9.07707691730687e-12\\
3.45899999999993	-8.75641569149809e-12\\
3.45999999999997	-8.67847863817919e-12\\
3.46	-8.67847863817699e-12\\
3.465	-8.30178202135442e-12\\
3.46500000000003	-8.30178202135234e-12\\
3.46999999999997	-7.94606389936375e-12\\
3.47	-7.94606389936179e-12\\
3.47499999999994	-7.61045248817232e-12\\
3.47999999999988	-7.29412529135568e-12\\
3.47999999999994	-7.29412529135216e-12\\
3.47999999999999	-7.29412529134867e-12\\
3.48999999999987	-6.74676202698775e-12\\
3.48999999999996	-6.7467620269833e-12\\
3.48999999999999	-6.74676202698193e-12\\
3.49999999999987	-6.28108238255635e-12\\
3.49999999999996	-6.28108238255219e-12\\
3.49999999999999	-6.28108238255091e-12\\
3.50899999999999	-5.88697449801625e-12\\
3.50900000000002	-5.88697449801504e-12\\
3.50999999999999	-5.84450180558707e-12\\
3.51000000000002	-5.84450180558586e-12\\
3.51100000000001	-5.80227708211787e-12\\
3.512	-5.76029618630208e-12\\
3.51399999999999	-5.67704944286389e-12\\
3.51799999999996	-5.51330239871449e-12\\
3.51999999999997	-5.43273789594903e-12\\
3.52	-5.43273789594789e-12\\
3.52799999999994	-5.12675371196895e-12\\
3.52999999999997	-5.05469343172973e-12\\
3.53	-5.05469343172872e-12\\
3.53499999999998	-4.88199347124512e-12\\
3.535	-4.88199347124417e-12\\
3.53799999999997	-4.78334756684714e-12\\
3.53799999999999	-4.78334756684623e-12\\
3.53999999999997	-4.71960194109289e-12\\
3.54	-4.71960194109199e-12\\
3.54199999999998	-4.65726093066475e-12\\
3.54399999999996	-4.59611931052526e-12\\
3.54799999999992	-4.47733881935329e-12\\
3.54999999999997	-4.41965337672261e-12\\
3.55	-4.4196533767218e-12\\
3.55799999999992	-4.19975427755904e-12\\
3.56	-4.14738240323028e-12\\
3.56000000000003	-4.14738240322954e-12\\
3.56699999999996	-3.96970437885065e-12\\
3.56699999999999	-3.96970437884994e-12\\
3.56999999999998	-3.89579765989381e-12\\
3.57000000000001	-3.89579765989312e-12\\
3.57299999999999	-3.82314081415196e-12\\
3.57599999999997	-3.75166974502726e-12\\
3.57999999999997	-3.65811069832464e-12\\
3.58	-3.65811069832398e-12\\
3.58599999999997	-3.52077931836244e-12\\
3.58999999999997	-3.43083591917435e-12\\
3.59	-3.43083591917371e-12\\
3.59599999999997	-3.29798439624951e-12\\
3.596	-3.29798439624889e-12\\
3.6	-3.21058985795257e-12\\
3.60000000000003	-3.21058985795195e-12\\
3.60400000000003	-3.12397371909621e-12\\
3.60499999999998	-3.1024253659982e-12\\
3.605	-3.10242536599759e-12\\
3.60900000000001	-3.0165913162159e-12\\
3.61	-2.99521212858456e-12\\
3.61000000000003	-2.99521212858395e-12\\
3.61400000000003	-2.91157677322764e-12\\
3.61800000000004	-2.83150478164585e-12\\
3.61999999999997	-2.79276558775134e-12\\
3.62	-2.7927655877508e-12\\
3.62499999999999	-2.69957845574993e-12\\
3.62500000000002	-2.69957845574941e-12\\
3.62999999999998	-2.61143993886586e-12\\
3.63000000000001	-2.61143993886537e-12\\
3.63499999999997	-2.52770267319305e-12\\
3.63999999999993	-2.44773008393152e-12\\
3.63999999999997	-2.44773008393098e-12\\
3.64	-2.44773008393043e-12\\
3.64999999999993	-2.29830371275204e-12\\
3.64999999999997	-2.29830371275133e-12\\
3.65	-2.29830371275093e-12\\
3.65399999999996	-2.24186112647559e-12\\
3.65399999999999	-2.24186112647519e-12\\
3.65799999999995	-2.18670203552576e-12\\
3.65999999999997	-2.15957668875149e-12\\
3.66	-2.15957668875111e-12\\
3.66399999999996	-2.10618121610029e-12\\
3.66799999999993	-2.05385643619316e-12\\
3.66999999999997	-2.02806984741587e-12\\
3.67	-2.0280698474155e-12\\
3.67499999999998	-1.96461846743619e-12\\
3.67500000000001	-1.96461846743583e-12\\
3.67999999999998	-1.90249323602781e-12\\
3.68000000000001	-1.90249323602746e-12\\
3.68299999999996	-1.86621791613807e-12\\
3.68299999999999	-1.86621791613773e-12\\
3.68599999999995	-1.83118660230184e-12\\
3.68899999999991	-1.79736839142776e-12\\
3.68999999999998	-1.7863600461193e-12\\
3.69000000000001	-1.78636004611898e-12\\
3.69599999999992	-1.72301122731574e-12\\
3.69999999999997	-1.68327609679402e-12\\
3.7	-1.68327609679375e-12\\
3.70599999999992	-1.62653687151272e-12\\
3.71	-1.59021020007978e-12\\
3.71000000000003	-1.59021020007953e-12\\
3.71199999999996	-1.57247517091489e-12\\
3.71199999999999	-1.57247517091464e-12\\
3.71399999999993	-1.5550163721993e-12\\
3.71599999999986	-1.53782695885147e-12\\
3.71999999999973	-1.50422943470876e-12\\
3.71999999999997	-1.50422943470671e-12\\
3.72	-1.50422943470648e-12\\
3.72799999999974	-1.43894586721611e-12\\
3.72999999999997	-1.42287841822669e-12\\
3.73	-1.42287841822646e-12\\
3.73799999999974	-1.35943335690188e-12\\
3.73999999999997	-1.34374718242011e-12\\
3.74	-1.34374718241989e-12\\
3.74099999999996	-1.33592450997468e-12\\
3.74099999999999	-1.33592450997446e-12\\
3.74199999999998	-1.32811228419084e-12\\
3.74299999999998	-1.32030973904822e-12\\
3.74499999999996	-1.3047306326375e-12\\
3.74500000000001	-1.30473063263715e-12\\
3.74899999999997	-1.27365499372016e-12\\
3.74999999999997	-1.26589884427959e-12\\
3.75	-1.26589884427937e-12\\
3.75399999999997	-1.23490248038562e-12\\
3.75799999999994	-1.20391483933303e-12\\
3.75999999999997	-1.18840910311248e-12\\
3.76	-1.18840910311226e-12\\
3.76799999999994	-1.12616478981275e-12\\
3.76999999999996	-1.11051786509164e-12\\
3.76999999999999	-1.11051786509141e-12\\
3.77799999999993	-1.04995351630452e-12\\
3.77999999999996	-1.03544793405671e-12\\
3.77999999999999	-1.03544793405651e-12\\
3.78799999999993	-9.79796963995601e-13\\
3.78999999999996	-9.66449788002072e-13\\
3.78999999999999	-9.66449788001884e-13\\
3.79799999999993	-9.15150941002443e-13\\
3.79899999999996	-9.08962871059808e-13\\
3.79899999999999	-9.08962871059633e-13\\
3.79999999999996	-9.02822592760261e-13\\
3.79999999999999	-9.02822592760087e-13\\
3.80099999999998	-8.9672950456746e-13\\
3.80199999999997	-8.90683009016079e-13\\
3.80399999999995	-8.78727429867746e-13\\
3.80799999999991	-8.55349606863182e-13\\
3.80999999999996	-8.43918197008127e-13\\
3.80999999999999	-8.43918197007966e-13\\
3.81499999999998	-8.15975383097839e-13\\
3.81500000000001	-8.15975383097682e-13\\
3.81999999999999	-7.88840799030302e-13\\
3.82000000000002	-7.8884079903015e-13\\
3.825	-7.62447944208307e-13\\
3.82799999999996	-7.46941282714382e-13\\
3.82799999999999	-7.46941282714236e-13\\
3.82999999999999	-7.36732136444989e-13\\
3.83000000000002	-7.36732136444845e-13\\
3.83200000000002	-7.26621217321988e-13\\
3.83400000000002	-7.16604561263466e-13\\
3.83800000000001	-6.96838365587595e-13\\
3.83999999999997	-6.87081076029482e-13\\
3.84	-6.87081076029344e-13\\
3.84799999999999	-6.50160464939608e-13\\
3.84999999999997	-6.41524234131149e-13\\
3.85	-6.41524234131028e-13\\
3.85699999999996	-6.11815381982768e-13\\
3.85699999999999	-6.11815381982647e-13\\
3.85999999999997	-5.99131020047213e-13\\
3.86	-5.99131020047093e-13\\
3.86299999999998	-5.86459287561072e-13\\
3.86599999999997	-5.73789005770894e-13\\
3.86999999999997	-5.56878233748017e-13\\
3.87	-5.56878233747896e-13\\
3.87599999999997	-5.3142161779762e-13\\
3.87999999999997	-5.14351418679477e-13\\
3.88	-5.14351418679355e-13\\
3.88500000000001	-4.93409839818484e-13\\
3.88500000000003	-4.93409839818367e-13\\
3.88599999999996	-4.89329381498576e-13\\
3.88599999999999	-4.89329381498461e-13\\
3.88699999999998	-4.85283934389823e-13\\
3.88799999999997	-4.8127310196119e-13\\
3.88999999999995	-4.7335371208484e-13\\
3.89	-4.73353712084647e-13\\
3.89399999999996	-4.57913234540109e-13\\
3.89799999999992	-4.42983732659483e-13\\
3.9	-4.35703249727157e-13\\
3.90000000000003	-4.35703249727055e-13\\
3.90799999999995	-4.07923415634119e-13\\
3.91	-4.01311593417117e-13\\
3.91000000000003	-4.01311593417024e-13\\
3.91499999999996	-3.85336460384293e-13\\
3.91499999999999	-3.85336460384205e-13\\
3.91999999999993	-3.70122269912516e-13\\
3.92	-3.7012226991229e-13\\
3.92000000000003	-3.70122269912205e-13\\
3.92499999999997	-3.5563173550312e-13\\
3.9299999999999	-3.41829344619301e-13\\
3.93000000000003	-3.41829344618948e-13\\
3.93000000000006	-3.41829344618872e-13\\
3.93999999999993	-3.17320749675053e-13\\
3.93999999999998	-3.17320749674947e-13\\
3.94	-3.17320749674885e-13\\
3.94399999999996	-3.08551938701094e-13\\
3.94399999999999	-3.08551938701032e-13\\
3.94799999999995	-2.99930586884422e-13\\
3.94999999999998	-2.95670973821325e-13\\
3.95	-2.95670973821265e-13\\
3.95399999999996	-2.87245522843779e-13\\
3.95499999999998	-2.85157491221809e-13\\
3.95500000000001	-2.8515749122175e-13\\
3.95899999999997	-2.76872549279073e-13\\
3.96	-2.7481709491124e-13\\
3.96000000000003	-2.74817094911182e-13\\
3.96399999999999	-2.66652357761793e-13\\
3.96799999999995	-2.58569365010945e-13\\
3.97	-2.54554556514455e-13\\
3.97000000000003	-2.54554556514398e-13\\
3.97299999999996	-2.48636804435287e-13\\
3.97299999999999	-2.48636804435232e-13\\
3.97599999999992	-2.42900198771178e-13\\
3.97899999999986	-2.37339678609551e-13\\
3.97999999999997	-2.35524449763293e-13\\
3.98	-2.35524449763242e-13\\
3.98599999999987	-2.2502263670499e-13\\
3.98999999999997	-2.18379955024267e-13\\
3.99	-2.1837995502422e-13\\
3.99599999999987	-2.08927674346956e-13\\
3.99999999999997	-2.0295290216578e-13\\
4	-2.02952902165738e-13\\
4.00199999999993	-2.00085179746725e-13\\
4.00199999999999	-2.00085179746645e-13\\
4.00399999999992	-1.9732931051357e-13\\
4.00599999999986	-1.94684213965039e-13\\
4.00999999999972	-1.89722233890722e-13\\
4.00999999999995	-1.89722233890456e-13\\
4.01	-1.89722233890388e-13\\
4.01799999999973	-1.80373970631728e-13\\
4.01999999999995	-1.78077473266697e-13\\
4.02	-1.78077473266632e-13\\
4.02500000000001	-1.72397351036583e-13\\
4.02500000000006	-1.72397351036519e-13\\
4.02999999999995	-1.6679343343451e-13\\
4.03	-1.66793433434447e-13\\
4.03099999999999	-1.65680582965713e-13\\
4.03100000000005	-1.6568058296565e-13\\
4.03200000000004	-1.64570121774813e-13\\
4.03300000000003	-1.63461941015748e-13\\
4.035	-1.61251986552836e-13\\
4.03899999999996	-1.56854677492997e-13\\
4.03999999999994	-1.55759429715498e-13\\
4.04	-1.55759429715436e-13\\
4.04799999999991	-1.47297263977629e-13\\
4.04999999999994	-1.45270158350113e-13\\
4.05	-1.45270158350056e-13\\
4.05799999999991	-1.37491629793765e-13\\
4.05999999999994	-1.35625659186116e-13\\
4.06	-1.35625659186064e-13\\
};
\end{axis}
\end{tikzpicture}%
}
      \caption{Evolution of the angular displacement of pendulum $P_1$ for
        $C_1 = 10$ ms. \texttt{Blue}: RM scheduling, \texttt{Red}: EDF scheduling}
      \label{fig:02.6.10.1}
    \end{figure}
  \end{minipage}
  \hfill
  \begin{minipage}{0.45\linewidth}
    \begin{figure}[H]\centering
      \scalebox{0.7}{% This file was created by matlab2tikz.
%
%The latest updates can be retrieved from
%  http://www.mathworks.com/matlabcentral/fileexchange/22022-matlab2tikz-matlab2tikz
%where you can also make suggestions and rate matlab2tikz.
%
\definecolor{mycolor1}{rgb}{0.00000,0.44700,0.74100}%
%
\begin{tikzpicture}

\begin{axis}[%
width=4.133in,
height=3.26in,
at={(0.693in,0.44in)},
scale only axis,
xmin=0,
xmax=1.2,
xmajorgrids,
ymin=-0.05,
ymax=0.1,
ymajorgrids,
axis background/.style={fill=white}
]
\pgfplotsset{max space between ticks=50}
\addplot [color=mycolor1,solid,forget plot]
  table[row sep=crcr]{%
0	0\\
3.15544362088405e-30	0\\
0.000656101980281985	0\\
0.00393661188169191	0\\
0.00999999999999994	0\\
0.01	0\\
0.0199999999999999	0\\
0.02	0\\
0.0289999999999998	0\\
0.029	0\\
0.03	0\\
0.0300000000000002	0\\
0.0349999999999996	0.00068493227514009\\
0.035	0.000684932275140243\\
0.0399999999999993	0.00274140816002566\\
0.04	0.00274140816002602\\
0.0449999999999993	0.00534437079003229\\
0.0499999999999987	0.00767010281067138\\
0.05	0.00767010281067197\\
0.0500000000000004	0.00767010281067217\\
0.0579999999999996	0.0129527826502965\\
0.058	0.0129527826502969\\
0.0599999999999996	0.0148320020673268\\
0.06	0.0148320020673273\\
0.0619999999999995	0.0167791198495251\\
0.0639999999999991	0.0186380361710611\\
0.0679999999999982	0.0220941452997054\\
0.0699999999999991	0.0236926931885511\\
0.07	0.0236926931885518\\
0.0779999999999982	0.0305652340976935\\
0.0799999999999991	0.0378100613777975\\
0.08	0.0378100613777983\\
0.087	0.0561576567758901\\
0.0870000000000009	0.0561576567758906\\
0.09	0.0602065942806531\\
0.0900000000000009	0.0552046674150938\\
0.0929999999999999	0.0566552387530397\\
0.095999999999999	0.064774944687725\\
0.0999999999999991	0.0679076162658781\\
0.1	0.0679076162658782\\
0.104999999999999	0.0724888812775372\\
0.105	0.0665016384324846\\
0.109999999999999	0.0685522291622384\\
0.11	0.0685522291622384\\
0.114999999999999	0.068227955660876\\
0.115999999999999	0.0695470391532237\\
0.116	0.062866816362474\\
0.119999999999999	0.0674719213384736\\
0.12	0.0619854650975131\\
0.123999999999999	0.0645613856423845\\
0.127999999999998	0.0670344645157313\\
0.129999999999998	0.0644851079290965\\
0.13	0.0644851079290965\\
0.137999999999998	0.0625083248462925\\
0.139999999999998	0.0600631182051521\\
0.14	0.0593558650398579\\
0.144999999999998	0.0569485689455489\\
0.145	0.0556624892926128\\
0.149999999999998	0.0515786474607335\\
0.15	0.0513514778565428\\
0.154999999999998	0.0460274468173092\\
0.159999999999996	0.0401730165815378\\
0.16	0.0403208624139119\\
0.169999999999996	0.0255763243230786\\
0.17	0.0264830527967936\\
0.173999999999998	0.0202045435983293\\
0.174	0.020204543598327\\
0.174999999999998	0.0199711364251127\\
0.175	0.0216506350984745\\
0.176	0.020938048304053\\
0.177	0.0192842467300469\\
0.179000000000001	0.0203984247913287\\
0.179999999999998	0.0200148044353173\\
0.18	0.0200148044353155\\
0.184000000000001	0.0159981672503822\\
0.188000000000002	0.00927287744407729\\
0.189999999999998	0.00656100345014041\\
0.19	0.0065610034501386\\
0.198000000000002	-0.00649091641637699\\
0.199999999999998	-0.00929419909473431\\
0.2	-0.00791580907120871\\
0.202999999999998	-0.0123378919416167\\
0.203	-0.0123378919416184\\
0.205999999999998	-0.0143717285158986\\
0.208999999999996	-0.0160781719286876\\
0.209999999999998	-0.0160770659955318\\
0.21	-0.0160770659955331\\
0.215999999999996	-0.0216101365368742\\
0.219999999999998	-0.0269160999187173\\
0.22	-0.0240197956144751\\
0.225999999999996	-0.0300413244089089\\
0.229999999999998	-0.0346776157007166\\
0.23	-0.0325629586956899\\
0.231999999999998	-0.0347851220167335\\
0.232	-0.0326991803247014\\
0.233999999999998	-0.0327994587963916\\
0.235999999999997	-0.034216134186806\\
0.239999999999993	-0.0382282305883098\\
0.239999999999996	-0.0342929429817149\\
0.24	-0.0318214944391811\\
0.244999999999998	-0.036329531241162\\
0.245	-0.0328407816521247\\
0.249999999999998	-0.034456642379395\\
0.25	-0.0344566423793953\\
0.254999999999999	-0.0365124563043245\\
0.259999999999997	-0.0389268167866508\\
0.26	-0.0379126342777017\\
0.260999999999996	-0.0373357749534266\\
0.261	-0.0354742930128887\\
0.262	-0.0358565977228592\\
0.263	-0.0362176030102689\\
0.265	-0.0347318888334962\\
0.269	-0.0357942166292505\\
0.269999999999997	-0.0340613607299935\\
0.27	-0.0340613607299933\\
0.278	-0.0350062636930079\\
0.279999999999996	-0.0343080061379221\\
0.28	-0.0343080061379209\\
0.288	-0.0331927574246004\\
0.289999999999996	-0.0323097149252547\\
0.29	-0.0310632438565711\\
0.298	-0.0297461826323198\\
0.299999999999996	-0.0293038224781857\\
0.3	-0.0290099384639334\\
0.308	-0.0267829138885945\\
0.309999999999996	-0.0258238212340273\\
0.31	-0.0255439788391393\\
0.314999999999997	-0.0232433102174589\\
0.315	-0.0222486828773331\\
0.319	-0.0205726015838666\\
0.319000000000004	-0.0205726015838645\\
0.319999999999996	-0.0185507712997018\\
0.32	-0.018189162861811\\
0.321	-0.0178207282386844\\
0.321999999999999	-0.0162085313035754\\
0.323999999999998	-0.0152008630395249\\
0.327999999999996	-0.0136774448766829\\
0.329999999999996	-0.0120062915256253\\
0.33	-0.0118146957268515\\
0.337999999999996	-0.00882432673368501\\
0.339999999999996	-0.00747740110108964\\
0.34	-0.00734867835779984\\
0.347999999999996	-0.00467540343188001\\
0.348	-0.00438922447439531\\
0.349999999999996	-0.00375819276746477\\
0.35	-0.0035579039769294\\
0.351999999999996	-0.00294128366245333\\
0.353999999999993	-0.00229269833652413\\
0.357999999999985	-0.00112936643124954\\
0.359999999999996	-0.000523460255354133\\
0.36	-0.000480452786193243\\
0.367999999999985	0.00153433801847599\\
0.369999999999996	0.00207346797542509\\
0.37	0.00213631055383558\\
0.377	0.00332839343427026\\
0.377000000000004	0.00355150392082686\\
0.379999999999997	0.00400988077268941\\
0.38	0.00400988077268998\\
0.382999999999993	0.00442143149738016\\
0.384999999999997	0.00465342929236755\\
0.385	0.00467813841562448\\
0.387999999999993	0.00501222914037083\\
0.389999999999997	0.0052405308250844\\
0.39	0.00532895683021906\\
0.392999999999993	0.00556307625604454\\
0.395999999999986	0.00575203839726219\\
0.399999999999997	0.00597545198229148\\
0.4	0.00597545198229167\\
0.405999999999986	0.00614085419804524\\
0.406	0.00618187328361762\\
0.406000000000004	0.00626501326362735\\
0.41	0.00629217930624101\\
0.410000000000004	0.00629217930624104\\
0.414	0.00637156184908278\\
0.417999999999996	0.00624376170159563\\
0.419999999999997	0.00612809300787233\\
0.42	0.00629130677415297\\
0.427999999999993	0.00570440321791966\\
0.429999999999997	0.00560533477151812\\
0.43	0.00560533477151789\\
0.435	0.00520137626112811\\
0.435000000000004	0.00525434919840185\\
0.439999999999997	0.00470448369721973\\
0.44	0.00478633489860541\\
0.444999999999993	0.00425120311320941\\
0.449999999999986	0.00364564502855115\\
0.449999999999993	0.0037320277061527\\
0.45	0.00382066910487137\\
0.454999999999997	0.00322945632125618\\
0.455	0.00322945632125584\\
0.459999999999997	0.00268271633322046\\
0.46	0.00268271633322011\\
0.463999999999997	0.00226558721886598\\
0.464	0.00236233414345739\\
0.467999999999997	0.00186400189191562\\
0.469999999999997	0.00161947186652028\\
0.47	0.00175254729189943\\
0.473999999999997	0.00127228006543229\\
0.477999999999993	0.000939379135697669\\
0.479999999999997	0.000846739433273147\\
0.48	0.000916468410943534\\
0.487999999999993	1.79632333124815e-05\\
0.489999999999997	8.32938845301036e-05\\
0.49	0.000155889307965259\\
0.492999999999997	-0.000160726937952286\\
0.493	2.08239772474979e-05\\
0.495999999999997	-0.000275224922789865\\
0.498999999999993	-0.000370616473472868\\
0.499999999999997	-0.00045806079365635\\
0.5	-0.00027846734061223\\
0.505999999999993	-0.000577511733839511\\
0.509999999999993	-0.000852814361862986\\
0.51	-0.000852814361863318\\
0.515999999999993	-0.00102943777612123\\
0.519999999999993	-0.00122829453605552\\
0.52	-0.0010556694736217\\
0.521999999999993	-0.00114416046899017\\
0.522	-0.00100889331717545\\
0.523999999999993	-0.00109013992859205\\
0.524999999999993	-0.000995077547462749\\
0.525	-0.000929380703579595\\
0.526999999999993	-0.000999819582179469\\
0.528999999999986	-0.000934244334608422\\
0.529999999999993	-0.000838314101836334\\
0.53	-0.000777295912747004\\
0.533999999999986	-0.000880875022107249\\
0.537999999999972	-0.000732089856654927\\
0.539999999999993	-0.000712182843568338\\
0.54	-0.00071218284356836\\
0.547999999999972	-0.000733597112547848\\
0.549999999999993	-0.000749173295959557\\
0.55	-0.000668450435055166\\
0.550999999999993	-0.000596293255474881\\
0.551	-0.000570683169931345\\
0.551999999999997	-0.000575839462684142\\
0.552999999999993	-0.000431483113560217\\
0.554999999999986	-0.000342340154752342\\
0.558999999999972	-0.00033984159678883\\
0.559999999999993	-0.000268478210220857\\
0.56	-0.000268478210220671\\
0.567999999999972	-0.000155656221094973\\
0.57	-7.50143742902095e-05\\
0.570000000000007	-5.42804668669733e-05\\
0.577999999999979	1.65249840498156e-05\\
0.579999999999993	7.45725095979317e-05\\
0.58	7.45725095981364e-05\\
0.587999999999972	0.000182102691192695\\
0.589999999999993	0.000218268737085636\\
0.59	0.000218268737085833\\
0.594999999999993	0.000297714920884168\\
0.595	0.000335283280652613\\
0.599999999999993	0.000394081136896393\\
0.6	0.00039408113689658\\
0.604999999999993	0.000501116612397148\\
0.608999999999993	0.000591006094969457\\
0.609	0.000623922541701254\\
0.609999999999993	0.000628395836698279\\
0.61	0.000752164675345437\\
0.610999999999997	0.000785297022086692\\
0.611999999999993	0.000789031565058532\\
0.613999999999986	0.000809637478698568\\
0.617999999999972	0.000817947043507307\\
0.619999999999993	0.000833970934555851\\
0.62	0.000847877504820039\\
0.627999999999972	0.000867682552084072\\
0.629999999999993	0.000914615930847189\\
0.63	0.000927387353535091\\
0.637999999999972	0.000888847632327721\\
0.637999999999993	0.000985467810182439\\
0.638	0.00100812412117069\\
0.639999999999993	0.00099458326916569\\
0.64	0.00100570235270808\\
0.641999999999993	0.000990842122676555\\
0.643999999999986	0.000985903190759035\\
0.647999999999971	0.000961664361284447\\
0.649999999999993	0.000963845817188636\\
0.65	0.00100514346266071\\
0.657999999999971	0.000927332798105774\\
0.659999999999993	0.000902559163633702\\
0.66	0.000978699009662511\\
0.664999999999993	0.000929668038038491\\
0.665	0.000929668038038454\\
0.666999999999993	0.000944017565119603\\
0.667	0.000944017565119558\\
0.668999999999993	0.000955018475013841\\
0.669999999999993	0.000939568889481638\\
0.67	0.000978508973255845\\
0.671999999999993	0.000976955846713778\\
0.673999999999986	0.000945499640682845\\
0.677999999999972	0.000889254894506935\\
0.679999999999993	0.000857200890963601\\
0.68	0.000864371239402974\\
0.687999999999972	0.000740891119901715\\
0.689999999999993	0.000721608912730021\\
0.69	0.000748906083749279\\
0.695999999999993	0.00066337747396225\\
0.696	0.000663377473962181\\
0.699999999999993	0.000649940709617091\\
0.7	0.000662108143300361\\
0.703999999999993	0.000598959076869413\\
0.707999999999986	0.0005836057844274\\
0.709999999999993	0.000552917690096625\\
0.71	0.000564154866423956\\
0.717999999999986	0.000443539264553422\\
0.719999999999993	0.000424926623907624\\
0.72	0.000435826140015579\\
0.724999999999993	0.00038633818380508\\
0.725	0.000396763105025999\\
0.729999999999993	0.00033220568628181\\
0.730000000000001	0.000372419960817038\\
0.734999999999994	0.000312264187110281\\
0.735000000000001	0.00031226418711026\\
0.739999999999994	0.000264376177420039\\
0.740000000000001	0.000264376177419993\\
0.744999999999994	0.000218780876310411\\
0.749999999999987	0.000175346847310305\\
0.750000000000001	0.000180017711107205\\
0.753999999999993	0.000139283400010104\\
0.754	0.000148532331294561\\
0.757999999999993	0.000109117441939926\\
0.759999999999993	9.90136018425853e-05\\
0.76	0.00010353356488159\\
0.763999999999993	6.68237529298463e-05\\
0.767999999999986	4.19128080856222e-05\\
0.77	3.48569608572523e-05\\
0.770000000000007	5.2131303819936e-05\\
0.777999999999993	3.96589834319658e-06\\
0.779999999999993	-8.49922601498638e-06\\
0.78	2.42481354694909e-05\\
0.782999999999993	1.45948280520552e-05\\
0.783	1.45948280520429e-05\\
0.785999999999993	2.13616431570991e-05\\
0.788999999999986	6.20385021039073e-06\\
0.79	1.62953069496284e-05\\
0.790000000000007	1.62953069496208e-05\\
0.795999999999993	4.799911545652e-06\\
0.8	4.17093886594561e-06\\
0.800000000000007	1.10795261113677e-05\\
0.804999999999993	-5.07777149058591e-06\\
0.805	1.50487039581203e-05\\
0.809999999999987	7.81923904430437e-07\\
0.809999999999997	1.37027018760468e-05\\
0.810000000000007	1.37027018760416e-05\\
0.811999999999993	2.10643606374329e-05\\
0.812	2.41450310236953e-05\\
0.813999999999987	1.92544829798149e-05\\
0.815999999999973	2.67642326647766e-05\\
0.819999999999945	2.14149955001063e-05\\
0.819999999999987	2.14149955000461e-05\\
0.82	3.30808643413297e-05\\
0.827999999999944	2.27341714062649e-05\\
0.829999999999993	2.01425983156538e-05\\
0.830000000000001	3.13480930336102e-05\\
0.837999999999945	2.58246786611323e-05\\
0.839999999999993	2.42329413033819e-05\\
0.84	3.50055236560519e-05\\
0.840999999999993	4.47258428275821e-05\\
0.841000000000001	4.98309095189456e-05\\
0.841999999999997	4.91530747159809e-05\\
0.842999999999993	5.84839867621644e-05\\
0.844999999999986	5.736352106364e-05\\
0.848999999999972	6.53130330613527e-05\\
0.849999999999993	6.97038669218116e-05\\
0.85	6.97038669218262e-05\\
0.857999999999972	7.67291650574543e-05\\
0.859999999999993	8.50864122271879e-05\\
0.86	8.94084124125961e-05\\
0.867999999999972	8.75257450306235e-05\\
0.869999999999993	9.55073146009251e-05\\
0.87	9.55073146009382e-05\\
0.874999999999994	0.00010227026809383\\
0.875000000000001	0.00010622854335604\\
0.879999999999994	0.000104306522503362\\
0.880000000000001	0.000110115511959694\\
0.884999999999994	0.000107632833432423\\
0.889999999999987	0.000110242839183159\\
0.890000000000001	0.000115757694907402\\
0.890000000000008	0.000117564697816638\\
0.899	0.000110437308843212\\
0.899000000000008	0.000120942539447799\\
0.9	0.000126702487368451\\
0.900000000000007	0.000126702487368456\\
0.901000000000004	0.000133775120834014\\
0.902	0.000132786351213454\\
0.903999999999993	0.000138405596811483\\
0.907999999999979	0.000133933633264088\\
0.909999999999993	0.000131523121424754\\
0.910000000000001	0.000134500951480342\\
0.917999999999972	0.000123672779037363\\
0.919999999999993	0.000123587337188061\\
0.920000000000001	0.000126457531436117\\
0.927999999999972	0.00011932241071156\\
0.927999999999994	0.000119322410711582\\
0.928000000000001	0.000119322410711589\\
0.929999999999994	0.000126579287282659\\
0.930000000000001	0.000129107160905966\\
0.931999999999994	0.00012572986920273\\
0.933999999999987	0.000124783928333218\\
0.937999999999972	0.000117768594302316\\
0.939999999999993	0.000116616153854852\\
0.940000000000001	0.000119009251387978\\
0.944999999999994	0.000114441686479123\\
0.945000000000001	0.000114441686479126\\
0.949999999999994	0.000104840496234291\\
0.950000000000001	0.000105978671314642\\
0.954999999999994	9.59777039634941e-05\\
0.956999999999994	9.29878553181889e-05\\
0.957000000000001	9.4105654496068e-05\\
0.959999999999993	9.00038183265322e-05\\
0.960000000000001	9.42982570466715e-05\\
0.962999999999993	8.90186754843785e-05\\
0.965999999999986	8.28015629052368e-05\\
0.969999999999993	8.2744144019097e-05\\
0.970000000000001	8.46719120975431e-05\\
0.975999999999986	7.28140984268036e-05\\
0.979999999999993	7.248784331913e-05\\
0.980000000000001	7.42527241660859e-05\\
0.985999999999986	6.29673620195949e-05\\
0.985999999999993	6.72455435643123e-05\\
0.986000000000001	6.72455435643047e-05\\
0.989999999999993	6.39792614247078e-05\\
0.990000000000001	6.39792614247006e-05\\
0.993999999999993	6.08632303162e-05\\
0.997999999999986	5.8103332464651e-05\\
0.999999999999993	5.50382198371353e-05\\
1	6.14019897787173e-05\\
1.00799999999999	5.03948860832903e-05\\
1.00999999999999	4.86282132493368e-05\\
1.01	4.86282132493246e-05\\
1.01499999999999	4.36449897862947e-05\\
1.015	4.43107415397417e-05\\
1.01999999999999	4.05679761087818e-05\\
1.02	4.3132250316465e-05\\
1.02499999999999	3.99100032649682e-05\\
1.02999999999997	3.60110014959743e-05\\
1.03	4.08061875101968e-05\\
1.03999999999997	3.47739802870618e-05\\
1.04	3.47739802870491e-05\\
1.04399999999999	3.30422811271001e-05\\
1.044	3.30422811270963e-05\\
1.04799999999999	3.14944476269063e-05\\
1.04999999999999	3.10691672789729e-05\\
1.05	3.21804892693835e-05\\
1.05399999999999	3.24537877056063e-05\\
1.05799999999997	3.06252812696805e-05\\
1.05999999999999	2.97351826574971e-05\\
1.06	3.07933286199607e-05\\
1.06799999999997	2.73858932241038e-05\\
1.06999999999999	2.76113524843936e-05\\
1.07	2.86346000349441e-05\\
1.07299999999999	2.9433372028274e-05\\
1.073	3.04064978743431e-05\\
1.07599999999999	2.92400799533345e-05\\
1.07899999999997	3.0466949364358e-05\\
1.07999999999999	3.00942870783599e-05\\
1.08	3.23643808052446e-05\\
1.08499999999999	3.05471624325904e-05\\
1.085	3.2725804220305e-05\\
1.08999999999999	3.18283559220893e-05\\
1.09	3.18283559220873e-05\\
1.09499999999999	3.13308417665606e-05\\
1.09999999999997	2.95223834178167e-05\\
1.09999999999999	3.07352563042091e-05\\
1.1	3.19188961035148e-05\\
1.10199999999999	3.2710064419843e-05\\
1.102	3.27100644198404e-05\\
1.10399999999999	3.41636386396578e-05\\
1.10599999999997	3.37537226582566e-05\\
1.10999999999994	3.21814527542195e-05\\
1.10999999999999	3.32360090475653e-05\\
1.11	3.32360090475621e-05\\
1.11799999999994	3.09610212714008e-05\\
1.11999999999999	3.11039260509663e-05\\
1.12	3.11039260509625e-05\\
1.12799999999994	2.76393261174578e-05\\
1.12999999999999	2.80757849382245e-05\\
1.13	2.80757849382206e-05\\
1.13099999999999	2.89023663323173e-05\\
1.131	3.01194651400671e-05\\
1.132	3.02802864487993e-05\\
1.13299999999999	2.98475721768329e-05\\
1.13499999999999	3.12596365313524e-05\\
1.13899999999997	3.00709754464977e-05\\
1.13999999999999	2.96361649684803e-05\\
1.14	3.09633174613825e-05\\
1.14799999999997	2.74643367895766e-05\\
1.14999999999999	2.78541160297534e-05\\
1.15	2.78541160297506e-05\\
1.15499999999999	2.69190522317273e-05\\
1.155	2.80893408917313e-05\\
1.15999999999999	2.60407956533718e-05\\
1.16	2.71629268756369e-05\\
1.16499999999999	2.51730245171175e-05\\
1.16999999999997	2.42694424274991e-05\\
1.16999999999999	2.42694424274964e-05\\
1.17	2.50972128035218e-05\\
1.17999999999997	2.11548578369584e-05\\
1.17999999999999	2.19554215092433e-05\\
1.18	2.23458948113081e-05\\
1.18899999999999	1.87881167650489e-05\\
1.189	1.95493158145514e-05\\
1.18999999999999	1.98874145230053e-05\\
1.19	2.02456812377267e-05\\
1.191	1.98503400142273e-05\\
1.19199999999999	2.08330054747315e-05\\
1.19399999999999	2.03921161780587e-05\\
1.19799999999997	1.88994061631818e-05\\
1.19999999999999	1.86599189516751e-05\\
1.2	1.86599189516723e-05\\
1.20799999999997	1.63875613787851e-05\\
1.20999999999999	1.61965628744254e-05\\
1.21	1.6346953877552e-05\\
1.21799999999997	1.38987678394172e-05\\
1.218	1.463179133506e-05\\
1.21999999999999	1.40575119088483e-05\\
1.22	1.47602890060815e-05\\
1.22199999999999	1.4200657370189e-05\\
1.22399999999997	1.40630077593109e-05\\
1.22499999999999	1.37957497309396e-05\\
1.225	1.41927680916405e-05\\
1.22899999999997	1.35455861818851e-05\\
1.22999999999999	1.34224502453959e-05\\
1.23	1.34224502453941e-05\\
1.23399999999997	1.2703553491914e-05\\
1.23799999999994	1.17804861819598e-05\\
1.23999999999999	1.15817988709657e-05\\
1.24	1.18236903338996e-05\\
1.24699999999999	1.08243563928743e-05\\
1.247	1.08243563928765e-05\\
1.24999999999999	1.0233308896685e-05\\
1.25	1.11295448074681e-05\\
1.25299999999999	1.07759206886495e-05\\
1.25599999999997	1.02301560234963e-05\\
1.25999999999999	1.03590095133783e-05\\
1.26	1.05561507229295e-05\\
1.26599999999997	9.60704278900732e-06\\
1.26999999999999	9.14035092836506e-06\\
1.27	9.14035092836452e-06\\
1.27599999999997	8.49166987233589e-06\\
1.276	8.58731183048856e-06\\
1.27999999999999	8.34545448762399e-06\\
1.28	8.71342251173086e-06\\
1.28399999999999	8.41299390681406e-06\\
1.28799999999997	8.05023424922581e-06\\
1.28999999999999	8.5725245337594e-06\\
1.29	8.73875110058007e-06\\
1.295	8.34006707323744e-06\\
1.29500000000001	8.42213522361061e-06\\
1.3	8.06445596797258e-06\\
1.30000000000001	8.14581169880024e-06\\
1.305	7.90891191120986e-06\\
1.30500000000001	8.0681327128725e-06\\
1.31	7.78993158059562e-06\\
1.31000000000001	8.10026315312839e-06\\
1.315	7.92837025930315e-06\\
1.31999999999998	7.70200506201928e-06\\
1.32	7.99939883165749e-06\\
1.32999999999997	7.82307943399283e-06\\
1.33	7.96332183377487e-06\\
1.33399999999999	7.75910609415351e-06\\
1.334	8.2966368561946e-06\\
1.33799999999999	8.21000534909964e-06\\
1.33999999999999	8.09805056266833e-06\\
1.34	8.59177831030684e-06\\
1.34399999999999	8.41852646025967e-06\\
1.34799999999997	8.17382528652482e-06\\
1.34999999999999	8.10580155781144e-06\\
1.35	8.10580155781136e-06\\
1.35799999999997	7.63350343063709e-06\\
1.35999999999999	7.55336231749572e-06\\
1.36	7.66776993475164e-06\\
1.36299999999999	7.67936642523972e-06\\
1.363	7.78804901679255e-06\\
1.36499999999999	7.64418437604677e-06\\
1.365	8.06110873486694e-06\\
1.36699999999999	8.01482063831782e-06\\
1.36899999999997	7.86608391474036e-06\\
1.36999999999999	8.17406774408719e-06\\
1.37	8.1740677440874e-06\\
1.37399999999997	7.86926731303172e-06\\
1.37799999999995	7.65137473349051e-06\\
1.37999999999999	7.49454286300504e-06\\
1.38	7.58480558896507e-06\\
1.38799999999995	7.03155931608356e-06\\
1.38999999999999	6.91090130368109e-06\\
1.39	6.91090130368054e-06\\
1.39199999999999	6.91734251649787e-06\\
1.392	6.95940164078866e-06\\
1.39399999999998	6.79415563394122e-06\\
1.39599999999997	6.7935752321119e-06\\
1.39999999999994	6.6218521025575e-06\\
1.39999999999999	6.62185210255471e-06\\
1.4	6.62185210255457e-06\\
1.40799999999994	6.03463651321561e-06\\
1.41	5.86742293208918e-06\\
1.41000000000001	5.94367837282989e-06\\
1.41799999999995	5.34375878934075e-06\\
1.41999999999999	5.28335640909348e-06\\
1.42	5.28335640909278e-06\\
1.42099999999999	5.30578236728429e-06\\
1.421	5.30578236728359e-06\\
1.422	5.32641086733776e-06\\
1.42299999999999	5.34523978399715e-06\\
1.42499999999998	5.3134031006544e-06\\
1.42899999999997	4.99556087106745e-06\\
1.42999999999999	5.10824285200752e-06\\
1.43	5.22980538241782e-06\\
1.43499999999999	4.85488132232257e-06\\
1.435	5.02967862016918e-06\\
1.43999999999998	4.67414564498382e-06\\
1.44	4.78578239042669e-06\\
1.44499999999999	4.44969785429753e-06\\
1.44999999999997	4.24109285255962e-06\\
1.44999999999998	4.34547750870498e-06\\
1.45	4.39639318087126e-06\\
1.45999999999997	3.81893452505295e-06\\
1.45999999999998	3.94248836503843e-06\\
1.46	3.94248836503795e-06\\
1.46999999999997	3.45288976314825e-06\\
1.46999999999998	3.45288976314785e-06\\
1.47	3.47676250188582e-06\\
1.47899999999998	3.08359877018233e-06\\
1.479	3.13033849920932e-06\\
1.47999999999999	3.08652791666706e-06\\
1.48	3.17767212428671e-06\\
1.481	3.15660992797468e-06\\
1.48199999999999	3.11361770777867e-06\\
1.48399999999999	3.11618634203146e-06\\
1.48799999999997	2.97229639231984e-06\\
1.48999999999999	2.89178597104704e-06\\
1.49	2.97560841127912e-06\\
1.49799999999997	2.74807815817698e-06\\
1.49999999999999	2.71364644191394e-06\\
1.5	2.7136464419137e-06\\
1.50499999999999	2.68561931826711e-06\\
1.505	2.72191761958057e-06\\
1.50799999999998	2.61779372508793e-06\\
1.508	2.68855878395575e-06\\
1.50999999999999	2.62055193275675e-06\\
1.51	2.6889158243903e-06\\
1.51199999999999	2.6558431900743e-06\\
1.51399999999997	2.59157139586643e-06\\
1.51799999999994	2.53423788046763e-06\\
1.51999999999999	2.53898839839982e-06\\
1.52	2.56962800781991e-06\\
1.52799999999994	2.34347709832529e-06\\
1.52999999999999	2.3472295301693e-06\\
1.53	2.34722953016911e-06\\
1.53699999999998	2.2102726380053e-06\\
1.537	2.23839279222981e-06\\
1.53999999999999	2.15540673666286e-06\\
1.54	2.19670301841097e-06\\
1.54299999999999	2.11387523754712e-06\\
1.54599999999997	2.05808995770665e-06\\
1.54999999999999	1.94777291020662e-06\\
1.55	1.97428371442492e-06\\
1.55599999999997	1.83910983122857e-06\\
1.56	1.74908836967981e-06\\
1.56000000000001	1.74908836967963e-06\\
1.56599999999999	1.65165286676889e-06\\
1.566	1.70035602769292e-06\\
1.57	1.62962892399672e-06\\
1.57000000000001	1.62962892399656e-06\\
1.57400000000001	1.6293100443193e-06\\
1.57499999999999	1.6287681604847e-06\\
1.575	1.62876816048454e-06\\
1.579	1.56256668087594e-06\\
1.57999999999999	1.54109344496212e-06\\
1.58	1.56218384378283e-06\\
1.584	1.50040670035023e-06\\
1.588	1.46430315212391e-06\\
1.59	1.4282577557737e-06\\
1.59000000000001	1.42825775577359e-06\\
1.59499999999998	1.41769231363599e-06\\
1.595	1.4361142064414e-06\\
1.59999999999997	1.35036090424861e-06\\
1.6	1.36847356186723e-06\\
1.60000000000001	1.36847356186712e-06\\
1.60499999999998	1.30195464573961e-06\\
1.60999999999995	1.23635779178789e-06\\
1.60999999999998	1.27046400806205e-06\\
1.61	1.2704640080618e-06\\
1.61999999999994	1.11549227211584e-06\\
1.62	1.11549227211498e-06\\
1.62000000000001	1.12376872343003e-06\\
1.62399999999998	1.06851912403027e-06\\
1.624	1.08472788449051e-06\\
1.62799999999997	1.05535742151132e-06\\
1.63000000000001	1.03350081252237e-06\\
1.63000000000003	1.03350081252227e-06\\
1.634	1.01376504429724e-06\\
1.63799999999997	9.58533597140299e-07\\
1.63999999999999	9.67871070173752e-07\\
1.64	9.67871070173683e-07\\
1.64499999999999	9.37705464810748e-07\\
1.645	9.70932722973156e-07\\
1.64999999999998	9.09103170025161e-07\\
1.65	9.09103170025046e-07\\
1.653	9.35733752664434e-07\\
1.65300000000001	9.35733752664371e-07\\
1.65600000000001	9.00988304907755e-07\\
1.65900000000001	9.24494615115417e-07\\
1.66	9.13438374563191e-07\\
1.66000000000002	9.13438374563121e-07\\
1.66600000000001	8.96784900843523e-07\\
1.67	8.55882658594984e-07\\
1.67000000000002	8.60946983890547e-07\\
1.67600000000001	8.0364178263509e-07\\
1.67999999999998	7.7365964425959e-07\\
1.68	7.78639027921126e-07\\
1.68199999999998	7.71635240212619e-07\\
1.682	7.90817830816767e-07\\
1.68399999999998	7.83403369448677e-07\\
1.68599999999997	7.66693560778972e-07\\
1.68999999999994	7.69292841784085e-07\\
1.68999999999998	7.73605122061901e-07\\
1.69	7.73605122061844e-07\\
1.69799999999994	7.11795364879067e-07\\
1.69999999999998	6.95328194211674e-07\\
1.7	6.99567938655597e-07\\
1.70799999999994	6.37926495214734e-07\\
1.70999999999998	6.29741544044697e-07\\
1.71	6.33841176039023e-07\\
1.711	6.25624656984822e-07\\
1.71100000000001	6.37721052824856e-07\\
1.71200000000001	6.29569509682618e-07\\
1.71300000000001	6.29381600718474e-07\\
1.715	6.13396233068591e-07\\
1.71500000000001	6.21163333356336e-07\\
1.71700000000001	6.13058895877808e-07\\
1.71900000000001	6.12440803222561e-07\\
1.71999999999998	6.12031895043435e-07\\
1.72	6.12031895043378e-07\\
1.724	6.09778845213565e-07\\
1.72799999999999	5.87262267043992e-07\\
1.72999999999998	5.72985169151607e-07\\
1.73	5.95450465072957e-07\\
1.73799999999999	5.40319268653287e-07\\
1.74	5.36226363129747e-07\\
1.74000000000001	5.36226363129696e-07\\
1.74800000000001	4.95735989715489e-07\\
1.74999999999998	4.93183443040926e-07\\
1.75	5.04539623497346e-07\\
1.75799999999999	4.6061563358697e-07\\
1.75999999999998	4.66110628704122e-07\\
1.76	4.68750779422887e-07\\
1.76799999999999	4.25887847056256e-07\\
1.76899999999998	4.28360481133617e-07\\
1.769	4.28360481133578e-07\\
1.76999999999998	4.30645740064659e-07\\
1.77	4.38048335748003e-07\\
1.771	4.32749800799308e-07\\
1.77199999999999	4.37046116619972e-07\\
1.77399999999998	4.2648178810436e-07\\
1.77799999999997	4.14683018466155e-07\\
1.77999999999998	4.06438831417674e-07\\
1.78	4.06438831417632e-07\\
1.78499999999998	3.8965574686656e-07\\
1.785	3.91828489138605e-07\\
1.78999999999998	3.6733461100914e-07\\
1.79	3.75844378515583e-07\\
1.79499999999998	3.60688028793078e-07\\
1.798	3.511780618485e-07\\
1.79800000000001	3.51178061848465e-07\\
1.8	3.52106200631837e-07\\
1.80000000000001	3.52106200631802e-07\\
1.802	3.52785630466744e-07\\
1.80399999999999	3.442947248964e-07\\
1.80799999999996	3.36721771558477e-07\\
1.81	3.37268134115627e-07\\
1.81000000000001	3.37268134115595e-07\\
1.81799999999996	3.06578877407926e-07\\
1.81999999999998	3.07473465697648e-07\\
1.82	3.07473465697619e-07\\
1.82699999999998	2.90616714875071e-07\\
1.827	2.9061671487505e-07\\
1.82999999999998	2.86482586506949e-07\\
1.83	2.86482586506922e-07\\
1.83299999999999	2.82376208946729e-07\\
1.83599999999997	2.7548367510262e-07\\
1.84	2.62804398403692e-07\\
1.84000000000001	2.6838166942097e-07\\
1.84599999999999	2.55383628764203e-07\\
1.85	2.46135164176027e-07\\
1.85000000000001	2.46135164176004e-07\\
1.85499999999998	2.38064250063784e-07\\
1.855	2.38064250063762e-07\\
1.85599999999998	2.41338949478374e-07\\
1.856	2.41338949478352e-07\\
1.85699999999999	2.42062906327916e-07\\
1.85799999999999	2.39256188054139e-07\\
1.85999999999998	2.37144795263605e-07\\
1.86	2.40512681990983e-07\\
1.86399999999999	2.30660087102647e-07\\
1.86799999999997	2.19926761983164e-07\\
1.86999999999999	2.21068354482666e-07\\
1.87	2.25179886257982e-07\\
1.87799999999997	2.05287832464162e-07\\
1.87999999999999	2.0366353784825e-07\\
1.88	2.03663537848232e-07\\
1.88499999999998	1.9563602058068e-07\\
1.885	1.98479188372391e-07\\
1.88999999999998	1.89303194715895e-07\\
1.89	1.89303194715873e-07\\
1.89499999999998	1.81734382696026e-07\\
1.89999999999996	1.72888278000068e-07\\
1.89999999999998	1.74697427895042e-07\\
1.9	1.76476554889734e-07\\
1.90999999999996	1.63857685292465e-07\\
1.91	1.63857685292451e-07\\
1.91000000000001	1.63857685292442e-07\\
1.91399999999998	1.64512122397848e-07\\
1.914	1.66091916288026e-07\\
1.91799999999997	1.60306660785942e-07\\
1.91999999999999	1.63529966687031e-07\\
1.92	1.6498299191127e-07\\
1.92399999999997	1.59417115349903e-07\\
1.92499999999998	1.58764557997931e-07\\
1.925	1.58764557997922e-07\\
1.92899999999997	1.54077686833335e-07\\
1.92999999999999	1.53453056434172e-07\\
1.93	1.54845085805457e-07\\
1.93399999999997	1.49601538127199e-07\\
1.93799999999994	1.45840652460149e-07\\
1.93999999999999	1.44661522452915e-07\\
1.94	1.45325532401434e-07\\
1.94299999999998	1.41617560638609e-07\\
1.943	1.44218210544771e-07\\
1.94599999999998	1.4314098321959e-07\\
1.94899999999996	1.40887876333623e-07\\
1.95	1.39752417240387e-07\\
1.95000000000002	1.40358093328541e-07\\
1.95599999999998	1.33526598209488e-07\\
1.95999999999998	1.29504135860807e-07\\
1.96	1.30099386945503e-07\\
1.96599999999996	1.24219250114535e-07\\
1.97	1.21715000258867e-07\\
1.97000000000001	1.22275575733762e-07\\
1.97199999999998	1.19847864741603e-07\\
1.972	1.24166704337322e-07\\
1.97399999999997	1.22753172874585e-07\\
1.97599999999994	1.20281659993291e-07\\
1.97999999999988	1.19237108049575e-07\\
1.98	1.20187663971726e-07\\
1.98000000000002	1.20187663971714e-07\\
1.9879999999999	1.13899972661772e-07\\
1.99	1.11911700155382e-07\\
1.99000000000002	1.11911700155371e-07\\
1.99499999999998	1.06366983227128e-07\\
1.995	1.06366983227117e-07\\
1.99999999999997	1.00932589910134e-07\\
1.99999999999998	1.01360541247638e-07\\
2	1.02205403771855e-07\\
2.00099999999997	1.02693110512112e-07\\
2.001	1.03495609921055e-07\\
2.00199999999999	1.02344065826376e-07\\
2.00299999999999	1.03147803073102e-07\\
2.00499999999998	1.00881029916298e-07\\
2.00899999999997	9.82992368369593e-08\\
2.00999999999997	9.72034123073144e-08\\
2.01	9.89944730335925e-08\\
2.01799999999997	9.14902618049167e-08\\
2.01999999999997	8.94131076983563e-08\\
2.02	9.00906135979636e-08\\
2.02799999999997	8.19913403831239e-08\\
2.02999999999997	8.06810103035592e-08\\
2.03	8.13360089114107e-08\\
2.03799999999997	7.52064325398482e-08\\
2.03999999999997	7.41193066739066e-08\\
2.04	7.41193066738871e-08\\
2.04799999999997	7.02127564767172e-08\\
2.04999999999997	6.87723867840024e-08\\
2.05	6.87723867839864e-08\\
2.05799999999997	6.39922966600587e-08\\
2.05899999999997	6.33652443191458e-08\\
2.059	6.39276136077674e-08\\
2.05999999999997	6.38626533581651e-08\\
2.06	6.46749868615192e-08\\
2.06099999999999	6.40661010333613e-08\\
2.06199999999999	6.42586164156106e-08\\
2.06399999999998	6.30855768616201e-08\\
2.06499999999997	6.32848780464496e-08\\
2.065	6.40376597537922e-08\\
2.06899999999998	6.28057792665756e-08\\
2.06999999999997	6.22757891799262e-08\\
2.07	6.36765134632778e-08\\
2.07399999999998	6.25048155680383e-08\\
2.07799999999997	6.0506394786174e-08\\
2.07999999999997	6.06097953114446e-08\\
2.08	6.06097953114339e-08\\
2.08799999999997	5.70794922999167e-08\\
2.088	5.7079492299906e-08\\
2.08999999999997	5.63900585486168e-08\\
2.09	5.65965142317566e-08\\
2.09199999999997	5.61207807879647e-08\\
2.09399999999994	5.5251282377862e-08\\
2.09799999999988	5.35526534900181e-08\\
2.09999999999997	5.35178095367888e-08\\
2.1	5.42866034871064e-08\\
2.10799999999988	5.15185906267741e-08\\
2.10999999999997	5.0778707128966e-08\\
2.11	5.18658749386863e-08\\
2.11699999999997	4.92794834933073e-08\\
2.117	4.99737458229146e-08\\
2.11999999999997	4.88463127181685e-08\\
2.12	4.95174379214875e-08\\
2.12299999999997	4.85414757340425e-08\\
2.12599999999994	4.73873980630228e-08\\
2.12999999999997	4.64690402811287e-08\\
2.13	4.66266688318055e-08\\
2.13499999999997	4.46332749556333e-08\\
2.135	4.52504260007247e-08\\
2.13999999999997	4.38033087039202e-08\\
2.14	4.40938690011406e-08\\
2.14499999999998	4.20440724646272e-08\\
2.14599999999997	4.27532387966127e-08\\
2.146	4.30204862084463e-08\\
2.14999999999997	4.1413075922289e-08\\
2.15	4.19340924409157e-08\\
2.15399999999998	4.03472226003241e-08\\
2.15799999999995	3.92817559294642e-08\\
2.15999999999997	3.87486454020516e-08\\
2.16	3.87486454020423e-08\\
2.16799999999995	3.61849680662014e-08\\
2.16999999999997	3.55603988676224e-08\\
2.17	3.55603988676136e-08\\
2.17499999999997	3.41947818753428e-08\\
2.175	3.43071270275203e-08\\
2.17999999999997	3.25217164211813e-08\\
2.18	3.2961572692464e-08\\
2.18499999999997	3.12202114008602e-08\\
2.18999999999994	2.9953028962546e-08\\
2.18999999999997	3.01601171349766e-08\\
2.19	3.01601171349686e-08\\
2.19999999999994	2.72112284633629e-08\\
2.19999999999997	2.7211228463354e-08\\
2.2	2.75077980066041e-08\\
2.20399999999997	2.65391890599798e-08\\
2.204	2.66340338717544e-08\\
2.20499999999997	2.63230439109554e-08\\
2.205	2.68752948474458e-08\\
2.206	2.69183010639374e-08\\
2.20699999999999	2.66099419690512e-08\\
2.20899999999998	2.65041325039175e-08\\
2.20999999999997	2.65237726714056e-08\\
2.21	2.65237726713981e-08\\
2.21399999999998	2.5500825905188e-08\\
2.21799999999997	2.43888037964897e-08\\
2.21999999999997	2.40135183512684e-08\\
2.22	2.41662496636428e-08\\
2.22799999999997	2.25054218126873e-08\\
2.22999999999997	2.20558566154182e-08\\
2.23	2.2055856615413e-08\\
2.23299999999997	2.1973050596497e-08\\
2.233	2.21088044324174e-08\\
2.23599999999997	2.14866814288852e-08\\
2.23899999999994	2.10243152749864e-08\\
2.23999999999997	2.08314488768797e-08\\
2.24	2.09626322283915e-08\\
2.24599999999994	1.99292293419772e-08\\
2.24999999999997	1.92520649510359e-08\\
2.25	1.937990273203e-08\\
2.25599999999994	1.8548974385183e-08\\
2.25999999999997	1.80020989364018e-08\\
2.26	1.80020989363984e-08\\
2.26199999999997	1.77674409172136e-08\\
2.262	1.77674409172111e-08\\
2.26399999999997	1.75376328054464e-08\\
2.26599999999994	1.73125665635304e-08\\
2.26999999999988	1.68762375648228e-08\\
2.26999999999997	1.71092772578838e-08\\
2.27	1.71663455441428e-08\\
2.27499999999997	1.64914819628987e-08\\
2.275	1.69307770224698e-08\\
2.27999999999997	1.6374780862758e-08\\
2.28	1.63747808627557e-08\\
2.28499999999997	1.61291470050472e-08\\
2.28999999999995	1.55872230010453e-08\\
2.29	1.55872230010399e-08\\
2.29099999999997	1.58314714045505e-08\\
2.291	1.5920280850836e-08\\
2.29199999999999	1.57940054383198e-08\\
2.29299999999999	1.60087142515295e-08\\
2.29499999999998	1.58394530201462e-08\\
2.29899999999997	1.53409927220129e-08\\
2.29999999999997	1.55301194934917e-08\\
2.3	1.5567766767126e-08\\
2.30799999999997	1.46134084292778e-08\\
2.30999999999997	1.44231554977244e-08\\
2.31	1.44231554977223e-08\\
2.31799999999997	1.35911973834872e-08\\
2.31999999999997	1.34205695219526e-08\\
2.32	1.34930544774023e-08\\
2.32799999999997	1.27280274260107e-08\\
2.32999999999997	1.25340127737661e-08\\
2.33	1.26742350098043e-08\\
2.33799999999997	1.19549905199563e-08\\
2.33999999999997	1.17718443370179e-08\\
2.34	1.19062875487586e-08\\
2.34499999999997	1.15843017864929e-08\\
2.345	1.1616084890926e-08\\
2.34899999999997	1.12591815732141e-08\\
2.349	1.12906956770264e-08\\
2.34999999999997	1.12020347403182e-08\\
2.35	1.12332829308084e-08\\
2.35099999999999	1.11758158810184e-08\\
2.35199999999999	1.11487576229536e-08\\
2.35399999999998	1.10921838976899e-08\\
2.35799999999996	1.07423230706754e-08\\
2.35999999999997	1.05682650517345e-08\\
2.36	1.06264838524083e-08\\
2.36799999999996	9.93631180233428e-09\\
2.36999999999997	9.82230289481171e-09\\
2.37	9.8785616588607e-09\\
2.37799999999996	9.30635903595383e-09\\
2.378	9.35979667837751e-09\\
2.37999999999997	9.18978615946246e-09\\
2.38	9.39476107365348e-09\\
2.38199999999997	9.27394884381868e-09\\
2.38399999999994	9.10396772322968e-09\\
2.38799999999988	8.92923115188472e-09\\
2.38999999999997	8.75879414552845e-09\\
2.39	8.82675471320852e-09\\
2.39799999999988	8.17041364040943e-09\\
2.39999999999997	8.08042054322279e-09\\
2.4	8.14502788930141e-09\\
2.40699999999997	7.7051567573435e-09\\
2.407	7.70515675734203e-09\\
2.40999999999997	7.59212943484021e-09\\
2.41	7.59212943483878e-09\\
2.41299999999997	7.42348070217924e-09\\
2.41499999999997	7.28816153908719e-09\\
2.415	7.32657371797636e-09\\
2.41799999999997	7.14716938900019e-09\\
2.41999999999997	7.01792402476247e-09\\
2.42	7.05537057508051e-09\\
2.42299999999997	6.90269569055564e-09\\
2.42599999999994	6.75450799613444e-09\\
2.42999999999997	6.51629387774695e-09\\
2.42999999999999	6.58693826521822e-09\\
2.43599999999994	6.24325289544034e-09\\
2.43599999999997	6.3115478423583e-09\\
2.436	6.3775273252133e-09\\
2.43999999999997	6.18823409398401e-09\\
2.43999999999999	6.18823409398291e-09\\
2.44399999999996	6.09585673373296e-09\\
2.44799999999993	5.91511454205375e-09\\
2.44999999999999	5.8118765730539e-09\\
2.45000000000002	5.88363863513812e-09\\
2.45799999999996	5.482549174217e-09\\
2.45999999999997	5.4537878884569e-09\\
2.46	5.45378788845599e-09\\
2.46499999999997	5.28523873895361e-09\\
2.465	5.34843915650292e-09\\
2.46999999999996	5.12965235437078e-09\\
2.47	5.1296523543695e-09\\
2.47499999999997	4.97961454708062e-09\\
2.47999999999993	4.76969052497285e-09\\
2.48	4.82772471439975e-09\\
2.48000000000003	4.82772471439894e-09\\
2.48499999999997	4.66267120029936e-09\\
2.485	4.66267120029848e-09\\
2.48999999999994	4.46353878033218e-09\\
2.48999999999999	4.46353878033068e-09\\
2.49000000000003	4.4743873558364e-09\\
2.49399999999997	4.31650455295221e-09\\
2.494	4.33775710945932e-09\\
2.49799999999993	4.17880559206744e-09\\
2.49999999999997	4.09369678657699e-09\\
2.5	4.13479029982351e-09\\
2.50399999999994	4.00674624816126e-09\\
2.50799999999988	3.86370188630823e-09\\
2.50999999999997	3.78443346623884e-09\\
2.51	3.85864503911224e-09\\
2.51799999999988	3.57149591818438e-09\\
2.51999999999997	3.49814286296207e-09\\
2.52	3.52427454964188e-09\\
2.52299999999997	3.41654785768114e-09\\
2.523	3.44201040983503e-09\\
2.52599999999996	3.36198569148389e-09\\
2.52899999999993	3.26807074853681e-09\\
2.52999999999997	3.23466025339792e-09\\
2.53	3.28203902846558e-09\\
2.53599999999993	3.11799865379945e-09\\
2.53999999999997	2.99404464538969e-09\\
2.54	3.0161058354819e-09\\
2.54599999999993	2.83798390016734e-09\\
2.54999999999997	2.74561770230277e-09\\
2.55	2.76656805036997e-09\\
2.55199999999997	2.71790025895766e-09\\
2.552	2.71790025895707e-09\\
2.55399999999996	2.67682761040473e-09\\
2.55499999999997	2.64984440033936e-09\\
2.555	2.663175543659e-09\\
2.55699999999997	2.62296964192583e-09\\
2.55899999999994	2.59608065348004e-09\\
2.55999999999997	2.57016005534196e-09\\
2.56	2.5701600553414e-09\\
2.56399999999994	2.49927605977906e-09\\
2.56799999999987	2.40073481677273e-09\\
2.56999999999997	2.38206515936409e-09\\
2.57	2.38206515936374e-09\\
2.57799999999987	2.23152167361286e-09\\
2.57999999999997	2.21770966810392e-09\\
2.58	2.21770966810393e-09\\
2.58099999999997	2.19780814378205e-09\\
2.581	2.20309493251546e-09\\
2.58199999999999	2.18356166260779e-09\\
2.58299999999999	2.16963576432569e-09\\
2.58499999999998	2.13758923164779e-09\\
2.58899999999996	2.07767826766031e-09\\
2.58999999999997	2.08102008668648e-09\\
2.59	2.08591320104882e-09\\
2.59799999999997	1.96135271937388e-09\\
2.59999999999997	1.97012681596115e-09\\
2.6	1.97916629034583e-09\\
2.60799999999997	1.87199531541145e-09\\
2.60999999999997	1.85174447753613e-09\\
2.61	1.85174447753591e-09\\
2.61799999999996	1.76533642064709e-09\\
2.62	1.74897612068931e-09\\
2.62000000000002	1.75763615939522e-09\\
2.62499999999997	1.72582704033739e-09\\
2.625	1.72995796476464e-09\\
2.62999999999995	1.68561912707013e-09\\
2.62999999999999	1.71744498484639e-09\\
2.63000000000002	1.72507640035456e-09\\
2.63499999999997	1.68237949253693e-09\\
2.63899999999997	1.67671335357823e-09\\
2.639	1.68371804185563e-09\\
2.63999999999997	1.6748759657072e-09\\
2.64	1.69187859361906e-09\\
2.641	1.68298972462107e-09\\
2.64199999999999	1.68715356210145e-09\\
2.64399999999999	1.66913399524152e-09\\
2.64799999999997	1.63569213521441e-09\\
2.64999999999997	1.6170473671719e-09\\
2.65	1.62022846960827e-09\\
2.65799999999997	1.54649947011609e-09\\
2.65999999999997	1.53289105616445e-09\\
2.66	1.54502566369874e-09\\
2.66799999999997	1.4708686782084e-09\\
2.668	1.4708686782082e-09\\
2.66999999999997	1.4734335839806e-09\\
2.67	1.4788727087151e-09\\
2.67199999999998	1.45869842860583e-09\\
2.67399999999995	1.45930860534856e-09\\
2.6779999999999	1.41860872031057e-09\\
2.67999999999997	1.39813106958222e-09\\
2.68	1.40312350804481e-09\\
2.6879999999999	1.32118477345047e-09\\
2.68999999999997	1.30562460191817e-09\\
2.69	1.31045117234251e-09\\
2.69499999999997	1.26863431788405e-09\\
2.695	1.27322620232337e-09\\
2.69699999999997	1.25268853166805e-09\\
2.697	1.27028939618664e-09\\
2.69899999999996	1.24971392645067e-09\\
2.69999999999997	1.23940949670543e-09\\
2.7	1.24362427400298e-09\\
2.70199999999997	1.2230507429089e-09\\
2.70399999999993	1.20671072532403e-09\\
2.70799999999986	1.17005413949094e-09\\
2.70999999999997	1.15577665493072e-09\\
2.71	1.15577665493049e-09\\
2.71799999999986	1.08105941387199e-09\\
2.71999999999997	1.06100514385911e-09\\
2.72	1.06669791625334e-09\\
2.72599999999997	1.01032664216834e-09\\
2.726	1.01032664216811e-09\\
2.72999999999997	9.75708592820849e-10\\
2.73	9.79266842939801e-10\\
2.73399999999998	9.40064690123834e-10\\
2.73799999999995	9.07848003069987e-10\\
2.73999999999997	8.94864028237352e-10\\
2.74	9.01463999918023e-10\\
2.74799999999995	8.33257986005246e-10\\
2.74999999999997	8.26690260139298e-10\\
2.75	8.2669026013912e-10\\
2.75499999999997	7.88174719750825e-10\\
2.755	7.94234355930307e-10\\
2.75999999999996	7.5794132854279e-10\\
2.76	7.63799020892517e-10\\
2.76499999999997	7.35304496085225e-10\\
2.765	7.38067468085095e-10\\
2.76500000000003	7.38067468084946e-10\\
2.77	7.13968757902715e-10\\
2.77000000000003	7.13968757902575e-10\\
2.77499999999999	6.88700899750153e-10\\
2.77999999999996	6.59723707049839e-10\\
2.78	6.64668958105903e-10\\
2.78399999999997	6.43662566626117e-10\\
2.784	6.43662566625996e-10\\
2.78799999999996	6.27008349596609e-10\\
2.78999999999997	6.17768100721538e-10\\
2.79	6.17768100721424e-10\\
2.79399999999997	6.01989022100553e-10\\
2.79799999999993	5.86600832673528e-10\\
2.79999999999997	5.79045756109751e-10\\
2.8	5.79045756109644e-10\\
2.80799999999993	5.46667986829453e-10\\
2.80999999999997	5.37558311741725e-10\\
2.81	5.42543095823169e-10\\
2.81299999999997	5.29076967991165e-10\\
2.813	5.3291614639354e-10\\
2.81599999999996	5.19677537196563e-10\\
2.81899999999993	5.10363782762604e-10\\
2.81999999999997	5.07876354962549e-10\\
2.82	5.07876354962453e-10\\
2.82599999999993	4.86714959704943e-10\\
2.82999999999997	4.74769085775784e-10\\
2.83	4.76431586265775e-10\\
2.83499999999997	4.58436656684327e-10\\
2.835	4.61677813728149e-10\\
2.83999999999997	4.44973212777626e-10\\
2.84	4.48104481898184e-10\\
2.84199999999997	4.43267427093769e-10\\
2.842	4.43267427093701e-10\\
2.84399999999996	4.3927643250919e-10\\
2.84599999999993	4.33172913392207e-10\\
2.84999999999985	4.23527274955489e-10\\
2.85	4.25658811369901e-10\\
2.85000000000003	4.26357695360256e-10\\
2.85799999999989	4.04136091544205e-10\\
2.85999999999997	4.02937368579858e-10\\
2.86	4.05536064187916e-10\\
2.86799999999986	3.85074315914869e-10\\
2.86999999999997	3.83808645001383e-10\\
2.87	3.86198176350075e-10\\
2.87099999999997	3.83712464367222e-10\\
2.871	3.84877508829645e-10\\
2.87199999999999	3.82401852964731e-10\\
2.87299999999999	3.81081027916989e-10\\
2.87499999999998	3.76742956438264e-10\\
2.87899999999996	3.67043723208399e-10\\
2.87999999999997	3.65755674332289e-10\\
2.88	3.66852571524291e-10\\
2.88799999999997	3.4842808232978e-10\\
2.88999999999997	3.43751669514829e-10\\
2.89	3.45872457307914e-10\\
2.89799999999997	3.29904201821261e-10\\
2.89999999999997	3.26592548331874e-10\\
2.9	3.26592548331827e-10\\
2.90499999999997	3.16813458522467e-10\\
2.905	3.16813458522419e-10\\
2.90999999999998	3.06848806509671e-10\\
2.91000000000001	3.07798938431916e-10\\
2.91499999999999	2.98524343450218e-10\\
2.91999999999997	2.87174872702282e-10\\
2.92	2.87174872702217e-10\\
2.92899999999997	2.66548697301015e-10\\
2.92899999999999	2.6654869730096e-10\\
2.92999999999997	2.64598674233154e-10\\
2.93	2.65044377095049e-10\\
2.93099999999999	2.63525689158258e-10\\
2.93199999999999	2.62857097758219e-10\\
2.93399999999997	2.58544523285751e-10\\
2.93799999999994	2.49225942512495e-10\\
2.94	2.47873875608068e-10\\
2.94000000000003	2.48649037859479e-10\\
2.94799999999997	2.30718406204144e-10\\
2.95	2.29306787863973e-10\\
2.95000000000003	2.30019757102015e-10\\
2.95799999999997	2.1304685012166e-10\\
2.958	2.15779687741507e-10\\
2.96	2.12338529775224e-10\\
2.96000000000003	2.12338529775175e-10\\
2.96200000000003	2.09887250312039e-10\\
2.96400000000003	2.05901722764019e-10\\
2.96800000000003	1.99610760990733e-10\\
2.97	1.9577863479662e-10\\
2.97000000000003	1.97239181729452e-10\\
2.97499999999997	1.88990092743691e-10\\
2.975	1.88990092743646e-10\\
2.97999999999995	1.80169777869945e-10\\
2.98	1.80169777869852e-10\\
2.98499999999995	1.7183509368737e-10\\
2.98699999999997	1.68859520859735e-10\\
2.98699999999999	1.69393644979093e-10\\
2.98999999999997	1.65773483138875e-10\\
2.99	1.66281636099995e-10\\
2.99299999999998	1.61864371791306e-10\\
2.99599999999996	1.59635746085312e-10\\
2.99999999999997	1.54669769264705e-10\\
3	1.54669769264676e-10\\
3.00599999999996	1.47744029160907e-10\\
3.00999999999997	1.43331271480212e-10\\
3.01	1.43561531402361e-10\\
3.01599999999996	1.37692236650362e-10\\
3.01599999999999	1.38143161346684e-10\\
3.01600000000002	1.39022740187978e-10\\
3.01999999999997	1.35296043099392e-10\\
3.02	1.35725042085464e-10\\
3.02399999999995	1.32195684128302e-10\\
3.0279999999999	1.29213451052491e-10\\
3.02999999999997	1.27971305459939e-10\\
3.03	1.28779312088502e-10\\
3.0379999999999	1.22833374300311e-10\\
3.03999999999997	1.21317116141065e-10\\
3.04	1.22827520261657e-10\\
3.04499999999997	1.19512672937771e-10\\
3.04499999999999	1.19512672937756e-10\\
3.04999999999996	1.16869630382777e-10\\
3.05	1.16869630382753e-10\\
3.05499999999997	1.13859172595066e-10\\
3.05999999999993	1.10693621567545e-10\\
3.05999999999997	1.11028894666975e-10\\
3.06	1.11194403859188e-10\\
3.06999999999993	1.04906844214204e-10\\
3.06999999999997	1.05233688993237e-10\\
3.07	1.05555075521828e-10\\
3.07399999999997	1.0358204524916e-10\\
3.07399999999999	1.03887804731507e-10\\
3.07799999999996	1.01233284429459e-10\\
3.07999999999997	1.0091411554656e-10\\
3.08	1.00914115546545e-10\\
3.08399999999997	9.85926519054759e-11\\
3.08799999999993	9.57832387344396e-11\\
3.08999999999997	9.47648839560462e-11\\
3.09	9.51668001776705e-11\\
3.09799999999993	8.94708154001273e-11\\
3.09999999999997	8.85843116030884e-11\\
3.1	8.85843116030721e-11\\
3.10299999999997	8.69874178564513e-11\\
3.10299999999999	8.74736015334317e-11\\
3.10599999999996	8.5621293161818e-11\\
3.10899999999993	8.35411068458303e-11\\
3.10999999999997	8.37582277719007e-11\\
3.11	8.39761510159633e-11\\
3.11499999999997	8.05572934301557e-11\\
3.115	8.10860873506863e-11\\
3.11999999999998	7.77371805313334e-11\\
3.12000000000001	7.82437208496189e-11\\
3.12499999999998	7.49565574899222e-11\\
3.12999999999996	7.17230816060082e-11\\
3.13000000000001	7.22086097360893e-11\\
3.13199999999999	7.09318640361395e-11\\
3.13200000000002	7.09318640361244e-11\\
3.13400000000001	7.01344274102728e-11\\
3.136	6.88843927711402e-11\\
3.13999999999997	6.64216916254036e-11\\
3.14	6.68681163596875e-11\\
3.14000000000003	6.68681163596751e-11\\
3.14799999999998	6.25064030213849e-11\\
3.14999999999998	6.13346647739284e-11\\
3.15	6.13346647739147e-11\\
3.15799999999995	5.71127304823236e-11\\
3.15999999999997	5.6009570038036e-11\\
3.16	5.63280232667755e-11\\
3.16099999999997	5.59366592473455e-11\\
3.16099999999999	5.59366592473322e-11\\
3.16199999999999	5.5695974654528e-11\\
3.16299999999998	5.54481631178987e-11\\
3.16499999999997	5.45192302003335e-11\\
3.16899999999994	5.2411597930113e-11\\
3.16999999999997	5.24386000279841e-11\\
3.17	5.25697260785608e-11\\
3.17799999999994	4.85048196613132e-11\\
3.17999999999997	4.77053917421249e-11\\
3.18	4.77053917421127e-11\\
3.185	4.55191208834748e-11\\
3.18500000000003	4.5701982401618e-11\\
3.18999999999997	4.3573731919699e-11\\
3.18999999999999	4.35737319196888e-11\\
3.19499999999993	4.19150413457351e-11\\
3.19999999999986	4.03059132447715e-11\\
3.19999999999999	4.0305913244728e-11\\
3.20000000000002	4.04684762711741e-11\\
3.20999999999989	3.72352820477068e-11\\
3.21000000000002	3.73413361389606e-11\\
3.21000000000005	3.73413361389536e-11\\
3.21899999999997	3.49623367371864e-11\\
3.21899999999999	3.50648149128519e-11\\
3.22	3.48634991316827e-11\\
3.22000000000003	3.48634991316771e-11\\
3.22100000000004	3.47151701179889e-11\\
3.22200000000005	3.44695535919935e-11\\
3.22400000000006	3.408430166087e-11\\
3.22800000000009	3.32442659495822e-11\\
3.23000000000003	3.29792137952431e-11\\
3.23000000000006	3.29792137952487e-11\\
3.23800000000012	3.12596812401681e-11\\
3.23999999999997	3.12108182374106e-11\\
3.24	3.1296600254733e-11\\
3.24799999999997	2.97546409905604e-11\\
3.24799999999999	3.00836815782071e-11\\
3.24999999999997	2.97979335940689e-11\\
3.25	2.97979335940649e-11\\
3.25199999999998	2.94812624452433e-11\\
3.25399999999996	2.91354875296057e-11\\
3.25499999999998	2.90048722995161e-11\\
3.255	2.90431587181011e-11\\
3.25899999999996	2.84657715886528e-11\\
3.25999999999997	2.84557770356385e-11\\
3.26	2.84918331104704e-11\\
3.26399999999996	2.78856862267826e-11\\
3.26799999999992	2.759371333723e-11\\
3.26999999999997	2.73888069236119e-11\\
3.27	2.73888069236091e-11\\
3.27699999999999	2.65048089286194e-11\\
3.27700000000002	2.65048089286166e-11\\
3.27999999999997	2.61548531992801e-11\\
3.28	2.61871758195826e-11\\
3.28299999999995	2.58741745184288e-11\\
3.2859999999999	2.56271595142595e-11\\
3.28999999999997	2.5170249978454e-11\\
3.29	2.51702499784514e-11\\
3.2959999999999	2.45883095671598e-11\\
3.29999999999997	2.40942626316382e-11\\
3.3	2.42366083419208e-11\\
3.3059999999999	2.34828750939247e-11\\
3.30599999999995	2.36193274208285e-11\\
3.30599999999999	2.37500373339728e-11\\
3.30999999999997	2.3237957304586e-11\\
3.31	2.34597782763775e-11\\
3.31399999999998	2.29392785914287e-11\\
3.31799999999996	2.24333710618531e-11\\
3.31999999999997	2.216494568002e-11\\
3.32	2.21883863736448e-11\\
3.32499999999998	2.15290646094732e-11\\
3.325	2.15749407258142e-11\\
3.32999999999998	2.09640316456462e-11\\
3.33000000000001	2.10076072603197e-11\\
3.33499999999998	2.02877990080313e-11\\
3.33500000000001	2.04549452970211e-11\\
3.33999999999998	1.97539699261498e-11\\
3.34000000000001	1.9753969926146e-11\\
3.34499999999998	1.91642380319105e-11\\
3.34999999999996	1.8437514899069e-11\\
3.35000000000001	1.84375148990619e-11\\
3.35999999999996	1.70636732107856e-11\\
3.36	1.70636732107798e-11\\
3.36399999999997	1.65521507726901e-11\\
3.36399999999999	1.65877164628517e-11\\
3.36799999999996	1.6121892138228e-11\\
3.36999999999997	1.58925507121334e-11\\
3.37	1.58925507121302e-11\\
3.37399999999997	1.54566272341641e-11\\
3.37799999999993	1.49490276828233e-11\\
3.37999999999997	1.47464807214847e-11\\
3.38	1.47464807214816e-11\\
3.38799999999993	1.38050615583317e-11\\
3.38999999999997	1.3568856693223e-11\\
3.39	1.35993923269942e-11\\
3.39299999999997	1.32799175547666e-11\\
3.39299999999999	1.33240256556404e-11\\
3.39499999999998	1.30940966951115e-11\\
3.395	1.31371227818892e-11\\
3.39699999999999	1.29095455840263e-11\\
3.39899999999997	1.27261794132157e-11\\
3.39999999999997	1.26552172477774e-11\\
3.4	1.2708107689131e-11\\
3.40399999999997	1.2265978567287e-11\\
3.40799999999993	1.19077341979613e-11\\
3.40999999999997	1.17419399575992e-11\\
3.41	1.17419399575965e-11\\
3.41799999999993	1.10054182034671e-11\\
3.41999999999997	1.08184197542238e-11\\
3.42	1.08184197542215e-11\\
3.42199999999997	1.0679545302488e-11\\
3.42199999999999	1.06795453024857e-11\\
3.42399999999996	1.05403602928973e-11\\
3.42599999999992	1.04008609210073e-11\\
3.42999999999985	1.00617692473487e-11\\
3.42999999999997	1.0061769247338e-11\\
3.43	1.0120799375867e-11\\
3.43799999999986	9.41206812946868e-12\\
3.43999999999997	9.2742951462452e-12\\
3.44	9.27429514624297e-12\\
3.44799999999986	8.61364644289815e-12\\
3.44999999999997	8.44915032584905e-12\\
3.45	8.44915032584684e-12\\
3.45099999999996	8.39792593485624e-12\\
3.45099999999999	8.40648472343146e-12\\
3.45199999999999	8.32125815161916e-12\\
3.45299999999998	8.27048063620945e-12\\
3.45499999999996	8.13706537956138e-12\\
3.45899999999993	7.83218078991354e-12\\
3.45999999999997	7.75424373659487e-12\\
3.46	7.81475937604224e-12\\
3.465	7.45257360005501e-12\\
3.46500000000003	7.45257360005314e-12\\
3.46999999999997	7.12514534325156e-12\\
3.47	7.12514534324979e-12\\
3.47499999999994	6.81686466047144e-12\\
3.47999999999988	6.51385660412044e-12\\
3.47999999999994	6.51385660411711e-12\\
3.47999999999999	6.53982854839158e-12\\
3.48999999999987	6.01758257972631e-12\\
3.48999999999996	6.02983309604524e-12\\
3.48999999999999	6.02983309604405e-12\\
3.49999999999987	5.58803666135672e-12\\
3.49999999999996	5.58803666135273e-12\\
3.49999999999999	5.6111101371057e-12\\
3.50899999999999	5.22824669727846e-12\\
3.50900000000002	5.22824669727741e-12\\
3.50999999999999	5.20228773107363e-12\\
3.51000000000002	5.20228773107258e-12\\
3.51100000000001	5.17616612630884e-12\\
3.512	5.14989194707629e-12\\
3.51399999999999	5.07179509120752e-12\\
3.51799999999996	4.90804804705827e-12\\
3.51999999999997	4.85257923161766e-12\\
3.52	4.85257923161666e-12\\
3.52799999999994	4.57063461641887e-12\\
3.52999999999997	4.49857433617979e-12\\
3.53	4.5216167012014e-12\\
3.53499999999998	4.37101837283553e-12\\
3.535	4.3710183728348e-12\\
3.53799999999997	4.2723724684379e-12\\
3.53799999999999	4.28095736837726e-12\\
3.53999999999997	4.21721174262404e-12\\
3.54	4.22564906783089e-12\\
3.54199999999998	4.17160111699263e-12\\
3.54399999999996	4.12662576910571e-12\\
3.54799999999992	4.00784527793472e-12\\
3.54999999999997	3.95015983530416e-12\\
3.55	3.98090061523142e-12\\
3.55799999999992	3.76837274387784e-12\\
3.56	3.71600086954918e-12\\
3.56000000000003	3.72324851626491e-12\\
3.56699999999996	3.54557049188611e-12\\
3.56699999999999	3.55269355189226e-12\\
3.56999999999998	3.48578809928968e-12\\
3.57000000000001	3.49943618378236e-12\\
3.57299999999999	3.42677933804163e-12\\
3.57599999999997	3.35530826891702e-12\\
3.57999999999997	3.26508935503412e-12\\
3.58	3.26508935503355e-12\\
3.58599999999997	3.13107001129055e-12\\
3.58999999999997	3.04441087648662e-12\\
3.59	3.05089737462905e-12\\
3.59599999999997	2.93069957071836e-12\\
3.596	2.9337982968945e-12\\
3.6	2.84640375859827e-12\\
3.60000000000003	2.87025685032713e-12\\
3.60400000000003	2.78935267326322e-12\\
3.60499999999998	2.76780432016529e-12\\
3.605	2.78166908492518e-12\\
3.60900000000001	2.69583503514357e-12\\
3.61	2.68775456582624e-12\\
3.61000000000003	2.68775456582571e-12\\
3.61400000000003	2.61686733651355e-12\\
3.61800000000004	2.54900698312536e-12\\
3.61999999999997	2.51026778923098e-12\\
3.62	2.5332000707791e-12\\
3.62499999999999	2.44001293877835e-12\\
3.62500000000002	2.45905788501583e-12\\
3.62999999999998	2.37091936813234e-12\\
3.63000000000001	2.3729462540641e-12\\
3.63499999999997	2.28920898839183e-12\\
3.63999999999993	2.21124618210047e-12\\
3.63999999999997	2.21323905922601e-12\\
3.64	2.21717487559831e-12\\
3.64999999999993	2.07542575256033e-12\\
3.64999999999997	2.0791701944308e-12\\
3.65	2.07917019443045e-12\\
3.65399999999996	2.03708019305391e-12\\
3.65399999999999	2.03880596147051e-12\\
3.65799999999995	1.98364687052113e-12\\
3.65999999999997	1.95823268569957e-12\\
3.66	1.95823268569924e-12\\
3.66399999999996	1.90653393617181e-12\\
3.66799999999993	1.85589160693357e-12\\
3.66999999999997	1.83342776033851e-12\\
3.67	1.83990913899933e-12\\
3.67499999999998	1.77645775902021e-12\\
3.67500000000001	1.7764577590199e-12\\
3.67999999999998	1.71749370531642e-12\\
3.68000000000001	1.71749370531612e-12\\
3.68299999999996	1.6843265647448e-12\\
3.68299999999999	1.68738131479687e-12\\
3.68599999999995	1.65385771507931e-12\\
3.68899999999991	1.62003950420526e-12\\
3.68999999999998	1.61493437596852e-12\\
3.69000000000001	1.61637899361064e-12\\
3.69599999999992	1.55303017480743e-12\\
3.69999999999997	1.51895301631112e-12\\
3.7	1.52442480367425e-12\\
3.70599999999992	1.47035434581409e-12\\
3.71	1.43402767438118e-12\\
3.71000000000003	1.44301662527775e-12\\
3.71199999999996	1.42528159611316e-12\\
3.71199999999999	1.42897172544119e-12\\
3.71399999999993	1.41151292672588e-12\\
3.71599999999986	1.3979210184572e-12\\
3.71999999999973	1.36783154880464e-12\\
3.71999999999997	1.37237514718724e-12\\
3.72	1.37237514718704e-12\\
3.72799999999974	1.31362404813796e-12\\
3.72999999999997	1.29861289081209e-12\\
3.73	1.29861289081189e-12\\
3.73799999999974	1.23828315474138e-12\\
3.73999999999997	1.22259698025964e-12\\
3.74	1.22563410920993e-12\\
3.74099999999996	1.22077304866282e-12\\
3.74099999999999	1.22077304866264e-12\\
3.74199999999998	1.21296082287907e-12\\
3.74299999999998	1.20899412893674e-12\\
3.74499999999996	1.19341502252605e-12\\
3.74500000000001	1.19712330355132e-12\\
3.74899999999997	1.16963022063305e-12\\
3.74999999999997	1.1636199657268e-12\\
3.75	1.16361996572661e-12\\
3.75399999999997	1.13686144828645e-12\\
3.75799999999994	1.10587380723389e-12\\
3.75999999999997	1.09443290206035e-12\\
3.76	1.09443290206015e-12\\
3.76799999999994	1.03608512805947e-12\\
3.76999999999996	1.02043820333837e-12\\
3.76999999999999	1.02417076198596e-12\\
3.77799999999993	9.63606413199083e-13\\
3.77999999999996	9.52678556480156e-13\\
3.77999999999999	9.561102169662e-13\\
3.78799999999993	9.00459246905332e-13\\
3.78999999999996	8.90401654480946e-13\\
3.78999999999999	8.90401654480777e-13\\
3.79799999999993	8.4225395295503e-13\\
3.79899999999996	8.36065883012413e-13\\
3.79899999999999	8.38492324765827e-13\\
3.79999999999996	8.32352046466472e-13\\
3.79999999999999	8.34698642248643e-13\\
3.80099999999998	8.29750064539544e-13\\
3.80199999999997	8.2370356898818e-13\\
3.80399999999995	8.1397929043079e-13\\
3.80799999999991	7.9275711831857e-13\\
3.80999999999996	7.82376226335869e-13\\
3.80999999999999	7.82376226335723e-13\\
3.81499999999998	7.5846296346268e-13\\
3.81500000000001	7.59429195895401e-13\\
3.81999999999999	7.32294611828034e-13\\
3.82000000000002	7.33713514517609e-13\\
3.825	7.07320659695779e-13\\
3.82799999999996	6.92739672580248e-13\\
3.82799999999999	6.92739672580115e-13\\
3.82999999999999	6.83440416453337e-13\\
3.83000000000002	6.84334879083655e-13\\
3.83200000000002	6.74665515779105e-13\\
3.83400000000002	6.64648859720595e-13\\
3.83800000000001	6.46612056231732e-13\\
3.83999999999997	6.38527250427171e-13\\
3.84	6.39342974768525e-13\\
3.84799999999999	6.024223636788e-13\\
3.84999999999997	5.94984015929595e-13\\
3.85	5.94984015929485e-13\\
3.85699999999996	5.66442373655262e-13\\
3.85699999999999	5.67579940071012e-13\\
3.85999999999997	5.55268353693547e-13\\
3.86	5.55268353693437e-13\\
3.86299999999998	5.43332772132783e-13\\
3.86599999999997	5.30662490342615e-13\\
3.86999999999997	5.14475575638957e-13\\
3.87	5.15187423155161e-13\\
3.87599999999997	4.91119584552486e-13\\
3.87999999999997	4.74049385434387e-13\\
3.88	4.74049385434275e-13\\
3.88500000000001	4.55747468236192e-13\\
3.88500000000003	4.5637957273328e-13\\
3.88599999999996	4.52299114413497e-13\\
3.88599999999999	4.54723725554502e-13\\
3.88699999999998	4.51259667763305e-13\\
3.88799999999997	4.47248835334679e-13\\
3.88999999999995	4.39616508743682e-13\\
3.89	4.39616508743497e-13\\
3.89399999999996	4.24460610023226e-13\\
3.89799999999992	4.0981323041062e-13\\
3.9	4.03089732711203e-13\\
3.90000000000003	4.04175425342188e-13\\
3.90799999999995	3.76661278716818e-13\\
3.91	3.70049456499824e-13\\
3.91000000000003	3.71344778992314e-13\\
3.91499999999996	3.5536964595959e-13\\
3.91499999999999	3.56612085284889e-13\\
3.91999999999993	3.41397894813213e-13\\
3.92	3.41397894812993e-13\\
3.92000000000003	3.41640242819061e-13\\
3.92499999999997	3.27149708409983e-13\\
3.9299999999999	3.13587568079769e-13\\
3.93000000000003	3.13825744729637e-13\\
3.93000000000006	3.14295969494025e-13\\
3.93999999999993	2.90703951026278e-13\\
3.93999999999998	2.9092825289204e-13\\
3.94	2.90928252891983e-13\\
3.94399999999996	2.83887529419223e-13\\
3.94399999999999	2.84301901686327e-13\\
3.94799999999995	2.75680549869723e-13\\
3.94999999999998	2.73009241112833e-13\\
3.95	2.73389583151261e-13\\
3.95399999999996	2.6496413217378e-13\\
3.95499999999998	2.64335007347654e-13\\
3.95500000000001	2.64510639331296e-13\\
3.95899999999997	2.56225697388625e-13\\
3.96	2.54344437366407e-13\\
3.96000000000003	2.54344437366354e-13\\
3.96399999999999	2.46352428352016e-13\\
3.96799999999995	2.38440668825203e-13\\
3.97	2.34763909029407e-13\\
3.97000000000003	2.35422796319052e-13\\
3.97299999999996	2.2982614116621e-13\\
3.97299999999999	2.29826141166159e-13\\
3.97599999999992	2.25321191328581e-13\\
3.97899999999986	2.20056005759091e-13\\
3.97999999999997	2.18240776912838e-13\\
3.98	2.18957410597765e-13\\
3.98599999999987	2.08455597539517e-13\\
3.98999999999997	2.02228310667483e-13\\
3.99	2.0222831066744e-13\\
3.99599999999987	1.93047110278569e-13\\
3.99999999999997	1.87072338097397e-13\\
4	1.87338862975489e-13\\
4.00199999999993	1.84733214515862e-13\\
4.00199999999999	1.85244418959251e-13\\
4.00399999999992	1.82737880643827e-13\\
4.00599999999986	1.800927840953e-13\\
4.00999999999972	1.76086498697339e-13\\
4.00999999999995	1.76086498697076e-13\\
4.01	1.76086498697013e-13\\
4.01799999999973	1.66967090110191e-13\\
4.01999999999995	1.64670592745164e-13\\
4.02	1.64895601579526e-13\\
4.02500000000001	1.59436730575729e-13\\
4.02500000000006	1.59761751253215e-13\\
4.02999999999995	1.54157833651209e-13\\
4.03	1.54474881259466e-13\\
4.03099999999999	1.53362030790735e-13\\
4.03100000000005	1.53671137853154e-13\\
4.03200000000004	1.52861868698077e-13\\
4.03300000000003	1.52143423311826e-13\\
4.035	1.49933468848917e-13\\
4.03899999999996	1.46096595122715e-13\\
4.03999999999994	1.45359713642737e-13\\
4.04	1.45359713642677e-13\\
4.04799999999991	1.37412781939805e-13\\
4.04999999999994	1.35385676312291e-13\\
4.05	1.35714703389989e-13\\
4.05799999999991	1.279361748337e-13\\
4.05999999999994	1.26388266128213e-13\\
4.06	1.26466127119496e-13\\
};
\end{axis}
\end{tikzpicture}%
}
      \caption{The evolution of the difference in angular displacement between
        RM and EDF of pendulum $P_1$ for execution time $C_1 = 10$ ms.}
      \label{fig:02.6.10.1_diff}
    \end{figure}
  \end{minipage}
\end{minipage}
}

\noindent\makebox[\textwidth][c]{%
\begin{minipage}{\linewidth}
  \begin{minipage}{0.45\linewidth}
    \begin{figure}[H]\centering
      \scalebox{0.7}{\input{./figures/02.earliest_deadline_first/6/6.10.2.tex}}
      \caption{Evolution of the angular displacement of pendulum $P_2$.
        $C_2 = 10$ ms. \texttt{Blue}: RM scheduling, \texttt{Red}: EDF scheduling}
      \label{fig:02.6.10.2}
    \end{figure}
  \end{minipage}
  \hfill
  \begin{minipage}{0.45\linewidth}
    \begin{figure}[H]\centering
      \scalebox{0.7}{% This file was created by matlab2tikz.
%
%The latest updates can be retrieved from
%  http://www.mathworks.com/matlabcentral/fileexchange/22022-matlab2tikz-matlab2tikz
%where you can also make suggestions and rate matlab2tikz.
%
\definecolor{mycolor1}{rgb}{0.00000,0.44700,0.74100}%
%
\begin{tikzpicture}

\begin{axis}[%
width=4.133in,
height=3.26in,
at={(0.693in,0.44in)},
scale only axis,
xmin=0,
xmax=1.2,
xmajorgrids,
ymin=-0.03,
ymax=0.06,
ymajorgrids,
axis background/.style={fill=white}
]
\pgfplotsset{max space between ticks=50}
\addplot [color=mycolor1,solid,forget plot]
  table[row sep=crcr]{%
0	0\\
3.15544362088405e-30	0\\
0.000656101980281985	0\\
0.00393661188169191	0\\
0.00999999999999994	0\\
0.01	0\\
0.0199999999999999	0\\
0.02	0\\
0.0289999999999998	0\\
0.029	0\\
0.03	0\\
0.0300000000000002	0\\
0.0349999999999996	0\\
0.035	0\\
0.0399999999999993	0\\
0.04	0\\
0.0449999999999993	0.000441679030054043\\
0.0499999999999987	0.00176725723152613\\
0.05	0.0017672572315266\\
0.0500000000000004	0.00176725723152674\\
0.0579999999999996	0.00448171110221693\\
0.058	0.00448171110221707\\
0.0599999999999996	0.00512515329508091\\
0.06	0.00512515329508104\\
0.0619999999999995	0.00575478331183131\\
0.0639999999999991	0.00637072457215719\\
0.0679999999999982	0.00756202106998481\\
0.0699999999999991	0.00813760981439746\\
0.07	0.00813760981439769\\
0.0779999999999982	0.010308837035746\\
0.0799999999999991	0.0153958902792518\\
0.08	0.0153958902792521\\
0.087	0.0296040486139819\\
0.0870000000000009	0.0296040486139824\\
0.09	0.0335698695083808\\
0.0900000000000009	0.0290828375670482\\
0.0929999999999999	0.0308580700817593\\
0.095999999999999	0.0391344957316273\\
0.0999999999999991	0.0431475596052122\\
0.1	0.0431475596052125\\
0.104999999999999	0.0493416613600214\\
0.105	0.0426696780556582\\
0.109999999999999	0.0461104852399028\\
0.11	0.046110485239903\\
0.114999999999999	0.0471136921995653\\
0.115999999999999	0.0494078841624792\\
0.116	0.0392447220368222\\
0.119999999999999	0.0482145117485957\\
0.12	0.0386595902590375\\
0.123999999999999	0.0453063433702096\\
0.127999999999998	0.0531567189357926\\
0.129999999999998	0.0497262202243989\\
0.13	0.0497262202244005\\
0.137999999999998	0.0561779401085994\\
0.139999999999998	0.052715337836028\\
0.14	0.0496647367418693\\
0.144999999999998	0.0562760316709695\\
0.145	0.0476649976805948\\
0.149999999999998	0.0533253905246989\\
0.15	0.0480645792331894\\
0.154999999999998	0.052781004742135\\
0.159999999999996	0.0517192580236085\\
0.16	0.0505745647405536\\
0.169999999999996	0.0551852869751382\\
0.17	0.0508660410430244\\
0.173999999999998	0.0505717612425104\\
0.174	0.0505717612425088\\
0.174999999999998	0.0468480537943738\\
0.175	0.0434532244532215\\
0.176	0.0419483790986326\\
0.177	0.0419483870506123\\
0.179000000000001	0.0366713696883023\\
0.179999999999998	0.0355172172299605\\
0.18	0.0355172172299594\\
0.184000000000001	0.0330497165499365\\
0.188000000000002	0.0320579272512746\\
0.189999999999998	0.0310764854249459\\
0.19	0.0310764854249447\\
0.198000000000002	0.0275849423545264\\
0.199999999999998	0.0262915876616091\\
0.2	0.0257136496076842\\
0.202999999999998	0.0238758653398399\\
0.203	0.0238758653398385\\
0.205999999999998	0.0213720264061924\\
0.208999999999996	0.0191055790658144\\
0.209999999999998	0.018406029666431\\
0.21	0.0184060296664297\\
0.215999999999996	0.0142139527866402\\
0.219999999999998	0.0108742956273331\\
0.22	0.0114459144999536\\
0.225999999999996	0.00655815799012124\\
0.229999999999998	0.00302220891923227\\
0.23	0.00368828596368793\\
0.231999999999998	0.00191045021780206\\
0.232	0.00270639492919252\\
0.233999999999998	0.00184768427891718\\
0.235999999999997	0.000393149631130671\\
0.239999999999993	-0.00321494592956934\\
0.239999999999996	-0.000882817236339525\\
0.24	0.00096457444825903\\
0.244999999999998	-0.00359339373490569\\
0.245	-0.000562574591503016\\
0.249999999999998	-0.00306417145090604\\
0.25	-0.00306417145090677\\
0.254999999999999	-0.00652834094058509\\
0.259999999999997	-0.0108884524131659\\
0.26	-0.00978266446080452\\
0.260999999999996	-0.00950678433526607\\
0.261	-0.00719556605829292\\
0.262	-0.00803786863403162\\
0.263	-0.00887285604823165\\
0.265	-0.00751691215481554\\
0.269	-0.0107272919768689\\
0.269999999999997	-0.00837671601954457\\
0.27	-0.00837671601954568\\
0.278	-0.0144072139620252\\
0.279999999999996	-0.0145552649198164\\
0.28	-0.0145552649198167\\
0.288	-0.0187513467914475\\
0.289999999999996	-0.0187638919354637\\
0.29	-0.0161784918177907\\
0.298	-0.0209378367351971\\
0.299999999999996	-0.0219942772171013\\
0.3	-0.0213497069742262\\
0.308	-0.0250480988280803\\
0.309999999999996	-0.025197838872591\\
0.31	-0.0245545344811256\\
0.314999999999997	-0.0250264098925394\\
0.315	-0.0224626637030952\\
0.319	-0.0229954284808277\\
0.319000000000004	-0.0229954284808264\\
0.319999999999996	-0.0181479914308351\\
0.32	-0.0168726731299449\\
0.321	-0.0171226055382166\\
0.321999999999999	-0.0125121844003406\\
0.323999999999998	-0.0118304245144247\\
0.327999999999996	-0.0126289462986541\\
0.329999999999996	-0.00873159608145566\\
0.33	-0.00774537755548404\\
0.337999999999996	-0.00876116543286624\\
0.339999999999996	-0.00522636003581052\\
0.34	-0.00435971622850647\\
0.347999999999996	-0.00465300695706541\\
0.348	-0.00258590926337822\\
0.349999999999996	-0.00256911186656355\\
0.35	-0.00101646080853817\\
0.351999999999996	-0.000974503996187932\\
0.353999999999993	-0.000543972144484631\\
0.357999999999985	-0.000388596474391562\\
0.359999999999996	7.93506023012572e-05\\
0.36	0.000442638252461877\\
0.367999999999985	0.00169314619667634\\
0.369999999999996	0.00321946407140174\\
0.37	0.0038647440994769\\
0.377	0.00457802495869983\\
0.377000000000004	0.00694664362505648\\
0.379999999999997	0.00757947881924396\\
0.38	0.00757947881924536\\
0.382999999999993	0.00822173441711928\\
0.384999999999997	0.00846992270312917\\
0.385	0.00873162306992736\\
0.387999999999993	0.00935746750658201\\
0.389999999999997	0.0100985779196242\\
0.39	0.0110275272285672\\
0.392999999999993	0.0113918830182667\\
0.395999999999986	0.0117533490189651\\
0.399999999999997	0.0126636933881304\\
0.4	0.0126636933881316\\
0.405999999999986	0.013783493082334\\
0.406	0.0141739254059251\\
0.406000000000004	0.0148919471575738\\
0.41	0.0156869078997263\\
0.410000000000004	0.0156869078997272\\
0.414	0.0172575548225396\\
0.417999999999996	0.0179057138372304\\
0.419999999999997	0.018099275827235\\
0.42	0.0187796747944031\\
0.427999999999993	0.0194684072515148\\
0.429999999999997	0.0198433939419257\\
0.43	0.0198433939419262\\
0.435	0.0203676526393938\\
0.435000000000004	0.0204676577967708\\
0.439999999999997	0.0207560759246577\\
0.44	0.0208725269052239\\
0.444999999999993	0.0211643563181836\\
0.449999999999986	0.0213493167976932\\
0.449999999999993	0.0213987479301556\\
0.45	0.0214080014728658\\
0.454999999999997	0.0215290462381684\\
0.455	0.0215290462381685\\
0.459999999999997	0.0215552013685542\\
0.46	0.0215552013685541\\
0.463999999999997	0.0214986604672965\\
0.464	0.0214150090559736\\
0.467999999999997	0.0213770176242343\\
0.469999999999997	0.0213425216959107\\
0.47	0.0211939153320844\\
0.473999999999997	0.0210938892426399\\
0.477999999999993	0.0207613333412793\\
0.479999999999997	0.0204413125017835\\
0.48	0.0203084181374132\\
0.487999999999993	0.0198972865110075\\
0.489999999999997	0.0191980727813878\\
0.49	0.0190432498072475\\
0.492999999999997	0.0188570977211523\\
0.493	0.0184535093017505\\
0.495999999999997	0.0182544204920613\\
0.498999999999993	0.0176147040054395\\
0.499999999999997	0.01754108921826\\
0.5	0.0170889474680753\\
0.505999999999993	0.0161392096089635\\
0.509999999999993	0.0157945727830618\\
0.51	0.0157945727830608\\
0.515999999999993	0.0147295788131559\\
0.519999999999993	0.0143243438750522\\
0.52	0.0138111835906687\\
0.521999999999993	0.013602753310812\\
0.522	0.0131959753123341\\
0.523999999999993	0.0129882268903681\\
0.524999999999993	0.0124808054989124\\
0.525	0.0122799318497642\\
0.526999999999993	0.0120731299040012\\
0.528999999999986	0.0114670069337702\\
0.529999999999993	0.0109663999851921\\
0.53	0.0107682770136368\\
0.533999999999986	0.0103579025092742\\
0.537999999999972	0.00919672846990224\\
0.539999999999993	0.008816887619033\\
0.54	0.00881688761903197\\
0.547999999999972	0.0077494390966832\\
0.549999999999993	0.00754688716981493\\
0.55	0.007302216908426\\
0.550999999999993	0.00696724588075814\\
0.551	0.00689148484826729\\
0.551999999999997	0.00679167039319456\\
0.552999999999993	0.00626220764117588\\
0.554999999999986	0.00580280239048493\\
0.558999999999972	0.00542410980077546\\
0.559999999999993	0.00514506820321446\\
0.56	0.00514506820321338\\
0.567999999999972	0.00426139841056228\\
0.57	0.00392592368724712\\
0.570000000000007	0.00387211340703067\\
0.577999999999979	0.00323503298901789\\
0.579999999999993	0.00297964622268596\\
0.58	0.00297964622268507\\
0.587999999999972	0.00231739474398838\\
0.589999999999993	0.00213905706555207\\
0.59	0.00213905706555128\\
0.594999999999993	0.00173923828884435\\
0.595	0.00164893852921946\\
0.599999999999993	0.00132614868195774\\
0.6	0.00132614868195705\\
0.604999999999993	0.00090844911126289\\
0.608999999999993	0.00057582577983626\\
0.609	0.000507476125249256\\
0.609999999999993	0.000464016231311249\\
0.61	0.000233245065393619\\
0.610999999999997	0.000143846283160944\\
0.611999999999993	0.000102208253381094\\
0.613999999999986	-2.78295048405019e-07\\
0.617999999999972	-0.000151177267123281\\
0.619999999999993	-0.000242201363381829\\
0.62	-0.000261881311423405\\
0.627999999999972	-0.000544891949150353\\
0.629999999999993	-0.000657555295813419\\
0.63	-0.000669743214411862\\
0.637999999999972	-0.000848981088879843\\
0.637999999999993	-0.00090809426835295\\
0.638	-0.00091222142507734\\
0.639999999999993	-0.000948660154373839\\
0.64	-0.000949127120185938\\
0.641999999999993	-0.000982232819888839\\
0.643999999999986	-0.0010116716268384\\
0.647999999999971	-0.00106038237657792\\
0.649999999999993	-0.00107678196216255\\
0.65	-0.00106330997106194\\
0.657999999999971	-0.00111159173207449\\
0.659999999999993	-0.00111870758038991\\
0.66	-0.00106086861024457\\
0.664999999999993	-0.00104793483830313\\
0.665	-0.00104793483830306\\
0.666999999999993	-0.000986757730005801\\
0.667	-0.000986757730005697\\
0.668999999999993	-0.000909174048647339\\
0.669999999999993	-0.000906295404184681\\
0.67	-0.000823318360035316\\
0.671999999999993	-0.000745975385826051\\
0.673999999999986	-0.00073592352651429\\
0.677999999999972	-0.000690536962349693\\
0.679999999999993	-0.000673107574754788\\
0.68	-0.000655005420532894\\
0.687999999999972	-0.000557663207382975\\
0.689999999999993	-0.000499339842368052\\
0.69	-0.000422652507510342\\
0.695999999999993	-0.000316252112373571\\
0.696	-0.000316252112373347\\
0.699999999999993	-0.000110275413676911\\
0.7	-7.0994206429816e-05\\
0.703999999999993	-2.11840177861608e-05\\
0.707999999999986	0.000186979385993948\\
0.709999999999993	0.000213986921384934\\
0.71	0.000252726405412945\\
0.717999999999986	0.000356477049411099\\
0.719999999999993	0.000419370671425143\\
0.72	0.000457946755203066\\
0.724999999999993	0.000592575434510079\\
0.725	0.000630963096316519\\
0.729999999999993	0.000684528962268884\\
0.730000000000001	0.00083767727312434\\
0.734999999999994	0.000887194613897017\\
0.735000000000001	0.000887194613897329\\
0.739999999999994	0.000970610080928986\\
0.740000000000001	0.000970610080929182\\
0.744999999999994	0.0010494071064579\\
0.749999999999987	0.00112364394433773\\
0.750000000000001	0.00114165181426974\\
0.753999999999993	0.00116780497076638\\
0.754	0.00120332030522588\\
0.757999999999993	0.00122452636606211\\
0.759999999999993	0.00126813349767954\\
0.76	0.00128531456238315\\
0.763999999999993	0.00129912994831967\\
0.767999999999986	0.00134190711524239\\
0.77	0.00137774843613861\\
0.770000000000007	0.00144230056452057\\
0.777999999999993	0.00147502098531192\\
0.779999999999993	0.00147330257496011\\
0.78	0.0015893273171027\\
0.782999999999993	0.00161170733730422\\
0.783	0.0016117073373043\\
0.785999999999993	0.0016793269612197\\
0.788999999999986	0.00167147481858709\\
0.79	0.00171263423919993\\
0.790000000000007	0.00171263423919998\\
0.795999999999993	0.00173040368374607\\
0.8	0.00174768881132161\\
0.800000000000007	0.00176450904307513\\
0.804999999999993	0.00173577545668784\\
0.805	0.00178056361719398\\
0.809999999999987	0.00174655275935067\\
0.809999999999997	0.00177202294621144\\
0.810000000000007	0.00177202294621141\\
0.811999999999993	0.00177952322841854\\
0.812	0.00178462720516972\\
0.813999999999987	0.00176939235208653\\
0.815999999999973	0.00177225869477956\\
0.819999999999945	0.00174399890443909\\
0.819999999999987	0.00174399890443877\\
0.82	0.00175863616512728\\
0.827999999999944	0.0016922764959187\\
0.829999999999993	0.00167388405625389\\
0.830000000000001	0.00168469186615213\\
0.837999999999945	0.0016092084184897\\
0.839999999999993	0.00158872783669467\\
0.84	0.00159609320643645\\
0.840999999999993	0.00159081742273873\\
0.841000000000001	0.00159249001822401\\
0.841999999999997	0.00158215702850182\\
0.842999999999993	0.00157345834734279\\
0.844999999999986	0.00155283491034549\\
0.848999999999972	0.0015110059705831\\
0.849999999999993	0.00149953708415206\\
0.85	0.00149953708415198\\
0.857999999999972	0.00141335713937024\\
0.859999999999993	0.00138645392531041\\
0.86	0.00138238936520984\\
0.867999999999972	0.00130034422848676\\
0.869999999999993	0.00126994254876382\\
0.87	0.00126994254876372\\
0.874999999999994	0.00120699784027917\\
0.875000000000001	0.00120089645508975\\
0.879999999999994	0.00114924724424417\\
0.880000000000001	0.0011396450686689\\
0.884999999999994	0.00108930603085129\\
0.889999999999987	0.00103163534130275\\
0.890000000000001	0.00102094089272528\\
0.890000000000008	0.00101725340749752\\
0.899	0.000938617754665764\\
0.899000000000008	0.000915192011279946\\
0.9	0.000890121938268286\\
0.900000000000007	0.000890121938268197\\
0.901000000000004	0.000860445488156119\\
0.902	0.000852457230716766\\
0.903999999999993	0.000815472692986411\\
0.907999999999979	0.000785400204816008\\
0.909999999999993	0.000770983140684844\\
0.910000000000001	0.000762520619980295\\
0.917999999999972	0.000709123932071891\\
0.919999999999993	0.000688415750223483\\
0.920000000000001	0.000680022898650296\\
0.927999999999972	0.000618406536839953\\
0.927999999999994	0.000618406536839667\\
0.928000000000001	0.000618406536839573\\
0.929999999999994	0.000575169424553227\\
0.930000000000001	0.000566968905928989\\
0.931999999999994	0.0005571875621913\\
0.933999999999987	0.00053963618815569\\
0.937999999999972	0.000522113633482168\\
0.939999999999993	0.000505852993821866\\
0.940000000000001	0.000497871478509662\\
0.944999999999994	0.000463650015435777\\
0.945000000000001	0.000463650015435699\\
0.949999999999994	0.000447516531190832\\
0.950000000000001	0.000443693673171866\\
0.954999999999994	0.000429924432374509\\
0.956999999999994	0.000421251185530479\\
0.957000000000001	0.000417487955198408\\
0.959999999999993	0.000403398648582891\\
0.960000000000001	0.000388871213960151\\
0.962999999999993	0.00037946009856438\\
0.965999999999986	0.00037440709365363\\
0.969999999999993	0.000341428829227434\\
0.970000000000001	0.000334831984054052\\
0.975999999999986	0.000328029045813047\\
0.979999999999993	0.000299716872747529\\
0.980000000000001	0.000294275913819904\\
0.985999999999986	0.000288749297163395\\
0.985999999999993	0.000276203051045877\\
0.986000000000001	0.000276203051045854\\
0.989999999999993	0.000261887650130834\\
0.990000000000001	0.000261887650130814\\
0.993999999999993	0.000249145693294086\\
0.997999999999986	0.000237721323481904\\
0.999999999999993	0.000236541967076151\\
1	0.000223122518284479\\
1.00799999999999	0.000218393565018997\\
1.00999999999999	0.00021585770296349\\
1.01	0.000215857702963472\\
1.01499999999999	0.000211403387920425\\
1.015	0.000210182072947959\\
1.01999999999999	0.000204513407853983\\
1.02	0.00020031464058348\\
1.02499999999999	0.000194952187846985\\
1.02999999999997	0.000191306986682732\\
1.03	0.000185330552780972\\
1.03999999999997	0.000176981162314886\\
1.04	0.000176981162314859\\
1.04399999999999	0.000173128249400326\\
1.044	0.000173128249400311\\
1.04799999999999	0.000169167253797082\\
1.04999999999999	0.000166905794392093\\
1.05	0.000166019929994111\\
1.05399999999999	0.000160939156658654\\
1.05799999999997	0.000157138716731929\\
1.05999999999999	0.000155183536436443\\
1.06	0.000154644698584492\\
1.06799999999997	0.000146448547969113\\
1.06999999999999	0.000143880241104669\\
1.07	0.000143572188457047\\
1.07299999999999	0.000140031325489069\\
1.073	0.000140069054243101\\
1.07599999999999	0.000136749573704053\\
1.07899999999997	0.000133846399587096\\
1.07999999999999	0.00013271167626079\\
1.08	0.00013363538624324\\
1.08499999999999	0.000127873418659912\\
1.085	0.00012925531138508\\
1.08999999999999	0.0001240222974978\\
1.09	0.000124022297497785\\
1.09499999999999	0.000119108534103141\\
1.09999999999997	0.000112867268473409\\
1.09999999999999	0.000114193985047053\\
1.1	0.00011568764342246\\
1.10199999999999	0.000115395012475681\\
1.102	0.000115395012475667\\
1.10399999999999	0.000116762721152161\\
1.10599999999997	0.000114883185777586\\
1.10999999999994	0.000109589355648938\\
1.10999999999999	0.000111817253374903\\
1.11	0.000111817253374889\\
1.11799999999994	0.000103565047039856\\
1.11999999999999	0.000103332473036368\\
1.12	0.000103332473036356\\
1.12799999999994	9.31729262421098e-05\\
1.12999999999999	9.39142433579931e-05\\
1.13	9.39142433579813e-05\\
1.13099999999999	9.60266690332074e-05\\
1.131	9.95004747913745e-05\\
1.132	0.0001000449780472\\
1.13299999999999	9.88109555139831e-05\\
1.13499999999999	0.000103432588474426\\
1.13899999999997	0.000100300428447999\\
1.13999999999999	9.9092566825851e-05\\
1.14	0.000103375400335742\\
1.14799999999997	9.39771375727541e-05\\
1.14999999999999	9.59112257527761e-05\\
1.15	9.59112257527719e-05\\
1.15499999999999	9.4493910783785e-05\\
1.155	9.85654246813369e-05\\
1.15999999999999	9.31898041300771e-05\\
1.16	9.72009633051754e-05\\
1.16499999999999	9.19973081631107e-05\\
1.16999999999997	9.08608676341594e-05\\
1.16999999999999	9.08608676341562e-05\\
1.17	9.3867573080846e-05\\
1.17999999999997	8.43364604899047e-05\\
1.17999999999999	8.7243863824142e-05\\
1.18	8.86610425527843e-05\\
1.18899999999999	8.07141123160066e-05\\
1.189	8.34768128047827e-05\\
1.18999999999999	8.5298211491106e-05\\
1.19	8.65981913639207e-05\\
1.191	8.57575672451313e-05\\
1.19199999999999	8.97499542595332e-05\\
1.19399999999999	8.92079067997481e-05\\
1.19799999999997	8.59949145240048e-05\\
1.19999999999999	8.60102935735402e-05\\
1.2	8.60102935735364e-05\\
1.20799999999997	8.14967801510176e-05\\
1.20999999999999	8.14624083365286e-05\\
1.21	8.19104416280746e-05\\
1.21799999999997	7.64073807117994e-05\\
1.218	7.85008275397545e-05\\
1.21999999999999	7.71854991805946e-05\\
1.22	7.90362502182719e-05\\
1.22199999999999	7.77445739580717e-05\\
1.22399999999997	7.7471208771045e-05\\
1.22499999999999	7.68457668218112e-05\\
1.225	7.77672255473694e-05\\
1.22899999999997	7.61837921443637e-05\\
1.22999999999999	7.5860502571711e-05\\
1.23	7.58605025717064e-05\\
1.23399999999997	7.40558466695935e-05\\
1.23799999999994	7.18117049614448e-05\\
1.23999999999999	7.12229191123914e-05\\
1.24	7.16973774508718e-05\\
1.24699999999999	6.89780251533949e-05\\
1.247	6.89780251533969e-05\\
1.24999999999999	6.75529549591605e-05\\
1.25	6.89797843789193e-05\\
1.25299999999999	6.79135992356726e-05\\
1.25599999999997	6.66219727125938e-05\\
1.25999999999999	6.59408813091594e-05\\
1.26	6.61204226600382e-05\\
1.26599999999997	6.39074466896503e-05\\
1.26999999999999	6.26564766757883e-05\\
1.27	6.26564766757848e-05\\
1.27599999999997	6.08901927529182e-05\\
1.276	6.09607407458086e-05\\
1.27999999999999	5.99355260247306e-05\\
1.28	6.0119109405585e-05\\
1.28399999999999	5.90630871942425e-05\\
1.28799999999997	5.80356136325605e-05\\
1.28999999999999	5.76054270772476e-05\\
1.29	5.75660427756698e-05\\
1.295	5.64029926943679e-05\\
1.29500000000001	5.63751020730867e-05\\
1.3	5.53037199068777e-05\\
1.30000000000001	5.52720465254657e-05\\
1.305	5.42244394366961e-05\\
1.30500000000001	5.41505874893601e-05\\
1.31	5.31602523747297e-05\\
1.31000000000001	5.29872724524968e-05\\
1.315	5.19706999239101e-05\\
1.31999999999998	5.10237500595174e-05\\
1.32	5.08084744103407e-05\\
1.32999999999997	4.87239578596394e-05\\
1.33	4.85864890177893e-05\\
1.33399999999999	4.78643040954415e-05\\
1.334	4.72283201744427e-05\\
1.33799999999999	4.63193921491361e-05\\
1.33999999999999	4.5952815299945e-05\\
1.34	4.52151526948254e-05\\
1.34399999999999	4.43848195447947e-05\\
1.34799999999997	4.36389004269245e-05\\
1.34999999999999	4.31710796014096e-05\\
1.35	4.31710796014056e-05\\
1.35799999999997	4.15537437379747e-05\\
1.35999999999999	4.10747879983683e-05\\
1.36	4.08913340836316e-05\\
1.36299999999999	3.99402852602438e-05\\
1.363	3.97570526805175e-05\\
1.36499999999999	3.9365310199263e-05\\
1.365	3.86622985708681e-05\\
1.36699999999999	3.81022160180834e-05\\
1.36899999999997	3.77068258327723e-05\\
1.36999999999999	3.68780019614505e-05\\
1.37	3.68780019614456e-05\\
1.37399999999997	3.60792866021865e-05\\
1.37799999999995	3.51232266424487e-05\\
1.37999999999999	3.47161973240972e-05\\
1.38	3.45708547299961e-05\\
1.38799999999995	3.2776288076197e-05\\
1.38999999999999	3.22869073139033e-05\\
1.39	3.22869073138993e-05\\
1.39199999999999	3.159415695078e-05\\
1.392	3.15286840245206e-05\\
1.39399999999998	3.11033386648162e-05\\
1.39599999999997	3.04214352741847e-05\\
1.39999999999994	2.93151666008621e-05\\
1.39999999999999	2.93151666008504e-05\\
1.4	2.93151666008457e-05\\
1.40799999999994	2.74880221250415e-05\\
1.41	2.70692860046328e-05\\
1.41000000000001	2.69600660726191e-05\\
1.41799999999995	2.52163977192575e-05\\
1.41999999999999	2.46684955998555e-05\\
1.42	2.4668495599852e-05\\
1.42099999999999	2.43340615721643e-05\\
1.421	2.43340615721609e-05\\
1.422	2.4011882513137e-05\\
1.42299999999999	2.37018942411087e-05\\
1.42499999999998	2.31765823346091e-05\\
1.42899999999997	2.23983914804341e-05\\
1.42999999999999	2.20464049515036e-05\\
1.43	2.19659787723242e-05\\
1.43499999999999	2.10187551899269e-05\\
1.435	2.09365734650681e-05\\
1.43999999999998	2.00179033300902e-05\\
1.44	1.99886912306983e-05\\
1.44499999999999	1.90974490507672e-05\\
1.44999999999997	1.8223762357914e-05\\
1.44999999999998	1.8235406403042e-05\\
1.45	1.82488898890998e-05\\
1.45999999999997	1.65938783837666e-05\\
1.45999999999998	1.66499552373727e-05\\
1.46	1.66499552373704e-05\\
1.46999999999997	1.51030218254012e-05\\
1.46999999999998	1.5103021825399e-05\\
1.47	1.5119128874586e-05\\
1.47899999999998	1.38406699063807e-05\\
1.479	1.38770141141359e-05\\
1.47999999999999	1.37417134732625e-05\\
1.48	1.38243453102972e-05\\
1.481	1.37134814435127e-05\\
1.48199999999999	1.35815771002439e-05\\
1.48399999999999	1.34221116590081e-05\\
1.48799999999997	1.29515845496758e-05\\
1.48999999999999	1.27127968439897e-05\\
1.49	1.28288554280659e-05\\
1.49799999999997	1.20684222212427e-05\\
1.49999999999999	1.19318768731482e-05\\
1.5	1.19318768731472e-05\\
1.50499999999999	1.17325955625073e-05\\
1.505	1.18094061218284e-05\\
1.50799999999998	1.15251313523097e-05\\
1.508	1.16823640739601e-05\\
1.50999999999999	1.14980300785226e-05\\
1.51	1.16601813968632e-05\\
1.51199999999999	1.15629108552272e-05\\
1.51399999999997	1.13867490219768e-05\\
1.51799999999994	1.12162621135592e-05\\
1.51999999999999	1.12271142297982e-05\\
1.52	1.13166297608904e-05\\
1.52799999999994	1.06990273919781e-05\\
1.52999999999999	1.07373726255431e-05\\
1.53	1.07373726255428e-05\\
1.53699999999998	1.04332904190662e-05\\
1.537	1.05259642029287e-05\\
1.53999999999999	1.03222607422275e-05\\
1.54	1.0461125136215e-05\\
1.54299999999999	1.02610036335543e-05\\
1.54599999999997	1.01568810338685e-05\\
1.54999999999999	9.89999561087379e-06\\
1.55	9.9924651214165e-06\\
1.55599999999997	9.71050653864957e-06\\
1.56	9.51411756976403e-06\\
1.56000000000001	9.51411756976383e-06\\
1.56599999999999	9.34116545700489e-06\\
1.566	9.52629578438699e-06\\
1.57	9.38234363344871e-06\\
1.57000000000001	9.38234363344853e-06\\
1.57400000000001	9.5036667853614e-06\\
1.57499999999999	9.52978900979095e-06\\
1.575	9.52978900979072e-06\\
1.579	9.37897727191345e-06\\
1.57999999999999	9.32067289008751e-06\\
1.58	9.401987568367e-06\\
1.584	9.24892383628583e-06\\
1.588	9.17117790576031e-06\\
1.59	9.05540494935035e-06\\
1.59000000000001	9.05540494935007e-06\\
1.59499999999998	9.05519439404836e-06\\
1.595	9.12337841819294e-06\\
1.59999999999997	8.83471933517247e-06\\
1.6	8.90132548467627e-06\\
1.60000000000001	8.90132548467592e-06\\
1.60499999999998	8.68151023603294e-06\\
1.60999999999995	8.46741421746469e-06\\
1.60999999999998	8.59071244736108e-06\\
1.61	8.59071244736028e-06\\
1.61999999999994	8.08504660257074e-06\\
1.62	8.085046602568e-06\\
1.62000000000001	8.11467800028687e-06\\
1.62399999999998	7.93671739896885e-06\\
1.624	7.99429924447571e-06\\
1.62799999999997	7.90197669529319e-06\\
1.63000000000001	7.82855646742214e-06\\
1.63000000000003	7.82855646742182e-06\\
1.634	7.76041549698357e-06\\
1.63799999999997	7.56571371758409e-06\\
1.63999999999999	7.5915933608176e-06\\
1.64	7.59159336081732e-06\\
1.64499999999999	7.4670280859284e-06\\
1.645	7.56779311093803e-06\\
1.64999999999998	7.33568775329004e-06\\
1.65	7.33568775328952e-06\\
1.653	7.3661120494522e-06\\
1.65300000000001	7.36611204945181e-06\\
1.65600000000001	7.23052693832266e-06\\
1.65900000000001	7.22984702549612e-06\\
1.66	7.18546075796022e-06\\
1.66000000000002	7.18546075795978e-06\\
1.66600000000001	7.02310753786966e-06\\
1.67	6.85109894965864e-06\\
1.67000000000002	6.86113544190815e-06\\
1.67600000000001	6.60760356218024e-06\\
1.67999999999998	6.45118921612099e-06\\
1.68	6.4607788718681e-06\\
1.68199999999998	6.39696611765307e-06\\
1.682	6.43133122409538e-06\\
1.68399999999998	6.36537159619814e-06\\
1.68599999999997	6.28404031332699e-06\\
1.68999999999994	6.1774752915166e-06\\
1.68999999999998	6.18333397914238e-06\\
1.69	6.18333397914189e-06\\
1.69799999999994	5.8716587614097e-06\\
1.69999999999998	5.79330587821225e-06\\
1.7	5.79884188086006e-06\\
1.70799999999994	5.50217803390971e-06\\
1.70999999999998	5.44028009577569e-06\\
1.71	5.44524711595058e-06\\
1.711	5.40959679062413e-06\\
1.71100000000001	5.4236490714119e-06\\
1.71200000000001	5.38831546788974e-06\\
1.71300000000001	5.36196000308834e-06\\
1.715	5.29286089792969e-06\\
1.71500000000001	5.30096201291101e-06\\
1.71700000000001	5.24064314507954e-06\\
1.71900000000001	5.1874156787567e-06\\
1.71999999999998	5.16042256533643e-06\\
1.72	5.16042256533601e-06\\
1.724	5.049832333731e-06\\
1.72799999999999	4.92507669303429e-06\\
1.72999999999998	4.86117685168015e-06\\
1.73	4.87243721844843e-06\\
1.73799999999999	4.61974529798777e-06\\
1.74	4.56075697870152e-06\\
1.74000000000001	4.56075697870109e-06\\
1.74800000000001	4.31575547067129e-06\\
1.74999999999998	4.25629117978966e-06\\
1.75	4.25769028779368e-06\\
1.75799999999999	4.01368792540409e-06\\
1.75999999999998	3.95255289685223e-06\\
1.76	3.95216731048528e-06\\
1.76799999999999	3.71749465481983e-06\\
1.76899999999998	3.68778544486406e-06\\
1.769	3.68778544486366e-06\\
1.76999999999998	3.65802079073325e-06\\
1.77	3.65602140390083e-06\\
1.771	3.62819856246056e-06\\
1.77199999999999	3.5974906885207e-06\\
1.77399999999998	3.54297324925883e-06\\
1.77799999999997	3.43297671546654e-06\\
1.77999999999998	3.38011419163146e-06\\
1.78	3.38011419163108e-06\\
1.78499999999998	3.24989574880882e-06\\
1.785	3.24874794152888e-06\\
1.78999999999998	3.12801699915909e-06\\
1.79	3.12339945262857e-06\\
1.79499999999998	3.00329625427681e-06\\
1.798	2.93418545452225e-06\\
1.79800000000001	2.93418545452193e-06\\
1.8	2.88508264737299e-06\\
1.80000000000001	2.88508264737267e-06\\
1.802	2.83688029337917e-06\\
1.80399999999999	2.79490243829562e-06\\
1.80799999999996	2.70794569860341e-06\\
1.81	2.6630394162249e-06\\
1.81000000000001	2.66303941622461e-06\\
1.81799999999996	2.51192346448341e-06\\
1.81999999999998	2.47081594199901e-06\\
1.82	2.47081594199874e-06\\
1.82699999999998	2.34579883880259e-06\\
1.827	2.34579883880234e-06\\
1.82999999999998	2.29342078449263e-06\\
1.83	2.29342078449239e-06\\
1.83299999999999	2.24316432004468e-06\\
1.83599999999997	2.19663945323927e-06\\
1.84	2.13966738430359e-06\\
1.84000000000001	2.13696749380524e-06\\
1.84599999999999	2.05568301026941e-06\\
1.85	2.00631643360788e-06\\
1.85000000000001	2.0063164336077e-06\\
1.85499999999998	1.94849054724857e-06\\
1.855	1.94849054724842e-06\\
1.85599999999998	1.93644801914427e-06\\
1.856	1.93644801914412e-06\\
1.85699999999999	1.92555951282441e-06\\
1.85799999999999	1.91548967508917e-06\\
1.85999999999998	1.89560625940197e-06\\
1.86	1.89547983078874e-06\\
1.86399999999999	1.85812495487816e-06\\
1.86799999999997	1.82240663882277e-06\\
1.86999999999999	1.80578982661055e-06\\
1.87	1.80674933646992e-06\\
1.87799999999997	1.74173326494241e-06\\
1.87999999999999	1.72746149715835e-06\\
1.88	1.72746149715825e-06\\
1.88499999999998	1.69216789078885e-06\\
1.885	1.69354995943885e-06\\
1.88999999999998	1.65991313704511e-06\\
1.89	1.659913137045e-06\\
1.89499999999998	1.62915867552399e-06\\
1.89999999999996	1.59963468335316e-06\\
1.89999999999998	1.60076792264086e-06\\
1.9	1.6019786053429e-06\\
1.90999999999996	1.5500431916706e-06\\
1.91	1.55004319167045e-06\\
1.91000000000001	1.55004319167038e-06\\
1.91399999999998	1.53565235214465e-06\\
1.914	1.53741263937369e-06\\
1.91799999999997	1.51743002786349e-06\\
1.91999999999999	1.51512413090576e-06\\
1.92	1.51706245504525e-06\\
1.92399999999997	1.49740813694641e-06\\
1.92499999999998	1.4933761524687e-06\\
1.925	1.49337615246864e-06\\
1.92899999999997	1.47390793533838e-06\\
1.92999999999999	1.46968813762294e-06\\
1.93	1.47170652597079e-06\\
1.93399999999997	1.45042130591637e-06\\
1.93799999999994	1.43050607510897e-06\\
1.93999999999999	1.42134634674698e-06\\
1.94	1.42240099975326e-06\\
1.94299999999998	1.40520292568008e-06\\
1.943	1.40950638398581e-06\\
1.94599999999998	1.39634472287687e-06\\
1.94899999999996	1.38060202057618e-06\\
1.95	1.37450334141941e-06\\
1.95000000000002	1.37565368453654e-06\\
1.95599999999998	1.3380622849012e-06\\
1.95999999999998	1.31318269702462e-06\\
1.96	1.31434021830313e-06\\
1.96599999999996	1.27674259185057e-06\\
1.97	1.25447611567137e-06\\
1.97000000000001	1.2556477938376e-06\\
1.97199999999998	1.24207809246724e-06\\
1.972	1.2515374084537e-06\\
1.97399999999997	1.24028578534116e-06\\
1.97599999999994	1.22656794034485e-06\\
1.97999999999988	1.20857650252546e-06\\
1.98	1.21102868588922e-06\\
1.98000000000002	1.21102868588914e-06\\
1.9879999999999	1.16415696891739e-06\\
1.99	1.15097558922265e-06\\
1.99000000000002	1.15097558922256e-06\\
1.99499999999998	1.11627452327677e-06\\
1.995	1.11627452327668e-06\\
1.99999999999997	1.0817218630694e-06\\
1.99999999999998	1.08283194707664e-06\\
2	1.08502207483947e-06\\
2.00099999999997	1.08216427237071e-06\\
2.001	1.08423656666472e-06\\
2.00199999999999	1.07712142186891e-06\\
2.00299999999999	1.07504489971116e-06\\
2.00499999999998	1.06083269525919e-06\\
2.00899999999997	1.03728296927179e-06\\
2.00999999999997	1.03018924230756e-06\\
2.01	1.03484695921036e-06\\
2.01799999999997	9.80815941899468e-07\\
2.01999999999997	9.66621468011435e-07\\
2.02	9.68391398150964e-07\\
2.02799999999997	9.12450826626987e-07\\
2.02999999999997	9.00463314268583e-07\\
2.03	9.02177799938456e-07\\
2.03799999999997	8.51548887132437e-07\\
2.03999999999997	8.39910753982265e-07\\
2.04	8.39910753982089e-07\\
2.04799999999997	7.93909634701133e-07\\
2.04999999999997	7.81030206461477e-07\\
2.05	7.81030206461305e-07\\
2.05799999999997	7.316983756331e-07\\
2.05899999999997	7.25415899810209e-07\\
2.059	7.26886264716292e-07\\
2.05999999999997	7.22057545876351e-07\\
2.06	7.24157303249043e-07\\
2.06099999999999	7.17930330933874e-07\\
2.06199999999999	7.1379998852314e-07\\
2.06399999999998	7.01676090704307e-07\\
2.06499999999997	6.9767421833583e-07\\
2.065	6.99568498740349e-07\\
2.06899999999998	6.78783130289253e-07\\
2.06999999999997	6.7314181198689e-07\\
2.07	6.76523178463948e-07\\
2.07399999999998	6.56666064613371e-07\\
2.07799999999997	6.35735800565109e-07\\
2.07999999999997	6.28034504073944e-07\\
2.08	6.28034504073809e-07\\
2.08799999999997	5.90530655801385e-07\\
2.088	5.90530655801255e-07\\
2.08999999999997	5.82081698338196e-07\\
2.09	5.82516602665306e-07\\
2.09199999999997	5.74686383209765e-07\\
2.09399999999994	5.6621285658512e-07\\
2.09799999999988	5.49860556606659e-07\\
2.09999999999997	5.43562471664941e-07\\
2.1	5.45009221171094e-07\\
2.10799999999988	5.16079148716221e-07\\
2.10999999999997	5.09152686086657e-07\\
2.11	5.10966879013624e-07\\
2.11699999999997	4.88187376059302e-07\\
2.117	4.89247212524495e-07\\
2.11999999999997	4.80171756517107e-07\\
2.12	4.81134641984949e-07\\
2.12299999999997	4.72615274280413e-07\\
2.12599999999994	4.64120793191336e-07\\
2.12999999999997	4.54020852399633e-07\\
2.13	4.54216615873609e-07\\
2.13499999999997	4.41142240609762e-07\\
2.135	4.41865886662853e-07\\
2.13999999999997	4.30076705623739e-07\\
2.14	4.30355877856905e-07\\
2.14499999999998	4.18578660073979e-07\\
2.14599999999997	4.17230266179591e-07\\
2.146	4.17420095299539e-07\\
2.14999999999997	4.08549231039688e-07\\
2.15	4.08877531550992e-07\\
2.15399999999998	4.00306134126781e-07\\
2.15799999999995	3.92194752157914e-07\\
2.15999999999997	3.88186462902338e-07\\
2.16	3.88186462902281e-07\\
2.16799999999995	3.72360917377944e-07\\
2.16999999999997	3.68506893767141e-07\\
2.17	3.68506893767086e-07\\
2.17499999999997	3.5906212556297e-07\\
2.175	3.5906983064974e-07\\
2.17999999999997	3.49813067063251e-07\\
2.18	3.49801532977574e-07\\
2.18499999999997	3.40794619357198e-07\\
2.18999999999994	3.31969111461139e-07\\
2.18999999999997	3.31931404641167e-07\\
2.19	3.31931404641117e-07\\
2.19999999999994	3.15224437093504e-07\\
2.19999999999997	3.15224437093451e-07\\
2.2	3.15143049518406e-07\\
2.20399999999997	3.08754503417861e-07\\
2.204	3.08719816954062e-07\\
2.20499999999997	3.07177772715761e-07\\
2.205	3.06937745988675e-07\\
2.206	3.05217534616724e-07\\
2.20699999999999	3.03700189544844e-07\\
2.20899999999998	3.00401205835706e-07\\
2.20999999999997	2.98723244630533e-07\\
2.21	2.9872324463049e-07\\
2.21399999999998	2.92821742329951e-07\\
2.21799999999997	2.87205552365389e-07\\
2.21999999999997	2.84369563127739e-07\\
2.22	2.84273262306901e-07\\
2.22799999999997	2.73408216911538e-07\\
2.22999999999997	2.70797203827336e-07\\
2.23	2.70797203827299e-07\\
2.23299999999997	2.66546723042378e-07\\
2.233	2.66454235549167e-07\\
2.23599999999997	2.62626546378245e-07\\
2.23899999999994	2.58752916198065e-07\\
2.23999999999997	2.57502328220878e-07\\
2.24	2.57412345507855e-07\\
2.24599999999994	2.49965374281637e-07\\
2.24999999999997	2.45120827274986e-07\\
2.25	2.45036231594096e-07\\
2.25599999999994	2.37815398116981e-07\\
2.25999999999997	2.33095579583714e-07\\
2.26	2.33095579583681e-07\\
2.26199999999997	2.30746189102876e-07\\
2.262	2.30746189102843e-07\\
2.26399999999997	2.28425668920766e-07\\
2.26599999999994	2.26133569696077e-07\\
2.26999999999988	2.21632863914395e-07\\
2.26999999999997	2.2150257166841e-07\\
2.27	2.21472180955505e-07\\
2.27499999999997	2.16081444206338e-07\\
2.275	2.15869518651073e-07\\
2.27999999999997	2.10591751619488e-07\\
2.28	2.10591751619458e-07\\
2.28499999999997	2.05288257059885e-07\\
2.28999999999995	2.00057509124878e-07\\
2.29	2.00057509124822e-07\\
2.29099999999997	1.98981377269757e-07\\
2.291	1.98986768931793e-07\\
2.29199999999999	1.97937735448239e-07\\
2.29299999999999	1.96946624921706e-07\\
2.29499999999998	1.9486282320966e-07\\
2.29899999999997	1.90623205946242e-07\\
2.29999999999997	1.89688202087745e-07\\
2.3	1.89708479663082e-07\\
2.30799999999997	1.81099845324665e-07\\
2.30999999999997	1.78943002446992e-07\\
2.31	1.78943002446962e-07\\
2.31799999999997	1.70240176374196e-07\\
2.31999999999997	1.68083014745581e-07\\
2.32	1.6813159155481e-07\\
2.32799999999997	1.59428399967331e-07\\
2.32999999999997	1.57241960912104e-07\\
2.33	1.57353457986004e-07\\
2.33799999999997	1.4860803400964e-07\\
2.33999999999997	1.46404548482266e-07\\
2.34	1.46534096157711e-07\\
2.34499999999997	1.4114796601266e-07\\
2.345	1.41186278355082e-07\\
2.34899999999997	1.36736381414031e-07\\
2.349	1.36775402366121e-07\\
2.34999999999997	1.35658903396179e-07\\
2.35	1.35698424991199e-07\\
2.35099999999999	1.3462532980398e-07\\
2.35199999999999	1.33602198261613e-07\\
2.35399999999998	1.31587109561089e-07\\
2.35799999999996	1.27317170425432e-07\\
2.35999999999997	1.25230729867755e-07\\
2.36	1.25318445913751e-07\\
2.36799999999996	1.17286740605108e-07\\
2.36999999999997	1.1544516127101e-07\\
2.37	1.15537063999802e-07\\
2.37799999999996	1.08296555836409e-07\\
2.378	1.08394874739421e-07\\
2.37999999999997	1.06609087829758e-07\\
2.38	1.07008503284635e-07\\
2.38199999999997	1.05353696816118e-07\\
2.38399999999994	1.0363096771762e-07\\
2.38799999999988	1.00636100636455e-07\\
2.38999999999997	9.90090962133669e-08\\
2.39	9.91622244479353e-08\\
2.39799999999988	9.2963731052556e-08\\
2.39999999999997	9.16439900810366e-08\\
2.4	9.17989697292146e-08\\
2.40699999999997	8.70815610726437e-08\\
2.407	8.70815610726259e-08\\
2.40999999999997	8.53423492237648e-08\\
2.41	8.53423492237476e-08\\
2.41299999999997	8.35101035576215e-08\\
2.41499999999997	8.22528870983578e-08\\
2.415	8.23590028386467e-08\\
2.41799999999997	8.05779694689841e-08\\
2.41999999999997	7.93897381364428e-08\\
2.42	7.9494796871954e-08\\
2.42299999999997	7.78775130099947e-08\\
2.42599999999994	7.6339549261611e-08\\
2.42999999999997	7.42738004913291e-08\\
2.42999999999999	7.44793491390141e-08\\
2.43599999999994	7.16414022657234e-08\\
2.43599999999997	7.18445594966933e-08\\
2.436	7.20454845966234e-08\\
2.43999999999997	7.04248156975268e-08\\
2.43999999999999	7.04248156975165e-08\\
2.44399999999996	6.92231325797469e-08\\
2.44799999999993	6.78641727372824e-08\\
2.44999999999999	6.71891520320638e-08\\
2.45000000000002	6.74094557244328e-08\\
2.45799999999996	6.49102010590207e-08\\
2.45999999999997	6.45363459975563e-08\\
2.46	6.45363459975496e-08\\
2.46499999999997	6.3348239706791e-08\\
2.465	6.3536836327393e-08\\
2.46999999999996	6.22503530389162e-08\\
2.47	6.22503530389077e-08\\
2.47499999999997	6.12413409397504e-08\\
2.47999999999993	6.01513318307591e-08\\
2.48	6.03202156510645e-08\\
2.48000000000003	6.03202156510597e-08\\
2.48499999999997	5.94143935833495e-08\\
2.485	5.94143935833445e-08\\
2.48999999999994	5.84252329917555e-08\\
2.48999999999999	5.84252329917476e-08\\
2.49000000000003	5.84560935627243e-08\\
2.49399999999997	5.76794882360682e-08\\
2.494	5.77394287059677e-08\\
2.49799999999993	5.69692480388058e-08\\
2.49999999999997	5.65720493906759e-08\\
2.5	5.66863096859567e-08\\
2.50399999999994	5.60054729476887e-08\\
2.50799999999988	5.52741549346378e-08\\
2.50999999999997	5.48843702648443e-08\\
2.51	5.50775087875815e-08\\
2.51799999999988	5.35369999225792e-08\\
2.51999999999997	5.31317466528196e-08\\
2.52	5.31948687317533e-08\\
2.52299999999997	5.25798483969362e-08\\
2.523	5.26393019245597e-08\\
2.52599999999996	5.20712829000922e-08\\
2.52899999999993	5.14561662902516e-08\\
2.52999999999997	5.12431275258776e-08\\
2.53	5.13415256848365e-08\\
2.53599999999993	5.0098518671813e-08\\
2.53999999999997	4.92092548804656e-08\\
2.54	4.92483683068383e-08\\
2.54599999999993	4.78795034879073e-08\\
2.54999999999997	4.69789485388419e-08\\
2.55	4.70133949368337e-08\\
2.55199999999997	4.65512880459159e-08\\
2.552	4.65512880459095e-08\\
2.55399999999996	4.61001923162573e-08\\
2.55499999999997	4.58641002534687e-08\\
2.555	4.58847295966822e-08\\
2.55699999999997	4.54326619618511e-08\\
2.55899999999994	4.49982349157598e-08\\
2.55999999999997	4.47625938114915e-08\\
2.56	4.47625938114851e-08\\
2.56399999999994	4.386213808004e-08\\
2.56799999999987	4.29201635926521e-08\\
2.56999999999997	4.24855331971726e-08\\
2.57	4.24855331971663e-08\\
2.57799999999987	4.06295781430805e-08\\
2.57999999999997	4.01843970051992e-08\\
2.58	4.0184397005193e-08\\
2.58099999999997	3.99478755144091e-08\\
2.581	3.99530301691331e-08\\
2.58199999999999	3.97170775808764e-08\\
2.58299999999999	3.94866822914022e-08\\
2.58499999999998	3.90223632065949e-08\\
2.58899999999996	3.80994449367677e-08\\
2.58999999999997	3.78842433359192e-08\\
2.59	3.78880139300758e-08\\
2.59799999999997	3.60493094136416e-08\\
2.59999999999997	3.56182939058937e-08\\
2.6	3.56229454805595e-08\\
2.60799999999997	3.38134113997173e-08\\
2.60999999999997	3.33659945899341e-08\\
2.61	3.33659945899277e-08\\
2.61799999999996	3.16002661706653e-08\\
2.62	3.11687982786685e-08\\
2.62000000000002	3.11726479060551e-08\\
2.62499999999997	3.01102821884995e-08\\
2.625	3.01120289059459e-08\\
2.62999999999995	2.90612068907462e-08\\
2.62999999999999	2.90740321379931e-08\\
2.63000000000002	2.90769222902637e-08\\
2.63499999999997	2.80436661183202e-08\\
2.63899999999997	2.72391702089347e-08\\
2.639	2.724143906738e-08\\
2.63999999999997	2.70392917473197e-08\\
2.64	2.70449018939336e-08\\
2.641	2.6843356678357e-08\\
2.64199999999999	2.66470268672122e-08\\
2.64399999999999	2.62468582191698e-08\\
2.64799999999997	2.54543713124604e-08\\
2.64999999999997	2.50607224666593e-08\\
2.65	2.50619052931237e-08\\
2.65799999999997	2.35556393071486e-08\\
2.65999999999997	2.32005774736202e-08\\
2.66	2.32054848342237e-08\\
2.66799999999997	2.1853606067052e-08\\
2.668	2.18536060670475e-08\\
2.66999999999997	2.15447280626385e-08\\
2.67	2.15478680459944e-08\\
2.67199999999998	2.12329798376295e-08\\
2.67399999999995	2.09350372980638e-08\\
2.6779999999999	2.03199422708486e-08\\
2.67999999999997	2.00166298163801e-08\\
2.68	2.00204735525857e-08\\
2.6879999999999	1.88340765058389e-08\\
2.68999999999997	1.85478863975056e-08\\
2.69	1.85520167144526e-08\\
2.69499999999997	1.78458850619013e-08\\
2.695	1.78504513929572e-08\\
2.69699999999997	1.75686197456594e-08\\
2.697	1.75883653523525e-08\\
2.69899999999996	1.73087928699073e-08\\
2.69999999999997	1.71698366741817e-08\\
2.7	1.71751358888932e-08\\
2.70199999999997	1.68988495437339e-08\\
2.70399999999993	1.66301131995832e-08\\
2.70799999999986	1.60934967532602e-08\\
2.70999999999997	1.58339593731784e-08\\
2.71	1.58339593731747e-08\\
2.71799999999986	1.48167729547134e-08\\
2.71999999999997	1.45733446536112e-08\\
2.72	1.45825188337992e-08\\
2.72599999999997	1.38891596003749e-08\\
2.726	1.38891596003718e-08\\
2.72999999999997	1.34574930222891e-08\\
2.73	1.34641212617981e-08\\
2.73399999999998	1.30424287611126e-08\\
2.73799999999995	1.26501499261913e-08\\
2.73999999999997	1.24662853844662e-08\\
2.74	1.24799821939459e-08\\
2.74799999999995	1.17459142494439e-08\\
2.74999999999997	1.15933011713692e-08\\
2.75	1.15933011713668e-08\\
2.75499999999997	1.11768175556135e-08\\
2.755	1.11909508095234e-08\\
2.75999999999996	1.07986349315083e-08\\
2.76	1.08129620308135e-08\\
2.76499999999997	1.04588654293103e-08\\
2.765	1.04662119769296e-08\\
2.76500000000003	1.04662119769277e-08\\
2.77	1.01431974695332e-08\\
2.77000000000003	1.01431974695314e-08\\
2.77499999999999	9.83568821330163e-09\\
2.77999999999996	9.53552437866128e-09\\
2.78	9.55055816930524e-09\\
2.78399999999997	9.33032649872619e-09\\
2.784	9.33032649872475e-09\\
2.78799999999996	9.1357857708588e-09\\
2.78999999999997	9.04012634280062e-09\\
2.79	9.04012634279934e-09\\
2.79399999999997	8.86693143333926e-09\\
2.79799999999993	8.70779188325397e-09\\
2.79999999999997	8.63345344724837e-09\\
2.8	8.63345344724734e-09\\
2.80799999999993	8.35899808087815e-09\\
2.80999999999997	8.29432436636614e-09\\
2.81	8.31210548778097e-09\\
2.81299999999997	8.21908737581119e-09\\
2.813	8.23286046163502e-09\\
2.81599999999996	8.14270506733686e-09\\
2.81899999999993	8.06875582496802e-09\\
2.81999999999997	8.04681432316839e-09\\
2.82	8.04681432316768e-09\\
2.82599999999993	7.89511086842428e-09\\
2.82999999999997	7.8038999444265e-09\\
2.83	7.80999329162903e-09\\
2.83499999999997	7.68703749891894e-09\\
2.835	7.69896313021925e-09\\
2.83999999999997	7.58322830023235e-09\\
2.84	7.59481181051463e-09\\
2.84199999999997	7.5561733270229e-09\\
2.842	7.55617332702236e-09\\
2.84399999999996	7.52133702355521e-09\\
2.84599999999993	7.47927221718486e-09\\
2.84999999999985	7.40661787517369e-09\\
2.85	7.41460107440901e-09\\
2.85000000000003	7.41722307075165e-09\\
2.85799999999989	7.2612093013127e-09\\
2.85999999999997	7.23788982495774e-09\\
2.86	7.24773723081501e-09\\
2.86799999999986	7.09432239589427e-09\\
2.86999999999997	7.07021168813421e-09\\
2.87	7.07891081931485e-09\\
2.87099999999997	7.05976701679024e-09\\
2.871	7.06393208894427e-09\\
2.87199999999999	7.04454936132425e-09\\
2.87299999999999	7.02896993167199e-09\\
2.87499999999998	6.99097736628876e-09\\
2.87899999999996	6.90811838934426e-09\\
2.87999999999997	6.8906508774175e-09\\
2.88	6.89439271780808e-09\\
2.88799999999997	6.71656960039017e-09\\
2.88999999999997	6.66911271881105e-09\\
2.89	6.67612107544106e-09\\
2.89799999999997	6.48237455333603e-09\\
2.89999999999997	6.4327344645399e-09\\
2.9	6.43273446453919e-09\\
2.90499999999997	6.2991036767899e-09\\
2.905	6.29910367678914e-09\\
2.90999999999998	6.15842809585182e-09\\
2.91000000000001	6.16124930647704e-09\\
2.91499999999999	6.01846692442477e-09\\
2.91999999999997	5.86786021066014e-09\\
2.92	5.86786021065914e-09\\
2.92899999999997	5.591501116183e-09\\
2.92899999999999	5.59150111618214e-09\\
2.92999999999997	5.5613915916647e-09\\
2.93	5.56267970146911e-09\\
2.93099999999999	5.53367636442505e-09\\
2.93199999999999	5.50689119325218e-09\\
2.93399999999997	5.44440286620043e-09\\
2.93799999999994	5.31557265739609e-09\\
2.94	5.25924030358337e-09\\
2.94000000000003	5.26130167178609e-09\\
2.94799999999997	5.00071118017378e-09\\
2.95	4.94346803487472e-09\\
2.95000000000003	4.94519086007012e-09\\
2.95799999999997	4.68686595617988e-09\\
2.958	4.6930850591044e-09\\
2.96	4.63006434175163e-09\\
2.96000000000003	4.63006434175073e-09\\
2.96200000000003	4.56901298609173e-09\\
2.96400000000003	4.50467867800123e-09\\
2.96800000000003	4.37922137572297e-09\\
2.97	4.31490448406163e-09\\
2.97000000000003	4.31793176331663e-09\\
2.97499999999997	4.16119391685193e-09\\
2.975	4.16119391685106e-09\\
2.97999999999995	4.00607318838779e-09\\
2.98	4.00607318838608e-09\\
2.98499999999995	3.85412891011947e-09\\
2.98699999999997	3.79453181556417e-09\\
2.98699999999999	3.79562864741016e-09\\
2.98999999999997	3.70840188925731e-09\\
2.99	3.70943626202981e-09\\
2.99299999999998	3.62109090153924e-09\\
2.99599999999996	3.53765413434178e-09\\
2.99999999999997	3.42307489441324e-09\\
3	3.42307489441244e-09\\
3.00599999999996	3.25388619223171e-09\\
3.00999999999997	3.14259970149029e-09\\
3.01	3.14306281517123e-09\\
3.01599999999996	2.98325766874741e-09\\
3.01599999999999	2.98417072224396e-09\\
3.01600000000002	2.98596597985128e-09\\
3.01999999999997	2.88529046484884e-09\\
3.02	2.88617302083562e-09\\
3.02399999999995	2.79033943245301e-09\\
3.0279999999999	2.70014537080264e-09\\
3.02999999999997	2.65724015069851e-09\\
3.03	2.65893807205871e-09\\
3.0379999999999	2.49626971957998e-09\\
3.03999999999997	2.45824621561384e-09\\
3.04	2.46148737236982e-09\\
3.04499999999997	2.37152457067933e-09\\
3.04499999999999	2.37152457067884e-09\\
3.04999999999996	2.28842592084135e-09\\
3.05	2.28842592084067e-09\\
3.05499999999997	2.20978115773398e-09\\
3.05999999999993	2.13590997926124e-09\\
3.05999999999997	2.13665345712919e-09\\
3.06	2.13702183719693e-09\\
3.06999999999993	2.00550255730652e-09\\
3.06999999999997	2.00623268996489e-09\\
3.07	2.00695410506334e-09\\
3.07399999999997	1.9610120536991e-09\\
3.07399999999999	1.96170815782995e-09\\
3.07799999999996	1.91630042091985e-09\\
3.07999999999997	1.89669017081367e-09\\
3.08	1.89669017081337e-09\\
3.08399999999997	1.85513265096131e-09\\
3.08799999999993	1.81445131945042e-09\\
3.08999999999997	1.79577517421016e-09\\
3.09	1.79673650678033e-09\\
3.09799999999993	1.72263951008837e-09\\
3.09999999999997	1.70648523103022e-09\\
3.1	1.70648523102999e-09\\
3.10299999999997	1.68215160536701e-09\\
3.10299999999999	1.68336853013029e-09\\
3.10599999999996	1.65883498926733e-09\\
3.10899999999993	1.63412773520376e-09\\
3.10999999999997	1.62830791434472e-09\\
3.11	1.62886742422617e-09\\
3.11499999999997	1.58883307067412e-09\\
3.115	1.59019590514722e-09\\
3.11999999999998	1.55125825293439e-09\\
3.12000000000001	1.55257107242439e-09\\
3.12499999999998	1.51468241804038e-09\\
3.12999999999996	1.47779634314532e-09\\
3.13000000000001	1.47906075602974e-09\\
3.13199999999999	1.4645768852561e-09\\
3.13200000000002	1.4645768852559e-09\\
3.13400000000001	1.45146090062443e-09\\
3.136	1.43727488151977e-09\\
3.13999999999997	1.40933431226969e-09\\
3.14	1.41050650172881e-09\\
3.14000000000003	1.41050650172862e-09\\
3.14799999999998	1.35781414123319e-09\\
3.14999999999998	1.34481207449375e-09\\
3.15	1.34481207449358e-09\\
3.15799999999995	1.29543190983e-09\\
3.15999999999997	1.28328860434385e-09\\
3.16	1.28415604294519e-09\\
3.16099999999997	1.27857320868231e-09\\
3.16099999999999	1.27857320868215e-09\\
3.16199999999999	1.27344515258799e-09\\
3.16299999999998	1.2683407930765e-09\\
3.16499999999997	1.25701023768541e-09\\
3.16899999999994	1.23401097924271e-09\\
3.16999999999997	1.22991678306797e-09\\
3.17	1.23028886230898e-09\\
3.17799999999994	1.18579031132041e-09\\
3.17999999999997	1.17538002158238e-09\\
3.18	1.17538002158223e-09\\
3.185	1.14879358123262e-09\\
3.18500000000003	1.14930859363524e-09\\
3.18999999999997	1.12272312203111e-09\\
3.18999999999999	1.12272312203096e-09\\
3.19499999999993	1.09727534960684e-09\\
3.19999999999986	1.07176780290919e-09\\
3.19999999999999	1.07176780290852e-09\\
3.20000000000002	1.07221571385143e-09\\
3.20999999999989	1.02072723653847e-09\\
3.21000000000002	1.02101953232925e-09\\
3.21000000000005	1.02101953232911e-09\\
3.21899999999997	9.76439048436529e-10\\
3.21899999999999	9.76724714274348e-10\\
3.22	9.72008661769153e-10\\
3.22000000000003	9.72008661769019e-10\\
3.22100000000004	9.67456514706732e-10\\
3.22200000000005	9.62647972862646e-10\\
3.22400000000006	9.53380978214313e-10\\
3.22800000000009	9.34849264536218e-10\\
3.23000000000003	9.26124085764059e-10\\
3.23000000000006	9.26124085763963e-10\\
3.23800000000012	8.89939631679775e-10\\
3.23999999999997	8.82149324767989e-10\\
3.24	8.82403744444209e-10\\
3.24799999999997	8.47672731088401e-10\\
3.24799999999999	8.486452472502e-10\\
3.24999999999997	8.4032683390253e-10\\
3.25	8.40326833902412e-10\\
3.25199999999998	8.31941178793142e-10\\
3.25399999999996	8.23490221850806e-10\\
3.25499999999998	8.19394805934222e-10\\
3.255	8.19506068387205e-10\\
3.25899999999996	8.03006181028703e-10\\
3.25999999999997	7.99277517761516e-10\\
3.26	7.99380606451371e-10\\
3.26399999999996	7.82722049355997e-10\\
3.26799999999992	7.66700159993464e-10\\
3.26999999999997	7.58422779304026e-10\\
3.27	7.58422779303908e-10\\
3.27699999999999	7.28556292804906e-10\\
3.27700000000002	7.28556292804786e-10\\
3.27999999999997	7.15633104135348e-10\\
3.28	7.15722610040181e-10\\
3.28299999999995	7.02774257693954e-10\\
3.2859999999999	6.89879653485633e-10\\
3.28999999999997	6.72111846375638e-10\\
3.29	6.72111846375513e-10\\
3.2959999999999	6.4531080598281e-10\\
3.29999999999997	6.26841550718844e-10\\
3.3	6.27257603751748e-10\\
3.3059999999999	5.9977381801348e-10\\
3.30599999999995	6.00180407511126e-10\\
3.30599999999999	6.00578031596568e-10\\
3.30999999999997	5.82707092518514e-10\\
3.31	5.83383846029435e-10\\
3.31399999999998	5.65857786910495e-10\\
3.31799999999996	5.4873427126169e-10\\
3.31999999999997	5.4025676496927e-10\\
3.32	5.4032742494401e-10\\
3.32499999999998	5.1954013120305e-10\\
3.325	5.19677757665349e-10\\
3.32999999999998	4.99547468056951e-10\\
3.33000000000001	4.99676397915734e-10\\
3.33499999999998	4.79877772887367e-10\\
3.33500000000001	4.80379371686635e-10\\
3.33999999999998	4.6144027250759e-10\\
3.34000000000001	4.61440272507485e-10\\
3.34499999999998	4.43573714554828e-10\\
3.34999999999996	4.25915858624114e-10\\
3.35000000000001	4.25915858623935e-10\\
3.35999999999996	3.92705193704675e-10\\
3.36	3.92705193704531e-10\\
3.36399999999997	3.80203763301805e-10\\
3.36399999999999	3.80320400624147e-10\\
3.36799999999996	3.68324652817614e-10\\
3.36999999999997	3.62469719859354e-10\\
3.37	3.62469719859271e-10\\
3.37399999999997	3.51240869544224e-10\\
3.37799999999993	3.40387724748674e-10\\
3.37999999999997	3.3537287872934e-10\\
3.38	3.35372878729269e-10\\
3.38799999999993	3.1635879722904e-10\\
3.38999999999997	3.11974056666261e-10\\
3.39	3.12084018488378e-10\\
3.39299999999997	3.05909858632887e-10\\
3.39299999999999	3.06072158417042e-10\\
3.39499999999998	3.02077935395706e-10\\
3.395	3.02238561721822e-10\\
3.39699999999999	2.98399148944001e-10\\
3.39899999999997	2.94872817578966e-10\\
3.39999999999997	2.93245175993363e-10\\
3.4	2.93452900371856e-10\\
3.40399999999997	2.86582138171111e-10\\
3.40799999999993	2.80397148161362e-10\\
3.40999999999997	2.77489677524096e-10\\
3.41	2.77489677524054e-10\\
3.41799999999993	2.66295423895717e-10\\
3.41999999999997	2.6365696468111e-10\\
3.42	2.63656964681074e-10\\
3.42199999999997	2.6128989739447e-10\\
3.42199999999999	2.61289897394436e-10\\
3.42399999999996	2.59007992842137e-10\\
3.42599999999992	2.56810906574378e-10\\
3.42999999999985	2.52428791878391e-10\\
3.42999999999997	2.52428791878259e-10\\
3.43	2.52669878892296e-10\\
3.43799999999986	2.44801768747224e-10\\
3.43999999999997	2.4320831985778e-10\\
3.44	2.43208319857757e-10\\
3.44799999999986	2.36956816494954e-10\\
3.44999999999997	2.35522208082338e-10\\
3.45	2.35522208082319e-10\\
3.45099999999996	2.34949909463547e-10\\
3.45099999999999	2.34985274473024e-10\\
3.45199999999999	2.34281446012718e-10\\
3.45299999999998	2.33729025279239e-10\\
3.45499999999996	2.32519072747149e-10\\
3.45899999999993	2.30046660845332e-10\\
3.45999999999997	2.29443365741632e-10\\
3.46	2.29692004499351e-10\\
3.465	2.26717144886664e-10\\
3.46500000000003	2.26717144886647e-10\\
3.46999999999997	2.23698026201988e-10\\
3.47	2.2369802620197e-10\\
3.47499999999994	2.20570943567866e-10\\
3.47999999999988	2.17281974817165e-10\\
3.47999999999994	2.17281974817127e-10\\
3.47999999999999	2.17380687945284e-10\\
3.48999999999987	2.1045784988457e-10\\
3.48999999999996	2.10502451377288e-10\\
3.48999999999999	2.10502451377268e-10\\
3.49999999999987	2.0309560500259e-10\\
3.49999999999996	2.03095605002519e-10\\
3.49999999999999	2.0317651957077e-10\\
3.50899999999999	1.96096844745668e-10\\
3.50900000000002	1.96096844745646e-10\\
3.50999999999999	1.95343936548272e-10\\
3.51000000000002	1.9534393654825e-10\\
3.51100000000001	1.94585258360438e-10\\
3.512	1.93820794795911e-10\\
3.51399999999999	1.9218940659152e-10\\
3.51799999999996	1.88841181144476e-10\\
3.51999999999997	1.87226300683801e-10\\
3.52	1.87226300683777e-10\\
3.52799999999994	1.80327158311086e-10\\
3.52999999999997	1.78535600410454e-10\\
3.53	1.78611129302068e-10\\
3.53499999999998	1.74160843317896e-10\\
3.535	1.74160843317871e-10\\
3.53799999999997	1.7142992154446e-10\\
3.53799999999999	1.71456695391797e-10\\
3.53999999999997	1.6962763700807e-10\\
3.54	1.69653616508894e-10\\
3.54199999999998	1.6784257522118e-10\\
3.54399999999996	1.66046842957534e-10\\
3.54799999999992	1.62335267510553e-10\\
3.54999999999997	1.6046672915035e-10\\
3.55	1.60553529766104e-10\\
3.55799999999992	1.53005600724457e-10\\
3.56	1.51088422836295e-10\\
3.56000000000003	1.5110746400314e-10\\
3.56699999999996	1.44309434121331e-10\\
3.56699999999999	1.44327955179407e-10\\
3.56999999999998	1.41388030536439e-10\\
3.57000000000001	1.41423655341512e-10\\
3.57299999999999	1.3847703201995e-10\\
3.57599999999997	1.35580890699026e-10\\
3.57999999999997	1.31804424814727e-10\\
3.58	1.31804424814701e-10\\
3.58599999999997	1.26292837781032e-10\\
3.58999999999997	1.22722089595587e-10\\
3.59	1.22739079381715e-10\\
3.59599999999997	1.17547462011226e-10\\
3.596	1.17555593205125e-10\\
3.6	1.14164557183194e-10\\
3.60000000000003	1.14227306483665e-10\\
3.60400000000003	1.10936900216726e-10\\
3.60499999999998	1.10125933037238e-10\\
3.605	1.10162399928979e-10\\
3.60900000000001	1.06978939017308e-10\\
3.61	1.06232711089678e-10\\
3.61000000000003	1.06232711089656e-10\\
3.61400000000003	1.03200394157783e-10\\
3.61800000000004	1.00258199105652e-10\\
3.61999999999997	9.88050946025037e-11\\
3.62	9.88659169113751e-11\\
3.62499999999999	9.52863557520639e-11\\
3.62500000000002	9.53407170157264e-11\\
3.62999999999998	9.18092195096421e-11\\
3.63000000000001	9.1815247376558e-11\\
3.63499999999997	8.83274869739464e-11\\
3.63999999999993	8.48852172280294e-11\\
3.63999999999997	8.48912426482359e-11\\
3.64	8.49032914426572e-11\\
3.64999999999993	7.81323339090522e-11\\
3.64999999999997	7.81443875081106e-11\\
3.65	7.81443875080916e-11\\
3.65399999999996	7.55066662677143e-11\\
3.65399999999999	7.55127185876852e-11\\
3.65799999999995	7.28418149784948e-11\\
3.65999999999997	7.1517430548426e-11\\
3.66	7.15174305484073e-11\\
3.66399999999996	6.8915636202064e-11\\
3.66799999999993	6.6412385534095e-11\\
3.66999999999997	6.520615904442e-11\\
3.67	6.52303055904884e-11\\
3.67499999999998	6.2288600203794e-11\\
3.67500000000001	6.22886002037778e-11\\
3.67999999999998	5.95047754194664e-11\\
3.68000000000001	5.95047754194511e-11\\
3.68299999999996	5.79078642882956e-11\\
3.68299999999999	5.79199303512155e-11\\
3.68599999999995	5.63675353118064e-11\\
3.68899999999991	5.48589658754249e-11\\
3.68999999999998	5.43912245500494e-11\\
3.69000000000001	5.43972673067701e-11\\
3.69599999999992	5.15631209224276e-11\\
3.69999999999997	4.98082015843964e-11\\
3.7	4.98317804125632e-11\\
3.70599999999992	4.73374106982196e-11\\
3.71	4.57730363473724e-11\\
3.71000000000003	4.58130960083433e-11\\
3.71199999999996	4.50623338012933e-11\\
3.71199999999999	4.50791826382729e-11\\
3.71399999999993	4.43491728474529e-11\\
3.71599999999986	4.3656440241394e-11\\
3.71999999999973	4.2315409361121e-11\\
3.71999999999997	4.23371441354197e-11\\
3.72	4.23371441354105e-11\\
3.72799999999974	3.98956400963205e-11\\
3.72999999999997	3.93321250365492e-11\\
3.73	3.93321250365414e-11\\
3.73799999999974	3.72812404780849e-11\\
3.73999999999997	3.6817083484269e-11\\
3.74	3.68325707143938e-11\\
3.74099999999996	3.66235566656168e-11\\
3.74099999999999	3.66235566656107e-11\\
3.74199999999998	3.64044790033139e-11\\
3.74299999999998	3.62105727696379e-11\\
3.74499999999996	3.57982795493107e-11\\
3.74500000000001	3.58178851807514e-11\\
3.74899999999997	3.50742595436951e-11\\
3.74999999999997	3.49058326846195e-11\\
3.75	3.49058326846146e-11\\
3.75399999999997	3.42682084627144e-11\\
3.75799999999994	3.36886715857118e-11\\
3.75999999999997	3.34511965211379e-11\\
3.76	3.34511965211344e-11\\
3.76799999999994	3.25420162394281e-11\\
3.76999999999996	3.23303801021294e-11\\
3.76999999999999	3.23503524458014e-11\\
3.77799999999993	3.15861755142124e-11\\
3.77999999999996	3.14345285959178e-11\\
3.77999999999999	3.14525647997991e-11\\
3.78799999999993	3.08506217384685e-11\\
3.78999999999996	3.07372441499201e-11\\
3.78999999999999	3.07372441499184e-11\\
3.79799999999993	3.0310865878868e-11\\
3.79899999999996	3.02644047660823e-11\\
3.79899999999999	3.02769200395402e-11\\
3.79999999999996	3.02324221926753e-11\\
3.79999999999999	3.02444947488109e-11\\
3.80099999999998	3.02069213750939e-11\\
3.80199999999997	3.01636140408532e-11\\
3.80399999999995	3.00888348219936e-11\\
3.80799999999991	2.99290088865019e-11\\
3.80999999999996	2.9849696916231e-11\\
3.80999999999999	2.98496969162299e-11\\
3.81499999999998	2.96602452784531e-11\\
3.81500000000001	2.96649498090885e-11\\
3.81999999999999	2.94581232129213e-11\\
3.82000000000002	2.94649381041998e-11\\
3.825	2.92430722120441e-11\\
3.82799999999996	2.90985867140259e-11\\
3.82799999999999	2.90985867140245e-11\\
3.82999999999999	2.89969952647583e-11\\
3.83000000000002	2.90011254467917e-11\\
3.83200000000002	2.8891990142005e-11\\
3.83400000000002	2.87755094264199e-11\\
3.83800000000001	2.85342343384708e-11\\
3.83999999999997	2.84088974866054e-11\\
3.84	2.84123477322095e-11\\
3.84799999999999	2.78274058395651e-11\\
3.84999999999997	2.76723436981642e-11\\
3.85	2.76723436981619e-11\\
3.85699999999996	2.70729624579218e-11\\
3.85699999999999	2.70775653127605e-11\\
3.85999999999997	2.67989102388541e-11\\
3.86	2.67989102388515e-11\\
3.86299999999998	2.65108092737027e-11\\
3.86599999999997	2.62107346842794e-11\\
3.86999999999997	2.57992896485995e-11\\
3.87	2.58021319870425e-11\\
3.87599999999997	2.51549248793008e-11\\
3.87999999999997	2.4698487011041e-11\\
3.88	2.46984870110377e-11\\
3.88500000000001	2.41136820417869e-11\\
3.88500000000003	2.41161226398419e-11\\
3.88599999999996	2.39937607024747e-11\\
3.88599999999999	2.40029266296774e-11\\
3.88699999999998	2.3881580579373e-11\\
3.88799999999997	2.37569610399647e-11\\
3.88999999999995	2.3505350654033e-11\\
3.89	2.35053506540268e-11\\
3.89399999999996	2.29940259161378e-11\\
3.89799999999992	2.24775061706495e-11\\
3.9	2.2218662139291e-11\\
3.90000000000003	2.22224777990469e-11\\
3.90799999999995	2.11650492356338e-11\\
3.91	2.08964554449403e-11\\
3.91000000000003	2.09008318977176e-11\\
3.91499999999996	2.02217603771695e-11\\
3.91499999999999	2.02257850116966e-11\\
3.91999999999993	1.95352076876042e-11\\
3.92	1.95352076875936e-11\\
3.92000000000003	1.95359708377277e-11\\
3.92499999999997	1.88438645020696e-11\\
3.9299999999999	1.81609583604297e-11\\
3.93000000000003	1.81617087821701e-11\\
3.93000000000006	1.81631981895294e-11\\
3.93999999999993	1.68208082057545e-11\\
3.93999999999998	1.6821532619614e-11\\
3.94	1.68215326196102e-11\\
3.94399999999996	1.62961150882146e-11\\
3.94399999999999	1.62974988742599e-11\\
3.94799999999995	1.57698096530559e-11\\
3.94999999999998	1.55125186508496e-11\\
3.95	1.55138405710342e-11\\
3.95399999999996	1.499047327562e-11\\
3.95499999999998	1.4865131047048e-11\\
3.95500000000001	1.48657494783321e-11\\
3.95899999999997	1.43452781720951e-11\\
3.96	1.42160804104608e-11\\
3.96000000000003	1.42160804104572e-11\\
3.96399999999999	1.37083788356481e-11\\
3.96799999999995	1.32218339321993e-11\\
3.97	1.29872711273018e-11\\
3.97000000000003	1.29896124093418e-11\\
3.97299999999996	1.26467571015341e-11\\
3.97299999999999	1.26467571015309e-11\\
3.97599999999992	1.23187330165689e-11\\
3.97899999999986	1.19985867261493e-11\\
3.97999999999997	1.18939758525932e-11\\
3.98	1.1896819008353e-11\\
3.98599999999987	1.1294872537962e-11\\
3.98999999999997	1.09194206114916e-11\\
3.99	1.0919420611489e-11\\
3.99599999999987	1.03892533554969e-11\\
3.99999999999997	1.00571391562503e-11\\
4	1.00582817566149e-11\\
4.00199999999993	9.89989756015037e-12\\
4.00199999999999	9.9021929311922e-12\\
4.00399999999992	9.74812597134987e-12\\
4.00599999999986	9.59718754680591e-12\\
4.00999999999972	9.31266817068434e-12\\
4.00999999999995	9.31266817066859e-12\\
4.01	9.31266817066462e-12\\
4.01799999999973	8.78563678166842e-12\\
4.01999999999995	8.66387556792611e-12\\
4.02	8.66504055488409e-12\\
4.02500000000001	8.37545575550129e-12\\
4.02500000000006	8.37720344029094e-12\\
4.02999999999995	8.10334125473106e-12\\
4.03	8.10509002727072e-12\\
4.03099999999999	8.05231420660544e-12\\
4.03100000000005	8.05406483813165e-12\\
4.03200000000004	8.00370170907975e-12\\
4.03300000000003	7.95458530546832e-12\\
4.035	7.8536268442257e-12\\
4.03899999999996	7.66295393177667e-12\\
4.03999999999994	7.61828380055447e-12\\
4.04	7.61828380055184e-12\\
4.04799999999991	7.2689232514098e-12\\
4.04999999999994	7.18699970428601e-12\\
4.05	7.18916127827644e-12\\
4.05799999999991	6.8770596288063e-12\\
4.05999999999994	6.80442453911737e-12\\
4.06	6.80495033765001e-12\\
};
\end{axis}
\end{tikzpicture}%
}
      \caption{The evolution of the difference in angular displacement between
        RM and EDF of pendulum $P_2$ for execution time $C_2 = 10$ ms.}
      \label{fig:02.6.10.2_diff}
    \end{figure}
  \end{minipage}
\end{minipage}
}
