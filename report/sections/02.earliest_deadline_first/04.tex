\subsubsection{Question 4}

Figure \ref{fig:02.4.1.small} shows the former schedule magnified over the
period of the first 60 ms.
As per the response to question 2, figure \ref{fig:02.4.2} illustrates that the
schedule is indeed feasible by plotting the overall usage of the CPU over the
aforementioned timespan.


\begin{sidewaysfigure}

  \begin{figure}[H]\centering
    \scalebox{1}{% This file was created by matlab2tikz.
%
%The latest updates can be retrieved from
%  http://www.mathworks.com/matlabcentral/fileexchange/22022-matlab2tikz-matlab2tikz
%where you can also make suggestions and rate matlab2tikz.
%
\definecolor{mycolor1}{rgb}{0.00000,0.44700,0.74100}%
\definecolor{mycolor2}{rgb}{0.85000,0.32500,0.09800}%
\definecolor{mycolor3}{rgb}{0.92900,0.69400,0.12500}%
\definecolor{mycolor4}{rgb}{0.63500,0.07800,0.18400}%
%
\begin{tikzpicture}

\begin{axis}[%
width=4.133in,
height=0.576in,
at={(0.693in,3.124in)},
scale only axis,
xmin=0,
xmax=0.06,
ymin=0,
ymax=1.1,
axis background/.style={fill=white}
]
\pgfplotsset{max space between ticks=50}
\addplot [color=mycolor1,solid,forget plot]
  table[row sep=crcr]{%
0	0\\
3.15544362088405e-30	1\\
0.000656101980281985	1\\
0.00393661188169191	1\\
0.00599999999999994	1\\
0.006	0\\
0.012	0\\
0.0120000000000001	0\\
0.018	0\\
0.0180000000000001	0\\
0.0199999999999998	0\\
0.02	1\\
0.026	1\\
0.0260000000000002	0\\
0.0289999999999998	0\\
0.029	0\\
0.0319999999999996	0\\
0.0349999999999991	0\\
0.035	0\\
0.0399999999999996	0\\
0.04	1\\
0.0449999999999996	1\\
0.0459999999999996	1\\
0.046	0\\
0.047	0\\
0.0470000000000004	0\\
0.0490000000000003	0\\
0.0510000000000002	0\\
0.055	0\\
0.0579999999999996	0\\
0.058	0\\
0.0599999999999996	0\\
0.06	1\\
0.0619999999999995	1\\
0.0639999999999991	1\\
0.0659999999999991	1\\
0.066	0\\
0.0699999999999991	0\\
0.07	0\\
0.0700000000000009	0\\
0.074	0\\
0.076	0\\
0.0760000000000009	0\\
0.08	0\\
0.0800000000000009	1\\
0.0839999999999999	1\\
0.086	1\\
0.0860000000000009	0\\
0.0869999999999991	0\\
0.087	0\\
0.0880000000000004	0\\
0.0890000000000009	0\\
0.0910000000000017	0\\
0.0929999999999991	0\\
0.093	0\\
0.0970000000000017	0\\
0.0999999999999991	0\\
0.1	1\\
0.104000000000002	1\\
0.104999999999999	1\\
0.105	1\\
0.105999999999999	1\\
0.106	0\\
0.106999999999999	0\\
0.107999999999998	0\\
0.109999999999997	0\\
0.111999999999999	0\\
0.112	0\\
0.115999999999997	0\\
0.115999999999998	0\\
0.116	0\\
0.119999999999997	0\\
0.119999999999998	0\\
0.12	1\\
0.123999999999997	1\\
0.125999999999999	1\\
0.126	0\\
0.127999999999998	0\\
0.128	0\\
0.129999999999998	0\\
0.131999999999996	0\\
0.135999999999993	0\\
0.139999999999998	0\\
0.14	1\\
0.144999999999998	1\\
0.145	1\\
0.145999999999998	1\\
0.146	0\\
0.146999999999999	0\\
0.147999999999998	0\\
0.149999999999997	0\\
0.151999999999998	0\\
0.152	0\\
0.155999999999997	0\\
0.157999999999998	0\\
0.158	0\\
0.16	0\\
0.160000000000002	1\\
0.162000000000002	1\\
0.164000000000002	1\\
0.166	1\\
0.166000000000002	0\\
0.170000000000002	0\\
0.174	0\\
0.174000000000001	0\\
0.175	0\\
0.175000000000002	0\\
0.176000000000001	0\\
0.177	0\\
0.178999999999998	0\\
0.179999999999998	0\\
0.18	1\\
0.183999999999997	1\\
0.186	1\\
0.186000000000002	0\\
0.189999999999998	0\\
0.192	0\\
0.192000000000002	0\\
0.195999999999998	0\\
0.199999999999995	0\\
0.199999999999997	0\\
0.2	1\\
0.202999999999998	1\\
0.203	1\\
0.205999999999998	1\\
0.206	0\\
0.208999999999998	0\\
0.209999999999998	0\\
0.21	0\\
0.211999999999998	0\\
0.212	0\\
0.213999999999998	0\\
0.215999999999997	0\\
0.217999999999998	0\\
0.218	0\\
0.219999999999998	0\\
0.22	1\\
0.221999999999998	1\\
0.223999999999996	1\\
0.225999999999998	1\\
0.226	0\\
0.229999999999996	0\\
0.231999999999998	0\\
0.232	0\\
0.235999999999996	0\\
0.237999999999998	0\\
0.238	0\\
0.239999999999998	0\\
0.24	1\\
0.241999999999998	1\\
0.243999999999996	1\\
0.245	1\\
0.245000000000002	1\\
0.245999999999998	1\\
0.246	0\\
0.246999999999999	0\\
0.247999999999998	0\\
0.249999999999997	0\\
0.252	0\\
0.252000000000003	0\\
0.256	0\\
0.259999999999997	0\\
0.26	1\\
0.260999999999996	1\\
0.261	1\\
0.261999999999998	1\\
0.262999999999996	1\\
0.264999999999993	1\\
0.265999999999997	1\\
0.266	0\\
0.269999999999993	0\\
0.271999999999997	0\\
0.272	0\\
0.275999999999993	0\\
0.279999999999986	0\\
0.279999999999993	0\\
0.28	1\\
0.285999999999996	1\\
0.286	0\\
0.289999999999996	0\\
0.29	0\\
0.293999999999996	0\\
0.295999999999997	0\\
0.296	0\\
0.297999999999997	0\\
0.298	0\\
0.299999999999997	0\\
0.3	1\\
0.301999999999997	1\\
0.303999999999993	1\\
0.305999999999997	1\\
0.306	0\\
0.309999999999993	0\\
0.313999999999986	0\\
0.314999999999997	0\\
0.315	0\\
0.318999999999997	0\\
0.319	0\\
0.319999999999996	0\\
0.32	1\\
0.320999999999998	1\\
0.321999999999996	1\\
0.323999999999993	1\\
0.325999999999996	1\\
0.326	0\\
0.329999999999993	0\\
0.331	0\\
0.331000000000004	0\\
0.333	0\\
0.333000000000004	0\\
0.335	0\\
0.336999999999996	0\\
0.339999999999996	0\\
0.34	1\\
0.343999999999993	1\\
0.345999999999997	1\\
0.346	0\\
0.347999999999997	0\\
0.348	0\\
0.349999999999997	0\\
0.35	0\\
0.351999999999997	0\\
0.353999999999993	0\\
0.354	0\\
0.357999999999993	0\\
0.359999999999996	0\\
0.36	1\\
0.363999999999993	1\\
0.365999999999996	1\\
0.366	0\\
0.369999999999993	0\\
0.373999999999986	0\\
0.376999999999997	0\\
0.377	0\\
0.379999999999997	0\\
0.38	1\\
0.382999999999996	1\\
0.384999999999997	1\\
0.385	1\\
0.385999999999997	1\\
0.386	0\\
0.386999999999998	0\\
0.387999999999996	0\\
0.388999999999997	0\\
0.389	0\\
0.390999999999997	0\\
0.392999999999993	0\\
0.394999999999997	0\\
0.395	0\\
0.398999999999993	0\\
0.399999999999997	0\\
0.4	1\\
0.403999999999993	1\\
0.405999999999997	1\\
0.406	0\\
0.409999999999993	0\\
0.411999999999997	0\\
0.412	0\\
0.415999999999993	0\\
0.419999999999986	0\\
0.419999999999996	0\\
0.42	1\\
0.426	1\\
0.426000000000004	0\\
0.432000000000004	0\\
0.432000000000007	0\\
0.434999999999997	0\\
0.435	0\\
0.43799999999999	0\\
0.439999999999997	0\\
0.44	1\\
0.44299999999999	1\\
0.445999999999979	1\\
0.445999999999995	1\\
0.446	0\\
0.447	0\\
0.447000000000004	0\\
0.448000000000004	0\\
0.449000000000004	0\\
0.451000000000004	0\\
0.454999999999997	0\\
0.455	0\\
0.459	0\\
0.459999999999997	0\\
0.46	1\\
0.463999999999997	1\\
0.464	1\\
0.465999999999997	1\\
0.466	0\\
0.467999999999996	0\\
0.469999999999993	0\\
0.471999999999997	0\\
0.472	0\\
0.473	0\\
0.473000000000004	0\\
0.474000000000004	0\\
0.475000000000004	0\\
0.477000000000004	0\\
0.479999999999997	0\\
0.48	1\\
0.484	1\\
0.485999999999997	1\\
0.486	0\\
0.489999999999997	0\\
0.49	0\\
0.492999999999997	0\\
0.493	0\\
0.495999999999997	0\\
0.498999999999993	0\\
0.499	0\\
0.499999999999997	0\\
0.5	1\\
0.500999999999998	1\\
0.501999999999997	1\\
0.503999999999993	1\\
0.505999999999993	1\\
0.506	0\\
0.507999999999993	0\\
0.508	0\\
0.509999999999993	0\\
0.511999999999986	0\\
0.515999999999972	0\\
0.519999999999993	0\\
0.52	1\\
0.521999999999993	1\\
0.522	1\\
0.523999999999993	1\\
0.524999999999993	1\\
0.525	1\\
0.525999999999993	1\\
0.526	0\\
0.526999999999998	0\\
0.527999999999997	0\\
0.529999999999993	0\\
0.531999999999993	0\\
0.532	0\\
0.535999999999993	0\\
0.538	0\\
0.538000000000007	0\\
0.539999999999993	0\\
0.54	1\\
0.541999999999986	1\\
0.543999999999972	1\\
0.545999999999993	1\\
0.546	0\\
0.549999999999972	0\\
0.550999999999993	0\\
0.551	0\\
0.554999999999972	0\\
0.556999999999993	0\\
0.557	0\\
0.559999999999993	0\\
0.56	1\\
0.562999999999993	1\\
0.565999999999986	1\\
0.565999999999993	1\\
0.566	0\\
0.571999999999986	0\\
0.571999999999993	0\\
0.572	0\\
0.577999999999986	0\\
0.579999999999993	0\\
0.58	1\\
0.585999999999986	1\\
0.585999999999993	1\\
0.586	0\\
0.591999999999986	0\\
0.591999999999993	0\\
0.592	0\\
0.594999999999993	0\\
0.595	0\\
0.597999999999993	0\\
0.599999999999993	0\\
0.6	1\\
};
\end{axis}

\begin{axis}[%
width=4.133in,
height=0.576in,
at={(0.693in,2.248in)},
scale only axis,
xmin=0,
xmax=0.06,
ymin=0,
ymax=1.1,
axis background/.style={fill=white}
]
\pgfplotsset{max space between ticks=50}
\addplot [color=mycolor2,solid,forget plot]
  table[row sep=crcr]{%
0	0\\
3.15544362088405e-30	0.5\\
0.000656101980281985	0.5\\
0.00393661188169191	0.5\\
0.00599999999999994	0.5\\
0.006	1\\
0.012	1\\
0.0120000000000001	0\\
0.018	0\\
0.0180000000000001	0\\
0.0199999999999998	0\\
0.02	0\\
0.026	0\\
0.0260000000000002	0\\
0.0289999999999998	0\\
0.029	1\\
0.0319999999999996	1\\
0.0349999999999991	1\\
0.035	0\\
0.0399999999999996	0\\
0.04	0\\
0.0449999999999996	0\\
0.0459999999999996	0\\
0.046	0\\
0.047	0\\
0.0470000000000004	0\\
0.0490000000000003	0\\
0.0510000000000002	0\\
0.055	0\\
0.0579999999999996	0\\
0.058	1\\
0.0599999999999996	1\\
0.06	0.5\\
0.0619999999999995	0.5\\
0.0639999999999991	0.5\\
0.0659999999999991	0.5\\
0.066	1\\
0.0699999999999991	1\\
0.07	1\\
0.0700000000000009	0\\
0.074	0\\
0.076	0\\
0.0760000000000009	0\\
0.08	0\\
0.0800000000000009	0\\
0.0839999999999999	0\\
0.086	0\\
0.0860000000000009	0\\
0.0869999999999991	0\\
0.087	1\\
0.0880000000000004	1\\
0.0890000000000009	1\\
0.0910000000000017	1\\
0.0929999999999991	1\\
0.093	0\\
0.0970000000000017	0\\
0.0999999999999991	0\\
0.1	0\\
0.104000000000002	0\\
0.104999999999999	0\\
0.105	0\\
0.105999999999999	0\\
0.106	0\\
0.106999999999999	0\\
0.107999999999998	0\\
0.109999999999997	0\\
0.111999999999999	0\\
0.112	0\\
0.115999999999997	0\\
0.115999999999998	0\\
0.116	1\\
0.119999999999997	1\\
0.119999999999998	1\\
0.12	0.5\\
0.123999999999997	0.5\\
0.125999999999999	0.5\\
0.126	1\\
0.127999999999998	1\\
0.128	0\\
0.129999999999998	0\\
0.131999999999996	0\\
0.135999999999993	0\\
0.139999999999998	0\\
0.14	0\\
0.144999999999998	0\\
0.145	0.5\\
0.145999999999998	0.5\\
0.146	1\\
0.146999999999999	1\\
0.147999999999998	1\\
0.149999999999997	1\\
0.151999999999998	1\\
0.152	0\\
0.155999999999997	0\\
0.157999999999998	0\\
0.158	0\\
0.16	0\\
0.160000000000002	0\\
0.162000000000002	0\\
0.164000000000002	0\\
0.166	0\\
0.166000000000002	0\\
0.170000000000002	0\\
0.174	0\\
0.174000000000001	1\\
0.175	1\\
0.175000000000002	1\\
0.176000000000001	1\\
0.177	1\\
0.178999999999998	1\\
0.179999999999998	1\\
0.18	0\\
0.183999999999997	0\\
0.186	0\\
0.186000000000002	0\\
0.189999999999998	0\\
0.192	0\\
0.192000000000002	0\\
0.195999999999998	0\\
0.199999999999995	0\\
0.199999999999997	0\\
0.2	0\\
0.202999999999998	0\\
0.203	0.5\\
0.205999999999998	0.5\\
0.206	1\\
0.208999999999998	1\\
0.209999999999998	1\\
0.21	1\\
0.211999999999998	1\\
0.212	0\\
0.213999999999998	0\\
0.215999999999997	0\\
0.217999999999998	0\\
0.218	0\\
0.219999999999998	0\\
0.22	0\\
0.221999999999998	0\\
0.223999999999996	0\\
0.225999999999998	0\\
0.226	0\\
0.229999999999996	0\\
0.231999999999998	0\\
0.232	1\\
0.235999999999996	1\\
0.237999999999998	1\\
0.238	0\\
0.239999999999998	0\\
0.24	0\\
0.241999999999998	0\\
0.243999999999996	0\\
0.245	0\\
0.245000000000002	0\\
0.245999999999998	0\\
0.246	0\\
0.246999999999999	0\\
0.247999999999998	0\\
0.249999999999997	0\\
0.252	0\\
0.252000000000003	0\\
0.256	0\\
0.259999999999997	0\\
0.26	0\\
0.260999999999996	0\\
0.261	0.5\\
0.261999999999998	0.5\\
0.262999999999996	0.5\\
0.264999999999993	0.5\\
0.265999999999997	0.5\\
0.266	1\\
0.269999999999993	1\\
0.271999999999997	1\\
0.272	0\\
0.275999999999993	0\\
0.279999999999986	0\\
0.279999999999993	0\\
0.28	0\\
0.285999999999996	0\\
0.286	0\\
0.289999999999996	0\\
0.29	1\\
0.293999999999996	1\\
0.295999999999997	1\\
0.296	0\\
0.297999999999997	0\\
0.298	0\\
0.299999999999997	0\\
0.3	0\\
0.301999999999997	0\\
0.303999999999993	0\\
0.305999999999997	0\\
0.306	0\\
0.309999999999993	0\\
0.313999999999986	0\\
0.314999999999997	0\\
0.315	0\\
0.318999999999997	0\\
0.319	1\\
0.319999999999996	1\\
0.32	0.5\\
0.320999999999998	0.5\\
0.321999999999996	0.5\\
0.323999999999993	0.5\\
0.325999999999996	0.5\\
0.326	1\\
0.329999999999993	1\\
0.331	1\\
0.331000000000004	0\\
0.333	0\\
0.333000000000004	0\\
0.335	0\\
0.336999999999996	0\\
0.339999999999996	0\\
0.34	0\\
0.343999999999993	0\\
0.345999999999997	0\\
0.346	0\\
0.347999999999997	0\\
0.348	1\\
0.349999999999997	1\\
0.35	1\\
0.351999999999997	1\\
0.353999999999993	1\\
0.354	0\\
0.357999999999993	0\\
0.359999999999996	0\\
0.36	0\\
0.363999999999993	0\\
0.365999999999996	0\\
0.366	0\\
0.369999999999993	0\\
0.373999999999986	0\\
0.376999999999997	0\\
0.377	1\\
0.379999999999997	1\\
0.38	0.5\\
0.382999999999996	0.5\\
0.384999999999997	0.5\\
0.385	0.5\\
0.385999999999997	0.5\\
0.386	1\\
0.386999999999998	1\\
0.387999999999996	1\\
0.388999999999997	1\\
0.389	0\\
0.390999999999997	0\\
0.392999999999993	0\\
0.394999999999997	0\\
0.395	0\\
0.398999999999993	0\\
0.399999999999997	0\\
0.4	0\\
0.403999999999993	0\\
0.405999999999997	0\\
0.406	1\\
0.409999999999993	1\\
0.411999999999997	1\\
0.412	0\\
0.415999999999993	0\\
0.419999999999986	0\\
0.419999999999996	0\\
0.42	0\\
0.426	0\\
0.426000000000004	0\\
0.432000000000004	0\\
0.432000000000007	0\\
0.434999999999997	0\\
0.435	1\\
0.43799999999999	1\\
0.439999999999997	1\\
0.44	0.5\\
0.44299999999999	0.5\\
0.445999999999979	0.5\\
0.445999999999995	0.5\\
0.446	1\\
0.447	1\\
0.447000000000004	0\\
0.448000000000004	0\\
0.449000000000004	0\\
0.451000000000004	0\\
0.454999999999997	0\\
0.455	0\\
0.459	0\\
0.459999999999997	0\\
0.46	0\\
0.463999999999997	0\\
0.464	0.5\\
0.465999999999997	0.5\\
0.466	1\\
0.467999999999996	1\\
0.469999999999993	1\\
0.471999999999997	1\\
0.472	0\\
0.473	0\\
0.473000000000004	0\\
0.474000000000004	0\\
0.475000000000004	0\\
0.477000000000004	0\\
0.479999999999997	0\\
0.48	0\\
0.484	0\\
0.485999999999997	0\\
0.486	0\\
0.489999999999997	0\\
0.49	0\\
0.492999999999997	0\\
0.493	1\\
0.495999999999997	1\\
0.498999999999993	1\\
0.499	0\\
0.499999999999997	0\\
0.5	0\\
0.500999999999998	0\\
0.501999999999997	0\\
0.503999999999993	0\\
0.505999999999993	0\\
0.506	0\\
0.507999999999993	0\\
0.508	0\\
0.509999999999993	0\\
0.511999999999986	0\\
0.515999999999972	0\\
0.519999999999993	0\\
0.52	0\\
0.521999999999993	0\\
0.522	0.5\\
0.523999999999993	0.5\\
0.524999999999993	0.5\\
0.525	0.5\\
0.525999999999993	0.5\\
0.526	1\\
0.526999999999998	1\\
0.527999999999997	1\\
0.529999999999993	1\\
0.531999999999993	1\\
0.532	0\\
0.535999999999993	0\\
0.538	0\\
0.538000000000007	0\\
0.539999999999993	0\\
0.54	0\\
0.541999999999986	0\\
0.543999999999972	0\\
0.545999999999993	0\\
0.546	0\\
0.549999999999972	0\\
0.550999999999993	0\\
0.551	1\\
0.554999999999972	1\\
0.556999999999993	1\\
0.557	0\\
0.559999999999993	0\\
0.56	0\\
0.562999999999993	0\\
0.565999999999986	0\\
0.565999999999993	0\\
0.566	0\\
0.571999999999986	0\\
0.571999999999993	0\\
0.572	0\\
0.577999999999986	0\\
0.579999999999993	0\\
0.58	0.5\\
0.585999999999986	0.5\\
0.585999999999993	0.5\\
0.586	1\\
0.591999999999986	1\\
0.591999999999993	1\\
0.592	0\\
0.594999999999993	0\\
0.595	0\\
0.597999999999993	0\\
0.599999999999993	0\\
0.6	0\\
};
\end{axis}

\begin{axis}[%
width=4.133in,
height=0.576in,
at={(0.693in,1.372in)},
scale only axis,
xmin=0,
xmax=0.06,
ymin=0,
ymax=1.1,
axis background/.style={fill=white}
]
\pgfplotsset{max space between ticks=50}
\addplot [color=mycolor3,solid,forget plot]
  table[row sep=crcr]{%
0	0\\
3.15544362088405e-30	0.5\\
0.000656101980281985	0.5\\
0.00393661188169191	0.5\\
0.00599999999999994	0.5\\
0.006	0.5\\
0.012	0.5\\
0.0120000000000001	1\\
0.018	1\\
0.0180000000000001	0\\
0.0199999999999998	0\\
0.02	0\\
0.026	0\\
0.0260000000000002	0\\
0.0289999999999998	0\\
0.029	0\\
0.0319999999999996	0\\
0.0349999999999991	0\\
0.035	1\\
0.0399999999999996	1\\
0.04	0.5\\
0.0449999999999996	0.5\\
0.0459999999999996	0.5\\
0.046	1\\
0.047	1\\
0.0470000000000004	0\\
0.0490000000000003	0\\
0.0510000000000002	0\\
0.055	0\\
0.0579999999999996	0\\
0.058	0\\
0.0599999999999996	0\\
0.06	0\\
0.0619999999999995	0\\
0.0639999999999991	0\\
0.0659999999999991	0\\
0.066	0\\
0.0699999999999991	0\\
0.07	0\\
0.0700000000000009	1\\
0.074	1\\
0.076	1\\
0.0760000000000009	0\\
0.08	0\\
0.0800000000000009	0\\
0.0839999999999999	0\\
0.086	0\\
0.0860000000000009	0\\
0.0869999999999991	0\\
0.087	0\\
0.0880000000000004	0\\
0.0890000000000009	0\\
0.0910000000000017	0\\
0.0929999999999991	0\\
0.093	0\\
0.0970000000000017	0\\
0.0999999999999991	0\\
0.1	0\\
0.104000000000002	0\\
0.104999999999999	0\\
0.105	0.5\\
0.105999999999999	0.5\\
0.106	1\\
0.106999999999999	1\\
0.107999999999998	1\\
0.109999999999997	1\\
0.111999999999999	1\\
0.112	0\\
0.115999999999997	0\\
0.115999999999998	0\\
0.116	0\\
0.119999999999997	0\\
0.119999999999998	0\\
0.12	0\\
0.123999999999997	0\\
0.125999999999999	0\\
0.126	0\\
0.127999999999998	0\\
0.128	0\\
0.129999999999998	0\\
0.131999999999996	0\\
0.135999999999993	0\\
0.139999999999998	0\\
0.14	0.5\\
0.144999999999998	0.5\\
0.145	0.5\\
0.145999999999998	0.5\\
0.146	0.5\\
0.146999999999999	0.5\\
0.147999999999998	0.5\\
0.149999999999997	0.5\\
0.151999999999998	0.5\\
0.152	1\\
0.155999999999997	1\\
0.157999999999998	1\\
0.158	0\\
0.16	0\\
0.160000000000002	0\\
0.162000000000002	0\\
0.164000000000002	0\\
0.166	0\\
0.166000000000002	0\\
0.170000000000002	0\\
0.174	0\\
0.174000000000001	0\\
0.175	0\\
0.175000000000002	0.5\\
0.176000000000001	0.5\\
0.177	0.5\\
0.178999999999998	0.5\\
0.179999999999998	0.5\\
0.18	0.5\\
0.183999999999997	0.5\\
0.186	0.5\\
0.186000000000002	1\\
0.189999999999998	1\\
0.192	1\\
0.192000000000002	0\\
0.195999999999998	0\\
0.199999999999995	0\\
0.199999999999997	0\\
0.2	0\\
0.202999999999998	0\\
0.203	0\\
0.205999999999998	0\\
0.206	0\\
0.208999999999998	0\\
0.209999999999998	0\\
0.21	0.5\\
0.211999999999998	0.5\\
0.212	1\\
0.213999999999998	1\\
0.215999999999997	1\\
0.217999999999998	1\\
0.218	0\\
0.219999999999998	0\\
0.22	0\\
0.221999999999998	0\\
0.223999999999996	0\\
0.225999999999998	0\\
0.226	0\\
0.229999999999996	0\\
0.231999999999998	0\\
0.232	0\\
0.235999999999996	0\\
0.237999999999998	0\\
0.238	0\\
0.239999999999998	0\\
0.24	0\\
0.241999999999998	0\\
0.243999999999996	0\\
0.245	0\\
0.245000000000002	0.5\\
0.245999999999998	0.5\\
0.246	1\\
0.246999999999999	1\\
0.247999999999998	1\\
0.249999999999997	1\\
0.252	1\\
0.252000000000003	0\\
0.256	0\\
0.259999999999997	0\\
0.26	0\\
0.260999999999996	0\\
0.261	0\\
0.261999999999998	0\\
0.262999999999996	0\\
0.264999999999993	0\\
0.265999999999997	0\\
0.266	0\\
0.269999999999993	0\\
0.271999999999997	0\\
0.272	0\\
0.275999999999993	0\\
0.279999999999986	0\\
0.279999999999993	0\\
0.28	0.5\\
0.285999999999996	0.5\\
0.286	1\\
0.289999999999996	1\\
0.29	0.5\\
0.293999999999996	0.5\\
0.295999999999997	0.5\\
0.296	1\\
0.297999999999997	1\\
0.298	0\\
0.299999999999997	0\\
0.3	0\\
0.301999999999997	0\\
0.303999999999993	0\\
0.305999999999997	0\\
0.306	0\\
0.309999999999993	0\\
0.313999999999986	0\\
0.314999999999997	0\\
0.315	1\\
0.318999999999997	1\\
0.319	0.5\\
0.319999999999996	0.5\\
0.32	0.5\\
0.320999999999998	0.5\\
0.321999999999996	0.5\\
0.323999999999993	0.5\\
0.325999999999996	0.5\\
0.326	0.5\\
0.329999999999993	0.5\\
0.331	0.5\\
0.331000000000004	1\\
0.333	1\\
0.333000000000004	0\\
0.335	0\\
0.336999999999996	0\\
0.339999999999996	0\\
0.34	0\\
0.343999999999993	0\\
0.345999999999997	0\\
0.346	0\\
0.347999999999997	0\\
0.348	0\\
0.349999999999997	0\\
0.35	0.5\\
0.351999999999997	0.5\\
0.353999999999993	0.5\\
0.354	1\\
0.357999999999993	1\\
0.359999999999996	1\\
0.36	0\\
0.363999999999993	0\\
0.365999999999996	0\\
0.366	0\\
0.369999999999993	0\\
0.373999999999986	0\\
0.376999999999997	0\\
0.377	0\\
0.379999999999997	0\\
0.38	0\\
0.382999999999996	0\\
0.384999999999997	0\\
0.385	0.5\\
0.385999999999997	0.5\\
0.386	0.5\\
0.386999999999998	0.5\\
0.387999999999996	0.5\\
0.388999999999997	0.5\\
0.389	1\\
0.390999999999997	1\\
0.392999999999993	1\\
0.394999999999997	1\\
0.395	0\\
0.398999999999993	0\\
0.399999999999997	0\\
0.4	0\\
0.403999999999993	0\\
0.405999999999997	0\\
0.406	0\\
0.409999999999993	0\\
0.411999999999997	0\\
0.412	0\\
0.415999999999993	0\\
0.419999999999986	0\\
0.419999999999996	0\\
0.42	0.5\\
0.426	0.5\\
0.426000000000004	1\\
0.432000000000004	1\\
0.432000000000007	0\\
0.434999999999997	0\\
0.435	0\\
0.43799999999999	0\\
0.439999999999997	0\\
0.44	0\\
0.44299999999999	0\\
0.445999999999979	0\\
0.445999999999995	0\\
0.446	0\\
0.447	0\\
0.447000000000004	0\\
0.448000000000004	0\\
0.449000000000004	0\\
0.451000000000004	0\\
0.454999999999997	0\\
0.455	1\\
0.459	1\\
0.459999999999997	1\\
0.46	0.5\\
0.463999999999997	0.5\\
0.464	0.5\\
0.465999999999997	0.5\\
0.466	0.5\\
0.467999999999996	0.5\\
0.469999999999993	0.5\\
0.471999999999997	0.5\\
0.472	1\\
0.473	1\\
0.473000000000004	0\\
0.474000000000004	0\\
0.475000000000004	0\\
0.477000000000004	0\\
0.479999999999997	0\\
0.48	0\\
0.484	0\\
0.485999999999997	0\\
0.486	0\\
0.489999999999997	0\\
0.49	1\\
0.492999999999997	1\\
0.493	0.5\\
0.495999999999997	0.5\\
0.498999999999993	0.5\\
0.499	1\\
0.499999999999997	1\\
0.5	0.5\\
0.500999999999998	0.5\\
0.501999999999997	0.5\\
0.503999999999993	0.5\\
0.505999999999993	0.5\\
0.506	1\\
0.507999999999993	1\\
0.508	0\\
0.509999999999993	0\\
0.511999999999986	0\\
0.515999999999972	0\\
0.519999999999993	0\\
0.52	0\\
0.521999999999993	0\\
0.522	0\\
0.523999999999993	0\\
0.524999999999993	0\\
0.525	0.5\\
0.525999999999993	0.5\\
0.526	0.5\\
0.526999999999998	0.5\\
0.527999999999997	0.5\\
0.529999999999993	0.5\\
0.531999999999993	0.5\\
0.532	1\\
0.535999999999993	1\\
0.538	1\\
0.538000000000007	0\\
0.539999999999993	0\\
0.54	0\\
0.541999999999986	0\\
0.543999999999972	0\\
0.545999999999993	0\\
0.546	0\\
0.549999999999972	0\\
0.550999999999993	0\\
0.551	0\\
0.554999999999972	0\\
0.556999999999993	0\\
0.557	0\\
0.559999999999993	0\\
0.56	0.5\\
0.562999999999993	0.5\\
0.565999999999986	0.5\\
0.565999999999993	0.5\\
0.566	1\\
0.571999999999986	1\\
0.571999999999993	1\\
0.572	0\\
0.577999999999986	0\\
0.579999999999993	0\\
0.58	0\\
0.585999999999986	0\\
0.585999999999993	0\\
0.586	0\\
0.591999999999986	0\\
0.591999999999993	0\\
0.592	0\\
0.594999999999993	0\\
0.595	1\\
0.597999999999993	1\\
0.599999999999993	1\\
0.6	0.5\\
};
\end{axis}

\begin{axis}[%
width=4.133in,
height=0.576in,
at={(0.693in,0.495in)},
scale only axis,
xmin=0,
xmax=0.06,
ymin=0,
ymax=1.1,
axis background/.style={fill=white}
]
\pgfplotsset{max space between ticks=50}
\addplot [color=mycolor4,solid,forget plot]
  table[row sep=crcr]{%
0	0\\
3.15544362088405e-30	1\\
0.000656101980281985	1\\
0.00393661188169191	1\\
0.00599999999999994	1\\
0.006	1\\
0.012	1\\
0.0120000000000001	1\\
0.018	1\\
0.0180000000000001	0\\
0.0199999999999998	0\\
0.02	1\\
0.026	1\\
0.0260000000000002	0\\
0.0289999999999998	0\\
0.029	1\\
0.0319999999999996	1\\
0.0349999999999991	1\\
0.035	1\\
0.0399999999999996	1\\
0.04	1\\
0.0449999999999996	1\\
0.0459999999999996	1\\
0.046	1\\
0.047	1\\
0.0470000000000004	0\\
0.0490000000000003	0\\
0.0510000000000002	0\\
0.055	0\\
0.0579999999999996	0\\
0.058	1\\
0.0599999999999996	1\\
0.06	1\\
0.0619999999999995	1\\
0.0639999999999991	1\\
0.0659999999999991	1\\
0.066	1\\
0.0699999999999991	1\\
0.07	1\\
0.0700000000000009	1\\
0.074	1\\
0.076	1\\
0.0760000000000009	0\\
0.08	0\\
0.0800000000000009	1\\
0.0839999999999999	1\\
0.086	1\\
0.0860000000000009	0\\
0.0869999999999991	0\\
0.087	1\\
0.0880000000000004	1\\
0.0890000000000009	1\\
0.0910000000000017	1\\
0.0929999999999991	1\\
0.093	0\\
0.0970000000000017	0\\
0.0999999999999991	0\\
0.1	1\\
0.104000000000002	1\\
0.104999999999999	1\\
0.105	1\\
0.105999999999999	1\\
0.106	1\\
0.106999999999999	1\\
0.107999999999998	1\\
0.109999999999997	1\\
0.111999999999999	1\\
0.112	0\\
0.115999999999997	0\\
0.115999999999998	0\\
0.116	1\\
0.119999999999997	1\\
0.119999999999998	1\\
0.12	1\\
0.123999999999997	1\\
0.125999999999999	1\\
0.126	1\\
0.127999999999998	1\\
0.128	0\\
0.129999999999998	0\\
0.131999999999996	0\\
0.135999999999993	0\\
0.139999999999998	0\\
0.14	1\\
0.144999999999998	1\\
0.145	1\\
0.145999999999998	1\\
0.146	1\\
0.146999999999999	1\\
0.147999999999998	1\\
0.149999999999997	1\\
0.151999999999998	1\\
0.152	1\\
0.155999999999997	1\\
0.157999999999998	1\\
0.158	0\\
0.16	0\\
0.160000000000002	1\\
0.162000000000002	1\\
0.164000000000002	1\\
0.166	1\\
0.166000000000002	0\\
0.170000000000002	0\\
0.174	0\\
0.174000000000001	1\\
0.175	1\\
0.175000000000002	1\\
0.176000000000001	1\\
0.177	1\\
0.178999999999998	1\\
0.179999999999998	1\\
0.18	1\\
0.183999999999997	1\\
0.186	1\\
0.186000000000002	1\\
0.189999999999998	1\\
0.192	1\\
0.192000000000002	0\\
0.195999999999998	0\\
0.199999999999995	0\\
0.199999999999997	0\\
0.2	1\\
0.202999999999998	1\\
0.203	1\\
0.205999999999998	1\\
0.206	1\\
0.208999999999998	1\\
0.209999999999998	1\\
0.21	1\\
0.211999999999998	1\\
0.212	1\\
0.213999999999998	1\\
0.215999999999997	1\\
0.217999999999998	1\\
0.218	0\\
0.219999999999998	0\\
0.22	1\\
0.221999999999998	1\\
0.223999999999996	1\\
0.225999999999998	1\\
0.226	0\\
0.229999999999996	0\\
0.231999999999998	0\\
0.232	1\\
0.235999999999996	1\\
0.237999999999998	1\\
0.238	0\\
0.239999999999998	0\\
0.24	1\\
0.241999999999998	1\\
0.243999999999996	1\\
0.245	1\\
0.245000000000002	1\\
0.245999999999998	1\\
0.246	1\\
0.246999999999999	1\\
0.247999999999998	1\\
0.249999999999997	1\\
0.252	1\\
0.252000000000003	0\\
0.256	0\\
0.259999999999997	0\\
0.26	1\\
0.260999999999996	1\\
0.261	1\\
0.261999999999998	1\\
0.262999999999996	1\\
0.264999999999993	1\\
0.265999999999997	1\\
0.266	1\\
0.269999999999993	1\\
0.271999999999997	1\\
0.272	0\\
0.275999999999993	0\\
0.279999999999986	0\\
0.279999999999993	0\\
0.28	1\\
0.285999999999996	1\\
0.286	1\\
0.289999999999996	1\\
0.29	1\\
0.293999999999996	1\\
0.295999999999997	1\\
0.296	1\\
0.297999999999997	1\\
0.298	0\\
0.299999999999997	0\\
0.3	1\\
0.301999999999997	1\\
0.303999999999993	1\\
0.305999999999997	1\\
0.306	0\\
0.309999999999993	0\\
0.313999999999986	0\\
0.314999999999997	0\\
0.315	1\\
0.318999999999997	1\\
0.319	1\\
0.319999999999996	1\\
0.32	1\\
0.320999999999998	1\\
0.321999999999996	1\\
0.323999999999993	1\\
0.325999999999996	1\\
0.326	1\\
0.329999999999993	1\\
0.331	1\\
0.331000000000004	1\\
0.333	1\\
0.333000000000004	0\\
0.335	0\\
0.336999999999996	0\\
0.339999999999996	0\\
0.34	1\\
0.343999999999993	1\\
0.345999999999997	1\\
0.346	0\\
0.347999999999997	0\\
0.348	1\\
0.349999999999997	1\\
0.35	1\\
0.351999999999997	1\\
0.353999999999993	1\\
0.354	1\\
0.357999999999993	1\\
0.359999999999996	1\\
0.36	1\\
0.363999999999993	1\\
0.365999999999996	1\\
0.366	0\\
0.369999999999993	0\\
0.373999999999986	0\\
0.376999999999997	0\\
0.377	1\\
0.379999999999997	1\\
0.38	1\\
0.382999999999996	1\\
0.384999999999997	1\\
0.385	1\\
0.385999999999997	1\\
0.386	1\\
0.386999999999998	1\\
0.387999999999996	1\\
0.388999999999997	1\\
0.389	1\\
0.390999999999997	1\\
0.392999999999993	1\\
0.394999999999997	1\\
0.395	0\\
0.398999999999993	0\\
0.399999999999997	0\\
0.4	1\\
0.403999999999993	1\\
0.405999999999997	1\\
0.406	1\\
0.409999999999993	1\\
0.411999999999997	1\\
0.412	0\\
0.415999999999993	0\\
0.419999999999986	0\\
0.419999999999996	0\\
0.42	1\\
0.426	1\\
0.426000000000004	1\\
0.432000000000004	1\\
0.432000000000007	0\\
0.434999999999997	0\\
0.435	1\\
0.43799999999999	1\\
0.439999999999997	1\\
0.44	1\\
0.44299999999999	1\\
0.445999999999979	1\\
0.445999999999995	1\\
0.446	1\\
0.447	1\\
0.447000000000004	0\\
0.448000000000004	0\\
0.449000000000004	0\\
0.451000000000004	0\\
0.454999999999997	0\\
0.455	1\\
0.459	1\\
0.459999999999997	1\\
0.46	1\\
0.463999999999997	1\\
0.464	1\\
0.465999999999997	1\\
0.466	1\\
0.467999999999996	1\\
0.469999999999993	1\\
0.471999999999997	1\\
0.472	1\\
0.473	1\\
0.473000000000004	0\\
0.474000000000004	0\\
0.475000000000004	0\\
0.477000000000004	0\\
0.479999999999997	0\\
0.48	1\\
0.484	1\\
0.485999999999997	1\\
0.486	0\\
0.489999999999997	0\\
0.49	1\\
0.492999999999997	1\\
0.493	1\\
0.495999999999997	1\\
0.498999999999993	1\\
0.499	1\\
0.499999999999997	1\\
0.5	1\\
0.500999999999998	1\\
0.501999999999997	1\\
0.503999999999993	1\\
0.505999999999993	1\\
0.506	1\\
0.507999999999993	1\\
0.508	0\\
0.509999999999993	0\\
0.511999999999986	0\\
0.515999999999972	0\\
0.519999999999993	0\\
0.52	1\\
0.521999999999993	1\\
0.522	1\\
0.523999999999993	1\\
0.524999999999993	1\\
0.525	1\\
0.525999999999993	1\\
0.526	1\\
0.526999999999998	1\\
0.527999999999997	1\\
0.529999999999993	1\\
0.531999999999993	1\\
0.532	1\\
0.535999999999993	1\\
0.538	1\\
0.538000000000007	0\\
0.539999999999993	0\\
0.54	1\\
0.541999999999986	1\\
0.543999999999972	1\\
0.545999999999993	1\\
0.546	0\\
0.549999999999972	0\\
0.550999999999993	0\\
0.551	1\\
0.554999999999972	1\\
0.556999999999993	1\\
0.557	0\\
0.559999999999993	0\\
0.56	1\\
0.562999999999993	1\\
0.565999999999986	1\\
0.565999999999993	1\\
0.566	1\\
0.571999999999986	1\\
0.571999999999993	1\\
0.572	0\\
0.577999999999986	0\\
0.579999999999993	0\\
0.58	1\\
0.585999999999986	1\\
0.585999999999993	1\\
0.586	1\\
0.591999999999986	1\\
0.591999999999993	1\\
0.592	0\\
0.594999999999993	0\\
0.595	1\\
0.597999999999993	1\\
0.599999999999993	1\\
0.6	1\\
};
\end{axis}
\end{tikzpicture}%
}
    \caption{The executed schedule for the three pendula restricted to the first
        60 ms. \texttt{Blue}: $P_1$, \texttt{Red}: $P_2$,
        \texttt{Orange}: $P_3$. $C_i = 6$ ms. The last figure shows the overall
      processor usage for verification purposes.}
    \label{fig:02.4.1.small}
  \end{figure}

  \begin{figure}[H]\centering
    \scalebox{0.7}{\begin{ganttchart}[vgrid, hgrid]{0}{47}
%\gantttitle{2016}{12}\\
\gantttitlelist{6,12,...,144}{6}\\
\ganttset{progress label text={},
       bar incomplete/.append style={fill=black!40},
       group/.append style={draw=black, fill=black},}
\ganttbar{Task 1}{0}{5}
\ganttbar{}{20}{25}
\ganttbar{}{40}{45}
\ganttbar{}{60}{65}
\ganttbar{}{80}{85}
\ganttbar{}{100}{105}
\ganttbar{}{120}{125}\\

\ganttbar[progress=00]{Task 2}{0}{5}
\ganttbar{}{6}{11}
\ganttbar{}{29}{34}
\ganttbar[progress=00]{}{60}{65}
\ganttbar{}{66}{69}
\ganttbar{}{87}{92}
\ganttbar{}{116}{119}
\ganttbar[progress=00]{}{120}{125}
\ganttbar{}{126}{127}\\

\ganttbar[progress=00]{Task 3}{0}{11}
\ganttbar{}{12}{17}
\ganttbar{}{35}{39}
\ganttbar[progress=00]{}{40}{45}
\ganttbar{}{46}{46}
\ganttbar{}{70}{75}
\ganttbar[progress=00]{}{105}{105}
\ganttbar{}{106}{111}
\end{ganttchart}
}
    \caption{A portion of the calculated EDF schedule $\sigma$ for tasks
      $J_1, J_2, J_3$. Shaded areas denote the waiting time.}
    \label{fig:edf_6}
  \end{figure}

\end{sidewaysfigure}

\begin{figure}[H]\centering
  \scalebox{0.7}{% This file was created by matlab2tikz.
%
%The latest updates can be retrieved from
%  http://www.mathworks.com/matlabcentral/fileexchange/22022-matlab2tikz-matlab2tikz
%where you can also make suggestions and rate matlab2tikz.
%
\definecolor{mycolor1}{rgb}{0.00000,0.44700,0.74100}%
%
\begin{tikzpicture}

\begin{axis}[%
width=4.133in,
height=3.26in,
at={(0.693in,0.44in)},
scale only axis,
xmin=0,
xmax=1700,
xmajorgrids,
ymin=0,
ymax=1.25,
ymajorgrids,
axis background/.style={fill=white}
]
\addplot [color=mycolor1,solid,forget plot]
  table[row sep=crcr]{%
1	0\\
2	1\\
3	1\\
4	1\\
5	1\\
6	0.75\\
7	0.75\\
8	0.75\\
9	0.75\\
10	1\\
11	1\\
12	0.75\\
13	0.75\\
14	1\\
15	1\\
16	1\\
17	1\\
18	1\\
19	1\\
20	0.75\\
21	0.75\\
22	1\\
23	1\\
24	1\\
25	1\\
26	1\\
27	1\\
28	1\\
29	1\\
30	1\\
31	1\\
32	1\\
33	1\\
34	1\\
35	1\\
36	1\\
37	1\\
38	1\\
39	1\\
40	1\\
41	1\\
42	1\\
43	1\\
44	1\\
45	1\\
46	1\\
47	1\\
48	1\\
49	1\\
50	1\\
51	1\\
52	1\\
53	1\\
54	1\\
55	1\\
56	1\\
57	1\\
58	1\\
59	1\\
60	1\\
61	1\\
62	1\\
63	1\\
64	1\\
65	1\\
66	1\\
67	1\\
68	1\\
69	1\\
70	1\\
71	1\\
72	1\\
73	1\\
74	1\\
75	1\\
76	1\\
77	1\\
78	1\\
79	1\\
80	1\\
81	1\\
82	1\\
83	1\\
84	1\\
85	1\\
86	1\\
87	1\\
88	1\\
89	1\\
90	1\\
91	1\\
92	1\\
93	1\\
94	1\\
95	1\\
96	1\\
97	1\\
98	1\\
99	1\\
100	1\\
101	1\\
102	1\\
103	1\\
104	1\\
105	1\\
106	1\\
107	1\\
108	1\\
109	1\\
110	1\\
111	1\\
112	1\\
113	1\\
114	1\\
115	1\\
116	1\\
117	1\\
118	1\\
119	1\\
120	1\\
121	1\\
122	1\\
123	1\\
124	1\\
125	1\\
126	1\\
127	1\\
128	1\\
129	1\\
130	1\\
131	1\\
132	1\\
133	1\\
134	1\\
135	1\\
136	1\\
137	1\\
138	1\\
139	1\\
140	1\\
141	1\\
142	1\\
143	1\\
144	1\\
145	1\\
146	1\\
147	1\\
148	1\\
149	1\\
150	1\\
151	1\\
152	1\\
153	1\\
154	1\\
155	1\\
156	1\\
157	1\\
158	1\\
159	1\\
160	1\\
161	1\\
162	1\\
163	1\\
164	1\\
165	1\\
166	1\\
167	1\\
168	1\\
169	1\\
170	1\\
171	1\\
172	1\\
173	1\\
174	1\\
175	1\\
176	1\\
177	1\\
178	1\\
179	1\\
180	1\\
181	1\\
182	1\\
183	1\\
184	1\\
185	1\\
186	1\\
187	1\\
188	1\\
189	1\\
190	1\\
191	1\\
192	1\\
193	1\\
194	1\\
195	1\\
196	1\\
197	1\\
198	1\\
199	1\\
200	1\\
201	1\\
202	1\\
203	1\\
204	1\\
205	1\\
206	1\\
207	1\\
208	1\\
209	1\\
210	1\\
211	1\\
212	1\\
213	1\\
214	1\\
215	1\\
216	1\\
217	1\\
218	1\\
219	1\\
220	1\\
221	1\\
222	1\\
223	1\\
224	1\\
225	1\\
226	1\\
227	1\\
228	1\\
229	1\\
230	1\\
231	1\\
232	1\\
233	1\\
234	1\\
235	1\\
236	1\\
237	1\\
238	1\\
239	1\\
240	1\\
241	1\\
242	1\\
243	1\\
244	1\\
245	1\\
246	1\\
247	1\\
248	1\\
249	1\\
250	1\\
251	1\\
252	1\\
253	1\\
254	1\\
255	1\\
256	1\\
257	1\\
258	1\\
259	1\\
260	1\\
261	1\\
262	1\\
263	1\\
264	1\\
265	1\\
266	1\\
267	1\\
268	1\\
269	1\\
270	1\\
271	1\\
272	1\\
273	1\\
274	1\\
275	1\\
276	1\\
277	1\\
278	1\\
279	1\\
280	1\\
281	1\\
282	1\\
283	1\\
284	1\\
285	1\\
286	1\\
287	1\\
288	1\\
289	1\\
290	1\\
291	1\\
292	1\\
293	1\\
294	1\\
295	1\\
296	1\\
297	1\\
298	1\\
299	1\\
300	1\\
301	1\\
302	1\\
303	1\\
304	1\\
305	1\\
306	1\\
307	1\\
308	1\\
309	1\\
310	1\\
311	1\\
312	1\\
313	1\\
314	1\\
315	1\\
316	1\\
317	1\\
318	1\\
319	1\\
320	1\\
321	1\\
322	1\\
323	1\\
324	1\\
325	1\\
326	1\\
327	1\\
328	1\\
329	1\\
330	1\\
331	1\\
332	1\\
333	1\\
334	1\\
335	1\\
336	1\\
337	1\\
338	1\\
339	1\\
340	1\\
341	1\\
342	1\\
343	1\\
344	1\\
345	1\\
346	1\\
347	1\\
348	1\\
349	1\\
350	1\\
351	1\\
352	1\\
353	1\\
354	1\\
355	1\\
356	1\\
357	1\\
358	1\\
359	1\\
360	1\\
361	1\\
362	1\\
363	1\\
364	1\\
365	1\\
366	1\\
367	1\\
368	1\\
369	1\\
370	1\\
371	1\\
372	1\\
373	1\\
374	1\\
375	1\\
376	1\\
377	1\\
378	1\\
379	1\\
380	1\\
381	1\\
382	1\\
383	1\\
384	1\\
385	1\\
386	1\\
387	1\\
388	1\\
389	1\\
390	1\\
391	1\\
392	1\\
393	1\\
394	1\\
395	1\\
396	1\\
397	1\\
398	1\\
399	1\\
400	1\\
401	1\\
402	1\\
403	1\\
404	1\\
405	1\\
406	1\\
407	1\\
408	1\\
409	1\\
410	1\\
411	1\\
412	1\\
413	1\\
414	1\\
415	1\\
416	1\\
417	1\\
418	1\\
419	1\\
420	1\\
421	1\\
422	1\\
423	1\\
424	1\\
425	1\\
426	1\\
427	1\\
428	1\\
429	1\\
430	1\\
431	1\\
432	1\\
433	1\\
434	1\\
435	1\\
436	1\\
437	1\\
438	1\\
439	1\\
440	1\\
441	1\\
442	1\\
443	1\\
444	1\\
445	1\\
446	1\\
447	1\\
448	1\\
449	1\\
450	1\\
451	1\\
452	1\\
453	1\\
454	1\\
455	1\\
456	1\\
457	1\\
458	1\\
459	1\\
460	1\\
461	1\\
462	1\\
463	1\\
464	1\\
465	1\\
466	1\\
467	1\\
468	1\\
469	1\\
470	1\\
471	1\\
472	1\\
473	1\\
474	1\\
475	1\\
476	1\\
477	1\\
478	1\\
479	1\\
480	1\\
481	1\\
482	1\\
483	1\\
484	1\\
485	1\\
486	1\\
487	1\\
488	1\\
489	1\\
490	1\\
491	1\\
492	1\\
493	1\\
494	1\\
495	1\\
496	1\\
497	1\\
498	1\\
499	1\\
500	1\\
501	1\\
502	1\\
503	1\\
504	1\\
505	1\\
506	1\\
507	1\\
508	1\\
509	1\\
510	1\\
511	1\\
512	1\\
513	1\\
514	1\\
515	1\\
516	1\\
517	1\\
518	1\\
519	1\\
520	1\\
521	1\\
522	1\\
523	1\\
524	1\\
525	1\\
526	1\\
527	1\\
528	1\\
529	1\\
530	1\\
531	1\\
532	1\\
533	1\\
534	1\\
535	1\\
536	1\\
537	1\\
538	1\\
539	1\\
540	1\\
541	1\\
542	1\\
543	1\\
544	1\\
545	1\\
546	1\\
547	1\\
548	1\\
549	1\\
550	1\\
551	1\\
552	1\\
553	1\\
554	1\\
555	1\\
556	1\\
557	1\\
558	1\\
559	1\\
560	1\\
561	1\\
562	1\\
563	1\\
564	1\\
565	1\\
566	1\\
567	1\\
568	1\\
569	1\\
570	1\\
571	1\\
572	1\\
573	1\\
574	1\\
575	1\\
576	1\\
577	1\\
578	1\\
579	1\\
580	1\\
581	1\\
582	1\\
583	1\\
584	1\\
585	1\\
586	1\\
587	1\\
588	1\\
589	1\\
590	1\\
591	1\\
592	1\\
593	1\\
594	1\\
595	1\\
596	1\\
597	1\\
598	1\\
599	1\\
600	1\\
601	1\\
602	1\\
603	1\\
604	1\\
605	1\\
606	1\\
607	1\\
608	1\\
609	1\\
610	1\\
611	1\\
612	1\\
613	1\\
614	1\\
615	1\\
616	1\\
617	1\\
618	1\\
619	1\\
620	1\\
621	1\\
622	1\\
623	1\\
624	1\\
625	1\\
626	1\\
627	1\\
628	1\\
629	1\\
630	1\\
631	1\\
632	1\\
633	1\\
634	1\\
635	1\\
636	1\\
637	1\\
638	1\\
639	1\\
640	1\\
641	1\\
642	1\\
643	1\\
644	1\\
645	1\\
646	1\\
647	1\\
648	1\\
649	1\\
650	1\\
651	1\\
652	1\\
653	1\\
654	1\\
655	1\\
656	1\\
657	1\\
658	1\\
659	1\\
660	1\\
661	1\\
662	1\\
663	1\\
664	1\\
665	1\\
666	1\\
667	1\\
668	1\\
669	1\\
670	1\\
671	1\\
672	1\\
673	1\\
674	1\\
675	1\\
676	1\\
677	1\\
678	1\\
679	1\\
680	1\\
681	1\\
682	1\\
683	1\\
684	1\\
685	1\\
686	1\\
687	1\\
688	1\\
689	1\\
690	1\\
691	1\\
692	1\\
693	1\\
694	1\\
695	1\\
696	1\\
697	1\\
698	1\\
699	1\\
700	1\\
701	1\\
702	1\\
703	1\\
704	1\\
705	1\\
706	1\\
707	1\\
708	1\\
709	1\\
710	1\\
711	1\\
712	1\\
713	1\\
714	1\\
715	1\\
716	1\\
717	1\\
718	1\\
719	1\\
720	1\\
721	1\\
722	1\\
723	1\\
724	1\\
725	1\\
726	1\\
727	1\\
728	1\\
729	1\\
730	1\\
731	1\\
732	1\\
733	1\\
734	1\\
735	1\\
736	1\\
737	1\\
738	1\\
739	1\\
740	1\\
741	1\\
742	1\\
743	1\\
744	1\\
745	1\\
746	1\\
747	1\\
748	1\\
749	1\\
750	1\\
751	1\\
752	1\\
753	1\\
754	1\\
755	1\\
756	1\\
757	1\\
758	1\\
759	1\\
760	1\\
761	1\\
762	1\\
763	1\\
764	1\\
765	1\\
766	1\\
767	1\\
768	1\\
769	1\\
770	1\\
771	1\\
772	1\\
773	1\\
774	1\\
775	1\\
776	1\\
777	1\\
778	1\\
779	1\\
780	1\\
781	1\\
782	1\\
783	1\\
784	1\\
785	1\\
786	1\\
787	1\\
788	1\\
789	1\\
790	1\\
791	1\\
792	1\\
793	1\\
794	1\\
795	1\\
796	1\\
797	1\\
798	1\\
799	1\\
800	1\\
801	1\\
802	1\\
803	1\\
804	1\\
805	1\\
806	1\\
807	1\\
808	1\\
809	1\\
810	1\\
811	1\\
812	1\\
813	1\\
814	1\\
815	1\\
816	1\\
817	1\\
818	1\\
819	1\\
820	1\\
821	1\\
822	1\\
823	1\\
824	1\\
825	1\\
826	1\\
827	1\\
828	1\\
829	1\\
830	1\\
831	1\\
832	1\\
833	1\\
834	1\\
835	1\\
836	1\\
837	1\\
838	1\\
839	1\\
840	1\\
841	1\\
842	1\\
843	1\\
844	1\\
845	1\\
846	1\\
847	1\\
848	1\\
849	1\\
850	1\\
851	1\\
852	1\\
853	1\\
854	1\\
855	1\\
856	1\\
857	1\\
858	1\\
859	1\\
860	1\\
861	1\\
862	1\\
863	1\\
864	1\\
865	1\\
866	1\\
867	1\\
868	1\\
869	1\\
870	1\\
871	1\\
872	1\\
873	1\\
874	1\\
875	1\\
876	1\\
877	1\\
878	1\\
879	1\\
880	1\\
881	1\\
882	1\\
883	1\\
884	1\\
885	1\\
886	1\\
887	1\\
888	1\\
889	1\\
890	1\\
891	1\\
892	1\\
893	1\\
894	1\\
895	1\\
896	1\\
897	1\\
898	1\\
899	1\\
900	1\\
901	1\\
902	1\\
903	1\\
904	1\\
905	1\\
906	1\\
907	1\\
908	1\\
909	1\\
910	1\\
911	1\\
912	1\\
913	1\\
914	1\\
915	1\\
916	1\\
917	1\\
918	1\\
919	1\\
920	1\\
921	1\\
922	1\\
923	1\\
924	1\\
925	1\\
926	1\\
927	1\\
928	1\\
929	1\\
930	1\\
931	1\\
932	1\\
933	1\\
934	1\\
935	1\\
936	1\\
937	1\\
938	1\\
939	1\\
940	1\\
941	1\\
942	1\\
943	1\\
944	1\\
945	1\\
946	1\\
947	1\\
948	1\\
949	1\\
950	1\\
951	1\\
952	1\\
953	1\\
954	1\\
955	1\\
956	1\\
957	1\\
958	1\\
959	1\\
960	1\\
961	1\\
962	1\\
963	1\\
964	1\\
965	1\\
966	1\\
967	1\\
968	1\\
969	1\\
970	1\\
971	1\\
972	1\\
973	1\\
974	1\\
975	1\\
976	1\\
977	1\\
978	1\\
979	1\\
980	1\\
981	1\\
982	1\\
983	1\\
984	1\\
985	1\\
986	1\\
987	1\\
988	1\\
989	1\\
990	1\\
991	1\\
992	1\\
993	1\\
994	1\\
995	1\\
996	1\\
997	1\\
998	1\\
999	1\\
1000	1\\
1001	1\\
1002	1\\
1003	1\\
1004	1\\
1005	1\\
1006	1\\
1007	1\\
1008	1\\
1009	1\\
1010	1\\
1011	1\\
1012	1\\
1013	1\\
1014	1\\
1015	1\\
1016	1\\
1017	1\\
1018	1\\
1019	1\\
1020	1\\
1021	1\\
1022	1\\
1023	1\\
1024	1\\
1025	1\\
1026	1\\
1027	1\\
1028	1\\
1029	1\\
1030	1\\
1031	1\\
1032	1\\
1033	1\\
1034	1\\
1035	1\\
1036	1\\
1037	1\\
1038	1\\
1039	1\\
1040	1\\
1041	1\\
1042	1\\
1043	1\\
1044	1\\
1045	1\\
1046	1\\
1047	1\\
1048	1\\
1049	1\\
1050	1\\
1051	1\\
1052	1\\
1053	1\\
1054	1\\
1055	1\\
1056	1\\
1057	1\\
1058	1\\
1059	1\\
1060	1\\
1061	1\\
1062	1\\
1063	1\\
1064	1\\
1065	1\\
1066	1\\
1067	1\\
1068	1\\
1069	1\\
1070	1\\
1071	1\\
1072	1\\
1073	1\\
1074	1\\
1075	1\\
1076	1\\
1077	1\\
1078	1\\
1079	1\\
1080	1\\
1081	1\\
1082	1\\
1083	1\\
1084	1\\
1085	1\\
1086	1\\
1087	1\\
1088	1\\
1089	1\\
1090	1\\
1091	1\\
1092	1\\
1093	1\\
1094	1\\
1095	1\\
1096	1\\
1097	1\\
1098	1\\
1099	1\\
1100	1\\
1101	1\\
1102	1\\
1103	1\\
1104	1\\
1105	1\\
1106	1\\
1107	1\\
1108	1\\
1109	1\\
1110	1\\
1111	1\\
1112	1\\
1113	1\\
1114	1\\
1115	1\\
1116	1\\
1117	1\\
1118	1\\
1119	1\\
1120	1\\
1121	1\\
1122	1\\
1123	1\\
1124	1\\
1125	1\\
1126	1\\
1127	1\\
1128	1\\
1129	1\\
1130	1\\
1131	1\\
1132	1\\
1133	1\\
1134	1\\
1135	1\\
1136	1\\
1137	1\\
1138	1\\
1139	1\\
1140	1\\
1141	1\\
1142	1\\
1143	1\\
1144	1\\
1145	1\\
1146	1\\
1147	1\\
1148	1\\
1149	1\\
1150	1\\
1151	1\\
1152	1\\
1153	1\\
1154	1\\
1155	1\\
1156	1\\
1157	1\\
1158	1\\
1159	1\\
1160	1\\
1161	1\\
1162	1\\
1163	1\\
1164	1\\
1165	1\\
1166	1\\
1167	1\\
1168	1\\
1169	1\\
1170	1\\
1171	1\\
1172	1\\
1173	1\\
1174	1\\
1175	1\\
1176	1\\
1177	1\\
1178	1\\
1179	1\\
1180	1\\
1181	1\\
1182	1\\
1183	1\\
1184	1\\
1185	1\\
1186	1\\
1187	1\\
1188	1\\
1189	1\\
1190	1\\
1191	1\\
1192	1\\
1193	1\\
1194	1\\
1195	1\\
1196	1\\
1197	1\\
1198	1\\
1199	1\\
1200	1\\
1201	1\\
1202	1\\
1203	1\\
1204	1\\
1205	1\\
1206	1\\
1207	1\\
1208	1\\
1209	1\\
1210	1\\
1211	1\\
1212	1\\
1213	1\\
1214	1\\
1215	1\\
1216	1\\
1217	1\\
1218	1\\
1219	1\\
1220	1\\
1221	1\\
1222	1\\
1223	1\\
1224	1\\
1225	1\\
1226	1\\
1227	1\\
1228	1\\
1229	1\\
1230	1\\
1231	1\\
1232	1\\
1233	1\\
1234	1\\
1235	1\\
1236	1\\
1237	1\\
1238	1\\
1239	1\\
1240	1\\
1241	1\\
1242	1\\
1243	1\\
1244	1\\
1245	1\\
1246	1\\
1247	1\\
1248	1\\
1249	1\\
1250	1\\
1251	1\\
1252	1\\
1253	1\\
1254	1\\
1255	1\\
1256	1\\
1257	1\\
1258	1\\
1259	1\\
1260	1\\
1261	1\\
1262	1\\
1263	1\\
1264	1\\
1265	1\\
1266	1\\
1267	1\\
1268	1\\
1269	1\\
1270	1\\
1271	1\\
1272	1\\
1273	1\\
1274	1\\
1275	1\\
1276	1\\
1277	1\\
1278	1\\
1279	1\\
1280	1\\
1281	1\\
1282	1\\
1283	1\\
1284	1\\
1285	1\\
1286	1\\
1287	1\\
1288	1\\
1289	1\\
1290	1\\
1291	1\\
1292	1\\
1293	1\\
1294	1\\
1295	1\\
1296	1\\
1297	1\\
1298	1\\
1299	1\\
1300	1\\
1301	1\\
1302	1\\
1303	1\\
1304	1\\
1305	1\\
1306	1\\
1307	1\\
1308	1\\
1309	1\\
1310	1\\
1311	1\\
1312	1\\
1313	1\\
1314	1\\
1315	1\\
1316	1\\
1317	1\\
1318	1\\
1319	1\\
1320	1\\
1321	1\\
1322	1\\
1323	1\\
1324	1\\
1325	1\\
1326	1\\
1327	1\\
1328	1\\
1329	1\\
1330	1\\
1331	1\\
1332	1\\
1333	1\\
1334	1\\
1335	1\\
1336	1\\
1337	1\\
1338	1\\
1339	1\\
1340	1\\
1341	1\\
1342	1\\
1343	1\\
1344	1\\
1345	1\\
1346	1\\
1347	1\\
1348	1\\
1349	1\\
1350	1\\
1351	1\\
1352	1\\
1353	1\\
1354	1\\
1355	1\\
1356	1\\
1357	1\\
1358	1\\
1359	1\\
1360	1\\
1361	1\\
1362	1\\
1363	1\\
1364	1\\
1365	1\\
1366	1\\
1367	1\\
1368	1\\
1369	1\\
1370	1\\
1371	1\\
1372	1\\
1373	1\\
1374	1\\
1375	1\\
1376	1\\
1377	1\\
1378	1\\
1379	1\\
1380	1\\
1381	1\\
1382	1\\
1383	1\\
1384	1\\
1385	1\\
1386	1\\
1387	1\\
1388	1\\
1389	1\\
1390	1\\
1391	1\\
1392	1\\
1393	1\\
1394	1\\
1395	1\\
1396	1\\
1397	1\\
1398	1\\
1399	1\\
1400	1\\
1401	1\\
1402	1\\
1403	1\\
1404	1\\
1405	1\\
1406	1\\
1407	1\\
1408	1\\
1409	1\\
1410	1\\
1411	1\\
1412	1\\
1413	1\\
1414	1\\
1415	1\\
1416	1\\
1417	1\\
1418	1\\
1419	1\\
1420	1\\
1421	1\\
1422	1\\
1423	1\\
1424	1\\
1425	1\\
1426	1\\
1427	1\\
1428	1\\
1429	1\\
1430	1\\
1431	1\\
1432	1\\
1433	1\\
1434	1\\
1435	1\\
1436	1\\
1437	1\\
1438	1\\
1439	1\\
1440	1\\
1441	1\\
1442	1\\
1443	1\\
1444	1\\
1445	1\\
1446	1\\
1447	1\\
1448	1\\
1449	1\\
1450	1\\
1451	1\\
1452	1\\
1453	1\\
1454	1\\
1455	1\\
1456	1\\
1457	1\\
1458	1\\
1459	1\\
1460	1\\
1461	1\\
1462	1\\
1463	1\\
1464	1\\
1465	1\\
1466	1\\
1467	1\\
1468	1\\
1469	1\\
1470	1\\
1471	1\\
1472	1\\
1473	1\\
1474	1\\
1475	1\\
1476	1\\
1477	1\\
1478	1\\
1479	1\\
1480	1\\
1481	1\\
1482	1\\
1483	1\\
1484	1\\
1485	1\\
1486	1\\
1487	1\\
1488	1\\
1489	1\\
1490	1\\
1491	1\\
1492	1\\
1493	1\\
1494	1\\
1495	1\\
1496	1\\
1497	1\\
1498	1\\
1499	1\\
1500	1\\
1501	1\\
1502	1\\
1503	1\\
1504	1\\
1505	1\\
1506	1\\
1507	1\\
1508	1\\
1509	1\\
1510	1\\
1511	1\\
1512	1\\
1513	1\\
1514	1\\
1515	1\\
1516	1\\
1517	1\\
1518	1\\
1519	1\\
1520	1\\
1521	1\\
1522	1\\
1523	1\\
1524	1\\
1525	1\\
1526	1\\
1527	1\\
1528	1\\
1529	1\\
1530	1\\
1531	1\\
1532	1\\
1533	1\\
1534	1\\
1535	1\\
1536	1\\
1537	1\\
1538	1\\
1539	1\\
1540	1\\
1541	1\\
1542	1\\
1543	1\\
1544	1\\
1545	1\\
1546	1\\
1547	1\\
1548	1\\
1549	1\\
1550	1\\
1551	1\\
1552	1\\
1553	1\\
1554	1\\
1555	1\\
1556	1\\
1557	1\\
1558	1\\
1559	1\\
1560	1\\
1561	1\\
1562	1\\
1563	1\\
1564	1\\
1565	1\\
1566	1\\
1567	1\\
1568	1\\
1569	1\\
1570	1\\
1571	1\\
1572	1\\
1573	1\\
1574	1\\
1575	1\\
1576	1\\
1577	1\\
1578	1\\
1579	1\\
1580	1\\
1581	1\\
1582	1\\
1583	1\\
1584	1\\
1585	1\\
1586	1\\
1587	1\\
1588	1\\
1589	1\\
1590	1\\
1591	1\\
1592	1\\
1593	1\\
1594	1\\
1595	1\\
1596	1\\
1597	1\\
1598	1\\
1599	1\\
1600	1\\
1601	1\\
1602	1\\
1603	1\\
1604	1\\
1605	1\\
1606	1\\
1607	1\\
1608	1\\
1609	1\\
1610	1\\
1611	1\\
1612	1\\
1613	1\\
1614	1\\
1615	1\\
1616	1\\
1617	1\\
1618	1\\
1619	1\\
1620	1\\
1621	1\\
1622	1\\
1623	1\\
1624	1\\
1625	1\\
1626	1\\
1627	1\\
1628	1\\
1629	1\\
1630	1\\
1631	1\\
1632	1\\
1633	1\\
1634	1\\
1635	1\\
1636	1\\
1637	1\\
1638	1\\
1639	1\\
1640	1\\
1641	1\\
1642	1\\
1643	1\\
1644	1\\
1645	1\\
1646	1\\
1647	1\\
1648	1\\
1649	1\\
1650	1\\
1651	1\\
1652	1\\
1653	1\\
1654	1\\
1655	1\\
1656	1\\
1657	1\\
1658	1\\
1659	1\\
1660	1\\
1661	1\\
1662	1\\
1663	1\\
1664	1\\
1665	1\\
1666	1\\
1667	1\\
1668	1\\
1669	1\\
1670	1\\
1671	1\\
1672	1\\
1673	1\\
1674	1\\
1675	1\\
1676	1\\
1677	1\\
1678	1\\
1679	1\\
1680	1\\
1681	1\\
1682	1\\
1683	1\\
1684	1\\
1685	1\\
1686	1\\
1687	1\\
1688	1\\
1689	1\\
1690	1\\
1691	1\\
1692	1\\
1693	1\\
1694	1\\
1695	1\\
1696	1\\
1697	1\\
1698	1\\
1699	1\\
1700	1\\
1701	1\\
1702	1\\
1703	1\\
1704	1\\
1705	1\\
1706	1\\
1707	1\\
1708	1\\
1709	1\\
1710	1\\
1711	1\\
1712	1\\
1713	1\\
1714	1\\
1715	1\\
1716	1\\
1717	1\\
1718	1\\
1719	1\\
1720	1\\
1721	1\\
1722	1\\
1723	1\\
1724	1\\
1725	1\\
1726	1\\
1727	1\\
1728	1\\
1729	1\\
1730	1\\
1731	1\\
1732	1\\
1733	1\\
1734	1\\
1735	1\\
1736	1\\
1737	1\\
1738	1\\
};
\end{axis}
\end{tikzpicture}%}
  \caption{The overall processing usage. Notice that it is at most at $100\%$.
     $C_i = 6$ ms.}
    \label{fig:02.4.2}
\end{figure}
