\subsection{Question 2}

In order for the system to be stable, the eigenvalues of matrix

\begin{align*}
  \begin{bmatrix}
    1 - (h-\tau)K & \tau \\
    -K            & 0 \\
  \end{bmatrix}
\end{align*}

should lie inside the unit circle.

These eigenvalues $\lambda_i$ can be found from

\begin{align*}
  \Bigg| \lambda I -
  \begin{bmatrix}
    1 - (h-\tau)K & \tau \\
    -K            & 0 \\
  \end{bmatrix}
  \Bigg|
  &=  0 \\
  \Bigg|
  \begin{bmatrix}
    \lambda -1 + (h-\tau)K & -\tau \\
    K                     & \lambda \\
  \end{bmatrix}
  \Bigg|
  &= 0 \\
  \lambda^2 + \lambda (K(h-\tau)-1) + K\tau &= 0 \\
\end{align*}

Where $\prod \lambda_i = K\tau$ and $\sum \lambda_i = -(K(h-\tau)-1)$.

The stability triangle dictates that the system is stable if the following
inequalities hold:

\begin{align*}
  \left.\begin{aligned}
      K\tau &<  1 \\
      K\tau &< K(h-\tau) -1 -1 \\
      K\tau &> -K(h-\tau) +1 -1 \\
  \end{aligned}\ \right\} \Leftrightarrow
\end{align*}

\begin{align*}
  \dfrac{Kh-2}{2} < &\ K\tau < 1  \Leftrightarrow \\
  \dfrac{1}{2} - \dfrac{1}{Kh} < &\ \dfrac{\tau}{h} < \dfrac{1}{Kh}
\end{align*}

And, since $K > 0$ and $0 \leq \tau \leq h$:

$$ \text{max} \Bigg(0, \dfrac{1}{2} - \dfrac{1}{Kh}\Bigg)
< \dfrac{\tau}{h} <
\text{min} \Bigg(1, \dfrac{1}{Kh} \Bigg) $$

The stability region is illustrated in figure \ref{fig:03.stability_region}.
The system is stable if the combination of the $Kh$ and $\tau/h$ quantities lies
inside the area bounded by the curves and the vertical axis.

\begin{figure}[H]\centering
  \scalebox{1}{% This file was created by matlab2tikz.
%
%The latest updates can be retrieved from
%  http://www.mathworks.com/matlabcentral/fileexchange/22022-matlab2tikz-matlab2tikz
%where you can also make suggestions and rate matlab2tikz.
%
\definecolor{mycolor1}{rgb}{0.00000,0.44700,0.74100}%
\definecolor{mycolor2}{rgb}{0.85000,0.32500,0.09800}%
\definecolor{mycolor3}{rgb}{0.92900,0.69400,0.12500}%
\definecolor{mycolor4}{rgb}{0.49400,0.18400,0.55600}%
\definecolor{mycolor5}{rgb}{0.46600,0.67400,0.18800}%
\definecolor{mycolor6}{rgb}{0.30100,0.74500,0.93300}%
\definecolor{mycolor7}{rgb}{0.63500,0.07800,0.18400}%
%
\begin{tikzpicture}

\begin{axis}[%
width=4.133in,
height=3.26in,
at={(0.693in,0.44in)},
scale only axis,
unbounded coords=jump,
xmin=0,
xmax=41,
xmajorgrids,
ymin=-0.1,
ymax=1.1,
ymajorgrids,
axis background/.style={fill=white}
]
\addplot [color=mycolor1,solid,forget plot]
  table[row sep=crcr]{%
1	-inf\\
2	-9.5\\
3	-4.5\\
4	-2.83333333333333\\
5	-2\\
6	-1.5\\
7	-1.16666666666667\\
8	-0.928571428571428\\
9	-0.75\\
10	-0.611111111111111\\
11	-0.5\\
12	-0.409090909090909\\
13	-0.333333333333333\\
14	-0.269230769230769\\
15	-0.214285714285714\\
16	-0.166666666666667\\
17	-0.125\\
18	-0.088235294117647\\
19	-0.0555555555555556\\
20	-0.0263157894736842\\
21	0\\
22	0.0238095238095238\\
23	0.0454545454545455\\
24	0.0652173913043479\\
25	0.0833333333333334\\
26	0.1\\
27	0.115384615384615\\
28	0.12962962962963\\
29	0.142857142857143\\
30	0.155172413793103\\
31	0.166666666666667\\
32	0.17741935483871\\
33	0.1875\\
34	0.196969696969697\\
35	0.205882352941177\\
36	0.214285714285714\\
37	0.222222222222222\\
38	0.22972972972973\\
39	0.236842105263158\\
40	0.243589743589744\\
41	0.25\\
42	0.25609756097561\\
43	0.261904761904762\\
44	0.267441860465116\\
45	0.272727272727273\\
46	0.277777777777778\\
47	0.282608695652174\\
48	0.287234042553192\\
49	0.291666666666667\\
50	0.295918367346939\\
51	0.3\\
52	0.303921568627451\\
53	0.307692307692308\\
54	0.311320754716981\\
55	0.314814814814815\\
56	0.318181818181818\\
57	0.321428571428571\\
58	0.324561403508772\\
59	0.327586206896552\\
60	0.330508474576271\\
61	0.333333333333333\\
62	0.336065573770492\\
63	0.338709677419355\\
64	0.341269841269841\\
65	0.34375\\
66	0.346153846153846\\
67	0.348484848484849\\
68	0.350746268656716\\
69	0.352941176470588\\
70	0.355072463768116\\
71	0.357142857142857\\
72	0.359154929577465\\
73	0.361111111111111\\
74	0.363013698630137\\
75	0.364864864864865\\
76	0.366666666666667\\
77	0.368421052631579\\
78	0.37012987012987\\
79	0.371794871794872\\
80	0.373417721518987\\
81	0.375\\
82	0.376543209876543\\
83	0.378048780487805\\
84	0.379518072289157\\
85	0.380952380952381\\
86	0.382352941176471\\
87	0.383720930232558\\
88	0.385057471264368\\
89	0.386363636363636\\
90	0.387640449438202\\
91	0.388888888888889\\
92	0.39010989010989\\
93	0.391304347826087\\
94	0.39247311827957\\
95	0.393617021276596\\
96	0.394736842105263\\
97	0.395833333333333\\
98	0.396907216494845\\
99	0.397959183673469\\
100	0.398989898989899\\
101	0.4\\
};
\addplot [color=mycolor2,solid,forget plot]
  table[row sep=crcr]{%
1	inf\\
2	10\\
3	5\\
4	3.33333333333333\\
5	2.5\\
6	2\\
7	1.66666666666667\\
8	1.42857142857143\\
9	1.25\\
10	1.11111111111111\\
11	1\\
12	0.909090909090909\\
13	0.833333333333333\\
14	0.769230769230769\\
15	0.714285714285714\\
16	0.666666666666667\\
17	0.625\\
18	0.588235294117647\\
19	0.555555555555556\\
20	0.526315789473684\\
21	0.5\\
22	0.476190476190476\\
23	0.454545454545455\\
24	0.434782608695652\\
25	0.416666666666667\\
26	0.4\\
27	0.384615384615385\\
28	0.37037037037037\\
29	0.357142857142857\\
30	0.344827586206897\\
31	0.333333333333333\\
32	0.32258064516129\\
33	0.3125\\
34	0.303030303030303\\
35	0.294117647058823\\
36	0.285714285714286\\
37	0.277777777777778\\
38	0.27027027027027\\
39	0.263157894736842\\
40	0.256410256410256\\
41	0.25\\
42	0.24390243902439\\
43	0.238095238095238\\
44	0.232558139534884\\
45	0.227272727272727\\
46	0.222222222222222\\
47	0.217391304347826\\
48	0.212765957446809\\
49	0.208333333333333\\
50	0.204081632653061\\
51	0.2\\
52	0.196078431372549\\
53	0.192307692307692\\
54	0.188679245283019\\
55	0.185185185185185\\
56	0.181818181818182\\
57	0.178571428571429\\
58	0.175438596491228\\
59	0.172413793103448\\
60	0.169491525423729\\
61	0.166666666666667\\
62	0.163934426229508\\
63	0.161290322580645\\
64	0.158730158730159\\
65	0.15625\\
66	0.153846153846154\\
67	0.151515151515152\\
68	0.149253731343284\\
69	0.147058823529412\\
70	0.144927536231884\\
71	0.142857142857143\\
72	0.140845070422535\\
73	0.138888888888889\\
74	0.136986301369863\\
75	0.135135135135135\\
76	0.133333333333333\\
77	0.131578947368421\\
78	0.12987012987013\\
79	0.128205128205128\\
80	0.126582278481013\\
81	0.125\\
82	0.123456790123457\\
83	0.121951219512195\\
84	0.120481927710843\\
85	0.119047619047619\\
86	0.117647058823529\\
87	0.116279069767442\\
88	0.114942528735632\\
89	0.113636363636364\\
90	0.112359550561798\\
91	0.111111111111111\\
92	0.10989010989011\\
93	0.108695652173913\\
94	0.10752688172043\\
95	0.106382978723404\\
96	0.105263157894737\\
97	0.104166666666667\\
98	0.103092783505155\\
99	0.102040816326531\\
100	0.101010101010101\\
101	0.1\\
};
\addplot [color=mycolor3,solid,forget plot]
  table[row sep=crcr]{%
1	1\\
2	1\\
3	1\\
4	1\\
5	1\\
6	1\\
7	1\\
8	1\\
9	1\\
10	1\\
11	1\\
};
\addplot [color=mycolor4,solid,forget plot]
  table[row sep=crcr]{%
1	0\\
2	0\\
3	0\\
4	0\\
5	0\\
6	0\\
7	0\\
8	0\\
9	0\\
10	0\\
11	0\\
12	0\\
13	0\\
14	0\\
15	0\\
16	0\\
17	0\\
18	0\\
19	0\\
20	0\\
21	0\\
};
\end{axis}
\end{tikzpicture}%
}
  \caption{The stability region is defined here as the area in the $x,y$ plane
    that is bounded by the curves: $x=0$, $y=1$, $y=0$, $y = 0.5-1/x$ and
    $y=1/x$. The $x$ axis represents the product $Kh$. The $y$ axis represents
    the magnitude of the fraction $\tau/h$.}
  \label{fig:03.stability_region}
\end{figure}
