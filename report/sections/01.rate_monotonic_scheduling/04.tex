\subsubsection{Question 4}

Figure \ref{fig:01.4.1.small} shows the executed schedule magnified over the
period of the first 60 ms.
As per the response to question 2, figure \ref{fig:01.4.2} illustrates that the
schedule is indeed feasible by plotting the overall usage of the CPU over the
aforementioned timespan.

\begin{sidewaysfigure}
  \begin{figure}[H]\centering
    \scalebox{1}{% This file was created by matlab2tikz.
%
%The latest updates can be retrieved from
%  http://www.mathworks.com/matlabcentral/fileexchange/22022-matlab2tikz-matlab2tikz
%where you can also make suggestions and rate matlab2tikz.
%
\definecolor{mycolor1}{rgb}{0.00000,0.44700,0.74100}%
\definecolor{mycolor2}{rgb}{0.85000,0.32500,0.09800}%
\definecolor{mycolor3}{rgb}{0.92900,0.69400,0.12500}%
\definecolor{mycolor4}{rgb}{0.63500,0.07800,0.18400}%
%
\begin{tikzpicture}

\begin{axis}[%
width=4.133in,
height=0.576in,
at={(0.693in,3.124in)},
scale only axis,
xmin=0,
xmax=0.06,
ymin=0,
ymax=1.1,
axis background/.style={fill=white}
]
\pgfplotsset{max space between ticks=50}
\addplot [color=mycolor1,solid,forget plot]
  table[row sep=crcr]{%
0	0\\
3.15544362088405e-30	1\\
0.000656101980281985	1\\
0.00393661188169191	1\\
0.00599999999999994	1\\
0.006	0\\
0.012	0\\
0.0120000000000001	0\\
0.018	0\\
0.0180000000000001	0\\
0.0199999999999998	0\\
0.02	1\\
0.026	1\\
0.0260000000000002	0\\
0.0289999999999998	0\\
0.029	0\\
0.0319999999999996	0\\
0.0349999999999991	0\\
0.035	0\\
0.0399999999999996	0\\
0.04	1\\
0.0449999999999996	1\\
0.0459999999999996	1\\
0.046	0\\
0.047	0\\
0.0470000000000004	0\\
0.0490000000000003	0\\
0.0510000000000002	0\\
0.055	0\\
0.0579999999999996	0\\
0.058	0\\
0.0599999999999996	0\\
0.06	1\\
0.0619999999999995	1\\
0.0639999999999991	1\\
0.0659999999999991	1\\
0.066	0\\
0.0699999999999991	0\\
0.07	0\\
0.0700000000000009	0\\
0.074	0\\
0.076	0\\
0.0760000000000009	0\\
0.08	0\\
0.0800000000000009	1\\
0.0839999999999999	1\\
0.086	1\\
0.0860000000000009	0\\
0.0869999999999991	0\\
0.087	0\\
0.0880000000000004	0\\
0.0890000000000009	0\\
0.0910000000000017	0\\
0.0929999999999991	0\\
0.093	0\\
0.0970000000000017	0\\
0.0999999999999991	0\\
0.1	1\\
0.104000000000002	1\\
0.104999999999999	1\\
0.105	1\\
0.105999999999999	1\\
0.106	0\\
0.106999999999999	0\\
0.107999999999998	0\\
0.109999999999997	0\\
0.111999999999999	0\\
0.112	0\\
0.115999999999997	0\\
0.115999999999998	0\\
0.116	0\\
0.119999999999997	0\\
0.119999999999998	0\\
0.12	1\\
0.123999999999997	1\\
0.125999999999999	1\\
0.126	0\\
0.127999999999998	0\\
0.128	0\\
0.129999999999998	0\\
0.131999999999996	0\\
0.135999999999993	0\\
0.139999999999998	0\\
0.14	1\\
0.144999999999998	1\\
0.145	1\\
0.145999999999998	1\\
0.146	0\\
0.146999999999999	0\\
0.147999999999998	0\\
0.149999999999997	0\\
0.151999999999998	0\\
0.152	0\\
0.155999999999997	0\\
0.157999999999998	0\\
0.158	0\\
0.16	0\\
0.160000000000002	1\\
0.162000000000002	1\\
0.164000000000002	1\\
0.166	1\\
0.166000000000002	0\\
0.170000000000002	0\\
0.174	0\\
0.174000000000001	0\\
0.175	0\\
0.175000000000002	0\\
0.176000000000001	0\\
0.177	0\\
0.178999999999998	0\\
0.179999999999998	0\\
0.18	1\\
0.183999999999997	1\\
0.186	1\\
0.186000000000002	0\\
0.189999999999998	0\\
0.192	0\\
0.192000000000002	0\\
0.195999999999998	0\\
0.199999999999995	0\\
0.199999999999997	0\\
0.2	1\\
0.202999999999998	1\\
0.203	1\\
0.205999999999998	1\\
0.206	0\\
0.208999999999998	0\\
0.209999999999998	0\\
0.21	0\\
0.211999999999998	0\\
0.212	0\\
0.213999999999998	0\\
0.215999999999997	0\\
0.217999999999998	0\\
0.218	0\\
0.219999999999998	0\\
0.22	1\\
0.221999999999998	1\\
0.223999999999996	1\\
0.225999999999998	1\\
0.226	0\\
0.229999999999996	0\\
0.231999999999998	0\\
0.232	0\\
0.235999999999996	0\\
0.237999999999998	0\\
0.238	0\\
0.239999999999998	0\\
0.24	1\\
0.241999999999998	1\\
0.243999999999996	1\\
0.245	1\\
0.245000000000002	1\\
0.245999999999998	1\\
0.246	0\\
0.246999999999999	0\\
0.247999999999998	0\\
0.249999999999997	0\\
0.252	0\\
0.252000000000003	0\\
0.256	0\\
0.259999999999997	0\\
0.26	1\\
0.260999999999996	1\\
0.261	1\\
0.261999999999998	1\\
0.262999999999996	1\\
0.264999999999993	1\\
0.265999999999997	1\\
0.266	0\\
0.269999999999993	0\\
0.271999999999997	0\\
0.272	0\\
0.275999999999993	0\\
0.279999999999986	0\\
0.279999999999993	0\\
0.28	1\\
0.285999999999996	1\\
0.286	0\\
0.289999999999996	0\\
0.29	0\\
0.293999999999996	0\\
0.295999999999997	0\\
0.296	0\\
0.297999999999997	0\\
0.298	0\\
0.299999999999997	0\\
0.3	1\\
0.301999999999997	1\\
0.303999999999993	1\\
0.305999999999997	1\\
0.306	0\\
0.309999999999993	0\\
0.313999999999986	0\\
0.314999999999997	0\\
0.315	0\\
0.318999999999997	0\\
0.319	0\\
0.319999999999996	0\\
0.32	1\\
0.320999999999998	1\\
0.321999999999996	1\\
0.323999999999993	1\\
0.325999999999996	1\\
0.326	0\\
0.329999999999993	0\\
0.331	0\\
0.331000000000004	0\\
0.333	0\\
0.333000000000004	0\\
0.335	0\\
0.336999999999996	0\\
0.339999999999996	0\\
0.34	1\\
0.343999999999993	1\\
0.345999999999997	1\\
0.346	0\\
0.347999999999997	0\\
0.348	0\\
0.349999999999997	0\\
0.35	0\\
0.351999999999997	0\\
0.353999999999993	0\\
0.354	0\\
0.357999999999993	0\\
0.359999999999996	0\\
0.36	1\\
0.363999999999993	1\\
0.365999999999996	1\\
0.366	0\\
0.369999999999993	0\\
0.373999999999986	0\\
0.376999999999997	0\\
0.377	0\\
0.379999999999997	0\\
0.38	1\\
0.382999999999996	1\\
0.384999999999997	1\\
0.385	1\\
0.385999999999997	1\\
0.386	0\\
0.386999999999998	0\\
0.387999999999996	0\\
0.388999999999997	0\\
0.389	0\\
0.390999999999997	0\\
0.392999999999993	0\\
0.394999999999997	0\\
0.395	0\\
0.398999999999993	0\\
0.399999999999997	0\\
0.4	1\\
0.403999999999993	1\\
0.405999999999997	1\\
0.406	0\\
0.409999999999993	0\\
0.411999999999997	0\\
0.412	0\\
0.415999999999993	0\\
0.419999999999986	0\\
0.419999999999996	0\\
0.42	1\\
0.426	1\\
0.426000000000004	0\\
0.432000000000004	0\\
0.432000000000007	0\\
0.434999999999997	0\\
0.435	0\\
0.43799999999999	0\\
0.439999999999997	0\\
0.44	1\\
0.44299999999999	1\\
0.445999999999979	1\\
0.445999999999995	1\\
0.446	0\\
0.447	0\\
0.447000000000004	0\\
0.448000000000004	0\\
0.449000000000004	0\\
0.451000000000004	0\\
0.454999999999997	0\\
0.455	0\\
0.459	0\\
0.459999999999997	0\\
0.46	1\\
0.463999999999997	1\\
0.464	1\\
0.465999999999997	1\\
0.466	0\\
0.467999999999996	0\\
0.469999999999993	0\\
0.471999999999997	0\\
0.472	0\\
0.473	0\\
0.473000000000004	0\\
0.474000000000004	0\\
0.475000000000004	0\\
0.477000000000004	0\\
0.479999999999997	0\\
0.48	1\\
0.484	1\\
0.485999999999997	1\\
0.486	0\\
0.489999999999997	0\\
0.49	0\\
0.492999999999997	0\\
0.493	0\\
0.495999999999997	0\\
0.498999999999993	0\\
0.499	0\\
0.499999999999997	0\\
0.5	1\\
0.500999999999998	1\\
0.501999999999997	1\\
0.503999999999993	1\\
0.505999999999993	1\\
0.506	0\\
0.507999999999993	0\\
0.508	0\\
0.509999999999993	0\\
0.511999999999986	0\\
0.515999999999972	0\\
0.519999999999993	0\\
0.52	1\\
0.521999999999993	1\\
0.522	1\\
0.523999999999993	1\\
0.524999999999993	1\\
0.525	1\\
0.525999999999993	1\\
0.526	0\\
0.526999999999998	0\\
0.527999999999997	0\\
0.529999999999993	0\\
0.531999999999993	0\\
0.532	0\\
0.535999999999993	0\\
0.538	0\\
0.538000000000007	0\\
0.539999999999993	0\\
0.54	1\\
0.541999999999986	1\\
0.543999999999972	1\\
0.545999999999993	1\\
0.546	0\\
0.549999999999972	0\\
0.550999999999993	0\\
0.551	0\\
0.554999999999972	0\\
0.556999999999993	0\\
0.557	0\\
0.559999999999993	0\\
0.56	1\\
0.562999999999993	1\\
0.565999999999986	1\\
0.565999999999993	1\\
0.566	0\\
0.571999999999986	0\\
0.571999999999993	0\\
0.572	0\\
0.577999999999986	0\\
0.579999999999993	0\\
0.58	1\\
0.585999999999986	1\\
0.585999999999993	1\\
0.586	0\\
0.591999999999986	0\\
0.591999999999993	0\\
0.592	0\\
0.594999999999993	0\\
0.595	0\\
0.597999999999993	0\\
0.599999999999993	0\\
0.6	1\\
};
\end{axis}

\begin{axis}[%
width=4.133in,
height=0.576in,
at={(0.693in,2.248in)},
scale only axis,
xmin=0,
xmax=0.06,
ymin=0,
ymax=1.1,
axis background/.style={fill=white}
]
\pgfplotsset{max space between ticks=50}
\addplot [color=mycolor2,solid,forget plot]
  table[row sep=crcr]{%
0	0\\
3.15544362088405e-30	0.5\\
0.000656101980281985	0.5\\
0.00393661188169191	0.5\\
0.00599999999999994	0.5\\
0.006	1\\
0.012	1\\
0.0120000000000001	0\\
0.018	0\\
0.0180000000000001	0\\
0.0199999999999998	0\\
0.02	0\\
0.026	0\\
0.0260000000000002	0\\
0.0289999999999998	0\\
0.029	1\\
0.0319999999999996	1\\
0.0349999999999991	1\\
0.035	0\\
0.0399999999999996	0\\
0.04	0\\
0.0449999999999996	0\\
0.0459999999999996	0\\
0.046	0\\
0.047	0\\
0.0470000000000004	0\\
0.0490000000000003	0\\
0.0510000000000002	0\\
0.055	0\\
0.0579999999999996	0\\
0.058	1\\
0.0599999999999996	1\\
0.06	0.5\\
0.0619999999999995	0.5\\
0.0639999999999991	0.5\\
0.0659999999999991	0.5\\
0.066	1\\
0.0699999999999991	1\\
0.07	1\\
0.0700000000000009	0\\
0.074	0\\
0.076	0\\
0.0760000000000009	0\\
0.08	0\\
0.0800000000000009	0\\
0.0839999999999999	0\\
0.086	0\\
0.0860000000000009	0\\
0.0869999999999991	0\\
0.087	1\\
0.0880000000000004	1\\
0.0890000000000009	1\\
0.0910000000000017	1\\
0.0929999999999991	1\\
0.093	0\\
0.0970000000000017	0\\
0.0999999999999991	0\\
0.1	0\\
0.104000000000002	0\\
0.104999999999999	0\\
0.105	0\\
0.105999999999999	0\\
0.106	0\\
0.106999999999999	0\\
0.107999999999998	0\\
0.109999999999997	0\\
0.111999999999999	0\\
0.112	0\\
0.115999999999997	0\\
0.115999999999998	0\\
0.116	1\\
0.119999999999997	1\\
0.119999999999998	1\\
0.12	0.5\\
0.123999999999997	0.5\\
0.125999999999999	0.5\\
0.126	1\\
0.127999999999998	1\\
0.128	0\\
0.129999999999998	0\\
0.131999999999996	0\\
0.135999999999993	0\\
0.139999999999998	0\\
0.14	0\\
0.144999999999998	0\\
0.145	0.5\\
0.145999999999998	0.5\\
0.146	1\\
0.146999999999999	1\\
0.147999999999998	1\\
0.149999999999997	1\\
0.151999999999998	1\\
0.152	0\\
0.155999999999997	0\\
0.157999999999998	0\\
0.158	0\\
0.16	0\\
0.160000000000002	0\\
0.162000000000002	0\\
0.164000000000002	0\\
0.166	0\\
0.166000000000002	0\\
0.170000000000002	0\\
0.174	0\\
0.174000000000001	1\\
0.175	1\\
0.175000000000002	1\\
0.176000000000001	1\\
0.177	1\\
0.178999999999998	1\\
0.179999999999998	1\\
0.18	0\\
0.183999999999997	0\\
0.186	0\\
0.186000000000002	0\\
0.189999999999998	0\\
0.192	0\\
0.192000000000002	0\\
0.195999999999998	0\\
0.199999999999995	0\\
0.199999999999997	0\\
0.2	0\\
0.202999999999998	0\\
0.203	0.5\\
0.205999999999998	0.5\\
0.206	1\\
0.208999999999998	1\\
0.209999999999998	1\\
0.21	1\\
0.211999999999998	1\\
0.212	0\\
0.213999999999998	0\\
0.215999999999997	0\\
0.217999999999998	0\\
0.218	0\\
0.219999999999998	0\\
0.22	0\\
0.221999999999998	0\\
0.223999999999996	0\\
0.225999999999998	0\\
0.226	0\\
0.229999999999996	0\\
0.231999999999998	0\\
0.232	1\\
0.235999999999996	1\\
0.237999999999998	1\\
0.238	0\\
0.239999999999998	0\\
0.24	0\\
0.241999999999998	0\\
0.243999999999996	0\\
0.245	0\\
0.245000000000002	0\\
0.245999999999998	0\\
0.246	0\\
0.246999999999999	0\\
0.247999999999998	0\\
0.249999999999997	0\\
0.252	0\\
0.252000000000003	0\\
0.256	0\\
0.259999999999997	0\\
0.26	0\\
0.260999999999996	0\\
0.261	0.5\\
0.261999999999998	0.5\\
0.262999999999996	0.5\\
0.264999999999993	0.5\\
0.265999999999997	0.5\\
0.266	1\\
0.269999999999993	1\\
0.271999999999997	1\\
0.272	0\\
0.275999999999993	0\\
0.279999999999986	0\\
0.279999999999993	0\\
0.28	0\\
0.285999999999996	0\\
0.286	0\\
0.289999999999996	0\\
0.29	1\\
0.293999999999996	1\\
0.295999999999997	1\\
0.296	0\\
0.297999999999997	0\\
0.298	0\\
0.299999999999997	0\\
0.3	0\\
0.301999999999997	0\\
0.303999999999993	0\\
0.305999999999997	0\\
0.306	0\\
0.309999999999993	0\\
0.313999999999986	0\\
0.314999999999997	0\\
0.315	0\\
0.318999999999997	0\\
0.319	1\\
0.319999999999996	1\\
0.32	0.5\\
0.320999999999998	0.5\\
0.321999999999996	0.5\\
0.323999999999993	0.5\\
0.325999999999996	0.5\\
0.326	1\\
0.329999999999993	1\\
0.331	1\\
0.331000000000004	0\\
0.333	0\\
0.333000000000004	0\\
0.335	0\\
0.336999999999996	0\\
0.339999999999996	0\\
0.34	0\\
0.343999999999993	0\\
0.345999999999997	0\\
0.346	0\\
0.347999999999997	0\\
0.348	1\\
0.349999999999997	1\\
0.35	1\\
0.351999999999997	1\\
0.353999999999993	1\\
0.354	0\\
0.357999999999993	0\\
0.359999999999996	0\\
0.36	0\\
0.363999999999993	0\\
0.365999999999996	0\\
0.366	0\\
0.369999999999993	0\\
0.373999999999986	0\\
0.376999999999997	0\\
0.377	1\\
0.379999999999997	1\\
0.38	0.5\\
0.382999999999996	0.5\\
0.384999999999997	0.5\\
0.385	0.5\\
0.385999999999997	0.5\\
0.386	1\\
0.386999999999998	1\\
0.387999999999996	1\\
0.388999999999997	1\\
0.389	0\\
0.390999999999997	0\\
0.392999999999993	0\\
0.394999999999997	0\\
0.395	0\\
0.398999999999993	0\\
0.399999999999997	0\\
0.4	0\\
0.403999999999993	0\\
0.405999999999997	0\\
0.406	1\\
0.409999999999993	1\\
0.411999999999997	1\\
0.412	0\\
0.415999999999993	0\\
0.419999999999986	0\\
0.419999999999996	0\\
0.42	0\\
0.426	0\\
0.426000000000004	0\\
0.432000000000004	0\\
0.432000000000007	0\\
0.434999999999997	0\\
0.435	1\\
0.43799999999999	1\\
0.439999999999997	1\\
0.44	0.5\\
0.44299999999999	0.5\\
0.445999999999979	0.5\\
0.445999999999995	0.5\\
0.446	1\\
0.447	1\\
0.447000000000004	0\\
0.448000000000004	0\\
0.449000000000004	0\\
0.451000000000004	0\\
0.454999999999997	0\\
0.455	0\\
0.459	0\\
0.459999999999997	0\\
0.46	0\\
0.463999999999997	0\\
0.464	0.5\\
0.465999999999997	0.5\\
0.466	1\\
0.467999999999996	1\\
0.469999999999993	1\\
0.471999999999997	1\\
0.472	0\\
0.473	0\\
0.473000000000004	0\\
0.474000000000004	0\\
0.475000000000004	0\\
0.477000000000004	0\\
0.479999999999997	0\\
0.48	0\\
0.484	0\\
0.485999999999997	0\\
0.486	0\\
0.489999999999997	0\\
0.49	0\\
0.492999999999997	0\\
0.493	1\\
0.495999999999997	1\\
0.498999999999993	1\\
0.499	0\\
0.499999999999997	0\\
0.5	0\\
0.500999999999998	0\\
0.501999999999997	0\\
0.503999999999993	0\\
0.505999999999993	0\\
0.506	0\\
0.507999999999993	0\\
0.508	0\\
0.509999999999993	0\\
0.511999999999986	0\\
0.515999999999972	0\\
0.519999999999993	0\\
0.52	0\\
0.521999999999993	0\\
0.522	0.5\\
0.523999999999993	0.5\\
0.524999999999993	0.5\\
0.525	0.5\\
0.525999999999993	0.5\\
0.526	1\\
0.526999999999998	1\\
0.527999999999997	1\\
0.529999999999993	1\\
0.531999999999993	1\\
0.532	0\\
0.535999999999993	0\\
0.538	0\\
0.538000000000007	0\\
0.539999999999993	0\\
0.54	0\\
0.541999999999986	0\\
0.543999999999972	0\\
0.545999999999993	0\\
0.546	0\\
0.549999999999972	0\\
0.550999999999993	0\\
0.551	1\\
0.554999999999972	1\\
0.556999999999993	1\\
0.557	0\\
0.559999999999993	0\\
0.56	0\\
0.562999999999993	0\\
0.565999999999986	0\\
0.565999999999993	0\\
0.566	0\\
0.571999999999986	0\\
0.571999999999993	0\\
0.572	0\\
0.577999999999986	0\\
0.579999999999993	0\\
0.58	0.5\\
0.585999999999986	0.5\\
0.585999999999993	0.5\\
0.586	1\\
0.591999999999986	1\\
0.591999999999993	1\\
0.592	0\\
0.594999999999993	0\\
0.595	0\\
0.597999999999993	0\\
0.599999999999993	0\\
0.6	0\\
};
\end{axis}

\begin{axis}[%
width=4.133in,
height=0.576in,
at={(0.693in,1.372in)},
scale only axis,
xmin=0,
xmax=0.06,
ymin=0,
ymax=1.1,
axis background/.style={fill=white}
]
\pgfplotsset{max space between ticks=50}
\addplot [color=mycolor3,solid,forget plot]
  table[row sep=crcr]{%
0	0\\
3.15544362088405e-30	0.5\\
0.000656101980281985	0.5\\
0.00393661188169191	0.5\\
0.00599999999999994	0.5\\
0.006	0.5\\
0.012	0.5\\
0.0120000000000001	1\\
0.018	1\\
0.0180000000000001	0\\
0.0199999999999998	0\\
0.02	0\\
0.026	0\\
0.0260000000000002	0\\
0.0289999999999998	0\\
0.029	0\\
0.0319999999999996	0\\
0.0349999999999991	0\\
0.035	1\\
0.0399999999999996	1\\
0.04	0.5\\
0.0449999999999996	0.5\\
0.0459999999999996	0.5\\
0.046	1\\
0.047	1\\
0.0470000000000004	0\\
0.0490000000000003	0\\
0.0510000000000002	0\\
0.055	0\\
0.0579999999999996	0\\
0.058	0\\
0.0599999999999996	0\\
0.06	0\\
0.0619999999999995	0\\
0.0639999999999991	0\\
0.0659999999999991	0\\
0.066	0\\
0.0699999999999991	0\\
0.07	0\\
0.0700000000000009	1\\
0.074	1\\
0.076	1\\
0.0760000000000009	0\\
0.08	0\\
0.0800000000000009	0\\
0.0839999999999999	0\\
0.086	0\\
0.0860000000000009	0\\
0.0869999999999991	0\\
0.087	0\\
0.0880000000000004	0\\
0.0890000000000009	0\\
0.0910000000000017	0\\
0.0929999999999991	0\\
0.093	0\\
0.0970000000000017	0\\
0.0999999999999991	0\\
0.1	0\\
0.104000000000002	0\\
0.104999999999999	0\\
0.105	0.5\\
0.105999999999999	0.5\\
0.106	1\\
0.106999999999999	1\\
0.107999999999998	1\\
0.109999999999997	1\\
0.111999999999999	1\\
0.112	0\\
0.115999999999997	0\\
0.115999999999998	0\\
0.116	0\\
0.119999999999997	0\\
0.119999999999998	0\\
0.12	0\\
0.123999999999997	0\\
0.125999999999999	0\\
0.126	0\\
0.127999999999998	0\\
0.128	0\\
0.129999999999998	0\\
0.131999999999996	0\\
0.135999999999993	0\\
0.139999999999998	0\\
0.14	0.5\\
0.144999999999998	0.5\\
0.145	0.5\\
0.145999999999998	0.5\\
0.146	0.5\\
0.146999999999999	0.5\\
0.147999999999998	0.5\\
0.149999999999997	0.5\\
0.151999999999998	0.5\\
0.152	1\\
0.155999999999997	1\\
0.157999999999998	1\\
0.158	0\\
0.16	0\\
0.160000000000002	0\\
0.162000000000002	0\\
0.164000000000002	0\\
0.166	0\\
0.166000000000002	0\\
0.170000000000002	0\\
0.174	0\\
0.174000000000001	0\\
0.175	0\\
0.175000000000002	0.5\\
0.176000000000001	0.5\\
0.177	0.5\\
0.178999999999998	0.5\\
0.179999999999998	0.5\\
0.18	0.5\\
0.183999999999997	0.5\\
0.186	0.5\\
0.186000000000002	1\\
0.189999999999998	1\\
0.192	1\\
0.192000000000002	0\\
0.195999999999998	0\\
0.199999999999995	0\\
0.199999999999997	0\\
0.2	0\\
0.202999999999998	0\\
0.203	0\\
0.205999999999998	0\\
0.206	0\\
0.208999999999998	0\\
0.209999999999998	0\\
0.21	0.5\\
0.211999999999998	0.5\\
0.212	1\\
0.213999999999998	1\\
0.215999999999997	1\\
0.217999999999998	1\\
0.218	0\\
0.219999999999998	0\\
0.22	0\\
0.221999999999998	0\\
0.223999999999996	0\\
0.225999999999998	0\\
0.226	0\\
0.229999999999996	0\\
0.231999999999998	0\\
0.232	0\\
0.235999999999996	0\\
0.237999999999998	0\\
0.238	0\\
0.239999999999998	0\\
0.24	0\\
0.241999999999998	0\\
0.243999999999996	0\\
0.245	0\\
0.245000000000002	0.5\\
0.245999999999998	0.5\\
0.246	1\\
0.246999999999999	1\\
0.247999999999998	1\\
0.249999999999997	1\\
0.252	1\\
0.252000000000003	0\\
0.256	0\\
0.259999999999997	0\\
0.26	0\\
0.260999999999996	0\\
0.261	0\\
0.261999999999998	0\\
0.262999999999996	0\\
0.264999999999993	0\\
0.265999999999997	0\\
0.266	0\\
0.269999999999993	0\\
0.271999999999997	0\\
0.272	0\\
0.275999999999993	0\\
0.279999999999986	0\\
0.279999999999993	0\\
0.28	0.5\\
0.285999999999996	0.5\\
0.286	1\\
0.289999999999996	1\\
0.29	0.5\\
0.293999999999996	0.5\\
0.295999999999997	0.5\\
0.296	1\\
0.297999999999997	1\\
0.298	0\\
0.299999999999997	0\\
0.3	0\\
0.301999999999997	0\\
0.303999999999993	0\\
0.305999999999997	0\\
0.306	0\\
0.309999999999993	0\\
0.313999999999986	0\\
0.314999999999997	0\\
0.315	1\\
0.318999999999997	1\\
0.319	0.5\\
0.319999999999996	0.5\\
0.32	0.5\\
0.320999999999998	0.5\\
0.321999999999996	0.5\\
0.323999999999993	0.5\\
0.325999999999996	0.5\\
0.326	0.5\\
0.329999999999993	0.5\\
0.331	0.5\\
0.331000000000004	1\\
0.333	1\\
0.333000000000004	0\\
0.335	0\\
0.336999999999996	0\\
0.339999999999996	0\\
0.34	0\\
0.343999999999993	0\\
0.345999999999997	0\\
0.346	0\\
0.347999999999997	0\\
0.348	0\\
0.349999999999997	0\\
0.35	0.5\\
0.351999999999997	0.5\\
0.353999999999993	0.5\\
0.354	1\\
0.357999999999993	1\\
0.359999999999996	1\\
0.36	0\\
0.363999999999993	0\\
0.365999999999996	0\\
0.366	0\\
0.369999999999993	0\\
0.373999999999986	0\\
0.376999999999997	0\\
0.377	0\\
0.379999999999997	0\\
0.38	0\\
0.382999999999996	0\\
0.384999999999997	0\\
0.385	0.5\\
0.385999999999997	0.5\\
0.386	0.5\\
0.386999999999998	0.5\\
0.387999999999996	0.5\\
0.388999999999997	0.5\\
0.389	1\\
0.390999999999997	1\\
0.392999999999993	1\\
0.394999999999997	1\\
0.395	0\\
0.398999999999993	0\\
0.399999999999997	0\\
0.4	0\\
0.403999999999993	0\\
0.405999999999997	0\\
0.406	0\\
0.409999999999993	0\\
0.411999999999997	0\\
0.412	0\\
0.415999999999993	0\\
0.419999999999986	0\\
0.419999999999996	0\\
0.42	0.5\\
0.426	0.5\\
0.426000000000004	1\\
0.432000000000004	1\\
0.432000000000007	0\\
0.434999999999997	0\\
0.435	0\\
0.43799999999999	0\\
0.439999999999997	0\\
0.44	0\\
0.44299999999999	0\\
0.445999999999979	0\\
0.445999999999995	0\\
0.446	0\\
0.447	0\\
0.447000000000004	0\\
0.448000000000004	0\\
0.449000000000004	0\\
0.451000000000004	0\\
0.454999999999997	0\\
0.455	1\\
0.459	1\\
0.459999999999997	1\\
0.46	0.5\\
0.463999999999997	0.5\\
0.464	0.5\\
0.465999999999997	0.5\\
0.466	0.5\\
0.467999999999996	0.5\\
0.469999999999993	0.5\\
0.471999999999997	0.5\\
0.472	1\\
0.473	1\\
0.473000000000004	0\\
0.474000000000004	0\\
0.475000000000004	0\\
0.477000000000004	0\\
0.479999999999997	0\\
0.48	0\\
0.484	0\\
0.485999999999997	0\\
0.486	0\\
0.489999999999997	0\\
0.49	1\\
0.492999999999997	1\\
0.493	0.5\\
0.495999999999997	0.5\\
0.498999999999993	0.5\\
0.499	1\\
0.499999999999997	1\\
0.5	0.5\\
0.500999999999998	0.5\\
0.501999999999997	0.5\\
0.503999999999993	0.5\\
0.505999999999993	0.5\\
0.506	1\\
0.507999999999993	1\\
0.508	0\\
0.509999999999993	0\\
0.511999999999986	0\\
0.515999999999972	0\\
0.519999999999993	0\\
0.52	0\\
0.521999999999993	0\\
0.522	0\\
0.523999999999993	0\\
0.524999999999993	0\\
0.525	0.5\\
0.525999999999993	0.5\\
0.526	0.5\\
0.526999999999998	0.5\\
0.527999999999997	0.5\\
0.529999999999993	0.5\\
0.531999999999993	0.5\\
0.532	1\\
0.535999999999993	1\\
0.538	1\\
0.538000000000007	0\\
0.539999999999993	0\\
0.54	0\\
0.541999999999986	0\\
0.543999999999972	0\\
0.545999999999993	0\\
0.546	0\\
0.549999999999972	0\\
0.550999999999993	0\\
0.551	0\\
0.554999999999972	0\\
0.556999999999993	0\\
0.557	0\\
0.559999999999993	0\\
0.56	0.5\\
0.562999999999993	0.5\\
0.565999999999986	0.5\\
0.565999999999993	0.5\\
0.566	1\\
0.571999999999986	1\\
0.571999999999993	1\\
0.572	0\\
0.577999999999986	0\\
0.579999999999993	0\\
0.58	0\\
0.585999999999986	0\\
0.585999999999993	0\\
0.586	0\\
0.591999999999986	0\\
0.591999999999993	0\\
0.592	0\\
0.594999999999993	0\\
0.595	1\\
0.597999999999993	1\\
0.599999999999993	1\\
0.6	0.5\\
};
\end{axis}

\begin{axis}[%
width=4.133in,
height=0.576in,
at={(0.693in,0.495in)},
scale only axis,
xmin=0,
xmax=0.06,
ymin=0,
ymax=1.1,
axis background/.style={fill=white}
]
\pgfplotsset{max space between ticks=50}
\addplot [color=mycolor4,solid,forget plot]
  table[row sep=crcr]{%
0	0\\
3.15544362088405e-30	1\\
0.000656101980281985	1\\
0.00393661188169191	1\\
0.00599999999999994	1\\
0.006	1\\
0.012	1\\
0.0120000000000001	1\\
0.018	1\\
0.0180000000000001	0\\
0.0199999999999998	0\\
0.02	1\\
0.026	1\\
0.0260000000000002	0\\
0.0289999999999998	0\\
0.029	1\\
0.0319999999999996	1\\
0.0349999999999991	1\\
0.035	1\\
0.0399999999999996	1\\
0.04	1\\
0.0449999999999996	1\\
0.0459999999999996	1\\
0.046	1\\
0.047	1\\
0.0470000000000004	0\\
0.0490000000000003	0\\
0.0510000000000002	0\\
0.055	0\\
0.0579999999999996	0\\
0.058	1\\
0.0599999999999996	1\\
0.06	1\\
0.0619999999999995	1\\
0.0639999999999991	1\\
0.0659999999999991	1\\
0.066	1\\
0.0699999999999991	1\\
0.07	1\\
0.0700000000000009	1\\
0.074	1\\
0.076	1\\
0.0760000000000009	0\\
0.08	0\\
0.0800000000000009	1\\
0.0839999999999999	1\\
0.086	1\\
0.0860000000000009	0\\
0.0869999999999991	0\\
0.087	1\\
0.0880000000000004	1\\
0.0890000000000009	1\\
0.0910000000000017	1\\
0.0929999999999991	1\\
0.093	0\\
0.0970000000000017	0\\
0.0999999999999991	0\\
0.1	1\\
0.104000000000002	1\\
0.104999999999999	1\\
0.105	1\\
0.105999999999999	1\\
0.106	1\\
0.106999999999999	1\\
0.107999999999998	1\\
0.109999999999997	1\\
0.111999999999999	1\\
0.112	0\\
0.115999999999997	0\\
0.115999999999998	0\\
0.116	1\\
0.119999999999997	1\\
0.119999999999998	1\\
0.12	1\\
0.123999999999997	1\\
0.125999999999999	1\\
0.126	1\\
0.127999999999998	1\\
0.128	0\\
0.129999999999998	0\\
0.131999999999996	0\\
0.135999999999993	0\\
0.139999999999998	0\\
0.14	1\\
0.144999999999998	1\\
0.145	1\\
0.145999999999998	1\\
0.146	1\\
0.146999999999999	1\\
0.147999999999998	1\\
0.149999999999997	1\\
0.151999999999998	1\\
0.152	1\\
0.155999999999997	1\\
0.157999999999998	1\\
0.158	0\\
0.16	0\\
0.160000000000002	1\\
0.162000000000002	1\\
0.164000000000002	1\\
0.166	1\\
0.166000000000002	0\\
0.170000000000002	0\\
0.174	0\\
0.174000000000001	1\\
0.175	1\\
0.175000000000002	1\\
0.176000000000001	1\\
0.177	1\\
0.178999999999998	1\\
0.179999999999998	1\\
0.18	1\\
0.183999999999997	1\\
0.186	1\\
0.186000000000002	1\\
0.189999999999998	1\\
0.192	1\\
0.192000000000002	0\\
0.195999999999998	0\\
0.199999999999995	0\\
0.199999999999997	0\\
0.2	1\\
0.202999999999998	1\\
0.203	1\\
0.205999999999998	1\\
0.206	1\\
0.208999999999998	1\\
0.209999999999998	1\\
0.21	1\\
0.211999999999998	1\\
0.212	1\\
0.213999999999998	1\\
0.215999999999997	1\\
0.217999999999998	1\\
0.218	0\\
0.219999999999998	0\\
0.22	1\\
0.221999999999998	1\\
0.223999999999996	1\\
0.225999999999998	1\\
0.226	0\\
0.229999999999996	0\\
0.231999999999998	0\\
0.232	1\\
0.235999999999996	1\\
0.237999999999998	1\\
0.238	0\\
0.239999999999998	0\\
0.24	1\\
0.241999999999998	1\\
0.243999999999996	1\\
0.245	1\\
0.245000000000002	1\\
0.245999999999998	1\\
0.246	1\\
0.246999999999999	1\\
0.247999999999998	1\\
0.249999999999997	1\\
0.252	1\\
0.252000000000003	0\\
0.256	0\\
0.259999999999997	0\\
0.26	1\\
0.260999999999996	1\\
0.261	1\\
0.261999999999998	1\\
0.262999999999996	1\\
0.264999999999993	1\\
0.265999999999997	1\\
0.266	1\\
0.269999999999993	1\\
0.271999999999997	1\\
0.272	0\\
0.275999999999993	0\\
0.279999999999986	0\\
0.279999999999993	0\\
0.28	1\\
0.285999999999996	1\\
0.286	1\\
0.289999999999996	1\\
0.29	1\\
0.293999999999996	1\\
0.295999999999997	1\\
0.296	1\\
0.297999999999997	1\\
0.298	0\\
0.299999999999997	0\\
0.3	1\\
0.301999999999997	1\\
0.303999999999993	1\\
0.305999999999997	1\\
0.306	0\\
0.309999999999993	0\\
0.313999999999986	0\\
0.314999999999997	0\\
0.315	1\\
0.318999999999997	1\\
0.319	1\\
0.319999999999996	1\\
0.32	1\\
0.320999999999998	1\\
0.321999999999996	1\\
0.323999999999993	1\\
0.325999999999996	1\\
0.326	1\\
0.329999999999993	1\\
0.331	1\\
0.331000000000004	1\\
0.333	1\\
0.333000000000004	0\\
0.335	0\\
0.336999999999996	0\\
0.339999999999996	0\\
0.34	1\\
0.343999999999993	1\\
0.345999999999997	1\\
0.346	0\\
0.347999999999997	0\\
0.348	1\\
0.349999999999997	1\\
0.35	1\\
0.351999999999997	1\\
0.353999999999993	1\\
0.354	1\\
0.357999999999993	1\\
0.359999999999996	1\\
0.36	1\\
0.363999999999993	1\\
0.365999999999996	1\\
0.366	0\\
0.369999999999993	0\\
0.373999999999986	0\\
0.376999999999997	0\\
0.377	1\\
0.379999999999997	1\\
0.38	1\\
0.382999999999996	1\\
0.384999999999997	1\\
0.385	1\\
0.385999999999997	1\\
0.386	1\\
0.386999999999998	1\\
0.387999999999996	1\\
0.388999999999997	1\\
0.389	1\\
0.390999999999997	1\\
0.392999999999993	1\\
0.394999999999997	1\\
0.395	0\\
0.398999999999993	0\\
0.399999999999997	0\\
0.4	1\\
0.403999999999993	1\\
0.405999999999997	1\\
0.406	1\\
0.409999999999993	1\\
0.411999999999997	1\\
0.412	0\\
0.415999999999993	0\\
0.419999999999986	0\\
0.419999999999996	0\\
0.42	1\\
0.426	1\\
0.426000000000004	1\\
0.432000000000004	1\\
0.432000000000007	0\\
0.434999999999997	0\\
0.435	1\\
0.43799999999999	1\\
0.439999999999997	1\\
0.44	1\\
0.44299999999999	1\\
0.445999999999979	1\\
0.445999999999995	1\\
0.446	1\\
0.447	1\\
0.447000000000004	0\\
0.448000000000004	0\\
0.449000000000004	0\\
0.451000000000004	0\\
0.454999999999997	0\\
0.455	1\\
0.459	1\\
0.459999999999997	1\\
0.46	1\\
0.463999999999997	1\\
0.464	1\\
0.465999999999997	1\\
0.466	1\\
0.467999999999996	1\\
0.469999999999993	1\\
0.471999999999997	1\\
0.472	1\\
0.473	1\\
0.473000000000004	0\\
0.474000000000004	0\\
0.475000000000004	0\\
0.477000000000004	0\\
0.479999999999997	0\\
0.48	1\\
0.484	1\\
0.485999999999997	1\\
0.486	0\\
0.489999999999997	0\\
0.49	1\\
0.492999999999997	1\\
0.493	1\\
0.495999999999997	1\\
0.498999999999993	1\\
0.499	1\\
0.499999999999997	1\\
0.5	1\\
0.500999999999998	1\\
0.501999999999997	1\\
0.503999999999993	1\\
0.505999999999993	1\\
0.506	1\\
0.507999999999993	1\\
0.508	0\\
0.509999999999993	0\\
0.511999999999986	0\\
0.515999999999972	0\\
0.519999999999993	0\\
0.52	1\\
0.521999999999993	1\\
0.522	1\\
0.523999999999993	1\\
0.524999999999993	1\\
0.525	1\\
0.525999999999993	1\\
0.526	1\\
0.526999999999998	1\\
0.527999999999997	1\\
0.529999999999993	1\\
0.531999999999993	1\\
0.532	1\\
0.535999999999993	1\\
0.538	1\\
0.538000000000007	0\\
0.539999999999993	0\\
0.54	1\\
0.541999999999986	1\\
0.543999999999972	1\\
0.545999999999993	1\\
0.546	0\\
0.549999999999972	0\\
0.550999999999993	0\\
0.551	1\\
0.554999999999972	1\\
0.556999999999993	1\\
0.557	0\\
0.559999999999993	0\\
0.56	1\\
0.562999999999993	1\\
0.565999999999986	1\\
0.565999999999993	1\\
0.566	1\\
0.571999999999986	1\\
0.571999999999993	1\\
0.572	0\\
0.577999999999986	0\\
0.579999999999993	0\\
0.58	1\\
0.585999999999986	1\\
0.585999999999993	1\\
0.586	1\\
0.591999999999986	1\\
0.591999999999993	1\\
0.592	0\\
0.594999999999993	0\\
0.595	1\\
0.597999999999993	1\\
0.599999999999993	1\\
0.6	1\\
};
\end{axis}
\end{tikzpicture}%
}
    \caption{The executed schedule for the three pendula restricted to the first
      60ms. \texttt{Blue}: $P_1$, \texttt{Red}: $P_2$,
      \texttt{Orange}: $P_3$. $C_i = 6$ ms. The last figure shows the overall
      processor usage for verification purposes.}
    \label{fig:01.4.1.small}
  \end{figure}

  \begin{figure}[H]\centering
    \scalebox{0.7}{\begin{ganttchart}[vgrid, hgrid]{0}{59}
%\gantttitle{2016}{12}\\
\gantttitlelist{10,20,...,60}{10}\\
\ganttset{progress label text={},
       bar incomplete/.append style={fill=black!40},
       group/.append style={draw=black, fill=black},}
\ganttbar{Task 1}{0}{9}
\ganttbar{}{20}{29}
\ganttbar{}{40}{49}\\

\ganttbar[progress=00]{Task 2}{0}{9}
\ganttbar{}{10}{19}
\ganttbar[progress=00]{}{29}{29}
\ganttbar{}{30}{39}
\ganttbar{}{58}{59}\\

\ganttbar[progress=00]{Task 3}{0}{49}
\ganttbar{}{50}{57}
\end{ganttchart}
}
    \caption{A portion of the calculated RM schedule $\sigma$ for tasks
      $J_1, J_2, J_3$.  Shaded areas denote the waiting time.}
    \label{fig:rm_6}
  \end{figure}
\end{sidewaysfigure}


\begin{figure}[H]\centering
  \scalebox{0.7}{% This file was created by matlab2tikz.
%
%The latest updates can be retrieved from
%  http://www.mathworks.com/matlabcentral/fileexchange/22022-matlab2tikz-matlab2tikz
%where you can also make suggestions and rate matlab2tikz.
%
\definecolor{mycolor1}{rgb}{0.00000,0.44700,0.74100}%
%
\begin{tikzpicture}

\begin{axis}[%
width=4.133in,
height=3.26in,
at={(0.693in,0.44in)},
scale only axis,
xmin=0,
xmax=2700,
xmajorgrids,
ymin=0,
ymax=1.25,
ymajorgrids,
axis background/.style={fill=white}
]
\pgfplotsset{max space between ticks=50}
\addplot [color=mycolor1,solid,forget plot]
  table[row sep=crcr]{%
1	0\\
2	1\\
3	1\\
4	1\\
5	1\\
6	0.75\\
7	0.75\\
8	0.5\\
9	0.5\\
10	0\\
11	0\\
12	0.5\\
13	0.5\\
14	0\\
15	0\\
16	0.5\\
17	0.5\\
18	0.5\\
19	0.5\\
20	0.5\\
21	0.75\\
22	0.75\\
23	0.75\\
24	0.5\\
25	0.5\\
26	0\\
27	0\\
28	0\\
29	0\\
30	0\\
31	0.5\\
32	0.5\\
33	0.75\\
34	0.75\\
35	0.75\\
36	0.75\\
37	0.5\\
38	0.5\\
39	0.5\\
40	0.5\\
41	0.5\\
42	0.5\\
43	0\\
44	0\\
45	0.5\\
46	0.5\\
47	0.5\\
48	0\\
49	0\\
50	0.5\\
51	0.5\\
52	0.5\\
53	0.5\\
54	0.5\\
55	0\\
56	0\\
57	0\\
58	0.5\\
59	0.5\\
60	0.5\\
61	0.75\\
62	0.75\\
63	0.5\\
64	0.5\\
65	0.5\\
66	0.5\\
67	0.5\\
68	0\\
69	0\\
70	0\\
71	0.5\\
72	0.5\\
73	0.5\\
74	0.75\\
75	0.75\\
76	0.75\\
77	0.5\\
78	0.5\\
79	0\\
80	0\\
81	0\\
82	0\\
83	0\\
84	0.75\\
85	0.75\\
86	1\\
87	1\\
88	0.75\\
89	0.75\\
90	0.75\\
91	0.75\\
92	0.75\\
93	0.5\\
94	0.5\\
95	0.5\\
96	0\\
97	0\\
98	0.5\\
99	0.5\\
100	0.5\\
101	0.5\\
102	0\\
103	0\\
104	0\\
105	0.5\\
106	0.5\\
107	0.75\\
108	0.75\\
109	0.75\\
110	0.75\\
111	0.75\\
112	0.75\\
113	0.75\\
114	0.75\\
115	0.5\\
116	0.5\\
117	0.5\\
118	0\\
119	0\\
120	0\\
121	0\\
122	0.5\\
123	0.5\\
124	0.75\\
125	0.75\\
126	0.5\\
127	0.5\\
128	0.5\\
129	0.75\\
130	0.75\\
131	0.5\\
132	0.5\\
133	0.5\\
134	0.5\\
135	0\\
136	0\\
137	0.5\\
138	0.5\\
139	0.5\\
140	0.5\\
141	0\\
142	0\\
143	0\\
144	0.5\\
145	0.5\\
146	0.5\\
147	0\\
148	0\\
149	0.5\\
150	0.5\\
151	0.5\\
152	0.5\\
153	0.75\\
154	0.75\\
155	0.5\\
156	0.5\\
157	0.5\\
158	0.5\\
159	0.5\\
160	0\\
161	0\\
162	0\\
163	0.5\\
164	0.5\\
165	0.75\\
166	0.75\\
167	0.75\\
168	0.75\\
169	0.75\\
170	0.5\\
171	0.5\\
172	0.5\\
173	0\\
174	0\\
175	0\\
176	0\\
177	0.75\\
178	0.75\\
179	0.5\\
180	0.5\\
181	0.75\\
182	0.75\\
183	0.5\\
184	0.5\\
185	0.5\\
186	0.5\\
187	0\\
188	0\\
189	0.5\\
190	0.5\\
191	0.5\\
192	0.5\\
193	0\\
194	0\\
195	0\\
196	0\\
197	0.5\\
198	0.5\\
199	0.75\\
200	0.75\\
201	1\\
202	1\\
203	1\\
204	1\\
205	1\\
206	0.75\\
207	0.75\\
208	0.75\\
209	0.5\\
210	0.5\\
211	0\\
212	0\\
213	0\\
214	0\\
215	0.5\\
216	0.5\\
217	0.5\\
218	0\\
219	0\\
220	0.5\\
221	0.5\\
222	0.75\\
223	0.75\\
224	0.75\\
225	0.5\\
226	0.5\\
227	0.5\\
228	0.5\\
229	0.5\\
230	0.5\\
231	0\\
232	0\\
233	0\\
234	0\\
235	0.5\\
236	0.5\\
237	0.75\\
238	0.75\\
239	0.75\\
240	1\\
241	1\\
242	0.75\\
243	0.75\\
244	0.75\\
245	0.75\\
246	0.5\\
247	0.5\\
248	0.5\\
249	0.5\\
250	0\\
251	0\\
252	0\\
253	0.5\\
254	0.5\\
255	0.5\\
256	0.5\\
257	0.5\\
258	0.5\\
259	0\\
260	0\\
261	0\\
262	0\\
263	0.75\\
264	0.75\\
265	0.5\\
266	0.5\\
267	0\\
268	0\\
269	0.5\\
270	0.5\\
271	0.5\\
272	0.75\\
273	0.75\\
274	0.75\\
275	0.75\\
276	0.5\\
277	0.5\\
278	0\\
279	0\\
280	0\\
281	0\\
282	0\\
283	0.5\\
284	0.5\\
285	0.5\\
286	0.75\\
287	0.75\\
288	1\\
289	1\\
290	0.75\\
291	0.75\\
292	0.5\\
293	0.5\\
294	0.5\\
295	0.5\\
296	0.5\\
297	0\\
298	0\\
299	0\\
300	0.5\\
301	0.5\\
302	0.5\\
303	0\\
304	0\\
305	0\\
306	0.5\\
307	0.5\\
308	0.75\\
309	0.75\\
310	0.75\\
311	0.75\\
312	0.5\\
313	0.5\\
314	0.75\\
315	0.75\\
316	0.75\\
317	0.75\\
318	0.75\\
319	0.5\\
320	0.5\\
321	0\\
322	0\\
323	0\\
324	0\\
325	0\\
326	0.5\\
327	0.5\\
328	0.75\\
329	0.75\\
330	0.75\\
331	1\\
332	1\\
333	0.75\\
334	0.75\\
335	0.75\\
336	0.75\\
337	0.75\\
338	0.5\\
339	0.5\\
340	0.5\\
341	0\\
342	0\\
343	0.5\\
344	0.5\\
345	0.5\\
346	0.5\\
347	0\\
348	0\\
349	0\\
350	0.5\\
351	0.5\\
352	0.5\\
353	0\\
354	0\\
355	0.75\\
356	0.75\\
357	0.75\\
358	0.75\\
359	0.5\\
360	0.5\\
361	0.5\\
362	0\\
363	0\\
364	0\\
365	0.75\\
366	0.75\\
367	0.75\\
368	0.5\\
369	0.5\\
370	0.5\\
371	0\\
372	0\\
373	0.5\\
374	0.5\\
375	0.5\\
376	0.75\\
377	0.75\\
378	0.75\\
379	0.5\\
380	0.5\\
381	0\\
382	0\\
383	0\\
384	0.5\\
385	0.5\\
386	0.5\\
387	0.5\\
388	0\\
389	0\\
390	0\\
391	0.5\\
392	0.5\\
393	0.5\\
394	0\\
395	0\\
396	0.5\\
397	0.5\\
398	0.5\\
399	0\\
400	0\\
401	0.5\\
402	0.5\\
403	0.75\\
404	0.75\\
405	0.75\\
406	0.75\\
407	0.5\\
408	0.5\\
409	0.5\\
410	0\\
411	0\\
412	0\\
413	0\\
414	0.5\\
415	0.5\\
416	0.75\\
417	0.75\\
418	0.5\\
419	0.5\\
420	0.75\\
421	0.75\\
422	0.75\\
423	0.75\\
424	0.75\\
425	0.5\\
426	0.5\\
427	0.5\\
428	0\\
429	0\\
430	0.5\\
431	0.5\\
432	0.5\\
433	0.5\\
434	0\\
435	0\\
436	0\\
437	0\\
438	0.5\\
439	0.5\\
440	1\\
441	1\\
442	1\\
443	0.75\\
444	0.75\\
445	0.5\\
446	0.5\\
447	0.5\\
448	0.5\\
449	0\\
450	0\\
451	0\\
452	0.5\\
453	0.5\\
454	0.5\\
455	0.75\\
456	0.75\\
457	0.5\\
458	0.5\\
459	0.5\\
460	0.5\\
461	0.5\\
462	0\\
463	0\\
464	0.5\\
465	0.5\\
466	0.5\\
467	0.75\\
468	0.75\\
469	0.75\\
470	0.5\\
471	0.5\\
472	0\\
473	0\\
474	0\\
475	0\\
476	0\\
477	0.5\\
478	0.5\\
479	0.5\\
480	0.5\\
481	0.5\\
482	0.5\\
483	0\\
484	0\\
485	0.5\\
486	0.5\\
487	0.5\\
488	0\\
489	0\\
490	0.5\\
491	0.5\\
492	0.75\\
493	0.75\\
494	0.5\\
495	0.5\\
496	0.5\\
497	0.5\\
498	0\\
499	0\\
500	0\\
501	0.5\\
502	0.5\\
503	0.75\\
504	0.75\\
505	0.5\\
506	0.5\\
507	0.5\\
508	0.5\\
509	0.5\\
510	0.5\\
511	0.5\\
512	0.5\\
513	0\\
514	0\\
515	0.5\\
516	0.5\\
517	0.5\\
518	0.5\\
519	0\\
520	0\\
521	0\\
522	0\\
523	0.75\\
524	0.75\\
525	1\\
526	1\\
527	1\\
528	1\\
529	1\\
530	0.75\\
531	0.75\\
532	0.75\\
533	0.5\\
534	0.5\\
535	0.5\\
536	0\\
537	0\\
538	0.5\\
539	0.5\\
540	0.5\\
541	0.5\\
542	0\\
543	0\\
544	0.5\\
545	0.5\\
546	0.5\\
547	0.75\\
548	0.75\\
549	0.5\\
550	0.5\\
551	0.5\\
552	0.5\\
553	0.75\\
554	0.75\\
555	0.75\\
556	0.75\\
557	0.5\\
558	0.5\\
559	0\\
560	0\\
561	0\\
562	0\\
563	0\\
564	0.5\\
565	0.5\\
566	0.75\\
567	0.75\\
568	0.75\\
569	0.75\\
570	0.75\\
571	0.5\\
572	0.5\\
573	0.5\\
574	0.75\\
575	0.75\\
576	0.5\\
577	0.5\\
578	0.5\\
579	0.5\\
580	0.5\\
581	0\\
582	0\\
583	0.5\\
584	0.5\\
585	0.5\\
586	0.5\\
587	0\\
588	0\\
589	0.5\\
590	0.5\\
591	0.5\\
592	0.5\\
593	0\\
594	0\\
595	0\\
596	0.5\\
597	0.5\\
598	0.5\\
599	0.75\\
600	0.75\\
601	0.5\\
602	0.5\\
603	0.5\\
604	0.5\\
605	0.5\\
606	0\\
607	0\\
608	0\\
609	0.5\\
610	0.5\\
611	0.75\\
612	0.75\\
613	0.75\\
614	0.5\\
615	0.5\\
616	0\\
617	0\\
618	0\\
619	0\\
620	0.75\\
621	0.75\\
622	0.75\\
623	0.75\\
624	0.5\\
625	0.5\\
626	0\\
627	0\\
628	0.5\\
629	0.5\\
630	0.5\\
631	0.5\\
632	0\\
633	0\\
634	0\\
635	0\\
636	0.75\\
637	0.75\\
638	1\\
639	1\\
640	1\\
641	0.75\\
642	0.75\\
643	0.5\\
644	0.5\\
645	0.5\\
646	0.5\\
647	0.5\\
648	0\\
649	0\\
650	0\\
651	0.5\\
652	0.5\\
653	0.75\\
654	0.75\\
655	0.5\\
656	0.5\\
657	0.5\\
658	0.75\\
659	0.75\\
660	0.5\\
661	0.5\\
662	0.5\\
663	0.5\\
664	0\\
665	0\\
666	0.5\\
667	0.5\\
668	0.5\\
669	0.5\\
670	0\\
671	0\\
672	0\\
673	0.5\\
674	0.5\\
675	0.5\\
676	0\\
677	0\\
678	0.5\\
679	0.5\\
680	0.5\\
681	0.5\\
682	0.5\\
683	0.75\\
684	0.75\\
685	0.5\\
686	0.5\\
687	0.5\\
688	0.5\\
689	0.5\\
690	0\\
691	0\\
692	0\\
693	0\\
694	0.5\\
695	0.5\\
696	0.75\\
697	0.75\\
698	0.75\\
699	0.75\\
700	0.5\\
701	0.5\\
702	0.5\\
703	0\\
704	0\\
705	0\\
706	0\\
707	0.75\\
708	0.75\\
709	0.5\\
710	0.5\\
711	0.75\\
712	0.75\\
713	0.5\\
714	0.5\\
715	0.5\\
716	0.5\\
717	0.5\\
718	0\\
719	0\\
720	0.5\\
721	0.5\\
722	0.5\\
723	0.5\\
724	0\\
725	0\\
726	0\\
727	0\\
728	0.5\\
729	0.5\\
730	1\\
731	1\\
732	1\\
733	0.75\\
734	0.75\\
735	0.75\\
736	0.5\\
737	0.5\\
738	0\\
739	0\\
740	0\\
741	0\\
742	0\\
743	0.5\\
744	0.5\\
745	0.5\\
746	0\\
747	0\\
748	0.5\\
749	0.5\\
750	0.75\\
751	0.75\\
752	0.75\\
753	0.75\\
754	0.75\\
755	0.5\\
756	0.5\\
757	0.5\\
758	0.75\\
759	0.75\\
760	0.75\\
761	0.5\\
762	0.5\\
763	0\\
764	0\\
765	0\\
766	0\\
767	0\\
768	0\\
769	0.5\\
770	0.5\\
771	0.75\\
772	0.75\\
773	0.75\\
774	0.75\\
775	1\\
776	1\\
777	0.75\\
778	0.75\\
779	0.75\\
780	0.75\\
781	0.5\\
782	0.5\\
783	0.5\\
784	0\\
785	0\\
786	0.5\\
787	0.5\\
788	0.5\\
789	0\\
790	0\\
791	0.5\\
792	0.5\\
793	0.5\\
794	0.5\\
795	0.5\\
796	0\\
797	0\\
798	0\\
799	0.75\\
800	0.75\\
801	0.75\\
802	0.5\\
803	0.5\\
804	0.5\\
805	0\\
806	0\\
807	0.5\\
808	0.5\\
809	0.75\\
810	0.75\\
811	0.75\\
812	0.5\\
813	0.5\\
814	0\\
815	0\\
816	0\\
817	0\\
818	0.5\\
819	0.5\\
820	0.5\\
821	0.75\\
822	0.75\\
823	0.75\\
824	1\\
825	1\\
826	0.75\\
827	0.75\\
828	0.5\\
829	0.5\\
830	0.5\\
831	0.5\\
832	0.5\\
833	0\\
834	0\\
835	0\\
836	0.5\\
837	0.5\\
838	0.5\\
839	0\\
840	0\\
841	0.5\\
842	0.5\\
843	0.75\\
844	0.75\\
845	0.75\\
846	0.75\\
847	0.75\\
848	0.75\\
849	0.5\\
850	0.5\\
851	0\\
852	0\\
853	0\\
854	0\\
855	0\\
856	0.5\\
857	0.5\\
858	0.75\\
859	0.75\\
860	1\\
861	1\\
862	0.75\\
863	0.75\\
864	0.75\\
865	0.75\\
866	0.75\\
867	0.5\\
868	0.5\\
869	0.5\\
870	0\\
871	0\\
872	0.5\\
873	0.5\\
874	0.5\\
875	0.5\\
876	0\\
877	0\\
878	0\\
879	0.5\\
880	0.5\\
881	0.5\\
882	0\\
883	0\\
884	0.75\\
885	0.75\\
886	0.75\\
887	0.75\\
888	0.5\\
889	0.5\\
890	0.5\\
891	0\\
892	0\\
893	0\\
894	0\\
895	0.5\\
896	0.5\\
897	0.75\\
898	0.75\\
899	0.75\\
900	0.75\\
901	0.75\\
902	0.5\\
903	0.5\\
904	0.5\\
905	0\\
906	0\\
907	0.5\\
908	0.5\\
909	0.5\\
910	0.75\\
911	0.75\\
912	0.75\\
913	0.75\\
914	0.5\\
915	0.5\\
916	0\\
917	0\\
918	0\\
919	0\\
920	0.5\\
921	0.5\\
922	0.5\\
923	0.5\\
924	0\\
925	0\\
926	0\\
927	0.5\\
928	0.5\\
929	0.5\\
930	0\\
931	0\\
932	0.5\\
933	0.5\\
934	0.5\\
935	0\\
936	0\\
937	0.5\\
938	0.5\\
939	0.75\\
940	0.75\\
941	0.75\\
942	0.75\\
943	0.75\\
944	0.5\\
945	0.5\\
946	0.5\\
947	0\\
948	0\\
949	0\\
950	0\\
951	0.5\\
952	0.5\\
953	0.75\\
954	0.75\\
955	0.5\\
956	0.5\\
957	0.5\\
958	0.75\\
959	0.75\\
960	0.75\\
961	0.75\\
962	0.5\\
963	0.5\\
964	0\\
965	0\\
966	0.5\\
967	0.5\\
968	0.5\\
969	0.5\\
970	0\\
971	0\\
972	0\\
973	0\\
974	0.5\\
975	0.5\\
976	1\\
977	1\\
978	1\\
979	1\\
980	0.75\\
981	0.75\\
982	0.5\\
983	0.5\\
984	0.5\\
985	0\\
986	0\\
987	0.5\\
988	0.5\\
989	0.5\\
990	0.5\\
991	0.5\\
992	0.5\\
993	0\\
994	0\\
995	0.5\\
996	0.5\\
997	0.5\\
998	0.75\\
999	0.75\\
1000	0.75\\
1001	0.5\\
1002	0.5\\
1003	0\\
1004	0\\
1005	0\\
1006	0\\
1007	0\\
1008	0\\
1009	0.5\\
1010	0.5\\
1011	0.75\\
1012	0.75\\
1013	0.75\\
1014	0.5\\
1015	0.5\\
1016	0\\
1017	0\\
1018	0\\
1019	0\\
1020	0.5\\
1021	0.5\\
1022	0.5\\
1023	0.5\\
1024	0\\
1025	0\\
1026	0\\
1027	0.5\\
1028	0.5\\
1029	0.75\\
1030	0.75\\
1031	0.5\\
1032	0.5\\
1033	0.5\\
1034	0.5\\
1035	0\\
1036	0\\
1037	0\\
1038	0\\
1039	0.5\\
1040	0.5\\
1041	0.75\\
1042	0.75\\
1043	0.5\\
1044	0.5\\
1045	0.5\\
1046	0.5\\
1047	0.5\\
1048	0\\
1049	0\\
1050	0.5\\
1051	0.5\\
1052	0.5\\
1053	0.5\\
1054	0.5\\
1055	0\\
1056	0\\
1057	0.5\\
1058	0.5\\
1059	0.5\\
1060	0.5\\
1061	0.5\\
1062	0\\
1063	0\\
1064	0\\
1065	0\\
1066	0.75\\
1067	0.75\\
1068	1\\
1069	1\\
1070	1\\
1071	1\\
1072	0.75\\
1073	0.75\\
1074	0.75\\
1075	0.5\\
1076	0.5\\
1077	0.5\\
1078	0\\
1079	0\\
1080	0.5\\
1081	0.5\\
1082	0.5\\
1083	0.5\\
1084	0\\
1085	0\\
1086	0\\
1087	0.5\\
1088	0.5\\
1089	0.75\\
1090	0.75\\
1091	0.5\\
1092	0.5\\
1093	0.5\\
1094	0.75\\
1095	0.75\\
1096	0.75\\
1097	0.75\\
1098	0.5\\
1099	0.5\\
1100	0\\
1101	0\\
1102	0\\
1103	0\\
1104	0.75\\
1105	0.75\\
1106	0.75\\
1107	0.5\\
1108	0.5\\
1109	0.75\\
1110	0.75\\
1111	0.5\\
1112	0.5\\
1113	0.5\\
1114	0.5\\
1115	0\\
1116	0\\
1117	0.5\\
1118	0.5\\
1119	0.5\\
1120	0.5\\
1121	0\\
1122	0\\
1123	0.5\\
1124	0.5\\
1125	0.5\\
1126	0.5\\
1127	0\\
1128	0\\
1129	0.5\\
1130	0.5\\
1131	0.75\\
1132	0.75\\
1133	0.5\\
1134	0.5\\
1135	0.5\\
1136	0.5\\
1137	0.5\\
1138	0\\
1139	0\\
1140	0\\
1141	0.5\\
1142	0.5\\
1143	0.75\\
1144	0.75\\
1145	0.75\\
1146	0.75\\
1147	0.5\\
1148	0.5\\
1149	0\\
1150	0\\
1151	0\\
1152	0\\
1153	0.75\\
1154	0.75\\
1155	0.5\\
1156	0.5\\
1157	0.75\\
1158	0.75\\
1159	0.75\\
1160	0.75\\
1161	0.75\\
1162	0.5\\
1163	0.5\\
1164	0.5\\
1165	0\\
1166	0\\
1167	0.5\\
1168	0.5\\
1169	0.5\\
1170	0.5\\
1171	0\\
1172	0\\
1173	0\\
1174	0\\
1175	0.5\\
1176	0.5\\
1177	0.75\\
1178	0.75\\
1179	0.75\\
1180	0.75\\
1181	0.75\\
1182	1\\
1183	1\\
1184	1\\
1185	0.75\\
1186	0.75\\
1187	0.5\\
1188	0.5\\
1189	0.5\\
1190	0.5\\
1191	0\\
1192	0\\
1193	0\\
1194	0.5\\
1195	0.5\\
1196	0.5\\
1197	0.75\\
1198	0.75\\
1199	0.5\\
1200	0.5\\
1201	0.5\\
1202	0.5\\
1203	0.75\\
1204	0.75\\
1205	0.5\\
1206	0.5\\
1207	0.5\\
1208	0.5\\
1209	0\\
1210	0\\
1211	0.5\\
1212	0.5\\
1213	0.5\\
1214	0.5\\
1215	0\\
1216	0\\
1217	0\\
1218	0\\
1219	0.5\\
1220	0.5\\
1221	0.5\\
1222	0.5\\
1223	0.75\\
1224	0.75\\
1225	0.5\\
1226	0.5\\
1227	0.5\\
1228	0.5\\
1229	0.5\\
1230	0\\
1231	0\\
1232	0\\
1233	0\\
1234	0.5\\
1235	0.5\\
1236	0.75\\
1237	0.75\\
1238	0.5\\
1239	0.5\\
1240	0.5\\
1241	0.5\\
1242	0\\
1243	0\\
1244	0\\
1245	0.75\\
1246	0.75\\
1247	0.75\\
1248	0.5\\
1249	0.5\\
1250	0.5\\
1251	0.5\\
1252	0.5\\
1253	0.5\\
1254	0\\
1255	0\\
1256	0.5\\
1257	0.5\\
1258	0.5\\
1259	0.5\\
1260	0\\
1261	0\\
1262	0\\
1263	0\\
1264	0.5\\
1265	0.5\\
1266	0.75\\
1267	0.75\\
1268	1\\
1269	1\\
1270	1\\
1271	1\\
1272	1\\
1273	0.75\\
1274	0.75\\
1275	0.75\\
1276	0.5\\
1277	0.5\\
1278	0\\
1279	0\\
1280	0\\
1281	0\\
1282	0\\
1283	0.5\\
1284	0.5\\
1285	0.5\\
1286	0\\
1287	0\\
1288	0.75\\
1289	0.75\\
1290	0.75\\
1291	0.5\\
1292	0.5\\
1293	0.75\\
1294	0.75\\
1295	0.75\\
1296	0.5\\
1297	0.5\\
1298	0\\
1299	0\\
1300	0\\
1301	0\\
1302	0\\
1303	0.5\\
1304	0.5\\
1305	0.75\\
1306	0.75\\
1307	0.75\\
1308	0.75\\
1309	0.75\\
1310	1\\
1311	1\\
1312	0.75\\
1313	0.75\\
1314	0.75\\
1315	0.75\\
1316	0.75\\
1317	0.5\\
1318	0.5\\
1319	0.5\\
1320	0\\
1321	0\\
1322	0.5\\
1323	0.5\\
1324	0.5\\
1325	0\\
1326	0\\
1327	0.5\\
1328	0.5\\
1329	0.5\\
1330	0.5\\
1331	0\\
1332	0\\
1333	0\\
1334	0.75\\
1335	0.75\\
1336	0.75\\
1337	0.5\\
1338	0.5\\
1339	0.5\\
1340	0\\
1341	0\\
1342	0\\
1343	0.5\\
1344	0.5\\
1345	0.75\\
1346	0.75\\
1347	0.75\\
1348	0.5\\
1349	0.5\\
1350	0\\
1351	0\\
1352	0\\
1353	0\\
1354	0.5\\
1355	0.5\\
1356	0.75\\
1357	0.75\\
1358	0.75\\
1359	0.75\\
1360	0.75\\
1361	0.5\\
1362	0.5\\
1363	0.5\\
1364	0.5\\
1365	0.5\\
1366	0\\
1367	0\\
1368	0\\
1369	0.5\\
1370	0.5\\
1371	0.5\\
1372	0\\
1373	0\\
1374	0.5\\
1375	0.5\\
1376	0.5\\
1377	0.75\\
1378	0.75\\
1379	0.75\\
1380	1\\
1381	1\\
1382	1\\
1383	0.75\\
1384	0.75\\
1385	0.5\\
1386	0.5\\
1387	0\\
1388	0\\
1389	0\\
1390	0\\
1391	0\\
1392	0\\
1393	0.5\\
1394	0.5\\
1395	0.75\\
1396	0.75\\
1397	1\\
1398	1\\
1399	0.75\\
1400	0.75\\
1401	0.75\\
1402	0.75\\
1403	0.75\\
1404	0.5\\
1405	0.5\\
1406	0.5\\
1407	0\\
1408	0\\
1409	0.5\\
1410	0.5\\
1411	0.5\\
1412	0.5\\
1413	0\\
1414	0\\
1415	0\\
1416	0.5\\
1417	0.5\\
1418	0.5\\
1419	0\\
1420	0\\
1421	0.75\\
1422	0.75\\
1423	0.75\\
1424	0.75\\
1425	0.75\\
1426	0.5\\
1427	0.5\\
1428	0.5\\
1429	0\\
1430	0\\
1431	0\\
1432	0.5\\
1433	0.5\\
1434	0.75\\
1435	0.75\\
1436	0.75\\
1437	0.75\\
1438	0.5\\
1439	0.5\\
1440	0.5\\
1441	0\\
1442	0\\
1443	0.5\\
1444	0.5\\
1445	0.5\\
1446	0.75\\
1447	0.75\\
1448	0.75\\
1449	0.75\\
1450	0.5\\
1451	0.5\\
1452	0\\
1453	0\\
1454	0\\
1455	0\\
1456	0.5\\
1457	0.5\\
1458	0.5\\
1459	0\\
1460	0\\
1461	0.5\\
1462	0.5\\
1463	0.5\\
1464	0\\
1465	0\\
1466	0.5\\
1467	0.5\\
1468	0.5\\
1469	0\\
1470	0\\
1471	0.75\\
1472	0.75\\
1473	0.75\\
1474	0.5\\
1475	0.5\\
1476	0.5\\
1477	0\\
1478	0\\
1479	0\\
1480	0\\
1481	0.5\\
1482	0.5\\
1483	0.75\\
1484	0.75\\
1485	0.5\\
1486	0.5\\
1487	0.5\\
1488	0.5\\
1489	0.75\\
1490	0.75\\
1491	0.75\\
1492	0.75\\
1493	0.5\\
1494	0.5\\
1495	0\\
1496	0\\
1497	0.5\\
1498	0.5\\
1499	0.5\\
1500	0.5\\
1501	0\\
1502	0\\
1503	0\\
1504	0\\
1505	0.5\\
1506	0.5\\
1507	1\\
1508	1\\
1509	1\\
1510	1\\
1511	0.75\\
1512	0.75\\
1513	0.5\\
1514	0.5\\
1515	0.5\\
1516	0\\
1517	0\\
1518	0.5\\
1519	0.5\\
1520	0.5\\
1521	0\\
1522	0\\
1523	0.5\\
1524	0.5\\
1525	0.5\\
1526	0.5\\
1527	0.5\\
1528	0\\
1529	0\\
1530	0.5\\
1531	0.5\\
1532	0.5\\
1533	0.5\\
1534	0.75\\
1535	0.75\\
1536	0.75\\
1537	0.5\\
1538	0.5\\
1539	0\\
1540	0\\
1541	0\\
1542	0\\
1543	0\\
1544	0\\
1545	0.5\\
1546	0.5\\
1547	0.75\\
1548	0.75\\
1549	0.75\\
1550	0.5\\
1551	0.5\\
1552	0\\
1553	0\\
1554	0.5\\
1555	0.5\\
1556	0.5\\
1557	0.5\\
1558	0\\
1559	0\\
1560	0.5\\
1561	0.5\\
1562	0.5\\
1563	0.75\\
1564	0.75\\
1565	0.5\\
1566	0.5\\
1567	0.5\\
1568	0.5\\
1569	0.5\\
1570	0\\
1571	0\\
1572	0\\
1573	0\\
1574	0.5\\
1575	0.5\\
1576	0.75\\
1577	0.75\\
1578	0.5\\
1579	0.5\\
1580	0.5\\
1581	0.5\\
1582	0.5\\
1583	0\\
1584	0\\
1585	0.5\\
1586	0.5\\
1587	0.5\\
1588	0.5\\
1589	0.5\\
1590	0.5\\
1591	0.5\\
1592	0\\
1593	0\\
1594	0\\
1595	0\\
1596	0.75\\
1597	0.75\\
1598	1\\
1599	1\\
1600	0.75\\
1601	0.75\\
1602	0.75\\
1603	0.75\\
1604	0.5\\
1605	0.5\\
1606	0.5\\
1607	0\\
1608	0\\
1609	0.5\\
1610	0.5\\
1611	0.5\\
1612	0.5\\
1613	0\\
1614	0\\
1615	0\\
1616	0.5\\
1617	0.5\\
1618	0.75\\
1619	0.75\\
1620	0.5\\
1621	0.5\\
1622	0.75\\
1623	0.75\\
1624	0.75\\
1625	0.75\\
1626	0.5\\
1627	0.5\\
1628	0.5\\
1629	0\\
1630	0\\
1631	0\\
1632	0\\
1633	0.5\\
1634	0.5\\
1635	0.75\\
1636	0.75\\
1637	0.75\\
1638	0.75\\
1639	0.75\\
1640	0.5\\
1641	0.5\\
1642	0.75\\
1643	0.75\\
1644	0.5\\
1645	0.5\\
1646	0.5\\
1647	0.5\\
1648	0\\
1649	0\\
1650	0.5\\
1651	0.5\\
1652	0.5\\
1653	0.5\\
1654	0\\
1655	0\\
1656	0.5\\
1657	0.5\\
1658	0.5\\
1659	0\\
1660	0\\
1661	0\\
1662	0.5\\
1663	0.5\\
1664	0.5\\
1665	0.75\\
1666	0.75\\
1667	0.5\\
1668	0.5\\
1669	0.5\\
1670	0.5\\
1671	0.5\\
1672	0\\
1673	0\\
1674	0\\
1675	0.5\\
1676	0.5\\
1677	0.75\\
1678	0.75\\
1679	0.75\\
1680	0.75\\
1681	0.75\\
1682	0.5\\
1683	0.5\\
1684	0.5\\
1685	0\\
1686	0\\
1687	0\\
1688	0\\
1689	0.75\\
1690	0.75\\
1691	0.5\\
1692	0.5\\
1693	0.75\\
1694	0.75\\
1695	0.75\\
1696	0.75\\
1697	0.5\\
1698	0.5\\
1699	0.5\\
1700	0\\
1701	0\\
1702	0.5\\
1703	0.5\\
1704	0.5\\
1705	0.5\\
1706	0\\
1707	0\\
1708	0\\
1709	0\\
1710	0.5\\
1711	0.5\\
1712	0.75\\
1713	0.75\\
1714	0.75\\
1715	1\\
1716	1\\
1717	1\\
1718	1\\
1719	0.75\\
1720	0.75\\
1721	0.5\\
1722	0.5\\
1723	0.5\\
1724	0\\
1725	0\\
1726	0\\
1727	0\\
1728	0.5\\
1729	0.5\\
1730	0.5\\
1731	0.5\\
1732	0.5\\
1733	0.75\\
1734	0.75\\
1735	0.5\\
1736	0.5\\
1737	0.5\\
1738	0.5\\
1739	0\\
1740	0\\
1741	0.5\\
1742	0.5\\
1743	0.5\\
1744	0.5\\
1745	0\\
1746	0\\
1747	0\\
1748	0\\
1749	0.5\\
1750	0.5\\
1751	0.75\\
1752	0.75\\
1753	1\\
1754	1\\
1755	0.75\\
1756	0.75\\
1757	0.5\\
1758	0.5\\
1759	0.5\\
1760	0.5\\
1761	0.5\\
1762	0\\
1763	0\\
1764	0\\
1765	0.5\\
1766	0.5\\
1767	0.75\\
1768	0.75\\
1769	0.5\\
1770	0.5\\
1771	0.5\\
1772	0.5\\
1773	0\\
1774	0\\
1775	0\\
1776	0.75\\
1777	0.75\\
1778	0.75\\
1779	0.5\\
1780	0.5\\
1781	0.5\\
1782	0\\
1783	0\\
1784	0.5\\
1785	0.5\\
1786	0.5\\
1787	0.5\\
1788	0.5\\
1789	0\\
1790	0\\
1791	0.5\\
1792	0.5\\
1793	0.5\\
1794	0.5\\
1795	0.5\\
1796	0\\
1797	0\\
1798	0\\
1799	0\\
1800	0.5\\
1801	0.5\\
1802	0.75\\
1803	0.75\\
1804	1\\
1805	1\\
1806	1\\
1807	1\\
1808	0.75\\
1809	0.75\\
1810	0.5\\
1811	0.5\\
1812	0.5\\
1813	0.5\\
1814	0.5\\
1815	0\\
1816	0\\
1817	0\\
1818	0.5\\
1819	0.5\\
1820	0.5\\
1821	0\\
1822	0\\
1823	0.5\\
1824	0.5\\
1825	0.75\\
1826	0.75\\
1827	0.75\\
1828	0.75\\
1829	0.75\\
1830	0.5\\
1831	0.5\\
1832	0.75\\
1833	0.75\\
1834	0.75\\
1835	0.5\\
1836	0.5\\
1837	0\\
1838	0\\
1839	0\\
1840	0\\
1841	0\\
1842	0.75\\
1843	0.75\\
1844	1\\
1845	1\\
1846	0.75\\
1847	0.75\\
1848	0.75\\
1849	0.75\\
1850	0.75\\
1851	0.5\\
1852	0.5\\
1853	0.5\\
1854	0\\
1855	0\\
1856	0.5\\
1857	0.5\\
1858	0.5\\
1859	0.5\\
1860	0\\
1861	0\\
1862	0.5\\
1863	0.5\\
1864	0.5\\
1865	0.5\\
1866	0\\
1867	0\\
1868	0.75\\
1869	0.75\\
1870	0.75\\
1871	0.5\\
1872	0.5\\
1873	0.5\\
1874	0\\
1875	0\\
1876	0\\
1877	0.5\\
1878	0.5\\
1879	0.75\\
1880	0.75\\
1881	0.75\\
1882	0.75\\
1883	0.5\\
1884	0.5\\
1885	0.5\\
1886	0\\
1887	0\\
1888	0\\
1889	0.5\\
1890	0.5\\
1891	0.5\\
1892	0.75\\
1893	0.75\\
1894	0.75\\
1895	0.5\\
1896	0.5\\
1897	0.5\\
1898	0.5\\
1899	0.5\\
1900	0.5\\
1901	0.5\\
1902	0\\
1903	0\\
1904	0\\
1905	0.5\\
1906	0.5\\
1907	0.5\\
1908	0\\
1909	0\\
1910	0.5\\
1911	0.5\\
1912	0.5\\
1913	0.5\\
1914	0.5\\
1915	0.75\\
1916	0.75\\
1917	0.75\\
1918	0.5\\
1919	0.5\\
1920	0\\
1921	0\\
1922	0\\
1923	0\\
1924	0\\
1925	0.5\\
1926	0.5\\
1927	1\\
1928	1\\
1929	0.75\\
1930	0.75\\
1931	0.75\\
1932	0.75\\
1933	0.75\\
1934	0.5\\
1935	0.5\\
1936	0.5\\
1937	0\\
1938	0\\
1939	0.5\\
1940	0.5\\
1941	0.5\\
1942	0.5\\
1943	0\\
1944	0\\
1945	0\\
1946	0\\
1947	0.5\\
1948	0.5\\
1949	0.75\\
1950	0.75\\
1951	0.5\\
1952	0.5\\
1953	0\\
1954	0\\
1955	0\\
1956	0.5\\
1957	0.5\\
1958	0.75\\
1959	0.75\\
1960	0.75\\
1961	0.5\\
1962	0.5\\
1963	0.5\\
1964	0.5\\
1965	0\\
1966	0\\
1967	0.5\\
1968	0.5\\
1969	0.5\\
1970	0.75\\
1971	0.75\\
1972	0.75\\
1973	0.75\\
1974	0.5\\
1975	0.5\\
1976	0\\
1977	0\\
1978	0\\
1979	0\\
1980	0\\
1981	0.5\\
1982	0.5\\
1983	0.5\\
1984	0\\
1985	0\\
1986	0.5\\
1987	0.5\\
1988	0.5\\
1989	0.5\\
1990	0\\
1991	0\\
1992	0.5\\
1993	0.5\\
1994	0.5\\
1995	0\\
1996	0\\
1997	0.5\\
1998	0.5\\
1999	0.75\\
2000	0.75\\
2001	0.75\\
2002	0.75\\
2003	0.75\\
2004	0.5\\
2005	0.5\\
2006	0.5\\
2007	0\\
2008	0\\
2009	0\\
2010	0\\
2011	0.5\\
2012	0.5\\
2013	0.75\\
2014	0.75\\
2015	0.5\\
2016	0.5\\
2017	0.5\\
2018	0.5\\
2019	0.5\\
2020	0.75\\
2021	0.75\\
2022	0.75\\
2023	0.5\\
2024	0.5\\
2025	0\\
2026	0\\
2027	0.5\\
2028	0.5\\
2029	0.5\\
2030	0.5\\
2031	0\\
2032	0\\
2033	0\\
2034	0\\
2035	0.5\\
2036	0.5\\
2037	1\\
2038	1\\
2039	1\\
2040	1\\
2041	1\\
2042	0.75\\
2043	0.75\\
2044	0.75\\
2045	0.5\\
2046	0.5\\
2047	0.5\\
2048	0\\
2049	0\\
2050	0.5\\
2051	0.5\\
2052	0.5\\
2053	0.5\\
2054	0\\
2055	0\\
2056	0.5\\
2057	0.5\\
2058	0.5\\
2059	0.5\\
2060	0\\
2061	0\\
2062	0.5\\
2063	0.5\\
2064	0.5\\
2065	0.5\\
2066	0.5\\
2067	0.75\\
2068	0.75\\
2069	0.75\\
2070	0.5\\
2071	0.5\\
2072	0\\
2073	0\\
2074	0\\
2075	0\\
2076	0\\
2077	0\\
2078	0.5\\
2079	0.5\\
2080	0.75\\
2081	0.75\\
2082	0.75\\
2083	0.5\\
2084	0.5\\
2085	0\\
2086	0\\
2087	0.5\\
2088	0.5\\
2089	0.5\\
2090	0.5\\
2091	0.5\\
2092	0\\
2093	0\\
2094	0\\
2095	0.5\\
2096	0.5\\
2097	0.5\\
2098	0.5\\
2099	0.5\\
2100	0.5\\
2101	0\\
2102	0\\
2103	0\\
2104	0\\
2105	0.5\\
2106	0.5\\
2107	0.75\\
2108	0.75\\
2109	0.5\\
2110	0.5\\
2111	0.5\\
2112	0.5\\
2113	0.5\\
2114	0\\
2115	0\\
2116	0.5\\
2117	0.5\\
2118	0.5\\
2119	0.75\\
2120	0.75\\
2121	0.75\\
2122	0.75\\
2123	0.5\\
2124	0.5\\
2125	0\\
2126	0\\
2127	0\\
2128	0\\
2129	0\\
2130	0\\
2131	0.75\\
2132	0.75\\
2133	1\\
2134	1\\
2135	0.75\\
2136	0.75\\
2137	0.75\\
2138	0.75\\
2139	0.5\\
2140	0.5\\
2141	0.5\\
2142	0\\
2143	0\\
2144	0.5\\
2145	0.5\\
2146	0.5\\
2147	0.5\\
2148	0\\
2149	0\\
2150	0\\
2151	0.5\\
2152	0.5\\
2153	0.75\\
2154	0.75\\
2155	0.75\\
2156	0.5\\
2157	0.5\\
2158	0.75\\
2159	0.75\\
2160	0.75\\
2161	0.75\\
2162	0.75\\
2163	0.5\\
2164	0.5\\
2165	0.5\\
2166	0\\
2167	0\\
2168	0\\
2169	0\\
2170	0.5\\
2171	0.5\\
2172	0.75\\
2173	0.75\\
2174	0.75\\
2175	0.75\\
2176	0.5\\
2177	0.5\\
2178	0.75\\
2179	0.75\\
2180	0.5\\
2181	0.5\\
2182	0.5\\
2183	0.5\\
2184	0\\
2185	0\\
2186	0.5\\
2187	0.5\\
2188	0.5\\
2189	0.5\\
2190	0\\
2191	0\\
2192	0\\
2193	0.5\\
2194	0.5\\
2195	0.5\\
2196	0\\
2197	0\\
2198	0.5\\
2199	0.5\\
2200	0.5\\
2201	0.75\\
2202	0.75\\
2203	0.5\\
2204	0.5\\
2205	0.5\\
2206	0.5\\
2207	0.5\\
2208	0\\
2209	0\\
2210	0\\
2211	0\\
2212	0.75\\
2213	0.75\\
2214	0.5\\
2215	0.5\\
2216	0\\
2217	0\\
2218	0\\
2219	0.75\\
2220	0.75\\
2221	0.5\\
2222	0.5\\
2223	0.75\\
2224	0.75\\
2225	0.5\\
2226	0.5\\
2227	0.5\\
2228	0.5\\
2229	0\\
2230	0\\
2231	0.5\\
2232	0.5\\
2233	0.5\\
2234	0.5\\
2235	0\\
2236	0\\
2237	0\\
2238	0\\
2239	0.5\\
2240	0.5\\
2241	0.75\\
2242	0.75\\
2243	1\\
2244	1\\
2245	1\\
2246	1\\
2247	0.75\\
2248	0.75\\
2249	0.5\\
2250	0.5\\
2251	0\\
2252	0\\
2253	0\\
2254	0\\
2255	0.5\\
2256	0.5\\
2257	0\\
2258	0\\
2259	0.5\\
2260	0.5\\
2261	0.5\\
2262	0.5\\
2263	0.75\\
2264	0.75\\
2265	0.75\\
2266	0.5\\
2267	0.5\\
2268	0.5\\
2269	0.5\\
2270	0\\
2271	0\\
2272	0.5\\
2273	0.5\\
2274	0.5\\
2275	0.5\\
2276	0.5\\
2277	0\\
2278	0\\
2279	0\\
2280	0\\
2281	0.5\\
2282	0.5\\
2283	0.75\\
2284	0.75\\
2285	0.75\\
2286	1\\
2287	1\\
2288	0.75\\
2289	0.75\\
2290	0.75\\
2291	0.5\\
2292	0.5\\
2293	0.5\\
2294	0.5\\
2295	0\\
2296	0\\
2297	0\\
2298	0.5\\
2299	0.5\\
2300	0.5\\
2301	0.75\\
2302	0.75\\
2303	0.5\\
2304	0.5\\
2305	0.5\\
2306	0.5\\
2307	0.5\\
2308	0\\
2309	0\\
2310	0\\
2311	0\\
2312	0.75\\
2313	0.75\\
2314	0.5\\
2315	0.5\\
2316	0\\
2317	0\\
2318	0.5\\
2319	0.5\\
2320	0.5\\
2321	0.5\\
2322	0.5\\
2323	0.5\\
2324	0.5\\
2325	0\\
2326	0\\
2327	0\\
2328	0\\
2329	0.5\\
2330	0.5\\
2331	0.75\\
2332	0.75\\
2333	1\\
2334	1\\
2335	0.75\\
2336	0.75\\
2337	0.5\\
2338	0.5\\
2339	0.5\\
2340	0.5\\
2341	0.5\\
2342	0\\
2343	0\\
2344	0\\
2345	0.5\\
2346	0.5\\
2347	0.5\\
2348	0\\
2349	0\\
2350	0\\
2351	0.5\\
2352	0.5\\
2353	0.75\\
2354	0.75\\
2355	0.75\\
2356	0.75\\
2357	0.5\\
2358	0.5\\
2359	0.75\\
2360	0.75\\
2361	0.75\\
2362	0.75\\
2363	0.5\\
2364	0.5\\
2365	0\\
2366	0\\
2367	0\\
2368	0\\
2369	0\\
2370	0.5\\
2371	0.5\\
2372	0.75\\
2373	0.75\\
2374	0.75\\
2375	0.75\\
2376	0.75\\
2377	1\\
2378	1\\
2379	0.75\\
2380	0.75\\
2381	0.75\\
2382	0.75\\
2383	0.75\\
2384	0.5\\
2385	0.5\\
2386	0.5\\
2387	0\\
2388	0\\
2389	0.5\\
2390	0.5\\
2391	0.5\\
2392	0.5\\
2393	0\\
2394	0\\
2395	0\\
2396	0.5\\
2397	0.5\\
2398	0.5\\
2399	0\\
2400	0\\
2401	0\\
2402	0.75\\
2403	0.75\\
2404	0.75\\
2405	0.5\\
2406	0.5\\
2407	0.5\\
2408	0\\
2409	0\\
2410	0\\
2411	0.5\\
2412	0.5\\
2413	0.75\\
2414	0.75\\
2415	0.75\\
2416	0.75\\
2417	0.75\\
2418	0.5\\
2419	0.5\\
2420	0.5\\
2421	0\\
2422	0\\
2423	0\\
2424	0.5\\
2425	0.5\\
2426	0.5\\
2427	0.75\\
2428	0.75\\
2429	0.75\\
2430	0.5\\
2431	0.5\\
2432	0\\
2433	0\\
2434	0.5\\
2435	0.5\\
2436	0.5\\
2437	0.5\\
2438	0.5\\
2439	0\\
2440	0\\
2441	0\\
2442	0.5\\
2443	0.5\\
2444	0.5\\
2445	0\\
2446	0\\
2447	0\\
2448	0.5\\
2449	0.5\\
2450	0.5\\
2451	0\\
2452	0\\
2453	0.5\\
2454	0.5\\
2455	0.5\\
2456	0.5\\
2457	0.75\\
2458	0.75\\
2459	0.75\\
2460	0.75\\
2461	0.5\\
2462	0.5\\
2463	0\\
2464	0\\
2465	0\\
2466	0\\
2467	0.5\\
2468	0.5\\
2469	0.75\\
2470	0.75\\
2471	0.75\\
2472	0.75\\
2473	0.75\\
2474	0.75\\
2475	0.75\\
2476	0.5\\
2477	0.5\\
2478	0.5\\
2479	0\\
2480	0\\
2481	0.5\\
2482	0.5\\
2483	0.5\\
2484	0.5\\
2485	0\\
2486	0\\
2487	0\\
2488	0\\
2489	0.5\\
2490	0.5\\
2491	1\\
2492	1\\
2493	1\\
2494	0.75\\
2495	0.75\\
2496	0.5\\
2497	0.5\\
2498	0.5\\
2499	0.5\\
2500	0.5\\
2501	0\\
2502	0\\
2503	0\\
2504	0.5\\
2505	0.5\\
2506	0.5\\
2507	0.75\\
2508	0.75\\
2509	0.5\\
2510	0.5\\
2511	0.5\\
2512	0.5\\
2513	0\\
2514	0\\
2515	0.5\\
2516	0.5\\
2517	0.5\\
2518	0.75\\
2519	0.75\\
2520	0.75\\
2521	0.75\\
2522	0.5\\
2523	0.5\\
2524	0\\
2525	0\\
2526	0\\
2527	0\\
2528	0\\
2529	0.5\\
2530	0.5\\
2531	0.5\\
2532	0\\
2533	0\\
2534	0.5\\
2535	0.5\\
2536	0.5\\
2537	0.5\\
2538	0.5\\
2539	0\\
2540	0\\
2541	0\\
2542	0.5\\
2543	0.5\\
2544	0.5\\
2545	0\\
2546	0\\
2547	0\\
2548	0.5\\
2549	0.5\\
2550	0.75\\
2551	0.75\\
2552	0.75\\
2553	0.75\\
2554	0.5\\
2555	0.5\\
2556	0.5\\
2557	0\\
2558	0\\
2559	0\\
2560	0\\
2561	0.5\\
2562	0.5\\
2563	0.75\\
2564	0.75\\
2565	0.5\\
2566	0.5\\
2567	0.5\\
2568	0.5\\
2569	0.5\\
2570	0.75\\
2571	0.75\\
2572	0.75\\
2573	0.5\\
2574	0.5\\
2575	0\\
2576	0\\
2577	0\\
2578	0.5\\
2579	0.5\\
2580	0.5\\
2581	0.5\\
2582	0\\
2583	0\\
2584	0\\
2585	0\\
2586	1\\
};
\end{axis}
\end{tikzpicture}%
}
  \caption{The overall processing usage. Notice that it is at most at $100\%$.
    $C_i = 6$ ms.}
  \label{fig:01.4.2}
\end{figure}
