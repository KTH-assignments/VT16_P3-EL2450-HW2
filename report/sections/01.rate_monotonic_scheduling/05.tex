\subsubsection{Question 5}
In the case where $T_1 = 20, T_2 = 29, T_3 = 35$ ms and $C_i = 10$ ms,
$i=\{1,2,3\}$, $U=1.131 > 1$. Hence tasks $J_1, J_2, J_3$ are not schedulable
under any scheduling scheme.


\begin{figure}[H]\centering
  \scalebox{0.7}{\begin{ganttchart}[vgrid, hgrid]{0}{47}
%\gantttitle{2016}{12}\\
\gantttitlelist{6,12,...,144}{6}\\
\ganttset{progress label text={},
       bar incomplete/.append style={fill=black!40},
       group/.append style={draw=black, fill=black},}
\ganttbar{Task 1}{0}{5}
\ganttbar{}{20}{25}
\ganttbar{}{40}{45}
\ganttbar{}{60}{65}
\ganttbar{}{80}{85}
\ganttbar{}{100}{105}
\ganttbar{}{120}{125}\\

\ganttbar[progress=00]{Task 2}{0}{5}
\ganttbar{}{6}{11}
\ganttbar{}{29}{34}
\ganttbar[progress=00]{}{60}{65}
\ganttbar{}{66}{69}
\ganttbar{}{87}{92}
\ganttbar{}{116}{119}
\ganttbar[progress=00]{}{120}{125}
\ganttbar{}{126}{127}\\

\ganttbar[progress=00]{Task 3}{0}{11}
\ganttbar{}{12}{17}
\ganttbar{}{35}{39}
\ganttbar[progress=00]{}{40}{45}
\ganttbar{}{46}{46}
\ganttbar{}{70}{75}
\ganttbar[progress=00]{}{105}{105}
\ganttbar{}{106}{111}
\end{ganttchart}
}
  \caption{A portion of the RM schedule $\sigma$ for tasks $J_1, J_2, J_3$ for
    $C_i = 10$ ms.  Shaded areas denote the waiting time. Notice that $J_3$
    misses its deadlines consecutively, indicative of the inability of
    schedulability.}
\end{figure}

Pendulum $P_3$ is left to its own devices and tends slowly but surely towards
instability. This makes sense since no control signal pertaining to $P_3$ is
assigned execution time in the processor. Figure \ref{fig:01.5.2} shows the
angular displacement of pendulum $P_3$ as a function of time.

For penduli $P_1$ and $P_2$, however, due to the increased execution time,
delays are introduced, and the control input is not as swift as before,
hence the increased magnitude of the overshoot and the larger rise and settling
times. Figure \ref{fig:01.5.1} shows the angular displacement of each stable
pendulum as a function of time.

Figures \ref{fig:01.5.6_10.1} and \ref{fig:01.5.6_10.2} show the angular
displacement of penduli $P_1$ and $P_2$ respectively, as a function of time, for
the two different cases of execution time.


\noindent\makebox[\textwidth][c]{%
\begin{minipage}{\linewidth}
  \begin{minipage}{0.45\linewidth}
    \begin{figure}[H]\centering
      \scalebox{0.7}{% This file was created by matlab2tikz.
%
%The latest updates can be retrieved from
%  http://www.mathworks.com/matlabcentral/fileexchange/22022-matlab2tikz-matlab2tikz
%where you can also make suggestions and rate matlab2tikz.
%
\definecolor{mycolor1}{rgb}{0.00000,0.44700,0.74100}%
\definecolor{mycolor2}{rgb}{0.85000,0.32500,0.09800}%
%
\begin{tikzpicture}

\begin{axis}[%
width=4.133in,
height=3.26in,
at={(0.693in,0.44in)},
scale only axis,
xmin=0,
xmax=1,
xmajorgrids,
ymin=-0.1,
ymax=0.2,
ymajorgrids,
axis background/.style={fill=white}
]
\addplot [color=mycolor1,solid,forget plot]
  table[row sep=crcr]{%
0	0.15314\\
3.15544362088405e-30	0.15314\\
0.000656101980281985	0.153143230512962\\
0.00393661188169191	0.153256312778436\\
0.00999999999999994	0.153891071773171\\
0.01	0.153891071773171\\
0.0199999999999999	0.150048203824684\\
0.02	0.150048203824684\\
0.0289999999999998	0.137414337712804\\
0.029	0.137414337712803\\
0.03	0.135470213386942\\
0.0300000000000002	0.135470213386942\\
0.0349999999999996	0.124799943342144\\
0.035	0.124799943342143\\
0.0399999999999993	0.112755527823867\\
0.04	0.112755527823865\\
0.0449999999999993	0.0993074487540181\\
0.0499999999999987	0.0844227483826206\\
0.05	0.0844227483826165\\
0.0500000000000004	0.0844227483826151\\
0.0579999999999996	0.0596508269858389\\
0.058	0.0596508269858376\\
0.0599999999999996	0.0535139519248594\\
0.06	0.0535139519248581\\
0.0619999999999995	0.0473946951543089\\
0.0639999999999991	0.0412906572523715\\
0.0679999999999982	0.0291186707010022\\
0.0699999999999991	0.0230459496574581\\
0.07	0.0230459496574554\\
0.0779999999999982	0.000115509555903549\\
0.0779999999999991	0.000115509555901112\\
0.078	0.000115509555898676\\
0.0799999999999991	-0.00520529938069268\\
0.08	-0.00520529938069501\\
0.0819999999999991	-0.0103656555449064\\
0.0839999999999982	-0.0153675824104684\\
0.0869999999999991	-0.0225776906978477\\
0.087	-0.0225776906978498\\
0.0899999999999991	-0.0294420913674593\\
0.09	-0.0294420913674613\\
0.0929999999999991	-0.0358989900454877\\
0.0959999999999982	-0.0418862328905422\\
0.0999999999999991	-0.0491477173641583\\
0.1	-0.0491477173641598\\
0.104999999999999	-0.0570807990847102\\
0.105	-0.0570807990847115\\
0.109999999999999	-0.0637610218754624\\
0.11	-0.0637610218754635\\
0.114999999999999	-0.0692474781164248\\
0.116	-0.0702078763842551\\
0.116000000000001	-0.0702078763842559\\
0.12	-0.0735963345728153\\
0.120000000000001	-0.073596334572816\\
0.124	-0.0762636029644891\\
0.127999999999999	-0.0782138647982409\\
0.129999999999999	-0.0789211179635345\\
0.130000000000001	-0.0789211179635351\\
0.135999999999999	-0.08020719761647\\
0.136000000000001	-0.0802071976164702\\
0.139999999999998	-0.0804343672206592\\
0.14	-0.0804343672206592\\
0.143999999999997	-0.0801578942840472\\
0.144999999999998	-0.0800100484516686\\
0.145	-0.0800100484516683\\
0.148999999999997	-0.0791033199779474\\
0.149999999999998	-0.0787976904879804\\
0.15	-0.0787976904879798\\
0.153999999999997	-0.0773811971506685\\
0.157999999999995	-0.0757016984773037\\
0.159999999999998	-0.0747625151777411\\
0.16	-0.0747625151777403\\
0.167999999999995	-0.0703339909543648\\
0.17	-0.0690567418759476\\
0.170000000000002	-0.0690567418759465\\
0.173999999999998	-0.0664014575579397\\
0.174	-0.0664014575579385\\
0.174999999999998	-0.0657273809182214\\
0.175	-0.0657273809182202\\
0.176	-0.065049070523484\\
0.177	-0.0643664598898951\\
0.179000000000001	-0.0629880698663664\\
0.179999999999998	-0.0622921553774208\\
0.18	-0.0622921553774196\\
0.184000000000001	-0.0594621041509566\\
0.188000000000002	-0.0565544783067479\\
0.189999999999998	-0.0550701547058472\\
0.19	-0.0550701547058459\\
0.193999999999998	-0.0521127014356976\\
0.194	-0.0521127014356963\\
0.197999999999998	-0.0492163971314518\\
0.199999999999998	-0.0477897527246364\\
0.2	-0.0477897527246351\\
0.202999999999998	-0.0456750957196064\\
0.203	-0.0456750957196052\\
0.205999999999998	-0.0435891540275711\\
0.208999999999997	-0.0415300875308923\\
0.209999999999998	-0.0408493896873756\\
0.21	-0.0408493896873744\\
0.215999999999997	-0.0369141020807761\\
0.22	-0.034442653538239\\
0.220000000000002	-0.0344426535382379\\
0.225999999999998	-0.0309539039491991\\
0.23	-0.0287683666824412\\
0.230000000000002	-0.0287683666824403\\
0.231999999999998	-0.0277201050108415\\
0.232	-0.0277201050108406\\
0.233999999999996	-0.0267059225018899\\
0.235999999999993	-0.0257254215228329\\
0.239999999999986	-0.0238639395822936\\
0.24	-0.0238639395822873\\
0.240000000000002	-0.0238639395822865\\
0.244999999999998	-0.0217199736534009\\
0.245	-0.0217199736534001\\
0.249999999999997	-0.0197743341989149\\
0.249999999999999	-0.0197743341989139\\
0.250000000000002	-0.019774334198913\\
0.251999999999996	-0.0190498245766449\\
0.252	-0.0190498245766436\\
0.253999999999995	-0.0183545927475019\\
0.255999999999989	-0.0176883661383964\\
0.259999999999979	-0.0164418950697132\\
0.259999999999995	-0.0164418950697086\\
0.26	-0.016441895069707\\
0.260999999999997	-0.0161480110554556\\
0.261	-0.0161480110554546\\
0.262	-0.0158611619153846\\
0.263	-0.0155813195204978\\
0.265	-0.0150425459404598\\
0.269	-0.0140479186003353\\
0.269999999999996	-0.0138163935677859\\
0.27	-0.0138163935677851\\
0.278	-0.0121608623944767\\
0.279999999999996	-0.0117992539565872\\
0.28	-0.0117992539565866\\
0.288	-0.0105576632685836\\
0.289999999999996	-0.0102978678801549\\
0.29	-0.0102978678801545\\
0.298	-0.0094028571170321\\
0.299999999999996	-0.00921126131825964\\
0.3	-0.00921126131825931\\
0.308	-0.00857125155207012\\
0.309999999999996	-0.00844252880878156\\
0.31	-0.00844252880878134\\
0.314999999999997	-0.00815634985129781\\
0.315	-0.00815634985129762\\
0.319	-0.00795606106076335\\
0.319000000000004	-0.00795606106076318\\
0.319999999999996	-0.00790992644388172\\
0.32	-0.00790992644388155\\
0.321	-0.0078653575850625\\
0.322	-0.0078223501159026\\
0.324	-0.00774100263846268\\
0.328	-0.0075968690952334\\
0.329999999999996	-0.00753402651682357\\
0.33	-0.00753402651682346\\
0.338	-0.00731091603026741\\
0.339000000000004	-0.00728524225051462\\
0.339000000000007	-0.00728524225051453\\
0.339999999999996	-0.00726005203286719\\
0.34	-0.0072600520328671\\
0.340999999999996	-0.00723534290961056\\
0.341999999999993	-0.00721111245776153\\
0.343999999999986	-0.00716407811562298\\
0.347999999999972	-0.00707565211048863\\
0.348	-0.00707565211048802\\
0.348000000000004	-0.00707565211048795\\
0.349999999999996	-0.00703422577719816\\
0.35	-0.00703422577719809\\
0.351999999999993	-0.00699355247184833\\
0.353999999999986	-0.00695253338627603\\
0.357999999999972	-0.00686939340626633\\
0.359999999999997	-0.00682723991434748\\
0.36	-0.00682723991434741\\
0.367999999999972	-0.00665434840384323\\
0.369999999999997	-0.00660997141115594\\
0.37	-0.00660997141115586\\
0.376999999999997	-0.006446757644875\\
0.377	-0.00644675764487491\\
0.379999999999997	-0.00637238566943073\\
0.38	-0.00637238566943065\\
0.382999999999997	-0.00629526613701222\\
0.384999999999997	-0.00624229319973811\\
0.385	-0.00624229319973802\\
0.387999999999997	-0.00616044199835192\\
0.39	-0.00610424880187564\\
0.390000000000004	-0.00610424880187554\\
0.393	-0.00601786612427311\\
0.395999999999997	-0.00592922472555357\\
0.397	-0.00589916219766626\\
0.397000000000004	-0.00589916219766615\\
0.399999999999997	-0.0058073871785191\\
0.4	-0.00580738717851899\\
0.402999999999993	-0.00571316776508011\\
0.405999999999986	-0.00561642084048825\\
0.405999999999997	-0.0056164208404879\\
0.406	-0.00561642084048778\\
0.409999999999997	-0.00548334541510819\\
0.41	-0.00548334541510807\\
0.413999999999997	-0.00534707708841722\\
0.417999999999993	-0.00520906329388098\\
0.419999999999997	-0.00513933431621019\\
0.42	-0.00513933431621006\\
0.427999999999993	-0.00485496423801058\\
0.429999999999997	-0.00478236881457504\\
0.43	-0.00478236881457491\\
0.434999999999997	-0.00460081789937476\\
0.435	-0.00460081789937463\\
0.439999999999997	-0.00442046671457951\\
0.44	-0.00442046671457939\\
0.444999999999997	-0.00424087326153496\\
0.449999999999993	-0.00406159740331977\\
0.45	-0.00406159740331952\\
0.450000000000004	-0.00406159740331939\\
0.454999999999997	-0.00388464242696522\\
0.455	-0.0038846424269651\\
0.459999999999993	-0.00371201736453094\\
0.46	-0.00371201736453071\\
0.463999999999997	-0.00357675021271568\\
0.464	-0.00357675021271557\\
0.467999999999997	-0.00344377110445797\\
0.469999999999997	-0.00337807426057455\\
0.47	-0.00337807426057443\\
0.473999999999997	-0.00324923018247093\\
0.477999999999993	-0.00312434644135401\\
0.479999999999997	-0.00306332825226448\\
0.48	-0.00306332825226437\\
0.487999999999993	-0.00282819266800841\\
0.489999999999997	-0.00277152783070031\\
0.49	-0.00277152783070021\\
0.492999999999997	-0.00268837657874228\\
0.493	-0.00268837657874219\\
0.495999999999997	-0.00260765371783775\\
0.498999999999993	-0.00252928803319682\\
0.499999999999997	-0.00250367794765325\\
0.5	-0.00250367794765316\\
0.505999999999993	-0.00235521701213284\\
0.509999999999993	-0.00226101849347193\\
0.51	-0.00226101849347176\\
0.512999999999993	-0.00219290696937763\\
0.513	-0.00219290696937747\\
0.515999999999993	-0.00212705890550972\\
0.518999999999986	-0.00206341621292921\\
0.519999999999993	-0.00204268230550604\\
0.52	-0.00204268230550589\\
0.521999999999993	-0.00200192274803122\\
0.522	-0.00200192274803108\\
0.523999999999993	-0.00196209425656451\\
0.524999999999993	-0.00194252426656321\\
0.525	-0.00194252426656308\\
0.526999999999993	-0.00190406320753698\\
0.528999999999986	-0.0018664948477686\\
0.529999999999993	-0.00184804081379888\\
0.53	-0.00184804081379874\\
0.533999999999986	-0.00177638202987842\\
0.537999999999972	-0.00170809081087881\\
0.539999999999993	-0.00167517436414704\\
0.54	-0.00167517436414693\\
0.547999999999972	-0.0015514055254998\\
0.549999999999993	-0.00152237689661286\\
0.55	-0.00152237689661275\\
0.550999999999993	-0.00150813397221752\\
0.551	-0.00150813397221742\\
0.551999999999993	-0.0014940599149228\\
0.552999999999987	-0.00148015334465862\\
0.554999999999973	-0.00145283722954938\\
0.558999999999945	-0.00140015595574187\\
0.559999999999993	-0.00138738453305394\\
0.56	-0.00138738453305385\\
0.567999999999945	-0.001290764355199\\
0.569999999999993	-0.0012681080442107\\
0.57	-0.00126810804421062\\
0.570999999999993	-0.00125698896066818\\
0.571	-0.0012569889606681\\
0.571999999999994	-0.00124599514542769\\
0.572999999999987	-0.00123512552046293\\
0.574999999999973	-0.00121375459208277\\
0.578999999999945	-0.00117245694661062\\
0.579999999999993	-0.00116242749231629\\
0.58	-0.00116242749231622\\
0.587999999999945	-0.00108628764628732\\
0.589999999999993	-0.00106835630058816\\
0.59	-0.0010683563005881\\
0.594999999999993	-0.00102518298860799\\
0.595	-0.00102518298860793\\
0.599999999999993	-0.000984176539447219\\
0.6	-0.000984176539447162\\
0.604999999999993	-0.000945236455672845\\
0.608999999999993	-0.000915507877246312\\
0.609	-0.00091550787724626\\
0.61	-0.000908267305150007\\
0.610000000000007	-0.000908267305149956\\
0.611000000000007	-0.000901096956710468\\
0.612000000000007	-0.000893991154167617\\
0.614000000000006	-0.000879970407259534\\
0.618000000000005	-0.000852673236240157\\
0.619999999999993	-0.000839386109424082\\
0.62	-0.000839386109424035\\
0.627999999999998	-0.000788526838264542\\
0.629999999999993	-0.000776359404581159\\
0.63	-0.000776359404581116\\
0.637999999999998	-0.000729611558851254\\
0.638000000000005	-0.000729611558851214\\
0.639999999999993	-0.000718374382523774\\
0.64	-0.000718374382523734\\
0.641999999999989	-0.000707308205907471\\
0.643999999999977	-0.000696408689799411\\
0.647999999999954	-0.000675092610098676\\
0.649999999999993	-0.00066466768887766\\
0.65	-0.000664667688877623\\
0.657999999999954	-0.000624453414342308\\
0.657999999999992	-0.00062445341434212\\
0.658000000000005	-0.000624453414342058\\
0.659999999999993	-0.000614753112835123\\
0.66	-0.000614753112835088\\
0.661999999999989	-0.000605186431259547\\
0.663999999999977	-0.000595749618389984\\
0.664999999999993	-0.000591078754592939\\
0.665	-0.000591078754592906\\
0.666999999999993	-0.000581829823308378\\
0.667	-0.000581829823308345\\
0.668999999999993	-0.000572701603768233\\
0.669999999999993	-0.000568181640729161\\
0.67	-0.000568181640729129\\
0.671999999999993	-0.000559235115436408\\
0.673999999999986	-0.000550415088342362\\
0.677999999999971	-0.000533140745379623\\
0.679999999999993	-0.000524679656571602\\
0.68	-0.000524679656571572\\
0.687999999999971	-0.000491932295087052\\
0.689999999999993	-0.000484003654470002\\
0.69	-0.000484003654469974\\
0.695999999999993	-0.00046087418664401\\
0.696	-0.000460874186643984\\
0.699999999999993	-0.000445995537281835\\
0.7	-0.000445995537281809\\
0.703999999999993	-0.000431522362865119\\
0.707999999999986	-0.000417431964163371\\
0.709999999999993	-0.000410523376917925\\
0.71	-0.000410523376917901\\
0.716	-0.000390396901469173\\
0.716000000000007	-0.000390396901469149\\
0.719999999999993	-0.000377476123497505\\
0.72	-0.000377476123497483\\
0.723999999999986	-0.000364929111947227\\
0.724999999999993	-0.000361848441560947\\
0.725000000000001	-0.000361848441560925\\
0.728999999999986	-0.000349741031181947\\
0.729999999999993	-0.000346766513055465\\
0.73	-0.000346766513055444\\
0.733999999999986	-0.000335100644214133\\
0.734999999999993	-0.000332242774229825\\
0.735	-0.000332242774229805\\
0.738999999999986	-0.00032103727951184\\
0.739999999999993	-0.000318291029310797\\
0.74	-0.000318291029310777\\
0.743999999999986	-0.000307518446958102\\
0.747999999999972	-0.000297073137188031\\
0.749999999999993	-0.000291968070496662\\
0.75	-0.000291968070496644\\
0.753999999999993	-0.000282006412962819\\
0.754	-0.000282006412962801\\
0.757999999999993	-0.000272378629803339\\
0.759999999999993	-0.000267685202435417\\
0.76	-0.000267685202435401\\
0.763999999999993	-0.000258530075133411\\
0.767999999999986	-0.000249672002432226\\
0.769999999999993	-0.000245350002246816\\
0.77	-0.000245350002246801\\
0.773999999999993	-0.000236927001864428\\
0.774000000000001	-0.000236927001864413\\
0.777999999999994	-0.000228797974828517\\
0.779999999999993	-0.000224839699566304\\
0.78	-0.00022483969956629\\
0.782999999999993	-0.000219030710109956\\
0.783	-0.000219030710109942\\
0.785999999999993	-0.000213371283756889\\
0.788999999999986	-0.000207856428032637\\
0.79	-0.000206049425123396\\
0.790000000000007	-0.000206049425123384\\
0.795999999999993	-0.000195544194518791\\
0.8	-0.000188852233065154\\
0.800000000000007	-0.000188852233065143\\
0.804999999999993	-0.000180819255107827\\
0.805	-0.000180819255107816\\
0.809999999999987	-0.000173136804447917\\
0.809999999999997	-0.000173136804447902\\
0.810000000000007	-0.000173136804447886\\
0.811999999999993	-0.000170158974392289\\
0.812	-0.000170158974392278\\
0.813999999999987	-0.000167235535321981\\
0.815999999999973	-0.000164365341073914\\
0.819999999999945	-0.00015878020642566\\
0.819999999999987	-0.000158780206425602\\
0.82	-0.000158780206425583\\
0.827999999999944	-0.000148200708436175\\
0.829999999999993	-0.000145672834812856\\
0.830000000000001	-0.000145672834812847\\
0.831999999999994	-0.000143190413239027\\
0.832000000000001	-0.000143190413239018\\
0.833999999999994	-0.000140753131532892\\
0.835999999999987	-0.000138360033999753\\
0.839999999999973	-0.000133702655540068\\
0.839999999999987	-0.000133702655540053\\
0.84	-0.000133702655540037\\
0.840999999999993	-0.000132564480459671\\
0.841000000000001	-0.000132564480459663\\
0.841999999999994	-0.000131436548631194\\
0.842999999999987	-0.0001303187494533\\
0.844999999999973	-0.000128113111807827\\
0.848999999999945	-0.000123818673087672\\
0.849999999999993	-0.000122768789122539\\
0.85	-0.000122768789122532\\
0.857999999999944	-0.000114702030170695\\
0.859999999999993	-0.000112774262092235\\
0.86	-0.000112774262092228\\
0.867999999999944	-0.000105396195871439\\
0.869999999999993	-0.000103631315024469\\
0.87	-0.000103631315024463\\
0.874999999999993	-9.93531334797321e-05\\
0.875000000000001	-9.93531334797261e-05\\
0.879999999999994	-9.5259379362353e-05\\
0.880000000000001	-9.52593793623474e-05\\
0.884999999999994	-9.13400198201427e-05\\
0.889999999999987	-8.75854495205424e-05\\
0.890000000000001	-8.7585449520532e-05\\
0.898999999999994	-8.12216795789393e-05\\
0.899000000000001	-8.12216795789345e-05\\
0.899999999999993	-8.05445188309035e-05\\
0.9	-8.05445188308987e-05\\
0.900999999999993	-7.98730955065059e-05\\
0.901999999999986	-7.92073437530437e-05\\
0.903999999999972	-7.7892594506006e-05\\
0.907999999999944	-7.53283202983093e-05\\
0.909999999999993	-7.40777899343446e-05\\
0.91	-7.40777899343402e-05\\
0.917999999999944	-6.92826039200973e-05\\
0.918999999999994	-6.87057273213319e-05\\
0.919000000000001	-6.87057273213278e-05\\
0.919999999999993	-6.81336642506684e-05\\
0.92	-6.81336642506644e-05\\
0.920999999999993	-6.75663586656561e-05\\
0.921999999999986	-6.70037549382955e-05\\
0.923999999999972	-6.58924329478781e-05\\
0.927999999999944	-6.37238152687967e-05\\
0.927999999999987	-6.37238152687741e-05\\
0.928000000000001	-6.37238152687666e-05\\
0.929999999999993	-6.26656693062966e-05\\
0.93	-6.26656693062929e-05\\
0.931999999999993	-6.16249210571799e-05\\
0.933999999999986	-6.06016735066236e-05\\
0.937999999999972	-5.86060825433679e-05\\
0.939999999999993	-5.76329566972932e-05\\
0.940000000000001	-5.76329566972898e-05\\
0.945000000000001	-5.52690413858537e-05\\
0.945000000000008	-5.52690413858504e-05\\
0.949999999999993	-5.29989476589604e-05\\
0.950000000000001	-5.29989476589573e-05\\
0.954999999999986	-5.08203058712376e-05\\
0.956999999999994	-4.9974146473985e-05\\
0.957000000000001	-4.9974146473982e-05\\
0.96	-4.87309705573583e-05\\
0.960000000000008	-4.87309705573554e-05\\
0.963000000000007	-4.75180976709576e-05\\
0.966000000000007	-4.63344578716468e-05\\
0.97	-4.47999440496539e-05\\
0.970000000000008	-4.47999440496513e-05\\
0.976000000000007	-4.25911829131623e-05\\
0.976999999999994	-4.22337744034521e-05\\
0.977000000000001	-4.22337744034496e-05\\
0.979999999999993	-4.11792181100863e-05\\
0.980000000000001	-4.11792181100839e-05\\
0.982999999999993	-4.01504451491753e-05\\
0.985999999999986	-3.91465479724907e-05\\
0.985999999999993	-3.91465479724883e-05\\
0.986000000000001	-3.91465479724859e-05\\
0.989999999999993	-3.78451864740688e-05\\
0.990000000000001	-3.78451864740665e-05\\
0.993999999999993	-3.65861076652143e-05\\
0.997999999999986	-3.53690088574584e-05\\
0.999999999999993	-3.4775602913599e-05\\
1	-3.47756029135969e-05\\
1.00799999999999	-3.24975514457578e-05\\
1.00999999999999	-3.19508152372063e-05\\
1.01	-3.19508152372025e-05\\
1.01499999999999	-3.0623662744294e-05\\
1.015	-3.06236627442903e-05\\
1.01999999999999	-2.93518723693235e-05\\
1.02	-2.93518723693199e-05\\
1.02499999999999	-2.81323272460508e-05\\
1.02999999999997	-2.69620385860408e-05\\
1.03	-2.69620385860343e-05\\
1.03499999999999	-2.58399073637635e-05\\
1.035	-2.58399073637604e-05\\
1.03999999999999	-2.47649529537662e-05\\
1.04	-2.47649529537632e-05\\
1.04399999999999	-2.39371825777321e-05\\
1.044	-2.39371825777292e-05\\
1.04799999999999	-2.31366189054387e-05\\
1.04999999999999	-2.27461456033682e-05\\
1.05	-2.27461456033655e-05\\
1.05399999999999	-2.19849465538573e-05\\
1.05799999999997	-2.12496843408677e-05\\
1.05999999999999	-2.08914176261406e-05\\
1.06	-2.08914176261381e-05\\
1.06799999999997	-1.95175300952296e-05\\
1.06999999999999	-1.91881797350223e-05\\
1.07	-1.918817973502e-05\\
1.07299999999999	-1.87046113248536e-05\\
1.073	-1.87046113248513e-05\\
1.07599999999999	-1.82335300099459e-05\\
1.07899999999997	-1.77745201964858e-05\\
1.07999999999999	-1.76241291933546e-05\\
1.08	-1.76241291933525e-05\\
1.08499999999999	-1.68911056977018e-05\\
1.085	-1.68911056976997e-05\\
1.08999999999999	-1.61883286004625e-05\\
1.09	-1.61883286004605e-05\\
1.09299999999999	-1.57807749244403e-05\\
1.093	-1.57807749244384e-05\\
1.09599999999999	-1.53837565637336e-05\\
1.09899999999997	-1.49969232861895e-05\\
1.09999999999999	-1.48701829763446e-05\\
1.1	-1.48701829763428e-05\\
1.10199999999999	-1.4619933839549e-05\\
1.102	-1.46199338395473e-05\\
1.10399999999999	-1.43739117687079e-05\\
1.10599999999997	-1.41320203057708e-05\\
1.10999999999994	-1.36602514504841e-05\\
1.10999999999999	-1.36602514504791e-05\\
1.11	-1.36602514504774e-05\\
1.11799999999994	-1.27640155396916e-05\\
1.11999999999999	-1.25494841871147e-05\\
1.12	-1.25494841871132e-05\\
1.12799999999994	-1.1726945276252e-05\\
1.12999999999999	-1.15298040666984e-05\\
1.13	-1.1529804066697e-05\\
1.13099999999999	-1.14324769036611e-05\\
1.131	-1.14324769036597e-05\\
1.13199999999999	-1.13359970733076e-05\\
1.13299999999999	-1.12403551151518e-05\\
1.13499999999997	-1.10515474010043e-05\\
1.13899999999994	-1.0683579376896e-05\\
1.13999999999999	-1.05935445346036e-05\\
1.14	-1.05935445346023e-05\\
1.14799999999994	-9.90008386585161e-06\\
1.14999999999999	-9.73385729902974e-06\\
1.15	-9.73385729902857e-06\\
1.15099999999999	-9.65178914865432e-06\\
1.151	-9.65178914865316e-06\\
1.15199999999999	-9.57043341782454e-06\\
1.15299999999999	-9.48978212907315e-06\\
1.15499999999997	-9.33056132740966e-06\\
1.155	-9.33056132740744e-06\\
1.15899999999997	-9.0202297548739e-06\\
1.15999999999999	-8.94429199936458e-06\\
1.16	-8.94429199936351e-06\\
1.16399999999997	-8.64689822972458e-06\\
1.16799999999994	-8.35932847993222e-06\\
1.16999999999999	-8.21908608014879e-06\\
1.17	-8.2190860801478e-06\\
1.17799999999994	-7.68155531810598e-06\\
1.17999999999999	-7.55283730330737e-06\\
1.18	-7.55283730330646e-06\\
1.18799999999994	-7.05910955566715e-06\\
1.18899999999999	-6.99967230567464e-06\\
1.189	-6.9996723056738e-06\\
1.18999999999999	-6.94072173435739e-06\\
1.19	-6.94072173435655e-06\\
1.19099999999999	-6.88226694406534e-06\\
1.19199999999999	-6.82431708040442e-06\\
1.19399999999997	-6.70990946314751e-06\\
1.19799999999994	-6.48690668950198e-06\\
1.19999999999999	-6.37822409794813e-06\\
1.2	-6.37822409794737e-06\\
1.20799999999994	-5.96129973912617e-06\\
1.20999999999999	-5.86131582659923e-06\\
1.21	-5.86131582659853e-06\\
1.21799999999994	-5.47802118299907e-06\\
1.21799999999997	-5.47802118299778e-06\\
1.218	-5.4780211829965e-06\\
1.21999999999999	-5.38622169400624e-06\\
1.22	-5.3862216940056e-06\\
1.22199999999999	-5.29595896804445e-06\\
1.22399999999997	-5.20719761178701e-06\\
1.22499999999999	-5.16336904982906e-06\\
1.225	-5.16336904982844e-06\\
1.22899999999997	-4.99163575668567e-06\\
1.22999999999999	-4.94957663239371e-06\\
1.23	-4.94957663239311e-06\\
1.23399999999997	-4.78489230637588e-06\\
1.23799999999995	-4.62577061555543e-06\\
1.23799999999997	-4.62577061555438e-06\\
1.238	-4.62577061555334e-06\\
1.23999999999999	-4.54821743614228e-06\\
1.24	-4.54821743614173e-06\\
1.24199999999999	-4.47196199539983e-06\\
1.24399999999997	-4.39697439246177e-06\\
1.24699999999999	-4.28680600665764e-06\\
1.247	-4.28680600665712e-06\\
1.24999999999999	-4.17932703682658e-06\\
1.25	-4.17932703682608e-06\\
1.25299999999999	-4.07452898997656e-06\\
1.25599999999997	-3.9724057379344e-06\\
1.25999999999999	-3.84024681488454e-06\\
1.26	-3.84024681488407e-06\\
1.26599999999997	-3.65017309488042e-06\\
1.26999999999999	-3.52861056446903e-06\\
1.27	-3.5286105644686e-06\\
1.27599999999997	-3.35381326662095e-06\\
1.276	-3.35381326662018e-06\\
1.28	-3.24217652117616e-06\\
1.28000000000001	-3.24217652117577e-06\\
1.28400000000002	-3.13425045416107e-06\\
1.28800000000002	-3.02986579801481e-06\\
1.29	-2.97895012584763e-06\\
1.29000000000001	-2.97895012584728e-06\\
1.29499999999999	-2.85539628586095e-06\\
1.295	-2.85539628586061e-06\\
1.29599999999999	-2.83131820547539e-06\\
1.296	-2.83131820547505e-06\\
1.29699999999999	-2.80744546673632e-06\\
1.29799999999999	-2.78377572965809e-06\\
1.29999999999997	-2.73703600063049e-06\\
1.3	-2.73703600062983e-06\\
1.30399999999997	-2.64589179300956e-06\\
1.30499999999999	-2.62357919660366e-06\\
1.305	-2.62357919660335e-06\\
1.30899999999997	-2.53615702186708e-06\\
1.30999999999999	-2.51474782049202e-06\\
1.31	-2.51474782049172e-06\\
1.31399999999997	-2.43092538025907e-06\\
1.31799999999994	-2.34994474738521e-06\\
1.31999999999999	-2.31048024809993e-06\\
1.32	-2.31048024809965e-06\\
1.32799999999994	-2.15910301963108e-06\\
1.32999999999999	-2.12280471831712e-06\\
1.33	-2.12280471831687e-06\\
1.33399999999999	-2.05203965944856e-06\\
1.334	-2.05203965944831e-06\\
1.33799999999999	-1.98367576781429e-06\\
1.33999999999999	-1.95036060996255e-06\\
1.34	-1.95036060996232e-06\\
1.34399999999999	-1.88539855919184e-06\\
1.34799999999997	-1.82257631940081e-06\\
1.35	-1.79193670998032e-06\\
1.35000000000001	-1.79193670998011e-06\\
1.354	-1.73220457150146e-06\\
1.35400000000001	-1.73220457150125e-06\\
1.358	-1.67450022506174e-06\\
1.35999999999999	-1.64638007083684e-06\\
1.36	-1.64638007083664e-06\\
1.36299999999999	-1.60508378908813e-06\\
1.363	-1.60508378908793e-06\\
1.36499999999999	-1.57812649371562e-06\\
1.365	-1.57812649371543e-06\\
1.36699999999999	-1.55161568949674e-06\\
1.36899999999997	-1.52554098282752e-06\\
1.36999999999999	-1.51266396488851e-06\\
1.37	-1.51266396488833e-06\\
1.37399999999997	-1.46224911178351e-06\\
1.37799999999995	-1.41354595085914e-06\\
1.37999999999999	-1.38981228194375e-06\\
1.38	-1.38981228194358e-06\\
1.38799999999994	-1.29878171355054e-06\\
1.38999999999999	-1.27695548636117e-06\\
1.39	-1.27695548636102e-06\\
1.39199999999999	-1.25549575148692e-06\\
1.392	-1.25549575148677e-06\\
1.39399999999998	-1.23440535266576e-06\\
1.39599999999997	-1.21367602004296e-06\\
1.39999999999994	-1.17326818392361e-06\\
1.39999999999999	-1.17326818392313e-06\\
1.4	-1.17326818392299e-06\\
1.40799999999994	-1.09643681678172e-06\\
1.41	-1.07801492397606e-06\\
1.41000000000001	-1.07801492397593e-06\\
1.41200000000001	-1.05990226635675e-06\\
1.41200000000003	-1.05990226635663e-06\\
1.41400000000003	-1.04210122593865e-06\\
1.41600000000003	-1.02460482267274e-06\\
1.41999999999999	-9.90498606398203e-07\\
1.42	-9.90498606398084e-07\\
1.42099999999999	-9.82151873289738e-07\\
1.421	-9.8215187328962e-07\\
1.42199999999999	-9.73875421974345e-07\\
1.42299999999999	-9.65668441242658e-07\\
1.42499999999997	-9.49459680782197e-07\\
1.42899999999994	-9.17840861158975e-07\\
1.42999999999999	-9.10098023260279e-07\\
1.43	-9.10098023260169e-07\\
1.43499999999999	-8.72367259820766e-07\\
1.435	-8.72367259820661e-07\\
1.43999999999998	-8.36225015465839e-07\\
1.44	-8.3622501546572e-07\\
1.44499999999999	-8.01582713850155e-07\\
1.44999999999997	-7.68355455687567e-07\\
1.44999999999998	-7.6835545568747e-07\\
1.45	-7.68355455687373e-07\\
1.45999999999997	-7.05993422014959e-07\\
1.45999999999999	-7.05993422014856e-07\\
1.46	-7.05993422014771e-07\\
1.46999999999997	-6.48697016418089e-07\\
1.46999999999998	-6.48697016418003e-07\\
1.47	-6.48697016417917e-07\\
1.47899999999998	-6.01112519549728e-07\\
1.479	-6.01112519549656e-07\\
1.47999999999999	-5.9604819425395e-07\\
1.48	-5.96048194253878e-07\\
1.48099999999999	-5.91026586032959e-07\\
1.48199999999999	-5.86047202371303e-07\\
1.48399999999997	-5.76213160690856e-07\\
1.48799999999994	-5.57030570086589e-07\\
1.49	-5.47674500032706e-07\\
1.49000000000001	-5.47674500032641e-07\\
1.49799999999996	-5.11811586780121e-07\\
1.49900000000001	-5.07499306501934e-07\\
1.49900000000003	-5.07499306501873e-07\\
1.49999999999999	-5.03223503581482e-07\\
1.5	-5.03223503581421e-07\\
1.50099999999999	-4.98983759137381e-07\\
1.50199999999999	-4.94779657435136e-07\\
1.50399999999997	-4.86476737526247e-07\\
1.50499999999999	-4.82377105531803e-07\\
1.505	-4.82377105531745e-07\\
1.50799999999998	-4.70280709691595e-07\\
1.508	-4.70280709691539e-07\\
1.50999999999999	-4.62381252114664e-07\\
1.51	-4.62381252114608e-07\\
1.51199999999999	-4.5461415182675e-07\\
1.51399999999998	-4.46980496605778e-07\\
1.51799999999995	-4.32101601373823e-07\\
1.51999999999999	-4.2485052762974e-07\\
1.52	-4.24850527629689e-07\\
1.52799999999995	-3.97036278163996e-07\\
1.52999999999999	-3.90366596075838e-07\\
1.53	-3.90366596075791e-07\\
1.53699999999998	-3.67901300154341e-07\\
1.537	-3.67901300154297e-07\\
1.53999999999999	-3.58678686899469e-07\\
1.54	-3.58678686899426e-07\\
1.54299999999999	-3.49687442931377e-07\\
1.54599999999997	-3.40919636321779e-07\\
1.54999999999999	-3.2956345586528e-07\\
1.55	-3.29563455865241e-07\\
1.55599999999997	-3.132360262e-07\\
1.55699999999998	-3.10595875481158e-07\\
1.557	-3.10595875481121e-07\\
1.55999999999999	-3.02809221447742e-07\\
1.56	-3.02809221447706e-07\\
1.56299999999999	-2.95217944971359e-07\\
1.56599999999998	-2.8781534928794e-07\\
1.566	-2.87815349287884e-07\\
1.56999999999999	-2.78227461571781e-07\\
1.57	-2.78227461571747e-07\\
1.57399999999999	-2.68958350615977e-07\\
1.57499999999999	-2.66690725288649e-07\\
1.575	-2.66690725288617e-07\\
1.57899999999999	-2.57812165579742e-07\\
1.57999999999999	-2.55639423307626e-07\\
1.58	-2.55639423307595e-07\\
1.58399999999999	-2.47129655801081e-07\\
1.58799999999998	-2.38899864641678e-07\\
1.58999999999999	-2.34885909832709e-07\\
1.59	-2.3488590983268e-07\\
1.59499999999998	-2.25145975673529e-07\\
1.595	-2.25145975673502e-07\\
1.59999999999998	-2.15815938359409e-07\\
1.6	-2.15815938359376e-07\\
1.60499999999998	-2.06872932110389e-07\\
1.60999999999997	-1.98295039940271e-07\\
1.60999999999998	-1.98295039940245e-07\\
1.61	-1.9829503994022e-07\\
1.61499999999998	-1.90072312831009e-07\\
1.615	-1.90072312830986e-07\\
1.61999999999998	-1.82195674957295e-07\\
1.62	-1.82195674957267e-07\\
1.624	-1.76130548918672e-07\\
1.62400000000001	-1.7613054891865e-07\\
1.62800000000001	-1.70265037423448e-07\\
1.62999999999999	-1.67404250529958e-07\\
1.63	-1.67404250529938e-07\\
1.634	-1.61826979512615e-07\\
1.638	-1.56438836596244e-07\\
1.63999999999999	-1.53813036000193e-07\\
1.64	-1.53813036000175e-07\\
1.64499999999998	-1.47439418189617e-07\\
1.645	-1.47439418189599e-07\\
1.64999999999998	-1.41326074765387e-07\\
1.65	-1.41326074765368e-07\\
1.65299999999998	-1.37779575105541e-07\\
1.653	-1.37779575105524e-07\\
1.65599999999998	-1.34323906131562e-07\\
1.65899999999997	-1.30956019404134e-07\\
1.65999999999999	-1.29852381958795e-07\\
1.66	-1.29852381958779e-07\\
1.66599999999997	-1.23422443326495e-07\\
1.67	-1.19310911551141e-07\\
1.67000000000002	-1.19310911551127e-07\\
1.673	-1.16316963685919e-07\\
1.67300000000001	-1.16316963685905e-07\\
1.676	-1.13399692615786e-07\\
1.67899999999998	-1.10556524824044e-07\\
1.67999999999998	-1.09624834266694e-07\\
1.68	-1.0962483426668e-07\\
1.68199999999998	-1.07784952156713e-07\\
1.682	-1.077849521567e-07\\
1.68399999999998	-1.05975802261697e-07\\
1.68599999999997	-1.04196675266974e-07\\
1.68999999999994	-1.00725711439961e-07\\
1.68999999999998	-1.00725711439922e-07\\
1.69	-1.0072571143991e-07\\
1.69799999999994	-9.41284725520489e-08\\
1.69999999999998	-9.25486786618495e-08\\
1.7	-9.25486786618384e-08\\
1.70799999999994	-8.64889798603042e-08\\
1.70999999999998	-8.50359546360451e-08\\
1.71	-8.50359546360348e-08\\
1.711	-8.43185188015419e-08\\
1.71100000000001	-8.43185188015318e-08\\
1.71200000000001	-8.36072925139183e-08\\
1.71300000000001	-8.29022060327357e-08\\
1.715	-8.15101766614325e-08\\
1.71500000000001	-8.15101766614227e-08\\
1.71700000000001	-8.0141884940795e-08\\
1.71900000000001	-7.87967944304307e-08\\
1.71999999999999	-7.81327844818945e-08\\
1.72	-7.81327844818851e-08\\
1.724	-7.55321345757059e-08\\
1.72799999999999	-7.30170688653024e-08\\
1.72999999999999	-7.17903922718982e-08\\
1.73	-7.17903922718896e-08\\
1.731	-7.11847161837202e-08\\
1.73100000000001	-7.11847161837116e-08\\
1.73200000000001	-7.05842820355433e-08\\
1.73300000000001	-6.99890309508308e-08\\
1.735	-6.88138451108618e-08\\
1.73899999999999	-6.65231365337793e-08\\
1.73999999999998	-6.59625610588673e-08\\
1.74	-6.59625610588593e-08\\
1.74799999999998	-6.16437214631233e-08\\
1.74999999999998	-6.06081211027463e-08\\
1.75	-6.0608121102739e-08\\
1.75799999999998	-5.66386191163846e-08\\
1.75999999999998	-5.56880631940834e-08\\
1.76	-5.56880631940767e-08\\
1.76799999999998	-5.20419343353018e-08\\
1.76899999999998	-5.16029902846923e-08\\
1.769	-5.16029902846861e-08\\
1.76999999999998	-5.11676389615332e-08\\
1.77	-5.11676389615271e-08\\
1.77099999999998	-5.07359517695034e-08\\
1.77199999999997	-5.03080004319789e-08\\
1.77399999999994	-4.94631379077416e-08\\
1.77799999999989	-4.78164216208364e-08\\
1.77999999999998	-4.70139222118604e-08\\
1.78	-4.70139222118547e-08\\
1.78499999999998	-4.50659934491632e-08\\
1.785	-4.50659934491578e-08\\
1.78999999999998	-4.31975838182109e-08\\
1.79	-4.31975838182057e-08\\
1.79499999999998	-4.14065230918965e-08\\
1.798	-4.0368336550432e-08\\
1.79800000000001	-4.03683365504272e-08\\
1.79999999999999	-3.96908306507905e-08\\
1.8	-3.96908306507858e-08\\
1.80199999999998	-3.90247089767893e-08\\
1.80399999999995	-3.83697103689099e-08\\
1.8079999999999	-3.70920594396854e-08\\
1.80999999999999	-3.64689061756224e-08\\
1.81	-3.6468906175618e-08\\
1.8179999999999	-3.4080328940984e-08\\
1.818	-3.40803289409557e-08\\
1.81800000000001	-3.40803289409516e-08\\
1.81999999999998	-3.35083484174997e-08\\
1.82	-3.35083484174956e-08\\
1.82199999999997	-3.29459791288564e-08\\
1.82399999999994	-3.23930005618548e-08\\
1.82699999999998	-3.15806670584833e-08\\
1.827	-3.15806670584795e-08\\
1.82999999999998	-3.07882591299664e-08\\
1.83	-3.07882591299627e-08\\
1.83299999999999	-3.00156959651519e-08\\
1.83599999999997	-2.92629142577932e-08\\
1.83999999999998	-2.82888311334205e-08\\
1.84	-2.8288831133417e-08\\
1.84599999999997	-2.68881068500505e-08\\
1.84999999999998	-2.59924217546587e-08\\
1.85	-2.59924217546556e-08\\
1.85499999999998	-2.49146880676918e-08\\
1.855	-2.49146880676888e-08\\
1.85599999999998	-2.4704666343567e-08\\
1.856	-2.4704666343564e-08\\
1.85699999999998	-2.44964378451197e-08\\
1.85799999999997	-2.42899821619672e-08\\
1.85999999999994	-2.38823084740925e-08\\
1.85999999999999	-2.38823084740829e-08\\
1.86	-2.388230847408e-08\\
1.86399999999994	-2.30873559494791e-08\\
1.86799999999988	-2.23185619991498e-08\\
1.86999999999999	-2.19435961207827e-08\\
1.87	-2.19435961207801e-08\\
1.876	-2.08564283110494e-08\\
1.87600000000001	-2.08564283110469e-08\\
1.87999999999999	-2.0162165981429e-08\\
1.88	-2.01621659814266e-08\\
1.88399999999998	-1.94910407780969e-08\\
1.88499999999998	-1.9326750530886e-08\\
1.885	-1.93267505308836e-08\\
1.88899999999997	-1.86830706542753e-08\\
1.89	-1.85254421035873e-08\\
1.89000000000001	-1.85254421035851e-08\\
1.89399999999999	-1.79082910584822e-08\\
1.89799999999996	-1.73120642448229e-08\\
1.89999999999999	-1.70215039475907e-08\\
1.9	-1.70215039475887e-08\\
1.90799999999995	-1.59069809316604e-08\\
1.90999999999999	-1.56397335198152e-08\\
1.91	-1.56397335198133e-08\\
1.914	-1.51187170011752e-08\\
1.91400000000001	-1.51187170011734e-08\\
1.91800000000001	-1.46153654942793e-08\\
1.91999999999999	-1.43700663638507e-08\\
1.92	-1.4370066363849e-08\\
1.924	-1.38917404648783e-08\\
1.92499999999998	-1.37746472188813e-08\\
1.925	-1.37746472188796e-08\\
1.92899999999999	-1.33158828323277e-08\\
1.92999999999999	-1.32035376801399e-08\\
1.93	-1.32035376801383e-08\\
1.934	-1.27636814088443e-08\\
1.93400000000001	-1.27636814088427e-08\\
1.93800000000001	-1.23387384155624e-08\\
1.93999999999999	-1.21316502431209e-08\\
1.94	-1.21316502431195e-08\\
1.94299999999998	-1.18275182977718e-08\\
1.943	-1.18275182977704e-08\\
1.94599999999998	-1.15309487545113e-08\\
1.94899999999996	-1.12416799939151e-08\\
1.94999999999999	-1.11468351821317e-08\\
1.95	-1.11468351821304e-08\\
1.95599999999997	-1.05945842456311e-08\\
1.95999999999998	-1.02419186767822e-08\\
1.96	-1.0241918676781e-08\\
1.96599999999996	-9.73478995943231e-09\\
1.97	-9.41051062892252e-09\\
1.97000000000001	-9.41051062892139e-09\\
1.97199999999998	-9.25240449106222e-09\\
1.972	-9.25240449106111e-09\\
1.97399999999997	-9.0970162275374e-09\\
1.97599999999994	-8.94428491515557e-09\\
1.97999999999988	-8.64655464802799e-09\\
1.98	-8.6465546480193e-09\\
1.98000000000002	-8.64655464801826e-09\\
1.9879999999999	-8.08040981111903e-09\\
1.99	-7.94465597519255e-09\\
1.99000000000002	-7.9446559751916e-09\\
1.99199999999998	-7.81117813049015e-09\\
1.992	-7.81117813048921e-09\\
1.99399999999997	-7.67999477897195e-09\\
1.995	-7.61524742837693e-09\\
1.99500000000001	-7.61524742837601e-09\\
1.99699999999998	-7.48740964737727e-09\\
1.99899999999995	-7.36173941741294e-09\\
1.99999999999999	-7.29970170455632e-09\\
2	-7.29970170455544e-09\\
2.00099999999997	-7.23818746885773e-09\\
2.001	-7.23818746885599e-09\\
2.00199999999997	-7.17719068150393e-09\\
2.00299999999995	-7.11670536394806e-09\\
2.00499999999989	-6.99724547501246e-09\\
2.00899999999979	-6.76420578193909e-09\\
2.00999999999997	-6.70713749567627e-09\\
2.01	-6.70713749567465e-09\\
2.01799999999979	-6.26784243609973e-09\\
2.01999999999997	-6.16264698220194e-09\\
2.02	-6.16264698220046e-09\\
2.02799999999979	-5.75914076666231e-09\\
2.02999999999997	-5.66238536401211e-09\\
2.03	-5.66238536401075e-09\\
2.03799999999979	-5.2915181987276e-09\\
2.03999999999997	-5.20270875243848e-09\\
2.04	-5.20270875243723e-09\\
2.04799999999979	-4.86205548788048e-09\\
2.05	-4.78037135584427e-09\\
2.05000000000003	-4.78037135584312e-09\\
2.05799999999981	-4.46727289198506e-09\\
2.05899999999997	-4.42962561834725e-09\\
2.059	-4.42962561834619e-09\\
2.05999999999997	-4.39229697761058e-09\\
2.06	-4.39229697760952e-09\\
2.06099999999998	-4.35528331285286e-09\\
2.06199999999995	-4.31858099464291e-09\\
2.0639999999999	-4.24609603919023e-09\\
2.06499999999997	-4.2103062975319e-09\\
2.065	-4.21030629753089e-09\\
2.0689999999999	-4.07008406148995e-09\\
2.06999999999997	-4.03574550597972e-09\\
2.07	-4.03574550597875e-09\\
2.0739999999999	-3.90130229423547e-09\\
2.0779999999998	-3.77141717299101e-09\\
2.07899999999997	-3.73963406855528e-09\\
2.079	-3.73963406855438e-09\\
2.07999999999997	-3.70811996473959e-09\\
2.08	-3.7081199647387e-09\\
2.08099999999998	-3.67687177424599e-09\\
2.08199999999995	-3.64588643298565e-09\\
2.08399999999991	-3.58469217608363e-09\\
2.08799999999981	-3.46532591974908e-09\\
2.088	-3.46532591974355e-09\\
2.08800000000003	-3.46532591974272e-09\\
2.08999999999997	-3.40710711906654e-09\\
2.09	-3.40710711906572e-09\\
2.09199999999995	-3.34986436741638e-09\\
2.0939999999999	-3.29360560336496e-09\\
2.09799999999979	-3.18395218991917e-09\\
2.09999999999997	-3.13051454749265e-09\\
2.1	-3.1305145474919e-09\\
2.10799999999979	-2.92553963329846e-09\\
2.10999999999997	-2.87638936530826e-09\\
2.11	-2.87638936530756e-09\\
2.11699999999997	-2.71084122853655e-09\\
2.117	-2.7108412285359e-09\\
2.11999999999997	-2.64288066085341e-09\\
2.12	-2.64288066085277e-09\\
2.12299999999998	-2.57662592839169e-09\\
2.12599999999995	-2.51201858231086e-09\\
2.13	-2.42833975048969e-09\\
2.13000000000003	-2.42833975048911e-09\\
2.13499999999997	-2.32765326384858e-09\\
2.135	-2.32765326384802e-09\\
2.137	-2.28857856085498e-09\\
2.13700000000002	-2.28857856085443e-09\\
2.13900000000002	-2.25016638196336e-09\\
2.14	-2.23120402283894e-09\\
2.14000000000003	-2.2312040228384e-09\\
2.14200000000003	-2.19375747251854e-09\\
2.14400000000002	-2.15693627124857e-09\\
2.14599999999997	-2.12072598313648e-09\\
2.146	-2.12072598313597e-09\\
2.14999999999999	-2.05008159566304e-09\\
2.15000000000003	-2.0500815956624e-09\\
2.15400000000002	-1.98178664874272e-09\\
2.15800000000002	-1.91580716588603e-09\\
2.15999999999997	-1.88365324679536e-09\\
2.16	-1.88365324679491e-09\\
2.16799999999999	-1.76031802689795e-09\\
2.16999999999997	-1.73074389239667e-09\\
2.17	-1.73074389239625e-09\\
2.17499999999997	-1.65898183031058e-09\\
2.175	-1.65898183031018e-09\\
2.17999999999997	-1.59023979053652e-09\\
2.18	-1.59023979053606e-09\\
2.18499999999997	-1.52434930211317e-09\\
2.18999999999994	-1.46114888456256e-09\\
2.18999999999997	-1.46114888456219e-09\\
2.19	-1.46114888456182e-09\\
2.19499999999997	-1.4005650889729e-09\\
2.195	-1.40056508897256e-09\\
2.19999999999996	-1.34253089954285e-09\\
2.2	-1.34253089954247e-09\\
2.20399999999997	-1.29784348199091e-09\\
2.204	-1.2978434819906e-09\\
2.20499999999997	-1.28690408242578e-09\\
2.205	-1.28690408242547e-09\\
2.20599999999997	-1.27605550691793e-09\\
2.20699999999995	-1.26529669215351e-09\\
2.20899999999989	-1.2440441356452e-09\\
2.20999999999997	-1.23354831088973e-09\\
2.21	-1.23354831088943e-09\\
2.21399999999989	-1.19245479764171e-09\\
2.21799999999979	-1.15275451918504e-09\\
2.21999999999997	-1.13340730156993e-09\\
2.22	-1.13340730156965e-09\\
2.22799999999979	-1.05919572869513e-09\\
2.22999999999997	-1.04140079054987e-09\\
2.23	-1.04140079054962e-09\\
2.23299999999997	-1.01526910386878e-09\\
2.233	-1.01526910386853e-09\\
2.23599999999997	-9.89806551713638e-10\\
2.23899999999994	-9.64990670704878e-10\\
2.24	-9.56858615180182e-10\\
2.24000000000003	-9.56858615179952e-10\\
2.24599999999997	-9.09479840111344e-10\\
2.25	-8.79183832895614e-10\\
2.25000000000003	-8.79183832895403e-10\\
2.25299999999997	-8.57122642802336e-10\\
2.253	-8.5712264280213e-10\\
2.25599999999994	-8.35626355366197e-10\\
2.25899999999988	-8.14676007298208e-10\\
2.26	-8.07810669816583e-10\\
2.26000000000003	-8.07810669816389e-10\\
2.26199999999997	-7.94253116853531e-10\\
2.262	-7.9425311685334e-10\\
2.26399999999994	-7.80921973532945e-10\\
2.26599999999988	-7.67812013104202e-10\\
2.26999999999977	-7.42235166518943e-10\\
2.27	-7.42235166517494e-10\\
2.27000000000003	-7.42235166517315e-10\\
2.27499999999997	-7.11459853571599e-10\\
2.275	-7.11459853571428e-10\\
2.27999999999994	-6.81979680510342e-10\\
2.28	-6.81979680510013e-10\\
2.28499999999995	-6.53722398195919e-10\\
2.28999999999989	-6.26618755366946e-10\\
2.29	-6.26618755366368e-10\\
2.29000000000003	-6.26618755366217e-10\\
2.29099999999997	-6.21331966632248e-10\\
2.291	-6.21331966632098e-10\\
2.29199999999997	-6.1609093398736e-10\\
2.29299999999994	-6.10895143241247e-10\\
2.29499999999989	-6.00637254863038e-10\\
2.29899999999979	-5.80642265249891e-10\\
2.29999999999997	-5.7574915088708e-10\\
2.3	-5.75749150886941e-10\\
2.30799999999979	-5.38051158993202e-10\\
2.31	-5.29011684608115e-10\\
2.31000000000003	-5.29011684607988e-10\\
2.31099999999999	-5.24548408170377e-10\\
2.31100000000002	-5.24548408170251e-10\\
2.31199999999999	-5.2012376022836e-10\\
2.31299999999996	-5.15737306916155e-10\\
2.31499999999989	-5.07077268209963e-10\\
2.31899999999976	-4.90196855620298e-10\\
2.31999999999997	-4.86065931192779e-10\\
2.32	-4.86065931192662e-10\\
2.32799999999974	-4.54240073416098e-10\\
2.32999999999997	-4.46608657907617e-10\\
2.33	-4.46608657907509e-10\\
2.33799999999974	-4.17357156437085e-10\\
2.33999999999997	-4.10352468159431e-10\\
2.34	-4.10352468159332e-10\\
2.345	-3.93349840247225e-10\\
2.34500000000003	-3.93349840247131e-10\\
2.34899999999997	-3.80249469645932e-10\\
2.349	-3.8024946964584e-10\\
2.34999999999997	-3.7704136446504e-10\\
2.35	-3.77041364464949e-10\\
2.35099999999997	-3.73860262028317e-10\\
2.35199999999994	-3.70706691050423e-10\\
2.35399999999988	-3.64480909807908e-10\\
2.35799999999976	-3.5234630227334e-10\\
2.35999999999997	-3.464327182308e-10\\
2.36	-3.46432718230716e-10\\
2.36799999999976	-3.23749533652638e-10\\
2.36999999999997	-3.18310408917856e-10\\
2.37	-3.1831040891778e-10\\
2.37799999999976	-2.97462040261316e-10\\
2.378	-2.97462040260732e-10\\
2.37800000000002	-2.97462040260661e-10\\
2.37999999999997	-2.92469601797779e-10\\
2.38	-2.92469601797709e-10\\
2.38199999999995	-2.87561059709314e-10\\
2.38399999999989	-2.82734489284681e-10\\
2.38799999999979	-2.7331972577108e-10\\
2.38999999999997	-2.68727841331465e-10\\
2.39	-2.687278413314e-10\\
2.39799999999979	-2.51126976812513e-10\\
2.398	-2.51126976812071e-10\\
2.39800000000002	-2.5112697681201e-10\\
2.39999999999997	-2.46912199514164e-10\\
2.4	-2.46912199514105e-10\\
2.40199999999994	-2.42768250278816e-10\\
2.40399999999988	-2.38693504203335e-10\\
2.40699999999997	-2.32707653723075e-10\\
2.407	-2.32707653723019e-10\\
2.40999999999997	-2.26868641743776e-10\\
2.41	-2.26868641743721e-10\\
2.41299999999997	-2.2117586934921e-10\\
2.41499999999997	-2.17461975551994e-10\\
2.415	-2.17461975551942e-10\\
2.41799999999997	-2.12009426343423e-10\\
2.41999999999997	-2.08451176224186e-10\\
2.42	-2.08451176224136e-10\\
2.42299999999997	-2.03225487419266e-10\\
2.42599999999994	-1.98129733328019e-10\\
2.42999999999997	-1.91529761647093e-10\\
2.42999999999999	-1.91529761647047e-10\\
2.43599999999994	-1.82040783276192e-10\\
2.43599999999997	-1.82040783276144e-10\\
2.436	-1.82040783276098e-10\\
2.43999999999997	-1.75981147096403e-10\\
2.44	-1.75981147096361e-10\\
2.44399999999998	-1.70123454746378e-10\\
2.44799999999996	-1.64458519221283e-10\\
2.44999999999997	-1.61695547221186e-10\\
2.45	-1.61695547221147e-10\\
2.45599999999999	-1.53684648158516e-10\\
2.45600000000002	-1.53684648158479e-10\\
2.45999999999997	-1.48568909606648e-10\\
2.46	-1.48568909606612e-10\\
2.46399999999995	-1.43623658550312e-10\\
2.46499999999997	-1.42413069849782e-10\\
2.465	-1.42413069849748e-10\\
2.46899999999995	-1.37670051515918e-10\\
2.46999999999997	-1.36508549964847e-10\\
2.47	-1.36508549964814e-10\\
2.47399999999995	-1.31961009909585e-10\\
2.4779999999999	-1.27567649097385e-10\\
2.47999999999997	-1.25426623389579e-10\\
2.48	-1.25426623389549e-10\\
2.48499999999997	-1.20229666858741e-10\\
2.485	-1.20229666858712e-10\\
2.48999999999997	-1.15244882777139e-10\\
2.49	-1.15244882777109e-10\\
2.49399999999997	-1.11405704374608e-10\\
2.494	-1.11405704374581e-10\\
2.49799999999996	-1.07696689012177e-10\\
2.49999999999997	-1.05889166908551e-10\\
2.5	-1.05889166908526e-10\\
2.50399999999997	-1.02364550392215e-10\\
2.50799999999994	-9.89559171402942e-11\\
2.50999999999997	-9.72934166501973e-11\\
2.51	-9.72934166501739e-11\\
2.51399999999997	-9.40522596062534e-11\\
2.514	-9.40522596062307e-11\\
2.51799999999996	-9.09209904855813e-11\\
2.51999999999997	-8.93950226438499e-11\\
2.52	-8.93950226438284e-11\\
2.523	-8.71539680151707e-11\\
2.52300000000002	-8.71539680151497e-11\\
2.52600000000002	-8.496863639077e-11\\
2.52900000000001	-8.28370999759311e-11\\
2.52999999999997	-8.21382159854936e-11\\
2.53	-8.21382159854738e-11\\
2.53599999999999	-7.80688352701193e-11\\
2.53999999999997	-7.54701396619871e-11\\
2.54	-7.54701396619689e-11\\
2.54599999999999	-7.17332429944116e-11\\
2.54999999999997	-6.93437116456493e-11\\
2.55	-6.93437116456326e-11\\
2.55199999999997	-6.81786671831394e-11\\
2.552	-6.8178667183123e-11\\
2.55399999999996	-6.70336497334305e-11\\
2.55499999999997	-6.64685105450542e-11\\
2.555	-6.64685105450382e-11\\
2.55699999999997	-6.5352694288635e-11\\
2.55899999999994	-6.42557970965652e-11\\
2.55999999999997	-6.37143084750512e-11\\
2.56	-6.37143084750358e-11\\
2.56399999999994	-6.15935206818848e-11\\
2.56799999999987	-5.95425208312899e-11\\
2.56999999999997	-5.85421819671317e-11\\
2.57	-5.85421819671176e-11\\
2.57199999999997	-5.75586141181745e-11\\
2.572	-5.75586141181607e-11\\
2.57399999999996	-5.65919537533894e-11\\
2.57599999999993	-5.56418218310819e-11\\
2.57999999999986	-5.3789659646388e-11\\
2.57999999999997	-5.37896596463375e-11\\
2.58	-5.37896596463245e-11\\
2.58099999999997	-5.33363749589487e-11\\
2.581	-5.33363749589359e-11\\
2.58199999999996	-5.28869032210876e-11\\
2.58299999999993	-5.24412003591238e-11\\
2.58499999999987	-5.15609268938935e-11\\
2.58899999999974	-4.98437086248207e-11\\
2.58999999999997	-4.94231848872425e-11\\
2.59	-4.94231848872306e-11\\
2.59799999999975	-4.61861155931048e-11\\
2.59999999999997	-4.54109533416286e-11\\
2.6	-4.54109533416177e-11\\
2.60799999999975	-4.24376046971412e-11\\
2.60999999999997	-4.17246354590366e-11\\
2.61	-4.17246354590266e-11\\
2.61799999999974	-3.89917978391211e-11\\
2.61999999999997	-3.83373810272982e-11\\
2.62	-3.8337381027289e-11\\
2.62499999999997	-3.67489013493981e-11\\
2.625	-3.67489013493893e-11\\
2.62999999999998	-3.52252733311733e-11\\
2.63000000000001	-3.52252733311648e-11\\
2.63499999999998	-3.37647263982788e-11\\
2.63899999999997	-3.26407104420996e-11\\
2.639	-3.26407104420918e-11\\
2.63999999999997	-3.23656446136411e-11\\
2.64	-3.23656446136333e-11\\
2.64099999999998	-3.20928997872129e-11\\
2.64199999999996	-3.18224492286012e-11\\
2.64399999999991	-3.12883251091983e-11\\
2.64799999999982	-3.02464565255819e-11\\
2.64999999999997	-2.97383035644194e-11\\
2.65	-2.97383035644122e-11\\
2.65799999999982	-2.77905343580327e-11\\
2.65899999999997	-2.75563327403653e-11\\
2.659	-2.75563327403587e-11\\
2.66	-2.73241133451885e-11\\
2.66000000000003	-2.73241133451819e-11\\
2.66100000000003	-2.70938534230022e-11\\
2.66200000000004	-2.68655303953196e-11\\
2.66400000000005	-2.64146056989637e-11\\
2.66799999999997	-2.55350268576424e-11\\
2.668	-2.55350268576363e-11\\
2.67	-2.51060278715391e-11\\
2.67000000000003	-2.5106027871533e-11\\
2.67200000000003	-2.46842211959289e-11\\
2.67400000000004	-2.42696653307534e-11\\
2.67800000000005	-2.34616587021671e-11\\
2.67999999999997	-2.30678911346804e-11\\
2.68	-2.30678911346749e-11\\
2.68800000000002	-2.15574870140616e-11\\
2.68999999999997	-2.11953123082948e-11\\
2.69	-2.11953123082896e-11\\
2.69499999999997	-2.03164901140331e-11\\
2.695	-2.03164901140283e-11\\
2.69699999999997	-1.99754342890612e-11\\
2.697	-1.99754342890564e-11\\
2.69899999999996	-1.96401611896043e-11\\
2.69999999999997	-1.94746519973704e-11\\
2.7	-1.94746519973657e-11\\
2.70199999999997	-1.91478072183097e-11\\
2.70399999999993	-1.88264206896965e-11\\
2.70799999999987	-1.81995204744218e-11\\
2.71	-1.78937609920568e-11\\
2.71000000000003	-1.78937609920525e-11\\
2.71699999999999	-1.68639009344399e-11\\
2.71700000000002	-1.68639009344359e-11\\
2.72	-1.64411244794381e-11\\
2.72000000000003	-1.64411244794341e-11\\
2.72300000000001	-1.60289599435184e-11\\
2.72599999999999	-1.56270437218736e-11\\
2.72600000000002	-1.56270437218699e-11\\
2.73	-1.5106485703964e-11\\
2.73000000000003	-1.51064857039604e-11\\
2.73400000000002	-1.46032400575701e-11\\
2.738	-1.41170563805699e-11\\
2.74	-1.38801234693045e-11\\
2.74000000000003	-1.38801234693011e-11\\
2.748	-1.29713019420419e-11\\
2.75	-1.27533786979597e-11\\
2.75000000000003	-1.27533786979567e-11\\
2.75499999999997	-1.22245847774068e-11\\
2.755	-1.22245847774039e-11\\
2.75999999999993	-1.17180444590995e-11\\
2.75999999999997	-1.17180444590961e-11\\
2.76	-1.17180444590926e-11\\
2.76499999999994	-1.1232516328979e-11\\
2.76499999999997	-1.1232516328976e-11\\
2.765	-1.12325163289731e-11\\
2.76999999999994	-1.07668104836721e-11\\
2.76999999999997	-1.07668104836692e-11\\
2.77	-1.07668104836664e-11\\
2.77499999999994	-1.03203857493651e-11\\
2.775	-1.03203857493602e-11\\
2.77999999999993	-9.8927482052384e-12\\
2.77999999999997	-9.89274820523547e-12\\
2.78	-9.89274820523257e-12\\
2.78399999999997	-9.5634591903642e-12\\
2.784	-9.5634591903619e-12\\
2.78799999999996	-9.2450059616068e-12\\
2.78999999999997	-9.08968604644116e-12\\
2.79	-9.08968604643898e-12\\
2.79399999999997	-8.78687936778501e-12\\
2.79799999999993	-8.4943390037188e-12\\
2.79999999999997	-8.3517746942569e-12\\
2.8	-8.35177469425489e-12\\
2.80799999999993	-7.80492996523655e-12\\
2.80999999999997	-7.67380391464512e-12\\
2.81	-7.67380391464327e-12\\
2.81299999999997	-7.481246617235e-12\\
2.813	-7.48124661723319e-12\\
2.81599999999996	-7.29361997272925e-12\\
2.81899999999993	-7.1107584545731e-12\\
2.81999999999997	-7.05083553181084e-12\\
2.82	-7.05083553180914e-12\\
2.82599999999993	-6.7017140793621e-12\\
2.83	-6.47847090626359e-12\\
2.83000000000003	-6.47847090626203e-12\\
2.83299999999999	-6.31590787980596e-12\\
2.83300000000002	-6.31590787980444e-12\\
2.835	-6.20985378851212e-12\\
2.83500000000003	-6.20985378851063e-12\\
2.83700000000001	-6.10560808987649e-12\\
2.83899999999999	-6.0031299142037e-12\\
2.84	-5.95254106543608e-12\\
2.84000000000003	-5.95254106543464e-12\\
2.84199999999997	-5.85263905487972e-12\\
2.842	-5.85263905487832e-12\\
2.84399999999993	-5.75440539073998e-12\\
2.84599999999987	-5.65780156012048e-12\\
2.84999999999974	-5.46933252897814e-12\\
2.85	-5.46933252896612e-12\\
2.85000000000003	-5.4693325289648e-12\\
2.85799999999978	-5.11110773673359e-12\\
2.85999999999997	-5.0253257194054e-12\\
2.86	-5.02532571940419e-12\\
2.86799999999975	-4.69628513175226e-12\\
2.86999999999997	-4.61738560030165e-12\\
2.87	-4.61738560030053e-12\\
2.87099999999997	-4.57842864316575e-12\\
2.871	-4.57842864316465e-12\\
2.87199999999996	-4.53980884876566e-12\\
2.87299999999993	-4.50152243017593e-12\\
2.87499999999986	-4.42593474286191e-12\\
2.87899999999973	-4.2785969135963e-12\\
2.87999999999997	-4.24254083875994e-12\\
2.88	-4.24254083875892e-12\\
2.88799999999974	-3.96475424075987e-12\\
2.88999999999997	-3.89814473036139e-12\\
2.89	-3.89814473036046e-12\\
2.89099999999997	-3.86525601994511e-12\\
2.89099999999999	-3.86525601994418e-12\\
2.89199999999996	-3.83265195310576e-12\\
2.89299999999993	-3.80032933279964e-12\\
2.89499999999986	-3.73651578724671e-12\\
2.89899999999973	-3.61212847480816e-12\\
2.89999999999997	-3.58168878211108e-12\\
2.9	-3.58168878211022e-12\\
2.90499999999997	-3.43328427141137e-12\\
2.905	-3.43328427141054e-12\\
2.90999999999998	-3.29093856112438e-12\\
2.91000000000001	-3.29093856112359e-12\\
2.91499999999999	-3.15448623421408e-12\\
2.91999999999997	-3.02377632106414e-12\\
2.92	-3.02377632106326e-12\\
2.92899999999997	-2.80195534926816e-12\\
2.92899999999999	-2.80195534926749e-12\\
2.92999999999997	-2.778315677716e-12\\
2.93	-2.77831567771533e-12\\
2.93099999999998	-2.75487498408676e-12\\
2.93199999999996	-2.73163716274709e-12\\
2.93399999999992	-2.68576104640216e-12\\
2.93799999999983	-2.59634437396695e-12\\
2.93999999999997	-2.55276875928941e-12\\
2.94	-2.5527687592888e-12\\
2.94799999999983	-2.38562247029478e-12\\
2.94999999999997	-2.34554302606487e-12\\
2.95	-2.34554302606431e-12\\
2.95799999999983	-2.19191702647578e-12\\
2.95799999999997	-2.19191702647326e-12\\
2.95799999999999	-2.19191702647273e-12\\
2.95999999999997	-2.15512909768254e-12\\
2.96	-2.15512909768203e-12\\
2.96199999999998	-2.11895937954043e-12\\
2.96399999999996	-2.08339368937501e-12\\
2.96799999999992	-2.01401885003045e-12\\
2.96999999999997	-1.98018250029204e-12\\
2.97	-1.98018250029157e-12\\
2.97499999999997	-1.89807810543891e-12\\
2.975	-1.89807810543846e-12\\
2.97799999999997	-1.85048651150019e-12\\
2.97799999999999	-1.85048651149974e-12\\
2.97999999999997	-1.8194289664213e-12\\
2.98	-1.81942896642086e-12\\
2.98199999999998	-1.78889333264634e-12\\
2.98399999999996	-1.75886763863053e-12\\
2.98699999999999	-1.71475953775343e-12\\
2.98700000000002	-1.71475953775302e-12\\
2.99	-1.67173345097212e-12\\
2.99000000000003	-1.67173345097172e-12\\
2.99300000000001	-1.62978496457475e-12\\
2.99599999999999	-1.5889106153921e-12\\
2.99999999999997	-1.53602017403538e-12\\
3	-1.53602017403501e-12\\
3.00599999999996	-1.45996428026922e-12\\
3.00999999999997	-1.41133089229158e-12\\
3.01	-1.41133089229124e-12\\
3.01599999999996	-1.34140920359236e-12\\
3.01599999999999	-1.34140920359196e-12\\
3.01600000000002	-1.34140920359164e-12\\
3.01999999999997	-1.29675740008657e-12\\
3.02	-1.29675740008626e-12\\
3.02399999999995	-1.25359366521829e-12\\
3.0279999999999	-1.21185030245838e-12\\
3.03	-1.19149071626197e-12\\
3.03000000000003	-1.19149071626168e-12\\
3.03600000000002	-1.13246058773013e-12\\
3.03600000000005	-1.13246058772986e-12\\
3.03999999999997	-1.09476410590214e-12\\
3.04	-1.09476410590188e-12\\
3.04399999999993	-1.0583238992227e-12\\
3.04499999999997	-1.04940340105033e-12\\
3.04499999999999	-1.04940340105008e-12\\
3.04899999999992	-1.0144533912607e-12\\
3.04999999999997	-1.00589460268304e-12\\
3.05	-1.0058946026828e-12\\
3.05399999999993	-9.72385018150591e-13\\
3.05799999999985	-9.40011537745488e-13\\
3.05999999999997	-9.24234901584549e-13\\
3.06	-9.24234901584326e-13\\
3.06799999999985	-8.63719262134748e-13\\
3.06999999999997	-8.49208421299411e-13\\
3.07	-8.49208421299207e-13\\
3.07399999999997	-8.20918556112193e-13\\
3.07399999999999	-8.20918556111995e-13\\
3.07799999999996	-7.93587827700889e-13\\
3.08	-7.80268687235237e-13\\
3.08000000000003	-7.8026868723505e-13\\
3.08399999999999	-7.54296742957097e-13\\
3.08799999999996	-7.29179447261439e-13\\
3.09	-7.16928930938055e-13\\
3.09000000000003	-7.16928930937882e-13\\
3.09399999999997	-6.93045721199628e-13\\
3.09399999999999	-6.93045721199461e-13\\
3.09799999999993	-6.69972245445211e-13\\
3.09999999999997	-6.58727800737794e-13\\
3.1	-6.58727800737636e-13\\
3.10299999999997	-6.4221407451344e-13\\
3.10299999999999	-6.42214074513286e-13\\
3.10599999999996	-6.26110955809025e-13\\
3.10899999999992	-6.10404239225784e-13\\
3.10999999999997	-6.05254351656362e-13\\
3.11	-6.05254351656217e-13\\
3.11499999999997	-5.80158664331374e-13\\
3.115	-5.80158664331235e-13\\
3.11999999999998	-5.56119095550072e-13\\
3.12	-5.56119095549938e-13\\
3.12499999999998	-5.33076730527318e-13\\
3.12999999999995	-5.10975098409589e-13\\
3.13	-5.10975098409365e-13\\
3.13000000000003	-5.10975098409241e-13\\
3.13199999999997	-5.02390198468968e-13\\
3.13199999999999	-5.02390198468847e-13\\
3.13399999999993	-4.93952873261101e-13\\
3.13599999999986	-4.85659813672119e-13\\
3.13999999999973	-4.69493541419547e-13\\
3.13999999999997	-4.69493541418566e-13\\
3.14	-4.69493541418453e-13\\
3.14799999999974	-4.38752761490381e-13\\
3.14999999999997	-4.313815336812e-13\\
3.15	-4.31381533681096e-13\\
3.15199999999997	-4.24133886964634e-13\\
3.15199999999999	-4.24133886964531e-13\\
3.15399999999996	-4.17010826957679e-13\\
3.15599999999992	-4.10009560604129e-13\\
3.15999999999985	-3.96361476110761e-13\\
3.15999999999997	-3.96361476110335e-13\\
3.16	-3.96361476110239e-13\\
3.16099999999997	-3.93021343290523e-13\\
3.16099999999999	-3.93021343290429e-13\\
3.16199999999996	-3.89709307071888e-13\\
3.16299999999993	-3.86425042687728e-13\\
3.16499999999986	-3.7993854454466e-13\\
3.16899999999973	-3.67284825531157e-13\\
3.16999999999997	-3.64186099354391e-13\\
3.17	-3.64186099354304e-13\\
3.17799999999974	-3.40333007624824e-13\\
3.17999999999997	-3.34621045832994e-13\\
3.18	-3.34621045832914e-13\\
3.18499999999997	-3.20756281072412e-13\\
3.185	-3.20756281072335e-13\\
3.18999999999998	-3.07457562758315e-13\\
3.19	-3.07457562758241e-13\\
3.19499999999998	-2.94709436714089e-13\\
3.19999999999995	-2.82497798520488e-13\\
3.2	-2.82497798520359e-13\\
3.20999999999995	-2.59565516971698e-13\\
3.21	-2.59565516971576e-13\\
3.21899999999997	-2.40520570733581e-13\\
3.21899999999999	-2.40520570733523e-13\\
3.21999999999997	-2.38493684801275e-13\\
3.22	-2.38493684801218e-13\\
3.22099999999998	-2.36483901831054e-13\\
3.22199999999996	-2.34491024704972e-13\\
3.22399999999992	-2.30555208332117e-13\\
3.22799999999984	-2.22877960191707e-13\\
3.22999999999997	-2.19133518320529e-13\\
3.23	-2.19133518320476e-13\\
3.23799999999984	-2.04780933421672e-13\\
3.23899999999997	-2.03055165004676e-13\\
3.23899999999999	-2.03055165004627e-13\\
3.23999999999997	-2.01344003051921e-13\\
3.24	-2.01344003051872e-13\\
3.24099999999998	-1.99647279928481e-13\\
3.24199999999996	-1.97964829259593e-13\\
3.24399999999992	-1.94642087077357e-13\\
3.24799999999984	-1.88160708416176e-13\\
3.24799999999997	-1.88160708415979e-13\\
3.24799999999999	-1.88160708415933e-13\\
3.24999999999997	-1.84999530711385e-13\\
3.25	-1.84999530711341e-13\\
3.25199999999998	-1.81891351393273e-13\\
3.25399999999996	-1.78836601340866e-13\\
3.25499999999998	-1.77328887222537e-13\\
3.255	-1.77328887222494e-13\\
3.25899999999996	-1.71425670150773e-13\\
3.25999999999997	-1.69981052508343e-13\\
3.26	-1.69981052508302e-13\\
3.26399999999996	-1.64323080482902e-13\\
3.26799999999992	-1.58851293119495e-13\\
3.26999999999997	-1.56182525698635e-13\\
3.27	-1.56182525698597e-13\\
3.27699999999999	-1.47193574801774e-13\\
3.27700000000002	-1.47193574801739e-13\\
3.28	-1.43503445473457e-13\\
3.28000000000003	-1.43503445473422e-13\\
3.28300000000001	-1.39905940394269e-13\\
3.28599999999999	-1.36397885904114e-13\\
3.28999999999998	-1.31854287519466e-13\\
3.29	-1.31854287519435e-13\\
3.29599999999996	-1.25321819078155e-13\\
3.29699999999999	-1.24265527414602e-13\\
3.29700000000002	-1.24265527414572e-13\\
3.29999999999997	-1.21150202160501e-13\\
3.3	-1.21150202160472e-13\\
3.30299999999995	-1.18113073209958e-13\\
3.30599999999991	-1.15151461311862e-13\\
3.30599999999995	-1.15151461311819e-13\\
3.30599999999999	-1.15151461311776e-13\\
3.31	-1.11315610111475e-13\\
3.31000000000003	-1.11315610111449e-13\\
3.31400000000004	-1.0760732908583e-13\\
3.31800000000005	-1.04024773087118e-13\\
3.31999999999997	-1.02278878552791e-13\\
3.32	-1.02278878552767e-13\\
3.32499999999998	-9.80410320991686e-14\\
3.325	-9.8041032099145e-14\\
3.32999999999998	-9.39762010521324e-14\\
3.33000000000001	-9.39762010521097e-14\\
3.33499999999998	-9.00796617532878e-14\\
3.33500000000001	-9.00796617532661e-14\\
3.33999999999998	-8.63471031054533e-14\\
3.34000000000001	-8.63471031054325e-14\\
3.34499999999998	-8.27693775765538e-14\\
3.34999999999996	-7.9337717090305e-14\\
3.35000000000001	-7.93377170902691e-14\\
3.35499999999998	-7.60481335211251e-14\\
3.35500000000001	-7.60481335211069e-14\\
3.35999999999998	-7.28969880474132e-14\\
3.36000000000001	-7.28969880473957e-14\\
3.36399999999999	-7.0470546293806e-14\\
3.36400000000002	-7.0470546293789e-14\\
3.36800000000001	-6.81239505114443e-14\\
3.36999999999998	-6.69794400279155e-14\\
3.37000000000001	-6.69794400278994e-14\\
3.37399999999999	-6.47481394369558e-14\\
3.37799999999998	-6.25924885446115e-14\\
3.37999999999997	-6.15419706722581e-14\\
3.38	-6.15419706722433e-14\\
3.38799999999998	-5.7512419635159e-14\\
3.38999999999997	-5.65461872022817e-14\\
3.39	-5.65461872022681e-14\\
3.39299999999997	-5.51272845125411e-14\\
3.39299999999999	-5.51272845125279e-14\\
3.395	-5.42016101341338e-14\\
3.39500000000003	-5.42016101341208e-14\\
3.39700000000004	-5.32917199916527e-14\\
3.39900000000005	-5.23972573611904e-14\\
3.39999999999997	-5.19557015428834e-14\\
3.4	-5.19557015428709e-14\\
3.40400000000002	-5.02263093558626e-14\\
3.40800000000004	-4.85538256023109e-14\\
3.40999999999997	-4.77381012608193e-14\\
3.41	-4.77381012608078e-14\\
3.41299999999996	-4.65402182015537e-14\\
3.41299999999999	-4.65402182015425e-14\\
3.41599999999996	-4.53730083275055e-14\\
3.41899999999992	-4.42354419116353e-14\\
3.41999999999997	-4.38626663536662e-14\\
3.42	-4.38626663536556e-14\\
3.42199999999997	-4.31265154282892e-14\\
3.42199999999999	-4.31265154282788e-14\\
3.42399999999996	-4.24026581090596e-14\\
3.42599999999992	-4.16908105927357e-14\\
3.42999999999985	-4.03020332451345e-14\\
3.42999999999997	-4.03020332450905e-14\\
3.43	-4.03020332450808e-14\\
3.43799999999985	-3.76623715822922e-14\\
3.43999999999997	-3.70302670850876e-14\\
3.44	-3.70302670850787e-14\\
3.44799999999985	-3.46056559439588e-14\\
3.44999999999997	-3.40242666265185e-14\\
3.45	-3.40242666265104e-14\\
3.45099999999999	-3.37372033411582e-14\\
3.45100000000002	-3.37372033411501e-14\\
3.45200000000001	-3.34526245168832e-14\\
3.453	-3.31705022488628e-14\\
3.45499999999999	-3.2613517015954e-14\\
3.45899999999995	-3.1527824384867e-14\\
3.46	-3.12621369173011e-14\\
3.46000000000003	-3.12621369172935e-14\\
3.46499999999998	-2.99668144247102e-14\\
3.465	-2.9966814424703e-14\\
3.46999999999995	-2.8724375099316e-14\\
3.47	-2.87243750993032e-14\\
3.47000000000003	-2.87243750992963e-14\\
3.47099999999999	-2.84820270931444e-14\\
3.47100000000002	-2.84820270931376e-14\\
3.47199999999999	-2.82417765395321e-14\\
3.47299999999996	-2.80035998893117e-14\\
3.47499999999989	-2.75333751248471e-14\\
3.47899999999976	-2.66167986487759e-14\\
3.47999999999996	-2.63924967829074e-14\\
3.47999999999999	-2.63924967829011e-14\\
3.48799999999973	-2.46644092818712e-14\\
3.48999999999997	-2.42500370147055e-14\\
3.48999999999999	-2.42500370146997e-14\\
3.49799999999973	-2.26617327084925e-14\\
3.49999999999999	-2.22813906700042e-14\\
3.50000000000002	-2.22813906699988e-14\\
3.50799999999976	-2.08224838741551e-14\\
3.50899999999999	-2.06468518904536e-14\\
3.50900000000002	-2.06468518904486e-14\\
3.50999999999999	-2.04726575448324e-14\\
3.51000000000002	-2.04726575448274e-14\\
3.51099999999999	-2.02999294097766e-14\\
3.51199999999996	-2.01286961857422e-14\\
3.51399999999991	-1.97906474850479e-14\\
3.51799999999979	-1.91317601953462e-14\\
3.51999999999997	-1.88106632690775e-14\\
3.52	-1.8810663269073e-14\\
3.52799999999977	-1.75790074425977e-14\\
3.52999999999997	-1.72836728504596e-14\\
3.53	-1.72836728504554e-14\\
3.53499999999998	-1.65670391654776e-14\\
3.535	-1.65670391654736e-14\\
3.53799999999997	-1.61516443567838e-14\\
3.53799999999999	-1.61516443567799e-14\\
3.53999999999997	-1.58805640683865e-14\\
3.54	-1.58805640683826e-14\\
3.54199999999998	-1.56140391902497e-14\\
3.54399999999996	-1.53519652308624e-14\\
3.54799999999993	-1.48407607873938e-14\\
3.54999999999997	-1.4591429869743e-14\\
3.55	-1.45914298697394e-14\\
3.55799999999993	-1.36357351933835e-14\\
3.55799999999996	-1.36357351933796e-14\\
3.55799999999999	-1.36357351933757e-14\\
3.56	-1.34068805215365e-14\\
3.56000000000003	-1.34068805215333e-14\\
3.56200000000004	-1.31818716871058e-14\\
3.56400000000005	-1.29606204608543e-14\\
3.56699999999996	-1.26355997833042e-14\\
3.56699999999999	-1.26355997833011e-14\\
3.56999999999998	-1.23185521749813e-14\\
3.57	-1.23185521749783e-14\\
3.57299999999999	-1.20094451124958e-14\\
3.57599999999997	-1.17082530767362e-14\\
3.57999999999997	-1.13185177039216e-14\\
3.58	-1.13185177039189e-14\\
3.58599999999997	-1.07580823702821e-14\\
3.58999999999997	-1.0399716072761e-14\\
3.59	-1.03997160727585e-14\\
3.59599999999997	-9.88448203782458e-15\\
3.596	-9.88448203782221e-15\\
3.59999999999997	-9.5554549600671e-15\\
3.6	-9.5554549600648e-15\\
3.60399999999998	-9.2373930579034e-15\\
3.60499999999998	-9.1595320665675e-15\\
3.605	-9.1595320665653e-15\\
3.60899999999998	-8.85447708457431e-15\\
3.60999999999998	-8.77977321150763e-15\\
3.61	-8.77977321150552e-15\\
3.61399999999998	-8.48729072669616e-15\\
3.61599999999997	-8.34479614443246e-15\\
3.616	-8.34479614443045e-15\\
3.61999999999997	-8.0670209539666e-15\\
3.62	-8.06702095396466e-15\\
3.62399999999998	-7.79850291370831e-15\\
3.62499999999996	-7.73277017216211e-15\\
3.62499999999999	-7.73277017216025e-15\\
3.62899999999997	-7.47523299019562e-15\\
3.62999999999997	-7.41216558893283e-15\\
3.63	-7.41216558893104e-15\\
3.63399999999998	-7.16524251149534e-15\\
3.63799999999996	-6.92669107835007e-15\\
3.63999999999997	-6.81043730763848e-15\\
3.64	-6.81043730763684e-15\\
3.64799999999996	-6.36451390492855e-15\\
3.65	-6.25758743787592e-15\\
3.65000000000003	-6.25758743787441e-15\\
3.65399999999999	-6.04912706171834e-15\\
3.65400000000002	-6.04912706171689e-15\\
3.65799999999998	-5.84773430538666e-15\\
3.65999999999997	-5.74958915644303e-15\\
3.66	-5.74958915644164e-15\\
3.66399999999997	-5.55820891335941e-15\\
3.66799999999993	-5.37312634482376e-15\\
3.66999999999997	-5.28285559852668e-15\\
3.67	-5.2828555985254e-15\\
3.67399999999999	-5.10686667611191e-15\\
3.67400000000002	-5.10686667611069e-15\\
3.67499999999998	-5.06381231762075e-15\\
3.67500000000001	-5.06381231761953e-15\\
3.67599999999997	-5.02112661276939e-15\\
3.67699999999994	-4.97880537775523e-15\\
3.67899999999988	-4.89523976086546e-15\\
3.67999999999997	-4.85398718853167e-15\\
3.68	-4.8539871885305e-15\\
3.68299999999996	-4.7323019999841e-15\\
3.68299999999999	-4.73230199998296e-15\\
3.68599999999995	-4.61364246739825e-15\\
3.68899999999992	-4.49790391502724e-15\\
3.69	-4.45995578443842e-15\\
3.69000000000003	-4.45995578443735e-15\\
3.69599999999995	-4.23899580476093e-15\\
3.69999999999997	-4.09789135267859e-15\\
3.7	-4.0978913526776e-15\\
3.70599999999993	-3.89498464729549e-15\\
3.70999999999997	-3.76523743084398e-15\\
3.71	-3.76523743084307e-15\\
3.71199999999999	-3.70197763004387e-15\\
3.71200000000002	-3.70197763004298e-15\\
3.71400000000001	-3.63980525962263e-15\\
3.716	-3.57869594361301e-15\\
3.71999999999998	-3.45957105028311e-15\\
3.72000000000001	-3.45957105028228e-15\\
3.72799999999997	-3.23305054724439e-15\\
3.73000000000001	-3.17873396086767e-15\\
3.73000000000004	-3.17873396086691e-15\\
3.73200000000002	-3.12532801019286e-15\\
3.73200000000005	-3.12532801019211e-15\\
3.73400000000003	-3.07284010532886e-15\\
3.73600000000002	-3.02124966499903e-15\\
3.73999999999999	-2.92068061748659e-15\\
3.74000000000002	-2.92068061748588e-15\\
3.74099999999999	-2.89606808109949e-15\\
3.74100000000002	-2.89606808109879e-15\\
3.742	-2.87166258097748e-15\\
3.74299999999997	-2.84746172400236e-15\\
3.74499999999993	-2.7996644715727e-15\\
3.745	-2.79966447157087e-15\\
3.74500000000003	-2.79966447157019e-15\\
3.74899999999994	-2.70642263469493e-15\\
3.75000000000001	-2.68358895960908e-15\\
3.75000000000004	-2.68358895960844e-15\\
3.75399999999995	-2.59418998005298e-15\\
3.75799999999986	-2.50782197068299e-15\\
3.76	-2.46573206740449e-15\\
3.76000000000003	-2.4657320674039e-15\\
3.76799999999985	-2.30428463445726e-15\\
3.76999999999996	-2.26557169972573e-15\\
3.76999999999999	-2.26557169972519e-15\\
3.77799999999981	-2.11718358190526e-15\\
3.77999999999996	-2.08164993834561e-15\\
3.77999999999999	-2.0816499383451e-15\\
3.78799999999981	-1.94535084780852e-15\\
3.78999999999996	-1.91266814916523e-15\\
3.78999999999999	-1.91266814916477e-15\\
3.79799999999981	-1.78739415041422e-15\\
3.79899999999996	-1.77233108604529e-15\\
3.79899999999999	-1.77233108604487e-15\\
3.79999999999997	-1.75739551162165e-15\\
3.8	-1.75739551162123e-15\\
3.80099999999999	-1.74258596397104e-15\\
3.80199999999997	-1.72790099092074e-15\\
3.80399999999993	-1.69889902356805e-15\\
3.80799999999986	-1.64232745653036e-15\\
3.80999999999997	-1.61473567620545e-15\\
3.81	-1.61473567620506e-15\\
3.81499999999998	-1.54778381986355e-15\\
3.81500000000001	-1.54778381986318e-15\\
3.81899999999996	-1.4962586341466e-15\\
3.81899999999999	-1.49625863414624e-15\\
3.81999999999996	-1.4836495446832e-15\\
3.81999999999999	-1.48364954468284e-15\\
3.82099999999996	-1.47114685052527e-15\\
3.82199999999993	-1.4587493262319e-15\\
3.82399999999988	-1.43426493708114e-15\\
3.82799999999976	-1.38650540940688e-15\\
3.82799999999996	-1.38650540940449e-15\\
3.82799999999999	-1.38650540940415e-15\\
3.82999999999996	-1.3632115452759e-15\\
3.82999999999999	-1.36321154527557e-15\\
3.83199999999996	-1.34030821194754e-15\\
3.83399999999993	-1.31779858444858e-15\\
3.83799999999988	-1.27392529812536e-15\\
3.83999999999996	-1.25254443741548e-15\\
3.83999999999999	-1.25254443741518e-15\\
3.84799999999988	-1.17053224902326e-15\\
3.84999999999996	-1.15086682332075e-15\\
3.84999999999999	-1.15086682332047e-15\\
3.85699999999996	-1.08462967412263e-15\\
3.85699999999999	-1.08462967412237e-15\\
3.85999999999997	-1.05743810867256e-15\\
3.86	-1.05743810867231e-15\\
3.86299999999999	-1.0309290662916e-15\\
3.86599999999997	-1.00507916113647e-15\\
3.86999999999997	-9.71598613003661e-16\\
3.87	-9.71598613003426e-16\\
3.87599999999997	-9.23462619136342e-16\\
3.87699999999996	-9.1567909143967e-16\\
3.87699999999999	-9.1567909143945e-16\\
3.87999999999996	-8.92723101879098e-16\\
3.87999999999999	-8.92723101878884e-16\\
3.88299999999996	-8.70343319737414e-16\\
3.88499999999998	-8.55733810323189e-16\\
3.88500000000001	-8.55733810322983e-16\\
3.88599999999996	-8.48520002322886e-16\\
3.88599999999999	-8.48520002322682e-16\\
3.88699999999996	-8.41365880794771e-16\\
3.88799999999993	-8.34270744552441e-16\\
3.88999999999986	-8.20254652048968e-16\\
3.88999999999996	-8.20254652048277e-16\\
3.88999999999999	-8.2025465204808e-16\\
3.89399999999986	-7.92929331691809e-16\\
3.89799999999973	-7.66530445048093e-16\\
3.89999999999996	-7.53665420200354e-16\\
3.89999999999999	-7.53665420200173e-16\\
3.90799999999973	-7.0431806898189e-16\\
3.90999999999996	-6.92485234259031e-16\\
3.90999999999999	-6.92485234258864e-16\\
3.91499999999996	-6.63772688641427e-16\\
3.91499999999999	-6.63772688641268e-16\\
3.91999999999997	-6.36268472895701e-16\\
3.92	-6.36268472895503e-16\\
3.92499999999998	-6.09905180502007e-16\\
3.92999999999995	-5.84618201876229e-16\\
3.93	-5.84618201875969e-16\\
3.93499999999996	-5.60378149612068e-16\\
3.93499999999999	-5.60378149611933e-16\\
3.93999999999995	-5.37158210229685e-16\\
3.93999999999999	-5.37158210229515e-16\\
3.94399999999996	-5.19278416684128e-16\\
3.94399999999999	-5.19278416684003e-16\\
3.94799999999997	-5.01986986502629e-16\\
3.94999999999996	-4.93553398031523e-16\\
3.94999999999999	-4.93553398031404e-16\\
3.95399999999996	-4.77111546931145e-16\\
3.95499999999998	-4.73089172258844e-16\\
3.95500000000001	-4.7308917225873e-16\\
3.95899999999998	-4.57340198022226e-16\\
3.95999999999999	-4.53486156111381e-16\\
3.96000000000002	-4.53486156111272e-16\\
3.96399999999999	-4.38391462091876e-16\\
3.96799999999997	-4.23793483269411e-16\\
3.97000000000002	-4.16673577535707e-16\\
3.97000000000005	-4.16673577535606e-16\\
3.97299999999996	-4.06218066978328e-16\\
3.97299999999999	-4.0621806697823e-16\\
3.97599999999991	-3.96030281891238e-16\\
3.97899999999983	-3.86101234483971e-16\\
3.97999999999997	-3.82847528909042e-16\\
3.98	-3.82847528908949e-16\\
3.98599999999984	-3.63890869853576e-16\\
3.98999999999997	-3.51769172113384e-16\\
3.99	-3.517691721133e-16\\
3.99299999999996	-3.42942295825928e-16\\
3.99299999999999	-3.42942295825845e-16\\
3.99599999999995	-3.34341441549184e-16\\
3.99899999999991	-3.25959021903905e-16\\
3.99999999999997	-3.23212139508281e-16\\
4	-3.23212139508203e-16\\
4.00199999999993	-3.17787642180681e-16\\
4.00199999999999	-3.17787642180529e-16\\
4.00399999999992	-3.12453733113924e-16\\
4.00599999999986	-3.07208321038727e-16\\
4.00999999999972	-2.96974795871342e-16\\
4.00999999999995	-2.96974795870776e-16\\
4.01	-2.96974795870633e-16\\
4.01799999999973	-2.77523842018259e-16\\
4.01999999999995	-2.72866034695425e-16\\
4.02	-2.72866034695294e-16\\
4.02500000000001	-2.61560041197087e-16\\
4.02500000000006	-2.61560041196961e-16\\
4.02999999999995	-2.50715629080336e-16\\
4.03	-2.50715629080215e-16\\
4.03099999999999	-2.48600337330702e-16\\
4.03100000000005	-2.48600337330582e-16\\
4.03200000000004	-2.46503352825188e-16\\
4.03300000000003	-2.4442447001921e-16\\
4.035	-2.40320196251141e-16\\
4.03899999999996	-2.32320020440631e-16\\
4.03999999999994	-2.30362241239899e-16\\
4.04	-2.30362241239788e-16\\
4.04799999999991	-2.15278934853653e-16\\
4.04999999999994	-2.1166215974488e-16\\
4.05	-2.11662159744778e-16\\
4.05099999999999	-2.09876362664372e-16\\
4.05100000000005	-2.09876362664271e-16\\
4.05200000000004	-2.0810602121176e-16\\
4.05300000000002	-2.0635096179344e-16\\
4.055	-2.02886002484005e-16\\
4.05899999999996	-1.96131997947154e-16\\
4.05999999999994	-1.94479178144627e-16\\
4.06	-1.94479178144534e-16\\
};
\addplot [color=mycolor2,solid,forget plot]
  table[row sep=crcr]{%
0	0.15313\\
3.15544362088405e-30	0.15313\\
0.000656101980281985	0.153131614989962\\
0.00393661188169191	0.153188143215565\\
0.00999999999999994	0.153505321610126\\
0.01	0.153505321610126\\
0.0199999999999999	0.154633126647887\\
0.02	0.154633126647887\\
0.0289999999999998	0.153421137821655\\
0.029	0.153421137821655\\
0.03	0.152969434437754\\
0.0300000000000002	0.152969434437754\\
0.0349999999999996	0.14975837658894\\
0.035	0.14975837658894\\
0.0399999999999993	0.144956580293631\\
0.04	0.14495658029363\\
0.0449999999999993	0.138999841807976\\
0.0499999999999987	0.132322542411715\\
0.05	0.132322542411713\\
0.0500000000000004	0.132322542411713\\
0.0579999999999996	0.120119344103204\\
0.058	0.120119344103203\\
0.0599999999999996	0.116772645889942\\
0.06	0.116772645889941\\
0.0619999999999995	0.113306311255493\\
0.0639999999999991	0.10971966074666\\
0.0679999999999982	0.102182576405751\\
0.0699999999999991	0.0982306652303514\\
0.07	0.0982306652303496\\
0.0779999999999982	0.0811822039884975\\
0.0779999999999991	0.0811822039884955\\
0.078	0.0811822039884935\\
0.0799999999999991	0.0766747750397316\\
0.08	0.0766747750397296\\
0.0819999999999991	0.0721779853201548\\
0.0839999999999982	0.0676909533788196\\
0.0869999999999991	0.0609767779339659\\
0.087	0.060976777933964\\
0.0899999999999991	0.0542796180244169\\
0.09	0.0542796180244149\\
0.0929999999999991	0.0475965200953159\\
0.0959999999999982	0.0409245367909504\\
0.0999999999999991	0.0320407632769474\\
0.1	0.0320407632769454\\
0.104999999999999	0.021256909365745\\
0.105	0.0212569093657432\\
0.109999999999999	0.0110937472400841\\
0.11	0.0110937472400824\\
0.114999999999999	0.00153882575052224\\
0.116	-0.000300212751249341\\
0.116000000000001	-0.000300212751250965\\
0.12	-0.00741956108284555\\
0.120000000000001	-0.00741956108284709\\
0.124	-0.0141641650924184\\
0.127999999999999	-0.0205393128972452\\
0.129999999999999	-0.0235899139914064\\
0.130000000000001	-0.0235899139914091\\
0.135999999999999	-0.0322009479817859\\
0.136000000000001	-0.0322009479817883\\
0.139999999999998	-0.0374617592732997\\
0.14	-0.0374617592733019\\
0.143999999999997	-0.0423017423766883\\
0.144999999999998	-0.0434464356597455\\
0.145	-0.0434464356597475\\
0.148999999999997	-0.0477656815918623\\
0.149999999999998	-0.0487808758306323\\
0.15	-0.0487808758306341\\
0.153999999999997	-0.0525846762078988\\
0.157999999999995	-0.0559795055490512\\
0.159999999999998	-0.0575244003628621\\
0.16	-0.0575244003628634\\
0.167999999999995	-0.0626813071469279\\
0.17	-0.0637153351896217\\
0.170000000000002	-0.0637153351896226\\
0.173999999999998	-0.0654786415475603\\
0.174	-0.065478641547561\\
0.174999999999998	-0.0658561064227037\\
0.175	-0.0658561064227043\\
0.176	-0.0662082654683606\\
0.177	-0.0665351359408366\\
0.179000000000001	-0.0671130739947607\\
0.179999999999998	-0.0673641698955957\\
0.18	-0.0673641698955961\\
0.184000000000001	-0.0681163386563733\\
0.188000000000002	-0.068465361489994\\
0.189999999999998	-0.0684887910453806\\
0.19	-0.0684887910453806\\
0.193999999999998	-0.0682335093779742\\
0.194	-0.068233509377974\\
0.197999999999998	-0.0676618905053519\\
0.199999999999998	-0.067289840431232\\
0.2	-0.0672898404312317\\
0.202999999999998	-0.0666237633867744\\
0.203	-0.066623763386774\\
0.205999999999998	-0.065827818675382\\
0.208999999999997	-0.0649016552724822\\
0.209999999999998	-0.0645639277109202\\
0.21	-0.0645639277109196\\
0.215999999999997	-0.0622317990176866\\
0.22	-0.0603844073330849\\
0.220000000000002	-0.0603844073330841\\
0.225999999999998	-0.0573535881896798\\
0.23	-0.0552388367632122\\
0.230000000000002	-0.0552388367632112\\
0.231999999999998	-0.054152583272455\\
0.232	-0.054152583272454\\
0.233999999999996	-0.0530467953200896\\
0.235999999999993	-0.0519212561636086\\
0.239999999999986	-0.0496100378866324\\
0.24	-0.0496100378866242\\
0.240000000000002	-0.0496100378866231\\
0.244999999999998	-0.0466059013476809\\
0.245	-0.0466059013476798\\
0.249999999999997	-0.0434704183131218\\
0.249999999999999	-0.0434704183131201\\
0.250000000000002	-0.0434704183131185\\
0.251999999999996	-0.0421785931221946\\
0.252	-0.0421785931221923\\
0.253999999999995	-0.0408776958240218\\
0.255999999999989	-0.0395802807330569\\
0.259999999999979	-0.0369948806153816\\
0.259999999999995	-0.0369948806153714\\
0.26	-0.036994880615368\\
0.260999999999997	-0.0363503103724911\\
0.261	-0.0363503103724888\\
0.262	-0.035706388834962\\
0.263	-0.0350630844434952\\
0.265	-0.033778201039262\\
0.269	-0.0312144548498167\\
0.269999999999996	-0.0305745893556221\\
0.27	-0.0305745893556199\\
0.278	-0.025466199863606\\
0.279999999999996	-0.0241908815627149\\
0.28	-0.0241908815627126\\
0.288	-0.0193397003059672\\
0.289999999999996	-0.018203890310217\\
0.29	-0.018203890310215\\
0.298	-0.0139619118989523\\
0.299999999999996	-0.0129756933729801\\
0.3	-0.0129756933729784\\
0.308	-0.00927737335173871\\
0.309999999999996	-0.00841072954443441\\
0.31	-0.00841072954443289\\
0.314999999999997	-0.0063436318507457\\
0.315	-0.00634363185074428\\
0.319	-0.00479098079271896\\
0.319000000000004	-0.00479098079271763\\
0.319999999999996	-0.00441668878355274\\
0.32	-0.00441668878355142\\
0.321	-0.0040479079892336\\
0.322	-0.00368462033907317\\
0.324	-0.0029744535348474\\
0.328	-0.00161927233280755\\
0.329999999999996	-0.000973992304732705\\
0.33	-0.000973992304731578\\
0.338	0.00139462636162467\\
0.339000000000004	0.00166703773321886\\
0.339000000000007	0.00166703773321982\\
0.339999999999996	0.00193413670471284\\
0.34	0.00193413670471378\\
0.340999999999996	0.00219583707151154\\
0.341999999999993	0.00245215166026715\\
0.343999999999986	0.00294867347822746\\
0.347999999999972	0.00387762278717002\\
0.348	0.00387762278717632\\
0.348000000000004	0.00387762278717711\\
0.349999999999996	0.00431023236252524\\
0.35	0.00431023236252599\\
0.351999999999993	0.00472171298242351\\
0.353999999999986	0.00511214530601302\\
0.357999999999972	0.00583016705766123\\
0.359999999999997	0.00615789722608327\\
0.36	0.00615789722608384\\
0.367999999999972	0.00728068066986741\\
0.369999999999997	0.00751580136352198\\
0.37	0.00751580136352238\\
0.376999999999997	0.0081962003306901\\
0.377	0.00819620033069039\\
0.379999999999997	0.00842021403749647\\
0.38	0.00842021403749671\\
0.382999999999997	0.0086038275331808\\
0.384999999999997	0.00870383269055749\\
0.385	0.00870383269055765\\
0.387999999999997	0.00882028367112367\\
0.39	0.00887556867862432\\
0.390000000000004	0.00887556867862441\\
0.393	0.00892499981108655\\
0.395999999999997	0.00893425335379649\\
0.397	0.00892841073051424\\
0.397000000000004	0.00892841073051421\\
0.399999999999997	0.00889231189782712\\
0.4	0.00889231189782707\\
0.402999999999993	0.00883244997578861\\
0.405999999999986	0.00874879856446565\\
0.405999999999997	0.00874879856446531\\
0.406	0.00874879856446519\\
0.409999999999997	0.00860019220063898\\
0.41	0.00860019220063883\\
0.413999999999997	0.00840911010059374\\
0.417999999999993	0.00817540244614365\\
0.419999999999997	0.0080425080817735\\
0.42	0.00804250808177325\\
0.427999999999993	0.00746024944809923\\
0.429999999999997	0.00730542647395913\\
0.43	0.00730542647395886\\
0.434999999999997	0.0069018380545573\\
0.435	0.00690183805455701\\
0.439999999999997	0.00647423490447306\\
0.44	0.00647423490447275\\
0.444999999999997	0.00602209315428838\\
0.449999999999993	0.00554485887381604\\
0.45	0.00554485887381536\\
0.450000000000004	0.00554485887381501\\
0.454999999999997	0.00504194738033945\\
0.455	0.00504194738033908\\
0.459999999999993	0.00452878709595633\\
0.46	0.00452878709595561\\
0.463999999999997	0.00412200909747846\\
0.464	0.0041220090974781\\
0.467999999999997	0.00371821895045717\\
0.469999999999997	0.00351734530130976\\
0.47	0.0035173453013094\\
0.473999999999997	0.00311744398515352\\
0.477999999999993	0.00271974288983303\\
0.479999999999997	0.00252161991827851\\
0.48	0.00252161991827816\\
0.487999999999993	0.00176906322921124\\
0.489999999999997	0.00159297228841402\\
0.49	0.00159297228841371\\
0.492999999999997	0.00133772944864621\\
0.493	0.00133772944864591\\
0.495999999999997	0.0010930591872577\\
0.498999999999993	0.000858853596700756\\
0.499999999999997	0.000783092564210614\\
0.5	0.000783092564210347\\
0.505999999999993	0.00035249817731223\\
0.509999999999993	8.80107191641623e-05\\
0.51	8.80107191637084e-05\\
0.512999999999993	-9.86520603868523e-05\\
0.513	-9.86520603872826e-05\\
0.515999999999993	-0.000275630403669611\\
0.518999999999986	-0.000443257033022569\\
0.519999999999993	-0.000497067313238426\\
0.52	-0.000497067313238805\\
0.521999999999993	-0.000601605875257211\\
0.522	-0.000601605875257575\\
0.523999999999993	-0.000702052096524967\\
0.524999999999993	-0.000750746763239926\\
0.525	-0.000750746763240269\\
0.526999999999993	-0.000845091139182644\\
0.528999999999986	-0.000935390898807118\\
0.529999999999993	-0.000979029609364891\\
0.53	-0.000979029609365198\\
0.533999999999986	-0.00114355930990597\\
0.537999999999972	-0.00129214376675162\\
0.539999999999993	-0.00136049342133831\\
0.54	-0.00136049342133855\\
0.547999999999972	-0.00159126458725587\\
0.549999999999993	-0.00163814999004629\\
0.55	-0.00163814999004645\\
0.550999999999993	-0.00165997840960948\\
0.551	-0.00165997840960963\\
0.551999999999993	-0.00168073208478697\\
0.552999999999987	-0.0017004120328283\\
0.554999999999973	-0.00173655455222289\\
0.558999999999945	-0.00179599377605356\\
0.559999999999993	-0.00180818169465182\\
0.56	-0.0018081816946519\\
0.567999999999945	-0.0018672948741246\\
0.569999999999993	-0.00187142203084885\\
0.57	-0.00187142203084886\\
0.570999999999993	-0.00187188899666083\\
0.571	-0.00187188899666083\\
0.571999999999994	-0.00187154864393448\\
0.572999999999987	-0.00187065799803123\\
0.574999999999973	-0.00186722562516012\\
0.578999999999945	-0.00185375363405945\\
0.579999999999993	-0.00184900820348367\\
0.58	-0.00184900820348364\\
0.587999999999945	-0.00179116923333827\\
0.589999999999993	-0.00177117527363048\\
0.59	-0.0017711752736304\\
0.594999999999993	-0.00171146441039013\\
0.595	-0.00171146441039004\\
0.599999999999993	-0.00163779874635414\\
0.6	-0.00163779874635403\\
0.604999999999993	-0.00155482170220468\\
0.608999999999993	-0.00148507599380066\\
0.609	-0.00148507599380054\\
0.61	-0.00146716529106077\\
0.610000000000007	-0.00146716529106064\\
0.611000000000007	-0.00144906313683881\\
0.612000000000007	-0.00143076864409049\\
0.614000000000006	-0.00139359904781055\\
0.618000000000005	-0.00131691171295292\\
0.619999999999993	-0.00127737894288338\\
0.62	-0.00127737894288324\\
0.627999999999998	-0.00111886322918495\\
0.629999999999993	-0.00107958202193794\\
0.63	-0.0010795820219378\\
0.637999999999998	-0.000923617064250694\\
0.638000000000005	-0.000923617064250556\\
0.639999999999993	-0.000884877580222641\\
0.64	-0.000884877580222504\\
0.641999999999989	-0.000846223584755479\\
0.643999999999977	-0.000807647500977641\\
0.647999999999954	-0.000730698838134866\\
0.649999999999993	-0.000692311176328506\\
0.65	-0.000692311176328369\\
0.657999999999954	-0.000539162865472987\\
0.657999999999992	-0.000539162865472252\\
0.658000000000005	-0.000539162865472008\\
0.659999999999993	-0.000501276878966948\\
0.66	-0.000501276878966814\\
0.661999999999989	-0.000464077253026777\\
0.663999999999977	-0.000427556695814266\\
0.664999999999993	-0.000409548825882352\\
0.665	-0.000409548825882225\\
0.666999999999993	-0.000374033491422775\\
0.667	-0.00037403349142265\\
0.668999999999993	-0.000339179576849402\\
0.669999999999993	-0.000321998512145828\\
0.67	-0.000321998512145706\\
0.671999999999993	-0.000288123958488455\\
0.673999999999986	-0.000254893986149715\\
0.677999999999971	-0.000190341857767758\\
0.679999999999993	-0.000159007048816407\\
0.68	-0.000159007048816297\\
0.687999999999971	-4.29823066736993e-05\\
0.689999999999993	-1.64751036363088e-05\\
0.69	-1.64751036362164e-05\\
0.695999999999993	5.71328625549299e-05\\
0.696	5.71328625550118e-05\\
0.699999999999993	0.000101324416735105\\
0.7	0.000101324416735181\\
0.703999999999993	0.000141652500755774\\
0.707999999999986	0.000178148733990258\\
0.709999999999993	0.00019496896574381\\
0.71	0.000194968965743868\\
0.716	0.000239757126250045\\
0.716000000000007	0.000239757126250093\\
0.719999999999993	0.000265227313110937\\
0.72	0.000265227313110979\\
0.723999999999986	0.000287581986330444\\
0.724999999999993	0.000292685963081671\\
0.725000000000001	0.000292685963081707\\
0.728999999999986	0.000311170607035509\\
0.729999999999993	0.000315310084454789\\
0.73	0.000315310084454818\\
0.733999999999986	0.000329947345143448\\
0.734999999999993	0.000333127394608294\\
0.735	0.000333127394608316\\
0.738999999999986	0.000343935204506619\\
0.739999999999993	0.000346159721975314\\
0.74	0.000346159721975329\\
0.743999999999986	0.000353525091717178\\
0.747999999999972	0.000358591717285046\\
0.749999999999993	0.000360264312770394\\
0.75	0.000360264312770399\\
0.753999999999993	0.000361889708221085\\
0.754	0.000361889708221086\\
0.757999999999993	0.000361222917688959\\
0.759999999999993	0.000360029859545878\\
0.76	0.000360029859545872\\
0.763999999999993	0.000355923248241808\\
0.767999999999986	0.000349519773041307\\
0.769999999999993	0.000345455212940812\\
0.77	0.000345455212940796\\
0.773999999999993	0.000335596464305072\\
0.774000000000001	0.000335596464305053\\
0.777999999999994	0.000324108495114803\\
0.779999999999993	0.000318007109925459\\
0.78	0.000318007109925437\\
0.782999999999993	0.000308404934350245\\
0.783	0.000308404934350221\\
0.785999999999993	0.000298258859566388\\
0.788999999999986	0.000287564410989037\\
0.79	0.000283876925761341\\
0.790000000000007	0.000283876925761315\\
0.795999999999993	0.000260451182375555\\
0.8	0.000243580934784975\\
0.800000000000007	0.000243580934784944\\
0.804999999999993	0.000221998327731054\\
0.805	0.000221998327731024\\
0.809999999999987	0.000200675108573026\\
0.809999999999997	0.000200675108572982\\
0.810000000000007	0.000200675108572938\\
0.811999999999993	0.000192212587868439\\
0.812	0.000192212587868409\\
0.813999999999987	0.000183785727583908\\
0.815999999999973	0.000175392876010764\\
0.819999999999945	0.000158702625238849\\
0.819999999999987	0.000158702625238675\\
0.82	0.000158702625238617\\
0.827999999999944	0.000125658254151261\\
0.829999999999993	0.000117457735527058\\
0.830000000000001	0.000117457735527029\\
0.831999999999994	0.000109278225116423\\
0.832000000000001	0.000109278225116394\\
0.833999999999994	0.000101171711119848\\
0.835999999999987	9.31901958076729e-05\\
0.839999999999973	7.75959278198192e-05\\
0.839999999999987	7.75959278197659e-05\\
0.84	7.75959278197126e-05\\
0.840999999999993	7.37730698007684e-05\\
0.841000000000001	7.37730698007414e-05\\
0.841999999999994	6.99801185746754e-05\\
0.842999999999987	6.62168882426209e-05\\
0.844999999999973	5.87788541116846e-05\\
0.848999999999945	4.42514194889586e-05\\
0.849999999999993	4.06911526263444e-05\\
0.85	4.06911526263192e-05\\
0.857999999999944	1.3212939114302e-05\\
0.859999999999993	6.61609394092825e-06\\
0.86	6.616093940905e-06\\
0.867999999999944	-1.75889111928683e-05\\
0.869999999999993	-2.30298701204867e-05\\
0.87	-2.30298701205056e-05\\
0.874999999999993	-3.55761162380169e-05\\
0.875000000000001	-3.55761162380336e-05\\
0.879999999999994	-4.66256500617711e-05\\
0.880000000000001	-4.66256500617857e-05\\
0.884999999999994	-5.64335494543234e-05\\
0.889999999999987	-6.52533710112184e-05\\
0.890000000000001	-6.52533710112419e-05\\
0.898999999999994	-7.86728198029097e-05\\
0.899000000000001	-7.8672819802919e-05\\
0.899999999999993	-7.9970804899513e-05\\
0.9	-7.99708048995221e-05\\
0.900999999999993	-8.12304240605533e-05\\
0.901999999999986	-8.24517390330102e-05\\
0.903999999999972	-8.4779693916931e-05\\
0.907999999999944	-8.89784611874238e-05\\
0.909999999999993	-9.08500965806667e-05\\
0.91	-9.08500965806731e-05\\
0.917999999999944	-9.68265304824112e-05\\
0.918999999999994	-9.74043092332666e-05\\
0.919000000000001	-9.74043092332706e-05\\
0.919999999999993	-9.79487953765544e-05\\
0.92	-9.79487953765582e-05\\
0.920999999999993	-9.84642346442333e-05\\
0.921999999999986	-9.89506523009617e-05\\
0.923999999999972	-9.98365166989302e-05\\
0.927999999999944	-0.000101260939743647\\
0.927999999999987	-0.000101260939743659\\
0.928000000000001	-0.000101260939743663\\
0.929999999999993	-0.000101799777595601\\
0.93	-0.000101799777595602\\
0.931999999999993	-0.000102223181343474\\
0.933999999999986	-0.00010253123399108\\
0.937999999999972	-0.000102801504905289\\
0.939999999999993	-0.000102763776151242\\
0.940000000000001	-0.000102763776151242\\
0.945000000000001	-0.000102297117257328\\
0.945000000000008	-0.000102297117257327\\
0.949999999999993	-0.00010137340727486\\
0.950000000000001	-0.000101373407274859\\
0.954999999999986	-9.9991514549674e-05\\
0.956999999999994	-9.93101059645939e-05\\
0.957000000000001	-9.93101059645914e-05\\
0.96	-9.81497460884982e-05\\
0.960000000000008	-9.81497460884953e-05\\
0.963000000000007	-9.68230295148333e-05\\
0.966000000000007	-9.53293711394087e-05\\
0.97	-9.30769902074456e-05\\
0.970000000000008	-9.30769902074413e-05\\
0.976000000000007	-8.91358071094371e-05\\
0.976999999999994	-8.84129010303857e-05\\
0.977000000000001	-8.84129010303805e-05\\
0.979999999999993	-8.61850033043572e-05\\
0.980000000000001	-8.61850033043519e-05\\
0.982999999999993	-8.38949329104704e-05\\
0.985999999999986	-8.15416798840167e-05\\
0.985999999999993	-8.1541679884011e-05\\
0.986000000000001	-8.15416798840052e-05\\
0.989999999999993	-7.83039272440832e-05\\
0.990000000000001	-7.83039272440774e-05\\
0.993999999999993	-7.49494630832039e-05\\
0.997999999999986	-7.14756573250192e-05\\
0.999999999999993	-6.96931538977346e-05\\
1	-6.96931538977282e-05\\
1.00799999999999	-6.26151671080816e-05\\
1.00999999999999	-6.08797032664715e-05\\
1.01	-6.08797032664592e-05\\
1.01499999999999	-5.65968697565506e-05\\
1.015	-5.65968697565386e-05\\
1.01999999999999	-5.23896018117568e-05\\
1.02	-5.2389601811745e-05\\
1.02499999999999	-4.82527449740554e-05\\
1.02999999999997	-4.4181231076488e-05\\
1.03	-4.41812310764652e-05\\
1.03499999999999	-4.01700719013518e-05\\
1.035	-4.01700719013405e-05\\
1.03999999999999	-3.62684475691184e-05\\
1.04	-3.62684475691075e-05\\
1.04399999999999	-3.32617421224035e-05\\
1.044	-3.3261742122393e-05\\
1.04799999999999	-3.03543387881424e-05\\
1.04999999999999	-2.89371600594872e-05\\
1.05	-2.89371600594772e-05\\
1.05399999999999	-2.6174459570689e-05\\
1.05799999999997	-2.3505504400891e-05\\
1.05999999999999	-2.22055245280643e-05\\
1.06	-2.22055245280551e-05\\
1.06799999999997	-1.73794045543739e-05\\
1.06999999999999	-1.6274496497497e-05\\
1.07	-1.62744964974893e-05\\
1.07299999999999	-1.4692450908133e-05\\
1.073	-1.46924509081257e-05\\
1.07599999999999	-1.32001679003735e-05\\
1.07899999999997	-1.1796989327222e-05\\
1.07999999999999	-1.13489560356659e-05\\
1.08	-1.13489560356596e-05\\
1.08499999999999	-9.25550920768639e-06\\
1.085	-9.25550920768079e-06\\
1.08999999999999	-7.40475816999422e-06\\
1.09	-7.40475816998923e-06\\
1.09299999999999	-6.40984600255705e-06\\
1.093	-6.40984600255254e-06\\
1.09599999999999	-5.48838727698545e-06\\
1.09899999999997	-4.62731926157693e-06\\
1.09999999999999	-4.35365034749533e-06\\
1.1	-4.35365034749149e-06\\
1.10199999999999	-3.82626220748261e-06\\
1.102	-3.82626220747896e-06\\
1.10399999999999	-3.32538805611812e-06\\
1.10599999999997	-2.85092971763011e-06\\
1.10999999999994	-1.98089365723358e-06\\
1.10999999999999	-1.98089365722464e-06\\
1.11	-1.98089365722173e-06\\
1.11799999999994	-5.5406423745634e-07\\
1.11999999999999	-2.62200488842525e-07\\
1.12	-2.62200488840543e-07\\
1.12799999999994	6.76533478320239e-07\\
1.12999999999999	8.56074829204535e-07\\
1.13	8.56074829205733e-07\\
1.13099999999999	9.37600973072676e-07\\
1.131	9.37600973073795e-07\\
1.13199999999999	1.01363617794279e-06\\
1.13299999999999	1.0841841708413e-06\\
1.13499999999997	1.20883207948458e-06\\
1.13899999999994	1.39241546034296e-06\\
1.13999999999999	1.42463472509868e-06\\
1.14	1.4246347250991e-06\\
1.14799999999994	1.48565058165763e-06\\
1.14999999999999	1.44626628008323e-06\\
1.15	1.44626628008287e-06\\
1.15099999999999	1.41837565880487e-06\\
1.151	1.41837565880444e-06\\
1.15199999999999	1.3867022773953e-06\\
1.15299999999999	1.35292920358639e-06\\
1.15499999999997	1.27907725625326e-06\\
1.155	1.27907725625216e-06\\
1.15899999999997	1.10609733402214e-06\\
1.15999999999999	1.05757430409747e-06\\
1.16	1.05757430409677e-06\\
1.16399999999997	8.42298654923144e-07\\
1.16799999999994	5.92999029280109e-07\\
1.16999999999999	4.55530187435144e-07\\
1.17	4.55530187434136e-07\\
1.17799999999994	-1.80453733562044e-07\\
1.17999999999999	-3.61132671177989e-07\\
1.18	-3.61132671179304e-07\\
1.18799999999994	-1.09879527629636e-06\\
1.18899999999999	-1.19071085292659e-06\\
1.189	-1.19071085292789e-06\\
1.18999999999999	-1.28257651514401e-06\\
1.19	-1.28257651514532e-06\\
1.19099999999999	-1.37439676339531e-06\\
1.19199999999999	-1.46617609788438e-06\\
1.19399999999997	-1.64963001261835e-06\\
1.19799999999994	-2.01624464678567e-06\\
1.19999999999999	-2.19947722650923e-06\\
1.2	-2.19947722651054e-06\\
1.20799999999994	-2.9024888549027e-06\\
1.20999999999999	-3.06904364077629e-06\\
1.21	-3.06904364077746e-06\\
1.21799999999994	-3.69944980899344e-06\\
1.21799999999997	-3.69944980899553e-06\\
1.218	-3.69944980899762e-06\\
1.21999999999999	-3.84825256128828e-06\\
1.22	-3.84825256128933e-06\\
1.22199999999999	-3.99359515538748e-06\\
1.22399999999997	-4.13550608128081e-06\\
1.22499999999999	-4.2051833870278e-06\\
1.225	-4.20518338702879e-06\\
1.22899999999997	-4.4754507430482e-06\\
1.22999999999999	-4.54092366930462e-06\\
1.23	-4.54092366930555e-06\\
1.23399999999997	-4.79453597266536e-06\\
1.23799999999995	-5.03504890921509e-06\\
1.23799999999997	-5.03504890921666e-06\\
1.238	-5.03504890921822e-06\\
1.23999999999999	-5.14945961507694e-06\\
1.24	-5.14945961507773e-06\\
1.24199999999999	-5.25867954708842e-06\\
1.24399999999997	-5.36273011469777e-06\\
1.24699999999999	-5.50915774423749e-06\\
1.247	-5.50915774423816e-06\\
1.24999999999999	-5.64406453134481e-06\\
1.25	-5.64406453134542e-06\\
1.25299999999999	-5.76750997207405e-06\\
1.25599999999997	-5.87954850789479e-06\\
1.25999999999999	-6.01127382787093e-06\\
1.26	-6.01127382787136e-06\\
1.26599999999997	-6.1705282962604e-06\\
1.26999999999999	-6.25095447543705e-06\\
1.27	-6.2509544754373e-06\\
1.27599999999997	-6.33313620029345e-06\\
1.276	-6.33313620029371e-06\\
1.28	-6.3623482996826e-06\\
1.28000000000001	-6.36234829968267e-06\\
1.28400000000002	-6.37112888452449e-06\\
1.28800000000002	-6.35948483939396e-06\\
1.29	-6.34600135333367e-06\\
1.29000000000001	-6.34600135333355e-06\\
1.29499999999999	-6.28992449972502e-06\\
1.295	-6.28992449972481e-06\\
1.29599999999999	-6.2748703198453e-06\\
1.296	-6.27487031984507e-06\\
1.29699999999999	-6.25876327065578e-06\\
1.29799999999999	-6.24183087144015e-06\\
1.29999999999997	-6.2054866636831e-06\\
1.3	-6.20548666368257e-06\\
1.30399999999997	-6.12285482664599e-06\\
1.30499999999999	-6.10011947573877e-06\\
1.305	-6.10011947573844e-06\\
1.30899999999997	-6.00083508307987e-06\\
1.30999999999999	-5.97392215645516e-06\\
1.31	-5.97392215645477e-06\\
1.31399999999997	-5.85786357237691e-06\\
1.31799999999994	-5.72828425311697e-06\\
1.31999999999999	-5.65839298417165e-06\\
1.32	-5.65839298417114e-06\\
1.32799999999994	-5.36284082318505e-06\\
1.32999999999999	-5.28603026386265e-06\\
1.33	-5.2860302638621e-06\\
1.33399999999999	-5.12879754221041e-06\\
1.334	-5.12879754220984e-06\\
1.33799999999999	-4.96664622386803e-06\\
1.33999999999999	-4.88368651402257e-06\\
1.34	-4.88368651402198e-06\\
1.34399999999999	-4.71391768815429e-06\\
1.34799999999997	-4.53890497709274e-06\\
1.35	-4.4493894459994e-06\\
1.35000000000001	-4.44938944599876e-06\\
1.354	-4.26625230570814e-06\\
1.35400000000001	-4.26625230570748e-06\\
1.358	-4.08058893535908e-06\\
1.35999999999999	-3.98791515149557e-06\\
1.36	-3.98791515149491e-06\\
1.36299999999999	-3.84905075750649e-06\\
1.363	-3.84905075750583e-06\\
1.36499999999999	-3.75654555418067e-06\\
1.365	-3.75654555418001e-06\\
1.36699999999999	-3.6640760436364e-06\\
1.36899999999997	-3.57162410154498e-06\\
1.36999999999999	-3.52539905581486e-06\\
1.37	-3.52539905581421e-06\\
1.37399999999997	-3.34045218885812e-06\\
1.37799999999995	-3.15532186147518e-06\\
1.37999999999999	-3.06264254762535e-06\\
1.38	-3.06264254762469e-06\\
1.38799999999994	-2.7060455882659e-06\\
1.38999999999999	-2.6212954073089e-06\\
1.39	-2.6212954073083e-06\\
1.39199999999999	-2.53827111357534e-06\\
1.392	-2.53827111357475e-06\\
1.39399999999998	-2.45695643529443e-06\\
1.39599999999997	-2.37733543321935e-06\\
1.39999999999994	-2.22311236270801e-06\\
1.39999999999999	-2.22311236270619e-06\\
1.4	-2.22311236270566e-06\\
1.40799999999994	-1.9343255443589e-06\\
1.41	-1.86614152021351e-06\\
1.41000000000001	-1.86614152021303e-06\\
1.41200000000001	-1.79953537070737e-06\\
1.41200000000003	-1.7995353707069e-06\\
1.41400000000003	-1.73441686130285e-06\\
1.41600000000003	-1.67069604696641e-06\\
1.41999999999999	-1.54739781706878e-06\\
1.42	-1.54739781706835e-06\\
1.42099999999999	-1.51742763234977e-06\\
1.421	-1.51742763234934e-06\\
1.42199999999999	-1.48779623462973e-06\\
1.42299999999999	-1.45850217193656e-06\\
1.42499999999997	-1.40092032642897e-06\\
1.42899999999994	-1.28974259636667e-06\\
1.42999999999999	-1.26277060381741e-06\\
1.43	-1.26277060381703e-06\\
1.43499999999999	-1.13277939823493e-06\\
1.435	-1.13277939823458e-06\\
1.43999999999998	-1.01078745577652e-06\\
1.44	-1.01078745577613e-06\\
1.44499999999999	-8.98533596392363e-07\\
1.44999999999997	-7.97768571382064e-07\\
1.44999999999998	-7.97768571381781e-07\\
1.45	-7.97768571381499e-07\\
1.45999999999997	-6.3022514976888e-07\\
1.45999999999999	-6.30225149768629e-07\\
1.46	-6.30225149768422e-07\\
1.46999999999997	-4.9679417040096e-07\\
1.46999999999998	-4.9679417040077e-07\\
1.47	-4.96794170400578e-07\\
1.47899999999998	-3.96316324462389e-07\\
1.479	-3.96316324462245e-07\\
1.47999999999999	-3.86279832212133e-07\\
1.48	-3.86279832211992e-07\\
1.48099999999999	-3.76466998722287e-07\\
1.48199999999999	-3.66877342974588e-07\\
1.48399999999997	-3.48365696037748e-07\\
1.48799999999994	-3.14000589594853e-07\\
1.49	-2.98140394143445e-07\\
1.49000000000001	-2.98140394143335e-07\\
1.49799999999996	-2.43444799507422e-07\\
1.49900000000001	-2.37586111879847e-07\\
1.49900000000003	-2.37586111879765e-07\\
1.49999999999999	-2.31907202833836e-07\\
1.5	-2.31907202833756e-07\\
1.50099999999999	-2.26371200185386e-07\\
1.50199999999999	-2.20977832597605e-07\\
1.50399999999997	-2.10617952475648e-07\\
1.50499999999999	-2.0565093230025e-07\\
1.505	-2.0565093230018e-07\\
1.50799999999998	-1.91598651511913e-07\\
1.508	-1.9159865151185e-07\\
1.50999999999999	-1.82935502833228e-07\\
1.51	-1.82935502833169e-07\\
1.51199999999999	-1.74834387851358e-07\\
1.51399999999998	-1.67293718720845e-07\\
1.51799999999995	-1.53887915592413e-07\\
1.51999999999999	-1.48020153900813e-07\\
1.52	-1.48020153900774e-07\\
1.52799999999995	-1.28487163241855e-07\\
1.52999999999999	-1.24483419875126e-07\\
1.53	-1.24483419875099e-07\\
1.53699999999998	-1.13223053106369e-07\\
1.537	-1.1322305310635e-07\\
1.53999999999999	-1.09703342333946e-07\\
1.54	-1.09703342333931e-07\\
1.54299999999999	-1.0696491965598e-07\\
1.54599999999997	-1.05006577338074e-07\\
1.54999999999999	-1.03607469333625e-07\\
1.55	-1.03607469333622e-07\\
1.55599999999997	-1.04104167722877e-07\\
1.55699999999998	-1.04489754089401e-07\\
1.557	-1.04489754089407e-07\\
1.55999999999999	-1.05918564554988e-07\\
1.56	-1.05918564554995e-07\\
1.56299999999999	-1.07632285313732e-07\\
1.56599999999998	-1.09631672145755e-07\\
1.566	-1.09631672145771e-07\\
1.56999999999999	-1.12743432932461e-07\\
1.57	-1.12743432932473e-07\\
1.57399999999999	-1.1636706643128e-07\\
1.57499999999999	-1.17353279012601e-07\\
1.575	-1.17353279012615e-07\\
1.57899999999999	-1.21620795960733e-07\\
1.57999999999999	-1.22768603240317e-07\\
1.58	-1.22768603240333e-07\\
1.58399999999999	-1.27386149770507e-07\\
1.58799999999998	-1.31928885437143e-07\\
1.58999999999999	-1.34173314487438e-07\\
1.59	-1.34173314487454e-07\\
1.59499999999998	-1.39709340738233e-07\\
1.595	-1.39709340738249e-07\\
1.59999999999998	-1.45143568131888e-07\\
1.6	-1.45143568131907e-07\\
1.60499999999998	-1.50482654313671e-07\\
1.60999999999997	-1.55733140331824e-07\\
1.60999999999998	-1.5573314033184e-07\\
1.61	-1.55733140331856e-07\\
1.61499999999998	-1.60901458822611e-07\\
1.615	-1.60901458822626e-07\\
1.61999999999998	-1.65769076333418e-07\\
1.62	-1.65769076333434e-07\\
1.624	-1.69288791799089e-07\\
1.62400000000001	-1.69288791799101e-07\\
1.62800000000001	-1.72478733572001e-07\\
1.62999999999999	-1.73950830956978e-07\\
1.63	-1.73950830956988e-07\\
1.634	-1.76650721455137e-07\\
1.638	-1.79026610391697e-07\\
1.63999999999999	-1.80093646306597e-07\\
1.64	-1.80093646306605e-07\\
1.64499999999998	-1.8232270444001e-07\\
1.645	-1.82322704440016e-07\\
1.64999999999998	-1.83877353584522e-07\\
1.65	-1.83877353584526e-07\\
1.65299999999998	-1.84487253395652e-07\\
1.653	-1.84487253395655e-07\\
1.65599999999998	-1.84855336441705e-07\\
1.65899999999997	-1.84981765054762e-07\\
1.65999999999999	-1.84970219647453e-07\\
1.66	-1.84970219647453e-07\\
1.66599999999997	-1.84337111607414e-07\\
1.67	-1.83377601747931e-07\\
1.67000000000002	-1.83377601747926e-07\\
1.673	-1.82375324610327e-07\\
1.67300000000001	-1.82375324610322e-07\\
1.676	-1.81165712069055e-07\\
1.67899999999998	-1.79783643418954e-07\\
1.67999999999998	-1.7928452713522e-07\\
1.68	-1.79284527135213e-07\\
1.68199999999998	-1.78228509139839e-07\\
1.682	-1.78228509139831e-07\\
1.68399999999998	-1.77095269852024e-07\\
1.68599999999997	-1.75884587149885e-07\\
1.68999999999994	-1.73229927086151e-07\\
1.68999999999998	-1.73229927086119e-07\\
1.69	-1.73229927086109e-07\\
1.69799999999994	-1.66979821909786e-07\\
1.69999999999998	-1.65219534680676e-07\\
1.7	-1.65219534680663e-07\\
1.70799999999994	-1.57776084664207e-07\\
1.70999999999998	-1.55837760524645e-07\\
1.71	-1.55837760524631e-07\\
1.711	-1.54856721705546e-07\\
1.71100000000001	-1.54856721705532e-07\\
1.71200000000001	-1.53867701096435e-07\\
1.71300000000001	-1.52870650225118e-07\\
1.715	-1.50852261877192e-07\\
1.71500000000001	-1.50852261877177e-07\\
1.71700000000001	-1.48801161051131e-07\\
1.71900000000001	-1.46716945724651e-07\\
1.71999999999999	-1.4566229271829e-07\\
1.72	-1.45662292718275e-07\\
1.724	-1.41358834412461e-07\\
1.72799999999999	-1.36917076449065e-07\\
1.72999999999999	-1.34643251072792e-07\\
1.73	-1.34643251072776e-07\\
1.731	-1.3349290795556e-07\\
1.73100000000001	-1.33492907955543e-07\\
1.73200000000001	-1.32337181458921e-07\\
1.73300000000001	-1.31179660180316e-07\\
1.735	-1.28859006311337e-07\\
1.73899999999999	-1.24193659328994e-07\\
1.73999999999998	-1.23021981162675e-07\\
1.74	-1.23021981162659e-07\\
1.74799999999998	-1.13562665176104e-07\\
1.74999999999998	-1.11171852115688e-07\\
1.75	-1.11171852115671e-07\\
1.75799999999998	-1.01490603198705e-07\\
1.75999999999998	-9.90384198341185e-08\\
1.76	-9.9038419834101e-08\\
1.76799999999998	-8.95404974600331e-08\\
1.76899999999998	-8.83999909393547e-08\\
1.769	-8.83999909393386e-08\\
1.76999999999998	-8.72696804720876e-08\\
1.77	-8.72696804720716e-08\\
1.77099999999998	-8.61495106855157e-08\\
1.77199999999997	-8.50394266781538e-08\\
1.77399999999994	-8.2849298915198e-08\\
1.77799999999989	-7.85877121746785e-08\\
1.77999999999998	-7.65154178804714e-08\\
1.78	-7.65154178804568e-08\\
1.78499999999998	-7.14806909976541e-08\\
1.785	-7.14806909976401e-08\\
1.78999999999998	-6.66374359191699e-08\\
1.79	-6.66374359191564e-08\\
1.79499999999998	-6.197971901615e-08\\
1.798	-5.92717641358894e-08\\
1.79800000000001	-5.92717641358767e-08\\
1.79999999999999	-5.75018339961457e-08\\
1.8	-5.75018339961332e-08\\
1.80199999999998	-5.57597968155233e-08\\
1.80399999999995	-5.40453111454558e-08\\
1.8079999999999	-5.06976554925649e-08\\
1.80999999999999	-4.90638293335076e-08\\
1.81	-4.90638293334961e-08\\
1.8179999999999	-4.27877900178846e-08\\
1.818	-4.27877900178101e-08\\
1.81800000000001	-4.27877900177993e-08\\
1.81999999999998	-4.12856094182071e-08\\
1.82	-4.12856094181965e-08\\
1.82199999999997	-3.9815244511936e-08\\
1.82399999999994	-3.83764070790161e-08\\
1.82699999999998	-3.62766497061462e-08\\
1.827	-3.62766497061364e-08\\
1.82999999999998	-3.42462700189901e-08\\
1.83	-3.42462700189806e-08\\
1.83299999999999	-3.22843725875455e-08\\
1.83599999999997	-3.03900921828571e-08\\
1.83999999999998	-2.79681273471839e-08\\
1.84	-2.79681273471755e-08\\
1.84599999999997	-2.45867608699581e-08\\
1.84999999999998	-2.25123684931811e-08\\
1.85	-2.2512368493174e-08\\
1.85499999999998	-2.01188400604365e-08\\
1.855	-2.01188400604301e-08\\
1.85599999999998	-1.96664914964406e-08\\
1.856	-1.96664914964342e-08\\
1.85699999999998	-1.92228759199655e-08\\
1.85799999999997	-1.87879715927307e-08\\
1.85999999999994	-1.79442118701757e-08\\
1.85999999999999	-1.7944211870156e-08\\
1.86	-1.79442118701501e-08\\
1.86399999999994	-1.63603182307745e-08\\
1.86799999999988	-1.49135687245113e-08\\
1.86999999999999	-1.42412643576148e-08\\
1.87	-1.42412643576101e-08\\
1.876	-1.24270714305463e-08\\
1.87600000000001	-1.24270714305423e-08\\
1.87999999999999	-1.13672349652606e-08\\
1.88	-1.13672349652571e-08\\
1.88399999999998	-1.04043494973313e-08\\
1.88499999999998	-1.01786865906946e-08\\
1.885	-1.01786865906914e-08\\
1.88899999999997	-9.33593706692618e-09\\
1.89	-9.14017359287341e-09\\
1.89000000000001	-9.14017359287067e-09\\
1.89399999999999	-8.41652753970751e-09\\
1.89799999999996	-7.78751897596076e-09\\
1.89999999999999	-7.50834674272613e-09\\
1.9	-7.50834674272423e-09\\
1.90799999999995	-6.57693036177996e-09\\
1.90999999999999	-6.38710124176791e-09\\
1.91	-6.38710124176663e-09\\
1.914	-6.05880073040123e-09\\
1.91400000000001	-6.05880073040019e-09\\
1.91800000000001	-5.79876827861214e-09\\
1.91999999999999	-5.69428677450082e-09\\
1.92	-5.69428677450014e-09\\
1.924	-5.5362908126955e-09\\
1.92499999999998	-5.50739836329102e-09\\
1.925	-5.50739836329064e-09\\
1.92899999999999	-5.43421225626421e-09\\
1.92999999999999	-5.42650716944095e-09\\
1.93	-5.42650716944087e-09\\
1.934	-5.43804125505915e-09\\
1.93400000000001	-5.43804125505931e-09\\
1.93800000000001	-5.49545090128984e-09\\
1.93999999999999	-5.53315772120561e-09\\
1.94	-5.5331577212059e-09\\
1.94299999999998	-5.60099081995362e-09\\
1.943	-5.60099081995398e-09\\
1.94599999999998	-5.68237839495383e-09\\
1.94899999999996	-5.77735633945086e-09\\
1.94999999999999	-5.81204280320595e-09\\
1.95	-5.81204280320645e-09\\
1.95599999999997	-6.0520695302492e-09\\
1.95999999999998	-6.24262372456705e-09\\
1.96	-6.24262372456777e-09\\
1.96599999999996	-6.54355680601461e-09\\
1.97	-6.74066554893863e-09\\
1.97000000000001	-6.74066554893932e-09\\
1.97199999999998	-6.83822408161124e-09\\
1.972	-6.83822408161193e-09\\
1.97399999999997	-6.93514425979036e-09\\
1.97599999999994	-7.03144508058978e-09\\
1.97999999999988	-7.22226403343146e-09\\
1.98	-7.2222640334371e-09\\
1.98000000000002	-7.22226403343777e-09\\
1.9879999999999	-7.59729248199696e-09\\
1.99	-7.68977997517223e-09\\
1.99000000000002	-7.68977997517289e-09\\
1.99199999999998	-7.78179602196335e-09\\
1.992	-7.781796021964e-09\\
1.99399999999997	-7.87177873495097e-09\\
1.995	-7.91542066031057e-09\\
1.99500000000001	-7.91542066031119e-09\\
1.99699999999998	-8.0000163411671e-09\\
1.99899999999995	-8.08104153993642e-09\\
1.99999999999999	-8.12022019870419e-09\\
2	-8.12022019870474e-09\\
2.00099999999997	-8.15851213789408e-09\\
2.001	-8.15851213789516e-09\\
2.00199999999997	-8.19591923378523e-09\\
2.00299999999995	-8.23244331936913e-09\\
2.00499999999989	-8.30284957518835e-09\\
2.00899999999979	-8.43314182107459e-09\\
2.00999999999997	-8.46353253394796e-09\\
2.01	-8.46353253394881e-09\\
2.01799999999979	-8.67545808918363e-09\\
2.01999999999997	-8.71981997798883e-09\\
2.02	-8.71981997798944e-09\\
2.02799999999979	-8.84983100586474e-09\\
2.02999999999997	-8.8696804738648e-09\\
2.03	-8.86968047386505e-09\\
2.03799999999979	-8.89858192225543e-09\\
2.03999999999997	-8.89319026018961e-09\\
2.04	-8.8931902601895e-09\\
2.04799999999979	-8.83341279999698e-09\\
2.05	-8.80966896079454e-09\\
2.05000000000003	-8.80966896079418e-09\\
2.05799999999981	-8.67935606750398e-09\\
2.05899999999997	-8.65907849213657e-09\\
2.059	-8.65907849213598e-09\\
2.05999999999997	-8.63791153311926e-09\\
2.06	-8.63791153311865e-09\\
2.06099999999998	-8.61585415344539e-09\\
2.06199999999995	-8.59290527213713e-09\\
2.0639999999999	-8.54432846287758e-09\\
2.06499999999997	-8.51869815463033e-09\\
2.065	-8.51869815462958e-09\\
2.0689999999999	-8.40720108069898e-09\\
2.06999999999997	-8.37707602274301e-09\\
2.07	-8.37707602274214e-09\\
2.0739999999999	-8.24752834726524e-09\\
2.0779999999998	-8.10342698563718e-09\\
2.07899999999997	-8.06511464318357e-09\\
2.079	-8.06511464318247e-09\\
2.07999999999997	-8.02609369106118e-09\\
2.08	-8.02609369106007e-09\\
2.08099999999998	-7.98657209600803e-09\\
2.08199999999995	-7.946547921075e-09\\
2.08399999999991	-7.86498396207298e-09\\
2.08799999999981	-7.69573713112357e-09\\
2.088	-7.69573713111541e-09\\
2.08800000000003	-7.69573713111417e-09\\
2.08999999999997	-7.60802108508809e-09\\
2.09	-7.60802108508682e-09\\
2.09199999999995	-7.51822049883055e-09\\
2.0939999999999	-7.42631777001138e-09\\
2.09799999999979	-7.23613341620939e-09\\
2.09999999999997	-7.13781451316666e-09\\
2.1	-7.13781451316525e-09\\
2.10799999999979	-6.73839905826255e-09\\
2.10999999999997	-6.63790127714492e-09\\
2.11	-6.63790127714349e-09\\
2.11699999999997	-6.28369129171821e-09\\
2.117	-6.28369129171677e-09\\
2.11999999999997	-6.13056305712551e-09\\
2.12	-6.13056305712405e-09\\
2.12299999999998	-5.97654307892396e-09\\
2.12599999999995	-5.82156343072524e-09\\
2.13	-5.61331288147189e-09\\
2.13000000000003	-5.6133128814704e-09\\
2.13499999999997	-5.35018068312386e-09\\
2.135	-5.35018068312235e-09\\
2.137	-5.24398467038055e-09\\
2.13700000000002	-5.24398467037904e-09\\
2.13900000000002	-5.13786893007249e-09\\
2.14	-5.08507854456881e-09\\
2.14000000000003	-5.08507854456731e-09\\
2.14200000000003	-4.98001980903943e-09\\
2.14400000000002	-4.87563997509583e-09\\
2.14599999999997	-4.77191858389025e-09\\
2.146	-4.77191858388879e-09\\
2.14999999999999	-4.56636993618967e-09\\
2.15000000000003	-4.56636993618778e-09\\
2.15400000000002	-4.36321270520412e-09\\
2.15800000000002	-4.16228760526055e-09\\
2.15999999999997	-4.06261280812488e-09\\
2.16	-4.06261280812347e-09\\
2.16799999999999	-3.67965964592755e-09\\
2.16999999999997	-3.5884481852551e-09\\
2.17	-3.58844818525382e-09\\
2.17499999999997	-3.36814449287551e-09\\
2.175	-3.36814449287428e-09\\
2.17999999999997	-3.15866215319304e-09\\
2.18	-3.15866215319165e-09\\
2.18499999999997	-2.95974452253307e-09\\
2.18999999999994	-2.77114790192347e-09\\
2.18999999999997	-2.77114790192236e-09\\
2.19	-2.77114790192126e-09\\
2.19499999999997	-2.59264123184379e-09\\
2.195	-2.59264123184281e-09\\
2.19999999999996	-2.42375741152347e-09\\
2.2	-2.42375741152235e-09\\
2.20399999999997	-2.29526052341299e-09\\
2.204	-2.2952605234121e-09\\
2.20499999999997	-2.26404111356492e-09\\
2.205	-2.26404111356404e-09\\
2.20599999999997	-2.23318054257863e-09\\
2.20699999999995	-2.20267729825124e-09\\
2.20899999999989	-2.14273682834608e-09\\
2.20999999999997	-2.11329666564264e-09\\
2.21	-2.11329666564181e-09\\
2.21399999999989	-1.99903637035538e-09\\
2.21799999999979	-1.89031007076117e-09\\
2.21999999999997	-1.83799536119601e-09\\
2.22	-1.83799536119528e-09\\
2.22799999999979	-1.64485683845248e-09\\
2.22999999999997	-1.60072368877682e-09\\
2.23	-1.60072368877621e-09\\
2.23299999999997	-1.53760160983678e-09\\
2.233	-1.5376016098362e-09\\
2.23599999999997	-1.47814808220684e-09\\
2.23899999999994	-1.42233688487599e-09\\
2.24	-1.40453825881486e-09\\
2.24000000000003	-1.40453825881436e-09\\
2.24599999999997	-1.30614009984935e-09\\
2.25	-1.24847737132973e-09\\
2.25000000000003	-1.24847737132935e-09\\
2.25299999999997	-1.20936394495035e-09\\
2.253	-1.20936394494999e-09\\
2.25599999999994	-1.17298613992152e-09\\
2.25899999999988	-1.13853974192294e-09\\
2.26	-1.12748416068537e-09\\
2.26000000000003	-1.12748416068506e-09\\
2.26199999999997	-1.10600955937873e-09\\
2.262	-1.10600955937843e-09\\
2.26399999999994	-1.08538021615823e-09\\
2.26599999999988	-1.06559208749543e-09\\
2.26999999999977	-1.0285241238169e-09\\
2.27	-1.02852412381486e-09\\
2.27000000000003	-1.02852412381461e-09\\
2.27499999999997	-9.86855387042123e-10\\
2.275	-9.86855387041901e-10\\
2.27999999999994	-9.50323987234743e-10\\
2.28	-9.50323987234357e-10\\
2.28499999999995	-9.18312723832458e-10\\
2.28999999999989	-8.90209934374831e-10\\
2.29	-8.90209934374262e-10\\
2.29000000000003	-8.90209934374113e-10\\
2.29099999999997	-8.8505527964348e-10\\
2.291	-8.85055279643336e-10\\
2.29199999999997	-8.80055319074056e-10\\
2.29299999999994	-8.75209807521751e-10\\
2.29499999999989	-8.65981189300331e-10\\
2.29899999999979	-8.49367399754311e-10\\
2.29999999999997	-8.45596805591199e-10\\
2.3	-8.45596805591094e-10\\
2.30799999999979	-8.20917482252762e-10\\
2.31	-8.16265907580453e-10\\
2.31000000000003	-8.16265907580391e-10\\
2.31099999999999	-8.14167160377177e-10\\
2.31100000000002	-8.1416716037712e-10\\
2.31199999999999	-8.12160483730895e-10\\
2.31299999999996	-8.10186629319306e-10\\
2.31499999999989	-8.0633700192655e-10\\
2.31899999999976	-7.9902746548877e-10\\
2.31999999999997	-7.97280748036335e-10\\
2.32	-7.97280748036285e-10\\
2.32799999999974	-7.84455500781731e-10\\
2.32999999999997	-7.81565348503442e-10\\
2.33	-7.81565348503402e-10\\
2.33799999999974	-7.71252380567427e-10\\
2.33999999999997	-7.68983522116406e-10\\
2.34	-7.68983522116374e-10\\
2.345	-7.63373375496646e-10\\
2.34500000000003	-7.63373375496614e-10\\
2.34899999999997	-7.58750199368782e-10\\
2.349	-7.58750199368749e-10\\
2.34999999999997	-7.57575178297802e-10\\
2.35	-7.57575178297768e-10\\
2.35099999999997	-7.56392351827789e-10\\
2.35199999999994	-7.55201661999016e-10\\
2.35399999999988	-7.52796458500768e-10\\
2.35799999999976	-7.47889097892229e-10\\
2.35999999999997	-7.45385978888841e-10\\
2.36	-7.45385978888806e-10\\
2.36799999999976	-7.34208243940452e-10\\
2.36999999999997	-7.31068260580098e-10\\
2.37	-7.31068260580053e-10\\
2.37799999999976	-7.17107662273996e-10\\
2.378	-7.17107662273559e-10\\
2.37800000000002	-7.17107662273506e-10\\
2.37999999999997	-7.1326392606355e-10\\
2.38	-7.13263926063494e-10\\
2.38199999999995	-7.09277290202743e-10\\
2.38399999999989	-7.05146973251639e-10\\
2.38799999999979	-6.96452029536267e-10\\
2.38999999999997	-6.91885698476349e-10\\
2.39	-6.91885698476283e-10\\
2.39799999999979	-6.72140091779193e-10\\
2.398	-6.72140091778647e-10\\
2.39800000000002	-6.72140091778573e-10\\
2.39999999999997	-6.66840877063099e-10\\
2.4	-6.66840877063023e-10\\
2.40199999999994	-6.61413854361904e-10\\
2.40399999999988	-6.55857959893151e-10\\
2.40699999999997	-6.47280094535823e-10\\
2.407	-6.47280094535741e-10\\
2.40999999999997	-6.38406028564223e-10\\
2.41	-6.38406028564137e-10\\
2.41299999999997	-6.29231848372742e-10\\
2.41499999999997	-6.22947005615683e-10\\
2.415	-6.22947005615592e-10\\
2.41799999999997	-6.1326357786174e-10\\
2.41999999999997	-6.06635338349725e-10\\
2.42	-6.0663533834963e-10\\
2.42299999999997	-5.96548548410694e-10\\
2.42599999999994	-5.86378681997985e-10\\
2.42999999999997	-5.72681872515567e-10\\
2.42999999999999	-5.72681872515469e-10\\
2.43599999999994	-5.51821251141923e-10\\
2.43599999999997	-5.51821251141816e-10\\
2.436	-5.5182125114171e-10\\
2.43999999999997	-5.37687997229574e-10\\
2.44	-5.37687997229473e-10\\
2.44399999999998	-5.23360897921508e-10\\
2.44799999999996	-5.08828720016225e-10\\
2.44999999999997	-5.01482172394479e-10\\
2.45	-5.01482172394375e-10\\
2.45599999999999	-4.79103382632827e-10\\
2.45600000000002	-4.79103382632719e-10\\
2.45999999999997	-4.64031673975844e-10\\
2.46	-4.64031673975737e-10\\
2.46399999999995	-4.48997883329356e-10\\
2.46499999999997	-4.45243978508672e-10\\
2.465	-4.45243978508566e-10\\
2.46899999999995	-4.3024101211144e-10\\
2.46999999999997	-4.26492514795603e-10\\
2.47	-4.26492514795497e-10\\
2.47399999999995	-4.11501992137555e-10\\
2.4779999999999	-3.96508201504441e-10\\
2.47999999999997	-3.89006407222263e-10\\
2.48	-3.89006407222157e-10\\
2.48499999999997	-3.70577402053518e-10\\
2.485	-3.70577402053415e-10\\
2.48999999999997	-3.52796280637683e-10\\
2.49	-3.52796280637577e-10\\
2.49399999999997	-3.39023194812885e-10\\
2.494	-3.39023194812789e-10\\
2.49799999999996	-3.25639999158652e-10\\
2.49999999999997	-3.19091319327772e-10\\
2.5	-3.1909131932768e-10\\
2.50399999999997	-3.06273377613631e-10\\
2.50799999999994	-2.9381964853178e-10\\
2.50999999999997	-2.8772630132853e-10\\
2.51	-2.87726301328444e-10\\
2.51399999999997	-2.75800670027453e-10\\
2.514	-2.7580067002737e-10\\
2.51799999999996	-2.64217159744458e-10\\
2.51999999999997	-2.58551526603449e-10\\
2.52	-2.58551526603369e-10\\
2.523	-2.50207600348613e-10\\
2.52300000000002	-2.50207600348535e-10\\
2.52600000000002	-2.42045837337729e-10\\
2.52900000000001	-2.34062638099511e-10\\
2.52999999999997	-2.31440641756315e-10\\
2.53	-2.31440641756241e-10\\
2.53599999999999	-2.1610995534186e-10\\
2.53999999999997	-2.06262549484042e-10\\
2.54	-2.06262549483973e-10\\
2.54599999999999	-1.92278414912268e-10\\
2.54999999999997	-1.83579283731091e-10\\
2.55	-1.83579283731031e-10\\
2.55199999999997	-1.7941421157646e-10\\
2.552	-1.79414211576402e-10\\
2.55399999999996	-1.75371038567265e-10\\
2.55499999999997	-1.73394915656862e-10\\
2.555	-1.73394915656806e-10\\
2.55699999999997	-1.69533112868027e-10\\
2.55899999999994	-1.6579127247684e-10\\
2.55999999999997	-1.63965108072526e-10\\
2.56	-1.63965108072475e-10\\
2.56399999999994	-1.56956751441789e-10\\
2.56799999999987	-1.50418407402933e-10\\
2.56999999999997	-1.47323873772532e-10\\
2.57	-1.47323873772489e-10\\
2.57199999999997	-1.44344949461336e-10\\
2.572	-1.44344949461294e-10\\
2.57399999999996	-1.41457126898646e-10\\
2.57599999999993	-1.38635916272592e-10\\
2.57999999999986	-1.3319113191282e-10\\
2.57999999999997	-1.33191131912673e-10\\
2.58	-1.33191131912635e-10\\
2.58099999999997	-1.31870719910768e-10\\
2.581	-1.31870719910731e-10\\
2.58199999999996	-1.30566490968613e-10\\
2.58299999999993	-1.29278381163306e-10\\
2.58499999999987	-1.26750267279321e-10\\
2.58899999999974	-1.21884744519207e-10\\
2.58999999999997	-1.20707745209955e-10\\
2.59	-1.20707745209922e-10\\
2.59799999999975	-1.11849964530719e-10\\
2.59999999999997	-1.09788596327074e-10\\
2.6	-1.09788596327045e-10\\
2.60799999999975	-1.02214965755642e-10\\
2.60999999999997	-1.00492140559324e-10\\
2.61	-1.004921405593e-10\\
2.61799999999974	-9.42730376338613e-11\\
2.61999999999997	-9.28847876376042e-11\\
2.62	-9.2884787637585e-11\\
2.62499999999997	-8.95708139484621e-11\\
2.625	-8.95708139484437e-11\\
2.62999999999998	-8.6402128505837e-11\\
2.63000000000001	-8.64021285058194e-11\\
2.63499999999998	-8.33748492499063e-11\\
2.63899999999997	-8.10523402844032e-11\\
2.639	-8.1052340284387e-11\\
2.63999999999997	-8.04852673830563e-11\\
2.64	-8.04852673830403e-11\\
2.64099999999998	-7.99235602487572e-11\\
2.64199999999996	-7.93671913565217e-11\\
2.64399999999991	-7.82703595096847e-11\\
2.64799999999982	-7.61395725466454e-11\\
2.64999999999997	-7.51051997733065e-11\\
2.65	-7.5105199773292e-11\\
2.65799999999982	-7.11698292694209e-11\\
2.65899999999997	-7.07002477859909e-11\\
2.659	-7.07002477859776e-11\\
2.66	-7.0234458608791e-11\\
2.66000000000003	-7.02344586087778e-11\\
2.66100000000003	-6.97713449270485e-11\\
2.66200000000004	-6.93108840428852e-11\\
2.66400000000005	-6.83978305454371e-11\\
2.66799999999997	-6.66025729373778e-11\\
2.668	-6.66025729373652e-11\\
2.67	-6.57200169498854e-11\\
2.67000000000003	-6.57200169498729e-11\\
2.67200000000003	-6.48472782099789e-11\\
2.67400000000004	-6.39841856574965e-11\\
2.67800000000005	-6.22862642966853e-11\\
2.67999999999997	-6.14511026778576e-11\\
2.68	-6.14511026778458e-11\\
2.68800000000002	-5.82099459213365e-11\\
2.68999999999997	-5.74244408295731e-11\\
2.69	-5.7424440829562e-11\\
2.69499999999997	-5.5502358348717e-11\\
2.695	-5.55023583487062e-11\\
2.69699999999997	-5.47498018827195e-11\\
2.697	-5.47498018827088e-11\\
2.69899999999996	-5.4006324014263e-11\\
2.69999999999997	-5.36379439460186e-11\\
2.7	-5.36379439460082e-11\\
2.70199999999997	-5.29078112872098e-11\\
2.70399999999993	-5.21863961883396e-11\\
2.70799999999987	-5.07691547526691e-11\\
2.71	-5.00730506214906e-11\\
2.71000000000003	-5.00730506214808e-11\\
2.71699999999999	-4.77000804432976e-11\\
2.71700000000002	-4.77000804432882e-11\\
2.72	-4.67109113776037e-11\\
2.72000000000003	-4.67109113775944e-11\\
2.72300000000001	-4.57358755665811e-11\\
2.72599999999999	-4.47745429961321e-11\\
2.72600000000002	-4.4774542996123e-11\\
2.73	-4.35133504565056e-11\\
2.73000000000003	-4.35133504564967e-11\\
2.73400000000002	-4.22747773583748e-11\\
2.738	-4.10578525948603e-11\\
2.74	-4.045720952113e-11\\
2.74000000000003	-4.04572095211215e-11\\
2.748	-3.81342636547448e-11\\
2.75	-3.75747537730609e-11\\
2.75000000000003	-3.7574753773053e-11\\
2.75499999999997	-3.62119192997308e-11\\
2.755	-3.62119192997232e-11\\
2.75999999999993	-3.48990998095095e-11\\
2.75999999999997	-3.48990998095006e-11\\
2.76	-3.48990998094917e-11\\
2.76499999999994	-3.36346869247027e-11\\
2.76499999999997	-3.3634686924695e-11\\
2.765	-3.36346869246872e-11\\
2.76999999999994	-3.24171315814151e-11\\
2.76999999999997	-3.24171315814076e-11\\
2.77	-3.24171315814001e-11\\
2.77499999999994	-3.12449421220853e-11\\
2.775	-3.12449421220724e-11\\
2.77999999999993	-3.01104167764235e-11\\
2.77999999999997	-3.01104167764156e-11\\
2.78	-3.01104167764079e-11\\
2.78399999999997	-2.92244680781259e-11\\
2.784	-2.92244680781196e-11\\
2.78799999999996	-2.83570294766106e-11\\
2.78999999999997	-2.79300383731651e-11\\
2.79	-2.79300383731591e-11\\
2.79399999999997	-2.70890940875259e-11\\
2.79799999999993	-2.62649856558159e-11\\
2.79999999999997	-2.58590424576445e-11\\
2.8	-2.58590424576387e-11\\
2.80799999999993	-2.43004732303696e-11\\
2.80999999999997	-2.39283939891962e-11\\
2.81	-2.3928393989191e-11\\
2.81299999999997	-2.33831476800482e-11\\
2.813	-2.33831476800431e-11\\
2.81599999999996	-2.28531352002142e-11\\
2.81899999999993	-2.23381227974411e-11\\
2.81999999999997	-2.21697459008016e-11\\
2.82	-2.21697459007969e-11\\
2.82599999999993	-2.119349716317e-11\\
2.83	-2.05744773076518e-11\\
2.83000000000003	-2.05744773076475e-11\\
2.83299999999999	-2.0126566364595e-11\\
2.83300000000002	-2.01265663645908e-11\\
2.835	-1.98342705731522e-11\\
2.83500000000003	-1.98342705731481e-11\\
2.83700000000001	-1.95453180762722e-11\\
2.83899999999999	-1.92596522383142e-11\\
2.84	-1.91180342941144e-11\\
2.84000000000003	-1.91180342941104e-11\\
2.84199999999997	-1.88371936587488e-11\\
2.842	-1.88371936587448e-11\\
2.84399999999993	-1.85595008878352e-11\\
2.84599999999987	-1.82849015525184e-11\\
2.84999999999974	-1.77447684950446e-11\\
2.85	-1.77447684950099e-11\\
2.85000000000003	-1.77447684950061e-11\\
2.85799999999978	-1.6699294759783e-11\\
2.85999999999997	-1.64448750834372e-11\\
2.86	-1.64448750834336e-11\\
2.86799999999975	-1.54723589215127e-11\\
2.86999999999997	-1.52414609397052e-11\\
2.87	-1.52414609397019e-11\\
2.87099999999997	-1.51278149215229e-11\\
2.871	-1.51278149215197e-11\\
2.87199999999996	-1.50153634416087e-11\\
2.87299999999993	-1.49041009885052e-11\\
2.87499999999986	-1.46851214131863e-11\\
2.87899999999973	-1.42611945723838e-11\\
2.87999999999997	-1.41581058824103e-11\\
2.88	-1.41581058824074e-11\\
2.88799999999974	-1.33742765562098e-11\\
2.88999999999997	-1.3189499302409e-11\\
2.89	-1.31894993024064e-11\\
2.89099999999997	-1.30987622646033e-11\\
2.89099999999999	-1.30987622646008e-11\\
2.89199999999996	-1.30088021121309e-11\\
2.89299999999993	-1.29192962071742e-11\\
2.89499999999986	-1.27416296186142e-11\\
2.89899999999973	-1.23915559010911e-11\\
2.89999999999997	-1.23051081446577e-11\\
2.9	-1.23051081446552e-11\\
2.90499999999997	-1.18790818565365e-11\\
2.905	-1.18790818565341e-11\\
2.90999999999998	-1.1463028823498e-11\\
2.91000000000001	-1.14630288234956e-11\\
2.91499999999999	-1.10564393256473e-11\\
2.91999999999997	-1.0658815240005e-11\\
2.92	-1.06588152400023e-11\\
2.92899999999997	-9.98206172895508e-12\\
2.92899999999999	-9.98206172895304e-12\\
2.92999999999997	-9.91065685946008e-12\\
2.93	-9.91065685945806e-12\\
2.93099999999998	-9.83999688459818e-12\\
2.93199999999996	-9.7700783407847e-12\\
2.93399999999992	-9.63245187836882e-12\\
2.93799999999983	-9.36594779472773e-12\\
2.93999999999997	-9.23701793583078e-12\\
2.94	-9.23701793582896e-12\\
2.94799999999983	-8.73541913644989e-12\\
2.94999999999997	-8.6125449942481e-12\\
2.95	-8.61254499424636e-12\\
2.95799999999983	-8.13042803162266e-12\\
2.95799999999997	-8.13042803161459e-12\\
2.95799999999999	-8.13042803161291e-12\\
2.95999999999997	-8.01212557061052e-12\\
2.96	-8.01212557060884e-12\\
2.96199999999998	-7.89466719088311e-12\\
2.96399999999996	-7.77802986845263e-12\\
2.96799999999992	-7.54712710677254e-12\\
2.96999999999997	-7.43281640815155e-12\\
2.97	-7.43281640814993e-12\\
2.97499999999997	-7.15018708595127e-12\\
2.975	-7.15018708594968e-12\\
2.97799999999997	-6.98265701812317e-12\\
2.97799999999999	-6.98265701812159e-12\\
2.97999999999997	-6.87187669117859e-12\\
2.98	-6.87187669117702e-12\\
2.98199999999998	-6.76191486899899e-12\\
2.98399999999996	-6.65274999870539e-12\\
2.98699999999999	-6.49045021449203e-12\\
2.98700000000002	-6.4904502144905e-12\\
2.99	-6.32982388831883e-12\\
2.99000000000003	-6.32982388831731e-12\\
2.99300000000001	-6.17080018140021e-12\\
2.99599999999999	-6.01330896167584e-12\\
2.99999999999997	-5.80558458313325e-12\\
3	-5.80558458313179e-12\\
3.00599999999996	-5.50271053821892e-12\\
3.00999999999997	-5.30823465638378e-12\\
3.01	-5.30823465638242e-12\\
3.01599999999996	-5.02729983304125e-12\\
3.01599999999999	-5.02729983303965e-12\\
3.01600000000002	-5.02729983303834e-12\\
3.01999999999997	-4.84698180030114e-12\\
3.02	-4.84698180029988e-12\\
3.02399999999995	-4.6720756727381e-12\\
3.0279999999999	-4.50244431467254e-12\\
3.03	-4.41956495186024e-12\\
3.03000000000003	-4.41956495185907e-12\\
3.03600000000002	-4.17847793779118e-12\\
3.03600000000005	-4.17847793779007e-12\\
3.03999999999997	-4.0238496450987e-12\\
3.04	-4.02384964509762e-12\\
3.04399999999993	-3.87390934097317e-12\\
3.04499999999997	-3.83714290697643e-12\\
3.04499999999999	-3.83714290697539e-12\\
3.04899999999992	-3.69289769015622e-12\\
3.04999999999997	-3.65753268065879e-12\\
3.05	-3.65753268065779e-12\\
3.05399999999993	-3.51880583271435e-12\\
3.05799999999985	-3.38437099284798e-12\\
3.05999999999997	-3.31873000707718e-12\\
3.06	-3.31873000707625e-12\\
3.06799999999985	-3.07009124933989e-12\\
3.06999999999997	-3.01158397335542e-12\\
3.07	-3.0115839733546e-12\\
3.07399999999997	-2.89887104545984e-12\\
3.07399999999999	-2.89887104545906e-12\\
3.07799999999996	-2.79182044041767e-12\\
3.08	-2.74039210771212e-12\\
3.08000000000003	-2.7403921077114e-12\\
3.08399999999999	-2.64167897951589e-12\\
3.08799999999996	-2.54842652106248e-12\\
3.09	-2.50382502827727e-12\\
3.09000000000003	-2.50382502827665e-12\\
3.09399999999997	-2.41862780558745e-12\\
3.09399999999999	-2.41862780558687e-12\\
3.09799999999993	-2.33771323731456e-12\\
3.09999999999997	-2.29846584891111e-12\\
3.1	-2.29846584891055e-12\\
3.10299999999997	-2.24108544478981e-12\\
3.10299999999999	-2.24108544478928e-12\\
3.10599999999996	-2.18547141918411e-12\\
3.10899999999992	-2.13159924539665e-12\\
3.10999999999997	-2.11402480390739e-12\\
3.11	-2.11402480390689e-12\\
3.11499999999997	-2.02898617686842e-12\\
3.115	-2.02898617686795e-12\\
3.11999999999998	-1.94859462926735e-12\\
3.12	-1.94859462926691e-12\\
3.12499999999998	-1.87306573762659e-12\\
3.12999999999995	-1.8026210358682e-12\\
3.13	-1.80262103586749e-12\\
3.13000000000003	-1.8026210358671e-12\\
3.13199999999997	-1.77584718850467e-12\\
3.13199999999999	-1.7758471885043e-12\\
3.13399999999993	-1.74986768765379e-12\\
3.13599999999986	-1.72467744022089e-12\\
3.13999999999973	-1.67664510986037e-12\\
3.13999999999997	-1.6766451098575e-12\\
3.14	-1.67664510985717e-12\\
3.14799999999974	-1.58984449407669e-12\\
3.14999999999997	-1.57004931873733e-12\\
3.15	-1.57004931873706e-12\\
3.15199999999997	-1.55100815186435e-12\\
3.15199999999999	-1.55100815186408e-12\\
3.15399999999996	-1.53248709376087e-12\\
3.15599999999992	-1.51425234605027e-12\\
3.15999999999985	-1.47862754094865e-12\\
3.15999999999997	-1.47862754094754e-12\\
3.16	-1.47862754094729e-12\\
3.16099999999997	-1.46989476602294e-12\\
3.16099999999999	-1.4698947660227e-12\\
3.16199999999996	-1.46123050088244e-12\\
3.16299999999993	-1.45263432087621e-12\\
3.16499999999986	-1.43564453472322e-12\\
3.16899999999973	-1.40246367158533e-12\\
3.16999999999997	-1.39433247766224e-12\\
3.17	-1.39433247766201e-12\\
3.17799999999974	-1.33158317716709e-12\\
3.17999999999997	-1.31652074635747e-12\\
3.18	-1.31652074635726e-12\\
3.18499999999997	-1.28005385459413e-12\\
3.185	-1.28005385459393e-12\\
3.18999999999998	-1.2453235176218e-12\\
3.19	-1.2453235176216e-12\\
3.19499999999998	-1.21228718616882e-12\\
3.19999999999995	-1.1809043865979e-12\\
3.2	-1.18090438659757e-12\\
3.20999999999995	-1.12008207770571e-12\\
3.21	-1.12008207770537e-12\\
3.21899999999997	-1.06572081402261e-12\\
3.21899999999999	-1.06572081402244e-12\\
3.21999999999997	-1.05969294708663e-12\\
3.22	-1.05969294708646e-12\\
3.22099999999998	-1.0536664512016e-12\\
3.22199999999996	-1.0476410309714e-12\\
3.22399999999992	-1.03559223652601e-12\\
3.22799999999984	-1.01149456638412e-12\\
3.22999999999997	-9.99440967292531e-13\\
3.23	-9.99440967292359e-13\\
3.23799999999984	-9.51139667835712e-13\\
3.23899999999997	-9.45087347845734e-13\\
3.23899999999999	-9.45087347845562e-13\\
3.23999999999997	-9.39034706433486e-13\\
3.24	-9.39034706433314e-13\\
3.24099999999998	-9.32985370404401e-13\\
3.24199999999996	-9.26939043274375e-13\\
3.24399999999992	-9.14854230758944e-13\\
3.24799999999984	-8.90707684673352e-13\\
3.24799999999997	-8.90707684672605e-13\\
3.24799999999999	-8.90707684672433e-13\\
3.24999999999997	-8.78641218127691e-13\\
3.25	-8.7864121812752e-13\\
3.25199999999998	-8.66576135872231e-13\\
3.25399999999996	-8.54510072937357e-13\\
3.25499999999998	-8.48475934671262e-13\\
3.255	-8.4847593467109e-13\\
3.25899999999996	-8.24325103166412e-13\\
3.25999999999997	-8.18282346431815e-13\\
3.26	-8.18282346431643e-13\\
3.26399999999996	-7.94309527844283e-13\\
3.26799999999992	-7.70730699665185e-13\\
3.26999999999997	-7.59083242946496e-13\\
3.27	-7.59083242946332e-13\\
3.27699999999999	-7.19023581964541e-13\\
3.27700000000002	-7.1902358196438e-13\\
3.28	-7.02174744974287e-13\\
3.28000000000003	-7.02174744974128e-13\\
3.28300000000001	-6.85506888448146e-13\\
3.28599999999999	-6.69012661465008e-13\\
3.28999999999998	-6.47277887084846e-13\\
3.29	-6.47277887084693e-13\\
3.29599999999996	-6.15193041698289e-13\\
3.29699999999999	-6.09902550348582e-13\\
3.29700000000002	-6.09902550348431e-13\\
3.29999999999997	-5.94176366059448e-13\\
3.3	-5.941763660593e-13\\
3.30299999999995	-5.78689135927983e-13\\
3.30599999999991	-5.63434029781097e-13\\
3.30599999999995	-5.63434029780873e-13\\
3.30599999999999	-5.63434029780649e-13\\
3.31	-5.43443356136929e-13\\
3.31000000000003	-5.43443356136788e-13\\
3.31400000000004	-5.23837724686929e-13\\
3.31800000000005	-5.04601763577628e-13\\
3.31999999999997	-4.95117685144386e-13\\
3.32	-4.95117685144252e-13\\
3.32499999999998	-4.72054464932063e-13\\
3.325	-4.72054464931935e-13\\
3.32999999999998	-4.50048077013678e-13\\
3.33000000000001	-4.50048077013556e-13\\
3.33499999999998	-4.29071560644823e-13\\
3.33500000000001	-4.29071560644707e-13\\
3.33999999999998	-4.09099216969682e-13\\
3.34000000000001	-4.09099216969572e-13\\
3.34499999999998	-3.90106577369569e-13\\
3.34999999999996	-3.72070373485909e-13\\
3.35000000000001	-3.72070373485722e-13\\
3.35499999999998	-3.54968508229181e-13\\
3.35500000000001	-3.54968508229086e-13\\
3.35999999999998	-3.38686862604296e-13\\
3.36000000000001	-3.38686862604205e-13\\
3.36399999999999	-3.26171589145044e-13\\
3.36400000000002	-3.26171589144956e-13\\
3.36800000000001	-3.14099033008136e-13\\
3.36999999999998	-3.0822580260644e-13\\
3.37000000000001	-3.08225802606358e-13\\
3.37399999999999	-2.96799681066666e-13\\
3.37799999999998	-2.85793247765051e-13\\
3.37999999999997	-2.80444704701789e-13\\
3.38	-2.80444704701714e-13\\
3.38799999999998	-2.60316098716094e-13\\
3.38999999999997	-2.55611568079499e-13\\
3.39	-2.55611568079433e-13\\
3.39299999999997	-2.48796676799765e-13\\
3.39299999999999	-2.48796676799701e-13\\
3.395	-2.44413387155417e-13\\
3.39500000000003	-2.44413387155355e-13\\
3.39700000000004	-2.40157068442025e-13\\
3.39900000000005	-2.36026886407075e-13\\
3.39999999999997	-2.34008842703082e-13\\
3.4	-2.34008842703025e-13\\
3.40400000000002	-2.2624802716638e-13\\
3.40800000000004	-2.18980877665631e-13\\
3.40999999999997	-2.15530632059631e-13\\
3.41	-2.15530632059583e-13\\
3.41299999999996	-2.10582514742852e-13\\
3.41299999999999	-2.10582514742807e-13\\
3.41599999999996	-2.0584043890365e-13\\
3.41899999999992	-2.01237584062374e-13\\
3.41999999999997	-1.9973388450433e-13\\
3.42	-1.99733884504288e-13\\
3.42199999999997	-1.96771920264999e-13\\
3.42199999999999	-1.96771920264957e-13\\
3.42399999999996	-1.93870053317123e-13\\
3.42599999999992	-1.91027714871167e-13\\
3.42999999999985	-1.85519406613776e-13\\
3.42999999999997	-1.85519406613602e-13\\
3.43	-1.85519406613564e-13\\
3.43799999999985	-1.75193387896375e-13\\
3.43999999999997	-1.72752789837843e-13\\
3.44	-1.72752789837809e-13\\
3.44799999999985	-1.63586862631611e-13\\
3.44999999999997	-1.61445200127268e-13\\
3.45	-1.61445200127238e-13\\
3.45099999999999	-1.60396546155083e-13\\
3.45100000000002	-1.60396546155054e-13\\
3.45200000000001	-1.59362608291644e-13\\
3.453	-1.58343335861402e-13\\
3.45499999999999	-1.5634858823837e-13\\
3.45899999999995	-1.52532928478764e-13\\
3.46	-1.51614955560505e-13\\
3.46000000000003	-1.51614955560479e-13\\
3.46499999999998	-1.47238502779404e-13\\
3.465	-1.4723850277938e-13\\
3.46999999999995	-1.43213868248431e-13\\
3.47	-1.4321386824839e-13\\
3.47000000000003	-1.43213868248368e-13\\
3.47099999999999	-1.42450718110256e-13\\
3.47100000000002	-1.42450718110234e-13\\
3.47199999999999	-1.41696435335632e-13\\
3.47299999999996	-1.40946013577412e-13\\
3.47499999999989	-1.39456606214246e-13\\
3.47899999999976	-1.36522518223525e-13\\
3.47999999999996	-1.35798104357712e-13\\
3.47999999999999	-1.35798104357692e-13\\
3.48799999999973	-1.30128440011399e-13\\
3.48999999999997	-1.28744653962377e-13\\
3.48999999999999	-1.28744653962358e-13\\
3.49799999999973	-1.23335891854934e-13\\
3.49999999999999	-1.22013971666649e-13\\
3.50000000000002	-1.2201397166663e-13\\
3.50799999999976	-1.16900347328858e-13\\
3.50899999999999	-1.1628191604106e-13\\
3.50900000000002	-1.16281916041043e-13\\
3.50999999999999	-1.15667998380342e-13\\
3.51000000000002	-1.15667998380325e-13\\
3.51099999999999	-1.15058564271332e-13\\
3.51199999999996	-1.14453583844692e-13\\
3.51399999999991	-1.13256865678601e-13\\
3.51799999999979	-1.10915583635259e-13\\
3.51999999999997	-1.09770560841717e-13\\
3.52	-1.097705608417e-13\\
3.52799999999977	-1.05226588574136e-13\\
3.52999999999997	-1.04090421988884e-13\\
3.53	-1.04090421988868e-13\\
3.53499999999998	-1.0124726622607e-13\\
3.535	-1.01247266226053e-13\\
3.53799999999997	-9.95383797396248e-14\\
3.53799999999999	-9.95383797396086e-14\\
3.53999999999997	-9.83974099467899e-14\\
3.54	-9.83974099467736e-14\\
3.54199999999998	-9.72548095798555e-14\\
3.54399999999996	-9.61103546853944e-14\\
3.54799999999993	-9.38149836390899e-14\\
3.54999999999997	-9.26636175706676e-14\\
3.55	-9.26636175706513e-14\\
3.55799999999993	-8.80283441254689e-14\\
3.55799999999996	-8.80283441254492e-14\\
3.55799999999999	-8.80283441254292e-14\\
3.56	-8.68630413646524e-14\\
3.56000000000003	-8.68630413646358e-14\\
3.56200000000004	-8.56980544032487e-14\\
3.56400000000005	-8.45331548832336e-14\\
3.56699999999996	-8.278547009037e-14\\
3.56699999999999	-8.27854700903534e-14\\
3.56999999999998	-8.10366975476699e-14\\
3.57	-8.10366975476534e-14\\
3.57299999999999	-7.92860660184684e-14\\
3.57599999999997	-7.75328034455906e-14\\
3.57999999999997	-7.51896906422172e-14\\
3.58	-7.51896906422005e-14\\
3.58599999999997	-7.17193774214219e-14\\
3.58999999999997	-6.94593025092553e-14\\
3.59	-6.94593025092394e-14\\
3.59599999999997	-6.61449606649467e-14\\
3.596	-6.61449606649312e-14\\
3.59999999999997	-6.39833866722024e-14\\
3.6	-6.39833866721872e-14\\
3.60399999999998	-6.18582208895072e-14\\
3.60499999999998	-6.1332422354761e-14\\
3.605	-6.13324223547461e-14\\
3.60899999999998	-5.92504269022408e-14\\
3.60999999999998	-5.87351001876604e-14\\
3.61	-5.87351001876458e-14\\
3.61399999999998	-5.66937243605164e-14\\
3.61599999999997	-5.56846453022969e-14\\
3.616	-5.56846453022826e-14\\
3.61999999999997	-5.36957184856008e-14\\
3.62	-5.36957184855868e-14\\
3.62399999999998	-5.1749138894093e-14\\
3.62499999999996	-5.12689311836847e-14\\
3.62499999999999	-5.12689311836711e-14\\
3.62899999999997	-4.93731431560622e-14\\
3.62999999999997	-4.89053407348684e-14\\
3.63	-4.89053407348551e-14\\
3.63399999999998	-4.70580216937147e-14\\
3.63799999999996	-4.52478455834767e-14\\
3.63999999999997	-4.43562414850055e-14\\
3.64	-4.43562414849929e-14\\
3.64799999999996	-4.09484368032976e-14\\
3.65	-4.01398510414467e-14\\
3.65000000000003	-4.01398510414353e-14\\
3.65399999999999	-3.85735967576355e-14\\
3.65400000000002	-3.85735967576246e-14\\
3.65799999999998	-3.70742143129458e-14\\
3.65999999999997	-3.63492306275563e-14\\
3.66	-3.63492306275462e-14\\
3.66399999999997	-3.49479678092178e-14\\
3.66799999999993	-3.36107341327484e-14\\
3.66999999999997	-3.29657986306557e-14\\
3.67	-3.29657986306467e-14\\
3.67399999999999	-3.17226582535347e-14\\
3.67400000000002	-3.17226582535261e-14\\
3.67499999999998	-3.14209850475682e-14\\
3.67500000000001	-3.14209850475597e-14\\
3.67599999999997	-3.11220714937663e-14\\
3.67699999999994	-3.08259029450339e-14\\
3.67899999999988	-3.02417429472744e-14\\
3.67999999999997	-2.99537228739846e-14\\
3.68	-2.99537228739764e-14\\
3.68299999999996	-2.91057333949796e-14\\
3.68299999999999	-2.91057333949717e-14\\
3.68599999999995	-2.82815605309508e-14\\
3.68899999999992	-2.74808408081505e-14\\
3.69	-2.72190847121041e-14\\
3.69000000000003	-2.72190847120967e-14\\
3.69599999999995	-2.57017368703901e-14\\
3.69999999999997	-2.47399255950896e-14\\
3.7	-2.47399255950829e-14\\
3.70599999999993	-2.33838201420579e-14\\
3.70999999999997	-2.25426192169134e-14\\
3.71	-2.25426192169076e-14\\
3.71199999999999	-2.21406285663209e-14\\
3.71200000000002	-2.21406285663153e-14\\
3.71400000000001	-2.17509385957753e-14\\
3.716	-2.13734729227774e-14\\
3.71999999999998	-2.06549209129948e-14\\
3.72000000000001	-2.06549209129899e-14\\
3.72799999999997	-1.93614119348742e-14\\
3.73000000000001	-1.90675771818961e-14\\
3.73000000000004	-1.9067577181892e-14\\
3.73200000000002	-1.87854407793811e-14\\
3.73200000000005	-1.87854407793771e-14\\
3.73400000000003	-1.8511871126943e-14\\
3.73600000000002	-1.82437382939931e-14\\
3.73999999999999	-1.77235739223079e-14\\
3.74000000000002	-1.77235739223043e-14\\
3.74099999999999	-1.75968499985432e-14\\
3.74100000000002	-1.75968499985396e-14\\
3.742	-1.7471440428247e-14\\
3.74299999999997	-1.73473390648628e-14\\
3.74499999999993	-1.71030366986578e-14\\
3.745	-1.71030366986485e-14\\
3.74500000000003	-1.71030366986451e-14\\
3.74899999999994	-1.66298671439814e-14\\
3.75000000000001	-1.6514756513655e-14\\
3.75000000000004	-1.65147565136517e-14\\
3.75399999999995	-1.6066871805887e-14\\
3.75799999999986	-1.56388184369852e-14\\
3.76	-1.54321230213357e-14\\
3.76000000000003	-1.54321230213328e-14\\
3.76799999999985	-1.46571193149398e-14\\
3.76999999999996	-1.44763627569683e-14\\
3.76999999999999	-1.44763627569657e-14\\
3.77799999999981	-1.38042509700107e-14\\
3.77999999999996	-1.36487869620697e-14\\
3.77999999999999	-1.36487869620676e-14\\
3.78799999999981	-1.30459655179011e-14\\
3.78999999999996	-1.28979773919453e-14\\
3.78999999999999	-1.28979773919432e-14\\
3.79799999999981	-1.23160235083557e-14\\
3.79899999999996	-1.22443452539914e-14\\
3.79899999999999	-1.22443452539894e-14\\
3.79999999999997	-1.21728921104518e-14\\
3.8	-1.21728921104497e-14\\
3.80099999999999	-1.21016605772905e-14\\
3.80199999999997	-1.20306471633623e-14\\
3.80399999999993	-1.18892607850304e-14\\
3.80799999999986	-1.1608953048753e-14\\
3.80999999999997	-1.14699767475403e-14\\
3.81	-1.14699767475383e-14\\
3.81499999999998	-1.11257522257637e-14\\
3.81500000000001	-1.11257522257617e-14\\
3.81899999999996	-1.08534583650381e-14\\
3.81899999999999	-1.08534583650362e-14\\
3.81999999999996	-1.07857626871095e-14\\
3.81999999999999	-1.07857626871076e-14\\
3.82099999999996	-1.07181748003756e-14\\
3.82199999999993	-1.06506913929571e-14\\
3.82399999999988	-1.05160247942156e-14\\
3.82799999999976	-1.02478001767703e-14\\
3.82799999999996	-1.02478001767566e-14\\
3.82799999999999	-1.02478001767547e-14\\
3.82999999999996	-1.01141895832226e-14\\
3.82999999999999	-1.01141895832207e-14\\
3.83199999999996	-9.98087853097389e-15\\
3.83399999999993	-9.84784088867086e-15\\
3.83799999999988	-9.58248157915955e-15\\
3.83999999999996	-9.45010789875122e-15\\
3.83999999999999	-9.45010789874934e-15\\
3.84799999999988	-8.92990189270999e-15\\
3.84999999999996	-8.80253586625342e-15\\
3.84999999999999	-8.80253586625162e-15\\
3.85699999999996	-8.36465636589612e-15\\
3.85699999999999	-8.36465636589437e-15\\
3.85999999999997	-8.18057264873354e-15\\
3.86	-8.18057264873181e-15\\
3.86299999999999	-7.99851857419281e-15\\
3.86599999999997	-7.81841385209582e-15\\
3.86999999999997	-7.58116938284241e-15\\
3.87	-7.58116938284073e-15\\
3.87599999999997	-7.23112971146507e-15\\
3.87699999999996	-7.17343281039885e-15\\
3.87699999999999	-7.17343281039721e-15\\
3.87999999999996	-7.00180826876802e-15\\
3.87999999999999	-7.00180826876641e-15\\
3.88299999999996	-6.83253446960107e-15\\
3.88499999999998	-6.72095433076203e-15\\
3.88500000000001	-6.72095433076045e-15\\
3.88599999999996	-6.66553675979805e-15\\
3.88599999999999	-6.66553675979648e-15\\
3.88699999999996	-6.61036389648823e-15\\
3.88799999999993	-6.5554330373375e-15\\
3.88999999999986	-6.44628657674261e-15\\
3.88999999999996	-6.4462865767372e-15\\
3.88999999999999	-6.44628657673566e-15\\
3.89399999999986	-6.23078005234936e-15\\
3.89799999999973	-6.0188482184759e-15\\
3.89999999999996	-5.9141707763168e-15\\
3.89999999999999	-5.91417077631532e-15\\
3.90799999999973	-5.51099085918341e-15\\
3.90999999999996	-5.41444300180267e-15\\
3.90999999999999	-5.41444300180131e-15\\
3.91499999999996	-5.18029747856448e-15\\
3.91499999999999	-5.18029747856318e-15\\
3.91999999999997	-4.95624415164757e-15\\
3.92	-4.95624415164595e-15\\
3.92499999999998	-4.74200852602223e-15\\
3.92999999999995	-4.53732813625725e-15\\
3.93	-4.53732813625515e-15\\
3.93499999999996	-4.34195221782897e-15\\
3.93499999999999	-4.34195221782788e-15\\
3.93999999999995	-4.15474796858554e-15\\
3.93999999999999	-4.15474796858417e-15\\
3.94399999999996	-4.01007046579708e-15\\
3.94399999999999	-4.01007046579607e-15\\
3.94799999999997	-3.86979063406472e-15\\
3.94999999999996	-3.80126534166682e-15\\
3.94999999999999	-3.80126534166586e-15\\
3.95399999999996	-3.6673768454759e-15\\
3.95499999999998	-3.63455390362956e-15\\
3.95500000000001	-3.63455390362863e-15\\
3.95899999999998	-3.50581061324929e-15\\
3.95999999999999	-3.4742540243245e-15\\
3.96000000000002	-3.47425402432361e-15\\
3.96399999999999	-3.35140279055303e-15\\
3.96799999999997	-3.23424195345618e-15\\
3.97000000000002	-3.17776655420574e-15\\
3.97000000000005	-3.17776655420495e-15\\
3.97299999999996	-3.09565350115056e-15\\
3.97299999999999	-3.0956535011498e-15\\
3.97599999999991	-3.0166284164241e-15\\
3.97899999999983	-2.94065644741392e-15\\
3.97999999999997	-2.91600509832809e-15\\
3.98	-2.9160050983274e-15\\
3.98599999999984	-2.77507603277661e-15\\
3.98999999999997	-2.68768650321334e-15\\
3.99	-2.68768650321273e-15\\
3.99299999999996	-2.62554076860017e-15\\
3.99299999999999	-2.62554076859959e-15\\
3.99599999999995	-2.5656165765489e-15\\
3.99899999999991	-2.50722840189015e-15\\
3.99999999999997	-2.48810254418684e-15\\
4	-2.4881025441863e-15\\
4.00199999999993	-2.45035049423439e-15\\
4.00199999999999	-2.45035049423333e-15\\
4.00399999999992	-2.41325851240015e-15\\
4.00599999999986	-2.37681932835731e-15\\
4.00999999999972	-2.30587091140678e-15\\
4.00999999999995	-2.30587091140286e-15\\
4.01	-2.30587091140187e-15\\
4.01799999999973	-2.17150179692448e-15\\
4.01999999999995	-2.13943920734999e-15\\
4.02	-2.13943920734909e-15\\
4.02500000000001	-2.06240260588939e-15\\
4.02500000000006	-2.06240260588854e-15\\
4.02999999999995	-1.99004078607611e-15\\
4.03	-1.99004078607531e-15\\
4.03099999999999	-1.97612149485718e-15\\
4.03100000000005	-1.9761214948564e-15\\
4.03200000000004	-1.96238494879604e-15\\
4.03300000000003	-1.94883047476899e-15\\
4.035	-1.92226509501615e-15\\
4.03899999999996	-1.87129057012396e-15\\
4.03999999999994	-1.85899249893376e-15\\
4.04	-1.85899249893306e-15\\
4.04799999999991	-1.76693133386828e-15\\
4.04999999999994	-1.74565194654486e-15\\
4.05	-1.74565194654426e-15\\
4.05099999999999	-1.73526968961211e-15\\
4.05100000000005	-1.73526968961153e-15\\
4.05200000000004	-1.72500171128585e-15\\
4.05300000000002	-1.71479084345032e-15\\
4.055	-1.69453844068539e-15\\
4.05899999999996	-1.65469713080699e-15\\
4.05999999999994	-1.64487218067451e-15\\
4.06	-1.64487218067395e-15\\
};
\end{axis}
\end{tikzpicture}%}
      \caption{The angular displacement of the stable penduli $P_1$ and $P_2$ as
        a function of time.  \texttt{Blue}: $P_1$, \texttt{Red}: $P_2$.
        $C_i = 10$ ms.}
      \label{fig:01.5.1}
    \end{figure}
  \end{minipage}
  \hfill
  \begin{minipage}{0.45\linewidth}
    \begin{figure}[H]\centering
      \scalebox{0.7}{% This file was created by matlab2tikz.
%
%The latest updates can be retrieved from
%  http://www.mathworks.com/matlabcentral/fileexchange/22022-matlab2tikz-matlab2tikz
%where you can also make suggestions and rate matlab2tikz.
%
\definecolor{mycolor1}{rgb}{0.00000,0.44700,0.74100}%
\definecolor{mycolor2}{rgb}{0.85000,0.32500,0.09800}%
\definecolor{mycolor3}{rgb}{0.92900,0.69400,0.12500}%
%
\begin{tikzpicture}

\begin{axis}[%
width=4.133in,
height=3.26in,
at={(0.693in,0.44in)},
scale only axis,
xmin=0,
xmax=1,
xlabel={Time (seconds)},
xmajorgrids,
ymin=-0.15,
ymax=0.2,
ymajorgrids,
axis background/.style={fill=white},
title style={font=\bfseries},
]
\addplot [color=mycolor1,solid,forget plot]
  table[row sep=crcr]{%
0	0.15314\\
3.15544362088405e-30	0.15314\\
0.000656101980281985	0.153143230512962\\
0.00393661188169191	0.153256312778436\\
0.00999999999999994	0.153891071773171\\
0.01	0.153891071773171\\
0.0199999999999999	0.150048203824684\\
0.02	0.150048203824684\\
0.0289999999999998	0.137414337712804\\
0.029	0.137414337712803\\
0.03	0.135470213386942\\
0.0300000000000002	0.135470213386942\\
0.0349999999999996	0.124115011067004\\
0.035	0.124115011067003\\
0.0399999999999993	0.110014119663841\\
0.04	0.110014119663839\\
0.0449999999999993	0.0939630779639858\\
0.0499999999999987	0.0767526455719492\\
0.05	0.0767526455719445\\
0.0500000000000004	0.0767526455719429\\
0.0579999999999996	0.0466980443355424\\
0.058	0.0466980443355407\\
0.0599999999999996	0.0386819498575326\\
0.06	0.0386819498575308\\
0.0619999999999995	0.0306155753047838\\
0.0639999999999991	0.0226526210813104\\
0.0679999999999982	0.00702452540129683\\
0.0699999999999991	-0.000646743531092999\\
0.07	-0.000646743531096385\\
0.0779999999999982	-0.0304497245417899\\
0.0799999999999991	-0.0376945518218964\\
0.08	-0.0376945518218996\\
0.087	-0.0613629561565828\\
0.0870000000000009	-0.0613629561565856\\
0.09	-0.0705722498255595\\
0.0900000000000009	-0.0705722498255621\\
0.0929999999999999	-0.0792329294508874\\
0.095999999999999	-0.0873526353855747\\
0.0999999999999991	-0.0973497076333374\\
0.1	-0.0973497076333395\\
0.104999999999999	-0.108387871323025\\
0.105	-0.108387871323027\\
0.109999999999999	-0.117699946526397\\
0.11	-0.117699946526398\\
0.114999999999999	-0.125308754745586\\
0.115999999999999	-0.126627838237935\\
0.116	-0.126627838237936\\
0.119999999999999	-0.131232943213937\\
0.12	-0.131232943213938\\
0.123999999999999	-0.13476926202664\\
0.127999999999998	-0.137242340899987\\
0.129999999999998	-0.138081442501912\\
0.13	-0.138081442501912\\
0.137999999999998	-0.138771927810782\\
0.139999999999998	-0.138276983003393\\
0.14	-0.138276983003392\\
0.144999999999998	-0.135869686909084\\
0.145	-0.135869686909083\\
0.149999999999998	-0.131785845077204\\
0.15	-0.131785845077202\\
0.154999999999998	-0.126461814037968\\
0.159999999999996	-0.120330910865585\\
0.16	-0.12033091086558\\
0.169999999999996	-0.105586372774747\\
0.17	-0.105586372774741\\
0.173999999999998	-0.0990022340863097\\
0.174	-0.0990022340863068\\
0.174999999999998	-0.0973523335757812\\
0.175	-0.0973523335757782\\
0.176	-0.0957005634817941\\
0.177	-0.0940467619077872\\
0.179000000000001	-0.0907324157456936\\
0.179999999999998	-0.0890715463112649\\
0.18	-0.0890715463112619\\
0.184000000000001	-0.0823996248083219\\
0.188000000000002	-0.0756743350020158\\
0.189999999999998	-0.0722883843683618\\
0.19	-0.0722883843683588\\
0.198000000000002	-0.058558154107107\\
0.199999999999998	-0.0550722607951608\\
0.2	-0.0550722607951577\\
0.202999999999998	-0.0499542634358041\\
0.203	-0.0499542634358012\\
0.205999999999998	-0.045090375635058\\
0.208999999999996	-0.0404763063780602\\
0.209999999999998	-0.0389930887103154\\
0.21	-0.0389930887103128\\
0.215999999999996	-0.0305025648988234\\
0.219999999999998	-0.0251966015169791\\
0.22	-0.0251966015169768\\
0.225999999999996	-0.0177484283157274\\
0.229999999999998	-0.0131121370239185\\
0.23	-0.0131121370239165\\
0.231999999999998	-0.0108899737028716\\
0.232	-0.0108899737028697\\
0.233999999999998	-0.00873062873450069\\
0.235999999999997	-0.00663325550056961\\
0.239999999999993	-0.00262115909906458\\
0.239999999999996	-0.0026211590990612\\
0.24	-0.00262115909905783\\
0.244999999999998	0.00188687770292411\\
0.245	0.00188687770292559\\
0.249999999999998	0.00568827569695383\\
0.25	0.00568827569695508\\
0.254999999999999	0.00879235129348301\\
0.259999999999997	0.0112067117758103\\
0.26	0.0112067117758117\\
0.260999999999996	0.0116103534305937\\
0.261	0.0116103534305951\\
0.262	0.0119926581405718\\
0.263	0.0123536634279823\\
0.265	0.0130119151800953\\
0.269	0.0140742429758504\\
0.269999999999997	0.0142870265310786\\
0.27	0.0142870265310794\\
0.278	0.015231929494095\\
0.279999999999996	0.0152581815612772\\
0.28	0.0152581815612772\\
0.288	0.0148381646770985\\
0.289999999999996	0.0146213487868583\\
0.29	0.0146213487868579\\
0.298	0.0133042875626112\\
0.299999999999996	0.0128619274084787\\
0.3	0.0128619274084778\\
0.308	0.01063490283314\\
0.309999999999996	0.00996265931864272\\
0.31	0.00996265931864148\\
0.314999999999997	0.00820076427699913\\
0.315	0.00820076427699786\\
0.319	0.00675620801608072\\
0.319000000000004	0.00675620801607942\\
0.319999999999996	0.00638990890522511\\
0.32	0.0063899089052238\\
0.321	0.00602147428209788\\
0.321999999999999	0.00565086803499177\\
0.323999999999998	0.00490299515937\\
0.327999999999996	0.00337957699652842\\
0.329999999999996	0.00260343440859324\\
0.33	0.00260343440859186\\
0.337999999999996	-0.000386934584574294\\
0.339999999999996	-0.00109385045098048\\
0.34	-0.00109385045098172\\
0.347999999999996	-0.00376712537690133\\
0.348	-0.0037671253769025\\
0.349999999999996	-0.00439815708383284\\
0.35	-0.00439815708383395\\
0.351999999999996	-0.00501477739830985\\
0.353999999999993	-0.00561722810735759\\
0.357999999999985	-0.00678056001263201\\
0.359999999999996	-0.00734189732970837\\
0.36	-0.00734189732970936\\
0.367999999999985	-0.00927534065693867\\
0.369999999999996	-0.00967033707065848\\
0.37	-0.00967033707065915\\
0.377	-0.0108624199510937\\
0.377000000000004	-0.0108624199510943\\
0.379999999999997	-0.011295123023204\\
0.38	-0.0112951230232045\\
0.382999999999993	-0.0116814835302473\\
0.384999999999997	-0.0119134813252346\\
0.385	-0.011913481325235\\
0.387999999999993	-0.0122233415981324\\
0.389999999999997	-0.0124046089407074\\
0.39	-0.0124046089407077\\
0.392999999999993	-0.0126387283665326\\
0.395999999999986	-0.0128276905077501\\
0.399999999999997	-0.0130096777594896\\
0.4	-0.0130096777594898\\
0.405999999999986	-0.0131344066698936\\
0.406	-0.0131344066698937\\
0.406000000000004	-0.0131344066698937\\
0.41	-0.0131194192205885\\
0.410000000000004	-0.0131194192205884\\
0.414	-0.013025910252926\\
0.417999999999996	-0.0128537331127516\\
0.419999999999997	-0.0127380644190282\\
0.42	-0.012738064419028\\
0.427999999999993	-0.0121511608627946\\
0.429999999999997	-0.0119777204409489\\
0.43	-0.0119777204409485\\
0.435	-0.0114966423981403\\
0.435000000000004	-0.01149664239814\\
0.439999999999997	-0.0109467768969577\\
0.44	-0.0109467768969573\\
0.444999999999993	-0.010355451915085\\
0.449999999999986	-0.00974989383042669\\
0.449999999999993	-0.00974989383042581\\
0.45	-0.00974989383042494\\
0.454999999999997	-0.00912861851892244\\
0.455	-0.00912861851892199\\
0.459999999999997	-0.00849010351173955\\
0.46	-0.00849010351173909\\
0.463999999999997	-0.00797875498394609\\
0.464	-0.00797875498394564\\
0.467999999999997	-0.00748042273240351\\
0.469999999999997	-0.00723589270700806\\
0.47	-0.00723589270700762\\
0.473999999999997	-0.00675562548054035\\
0.477999999999993	-0.00628645622411489\\
0.479999999999997	-0.00605580272715413\\
0.48	-0.00605580272715372\\
0.487999999999993	-0.00515729754952254\\
0.489999999999997	-0.00493825812254068\\
0.49	-0.0049382581225403\\
0.492999999999997	-0.00462164187662262\\
0.493	-0.00462164187662226\\
0.495999999999997	-0.00432559297658477\\
0.498999999999993	-0.00404985024110665\\
0.499999999999997	-0.00396240592092304\\
0.5	-0.00396240592092273\\
0.505999999999993	-0.00348408566948026\\
0.509999999999993	-0.00320878304145653\\
0.51	-0.00320878304145607\\
0.515999999999993	-0.00285520465084399\\
0.519999999999993	-0.00265634789090958\\
0.52	-0.00265634789090925\\
0.521999999999993	-0.00256785689554054\\
0.522	-0.00256785689554024\\
0.523999999999993	-0.00248661028412352\\
0.524999999999993	-0.00244869355699522\\
0.525	-0.00244869355699495\\
0.526999999999993	-0.00237825467839496\\
0.528999999999986	-0.00231498584786251\\
0.529999999999993	-0.00228603233951768\\
0.53	-0.00228603233951748\\
0.533999999999986	-0.00218245323015712\\
0.537999999999972	-0.00209610281135348\\
0.539999999999993	-0.00205934498713197\\
0.54	-0.00205934498713185\\
0.547999999999972	-0.00195477946619444\\
0.549999999999993	-0.00193920328278263\\
0.55	-0.00193920328278258\\
0.550999999999993	-0.00193299477772194\\
0.551	-0.0019329947777219\\
0.551999999999997	-0.00192783848496902\\
0.552999999999993	-0.00192373389857262\\
0.554999999999986	-0.00191867833871959\\
0.558999999999972	-0.00192117689668293\\
0.559999999999993	-0.00192442875915677\\
0.56	-0.0019244287591568\\
0.567999999999972	-0.00197140268441475\\
0.57	-0.001988401838639\\
0.570000000000007	-0.00198840183863906\\
0.577999999999979	-0.0020592072895557\\
0.579999999999993	-0.00207649525762915\\
0.58	-0.00207649525762921\\
0.587999999999972	-0.00214419694775721\\
0.589999999999993	-0.00216079300364885\\
0.59	-0.00216079300364891\\
0.594999999999993	-0.00220177812842115\\
0.595	-0.00220177812842121\\
0.599999999999993	-0.00224212195069527\\
0.6	-0.00224212195069533\\
0.604999999999993	-0.00227749864227557\\
0.608999999999993	-0.00229909690584826\\
0.609	-0.0022990969058483\\
0.609999999999993	-0.00230357020084521\\
0.61	-0.00230357020084524\\
0.610999999999997	-0.00230767391869955\\
0.611999999999993	-0.00231140846167129\\
0.613999999999986	-0.00231777145091609\\
0.617999999999972	-0.00232608101572473\\
0.619999999999993	-0.00232803084947865\\
0.62	-0.00232803084947866\\
0.627999999999972	-0.00232051978163345\\
0.629999999999993	-0.00231477188658906\\
0.63	-0.00231477188658903\\
0.637999999999972	-0.00227623216538157\\
0.637999999999993	-0.00227623216538144\\
0.638	-0.00227623216538139\\
0.639999999999993	-0.00226269131337631\\
0.64	-0.00226269131337626\\
0.641999999999993	-0.00224783108334466\\
0.643999999999986	-0.00223189833618672\\
0.647999999999971	-0.00219678988174738\\
0.649999999999993	-0.0021776004092714\\
0.65	-0.00217760040927134\\
0.657999999999971	-0.00208976029042206\\
0.659999999999993	-0.00206498665594992\\
0.66	-0.00206498665594983\\
0.664999999999993	-0.00199802433862666\\
0.665	-0.00199802433862656\\
0.666999999999993	-0.00196920055372759\\
0.667	-0.00196920055372749\\
0.668999999999993	-0.00193919501446106\\
0.669999999999993	-0.0019237454289288\\
0.67	-0.00192374542892869\\
0.671999999999993	-0.00189246372396009\\
0.673999999999986	-0.00186100751792911\\
0.677999999999972	-0.00179752219965694\\
0.679999999999993	-0.00176546819611356\\
0.68	-0.00176546819611344\\
0.687999999999972	-0.00163488227406933\\
0.689999999999993	-0.00160157931998955\\
0.69	-0.00160157931998944\\
0.695999999999993	-0.00150276358338633\\
0.696	-0.00150276358338622\\
0.699999999999993	-0.00143846754788163\\
0.7	-0.00143846754788152\\
0.703999999999993	-0.00137531848145053\\
0.707999999999986	-0.00131321734327865\\
0.709999999999993	-0.00128252924894784\\
0.71	-0.00128252924894773\\
0.717999999999986	-0.00116191364707716\\
0.719999999999993	-0.00113223482981509\\
0.72	-0.00113223482981499\\
0.724999999999993	-0.00106143079390376\\
0.725	-0.00106143079390366\\
0.729999999999993	-0.000996873375159433\\
0.730000000000001	-0.000996873375159346\\
0.734999999999994	-0.000936717601452401\\
0.735000000000001	-0.000936717601452317\\
0.739999999999994	-0.000879129290255161\\
0.740000000000001	-0.000879129290255081\\
0.744999999999994	-0.000823967307569958\\
0.749999999999987	-0.000771096465700289\\
0.750000000000001	-0.000771096465700144\\
0.753999999999993	-0.00073036215460301\\
0.754	-0.000730362154602938\\
0.757999999999993	-0.000690947265248271\\
0.759999999999993	-0.000671715205610819\\
0.76	-0.000671715205610751\\
0.763999999999993	-0.000635005393658975\\
0.767999999999986	-0.00060114792352203\\
0.77	-0.000585272049199614\\
0.770000000000007	-0.000585272049199559\\
0.777999999999993	-0.000528645554914799\\
0.779999999999993	-0.000516180430556586\\
0.78	-0.000516180430556543\\
0.782999999999993	-0.000498598482522057\\
0.783	-0.000498598482522017\\
0.785999999999993	-0.000482235829801109\\
0.788999999999986	-0.000467078036854374\\
0.79	-0.000462290844231463\\
0.790000000000007	-0.000462290844231429\\
0.795999999999993	-0.000436322274410771\\
0.8	-0.000421602903029317\\
0.800000000000007	-0.000421602903029293\\
0.804999999999993	-0.000405445605427315\\
0.805	-0.000405445605427293\\
0.809999999999987	-0.00039117882537358\\
0.809999999999997	-0.000391178825373552\\
0.810000000000007	-0.000391178825373524\\
0.811999999999993	-0.00038599347258466\\
0.812	-0.000385993472584642\\
0.813999999999987	-0.00038110292454074\\
0.815999999999973	-0.000376505263846724\\
0.819999999999945	-0.000368181508555571\\
0.819999999999987	-0.00036818150855549\\
0.82	-0.000368181508555463\\
0.827999999999944	-0.00035497694563609\\
0.829999999999993	-0.000352385372545459\\
0.830000000000001	-0.00035238537254545\\
0.837999999999945	-0.000344115707971929\\
0.839999999999993	-0.000342523970614159\\
0.84	-0.000342523970614154\\
0.840999999999993	-0.000341798980015613\\
0.841000000000001	-0.000341798980015608\\
0.841999999999997	-0.000341121145212625\\
0.842999999999993	-0.000340490399724983\\
0.844999999999986	-0.000339369934026441\\
0.848999999999972	-0.000337691662864691\\
0.849999999999993	-0.000337389069357229\\
0.85	-0.000337389069357227\\
0.857999999999972	-0.000335259240190865\\
0.859999999999993	-0.000334758414659414\\
0.86	-0.000334758414659412\\
0.867999999999972	-0.000332875747277424\\
0.869999999999993	-0.000332434316465353\\
0.87	-0.000332434316465352\\
0.874999999999994	-0.000331068242922347\\
0.875000000000001	-0.000331068242922344\\
0.879999999999994	-0.000329146222069652\\
0.880000000000001	-0.000329146222069649\\
0.884999999999994	-0.000326663543542365\\
0.889999999999987	-0.000323614122940048\\
0.890000000000001	-0.000323614122940039\\
0.890000000000008	-0.000323614122940034\\
0.899	-0.000316486733966596\\
0.899000000000008	-0.000316486733966589\\
0.9	-0.000315554720433605\\
0.900000000000007	-0.000315554720433599\\
0.901000000000004	-0.000314594375941841\\
0.902	-0.000313605606321269\\
0.903999999999993	-0.0003115424012594\\
0.907999999999979	-0.00030707043771199\\
0.909999999999993	-0.00030465992587264\\
0.910000000000001	-0.000304659925872631\\
0.917999999999972	-0.000293831753429642\\
0.919999999999993	-0.000290822872510042\\
0.920000000000001	-0.000290822872510032\\
0.927999999999972	-0.000278102617137219\\
0.927999999999994	-0.000278102617137184\\
0.928000000000001	-0.000278102617137172\\
0.929999999999994	-0.000274779995718834\\
0.930000000000001	-0.000274779995718822\\
0.931999999999994	-0.000271402704015578\\
0.933999999999987	-0.000267974341572245\\
0.937999999999972	-0.000260959007541334\\
0.939999999999993	-0.000257369285387744\\
0.940000000000001	-0.000257369285387731\\
0.944999999999994	-0.000248144342019191\\
0.945000000000001	-0.000248144342019178\\
0.949999999999994	-0.000238543151774327\\
0.950000000000001	-0.000238543151774313\\
0.954999999999994	-0.000228542184423157\\
0.956999999999994	-0.000224424403949383\\
0.957000000000001	-0.000224424403949368\\
0.959999999999993	-0.000218116930134359\\
0.960000000000001	-0.000218116930134344\\
0.962999999999993	-0.000211787464606918\\
0.965999999999986	-0.000205570352027769\\
0.969999999999993	-0.000197446174189792\\
0.970000000000001	-0.000197446174189778\\
0.975999999999986	-0.000185588360519031\\
0.979999999999993	-0.000177884039190569\\
0.980000000000001	-0.000177884039190555\\
0.985999999999986	-0.000166598677044058\\
0.985999999999993	-0.000166598677044044\\
0.986000000000001	-0.000166598677044031\\
0.989999999999993	-0.000159238640787061\\
0.990000000000001	-0.000159238640787048\\
0.993999999999993	-0.000152203250136343\\
0.997999999999986	-0.000145688781985193\\
1	-0.000142623669560464\\
};
\addplot [color=mycolor2,solid,forget plot]
  table[row sep=crcr]{%
0	0.15313\\
3.15544362088405e-30	0.15313\\
0.000656101980281985	0.153131614989962\\
0.00393661188169191	0.153188143215565\\
0.00999999999999994	0.153505321610126\\
0.01	0.153505321610126\\
0.0199999999999999	0.154633126647887\\
0.02	0.154633126647887\\
0.0289999999999998	0.153421137821655\\
0.029	0.153421137821655\\
0.03	0.152969434437754\\
0.0300000000000002	0.152969434437754\\
0.0349999999999996	0.14975837658894\\
0.035	0.14975837658894\\
0.0399999999999993	0.144956580293631\\
0.04	0.14495658029363\\
0.0449999999999993	0.138558162777922\\
0.0499999999999987	0.130555285180189\\
0.05	0.130555285180187\\
0.0500000000000004	0.130555285180186\\
0.0579999999999996	0.115637633000987\\
0.058	0.115637633000986\\
0.0599999999999996	0.111647492594861\\
0.06	0.11164749259486\\
0.0619999999999995	0.107551527943661\\
0.0639999999999991	0.103348936174503\\
0.0679999999999982	0.0946205553357658\\
0.0699999999999991	0.090093055415954\\
0.07	0.0900930554159519\\
0.0779999999999982	0.0708733669527515\\
0.0799999999999991	0.0657863137092437\\
0.08	0.0657863137092414\\
0.087	0.0470707264257497\\
0.0870000000000009	0.0470707264257473\\
0.09	0.038608115811774\\
0.0900000000000009	0.0386081158117714\\
0.0929999999999999	0.0301187078522066\\
0.095999999999999	0.0218422822023367\\
0.0999999999999991	0.0111320584192047\\
0.1	0.0111320584192024\\
0.104999999999999	-0.0017451412647055\\
0.105	-0.00174514126470775\\
0.109999999999999	-0.0140697219629554\\
0.11	-0.0140697219629576\\
0.114999999999999	-0.0258567828338202\\
0.115999999999999	-0.028150974796736\\
0.116	-0.0281509747967381\\
0.119999999999999	-0.0371207645085133\\
0.12	-0.0371207645085153\\
0.123999999999999	-0.045606556121459\\
0.127999999999998	-0.0534569316870435\\
0.129999999999998	-0.0571457813072444\\
0.13	-0.0571457813072476\\
0.137999999999998	-0.0703421052010178\\
0.139999999999998	-0.0732546507332732\\
0.14	-0.0732546507332757\\
0.144999999999998	-0.0798659456623785\\
0.145	-0.0798659456623807\\
0.149999999999998	-0.0855263385064872\\
0.15	-0.0855263385064891\\
0.154999999999998	-0.0902427640154369\\
0.159999999999996	-0.0940210004002968\\
0.16	-0.0940210004002991\\
0.169999999999996	-0.0986317226348858\\
0.17	-0.0986317226348867\\
0.173999999999998	-0.0993526370731427\\
0.174	-0.0993526370731429\\
0.174999999999998	-0.0994327300022726\\
0.175	-0.0994327300022727\\
0.176	-0.0994727794614947\\
0.177	-0.0994727874134758\\
0.179000000000001	-0.0993526768352302\\
0.179999999999998	-0.0992325524195822\\
0.18	-0.099232552419582\\
0.184000000000001	-0.0985283580974968\\
0.188000000000002	-0.0975365687988356\\
0.189999999999998	-0.0969325918476496\\
0.19	-0.096932591847649\\
0.198000000000002	-0.093793207822887\\
0.199999999999998	-0.0928267236024457\\
0.2	-0.0928267236024448\\
0.202999999999998	-0.0912400352354355\\
0.203	-0.0912400352354345\\
0.205999999999998	-0.0894883650625657\\
0.208999999999996	-0.0875709405558083\\
0.209999999999998	-0.0868948207118116\\
0.21	-0.0868948207118103\\
0.215999999999996	-0.0824474621646144\\
0.219999999999998	-0.0791078050053071\\
0.22	-0.0791078050053055\\
0.225999999999996	-0.0738479984213533\\
0.229999999999998	-0.0703120493504639\\
0.23	-0.0703120493504624\\
0.231999999999998	-0.0685342136045761\\
0.232	-0.0685342136045745\\
0.233999999999998	-0.0667493395513993\\
0.235999999999997	-0.0649570773420509\\
0.239999999999993	-0.0613489817813503\\
0.239999999999996	-0.0613489817813471\\
0.24	-0.061348981781344\\
0.244999999999998	-0.0567910135981784\\
0.245	-0.0567910135981768\\
0.249999999999998	-0.0521746653123061\\
0.25	-0.0521746653123044\\
0.254999999999999	-0.0476242423318699\\
0.259999999999997	-0.0432641308592881\\
0.26	-0.0432641308592851\\
0.260999999999996	-0.0424144718283425\\
0.261	-0.0424144718283395\\
0.262	-0.0415721692525926\\
0.263	-0.0407371818383915\\
0.265	-0.0390889891928653\\
0.269	-0.0358786093708109\\
0.269999999999997	-0.0350937022935772\\
0.27	-0.0350937022935744\\
0.278	-0.0290632043510933\\
0.279999999999996	-0.0276233282023782\\
0.28	-0.0276233282023757\\
0.288	-0.0221263490325744\\
0.289999999999996	-0.0208163887975932\\
0.29	-0.0208163887975909\\
0.298	-0.0160570438801743\\
0.299999999999996	-0.0150006033982667\\
0.3	-0.0150006033982649\\
0.308	-0.0113022115444085\\
0.309999999999996	-0.010508549962371\\
0.31	-0.0105085499623696\\
0.314999999999997	-0.00875179114672262\\
0.315	-0.00875179114672148\\
0.319	-0.00757916087479443\\
0.319000000000004	-0.00757916087479348\\
0.319999999999996	-0.00731820843277085\\
0.32	-0.00731820843276995\\
0.321	-0.00706827602449595\\
0.321999999999999	-0.0068275159056266\\
0.323999999999998	-0.00637346579579224\\
0.327999999999996	-0.00557494401156089\\
0.329999999999996	-0.00523031581749665\\
0.33	-0.00523031581749607\\
0.337999999999996	-0.00421452794011216\\
0.339999999999996	-0.00405101331592819\\
0.34	-0.00405101331592793\\
0.347999999999996	-0.00375772258736748\\
0.348	-0.00375772258736748\\
0.349999999999996	-0.00377451998418073\\
0.35	-0.00377451998418079\\
0.351999999999996	-0.00381647679652969\\
0.353999999999993	-0.0038727166390681\\
0.357999999999985	-0.00402809230915985\\
0.359999999999996	-0.00412725859153486\\
0.36	-0.00412725859153505\\
0.367999999999985	-0.00466759973152374\\
0.369999999999996	-0.00483873640420929\\
0.37	-0.00483873640420961\\
0.377	-0.0055520172634314\\
0.377000000000004	-0.00555201726343181\\
0.379999999999997	-0.0059124410860251\\
0.38	-0.00591244108602555\\
0.382999999999993	-0.00628759771240643\\
0.384999999999997	-0.00653578599841538\\
0.385	-0.00653578599841582\\
0.387999999999993	-0.00690531584631486\\
0.389999999999997	-0.00714990444139671\\
0.39	-0.00714990444139714\\
0.392999999999993	-0.0075142602310904\\
0.395999999999986	-0.00787572623178794\\
0.399999999999997	-0.00835346102560518\\
0.4	-0.0083534610256056\\
0.405999999999986	-0.00906178009991046\\
0.406	-0.00906178009991213\\
0.406000000000004	-0.00906178009991255\\
0.41	-0.00952901067364299\\
0.410000000000004	-0.00952901067364341\\
0.414	-0.00997687415267218\\
0.417999999999996	-0.0103899124737084\\
0.419999999999997	-0.0105834744637126\\
0.42	-0.010583474463713\\
0.427999999999993	-0.0112722069208244\\
0.429999999999997	-0.0114231799044292\\
0.43	-0.0114231799044295\\
0.435	-0.011763825106213\\
0.435000000000004	-0.0117638251062133\\
0.439999999999997	-0.0120522432341001\\
0.44	-0.0120522432341002\\
0.444999999999993	-0.0122887876395593\\
0.449999999999986	-0.0124737481190688\\
0.449999999999993	-0.012473748119069\\
0.45	-0.0124737481190693\\
0.454999999999997	-0.0126006355076542\\
0.455	-0.0126006355076543\\
0.459999999999997	-0.012662889470727\\
0.46	-0.0126628894707271\\
0.463999999999997	-0.0126662104915079\\
0.464	-0.0126662104915079\\
0.467999999999997	-0.012628219059769\\
0.469999999999997	-0.0125937231314455\\
0.47	-0.0125937231314454\\
0.473999999999997	-0.0124936970420011\\
0.477999999999993	-0.0123522232406856\\
0.479999999999997	-0.0122659100556399\\
0.48	-0.0122659100556397\\
0.487999999999993	-0.0118547784292342\\
0.489999999999997	-0.0117378233332886\\
0.49	-0.0117378233332884\\
0.492999999999997	-0.0115516712471934\\
0.493	-0.0115516712471932\\
0.495999999999997	-0.0113525824375043\\
0.498999999999993	-0.0111404691009665\\
0.499999999999997	-0.0110668543137872\\
0.5	-0.0110668543137869\\
0.505999999999993	-0.0105943507351475\\
0.509999999999993	-0.0102497139092464\\
0.51	-0.0102497139092458\\
0.515999999999993	-0.00968763143281645\\
0.519999999999993	-0.00928239649471308\\
0.52	-0.00928239649471234\\
0.521999999999993	-0.00907396621485642\\
0.522	-0.00907396621485568\\
0.523999999999993	-0.00886621779288999\\
0.524999999999993	-0.0087625865484552\\
0.525	-0.00876258654845447\\
0.526999999999993	-0.00855578460269177\\
0.528999999999986	-0.00834956294861666\\
0.529999999999993	-0.00824665709535903\\
0.53	-0.0082466570953583\\
0.533999999999986	-0.00783628259099601\\
0.537999999999972	-0.007427665240691\\
0.539999999999993	-0.00722391533061898\\
0.54	-0.00722391533061826\\
0.547999999999972	-0.00641170964803699\\
0.549999999999993	-0.00620915772116902\\
0.55	-0.0062091577211683\\
0.550999999999993	-0.00610839228405739\\
0.551	-0.00610839228405667\\
0.551999999999997	-0.00600857782898421\\
0.552999999999993	-0.00590970946386365\\
0.554999999999986	-0.00571479167132077\\
0.558999999999972	-0.00533609908161176\\
0.559999999999993	-0.00524372026360131\\
0.56	-0.00524372026360066\\
0.567999999999972	-0.00453702881423189\\
0.57	-0.00436918072026969\\
0.570000000000007	-0.0043691807202691\\
0.577999999999979	-0.00373210030225669\\
0.579999999999993	-0.00358125209794317\\
0.58	-0.00358125209794264\\
0.587999999999972	-0.00301944684051335\\
0.589999999999993	-0.002889803828792\\
0.59	-0.00288980382879155\\
0.594999999999993	-0.002584329428027\\
0.595	-0.00258432942802658\\
0.599999999999993	-0.00230517829132263\\
0.6	-0.00230517829132225\\
0.604999999999993	-0.00205200842116886\\
0.608999999999993	-0.00186796954658788\\
0.609	-0.00186796954658757\\
0.609999999999993	-0.0018245096526498\\
0.61	-0.00182450965264949\\
0.610999999999997	-0.00178199627320724\\
0.611999999999993	-0.00174035824342754\\
0.613999999999986	-0.00165970011456107\\
0.617999999999972	-0.00150880114248635\\
0.619999999999993	-0.00143853072140514\\
0.62	-0.00143853072140489\\
0.627999999999972	-0.00119166260307253\\
0.629999999999993	-0.00113843848024014\\
0.63	-0.00113843848023996\\
0.637999999999972	-0.000959200605772057\\
0.637999999999993	-0.000959200605771646\\
0.638	-0.000959200605771511\\
0.639999999999993	-0.000922761876475018\\
0.64	-0.000922761876474895\\
0.641999999999993	-0.000889656176771994\\
0.643999999999986	-0.000859877017096072\\
0.647999999999971	-0.000810275621453308\\
0.649999999999993	-0.000790443662997569\\
0.65	-0.000790443662997505\\
0.657999999999971	-0.000737416471409183\\
0.659999999999993	-0.000730300623093721\\
0.66	-0.0007303006230937\\
0.664999999999993	-0.000723240435327344\\
0.665	-0.000723240435327345\\
0.666999999999993	-0.000724706680384332\\
0.667	-0.000724706680384341\\
0.668999999999993	-0.0007286246977068\\
0.669999999999993	-0.000731503342169344\\
0.67	-0.000731503342169367\\
0.671999999999993	-0.000739100607974611\\
0.673999999999986	-0.000749152467286245\\
0.677999999999972	-0.000776628328711077\\
0.679999999999993	-0.000794057716305854\\
0.68	-0.00079405771630592\\
0.687999999999972	-0.000873105436707512\\
0.689999999999993	-0.000894259205442502\\
0.69	-0.000894259205442578\\
0.695999999999993	-0.000961126830509811\\
0.696	-0.000961126830509893\\
0.699999999999993	-0.00100858781550804\\
0.7	-0.00100858781550812\\
0.703999999999993	-0.00105839800415164\\
0.707999999999986	-0.00111059645024464\\
0.709999999999993	-0.00113760398563549\\
0.71	-0.00113760398563559\\
0.717999999999986	-0.0012413546296336\\
0.719999999999993	-0.00126559425618062\\
0.72	-0.00126559425618071\\
0.724999999999993	-0.00132327427264495\\
0.725	-0.00132327427264502\\
0.729999999999993	-0.00137684013859725\\
0.730000000000001	-0.00137684013859733\\
0.734999999999994	-0.00142635747936927\\
0.735000000000001	-0.00142635747936934\\
0.739999999999994	-0.00147188695989593\\
0.740000000000001	-0.001471886959896\\
0.744999999999994	-0.00151348435948468\\
0.749999999999987	-0.00155120064015199\\
0.750000000000001	-0.00155120064015209\\
0.753999999999993	-0.00157735379664861\\
0.754	-0.00157735379664865\\
0.757999999999993	-0.00159855985748476\\
0.759999999999993	-0.00160731307452895\\
0.76	-0.00160731307452897\\
0.763999999999993	-0.00162112846046538\\
0.767999999999986	-0.00163003107373084\\
0.77	-0.00163264242228832\\
0.770000000000007	-0.00163264242228833\\
0.777999999999993	-0.00163402803412833\\
0.779999999999993	-0.00163230962377641\\
0.78	-0.0016323096237764\\
0.782999999999993	-0.00162818244094053\\
0.783	-0.00162818244094052\\
0.785999999999993	-0.00162219409866477\\
0.788999999999986	-0.00161434195603207\\
0.79	-0.00161130982246483\\
0.790000000000007	-0.0016113098224648\\
0.795999999999993	-0.0015887511829903\\
0.8	-0.00156954007733136\\
0.800000000000007	-0.00156954007733132\\
0.804999999999993	-0.00154080649094398\\
0.805	-0.00154080649094393\\
0.809999999999987	-0.00150679563310058\\
0.809999999999997	-0.0015067956331005\\
0.810000000000007	-0.00150679563310043\\
0.811999999999993	-0.0014919412420881\\
0.812	-0.00149194124208805\\
0.813999999999987	-0.00147670638900483\\
0.815999999999973	-0.00146108808774405\\
0.819999999999945	-0.0014286888199843\\
0.819999999999987	-0.00142868881998395\\
0.82	-0.00142868881998383\\
0.827999999999944	-0.0013591491013104\\
0.829999999999993	-0.00134075666164557\\
0.830000000000001	-0.00134075666164551\\
0.837999999999945	-0.00126304869651438\\
0.839999999999993	-0.00124256811471934\\
0.84	-0.00124256811471927\\
0.840999999999993	-0.00123222570545369\\
0.841000000000001	-0.00123222570545361\\
0.841999999999997	-0.00122189271573142\\
0.842999999999993	-0.0012115686391217\\
0.844999999999986	-0.00119094520212441\\
0.848999999999972	-0.00114978305289414\\
0.849999999999993	-0.00113950722460618\\
0.85	-0.00113950722460611\\
0.857999999999972	-0.00105743389112843\\
0.859999999999993	-0.0010369341522691\\
0.86	-0.00103693415226903\\
0.867999999999972	-0.000954889015545967\\
0.869999999999993	-0.000934346084458745\\
0.87	-0.000934346084458672\\
0.874999999999994	-0.000882889345164364\\
0.875000000000001	-0.000882889345164291\\
0.879999999999994	-0.000831240134318731\\
0.880000000000001	-0.000831240134318657\\
0.884999999999994	-0.000780901096501069\\
0.889999999999987	-0.000733376481736367\\
0.890000000000001	-0.000733376481736241\\
0.890000000000008	-0.000733376481736175\\
0.899	-0.000654740828904449\\
0.899000000000008	-0.000654740828904391\\
0.9	-0.000646541003483311\\
0.900000000000007	-0.000646541003483253\\
0.901000000000004	-0.000638447160425065\\
0.902	-0.000630458902985742\\
0.903999999999993	-0.000614797584413385\\
0.907999999999979	-0.000584725096243026\\
0.909999999999993	-0.000570308032111906\\
0.910000000000001	-0.000570308032111856\\
0.917999999999972	-0.000516911344203481\\
0.919999999999993	-0.000504630022639575\\
0.920000000000001	-0.000504630022639532\\
0.927999999999972	-0.000459703911601104\\
0.927999999999994	-0.000459703911600993\\
0.928000000000001	-0.000459703911600956\\
0.929999999999994	-0.000449511170401967\\
0.930000000000001	-0.000449511170401931\\
0.931999999999994	-0.000439729826664271\\
0.933999999999987	-0.000430357963039267\\
0.937999999999972	-0.000412835408365774\\
0.939999999999993	-0.000404681282702018\\
0.940000000000001	-0.00040468128270199\\
0.944999999999994	-0.000386054087615957\\
0.945000000000001	-0.000386054087615933\\
0.949999999999994	-0.000369920603371119\\
0.950000000000001	-0.000369920603371098\\
0.954999999999994	-0.000356151362573767\\
0.956999999999994	-0.000351271066955804\\
0.957000000000001	-0.000351271066955787\\
0.959999999999993	-0.000344619794471207\\
0.960000000000001	-0.000344619794471192\\
0.962999999999993	-0.000338768945938035\\
0.965999999999986	-0.00033371594102731\\
0.969999999999993	-0.000328215890113132\\
0.970000000000001	-0.000328215890113124\\
0.975999999999986	-0.000321412951872142\\
0.979999999999993	-0.000317305783940397\\
0.980000000000001	-0.00031730578394039\\
0.985999999999986	-0.000311779167283901\\
0.985999999999993	-0.000311779167283894\\
0.986000000000001	-0.000311779167283888\\
0.989999999999993	-0.000308513300192605\\
0.990000000000001	-0.0003085133001926\\
0.993999999999993	-0.000305579242748409\\
0.997999999999986	-0.000302974694493122\\
1	-0.000301795338108548\\
};
\addplot [color=mycolor3,solid,forget plot]
  table[row sep=crcr]{%
0	0.10153\\
3.15544362088405e-30	0.10153\\
0.000656101980281985	0.101530709989553\\
0.00393661188169191	0.101555560666546\\
0.00999999999999994	0.101694978093407\\
0.01	0.101694978093407\\
0.0199999999999999	0.102190448599525\\
0.02	0.102190448599525\\
0.0289999999999998	0.10292025163065\\
0.029	0.10292025163065\\
0.03	0.103018021716247\\
0.0300000000000002	0.103018021716247\\
0.0349999999999996	0.103192147327812\\
0.035	0.103192147327812\\
0.0399999999999993	0.102720100364795\\
0.04	0.102720100364795\\
0.0449999999999993	0.101601497399549\\
0.0499999999999987	0.0998354297964996\\
0.05	0.099835429796499\\
0.0500000000000004	0.0998354297964988\\
0.0579999999999996	0.0956592171978746\\
0.058	0.0956592171978743\\
0.0599999999999996	0.0943546352961654\\
0.06	0.0943546352961651\\
0.0619999999999995	0.0929455231467101\\
0.0639999999999991	0.0914316976162889\\
0.0679999999999982	0.0880891058325206\\
0.0699999999999991	0.0862599051673818\\
0.07	0.086259905167381\\
0.0779999999999982	0.0783631580954561\\
0.0799999999999991	0.0762726013656067\\
0.08	0.0762726013656058\\
0.087	0.0685830418499841\\
0.0870000000000009	0.0685830418499831\\
0.09	0.065107935293602\\
0.0900000000000009	0.065107935293601\\
0.0929999999999999	0.0615236275464586\\
0.095999999999999	0.0578290704987055\\
0.0999999999999991	0.0527296235853229\\
0.1	0.0527296235853218\\
0.104999999999999	0.0460729370657436\\
0.105	0.0460729370657424\\
0.109999999999999	0.0390974387049891\\
0.11	0.0390974387049878\\
0.114999999999999	0.0320420412716109\\
0.115999999999999	0.0306502167315683\\
0.116	0.0306502167315671\\
0.119999999999999	0.0251455926314253\\
0.12	0.0251455926314241\\
0.123999999999999	0.0197391081211328\\
0.127999999999998	0.0144279526059265\\
0.129999999999998	0.011807258483974\\
0.13	0.0118072584839717\\
0.137999999999998	0.00154912692971668\\
0.139999999999998	-0.000960911168698507\\
0.14	-0.000960911168700728\\
0.144999999999998	-0.00705216971721133\\
0.145	-0.00705216971721343\\
0.149999999999998	-0.0128321349015648\\
0.15	-0.0128321349015668\\
0.154999999999998	-0.0183055017781117\\
0.159999999999996	-0.0234767163352839\\
0.16	-0.0234767163352874\\
0.169999999999996	-0.0329292486698219\\
0.17	-0.032929248669825\\
0.173999999999998	-0.0363834968265353\\
0.174	-0.0363834968265368\\
0.174999999999998	-0.0372182447500583\\
0.175	-0.0372182447500598\\
0.176	-0.0380415218348881\\
0.177	-0.0388533548293884\\
0.179000000000001	-0.0404427936812492\\
0.179999999999998	-0.0412204511793037\\
0.18	-0.0412204511793051\\
0.184000000000001	-0.0442179202275956\\
0.188000000000002	-0.0470354872327416\\
0.189999999999998	-0.0483772688386875\\
0.19	-0.0483772688386887\\
0.198000000000002	-0.0531152697228068\\
0.199999999999998	-0.0541316071613497\\
0.2	-0.0541316071613506\\
0.202999999999998	-0.0555305499846143\\
0.203	-0.0555305499846151\\
0.205999999999998	-0.0567791771522262\\
0.208999999999996	-0.057877853791348\\
0.209999999999998	-0.0582108137877957\\
0.21	-0.0582108137877963\\
0.215999999999996	-0.0598600144100443\\
0.219999999999998	-0.0606281454731618\\
0.22	-0.0606281454731621\\
0.225999999999996	-0.0612843363763174\\
0.229999999999998	-0.0613914581691176\\
0.23	-0.0613914581691176\\
0.231999999999998	-0.0613569673420705\\
0.232	-0.0613569673420704\\
0.233999999999998	-0.0612784599866984\\
0.235999999999997	-0.061155925900476\\
0.239999999999993	-0.0607787081136675\\
0.239999999999996	-0.0607787081136671\\
0.24	-0.0607787081136667\\
0.244999999999998	-0.0600591304216388\\
0.245	-0.0600591304216385\\
0.249999999999998	-0.0590633795606718\\
0.25	-0.0590633795606714\\
0.254999999999999	-0.0577906466877398\\
0.259999999999997	-0.0562398979683205\\
0.26	-0.0562398979683193\\
0.260999999999996	-0.0558962794551987\\
0.261	-0.0558962794551974\\
0.262	-0.0555414794859283\\
0.263	-0.0551754865306404\\
0.265	-0.0544098737339776\\
0.269	-0.0527437846910428\\
0.269999999999997	-0.0522990874689763\\
0.27	-0.0522990874689747\\
0.278	-0.0486075012896723\\
0.279999999999996	-0.0476563426766285\\
0.28	-0.0476563426766268\\
0.288	-0.0437349511915884\\
0.289999999999996	-0.042724777029565\\
0.29	-0.0427247770295632\\
0.298	-0.0385608363448714\\
0.299999999999996	-0.037488363729724\\
0.3	-0.0374883637297221\\
0.308	-0.0332378303916068\\
0.309999999999996	-0.0321949371574688\\
0.31	-0.032194937157467\\
0.314999999999997	-0.029620762633946\\
0.315	-0.0296207626339442\\
0.319	-0.0275943241006294\\
0.319000000000004	-0.0275943241006276\\
0.319999999999996	-0.0270921464080909\\
0.32	-0.0270921464080891\\
0.321	-0.0265917087750128\\
0.321999999999999	-0.0260929949421357\\
0.323999999999998	-0.0251006739202311\\
0.327999999999996	-0.0231360082320983\\
0.329999999999996	-0.0221634082319868\\
0.33	-0.0221634082319851\\
0.337999999999996	-0.0183349540470543\\
0.339999999999996	-0.0173927050188045\\
0.34	-0.0173927050188028\\
0.347999999999996	-0.0137940267762798\\
0.348	-0.0137940267762782\\
0.349999999999996	-0.0129435092749043\\
0.35	-0.0129435092749028\\
0.351999999999996	-0.0121124277201381\\
0.353999999999993	-0.0113006740988322\\
0.357999999999985	-0.00973473117336946\\
0.359999999999996	-0.00898033835435596\\
0.36	-0.00898033835435464\\
0.367999999999985	-0.00615103237994316\\
0.369999999999996	-0.00549031258752466\\
0.37	-0.0054903125875235\\
0.377	-0.00332268163643367\\
0.377000000000004	-0.00332268163643262\\
0.379999999999997	-0.00246208994964999\\
0.38	-0.002462089949649\\
0.382999999999993	-0.00164216481576596\\
0.384999999999997	-0.00111802170159374\\
0.385	-0.00111802170159282\\
0.387999999999993	-0.000365347197260713\\
0.389999999999997	0.000114170794730868\\
0.39	0.000114170794731703\\
0.392999999999993	0.000793425379057254\\
0.395999999999986	0.00141940400203702\\
0.399999999999997	0.00217148258534534\\
0.4	0.00217148258534597\\
0.405999999999986	0.00312346260189153\\
0.406	0.00312346260189354\\
0.406000000000004	0.00312346260189404\\
0.41	0.00364117185530289\\
0.410000000000004	0.00364117185530331\\
0.414	0.00406564691138126\\
0.417999999999996	0.00439710843664848\\
0.419999999999997	0.0045280148603674\\
0.42	0.00452801486036761\\
0.427999999999993	0.00481984009384359\\
0.429999999999997	0.00483489369551712\\
0.43	0.00483489369551712\\
0.435	0.00480207239530616\\
0.435000000000004	0.00480207239530611\\
0.439999999999997	0.00468619349116265\\
0.44	0.00468619349116254\\
0.444999999999993	0.00448716285586529\\
0.449999999999986	0.0042048188178285\\
0.449999999999993	0.00420481881782804\\
0.45	0.00420481881782758\\
0.454999999999997	0.00383893202858688\\
0.455	0.00383893202858659\\
0.459999999999997	0.00338920528286627\\
0.46	0.00338920528286592\\
0.463999999999997	0.00296881592364295\\
0.464	0.00296881592364255\\
0.467999999999997	0.00249431796400048\\
0.469999999999997	0.00223670227654669\\
0.47	0.00223670227654622\\
0.473999999999997	0.00170503456494174\\
0.477999999999993	0.00116753023061576\\
0.479999999999997	0.000896502130446468\\
0.48	0.000896502130445985\\
0.487999999999993	-0.000203607417958384\\
0.489999999999997	-0.000482812805593269\\
0.49	-0.000482812805593767\\
0.492999999999997	-0.000904901238874969\\
0.493	-0.000904901238875472\\
0.495999999999997	-0.00133103588832417\\
0.498999999999993	-0.00176134136389649\\
0.499999999999997	-0.00190572508498463\\
0.5	-0.00190572508498514\\
0.505999999999993	-0.00275260833343492\\
0.509999999999993	-0.00329432190879493\\
0.51	-0.00329432190879587\\
0.515999999999993	-0.00407328344997044\\
0.519999999999993	-0.0045705789638816\\
0.52	-0.00457057896388246\\
0.521999999999993	-0.0048127208643742\\
0.522	-0.00481272086437505\\
0.523999999999993	-0.00505056767222024\\
0.524999999999993	-0.0051678900978277\\
0.525	-0.00516789009782853\\
0.526999999999993	-0.00539935205137182\\
0.528999999999986	-0.00562659515125581\\
0.529999999999993	-0.00573864388133106\\
0.53	-0.00573864388133186\\
0.533999999999986	-0.00617643788416967\\
0.537999999999972	-0.00659775989691432\\
0.539999999999993	-0.00680231268562629\\
0.54	-0.00680231268562701\\
0.547999999999972	-0.00755442208911304\\
0.549999999999993	-0.00772442179205999\\
0.55	-0.00772442179206058\\
0.550999999999993	-0.00780673324749164\\
0.551	-0.00780673324749222\\
0.551999999999997	-0.00788725602367983\\
0.552999999999993	-0.00796599273723251\\
0.554999999999986	-0.00811811815101259\\
0.558999999999972	-0.00840104490687427\\
0.559999999999993	-0.00846734757065569\\
0.56	-0.00846734757065616\\
0.567999999999972	-0.00893431802287783\\
0.57	-0.00903350441148642\\
0.570000000000007	-0.00903350441148676\\
0.577999999999979	-0.0093604118139374\\
0.579999999999993	-0.00942473223335562\\
0.58	-0.00942473223335584\\
0.587999999999972	-0.00961263898270645\\
0.589999999999993	-0.00964230246399086\\
0.59	-0.00964230246399095\\
0.594999999999993	-0.00968719012414359\\
0.595	-0.00968719012414363\\
0.599999999999993	-0.00969084781869305\\
0.6	-0.00969084781869302\\
0.604999999999993	-0.00965327851924945\\
0.608999999999993	-0.00959352026206663\\
0.609	-0.0095935202620665\\
0.609999999999993	-0.00957445170845045\\
0.61	-0.00957445170845031\\
0.610999999999997	-0.00955373040056255\\
0.611999999999993	-0.0095313556651646\\
0.613999999999986	-0.00948164295030244\\
0.617999999999972	-0.00936234080248655\\
0.619999999999993	-0.00929273586475684\\
0.62	-0.00929273586475658\\
0.627999999999972	-0.00894773586386479\\
0.629999999999993	-0.00884478475601719\\
0.63	-0.00884478475601681\\
0.637999999999972	-0.00839861878260049\\
0.637999999999993	-0.00839861878259923\\
0.638	-0.00839861878259882\\
0.639999999999993	-0.00828046686037032\\
0.64	-0.0082804668603699\\
0.641999999999993	-0.00815964057471369\\
0.643999999999986	-0.00803612422239629\\
0.647999999999971	-0.00778095675744499\\
0.649999999999993	-0.00764927248252216\\
0.65	-0.00764927248252169\\
0.657999999999971	-0.00709479492220257\\
0.659999999999993	-0.00694915034179599\\
0.66	-0.00694915034179547\\
0.664999999999993	-0.00658266984273853\\
0.665	-0.00658266984273801\\
0.666999999999993	-0.00643670081249966\\
0.667	-0.00643670081249914\\
0.668999999999993	-0.00629105939838003\\
0.669999999999993	-0.00621835563086651\\
0.67	-0.00621835563086599\\
0.671999999999993	-0.00607317016424685\\
0.673999999999986	-0.00592826506924562\\
0.677999999999972	-0.00563922070073368\\
0.679999999999993	-0.00549504386215189\\
0.68	-0.00549504386215137\\
0.687999999999972	-0.00492020190943284\\
0.689999999999993	-0.00477686438809373\\
0.69	-0.00477686438809322\\
0.695999999999993	-0.00434744880791587\\
0.696	-0.00434744880791536\\
0.699999999999993	-0.00406148324064083\\
0.7	-0.00406148324064033\\
0.703999999999993	-0.00378043682998459\\
0.707999999999986	-0.00350900689331351\\
0.709999999999993	-0.00337685381341326\\
0.71	-0.00337685381341279\\
0.717999999999986	-0.00287159023084518\\
0.719999999999993	-0.00275102943883562\\
0.72	-0.00275102943883519\\
0.724999999999993	-0.00245952629397335\\
0.725	-0.00245952629397294\\
0.729999999999993	-0.00218197628778351\\
0.730000000000001	-0.00218197628778313\\
0.734999999999994	-0.00191815396685727\\
0.735000000000001	-0.0019181539668569\\
0.739999999999994	-0.0016678450296479\\
0.740000000000001	-0.00166784502964756\\
0.744999999999994	-0.00143328760783166\\
0.749999999999987	-0.00121673262796016\\
0.750000000000001	-0.00121673262795958\\
0.753999999999993	-0.00105633170278198\\
0.754	-0.00105633170278171\\
0.757999999999993	-0.000907256099697013\\
0.759999999999993	-0.000836940845916116\\
0.76	-0.000836940845915871\\
0.763999999999993	-0.000704709741995797\\
0.767999999999986	-0.000583621164516536\\
0.77	-0.00052723542596818\\
0.770000000000007	-0.000527235425967985\\
0.777999999999993	-0.000329249340358925\\
0.779999999999993	-0.000286609872933135\\
0.78	-0.000286609872932989\\
0.782999999999993	-0.000227770046491974\\
0.783	-0.000227770046491842\\
0.785999999999993	-0.000175058309258881\\
0.788999999999986	-0.000128459247121661\\
0.79	-0.000114282192066528\\
0.790000000000007	-0.000114282192066429\\
0.795999999999993	-4.22451738027334e-05\\
0.8	-6.41055769183859e-06\\
0.800000000000007	-6.41055769178359e-06\\
0.804999999999993	2.47021951101581e-05\\
0.805	2.47021951101915e-05\\
0.809999999999987	4.06373850288279e-05\\
0.809999999999997	4.06373850288453e-05\\
0.810000000000007	4.06373850288627e-05\\
0.811999999999993	4.27650568151917e-05\\
0.812	4.27650568151949e-05\\
0.813999999999987	4.24668040450288e-05\\
0.815999999999973	3.97425879671313e-05\\
0.819999999999945	2.70145344093136e-05\\
0.819999999999987	2.70145344091291e-05\\
0.82	2.70145344090681e-05\\
0.827999999999944	-1.49686270216265e-05\\
0.829999999999993	-2.7600809865971e-05\\
0.830000000000001	-2.76008098660174e-05\\
0.837999999999945	-8.67243667958422e-05\\
0.839999999999993	-0.000103663567035473\\
0.84	-0.000103663567035535\\
0.840999999999993	-0.00011245835966093\\
0.841000000000001	-0.000112458359660993\\
0.841999999999997	-0.000121470325897288\\
0.842999999999993	-0.000130699758575071\\
0.844999999999986	-0.000149812229775815\\
0.848999999999972	-0.000190660437977041\\
0.849999999999993	-0.000201420928829159\\
0.85	-0.000201420928829236\\
0.857999999999972	-0.000295458030977637\\
0.859999999999993	-0.000321190591083767\\
0.86	-0.00032119059108386\\
0.867999999999972	-0.000424070723010823\\
0.869999999999993	-0.000449229809659529\\
0.87	-0.000449229809659618\\
0.874999999999994	-0.000511181751801168\\
0.875000000000001	-0.000511181751801255\\
0.879999999999994	-0.000571822715192518\\
0.880000000000001	-0.000571822715192603\\
0.884999999999994	-0.000631201958367439\\
0.889999999999987	-0.000689367714786955\\
0.890000000000001	-0.000689367714787112\\
0.890000000000008	-0.000689367714787194\\
0.899	-0.000791158298843613\\
0.899000000000008	-0.000791158298843692\\
0.9	-0.000802246811158136\\
0.900000000000007	-0.000802246811158215\\
0.901000000000004	-0.000813224446872723\\
0.902	-0.000824023665330358\\
0.903999999999993	-0.000845088248142637\\
0.907999999999979	-0.000885091457673038\\
0.909999999999993	-0.000904035283336731\\
0.910000000000001	-0.000904035283336797\\
0.917999999999972	-0.000972815107728738\\
0.919999999999993	-0.0009882723676427\\
0.920000000000001	-0.000988272367642754\\
0.927999999999972	-0.00104321089333297\\
0.927999999999994	-0.0010432108933331\\
0.928000000000001	-0.00104321089333314\\
0.929999999999994	-0.00105523182111481\\
0.930000000000001	-0.00105523182111485\\
0.931999999999994	-0.00106657052520826\\
0.933999999999987	-0.00107722847928501\\
0.937999999999972	-0.00109650758952085\\
0.939999999999993	-0.00110513125125947\\
0.940000000000001	-0.0011051312512595\\
0.944999999999994	-0.00112367898475253\\
0.945000000000001	-0.00112367898475255\\
0.949999999999994	-0.00113790357842256\\
0.950000000000001	-0.00113790357842258\\
0.954999999999994	-0.00114781658692478\\
0.956999999999994	-0.00115057646770512\\
0.957000000000001	-0.00115057646770513\\
0.959999999999993	-0.00115342606255304\\
0.960000000000001	-0.00115342606255305\\
0.962999999999993	-0.00115472809295829\\
0.965999999999986	-0.00115448293965585\\
0.969999999999993	-0.00115174914931009\\
0.970000000000001	-0.00115174914931008\\
0.975999999999986	-0.00114248734805354\\
0.979999999999993	-0.00113286738899141\\
0.980000000000001	-0.00113286738899139\\
0.985999999999986	-0.00111457699106654\\
0.985999999999993	-0.00111457699106651\\
0.986000000000001	-0.00111457699106649\\
0.989999999999993	-0.00110038722398059\\
0.990000000000001	-0.00110038722398057\\
0.993999999999993	-0.00108459203725712\\
0.997999999999986	-0.00106718321966508\\
1	-0.00105787090403666\\
};
\end{axis}
\end{tikzpicture}%
}
      \caption{The angular displacement of the unstable pendulum $P_3$ as a
        function of time. $C_i = 10$ ms.}
      \label{fig:01.5.2}
    \end{figure}
  \end{minipage}
\end{minipage}
}



\noindent\makebox[\textwidth][c]{%
\begin{minipage}{\linewidth}
  \begin{minipage}{0.45\linewidth}
    \begin{figure}[H]\centering
      \scalebox{0.7}{% This file was created by matlab2tikz.
%
%The latest updates can be retrieved from
%  http://www.mathworks.com/matlabcentral/fileexchange/22022-matlab2tikz-matlab2tikz
%where you can also make suggestions and rate matlab2tikz.
%
\definecolor{mycolor1}{rgb}{0.00000,0.44700,0.74100}%
\definecolor{mycolor2}{rgb}{0.85000,0.32500,0.09800}%
%
\begin{tikzpicture}

\begin{axis}[%
width=4.133in,
height=3.26in,
at={(0.693in,0.44in)},
scale only axis,
xmin=0,
xmax=1,
xmajorgrids,
ymin=-0.15,
ymax=0.2,
ymajorgrids,
axis background/.style={fill=white}
]
\addplot [color=mycolor1,solid,forget plot]
  table[row sep=crcr]{%
0	0.15314\\
3.15544362088405e-30	0.15314\\
0.000656101980281985	0.153143230512962\\
0.00393661188169191	0.153256312778436\\
0.00599999999999994	0.153410244700375\\
0.006	0.153410244700375\\
0.012	0.152025843789547\\
0.0120000000000001	0.152025843789547\\
0.018	0.146785790333147\\
0.0180000000000001	0.146785790333147\\
0.0199999999999998	0.144179493489919\\
0.02	0.144179493489918\\
0.026	0.133767457951154\\
0.0260000000000002	0.133767457951153\\
0.0289999999999998	0.127361824092399\\
0.029	0.127361824092398\\
0.0319999999999996	0.120505893144387\\
0.0349999999999991	0.113193617075476\\
0.035	0.113193617075474\\
0.0399999999999996	0.0999746854104061\\
0.04	0.0999746854104049\\
0.0449999999999996	0.0854376461080997\\
0.0459999999999996	0.0823689879834041\\
0.046	0.0823689879834027\\
0.047	0.0792808135350665\\
0.0470000000000004	0.0792808135350652\\
0.0490000000000003	0.0731494760963212\\
0.0510000000000002	0.0670761531370235\\
0.055	0.0550940482536636\\
0.0579999999999996	0.0462422041576017\\
0.058	0.0462422041576004\\
0.0599999999999996	0.0404007870498711\\
0.06	0.0404007870498698\\
0.0619999999999995	0.0346045456083864\\
0.0639999999999991	0.0288512070597887\\
0.0659999999999991	0.0231385157858626\\
0.066	0.02313851578586\\
0.0699999999999991	0.0121295381531389\\
0.07	0.0121295381531366\\
0.0700000000000009	0.0121295381531342\\
0.074	0.00186375329871775\\
0.076	-0.00299551806340728\\
0.0760000000000009	-0.0029955180634094\\
0.08	-0.0121763447641911\\
0.0800000000000009	-0.0121763447641931\\
0.0839999999999999	-0.0206520990256193\\
0.086	-0.0246297706860801\\
0.0860000000000009	-0.0246297706860818\\
0.0869999999999991	-0.0265502128214379\\
0.087	-0.0265502128214396\\
0.0880000000000004	-0.0284199307246156\\
0.0890000000000009	-0.0302391076551602\\
0.0910000000000017	-0.0337265468553195\\
0.0929999999999991	-0.0370138976863233\\
0.093	-0.0370138976863247\\
0.0970000000000017	-0.0429934114909054\\
0.0999999999999991	-0.0469618347664145\\
0.1	-0.0469618347664156\\
0.104000000000002	-0.0515714390916565\\
0.104999999999999	-0.0526028899776093\\
0.105	-0.0526028899776102\\
0.105999999999999	-0.053586170072105\\
0.106	-0.0535861700721058\\
0.106999999999999	-0.05452585761117\\
0.107999999999998	-0.0554265265666187\\
0.109999999999997	-0.0571111580107877\\
0.111999999999999	-0.058640724871287\\
0.112	-0.0586407248712876\\
0.115999999999997	-0.0612370029316692\\
0.115999999999998	-0.0612370029316702\\
0.116	-0.0612370029316711\\
0.119999999999997	-0.063219432648722\\
0.119999999999998	-0.0632194326487227\\
0.12	-0.0632194326487234\\
0.123999999999997	-0.0645911231830114\\
0.125999999999999	-0.0650486508548064\\
0.126	-0.0650486508548066\\
0.127999999999998	-0.0653835388327254\\
0.128	-0.0653835388327257\\
0.129999999999998	-0.0656252314014744\\
0.131999999999996	-0.0657738233185884\\
0.135999999999993	-0.0657919017428447\\
0.139999999999998	-0.0654378024812815\\
0.14	-0.0654378024812812\\
0.144999999999998	-0.0644709017753106\\
0.145	-0.0644709017753102\\
0.145999999999998	-0.0642074456912333\\
0.146	-0.0642074456912328\\
0.146999999999999	-0.0639273473319215\\
0.147999999999998	-0.0636373505175543\\
0.149999999999997	-0.0630275468598978\\
0.151999999999998	-0.0623777956425841\\
0.152	-0.0623777956425835\\
0.155999999999997	-0.0609574158314218\\
0.157999999999998	-0.0601862303730122\\
0.158	-0.0601862303730115\\
0.16	-0.0593739834145983\\
0.160000000000002	-0.0593739834145976\\
0.162000000000002	-0.0585203565157101\\
0.164000000000002	-0.0576250150091663\\
0.166	-0.0566876078731048\\
0.166000000000002	-0.056687607873104\\
0.170000000000002	-0.0547623883316001\\
0.174	-0.0528189567458071\\
0.174000000000001	-0.0528189567458063\\
0.175	-0.0523298967241606\\
0.175000000000002	-0.0523298967241597\\
0.176000000000001	-0.0518394601814495\\
0.177	-0.051347599048052\\
0.178999999999998	-0.0503594100348185\\
0.179999999999998	-0.0498629853004126\\
0.18	-0.0498629853004117\\
0.183999999999997	-0.0478606143562568\\
0.186	-0.0468487381919073\\
0.186000000000002	-0.0468487381919064\\
0.189999999999998	-0.0448460616191702\\
0.192	-0.0438655851705541\\
0.192000000000002	-0.0438655851705532\\
0.195999999999998	-0.0419444338237903\\
0.199999999999995	-0.0400738489287693\\
0.199999999999997	-0.0400738489287682\\
0.2	-0.040073848928767\\
0.202999999999998	-0.0387023267293148\\
0.203	-0.038702326729314\\
0.205999999999998	-0.0373563865838793\\
0.206	-0.0373563865838785\\
0.208999999999998	-0.0360456737975151\\
0.209999999999998	-0.0356188041094374\\
0.21	-0.0356188041094366\\
0.211999999999998	-0.0347798648300974\\
0.212	-0.0347798648300967\\
0.213999999999998	-0.0339603857274477\\
0.215999999999997	-0.0331600455233567\\
0.217999999999998	-0.0323785304413606\\
0.218	-0.0323785304413599\\
0.219999999999998	-0.031615534089747\\
0.22	-0.0316155340897463\\
0.221999999999998	-0.0308707573368921\\
0.223999999999996	-0.030143908190053\\
0.225999999999998	-0.0294347016852164\\
0.226	-0.0294347016852158\\
0.229999999999996	-0.0280719343105698\\
0.231999999999998	-0.0274187948368041\\
0.232	-0.0274187948368036\\
0.235999999999996	-0.0261677239027614\\
0.237999999999998	-0.0255693019550791\\
0.238	-0.0255693019550785\\
0.239999999999998	-0.0249886405204315\\
0.24	-0.024988640520431\\
0.241999999999998	-0.0244255119513503\\
0.243999999999996	-0.0238796954709565\\
0.245	-0.0236132121289339\\
0.245000000000002	-0.0236132121289334\\
0.245999999999998	-0.0233509770892397\\
0.246	-0.0233509770892393\\
0.246999999999999	-0.0230927460166266\\
0.247999999999998	-0.0228382749662833\\
0.249999999999997	-0.022340513535244\\
0.252	-0.021857497646821\\
0.252000000000003	-0.0218574976468202\\
0.256	-0.020934950743504\\
0.259999999999997	-0.020069187370574\\
0.26	-0.0200691873705733\\
0.260999999999996	-0.019861457897877\\
0.261	-0.0198614578978763\\
0.261999999999998	-0.0196571717576136\\
0.262999999999996	-0.0194563089176112\\
0.264999999999993	-0.0190647747253879\\
0.265999999999997	-0.0188740649980124\\
0.266	-0.0188740649980117\\
0.269999999999993	-0.0181390028091065\\
0.271999999999997	-0.0177870605763722\\
0.272	-0.0177870605763716\\
0.275999999999993	-0.0171136638873358\\
0.279999999999986	-0.0164800440753242\\
0.279999999999993	-0.0164800440753232\\
0.28	-0.0164800440753221\\
0.285999999999996	-0.015602038849948\\
0.286	-0.0156020388499475\\
0.289999999999996	-0.0150588753922395\\
0.29	-0.015058875392239\\
0.291999999999996	-0.0147974749263449\\
0.292	-0.0147974749263445\\
0.293999999999996	-0.0145427244433615\\
0.295999999999993	-0.0142945240673259\\
0.297999999999996	-0.0140527764903207\\
0.298	-0.0140527764903202\\
0.299999999999996	-0.0138173869356104\\
0.3	-0.01381738693561\\
0.301999999999996	-0.0135882631190604\\
0.303999999999993	-0.013365315211743\\
0.305999999999996	-0.0131484558060676\\
0.306	-0.0131484558060673\\
0.309999999999993	-0.0127295733955093\\
0.313999999999986	-0.0123278675480914\\
0.314999999999997	-0.0122300505117705\\
0.315	-0.0122300505117702\\
0.318999999999997	-0.011848932449539\\
0.319	-0.0118489324495386\\
0.319999999999996	-0.0117561437671958\\
0.32	-0.0117561437671955\\
0.320999999999998	-0.0116643330826606\\
0.321999999999996	-0.0115734913971812\\
0.323999999999993	-0.0113946795037688\\
0.325999999999996	-0.0112196379830671\\
0.326	-0.0112196379830668\\
0.329999999999993	-0.0108791202965964\\
0.331	-0.01079578148901\\
0.331000000000004	-0.0107957814890097\\
0.333	-0.0106311951896303\\
0.333000000000004	-0.01063119518963\\
0.335	-0.0104693436531452\\
0.336999999999996	-0.0103101634250184\\
0.339999999999996	-0.0100762655512865\\
0.34	-0.0100762655512863\\
0.343999999999993	-0.00977317734586651\\
0.345999999999997	-0.00962526027170982\\
0.346	-0.00962526027170956\\
0.347999999999997	-0.00947960356540665\\
0.348	-0.00947960356540639\\
0.349999999999997	-0.00933607010583894\\
0.35	-0.00933607010583868\\
0.351999999999997	-0.00919460362014799\\
0.353999999999993	-0.00905514864622743\\
0.354	-0.00905514864622694\\
0.357999999999993	-0.00878205530691853\\
0.359999999999996	-0.00864830987459709\\
0.36	-0.00864830987459685\\
0.363999999999993	-0.00838615929020967\\
0.365999999999996	-0.00825765136098592\\
0.366	-0.00825765136098569\\
0.369999999999993	-0.00800583848852589\\
0.373999999999986	-0.00776084841554911\\
0.376999999999997	-0.00758135178776813\\
0.377	-0.00758135178776792\\
0.379999999999997	-0.00740531830425179\\
0.38	-0.00740531830425158\\
0.382999999999996	-0.00723259268666171\\
0.384999999999997	-0.00711920461197321\\
0.385	-0.00711920461197301\\
0.385999999999997	-0.00706302255451955\\
0.386	-0.00706302255451935\\
0.386999999999998	-0.00700720978382274\\
0.387999999999996	-0.00695179614931686\\
0.388999999999997	-0.00689677621988011\\
0.389	-0.00689677621987991\\
0.390999999999997	-0.00678789594421677\\
0.392999999999993	-0.00668052627018024\\
0.394999999999997	-0.00657462510307921\\
0.395	-0.00657462510307902\\
0.398999999999993	-0.00636706277582973\\
0.399999999999997	-0.00631602581140577\\
0.4	-0.00631602581140559\\
0.403999999999993	-0.00611514234694707\\
0.405999999999997	-0.00601659044921387\\
0.406	-0.0060165904492137\\
0.409999999999993	-0.00582364296493607\\
0.411999999999997	-0.00572931417942692\\
0.412	-0.00572931417942676\\
0.415999999999993	-0.00554476159961642\\
0.419999999999986	-0.00536544202466481\\
0.419999999999996	-0.00536544202466433\\
0.42	-0.00536544202466417\\
0.426	-0.005105661471752\\
0.426000000000004	-0.00510566147175185\\
0.432000000000004	-0.00485715184030643\\
0.432000000000007	-0.00485715184030629\\
0.434999999999997	-0.0047372395980762\\
0.435	-0.00473723959807606\\
0.43799999999999	-0.00462008052598476\\
0.439999999999997	-0.00454345288070487\\
0.44	-0.00454345288070474\\
0.44299999999999	-0.00443065390957599\\
0.445999999999979	-0.0043203376519555\\
0.445999999999995	-0.00432033765195493\\
0.446	-0.00432033765195474\\
0.447	-0.00428412121687065\\
0.447000000000004	-0.00428412121687052\\
0.448000000000004	-0.00424820789709132\\
0.449000000000004	-0.00421259417212973\\
0.451000000000004	-0.00414225157361891\\
0.454999999999997	-0.00400500985723923\\
0.455	-0.00400500985723911\\
0.459	-0.00387218082350991\\
0.459999999999997	-0.00383963848634947\\
0.46	-0.00383963848634935\\
0.463999999999997	-0.0037120331307038\\
0.464	-0.00371203313070369\\
0.465999999999997	-0.00364972516138714\\
0.466	-0.00364972516138703\\
0.466999999999997	-0.00361894787712166\\
0.467	-0.00361894787712155\\
0.467999999999998	-0.00358843606096643\\
0.468999999999997	-0.00355818672244122\\
0.470999999999993	-0.0034984636447164\\
0.472999999999997	-0.00343975522928542\\
0.473	-0.00343975522928531\\
0.476999999999993	-0.00332529070805253\\
0.479999999999997	-0.00324193742300783\\
0.48	-0.00324193742300773\\
0.483999999999993	-0.00313398671261266\\
0.485999999999997	-0.00308132950356731\\
0.486	-0.00308132950356722\\
0.489999999999993	-0.00297868852415155\\
0.49	-0.00297868852415136\\
0.490000000000004	-0.00297868852415127\\
0.492999999999997	-0.00290402940479505\\
0.493	-0.00290402940479497\\
0.495999999999993	-0.00283128690175424\\
0.498999999999986	-0.00276039684508185\\
0.498999999999993	-0.00276039684508169\\
0.499	-0.00276039684508152\\
0.499999999999997	-0.00273716762589667\\
0.5	-0.00273716762589659\\
0.500999999999998	-0.00271413499636033\\
0.501999999999997	-0.00269129669864501\\
0.503999999999993	-0.00264619416394156\\
0.505999999999993	-0.00260184233916467\\
0.506	-0.00260184233916451\\
0.507999999999997	-0.00255824665041905\\
0.508000000000004	-0.0025582466504189\\
0.51	-0.00251541282024629\\
0.511999999999997	-0.0024733240554731\\
0.51599999999999	-0.00239131600395722\\
0.519999999999993	-0.00231209387431912\\
0.52	-0.00231209387431898\\
0.521999999999993	-0.00227348853796277\\
0.522	-0.00227348853796264\\
0.523999999999993	-0.00223553342270299\\
0.524999999999993	-0.00221679503717306\\
0.525	-0.00221679503717292\\
0.526	-0.00219821364578049\\
0.526000000000007	-0.00219821364578036\\
0.527000000000007	-0.00217979147688287\\
0.528000000000007	-0.00216153077443716\\
0.530000000000007	-0.00212548662559605\\
0.532	-0.00209006706801044\\
0.532000000000007	-0.00209006706801032\\
0.536000000000007	-0.00202104642138933\\
0.538	-0.00198741827253858\\
0.538000000000007	-0.00198741827253846\\
0.539999999999993	-0.00195436058517631\\
0.54	-0.00195436058517619\\
0.541999999999986	-0.00192186039908287\\
0.543999999999972	-0.00188990497243143\\
0.546	-0.0018584817769738\\
0.546000000000007	-0.00185848177697369\\
0.549999999999979	-0.00179723505058651\\
0.550999999999993	-0.00178225327986291\\
0.551	-0.00178225327986281\\
0.554999999999972	-0.00172360178594778\\
0.556999999999993	-0.0016950213156329\\
0.557	-0.0016950213156328\\
0.559999999999993	-0.00165305062302131\\
0.56	-0.00165305062302121\\
0.562999999999993	-0.00161212710529833\\
0.565999999999986	-0.00157221466169964\\
0.565999999999993	-0.00157221466169954\\
0.566	-0.00157221466169945\\
0.571999999999986	-0.00149538666290301\\
0.571999999999993	-0.00149538666290292\\
0.572	-0.00149538666290283\\
0.577999999999986	-0.00142239911156992\\
0.579999999999993	-0.00139887846372692\\
0.58	-0.00139887846372684\\
0.585999999999986	-0.00133061300722733\\
0.585999999999993	-0.00133061300722725\\
0.586	-0.00133061300722717\\
0.591999999999986	-0.00126570205795457\\
0.591999999999993	-0.0012657020579545\\
0.592	-0.00126570205795442\\
0.594999999999993	-0.00123446833087104\\
0.595	-0.00123446833087096\\
0.597999999999993	-0.00120401212899401\\
0.599999999999993	-0.00118412665745423\\
0.6	-0.00118412665745416\\
0.602999999999993	-0.00115490695376066\\
0.605999999999986	-0.00112639458966449\\
0.606	-0.00112639458966436\\
0.606999999999993	-0.00111704573809149\\
0.607	-0.00111704573809143\\
0.607999999999999	-0.00110777666046391\\
0.608999999999997	-0.00109858644815858\\
0.609000000000004	-0.00109858644815852\\
0.611	-0.00108043902421022\\
0.612999999999997	-0.00106259635152698\\
0.614999999999997	-0.0010450514348206\\
0.615000000000004	-0.00104505143482053\\
0.618999999999997	-0.00101082746959056\\
0.619999999999993	-0.00100244696554173\\
0.62	-0.00100244696554167\\
0.623999999999993	-0.00096960225516176\\
0.625999999999993	-0.000953575039258187\\
0.626	-0.000953575039258131\\
0.629999999999993	-0.000922314872374938\\
0.63	-0.000922314872374883\\
0.633999999999993	-0.000892096646841945\\
0.635999999999993	-0.00087736338994188\\
0.636	-0.000877363389941828\\
0.637999999999993	-0.00086287296956934\\
0.638	-0.000862872969569289\\
0.639999999999993	-0.000848619704775458\\
0.64	-0.000848619704775408\\
0.641999999999993	-0.000834598007511279\\
0.643999999999986	-0.000820802380553502\\
0.645999999999993	-0.000807227415273633\\
0.646	-0.000807227415273585\\
0.649999999999986	-0.0007807494801423\\
0.65	-0.000780749480142208\\
0.650000000000007	-0.000780749480142162\\
0.653999999999993	-0.00075515388929724\\
0.657999999999979	-0.000730400499675287\\
0.659999999999993	-0.000718327447732739\\
0.66	-0.000718327447732696\\
0.664999999999993	-0.000688992555410453\\
0.665	-0.000688992555410412\\
0.665999999999993	-0.000683266293778606\\
0.666	-0.000683266293778566\\
0.666999999999998	-0.000677587299556187\\
0.667000000000006	-0.000677587299556147\\
0.668000000000004	-0.000671956698056672\\
0.669000000000002	-0.000666373937422595\\
0.670999999999998	-0.00065534975471149\\
0.673000000000005	-0.000644510429349712\\
0.673000000000013	-0.000644510429349674\\
0.677000000000005	-0.000623369423345519\\
0.678	-0.000618193153072279\\
0.678000000000007	-0.000618193153072243\\
0.679999999999993	-0.000607967823817857\\
0.68	-0.00060796782381782\\
0.681999999999986	-0.000597908775826181\\
0.683999999999972	-0.000588012065407668\\
0.686	-0.000578273812518879\\
0.686000000000007	-0.000578273812518844\\
0.689999999999979	-0.000559280295899554\\
0.69399999999995	-0.000540921264682568\\
0.695999999999993	-0.000531970644944006\\
0.696	-0.000531970644943975\\
0.699999999999993	-0.000514509651760496\\
0.7	-0.000514509651760466\\
0.703999999999993	-0.000497612928358538\\
0.705999999999993	-0.000489367852791729\\
0.706	-0.0004893678527917\\
0.707999999999993	-0.000481258764669065\\
0.708	-0.000481258764669036\\
0.709999999999993	-0.000473287275730152\\
0.711999999999986	-0.00046545026072113\\
0.713999999999993	-0.000457744647110169\\
0.714	-0.000457744647110142\\
0.717999999999986	-0.000442715590537679\\
0.719999999999993	-0.000435386255375497\\
0.72	-0.000435386255375471\\
0.723999999999986	-0.000421083602878829\\
0.724999999999993	-0.000417580061616165\\
0.725	-0.00041758006161614\\
0.725999999999993	-0.000414104678126119\\
0.726	-0.000414104678126094\\
0.726999999999999	-0.000410658112971318\\
0.727999999999997	-0.000407241029513212\\
0.729999999999993	-0.000400493970904834\\
0.731999999999993	-0.000393860857090771\\
0.732	-0.000393860857090748\\
0.734999999999993	-0.000384119156237618\\
0.735	-0.000384119156237595\\
0.737999999999993	-0.000374619396625751\\
0.74	-0.000368416488403757\\
0.740000000000007	-0.000368416488403735\\
0.743	-0.000359301443504857\\
0.745999999999993	-0.000350406446863639\\
0.746000000000007	-0.000350406446863598\\
0.746999999999993	-0.000347489822771658\\
0.747	-0.000347489822771637\\
0.747999999999999	-0.000344598163091137\\
0.748999999999997	-0.000341731184360338\\
0.750999999999993	-0.000336070148164364\\
0.753999999999993	-0.000327756753293898\\
0.754	-0.000327756753293879\\
0.757999999999993	-0.000316993431217072\\
0.759999999999993	-0.000311744671911115\\
0.76	-0.000311744671911096\\
0.763999999999993	-0.000301502667102156\\
0.766	-0.000296505406181009\\
0.766000000000007	-0.000296505406180992\\
0.77	-0.000286759795598509\\
0.770000000000007	-0.000286759795598492\\
0.774	-0.000277340867131346\\
0.776	-0.000272749272973379\\
0.776000000000007	-0.000272749272973363\\
0.779999999999993	-0.000263792823452599\\
0.78	-0.000263792823452584\\
0.782999999999993	-0.000257266983875963\\
0.783	-0.000257266983875948\\
0.785999999999993	-0.00025089887536838\\
0.786000000000001	-0.000250898875368365\\
0.788999999999994	-0.000244688191913039\\
0.791999999999987	-0.000238634766370875\\
0.792	-0.000238634766370847\\
0.792000000000008	-0.000238634766370833\\
0.797999999999994	-0.000226978462895243\\
0.799999999999993	-0.00022322092671551\\
0.8	-0.000223220926715497\\
0.804999999999993	-0.000214092868713142\\
0.805000000000001	-0.00021409286871313\\
0.805999999999993	-0.000212311385418157\\
0.806	-0.000212311385418145\\
0.806999999999994	-0.000210544699061964\\
0.807999999999987	-0.00020879313470276\\
0.809999999999973	-0.000205334686754476\\
0.811999999999993	-0.000201934685675467\\
0.812	-0.000201934685675455\\
0.815999999999973	-0.000195304714664832\\
0.817999999999993	-0.000192072145426957\\
0.818000000000001	-0.000192072145426945\\
0.819999999999993	-0.000188892823472461\\
0.82	-0.00018889282347245\\
0.821999999999993	-0.000185765502352623\\
0.823999999999986	-0.000182688955988815\\
0.825999999999993	-0.000179661978209226\\
0.826	-0.000179661978209215\\
0.829999999999986	-0.000173758748889766\\
0.833999999999972	-0.00016805330440493\\
0.839999999999993	-0.000159846491347907\\
0.84	-0.000159846491347897\\
0.840999999999993	-0.000158517949109501\\
0.841000000000001	-0.000158517949109491\\
0.841999999999994	-0.000157200273212734\\
0.842999999999987	-0.000155893334450983\\
0.844999999999973	-0.000153311156997933\\
0.845999999999993	-0.000152035665221624\\
0.846	-0.000152035665221615\\
0.849999999999973	-0.000147040493095658\\
0.851999999999993	-0.000144606132116096\\
0.852	-0.000144606132116087\\
0.855999999999973	-0.000139859088005421\\
0.857999999999993	-0.000137544543776967\\
0.858	-0.000137544543776958\\
0.86	-0.00013526810311292\\
0.860000000000007	-0.000135268103112912\\
0.862000000000007	-0.000133028873537915\\
0.864000000000007	-0.000130825977153128\\
0.866	-0.000128658550304601\\
0.866000000000007	-0.000128658550304593\\
0.87	-0.000124431571562097\\
0.870000000000007	-0.000124431571562089\\
0.874	-0.000120346159599326\\
0.874999999999994	-0.000119346170814537\\
0.875000000000001	-0.00011934617081453\\
0.876	-0.000118354530447473\\
0.876000000000007	-0.000118354530447466\\
0.877000000000007	-0.000117371141310725\\
0.878000000000006	-0.000116395907018226\\
0.879999999999998	-0.000114469521423438\\
0.880000000000006	-0.000114469521423431\\
0.882000000000005	-0.000112574618421282\\
0.884000000000004	-0.000110710455112767\\
0.886000000000005	-0.000108876300645523\\
0.886000000000013	-0.000108876300645517\\
0.888000000000007	-0.000107072464331694\\
0.888000000000014	-0.000107072464331687\\
0.890000000000009	-0.000105299267369401\\
0.892000000000004	-0.000103556014569742\\
0.895999999999993	-0.000100156619152621\\
0.898999999999993	-9.76806754823634e-05\\
0.899000000000001	-9.76806754823576e-05\\
0.899999999999993	-9.68689464304756e-05\\
0.9	-9.68689464304698e-05\\
0.900999999999994	-9.6063877013777e-05\\
0.901999999999987	-9.52653882900104e-05\\
0.903999999999973	-9.36878405177649e-05\\
0.905999999999993	-9.21356846300636e-05\\
0.906	-9.21356846300581e-05\\
0.909999999999973	-8.91086105839449e-05\\
0.909999999999987	-8.91086105839346e-05\\
0.910000000000001	-8.91086105839244e-05\\
0.910999999999993	-8.83678628003285e-05\\
0.911000000000001	-8.83678628003233e-05\\
0.911999999999994	-8.76333771741875e-05\\
0.912999999999987	-8.69050817155083e-05\\
0.914999999999973	-8.54667763781069e-05\\
0.916999999999993	-8.40523828949788e-05\\
0.917000000000001	-8.40523828949738e-05\\
0.919999999999993	-8.19744169173503e-05\\
0.92	-8.19744169173455e-05\\
0.922999999999993	-7.99471740298812e-05\\
0.925999999999986	-7.79688658961427e-05\\
0.925999999999993	-7.7968865896138e-05\\
0.926	-7.79688658961333e-05\\
0.927999999999993	-7.66770543219352e-05\\
0.928000000000001	-7.66770543219306e-05\\
0.929999999999994	-7.5407184820723e-05\\
0.931999999999987	-7.41587595138373e-05\\
0.933999999999994	-7.29312889513422e-05\\
0.934000000000001	-7.29312889513379e-05\\
0.937999999999987	-7.0537295177499e-05\\
0.939999999999993	-6.93698333914894e-05\\
0.940000000000001	-6.93698333914852e-05\\
0.943999999999987	-6.70916913305193e-05\\
0.944999999999994	-6.65336590054775e-05\\
0.945000000000001	-6.65336590054735e-05\\
0.945999999999993	-6.59801178998789e-05\\
0.946000000000001	-6.59801178998749e-05\\
0.946999999999994	-6.54311699499357e-05\\
0.947999999999987	-6.48869175379424e-05\\
0.949999999999973	-6.3812286413191e-05\\
0.952	-6.27558032123494e-05\\
0.952000000000008	-6.27558032123457e-05\\
0.95599999999998	-6.06956307672097e-05\\
0.956999999999994	-6.01912913074925e-05\\
0.957000000000001	-6.01912913074889e-05\\
0.96	-5.87031691074909e-05\\
0.960000000000008	-5.87031691074874e-05\\
0.963000000000007	-5.72513743971698e-05\\
0.966000000000007	-5.58346264728782e-05\\
0.966000000000014	-5.58346264728749e-05\\
0.969000000000007	-5.44528649793914e-05\\
0.969000000000014	-5.44528649793881e-05\\
0.972000000000007	-5.31060604824633e-05\\
0.975	-5.17930248772815e-05\\
0.979999999999994	-4.9676541243674e-05\\
0.980000000000001	-4.9676541243671e-05\\
0.985999999999987	-4.72490603455223e-05\\
0.986000000000001	-4.72490603455167e-05\\
0.991999999999987	-4.49400389592495e-05\\
0.992000000000001	-4.49400389592443e-05\\
0.997999999999987	-4.27453534169505e-05\\
0.998000000000001	-4.27453534169455e-05\\
0.999999999999993	-4.20378496340661e-05\\
1	-4.20378496340636e-05\\
1.00199999999999	-4.13419087208149e-05\\
1.00399999999999	-4.06572578191729e-05\\
1.00599999999999	-3.99836285091284e-05\\
1.006	-3.99836285091237e-05\\
1.00999999999999	-3.86698957854157e-05\\
1.01399999999997	-3.74001634276287e-05\\
1.01499999999999	-3.708937054333e-05\\
1.015	-3.70893705433256e-05\\
1.01999999999999	-3.55737274428047e-05\\
1.02	-3.55737274428004e-05\\
1.02499999999999	-3.41192521065527e-05\\
1.02599999999999	-3.3835380944933e-05\\
1.026	-3.3835380944929e-05\\
1.02699999999999	-3.35538656382162e-05\\
1.027	-3.35538656382122e-05\\
1.02799999999999	-3.32747585990048e-05\\
1.02899999999999	-3.29980324694628e-05\\
1.03099999999997	-3.24516146819163e-05\\
1.03299999999999	-3.19143980503788e-05\\
1.033	-3.1914398050375e-05\\
1.03699999999997	-3.08667293216363e-05\\
1.04	-3.01035897706801e-05\\
1.04000000000001	-3.01035897706765e-05\\
1.04399999999999	-2.91149397160916e-05\\
1.044	-2.91149397160882e-05\\
1.046	-2.86325503103997e-05\\
1.04600000000001	-2.86325503103963e-05\\
1.04800000000001	-2.81581355429319e-05\\
1.05	-2.76917801657761e-05\\
1.05000000000001	-2.76917801657728e-05\\
1.05200000000001	-2.723330134373e-05\\
1.05200000000002	-2.72333013437268e-05\\
1.05400000000002	-2.67825193296244e-05\\
1.05600000000002	-2.63392573938291e-05\\
1.05800000000001	-2.59033417538038e-05\\
1.05800000000002	-2.59033417538007e-05\\
1.05999999999999	-2.5474601509497e-05\\
1.06	-2.54746015094939e-05\\
1.06199999999996	-2.50528685738976e-05\\
1.06399999999992	-2.46379776049399e-05\\
1.06599999999999	-2.42297659430081e-05\\
1.066	-2.42297659430052e-05\\
1.06999999999992	-2.34336593983795e-05\\
1.07299999999999	-2.28541444051524e-05\\
1.073	-2.28541444051497e-05\\
1.07699999999992	-2.2103905639253e-05\\
1.07899999999999	-2.17380751798321e-05\\
1.079	-2.17380751798295e-05\\
1.07999999999999	-2.15574190393219e-05\\
1.08	-2.15574190393193e-05\\
1.08099999999999	-2.13782453018156e-05\\
1.08199999999999	-2.12005364057517e-05\\
1.08399999999997	-2.08494436105259e-05\\
1.08499999999999	-2.0676025299972e-05\\
1.085	-2.06760252999696e-05\\
1.08599999999999	-2.05040030060072e-05\\
1.086	-2.05040030060048e-05\\
1.08699999999999	-2.03334083333607e-05\\
1.08799999999999	-2.01642730254155e-05\\
1.08999999999997	-1.98303143369535e-05\\
1.09199999999999	-1.9501996010831e-05\\
1.092	-1.95019960108287e-05\\
1.09599999999997	-1.88617677405113e-05\\
1.09999999999995	-1.82425841060603e-05\\
1.09999999999997	-1.82425841060561e-05\\
1.1	-1.8242584106052e-05\\
1.10199999999999	-1.79405792921785e-05\\
1.102	-1.79405792921763e-05\\
1.10399999999999	-1.76434739873749e-05\\
1.10599999999997	-1.73511516923925e-05\\
1.10599999999999	-1.73511516923903e-05\\
1.106	-1.73511516923883e-05\\
1.10999999999997	-1.6781055835308e-05\\
1.11199999999999	-1.65032228232008e-05\\
1.112	-1.65032228231988e-05\\
1.11599999999997	-1.59614419683402e-05\\
1.11999999999994	-1.54374695874575e-05\\
1.12	-1.54374695874501e-05\\
1.12000000000001	-1.54374695874483e-05\\
1.126	-1.46831115783722e-05\\
1.12600000000001	-1.46831115783705e-05\\
1.13099999999999	-1.40826238211678e-05\\
1.131	-1.40826238211662e-05\\
1.132	-1.39655673521018e-05\\
1.13200000000001	-1.39655673521001e-05\\
1.13300000000001	-1.38494975788315e-05\\
1.13400000000001	-1.37344031242957e-05\\
1.13600000000001	-1.35070951440944e-05\\
1.138	-1.32835542945336e-05\\
1.13800000000001	-1.3283554294532e-05\\
1.13999999999999	-1.30636929367676e-05\\
1.14	-1.30636929367661e-05\\
1.14199999999997	-1.28474248744612e-05\\
1.14399999999994	-1.26346653188649e-05\\
1.14599999999999	-1.24253308567719e-05\\
1.146	-1.24253308567705e-05\\
1.14999999999994	-1.2017080284122e-05\\
1.15399999999989	-1.16225029095039e-05\\
1.15499999999999	-1.15259220212841e-05\\
1.155	-1.15259220212827e-05\\
1.15999999999999	-1.10549259910857e-05\\
1.16	-1.10549259910844e-05\\
1.16499999999999	-1.06029379264291e-05\\
1.16599999999999	-1.05147229999796e-05\\
1.166	-1.05147229999784e-05\\
1.17099999999999	-1.00847077908383e-05\\
1.17199999999999	-1.00008824871519e-05\\
1.172	-1.00008824871507e-05\\
1.173	-9.91776376173988e-06\\
1.17300000000001	-9.9177637617387e-06\\
1.17400000000001	-9.83534346881401e-06\\
1.17500000000001	-9.75361352960694e-06\\
1.17700000000001	-9.59219273768623e-06\\
1.17999999999999	-9.35504110010023e-06\\
1.18	-9.35504110009912e-06\\
1.184	-9.04780999310616e-06\\
1.18599999999999	-8.89790339915877e-06\\
1.186	-8.89790339915772e-06\\
1.18899999999999	-8.6777034118826e-06\\
1.189	-8.67770341188157e-06\\
1.18999999999999	-8.60555087815935e-06\\
1.19	-8.60555087815833e-06\\
1.19099999999999	-8.53401023871499e-06\\
1.19199999999999	-8.46307448174067e-06\\
1.19399999999997	-8.32298986403477e-06\\
1.19499999999999	-8.2538272732956e-06\\
1.195	-8.25382727329462e-06\\
1.19899999999997	-7.98288418604953e-06\\
1.19999999999999	-7.91654203892987e-06\\
1.2	-7.91654203892893e-06\\
1.20399999999997	-7.65655286739852e-06\\
1.20599999999999	-7.52969692657781e-06\\
1.206	-7.52969692657691e-06\\
1.20699999999999	-7.46704952780753e-06\\
1.207	-7.46704952780665e-06\\
1.20799999999999	-7.40493804245904e-06\\
1.20899999999999	-7.34335638272751e-06\\
1.21099999999997	-7.22175844887707e-06\\
1.21499999999995	-6.98465832714022e-06\\
1.21799999999999	-6.81198437922706e-06\\
1.218	-6.81198437922625e-06\\
1.21999999999999	-6.69923636513134e-06\\
1.22	-6.69923636513054e-06\\
1.22199999999999	-6.5883310412403e-06\\
1.22399999999997	-6.47922491998147e-06\\
1.22499999999999	-6.42533316579872e-06\\
1.225	-6.42533316579796e-06\\
1.22599999999999	-6.37187522576968e-06\\
1.226	-6.37187522576892e-06\\
1.22699999999999	-6.3188609252805e-06\\
1.22799999999999	-6.26630013280449e-06\\
1.22999999999997	-6.16251850981608e-06\\
1.23	-6.1625185098147e-06\\
1.23399999999997	-5.96017361010833e-06\\
1.236	-5.86153100355807e-06\\
1.23600000000001	-5.86153100355737e-06\\
1.23999999999999	-5.66911209821449e-06\\
1.24	-5.66911209821382e-06\\
1.24399999999997	-5.48293117095466e-06\\
1.24599999999999	-5.39208832980723e-06\\
1.246	-5.39208832980659e-06\\
1.247	-5.34722588613589e-06\\
1.24700000000001	-5.34722588613526e-06\\
1.24800000000001	-5.30274721643946e-06\\
1.24900000000001	-5.25864796117829e-06\\
1.25100000000001	-5.17157044198016e-06\\
1.253	-5.08595918946622e-06\\
1.25300000000001	-5.08595918946562e-06\\
1.25700000000001	-4.91900179135527e-06\\
1.25999999999999	-4.79738711157346e-06\\
1.26	-4.79738711157289e-06\\
1.264	-4.63983471791204e-06\\
1.266	-4.56296053566249e-06\\
1.26600000000001	-4.56296053566195e-06\\
1.27000000000001	-4.41303812081527e-06\\
1.272	-4.33997426245643e-06\\
1.27200000000001	-4.33997426245592e-06\\
1.276	-4.19749800956031e-06\\
1.27600000000001	-4.19749800955981e-06\\
1.27999999999999	-4.05970495345541e-06\\
1.28000000000001	-4.05970495345493e-06\\
1.28399999999999	-3.92637898522597e-06\\
1.28599999999999	-3.86132554959552e-06\\
1.28600000000001	-3.86132554959506e-06\\
1.288	-3.79734751779369e-06\\
1.28800000000001	-3.79734751779324e-06\\
1.29	-3.73445632351069e-06\\
1.29199999999999	-3.67262731000453e-06\\
1.29499999999999	-3.58182246485316e-06\\
1.295	-3.58182246485274e-06\\
1.29899999999998	-3.46424411588173e-06\\
1.29999999999999	-3.4354543087659e-06\\
1.3	-3.4354543087655e-06\\
1.30399999999998	-3.32262954319313e-06\\
1.30499999999999	-3.29499314494559e-06\\
1.305	-3.2949931449452e-06\\
1.30599999999999	-3.26757921342542e-06\\
1.306	-3.26757921342503e-06\\
1.30699999999999	-3.24039278693323e-06\\
1.30700000000001	-3.24039278693284e-06\\
1.308	-3.21343892586474e-06\\
1.30899999999999	-3.18671498846163e-06\\
1.31099999999998	-3.13394643003336e-06\\
1.31300000000001	-3.08206642353151e-06\\
1.31300000000002	-3.08206642353115e-06\\
1.31699999999999	-2.98089104891791e-06\\
1.31999999999999	-2.9071931457201e-06\\
1.32	-2.90719314571975e-06\\
1.32399999999997	-2.81171716818259e-06\\
1.32599999999999	-2.76513179317114e-06\\
1.326	-2.76513179317081e-06\\
1.32999999999997	-2.67427955214637e-06\\
1.33	-2.67427955214579e-06\\
1.33399999999997	-2.5864701600707e-06\\
1.334	-2.58647016007012e-06\\
1.33799999999997	-2.50156590033478e-06\\
1.34	-2.46016141016012e-06\\
1.34000000000001	-2.46016141015982e-06\\
1.34399999999999	-2.37936653988299e-06\\
1.346	-2.33994448383389e-06\\
1.34600000000001	-2.33994448383361e-06\\
1.348	-2.30117411735104e-06\\
1.34800000000001	-2.30117411735077e-06\\
1.35	-2.26306236891223e-06\\
1.35199999999999	-2.22559429665541e-06\\
1.35599999999996	-2.15253066967173e-06\\
1.35999999999999	-2.08186864330043e-06\\
1.36	-2.08186864330018e-06\\
1.36299999999999	-2.0303813715584e-06\\
1.363	-2.03038137155816e-06\\
1.36499999999999	-1.99674987871469e-06\\
1.365	-1.99674987871446e-06\\
1.36599999999999	-1.98013717357585e-06\\
1.366	-1.98013717357561e-06\\
1.36699999999999	-1.96366233525072e-06\\
1.36799999999999	-1.9473284303516e-06\\
1.36999999999997	-1.91507703091651e-06\\
1.37199999999999	-1.8833703309852e-06\\
1.372	-1.88337033098498e-06\\
1.37599999999997	-1.82154151613983e-06\\
1.378	-1.79139516099549e-06\\
1.37800000000001	-1.79139516099528e-06\\
1.37999999999999	-1.76174501581677e-06\\
1.38	-1.76174501581656e-06\\
1.38199999999997	-1.73257945630545e-06\\
1.38399999999994	-1.70388704798853e-06\\
1.38599999999999	-1.67565654189632e-06\\
1.386	-1.67565654189612e-06\\
1.38999999999994	-1.62060053392155e-06\\
1.39199999999999	-1.59376929346458e-06\\
1.392	-1.59376929346439e-06\\
1.39599999999994	-1.54144773968102e-06\\
1.39799999999999	-1.51593691348235e-06\\
1.398	-1.51593691348217e-06\\
1.39999999999999	-1.49084599621229e-06\\
1.4	-1.49084599621211e-06\\
1.40199999999999	-1.46616515101157e-06\\
1.40399999999997	-1.4418847016572e-06\\
1.40599999999999	-1.41799512890556e-06\\
1.406	-1.41799512890539e-06\\
1.40999999999997	-1.3714049434806e-06\\
1.412	-1.34869947465724e-06\\
1.41200000000001	-1.34869947465708e-06\\
1.41599999999999	-1.30442327301461e-06\\
1.41999999999996	-1.26160243071545e-06\\
1.41999999999998	-1.26160243071522e-06\\
1.42	-1.261602430715e-06\\
1.42099999999999	-1.25111670079919e-06\\
1.421	-1.25111670079905e-06\\
1.42199999999999	-1.24071669557977e-06\\
1.42299999999999	-1.23040139526944e-06\\
1.42499999999997	-1.21002087355164e-06\\
1.42599999999999	-1.19995365460354e-06\\
1.426	-1.1999536546034e-06\\
1.42999999999997	-1.16052752266519e-06\\
1.43199999999999	-1.14131341558744e-06\\
1.432	-1.1413134155873e-06\\
1.43499999999999	-1.11309472116344e-06\\
1.435	-1.11309472116331e-06\\
1.43799999999998	-1.08557690585826e-06\\
1.43999999999999	-1.06760906053696e-06\\
1.44	-1.06760906053683e-06\\
1.44299999999999	-1.0412057285083e-06\\
1.44599999999997	-1.01543985861267e-06\\
1.44599999999999	-1.01543985861253e-06\\
1.446	-1.0154398586124e-06\\
1.44699999999999	-1.00699134973081e-06\\
1.447	-1.00699134973069e-06\\
1.44799999999999	-9.98615113433553e-07\\
1.44899999999999	-9.90310328620577e-07\\
1.44999999999999	-9.82076181335089e-07\\
1.45	-9.82076181334972e-07\\
1.45199999999999	-9.65816578069283e-07\\
1.45399999999997	-9.49829929037836e-07\\
1.45599999999999	-9.34109966613451e-07\\
1.456	-9.3410996661334e-07\\
1.45999999999997	-9.03445551816085e-07\\
1.46	-9.03445551815871e-07\\
1.46000000000001	-9.03445551815764e-07\\
1.46399999999999	-8.73775240836027e-07\\
1.466	-8.59298273570955e-07\\
1.46600000000001	-8.59298273570853e-07\\
1.46999999999999	-8.3106484305882e-07\\
1.47	-8.31064843058722e-07\\
1.47399999999997	-8.03777013858898e-07\\
1.47599999999999	-7.90474269540854e-07\\
1.476	-7.90474269540761e-07\\
1.47899999999998	-7.70931916199948e-07\\
1.479	-7.70931916199856e-07\\
1.47999999999999	-7.64525042761406e-07\\
1.48	-7.64525042761315e-07\\
1.48099999999999	-7.58170740579864e-07\\
1.48199999999999	-7.51868386860326e-07\\
1.48399999999997	-7.39417059053101e-07\\
1.48599999999999	-7.27166177748748e-07\\
1.486	-7.27166177748662e-07\\
1.48999999999997	-7.0327412912771e-07\\
1.491	-6.97427583694441e-07\\
1.49100000000001	-6.97427583694358e-07\\
1.49499999999999	-6.74530076842024e-07\\
1.49899999999996	-6.52387687670176e-07\\
1.49999999999999	-6.46965982437778e-07\\
1.5	-6.46965982437701e-07\\
1.50499999999999	-6.20514293231628e-07\\
1.505	-6.20514293231554e-07\\
1.50599999999999	-6.15351693299492e-07\\
1.506	-6.15351693299419e-07\\
1.50699999999999	-6.10231937175109e-07\\
1.50799999999999	-6.05155977810956e-07\\
1.508	-6.05155977810884e-07\\
1.50999999999999	-5.951334636196e-07\\
1.51199999999997	-5.85280221386688e-07\\
1.51399999999999	-5.75592388110665e-07\\
1.514	-5.75592388110597e-07\\
1.51799999999997	-5.56697819413171e-07\\
1.518	-5.56697819413046e-07\\
1.51999999999999	-5.47483676301743e-07\\
1.52	-5.47483676301678e-07\\
1.52199999999999	-5.38420124017042e-07\\
1.52399999999997	-5.29503609157478e-07\\
1.52599999999999	-5.20730635969083e-07\\
1.526	-5.20730635969022e-07\\
1.52999999999997	-5.03621311646199e-07\\
1.53399999999994	-4.8708501728248e-07\\
1.537	-4.75043212661656e-07\\
1.53700000000001	-4.750432126616e-07\\
1.53999999999999	-4.63298508796156e-07\\
1.54	-4.63298508796101e-07\\
1.54299999999997	-4.51840545741969e-07\\
1.54599999999994	-4.40659215145519e-07\\
1.54599999999997	-4.40659215145417e-07\\
1.546	-4.40659215145314e-07\\
1.549	-4.2975402991908e-07\\
1.54900000000001	-4.29754029919029e-07\\
1.55200000000001	-4.1912474703272e-07\\
1.55500000000001	-4.08761989696897e-07\\
1.55500000000003	-4.08761989696848e-07\\
1.55999999999999	-3.9205828064636e-07\\
1.56	-3.92058280646314e-07\\
1.56499999999996	-3.76028684247008e-07\\
1.56599999999998	-3.72900173450336e-07\\
1.566	-3.72900173450292e-07\\
1.57099999999996	-3.57649839507866e-07\\
1.57199999999998	-3.54677005739556e-07\\
1.572	-3.54677005739514e-07\\
1.57499999999999	-3.45907703154635e-07\\
1.575	-3.45907703154594e-07\\
1.57799999999999	-3.37356207353412e-07\\
1.57999999999999	-3.31772480900963e-07\\
1.58	-3.31772480900924e-07\\
1.58299999999999	-3.23567324397437e-07\\
1.58599999999997	-3.15560267043704e-07\\
1.586	-3.15560267043631e-07\\
1.58699999999999	-3.12934790192868e-07\\
1.587	-3.1293479019283e-07\\
1.58799999999999	-3.10331772929613e-07\\
1.58899999999999	-3.07750960087046e-07\\
1.59099999999997	-3.02654938029343e-07\\
1.59499999999995	-2.92718360093961e-07\\
1.59499999999997	-2.92718360093896e-07\\
1.595	-2.9271836009383e-07\\
1.59999999999999	-2.80756674576221e-07\\
1.6	-2.80756674576188e-07\\
1.60499999999999	-2.69277727852186e-07\\
1.60599999999999	-2.67037371457235e-07\\
1.606	-2.67037371457203e-07\\
1.60699999999998	-2.64815607498324e-07\\
1.607	-2.64815607498293e-07\\
1.60799999999999	-2.62612849555249e-07\\
1.60899999999999	-2.60428881715541e-07\\
1.60999999999998	-2.58263489927241e-07\\
1.61	-2.5826348992721e-07\\
1.61199999999999	-2.53987587382857e-07\\
1.61399999999997	-2.49783465549024e-07\\
1.61599999999999	-2.45649476183591e-07\\
1.616	-2.45649476183562e-07\\
1.61999999999997	-2.37585438831863e-07\\
1.61999999999999	-2.37585438831834e-07\\
1.62	-2.37585438831805e-07\\
1.62399999999997	-2.29782828014192e-07\\
1.624	-2.29782828014144e-07\\
1.62599999999999	-2.25975717904888e-07\\
1.626	-2.25975717904861e-07\\
1.62799999999999	-2.22231543444522e-07\\
1.62999999999998	-2.18550973730036e-07\\
1.63199999999999	-2.14932565779908e-07\\
1.632	-2.14932565779882e-07\\
1.63599999999998	-2.07876584631267e-07\\
1.63999999999995	-2.01052533680171e-07\\
1.63999999999998	-2.0105253368013e-07\\
1.64	-2.0105253368009e-07\\
1.645	-1.92832350197346e-07\\
1.64500000000001	-1.92832350197323e-07\\
1.64599999999999	-1.9122800961036e-07\\
1.646	-1.91228009610337e-07\\
1.64699999999999	-1.89636983257357e-07\\
1.64799999999999	-1.88059567279348e-07\\
1.64999999999997	-1.84944949354369e-07\\
1.65199999999999	-1.81882934737465e-07\\
1.652	-1.81882934737443e-07\\
1.65299999999998	-1.80371277732718e-07\\
1.653	-1.80371277732697e-07\\
1.65399999999999	-1.78872322973086e-07\\
1.65499999999997	-1.77385923539781e-07\\
1.65699999999995	-1.7445020913795e-07\\
1.65899999999998	-1.71562983569048e-07\\
1.659	-1.71562983569027e-07\\
1.65999999999999	-1.70137199636775e-07\\
1.66	-1.70137199636754e-07\\
1.66099999999999	-1.68723114903204e-07\\
1.66199999999998	-1.67320590768011e-07\\
1.66399999999995	-1.64549675564482e-07\\
1.66599999999999	-1.61823367678176e-07\\
1.666	-1.61823367678156e-07\\
1.66999999999995	-1.56506437506861e-07\\
1.67399999999991	-1.51367583346046e-07\\
1.67999999999998	-1.43975637480808e-07\\
1.68	-1.43975637480791e-07\\
1.68199999999998	-1.41592131203646e-07\\
1.682	-1.41592131203629e-07\\
1.68399999999998	-1.39247292542355e-07\\
1.68599999999997	-1.36940202052067e-07\\
1.68599999999999	-1.36940202052049e-07\\
1.686	-1.36940202052033e-07\\
1.68999999999997	-1.32440842633399e-07\\
1.69199999999999	-1.30248104741518e-07\\
1.692	-1.30248104741502e-07\\
1.69599999999997	-1.25972213951827e-07\\
1.69799999999999	-1.23887384674211e-07\\
1.698	-1.23887384674196e-07\\
1.69999999999999	-1.21836871740851e-07\\
1.7	-1.21836871740837e-07\\
1.70199999999999	-1.19819871251032e-07\\
1.70399999999998	-1.17835592431731e-07\\
1.70599999999999	-1.15883257338806e-07\\
1.706	-1.15883257338792e-07\\
1.70999999999998	-1.12075752908577e-07\\
1.71099999999998	-1.1114403089721e-07\\
1.711	-1.11144030897197e-07\\
1.71499999999997	-1.0749501946948e-07\\
1.715	-1.07495019469458e-07\\
1.71699999999998	-1.05715990597716e-07\\
1.717	-1.05715990597704e-07\\
1.71899999999998	-1.03966345737004e-07\\
1.71999999999999	-1.03102327506724e-07\\
1.72	-1.03102327506712e-07\\
1.72199999999999	-1.01395476024732e-07\\
1.72399999999997	-9.97163146860125e-08\\
1.72599999999999	-9.80641851688996e-08\\
1.726	-9.8064185168888e-08\\
1.72899999999998	-9.56373482049413e-08\\
1.729	-9.56373482049299e-08\\
1.73199999999998	-9.32719104375856e-08\\
1.73499999999997	-9.09657851911036e-08\\
1.73999999999998	-8.72485462369585e-08\\
1.74	-8.72485462369481e-08\\
1.74599999999997	-8.29851061664371e-08\\
1.74599999999998	-8.29851061664257e-08\\
1.746	-8.29851061664141e-08\\
1.74999999999998	-8.02585158985349e-08\\
1.75	-8.02585158985254e-08\\
1.75199999999998	-7.89297272427249e-08\\
1.752	-7.89297272427156e-08\\
1.75399999999998	-7.76232453177377e-08\\
1.75599999999997	-7.6338557912159e-08\\
1.75799999999998	-7.50751613593442e-08\\
1.758	-7.50751613593353e-08\\
1.75999999999999	-7.38325603466569e-08\\
1.76	-7.38325603466481e-08\\
1.76199999999999	-7.26102677141145e-08\\
1.76399999999998	-7.1407804257055e-08\\
1.76599999999999	-7.02246985450567e-08\\
1.766	-7.02246985450484e-08\\
1.76899999999998	-6.8486817437662e-08\\
1.769	-6.84868174376539e-08\\
1.77199999999998	-6.67929048806059e-08\\
1.77499999999996	-6.5141466585487e-08\\
1.77499999999998	-6.51414665854777e-08\\
1.775	-6.51414665854685e-08\\
1.78	-6.24795162982505e-08\\
1.78000000000002	-6.24795162982431e-08\\
1.78499999999998	-5.99249944998296e-08\\
1.785	-5.99249944998225e-08\\
1.78600000000001	-5.94264261737771e-08\\
1.78600000000003	-5.94264261737701e-08\\
1.78700000000005	-5.89319954077242e-08\\
1.78800000000006	-5.84417942311079e-08\\
1.79000000000009	-5.7473888876745e-08\\
1.79200000000003	-5.65223306397248e-08\\
1.79200000000004	-5.65223306397181e-08\\
1.79600000000011	-5.46667695547242e-08\\
1.79799999999998	-5.37620392268632e-08\\
1.798	-5.37620392268568e-08\\
1.8	-5.28722007837155e-08\\
1.80000000000002	-5.28722007837092e-08\\
1.80200000000002	-5.19969053654383e-08\\
1.80400000000002	-5.11358098090003e-08\\
1.806	-5.02885765185956e-08\\
1.80600000000002	-5.02885765185896e-08\\
1.80999999999998	-4.86362758870654e-08\\
1.81	-4.86362758870596e-08\\
1.81399999999997	-4.70393145393403e-08\\
1.81799999999994	-4.54951878674641e-08\\
1.82	-4.47421775616881e-08\\
1.82000000000001	-4.47421775616828e-08\\
1.826	-4.25558305689674e-08\\
1.82600000000001	-4.25558305689624e-08\\
1.827	-4.22017639742739e-08\\
1.82700000000001	-4.22017639742689e-08\\
1.828	-4.18507262349338e-08\\
1.82899999999999	-4.15026829396494e-08\\
1.83099999999996	-4.0815443523925e-08\\
1.83200000000001	-4.04761800455945e-08\\
1.83200000000003	-4.04761800455897e-08\\
1.83599999999998	-3.91473950511301e-08\\
1.83800000000001	-3.84995090327869e-08\\
1.83800000000003	-3.84995090327824e-08\\
1.83999999999999	-3.78622872354582e-08\\
1.84	-3.78622872354538e-08\\
1.84199999999996	-3.72354798372176e-08\\
1.84399999999992	-3.66188410957449e-08\\
1.84599999999999	-3.60121292554251e-08\\
1.846	-3.60121292554209e-08\\
1.84999999999992	-3.48289010140466e-08\\
1.85399999999984	-3.36853017625717e-08\\
1.85499999999998	-3.34053824571342e-08\\
1.855	-3.34053824571302e-08\\
1.85599999999998	-3.31278001407826e-08\\
1.856	-3.31278001407786e-08\\
1.85699999999999	-3.2852527619997e-08\\
1.85799999999997	-3.25795379025201e-08\\
1.85999999999995	-3.20403000752706e-08\\
1.85999999999998	-3.20403000752638e-08\\
1.86	-3.20403000752569e-08\\
1.86399999999995	-3.09880554735901e-08\\
1.86599999999999	-3.04746361639989e-08\\
1.866	-3.04746361639953e-08\\
1.86799999999998	-2.99697042299132e-08\\
1.868	-2.99697042299096e-08\\
1.86999999999998	-2.94733499071916e-08\\
1.87199999999996	-2.89853785981544e-08\\
1.87299999999998	-2.87444766622487e-08\\
1.873	-2.87444766622453e-08\\
1.87699999999996	-2.78008784246803e-08\\
1.87999999999999	-2.71135450349128e-08\\
1.88	-2.71135450349096e-08\\
1.88399999999997	-2.62231013986987e-08\\
1.88499999999998	-2.60049874485378e-08\\
1.885	-2.60049874485347e-08\\
1.886	-2.57886292648668e-08\\
1.88600000000002	-2.57886292648638e-08\\
1.88700000000002	-2.55740666113488e-08\\
1.88800000000002	-2.53613394260269e-08\\
1.88999999999998	-2.4941308239916e-08\\
1.89	-2.4941308239913e-08\\
1.892	-2.45283710327764e-08\\
1.89200000000002	-2.45283710327735e-08\\
1.89400000000002	-2.41223659120891e-08\\
1.89600000000003	-2.37231337030588e-08\\
1.898	-2.33305178851724e-08\\
1.89800000000002	-2.33305178851696e-08\\
1.9	-2.29443645339956e-08\\
1.90000000000002	-2.29443645339929e-08\\
1.902	-2.2564522258656e-08\\
1.90399999999998	-2.2190842140487e-08\\
1.906	-2.18231776767582e-08\\
1.90600000000002	-2.18231776767556e-08\\
1.90999999999998	-2.11061470402425e-08\\
1.91399999999995	-2.04131313808562e-08\\
1.91399999999997	-2.04131313808518e-08\\
1.914	-2.04131313808474e-08\\
1.91999999999998	-1.94162682523969e-08\\
1.92	-1.94162682523946e-08\\
1.92499999999998	-1.86224195898359e-08\\
1.925	-1.86224195898337e-08\\
1.92599999999998	-1.84674834298463e-08\\
1.926	-1.84674834298441e-08\\
1.92699999999999	-1.8313833066864e-08\\
1.92799999999997	-1.81614971001522e-08\\
1.92999999999995	-1.78607087608587e-08\\
1.93199999999998	-1.75650004867345e-08\\
1.932	-1.75650004867324e-08\\
1.93599999999995	-1.69883623530699e-08\\
1.9399999999999	-1.64306783229831e-08\\
1.93999999999999	-1.64306783229713e-08\\
1.94	-1.64306783229694e-08\\
1.94299999999998	-1.60243266729119e-08\\
1.943	-1.602432667291e-08\\
1.94599999999998	-1.56277856954226e-08\\
1.946	-1.56277856954198e-08\\
1.94899999999998	-1.52410380971515e-08\\
1.95199999999997	-1.48640752480954e-08\\
1.95199999999998	-1.48640752480932e-08\\
1.952	-1.4864075248091e-08\\
1.95799999999997	-1.41381819830471e-08\\
1.95999999999998	-1.39041748992468e-08\\
1.96	-1.39041748992451e-08\\
1.96599999999996	-1.32247410380267e-08\\
1.96599999999998	-1.32247410380248e-08\\
1.966	-1.32247410380228e-08\\
1.97199999999996	-1.2578464393245e-08\\
1.97199999999998	-1.25784643932432e-08\\
1.972	-1.25784643932414e-08\\
1.97799999999996	-1.19641898798447e-08\\
1.97799999999998	-1.1964189879843e-08\\
1.978	-1.19641898798412e-08\\
1.98	-1.17661654672371e-08\\
1.98000000000002	-1.17661654672357e-08\\
1.98200000000002	-1.15713774588398e-08\\
1.98400000000002	-1.13797494839859e-08\\
1.986	-1.11912064141954e-08\\
1.98600000000002	-1.11912064141941e-08\\
1.99000000000002	-1.08235038738479e-08\\
1.99400000000003	-1.04681165230686e-08\\
1.995	-1.03811282006563e-08\\
1.99500000000001	-1.03811282006551e-08\\
1.99999999999999	-9.95691227019017e-09\\
2	-9.95691227018899e-09\\
2.00099999999997	-9.87415601302283e-09\\
2.001	-9.87415601302048e-09\\
2.00199999999999	-9.79207631566493e-09\\
2.00299999999997	-9.71066513273897e-09\\
2.00499999999995	-9.54981645834466e-09\\
2.00599999999997	-9.47036320173591e-09\\
2.006	-9.47036320173366e-09\\
2.00999999999995	-9.15920134191744e-09\\
2.01199999999997	-9.00755833307511e-09\\
2.012	-9.00755833307297e-09\\
2.01299999999998	-8.93269513185397e-09\\
2.01300000000001	-8.93269513185185e-09\\
2.014	-8.85846099580636e-09\\
2.01499999999999	-8.78484864893714e-09\\
2.01699999999996	-8.63946052385991e-09\\
2.01999999999997	-8.42586332845568e-09\\
2.02	-8.42586332845369e-09\\
2.02399999999995	-8.14914715807546e-09\\
2.02599999999997	-8.0141296679016e-09\\
2.026	-8.01412966789969e-09\\
2.02999999999995	-7.7508143724346e-09\\
2.03	-7.75081437243172e-09\\
2.03399999999995	-7.49631810159424e-09\\
2.03599999999997	-7.37225184535954e-09\\
2.036	-7.37225184535779e-09\\
2.03999999999995	-7.13023987191646e-09\\
2.04	-7.13023987191334e-09\\
2.04399999999995	-6.89607363978448e-09\\
2.04599999999997	-6.78181744392865e-09\\
2.046	-6.78181744392704e-09\\
2.04799999999997	-6.66945002629583e-09\\
2.048	-6.66945002629424e-09\\
2.04999999999996	-6.55899146794323e-09\\
2.05199999999993	-6.450398463156e-09\\
2.05599999999987	-6.23863953290441e-09\\
2.05899999999997	-6.08440591409081e-09\\
2.059	-6.08440591408937e-09\\
2.05999999999997	-6.03384111045324e-09\\
2.06	-6.03384111045181e-09\\
2.06099999999999	-5.98369121589198e-09\\
2.06199999999998	-5.93395131288756e-09\\
2.06399999999995	-5.83568202113935e-09\\
2.06499999999997	-5.78714300080149e-09\\
2.065	-5.78714300080012e-09\\
2.06599999999997	-5.73899470825349e-09\\
2.066	-5.73899470825212e-09\\
2.06699999999999	-5.69124599249692e-09\\
2.06799999999998	-5.64390574134267e-09\\
2.06999999999995	-5.55043211306818e-09\\
2.07099999999997	-5.50428957438758e-09\\
2.071	-5.50428957438628e-09\\
2.07499999999995	-5.32357617608778e-09\\
2.07699999999997	-5.23547166903123e-09\\
2.077	-5.23547166902999e-09\\
2.07999999999997	-5.10603279254529e-09\\
2.08	-5.10603279254408e-09\\
2.08299999999998	-4.9797540912234e-09\\
2.08599999999995	-4.8565241680537e-09\\
2.086	-4.85652416805177e-09\\
2.08799999999997	-4.77605678697469e-09\\
2.088	-4.77605678697356e-09\\
2.08999999999997	-4.69695635957358e-09\\
2.09199999999993	-4.61919187288798e-09\\
2.09399999999997	-4.54273283904935e-09\\
2.094	-4.54273283904828e-09\\
2.09799999999993	-4.3936117275811e-09\\
2.09999999999997	-4.32089118643073e-09\\
2.1	-4.3208911864297e-09\\
2.10399999999993	-4.17898757203687e-09\\
2.10599999999997	-4.10974886492132e-09\\
2.106	-4.10974886492035e-09\\
2.10999999999993	-3.9747173915453e-09\\
2.112	-3.90891055089514e-09\\
2.11200000000003	-3.90891055089421e-09\\
2.11599999999996	-3.78058565402447e-09\\
2.11699999999997	-3.74917121257253e-09\\
2.117	-3.74917121257164e-09\\
2.11999999999997	-3.65647879810409e-09\\
2.12	-3.65647879810322e-09\\
2.12299999999998	-3.56604941518713e-09\\
2.12599999999995	-3.47780329135854e-09\\
2.126	-3.47780329135713e-09\\
2.12899999999997	-3.39173658233563e-09\\
2.129	-3.39173658233483e-09\\
2.13199999999996	-3.30784737000392e-09\\
2.13499999999993	-3.22606165017266e-09\\
2.13499999999997	-3.22606165017175e-09\\
2.135	-3.22606165017085e-09\\
2.13999999999997	-3.09423140190775e-09\\
2.14	-3.09423140190701e-09\\
2.14499999999998	-2.9677214349161e-09\\
2.14599999999997	-2.94303037042066e-09\\
2.146	-2.94303037041996e-09\\
2.14699999999997	-2.91854421395811e-09\\
2.14699999999999	-2.91854421395741e-09\\
2.14799999999998	-2.89426752361201e-09\\
2.14899999999997	-2.87019791980234e-09\\
2.15099999999995	-2.82267055556964e-09\\
2.15299999999999	-2.77594348797241e-09\\
2.15300000000002	-2.77594348797175e-09\\
2.15699999999997	-2.68481727228376e-09\\
2.15999999999997	-2.61843934966851e-09\\
2.16	-2.61843934966789e-09\\
2.16399999999995	-2.53244643891227e-09\\
2.16599999999997	-2.49048811397047e-09\\
2.166	-2.49048811396988e-09\\
2.16999999999995	-2.40865968773891e-09\\
2.17	-2.40865968773797e-09\\
2.17399999999995	-2.32957188049521e-09\\
2.17499999999997	-2.31021352298225e-09\\
2.175	-2.3102135229817e-09\\
2.17899999999995	-2.23437754607426e-09\\
2.17999999999997	-2.21580862418798e-09\\
2.18	-2.21580862418745e-09\\
2.18399999999995	-2.14303862332668e-09\\
2.18599999999997	-2.10753212267536e-09\\
2.186	-2.10753212267486e-09\\
2.187	-2.08999735245014e-09\\
2.18700000000002	-2.08999735244965e-09\\
2.18800000000002	-2.07261258282432e-09\\
2.18800000000005	-2.07261258282383e-09\\
2.18900000000004	-2.05537610984398e-09\\
2.19000000000004	-2.03828624415599e-09\\
2.19200000000003	-2.00453964892216e-09\\
2.19600000000001	-1.93873298928637e-09\\
2.2	-1.87508939265602e-09\\
2.20000000000003	-1.87508939265558e-09\\
2.20399999999997	-1.81350904606906e-09\\
2.204	-1.81350904606863e-09\\
2.20499999999997	-1.79842495596648e-09\\
2.205	-1.79842495596605e-09\\
2.20599999999999	-1.78346228932641e-09\\
2.20600000000003	-1.78346228932579e-09\\
2.20700000000002	-1.76862379597292e-09\\
2.20800000000001	-1.7539122379575e-09\\
2.20999999999998	-1.7248641726959e-09\\
2.21200000000003	-1.69630670513471e-09\\
2.21200000000006	-1.69630670513431e-09\\
2.21600000000001	-1.64061896749459e-09\\
2.21800000000003	-1.6134668648184e-09\\
2.21800000000006	-1.61346686481801e-09\\
2.21999999999997	-1.58676168647414e-09\\
2.22	-1.58676168647377e-09\\
2.22199999999992	-1.56049296273329e-09\\
2.22399999999983	-1.53465039483897e-09\\
2.226	-1.50922385110997e-09\\
2.22600000000003	-1.50922385110961e-09\\
2.22999999999986	-1.45963621722653e-09\\
2.23299999999997	-1.42353954717499e-09\\
2.233	-1.42353954717465e-09\\
2.23699999999983	-1.37680885088518e-09\\
2.23899999999997	-1.35402207535656e-09\\
2.239	-1.35402207535624e-09\\
2.24	-1.34276939779072e-09\\
2.24000000000003	-1.3427693977904e-09\\
2.24100000000003	-1.3316090535144e-09\\
2.24200000000003	-1.32053994865601e-09\\
2.24400000000004	-1.29867112647859e-09\\
2.246	-1.27715435750312e-09\\
2.24600000000003	-1.27715435750281e-09\\
2.25000000000004	-1.23519168755036e-09\\
2.252	-1.21474141278389e-09\\
2.25200000000003	-1.2147414127836e-09\\
2.25600000000004	-1.17486289297273e-09\\
2.25999999999997	-1.13629518070236e-09\\
2.26	-1.13629518070209e-09\\
2.26199999999997	-1.11748389715364e-09\\
2.262	-1.11748389715338e-09\\
2.26399999999996	-1.0989777879204e-09\\
2.26599999999993	-1.08076959760946e-09\\
2.26599999999997	-1.08076959760914e-09\\
2.266	-1.08076959760882e-09\\
2.26999999999994	-1.04525942023253e-09\\
2.272	-1.02795373197208e-09\\
2.27200000000003	-1.02795373197183e-09\\
2.27499999999997	-1.00253782658167e-09\\
2.275	-1.00253782658143e-09\\
2.27799999999994	-9.77753186379527e-10\\
2.27999999999997	-9.61569976505633e-10\\
2.28	-9.61569976505405e-10\\
2.28299999999994	-9.37789124946285e-10\\
2.28599999999989	-9.14582420408248e-10\\
2.28599999999997	-9.1458242040758e-10\\
2.286	-9.14582420407362e-10\\
2.28699999999997	-9.06973049958816e-10\\
2.287	-9.069730499586e-10\\
2.28799999999999	-8.99428773710269e-10\\
2.28899999999997	-8.91948852116791e-10\\
2.29099999999995	-8.77179146689534e-10\\
2.291	-8.77179146689196e-10\\
2.29499999999995	-8.48380148659471e-10\\
2.29699999999997	-8.34339565336599e-10\\
2.297	-8.34339565336401e-10\\
2.29999999999997	-8.13711820037174e-10\\
2.3	-8.13711820036981e-10\\
2.30299999999998	-7.93587689175409e-10\\
2.30599999999995	-7.73949420208547e-10\\
2.306	-7.73949420208246e-10\\
2.30999999999997	-7.48520243150941e-10\\
2.31	-7.48520243150763e-10\\
2.31399999999997	-7.23942746888686e-10\\
2.31599999999997	-7.11961282718487e-10\\
2.316	-7.11961282718318e-10\\
2.31999999999997	-6.88589433850948e-10\\
2.32	-6.88589433850785e-10\\
2.32399999999997	-6.65975272601678e-10\\
2.32599999999997	-6.54941196445864e-10\\
2.326	-6.54941196445709e-10\\
2.32999999999997	-6.33422199008811e-10\\
2.33199999999997	-6.22935034795106e-10\\
2.332	-6.22935034794958e-10\\
2.33599999999997	-6.02484816352391e-10\\
2.33999999999994	-5.82706796871669e-10\\
2.34	-5.82706796871356e-10\\
2.34000000000003	-5.82706796871218e-10\\
2.34499999999997	-5.58882392537186e-10\\
2.345	-5.58882392537054e-10\\
2.34600000000003	-5.54232562163309e-10\\
2.34600000000006	-5.54232562163177e-10\\
2.34700000000009	-5.49621320185701e-10\\
2.34800000000012	-5.45049524904668e-10\\
2.34899999999997	-5.40516728236649e-10\\
2.349	-5.40516728236521e-10\\
2.35100000000006	-5.31566357473108e-10\\
2.35300000000011	-5.22766698891713e-10\\
2.35499999999997	-5.14114302553359e-10\\
2.355	-5.14114302553237e-10\\
2.35800000000003	-5.01404419877563e-10\\
2.35800000000006	-5.01404419877444e-10\\
2.35999999999997	-4.93105461450458e-10\\
2.36	-4.93105461450341e-10\\
2.36199999999992	-4.849421365681e-10\\
2.36399999999983	-4.76911244785728e-10\\
2.36599999999997	-4.69009637560487e-10\\
2.366	-4.69009637560376e-10\\
2.36999999999983	-4.5359967816564e-10\\
2.37399999999966	-4.3870583316e-10\\
2.37799999999997	-4.24304743013925e-10\\
2.378	-4.24304743013824e-10\\
2.37999999999997	-4.17281893796132e-10\\
2.38	-4.17281893796033e-10\\
2.38199999999997	-4.10373822535638e-10\\
2.38399999999995	-4.03577820478402e-10\\
2.38599999999997	-3.9689122322555e-10\\
2.386	-3.96891223225456e-10\\
2.38999999999995	-3.83850813957594e-10\\
2.39	-3.83850813957444e-10\\
2.39399999999995	-3.71247157499278e-10\\
2.396	-3.65102908450872e-10\\
2.39600000000002	-3.65102908450786e-10\\
2.39999999999997	-3.53117523537237e-10\\
2.4	-3.53117523537153e-10\\
2.40399999999995	-3.41520690614927e-10\\
2.40599999999997	-3.35862274369834e-10\\
2.406	-3.35862274369754e-10\\
2.40699999999997	-3.33067883855288e-10\\
2.407	-3.33067883855209e-10\\
2.40799999999998	-3.30297397860602e-10\\
2.40899999999997	-3.27550544839006e-10\\
2.41099999999995	-3.22126663124085e-10\\
2.41299999999997	-3.16794112256723e-10\\
2.413	-3.16794112256648e-10\\
2.41499999999997	-3.11550801618787e-10\\
2.415	-3.11550801618713e-10\\
2.41699999999997	-3.06394675577696e-10\\
2.41899999999995	-3.01323712653857e-10\\
2.41999999999997	-2.98819544794563e-10\\
2.42	-2.98819544794493e-10\\
2.42399999999995	-2.8900592727624e-10\\
2.426	-2.84217591254418e-10\\
2.42600000000003	-2.8421759125435e-10\\
2.427	-2.81852887036128e-10\\
2.42700000000003	-2.81852887036061e-10\\
2.42800000000002	-2.79508411600243e-10\\
2.429	-2.77183935155039e-10\\
2.43099999999998	-2.72594069882784e-10\\
2.43499999999993	-2.63644431616398e-10\\
2.43599999999997	-2.61453675942175e-10\\
2.436	-2.61453675942113e-10\\
2.43999999999997	-2.52870827160071e-10\\
2.44	-2.52870827160011e-10\\
2.44399999999998	-2.44566224542467e-10\\
2.446	-2.40514178717604e-10\\
2.44600000000003	-2.40514178717547e-10\\
2.448	-2.36529117583715e-10\\
2.44800000000002	-2.36529117583659e-10\\
2.44999999999999	-2.32611753307324e-10\\
2.45000000000002	-2.32611753307269e-10\\
2.45199999999998	-2.28760550070214e-10\\
2.45399999999995	-2.24973998003557e-10\\
2.45600000000002	-2.21250612574871e-10\\
2.45600000000005	-2.21250612574819e-10\\
2.45999999999998	-2.13987526843694e-10\\
2.46000000000001	-2.13987526843643e-10\\
2.46399999999994	-2.06959901743582e-10\\
2.46499999999997	-2.05238486699303e-10\\
2.465	-2.05238486699255e-10\\
2.46599999999998	-2.03530928644891e-10\\
2.46600000000001	-2.03530928644842e-10\\
2.46699999999999	-2.01837541406044e-10\\
2.46799999999998	-2.00158640184797e-10\\
2.46999999999996	-1.96843638999486e-10\\
2.47199999999998	-1.93584625429924e-10\\
2.47200000000001	-1.93584625429878e-10\\
2.47599999999996	-1.87229471830543e-10\\
2.47999999999991	-1.81083212218112e-10\\
2.47999999999997	-1.81083212218014e-10\\
2.48	-1.81083212217971e-10\\
2.48499999999997	-1.73679489399505e-10\\
2.485	-1.73679489399464e-10\\
2.48599999999997	-1.72234498173133e-10\\
2.486	-1.72234498173092e-10\\
2.48699999999999	-1.70801498759175e-10\\
2.48799999999998	-1.69380757884777e-10\\
2.48999999999995	-1.66575495952824e-10\\
2.49199999999997	-1.63817612563486e-10\\
2.492	-1.63817612563447e-10\\
2.494	-1.61106026492917e-10\\
2.49400000000002	-1.61106026492879e-10\\
2.49600000000002	-1.58439674667629e-10\\
2.49800000000001	-1.55817511733848e-10\\
2.49999999999997	-1.53238509662561e-10\\
2.5	-1.53238509662525e-10\\
2.50399999999999	-1.48205960263291e-10\\
2.50599999999997	-1.45750439904683e-10\\
2.506	-1.45750439904648e-10\\
2.50999999999999	-1.40961607966064e-10\\
2.51399999999998	-1.36333164834736e-10\\
2.51999999999997	-1.29675415851786e-10\\
2.52	-1.29675415851756e-10\\
2.52299999999997	-1.2646837729115e-10\\
2.523	-1.2646837729112e-10\\
2.52599999999996	-1.23338767245638e-10\\
2.526	-1.233387672456e-10\\
2.52899999999997	-1.20286449231785e-10\\
2.53199999999993	-1.17311355139667e-10\\
2.53199999999997	-1.17311355139635e-10\\
2.532	-1.17311355139603e-10\\
2.53799999999993	-1.11582406597712e-10\\
2.53799999999997	-1.11582406597681e-10\\
2.538	-1.11582406597651e-10\\
2.53999999999997	-1.09735558574493e-10\\
2.54	-1.09735558574467e-10\\
2.54199999999998	-1.0791889443866e-10\\
2.54399999999995	-1.06131701929468e-10\\
2.54599999999997	-1.04373280371184e-10\\
2.546	-1.04373280371159e-10\\
2.54999999999995	-1.00943952128948e-10\\
2.55199999999997	-9.92726880049976e-11\\
2.552	-9.9272688004974e-11\\
2.55499999999997	-9.68181949962042e-11\\
2.555	-9.68181949961812e-11\\
2.55799999999998	-9.44246652301689e-11\\
2.55800000000001	-9.44246652301465e-11\\
2.55999999999997	-9.28618023354243e-11\\
2.56	-9.28618023354023e-11\\
2.56199999999997	-9.13244820000885e-11\\
2.56399999999994	-8.98121015152303e-11\\
2.56599999999997	-8.83240679462208e-11\\
2.566	-8.83240679461998e-11\\
2.56999999999994	-8.54220587440507e-11\\
2.56999999999997	-8.5422058744027e-11\\
2.57	-8.54220587440032e-11\\
2.57399999999994	-8.26172443914214e-11\\
2.57799999999988	-7.99052259338734e-11\\
2.57999999999997	-7.85826803907801e-11\\
2.58	-7.85826803907615e-11\\
2.58099999999997	-7.79295453081817e-11\\
2.581	-7.79295453081632e-11\\
2.58199999999998	-7.72817498450949e-11\\
2.58299999999997	-7.66392304810246e-11\\
2.58499999999995	-7.53697686664986e-11\\
2.58599999999997	-7.47427017932331e-11\\
2.586	-7.47427017932153e-11\\
2.58999999999995	-7.22869271288568e-11\\
2.59	-7.22869271288283e-11\\
2.59199999999997	-7.10901189482887e-11\\
2.592	-7.10901189482719e-11\\
2.59399999999997	-6.99134019071945e-11\\
2.59599999999995	-6.87563146689936e-11\\
2.59799999999997	-6.76184035931566e-11\\
2.598	-6.76184035931406e-11\\
2.59999999999997	-6.64992225634527e-11\\
2.6	-6.64992225634369e-11\\
2.60199999999997	-6.53983328065227e-11\\
2.60399999999995	-6.43153027141899e-11\\
2.60599999999997	-6.32497076803159e-11\\
2.606	-6.32497076803009e-11\\
2.60999999999995	-6.11715512064784e-11\\
2.61	-6.11715512064525e-11\\
2.61399999999994	-5.91629968912518e-11\\
2.61599999999997	-5.81838319959368e-11\\
2.616	-5.8183831995923e-11\\
2.61999999999994	-5.62738071762015e-11\\
2.61999999999997	-5.62738071761869e-11\\
2.62	-5.62738071761723e-11\\
2.62399999999995	-5.44257030964152e-11\\
2.62499999999997	-5.3973010455304e-11\\
2.625	-5.39730104552912e-11\\
2.62599999999997	-5.35239618843806e-11\\
2.626	-5.35239618843679e-11\\
2.62699999999999	-5.30786399126141e-11\\
2.62799999999998	-5.26371274309284e-11\\
2.62999999999995	-5.17653582156297e-11\\
2.63199999999997	-5.09083124579923e-11\\
2.632	-5.09083124579802e-11\\
2.63599999999995	-4.92370529446546e-11\\
2.63899999999997	-4.80197990243561e-11\\
2.639	-4.80197990243447e-11\\
2.63999999999997	-4.76207277519393e-11\\
2.64	-4.7620727751928e-11\\
2.64099999999999	-4.72249310389222e-11\\
2.64199999999998	-4.68323700916334e-11\\
2.64399999999995	-4.60568019073524e-11\\
2.64599999999997	-4.52937191345926e-11\\
2.646	-4.52937191345819e-11\\
2.64999999999995	-4.38055314512521e-11\\
2.65099999999997	-4.34413618888486e-11\\
2.651	-4.34413618888383e-11\\
2.65499999999995	-4.20151222173587e-11\\
2.6589999999999	-4.06359173117102e-11\\
2.65999999999997	-4.02982101933277e-11\\
2.66	-4.02982101933181e-11\\
2.66599999999997	-3.83290197128048e-11\\
2.666	-3.83290197127957e-11\\
2.66799999999997	-3.76939490847073e-11\\
2.668	-3.76939490846984e-11\\
2.66999999999996	-3.70696668296457e-11\\
2.67199999999993	-3.64559281848261e-11\\
2.67199999999996	-3.64559281848151e-11\\
2.672	-3.64559281848042e-11\\
2.67599999999993	-3.52591232972855e-11\\
2.67799999999997	-3.46755878441791e-11\\
2.678	-3.46755878441709e-11\\
2.67999999999997	-3.41016573998113e-11\\
2.68	-3.41016573998032e-11\\
2.68199999999998	-3.35371069555499e-11\\
2.68399999999995	-3.29817151771637e-11\\
2.68599999999997	-3.2435264321158e-11\\
2.686	-3.24352643211503e-11\\
2.68999999999995	-3.13695589191206e-11\\
2.69399999999989	-3.03395464007265e-11\\
2.69499999999997	-3.00874297711861e-11\\
2.695	-3.0087429771179e-11\\
2.69699999999997	-2.95894866108812e-11\\
2.697	-2.95894866108742e-11\\
2.69899999999996	-2.90997679484516e-11\\
2.69999999999997	-2.88579326700239e-11\\
2.7	-2.88579326700171e-11\\
2.70199999999997	-2.83801917047306e-11\\
2.70399999999993	-2.79102011008219e-11\\
2.70599999999997	-2.74477765966589e-11\\
2.706	-2.74477765966524e-11\\
2.70899999999997	-2.67685145486583e-11\\
2.709	-2.67685145486519e-11\\
2.71199999999996	-2.61064378928702e-11\\
2.71299999999997	-2.58894632759895e-11\\
2.713	-2.58894632759833e-11\\
2.71599999999997	-2.52493944966632e-11\\
2.71899999999993	-2.46251710153901e-11\\
2.71999999999997	-2.44205221265747e-11\\
2.72	-2.44205221265689e-11\\
2.72599999999993	-2.32272021525107e-11\\
2.726	-2.32272021524986e-11\\
2.72600000000002	-2.32272021524931e-11\\
2.72999999999997	-2.24640403465451e-11\\
2.73	-2.24640403465398e-11\\
2.73199999999999	-2.20921176702104e-11\\
2.73200000000002	-2.20921176702051e-11\\
2.73400000000002	-2.17264385595445e-11\\
2.73600000000001	-2.13668596486061e-11\\
2.73799999999999	-2.10132399631027e-11\\
2.73800000000002	-2.10132399630977e-11\\
2.73999999999997	-2.06654408669829e-11\\
2.74	-2.0665440866978e-11\\
2.74199999999995	-2.03233260061008e-11\\
2.7439999999999	-1.99867612529741e-11\\
2.74599999999997	-1.96556146560785e-11\\
2.746	-1.96556146560738e-11\\
2.7499999999999	-1.90098022958202e-11\\
2.7539999999998	-1.83856196517471e-11\\
2.75499999999997	-1.82328381838975e-11\\
2.755	-1.82328381838931e-11\\
2.75999999999997	-1.74877688372908e-11\\
2.76	-1.74877688372867e-11\\
2.76499999999998	-1.67727683190952e-11\\
2.76500000000001	-1.67727683190912e-11\\
2.76599999999997	-1.66332210223947e-11\\
2.766	-1.66332210223908e-11\\
2.76699999999999	-1.64948318123411e-11\\
2.76700000000002	-1.64948318123371e-11\\
2.76800000000001	-1.63576264476108e-11\\
2.76899999999999	-1.6221591480655e-11\\
2.77099999999997	-1.59529795220935e-11\\
2.77299999999999	-1.56888906262995e-11\\
2.77300000000002	-1.56888906262958e-11\\
2.77699999999997	-1.51738696137965e-11\\
2.77999999999997	-1.47987193386536e-11\\
2.78	-1.47987193386501e-11\\
2.78399999999995	-1.43127103941947e-11\\
2.784	-1.43127103941897e-11\\
2.786	-1.40755731578462e-11\\
2.78600000000003	-1.40755731578429e-11\\
2.78800000000004	-1.3842356056594e-11\\
2.79000000000004	-1.36131007671595e-11\\
2.792	-1.33877174091302e-11\\
2.79200000000003	-1.3387717409127e-11\\
2.79600000000004	-1.29482145263775e-11\\
2.79999999999997	-1.25231581096245e-11\\
2.8	-1.25231581096216e-11\\
2.80400000000001	-1.21118815179885e-11\\
2.80599999999997	-1.19112082681299e-11\\
2.806	-1.19112082681271e-11\\
2.81000000000001	-1.15198490743189e-11\\
2.81199999999997	-1.13291223388397e-11\\
2.812	-1.1329122338837e-11\\
2.81299999999997	-1.12349642626375e-11\\
2.813	-1.12349642626349e-11\\
2.81399999999998	-1.1141597383761e-11\\
2.81499999999997	-1.10490125509372e-11\\
2.81699999999995	-1.08661528020932e-11\\
2.81899999999997	-1.0686313325234e-11\\
2.819	-1.06863133252314e-11\\
2.81999999999997	-1.05975041101858e-11\\
2.82	-1.05975041101833e-11\\
2.82099999999999	-1.05094236147193e-11\\
2.82199999999998	-1.04220632057012e-11\\
2.82399999999995	-1.02494684678246e-11\\
2.82599999999997	-1.00796522301221e-11\\
2.826	-1.00796522301197e-11\\
2.82999999999995	-9.74847133810219e-12\\
2.83399999999991	-9.4283824427818e-12\\
2.83499999999997	-9.35003413920577e-12\\
2.835	-9.35003413920355e-12\\
2.83999999999997	-8.96795298622807e-12\\
2.84	-8.96795298622594e-12\\
2.84199999999997	-8.81948917961099e-12\\
2.842	-8.8194891796089e-12\\
2.84399999999996	-8.67343389307826e-12\\
2.84599999999992	-8.52972986423031e-12\\
2.846	-8.52972986422483e-12\\
2.84600000000003	-8.5297298642228e-12\\
2.847	-8.45876209530731e-12\\
2.84700000000003	-8.4587620953053e-12\\
2.84800000000002	-8.38840141777417e-12\\
2.84900000000001	-8.31864093528458e-12\\
2.85099999999998	-8.18089326457407e-12\\
2.853	-8.04546507866881e-12\\
2.85300000000003	-8.0454650786669e-12\\
2.85699999999998	-7.78135567251753e-12\\
2.85999999999997	-7.58897378128907e-12\\
2.86	-7.58897378128727e-12\\
2.86399999999995	-7.33974213704654e-12\\
2.86599999999997	-7.21813510900272e-12\\
2.866	-7.218135109001e-12\\
2.86999999999995	-6.98097331253226e-12\\
2.87	-6.98097331252944e-12\\
2.87099999999997	-6.9229382215505e-12\\
2.871	-6.92293822154886e-12\\
2.87199999999998	-6.86539382472867e-12\\
2.87299999999997	-6.80833448190192e-12\\
2.87499999999995	-6.69564863558482e-12\\
2.87699999999997	-6.5848365036775e-12\\
2.87699999999999	-6.58483650367594e-12\\
2.87999999999997	-6.42203668407538e-12\\
2.88	-6.42203668407385e-12\\
2.88299999999998	-6.26321152773433e-12\\
2.88599999999996	-6.10822092672135e-12\\
2.886	-6.1082209267192e-12\\
2.888	-6.00701427630391e-12\\
2.88800000000003	-6.00701427630248e-12\\
2.89000000000003	-5.90752689152468e-12\\
2.89200000000003	-5.80971976629227e-12\\
2.89600000000002	-5.61899355657615e-12\\
2.89999999999997	-5.43453651189328e-12\\
2.9	-5.43453651189199e-12\\
2.90499999999997	-5.21234141214886e-12\\
2.905	-5.21234141214762e-12\\
2.90599999999997	-5.16897539515079e-12\\
2.90599999999999	-5.16897539514956e-12\\
2.90699999999998	-5.12596926764053e-12\\
2.90799999999997	-5.08333103444278e-12\\
2.90999999999995	-4.99914157066787e-12\\
2.91199999999997	-4.9163739970212e-12\\
2.91199999999999	-4.91637399702003e-12\\
2.91599999999995	-4.75497526945917e-12\\
2.91799999999997	-4.67628083853704e-12\\
2.91799999999999	-4.67628083853594e-12\\
2.91999999999997	-4.5988817200333e-12\\
2.92	-4.59888172003221e-12\\
2.92199999999998	-4.52274756973222e-12\\
2.92399999999996	-4.44784853893951e-12\\
2.926	-4.37415526320013e-12\\
2.92600000000003	-4.37415526319909e-12\\
2.92899999999997	-4.26590614373211e-12\\
2.92899999999999	-4.2659061437311e-12\\
2.93199999999993	-4.16039573693212e-12\\
2.93499999999987	-4.05753096658108e-12\\
2.93499999999997	-4.05753096657774e-12\\
2.935	-4.05753096657678e-12\\
2.93999999999997	-3.89172343657807e-12\\
2.94	-3.89172343657715e-12\\
2.94499999999998	-3.73260741135184e-12\\
2.94599999999997	-3.70155259306461e-12\\
2.946	-3.70155259306373e-12\\
2.95099999999998	-3.55017183054663e-12\\
2.95199999999997	-3.52066232309052e-12\\
2.952	-3.52066232308968e-12\\
2.95699999999998	-3.37678891866988e-12\\
2.95799999999999	-3.34872932271585e-12\\
2.95800000000002	-3.34872932271505e-12\\
2.95999999999997	-3.29330307555662e-12\\
2.96	-3.29330307555584e-12\\
2.96199999999995	-3.2387826844961e-12\\
2.9639999999999	-3.18514677428297e-12\\
2.96599999999997	-3.13237431674945e-12\\
2.966	-3.13237431674871e-12\\
2.9699999999999	-3.02945583386099e-12\\
2.96999999999999	-3.02945583385871e-12\\
2.97000000000002	-3.02945583385799e-12\\
2.97399999999992	-2.92998432362861e-12\\
2.97499999999997	-2.90563663812967e-12\\
2.975	-2.90563663812898e-12\\
2.9789999999999	-2.81025506900498e-12\\
2.97999999999997	-2.78690028439128e-12\\
2.98	-2.78690028439062e-12\\
2.9839999999999	-2.69537490091183e-12\\
2.986	-2.65071712916476e-12\\
2.98600000000003	-2.65071712916413e-12\\
2.98699999999997	-2.62866303338006e-12\\
2.98699999999999	-2.62866303337944e-12\\
2.98799999999998	-2.60679759860873e-12\\
2.98899999999997	-2.58511868172988e-12\\
2.99099999999995	-2.54231192068294e-12\\
2.99299999999999	-2.50022596768247e-12\\
2.99300000000002	-2.50022596768187e-12\\
2.99699999999997	-2.41815076312507e-12\\
2.99999999999997	-2.35836575463967e-12\\
3	-2.35836575463911e-12\\
3.00399999999995	-2.28091399510675e-12\\
3.00599999999997	-2.24312313460389e-12\\
3.006	-2.24312313460336e-12\\
3.00999999999995	-2.16942222671251e-12\\
3.01	-2.16942222671163e-12\\
3.01399999999995	-2.0981897288136e-12\\
3.01599999999997	-2.06346407521942e-12\\
3.01599999999999	-2.06346407521893e-12\\
3.01999999999995	-1.99572588261004e-12\\
3.02	-1.99572588260911e-12\\
3.02399999999995	-1.93018368184909e-12\\
3.02599999999997	-1.89820382521276e-12\\
3.026	-1.89820382521231e-12\\
3.02799999999997	-1.86675263011009e-12\\
3.02799999999999	-1.86675263010965e-12\\
3.02999999999996	-1.83583571715665e-12\\
3.03199999999992	-1.8054409652544e-12\\
3.03599999999985	-1.74617047944877e-12\\
3.03999999999997	-1.68884821277428e-12\\
3.04	-1.68884821277388e-12\\
3.04499999999997	-1.61979838715869e-12\\
3.04499999999999	-1.6197983871583e-12\\
3.04599999999997	-1.60632187072683e-12\\
3.046	-1.60632187072644e-12\\
3.04699999999998	-1.59295719432291e-12\\
3.04799999999996	-1.57970684554389e-12\\
3.04999999999992	-1.5535439472495e-12\\
3.05199999999997	-1.52782291857753e-12\\
3.052	-1.52782291857716e-12\\
3.05599999999992	-1.47766630389402e-12\\
3.05799999999997	-1.45321105378284e-12\\
3.058	-1.45321105378249e-12\\
3.05999999999997	-1.42915833773473e-12\\
3.06	-1.42915833773439e-12\\
3.06199999999997	-1.40549872591561e-12\\
3.06399999999995	-1.38222294248124e-12\\
3.066	-1.35932186207036e-12\\
3.06600000000003	-1.35932186207004e-12\\
3.06999999999997	-1.31465946503438e-12\\
3.07399999999992	-1.2714929126966e-12\\
3.07399999999996	-1.2714929126962e-12\\
3.07399999999999	-1.27149291269579e-12\\
3.07999999999997	-1.20940031238581e-12\\
3.07999999999999	-1.20940031238553e-12\\
3.08599999999997	-1.15030241338282e-12\\
3.08599999999999	-1.15030241338254e-12\\
3.09199999999997	-1.09408856533709e-12\\
3.09199999999999	-1.09408856533683e-12\\
3.09799999999997	-1.04065829763865e-12\\
3.09999999999997	-1.02343391807908e-12\\
3.1	-1.02343391807884e-12\\
3.10299999999997	-9.98123091956453e-13\\
3.10299999999999	-9.98123091956216e-13\\
3.10599999999996	-9.73423351779104e-13\\
3.106	-9.73423351778743e-13\\
3.10600000000003	-9.73423351778511e-13\\
3.10899999999999	-9.49333621515988e-13\\
3.11199999999996	-9.25853363615271e-13\\
3.11199999999999	-9.25853363614986e-13\\
3.11200000000003	-9.25853363614704e-13\\
3.11499999999997	-9.0296186462205e-13\\
3.115	-9.02961864621836e-13\\
3.11799999999995	-8.80638931411589e-13\\
3.11999999999997	-8.66063100972871e-13\\
3.12	-8.66063100972666e-13\\
3.12299999999995	-8.44644256216736e-13\\
3.12599999999989	-8.23742531969026e-13\\
3.12599999999995	-8.23742531968644e-13\\
3.126	-8.23742531968259e-13\\
3.127	-8.16888953834061e-13\\
3.12700000000003	-8.16888953833867e-13\\
3.12800000000003	-8.10094004491197e-13\\
3.12900000000003	-8.03357017849089e-13\\
3.13100000000003	-7.90054297103768e-13\\
3.13199999999997	-7.83487259171601e-13\\
3.13199999999999	-7.83487259171415e-13\\
3.13599999999999	-7.57766301793502e-13\\
3.13799999999997	-7.45225334866115e-13\\
3.13799999999999	-7.45225334865939e-13\\
3.13999999999997	-7.32890792582541e-13\\
3.14	-7.32890792582367e-13\\
3.14199999999998	-7.20757839203102e-13\\
3.14399999999996	-7.08821717955511e-13\\
3.14599999999997	-6.97077749237384e-13\\
3.146	-6.97077749237219e-13\\
3.14999999999996	-6.74174297319376e-13\\
3.15	-6.74174297319146e-13\\
3.15399999999996	-6.52037933782806e-13\\
3.15599999999997	-6.41246515336628e-13\\
3.156	-6.41246515336476e-13\\
3.15999999999996	-6.20196049604061e-13\\
3.16	-6.20196049603852e-13\\
3.16099999999997	-6.15041328744142e-13\\
3.16099999999999	-6.15041328743996e-13\\
3.16199999999996	-6.0992874940831e-13\\
3.16299999999992	-6.04857810491435e-13\\
3.16499999999985	-5.94838869927586e-13\\
3.16599999999997	-5.89889886301117e-13\\
3.166	-5.89889886300977e-13\\
3.16999999999986	-5.70508239582549e-13\\
3.17199999999997	-5.61062701401517e-13\\
3.172	-5.61062701401383e-13\\
3.17599999999986	-5.42643678376649e-13\\
3.17999999999972	-5.24830088889422e-13\\
3.17999999999997	-5.24830088888312e-13\\
3.18	-5.24830088888187e-13\\
3.18499999999997	-5.03372017259894e-13\\
3.185	-5.03372017259775e-13\\
3.18599999999997	-4.99184026163413e-13\\
3.186	-4.99184026163295e-13\\
3.18699999999997	-4.95030790713011e-13\\
3.18799999999994	-4.90913083959207e-13\\
3.18999999999989	-4.82782645671317e-13\\
3.18999999999994	-4.827826456711e-13\\
3.18999999999999	-4.82782645670883e-13\\
3.19399999999988	-4.66930584486508e-13\\
3.19599999999999	-4.59202746748628e-13\\
3.19600000000002	-4.59202746748519e-13\\
3.198	-4.51602980852058e-13\\
3.19800000000003	-4.51602980851951e-13\\
3.19999999999998	-4.44128307318988e-13\\
3.20000000000001	-4.44128307318882e-13\\
3.20199999999996	-4.36775795673916e-13\\
3.20399999999991	-4.29542563355874e-13\\
3.20599999999998	-4.22425774547926e-13\\
3.20600000000001	-4.22425774547826e-13\\
3.20999999999991	-4.08546392358818e-13\\
3.21399999999982	-3.9513186206234e-13\\
3.21899999999997	-3.78985412170138e-13\\
3.21899999999999	-3.78985412170048e-13\\
3.21999999999997	-3.75835832180895e-13\\
3.22	-3.75835832180806e-13\\
3.22099999999998	-3.72712096044634e-13\\
3.22199999999996	-3.69613897458912e-13\\
3.22399999999992	-3.63492900789264e-13\\
3.22599999999997	-3.57470442411411e-13\\
3.226	-3.57470442411326e-13\\
3.22999999999992	-3.45725257362968e-13\\
3.23099999999999	-3.42851131374777e-13\\
3.23100000000002	-3.42851131374696e-13\\
3.23499999999994	-3.31594857033304e-13\\
3.23699999999999	-3.26107004384073e-13\\
3.23700000000002	-3.26107004383996e-13\\
3.23999999999997	-3.18044516951811e-13\\
3.24	-3.18044516951736e-13\\
3.24299999999996	-3.10178870483153e-13\\
3.24599999999991	-3.02503126282153e-13\\
3.24599999999997	-3.02503126281985e-13\\
3.246	-3.02503126281913e-13\\
3.24799999999997	-2.97490974870232e-13\\
3.24799999999999	-2.97490974870162e-13\\
3.24999999999996	-2.92563968256655e-13\\
3.25199999999992	-2.87720174706215e-13\\
3.25399999999997	-2.82957695190835e-13\\
3.25399999999999	-2.82957695190767e-13\\
3.25499999999998	-2.8060636349376e-13\\
3.255	-2.80606363493693e-13\\
3.25599999999998	-2.78274662591899e-13\\
3.25699999999997	-2.75962363945579e-13\\
3.25899999999993	-2.71395068781511e-13\\
3.25999999999997	-2.69139624612817e-13\\
3.26	-2.69139624612753e-13\\
3.26399999999993	-2.60300733749432e-13\\
3.26599999999997	-2.55987993880336e-13\\
3.266	-2.55987993880275e-13\\
3.267	-2.53858161279608e-13\\
3.26700000000003	-2.53858161279548e-13\\
3.26800000000003	-2.5174654826002e-13\\
3.26900000000003	-2.49652947853616e-13\\
3.27100000000003	-2.4551896584704e-13\\
3.27500000000003	-2.37458240452527e-13\\
3.27699999999997	-2.33528336607141e-13\\
3.27699999999999	-2.33528336607086e-13\\
3.27999999999997	-2.27754712143853e-13\\
3.28	-2.27754712143798e-13\\
3.28299999999998	-2.2212204770616e-13\\
3.28599999999996	-2.16625374089402e-13\\
3.286	-2.16625374089331e-13\\
3.28899999999999	-2.1126445184219e-13\\
3.28900000000002	-2.11264451842139e-13\\
3.29	-2.09507848239217e-13\\
3.29000000000003	-2.09507848239167e-13\\
3.29100000000001	-2.07766141690374e-13\\
3.29199999999999	-2.06039161488317e-13\\
3.29399999999996	-2.02628704503076e-13\\
3.296	-1.99275140200586e-13\\
3.29600000000003	-1.99275140200539e-13\\
3.29999999999996	-1.92733452403307e-13\\
3.3	-1.92733452403236e-13\\
3.30000000000003	-1.9273345240319e-13\\
3.30399999999996	-1.86403838375003e-13\\
3.30599999999997	-1.83315444232568e-13\\
3.30599999999999	-1.83315444232524e-13\\
3.30999999999992	-1.77292362223209e-13\\
3.31199999999999	-1.74357046567384e-13\\
3.31200000000002	-1.74357046567343e-13\\
3.31599999999995	-1.68633111564474e-13\\
3.31999999999988	-1.63097322388287e-13\\
3.32	-1.63097322388114e-13\\
3.32000000000003	-1.63097322388076e-13\\
3.32499999999998	-1.56428966153141e-13\\
3.325	-1.56428966153104e-13\\
3.326	-1.55127497083498e-13\\
3.32600000000003	-1.55127497083461e-13\\
3.32700000000003	-1.53836828752955e-13\\
3.32800000000003	-1.52557201396182e-13\\
3.33000000000002	-1.50030569006517e-13\\
3.332	-1.47546609338581e-13\\
3.33200000000003	-1.47546609338546e-13\\
3.33499999999999	-1.43898555416479e-13\\
3.33500000000002	-1.43898555416445e-13\\
3.33799999999998	-1.40341109693515e-13\\
3.33999999999997	-1.38018264161172e-13\\
3.34	-1.38018264161139e-13\\
3.34299999999997	-1.34604896525945e-13\\
3.34599999999993	-1.3127393867434e-13\\
3.34599999999997	-1.31273938674299e-13\\
3.346	-1.31273938674259e-13\\
3.34699999999999	-1.30181733091829e-13\\
3.34700000000002	-1.30181733091799e-13\\
3.34800000000001	-1.29098870759209e-13\\
3.349	-1.28025245526321e-13\\
3.35099999999999	-1.25905286348911e-13\\
3.35499999999995	-1.21771642623318e-13\\
3.35999999999997	-1.16795548571875e-13\\
3.36	-1.16795548571848e-13\\
3.36399999999997	-1.1295983263934e-13\\
3.36399999999999	-1.12959832639314e-13\\
3.36599999999997	-1.11088280592854e-13\\
3.366	-1.11088280592828e-13\\
3.36799999999998	-1.09247667326159e-13\\
3.36999999999996	-1.07438321763781e-13\\
3.37199999999997	-1.0565953454481e-13\\
3.372	-1.05659534544785e-13\\
3.37599999999996	-1.02190857365482e-13\\
3.37799999999997	-1.00499607496602e-13\\
3.378	-1.00499607496579e-13\\
3.37999999999997	-9.88361956447507e-14\\
3.38	-9.88361956447273e-14\\
3.38199999999997	-9.71999696716479e-14\\
3.38399999999995	-9.55902880884394e-14\\
3.38599999999997	-9.40065198132321e-14\\
3.386	-9.40065198132098e-14\\
3.38999999999995	-9.09178058185998e-14\\
3.39299999999997	-8.86694168291371e-14\\
3.39299999999999	-8.8669416829116e-14\\
3.39499999999998	-8.72018349647445e-14\\
3.395	-8.72018349647238e-14\\
3.39699999999999	-8.57586557228615e-14\\
3.39899999999997	-8.43393132993619e-14\\
3.399	-8.43393132993419e-14\\
3.39999999999997	-8.36384066406688e-14\\
3.4	-8.3638406640649e-14\\
3.40099999999998	-8.29432512367137e-14\\
3.40199999999996	-8.22537789524355e-14\\
3.40399999999991	-8.08916139914134e-14\\
3.40599999999997	-7.95513777155253e-14\\
3.406	-7.95513777155064e-14\\
3.40999999999991	-7.69376073763618e-14\\
3.41099999999997	-7.62980001472681e-14\\
3.411	-7.629800014725e-14\\
3.41499999999991	-7.3793031862531e-14\\
3.41899999999982	-7.13706728335635e-14\\
3.41999999999997	-7.07775427744152e-14\\
3.42	-7.07775427743984e-14\\
3.42199999999997	-6.96058256731416e-14\\
3.42199999999999	-6.96058256731251e-14\\
3.42399999999996	-6.84531173353533e-14\\
3.42599999999992	-6.73189657675871e-14\\
3.426	-6.73189657675401e-14\\
3.42600000000003	-6.73189657675241e-14\\
3.42999999999996	-6.51071082089807e-14\\
3.43	-6.51071082089541e-14\\
3.432	-6.40291716784043e-14\\
3.43200000000003	-6.40291716783891e-14\\
3.43400000000003	-6.29693307567402e-14\\
3.43600000000003	-6.1927169929153e-14\\
3.438	-6.09022806124953e-14\\
3.43800000000003	-6.09022806124808e-14\\
3.43999999999997	-5.98942610004869e-14\\
3.44	-5.98942610004726e-14\\
3.44199999999995	-5.89027159004588e-14\\
3.44399999999989	-5.7927256573239e-14\\
3.44599999999997	-5.69675005861849e-14\\
3.446	-5.69675005861714e-14\\
3.44999999999989	-5.50957547025057e-14\\
3.45099999999996	-5.46377258617942e-14\\
3.45099999999999	-5.46377258617812e-14\\
3.45499999999988	-5.28438941673753e-14\\
3.45699999999996	-5.19693344571355e-14\\
3.45699999999999	-5.19693344571232e-14\\
3.45999999999997	-5.0684473669263e-14\\
3.46	-5.0684473669251e-14\\
3.46299999999998	-4.9430982012412e-14\\
3.46499999999998	-4.86122010339132e-14\\
3.465	-4.86122010339017e-14\\
3.46599999999997	-4.82077537155713e-14\\
3.466	-4.82077537155599e-14\\
3.46699999999997	-4.78066628550027e-14\\
3.46799999999994	-4.74090031111321e-14\\
3.46999999999988	-4.6623821406753e-14\\
3.47199999999997	-4.58519007688688e-14\\
3.472	-4.58519007688579e-14\\
3.47599999999988	-4.43466372539689e-14\\
3.47999999999976	-4.289085177134e-14\\
3.47999999999996	-4.28908517712666e-14\\
3.47999999999999	-4.28908517712564e-14\\
3.48599999999996	-4.07949706807939e-14\\
3.48599999999999	-4.07949706807842e-14\\
3.49199999999996	-3.88013712151697e-14\\
3.49199999999999	-3.88013712151603e-14\\
3.49799999999996	-3.69064902015406e-14\\
3.5	-3.62956351327353e-14\\
3.50000000000003	-3.62956351327267e-14\\
3.506	-3.45220323361206e-14\\
3.50600000000003	-3.45220323361124e-14\\
3.50899999999999	-3.3667700600039e-14\\
3.50900000000002	-3.3667700600031e-14\\
3.51199999999998	-3.28349835459348e-14\\
3.51200000000003	-3.2834983545922e-14\\
3.51499999999999	-3.20231465793283e-14\\
3.51799999999995	-3.12314735372487e-14\\
3.51799999999999	-3.12314735372389e-14\\
3.51800000000003	-3.12314735372292e-14\\
3.51999999999997	-3.07145480897088e-14\\
3.52	-3.07145480897015e-14\\
3.52199999999995	-3.02060709927274e-14\\
3.52399999999989	-2.97058428873405e-14\\
3.52599999999997	-2.92136676571637e-14\\
3.526	-2.92136676571568e-14\\
3.52999999999989	-2.82538122635131e-14\\
3.53399999999978	-2.73261045973481e-14\\
3.53499999999998	-2.70990291726791e-14\\
3.535	-2.70990291726727e-14\\
3.53799999999997	-2.64290896315196e-14\\
3.53799999999999	-2.64290896315134e-14\\
3.53999999999997	-2.59916504973033e-14\\
3.54	-2.59916504972971e-14\\
3.54199999999998	-2.55613606242436e-14\\
3.54399999999996	-2.51380513164979e-14\\
3.54599999999997	-2.47215566140195e-14\\
3.546	-2.47215566140137e-14\\
3.54999999999996	-2.39092957322407e-14\\
3.54999999999999	-2.39092957322349e-14\\
3.553	-2.33180210381976e-14\\
3.55300000000003	-2.33180210381921e-14\\
3.55600000000004	-2.2741526381495e-14\\
3.55900000000005	-2.2179303206204e-14\\
3.55999999999997	-2.19949808416657e-14\\
3.56	-2.19949808416605e-14\\
3.56599999999999	-2.09201860526347e-14\\
3.56600000000002	-2.09201860526297e-14\\
3.56699999999996	-2.07461290814083e-14\\
3.56699999999999	-2.07461290814034e-14\\
3.56799999999996	-2.05735610769371e-14\\
3.56899999999992	-2.04024651228437e-14\\
3.56999999999998	-2.02328244498974e-14\\
3.57	-2.02328244498926e-14\\
3.57199999999993	-1.9897842582458e-14\\
3.57299999999996	-1.97324685561817e-14\\
3.57299999999999	-1.9732468556177e-14\\
3.57499999999992	-1.9405873275197e-14\\
3.57699999999985	-1.90847085465714e-14\\
3.57899999999996	-1.87688484564101e-14\\
3.57899999999999	-1.87688484564056e-14\\
3.57999999999997	-1.86128688731163e-14\\
3.58	-1.86128688731119e-14\\
3.58099999999998	-1.84581691725508e-14\\
3.58199999999997	-1.83047341919548e-14\\
3.58399999999993	-1.80015983602447e-14\\
3.58599999999997	-1.77033425324064e-14\\
3.586	-1.77033425324022e-14\\
3.58999999999993	-1.71216747696748e-14\\
3.59399999999986	-1.65594883754431e-14\\
3.59599999999996	-1.62854239850297e-14\\
3.59599999999999	-1.62854239850258e-14\\
3.59999999999997	-1.57508155831232e-14\\
3.6	-1.57508155831194e-14\\
3.60399999999998	-1.52335385755425e-14\\
3.60499999999998	-1.51068316994099e-14\\
3.605	-1.51068316994063e-14\\
3.60599999999997	-1.49811447840816e-14\\
3.606	-1.4981144784078e-14\\
3.60699999999997	-1.48565009291083e-14\\
3.60799999999994	-1.47329233353427e-14\\
3.60799999999999	-1.47329233353364e-14\\
3.60999999999993	-1.44889185871781e-14\\
3.61199999999987	-1.42490348772403e-14\\
3.61399999999996	-1.40131781581774e-14\\
3.61399999999999	-1.40131781581741e-14\\
3.61799999999987	-1.35531773668312e-14\\
3.61999999999997	-1.33288529493994e-14\\
3.62	-1.33288529493962e-14\\
3.62399999999988	-1.28911163057463e-14\\
3.62499999999996	-1.2783892822292e-14\\
3.62499999999999	-1.27838928222889e-14\\
3.62599999999997	-1.2677532462906e-14\\
3.626	-1.2677532462903e-14\\
3.62699999999998	-1.25720547751537e-14\\
3.62799999999997	-1.24674793923341e-14\\
3.62999999999993	-1.22609946310141e-14\\
3.63199999999997	-1.20579972257929e-14\\
3.632	-1.20579972257901e-14\\
3.63599999999993	-1.16621474809316e-14\\
3.63999999999985	-1.12793093228497e-14\\
3.63999999999997	-1.12793093228382e-14\\
3.64	-1.12793093228355e-14\\
3.64599999999997	-1.07281407165758e-14\\
3.646	-1.07281407165732e-14\\
3.65199999999997	-1.02038698296567e-14\\
3.652	-1.02038698296543e-14\\
3.65399999999996	-1.00349705806596e-14\\
3.65399999999999	-1.00349705806573e-14\\
3.65599999999995	-9.86888888468687e-15\\
3.65799999999992	-9.7055596259728e-15\\
3.65999999999996	-9.54491877064701e-15\\
3.65999999999999	-9.54491877064474e-15\\
3.66399999999992	-9.23145138388975e-15\\
3.66599999999996	-9.07850195632958e-15\\
3.66599999999999	-9.07850195632742e-15\\
3.66999999999992	-8.78021524041745e-15\\
3.67399999999984	-8.49191882090594e-15\\
3.67499999999998	-8.42135237599986e-15\\
3.67500000000001	-8.42135237599786e-15\\
3.67999999999997	-8.0772210113429e-15\\
3.68	-8.07722101134098e-15\\
3.68299999999996	-7.8774610363059e-15\\
3.68299999999999	-7.87746103630403e-15\\
3.68599999999995	-7.68252391691928e-15\\
3.686	-7.68252391691614e-15\\
3.68699999999998	-7.61860494339898e-15\\
3.68700000000001	-7.61860494339717e-15\\
3.68799999999998	-7.55523276241712e-15\\
3.68899999999995	-7.49240116276722e-15\\
3.6909999999999	-7.36833512723599e-15\\
3.69299999999998	-7.24635819623115e-15\\
3.693	-7.24635819622943e-15\\
3.6969999999999	-7.00848116495709e-15\\
3.69999999999997	-6.83520739242763e-15\\
3.7	-6.83520739242601e-15\\
3.7039999999999	-6.61073040574145e-15\\
3.70599999999997	-6.50120186131134e-15\\
3.706	-6.5012018613098e-15\\
3.7099999999999	-6.28759589803682e-15\\
3.70999999999998	-6.28759589803265e-15\\
3.71000000000001	-6.28759589803115e-15\\
3.71199999999996	-6.18349621494276e-15\\
3.71199999999999	-6.18349621494129e-15\\
3.71399999999995	-6.08114408110302e-15\\
3.71599999999991	-5.98049936895461e-15\\
3.71799999999996	-5.88152262034658e-15\\
3.71799999999999	-5.88152262034519e-15\\
3.71999999999997	-5.7841750316048e-15\\
3.72	-5.78417503160343e-15\\
3.72199999999998	-5.6884184377422e-15\\
3.72399999999997	-5.59421529700798e-15\\
3.72599999999997	-5.50152867669773e-15\\
3.726	-5.50152867669643e-15\\
3.728	-5.41037426623489e-15\\
3.72800000000003	-5.41037426623361e-15\\
3.73000000000004	-5.32076835573408e-15\\
3.73200000000004	-5.23267581484353e-15\\
3.73600000000004	-5.06089327452333e-15\\
3.74	-4.89475722128985e-15\\
3.74000000000003	-4.89475722128869e-15\\
3.74099999999996	-4.8540747496314e-15\\
3.74099999999999	-4.85407474963024e-15\\
3.74199999999996	-4.8137248721349e-15\\
3.74299999999992	-4.77370363223674e-15\\
3.74499999999985	-4.69463140706647e-15\\
3.745	-4.69463140706046e-15\\
3.74500000000003	-4.69463140705934e-15\\
3.746	-4.65557267174186e-15\\
3.74600000000003	-4.65557267174076e-15\\
3.747	-4.61683808009048e-15\\
3.74799999999997	-4.57843484205317e-15\\
3.74999999999991	-4.50260740325773e-15\\
3.752	-4.42806062691068e-15\\
3.75200000000003	-4.42806062690963e-15\\
3.75599999999991	-4.28269264860725e-15\\
3.758	-4.21181445455413e-15\\
3.75800000000003	-4.21181445455313e-15\\
3.75999999999997	-4.14210291743815e-15\\
3.76	-4.14210291743717e-15\\
3.76199999999995	-4.07353070694509e-15\\
3.76399999999989	-4.006070939069e-15\\
3.76599999999997	-3.93969716594415e-15\\
3.766	-3.93969716594322e-15\\
3.76999999999989	-3.81025297569902e-15\\
3.76999999999996	-3.8102529756967e-15\\
3.76999999999999	-3.8102529756958e-15\\
3.77399999999988	-3.68514416406106e-15\\
3.77599999999999	-3.62415395060065e-15\\
3.77600000000002	-3.6241539505998e-15\\
3.77999999999991	-3.50518234326653e-15\\
3.77999999999996	-3.50518234326517e-15\\
3.78	-3.50518234326379e-15\\
3.78399999999989	-3.3900676544095e-15\\
3.78599999999997	-3.33390000667705e-15\\
3.786	-3.33390000667626e-15\\
3.78999999999989	-3.22436011859564e-15\\
3.79199999999997	-3.17097646085078e-15\\
3.792	-3.17097646085003e-15\\
3.79599999999989	-3.0668770647031e-15\\
3.79899999999996	-2.99105676449011e-15\\
3.79899999999999	-2.9910567644894e-15\\
3.79999999999997	-2.96619941700361e-15\\
3.8	-2.96619941700291e-15\\
3.80099999999999	-2.94154603524286e-15\\
3.80199999999997	-2.91709420282777e-15\\
3.80399999999993	-2.86878561941597e-15\\
3.80599999999997	-2.8212547272006e-15\\
3.806	-2.82125472719993e-15\\
3.80999999999993	-2.7285585074914e-15\\
3.81099999999996	-2.70587511739019e-15\\
3.81099999999999	-2.70587511738955e-15\\
3.81499999999992	-2.61703751485092e-15\\
3.81499999999998	-2.61703751484968e-15\\
3.81500000000001	-2.61703751484906e-15\\
3.81899999999993	-2.53112961335576e-15\\
3.81999999999997	-2.51009451659137e-15\\
3.82	-2.51009451659077e-15\\
3.82399999999993	-2.42765978940598e-15\\
3.826	-2.38743760086111e-15\\
3.82600000000003	-2.38743760086055e-15\\
3.82700000000001	-2.36757400345833e-15\\
3.82700000000003	-2.36757400345776e-15\\
3.82799999999998	-2.34788032852752e-15\\
3.82800000000001	-2.34788032852696e-15\\
3.82899999999997	-2.32835464580846e-15\\
3.82999999999993	-2.30899504158203e-15\\
3.83199999999986	-2.27076649492656e-15\\
3.83399999999998	-2.23317970090865e-15\\
3.83400000000001	-2.23317970090812e-15\\
3.83799999999985	-2.15987267334099e-15\\
3.84	-2.12412369947627e-15\\
3.84000000000003	-2.12412369947576e-15\\
3.84399999999988	-2.0543647493511e-15\\
3.846	-2.02032742381392e-15\\
3.84600000000003	-2.02032742381344e-15\\
3.84999999999988	-1.95394677621798e-15\\
3.84999999999998	-1.95394677621635e-15\\
3.85000000000001	-1.95394677621589e-15\\
3.85399999999985	-1.88978936657767e-15\\
3.85599999999998	-1.85851279973314e-15\\
3.85600000000001	-1.8585127997327e-15\\
3.85699999999996	-1.84306965218393e-15\\
3.85699999999999	-1.8430696521835e-15\\
3.85799999999995	-1.82775457293141e-15\\
3.85899999999992	-1.81256606088122e-15\\
3.85999999999997	-1.79750262739686e-15\\
3.86	-1.79750262739644e-15\\
3.86199999999993	-1.76774510276659e-15\\
3.86399999999985	-1.73847033256396e-15\\
3.86599999999997	-1.70966683949997e-15\\
3.866	-1.70966683949956e-15\\
3.86899999999996	-1.66735697157706e-15\\
3.86899999999999	-1.66735697157666e-15\\
3.87199999999995	-1.62611754717516e-15\\
3.87499999999991	-1.58591218685251e-15\\
3.87999999999997	-1.52110524205442e-15\\
3.88	-1.52110524205405e-15\\
3.88499999999998	-1.45891371496958e-15\\
3.88500000000001	-1.45891371496924e-15\\
3.88599999999999	-1.44677573849589e-15\\
3.88600000000002	-1.44677573849555e-15\\
3.88700000000001	-1.4347384936649e-15\\
3.88799999999999	-1.42280422099449e-15\\
3.88999999999996	-1.39923992338075e-15\\
3.89199999999999	-1.37607360718049e-15\\
3.89200000000002	-1.37607360718016e-15\\
3.89599999999996	-1.33089874227508e-15\\
3.89799999999999	-1.30887248258588e-15\\
3.89800000000002	-1.30887248258557e-15\\
3.89999999999997	-1.28720877571083e-15\\
3.9	-1.28720877571052e-15\\
3.90199999999996	-1.26589912842366e-15\\
3.90399999999991	-1.24493518619302e-15\\
3.90599999999997	-1.22430873002368e-15\\
3.906	-1.22430873002339e-15\\
3.90999999999991	-1.18408237569279e-15\\
3.91399999999982	-1.14520329293432e-15\\
3.91499999999996	-1.13568684228049e-15\\
3.91499999999999	-1.13568684228022e-15\\
3.91999999999997	-1.08927796973181e-15\\
3.92	-1.08927796973155e-15\\
3.92499999999999	-1.04474202410593e-15\\
3.92599999999997	-1.03604990338948e-15\\
3.926	-1.03604990338924e-15\\
3.92699999999996	-1.02742991749022e-15\\
3.92699999999999	-1.02742991748998e-15\\
3.92799999999995	-1.01888367101379e-15\\
3.92899999999992	-1.01041032626442e-15\\
3.93099999999984	-9.93679027106171e-16\\
3.93299999999996	-9.77229460443666e-16\\
3.93299999999999	-9.77229460443434e-16\\
3.93699999999984	-9.45149836470864e-16\\
3.93999999999997	-9.21782479682593e-16\\
3.94	-9.21782479682374e-16\\
3.94399999999985	-8.91509959172686e-16\\
3.94399999999996	-8.9150995917184e-16\\
3.94399999999999	-8.91509959171628e-16\\
3.94599999999997	-8.76739157259053e-16\\
3.946	-8.76739157258845e-16\\
3.94799999999999	-8.62212532689804e-16\\
3.94999999999997	-8.47932681423118e-16\\
3.95199999999997	-8.3389400498823e-16\\
3.952	-8.33894004988032e-16\\
3.95499999999998	-8.13276178842771e-16\\
3.95500000000001	-8.13276178842578e-16\\
3.95799999999998	-7.93170446052971e-16\\
3.95999999999998	-7.80042343706762e-16\\
3.96	-7.80042343706577e-16\\
3.96299999999998	-7.60750901812056e-16\\
3.96599999999995	-7.41925218018914e-16\\
3.966	-7.41925218018606e-16\\
3.96600000000003	-7.4192521801843e-16\\
3.96700000000001	-7.35752363892729e-16\\
3.96700000000003	-7.35752363892554e-16\\
3.96800000000001	-7.29632315319631e-16\\
3.96899999999998	-7.23564472367265e-16\\
3.97099999999993	-7.11583029535743e-16\\
3.97299999999996	-6.99803338026933e-16\\
3.97299999999999	-6.99803338026767e-16\\
3.97699999999989	-6.7683081345564e-16\\
3.97899999999996	-6.65628973923737e-16\\
3.97899999999999	-6.65628973923579e-16\\
3.97999999999997	-6.6009722642908e-16\\
3.98	-6.60097226428923e-16\\
3.98099999999999	-6.5461086941376e-16\\
3.98199999999997	-6.49169365137196e-16\\
3.98399999999994	-6.3841878584758e-16\\
3.98599999999997	-6.27841273753923e-16\\
3.986	-6.27841273753774e-16\\
3.98999999999994	-6.07212681718417e-16\\
3.98999999999997	-6.07212681718237e-16\\
3.99000000000001	-6.07212681718057e-16\\
3.99399999999994	-5.87274987872209e-16\\
3.99599999999998	-5.77555415065065e-16\\
3.99600000000001	-5.77555415064928e-16\\
3.99999999999994	-5.58595763443891e-16\\
3.99999999999997	-5.5859576344374e-16\\
4	-5.58595763443591e-16\\
4.00199999999993	-5.49348251423728e-16\\
4.00199999999999	-5.49348251423468e-16\\
4.00399999999992	-5.40250761228686e-16\\
4.00599999999986	-5.3129972615191e-16\\
4.00599999999993	-5.31299726151587e-16\\
4.006	-5.3129972615126e-16\\
4.00999999999987	-5.13843139970736e-16\\
4.01199999999995	-5.05335769383313e-16\\
4.012	-5.05335769383073e-16\\
4.01599999999987	-4.88746195376633e-16\\
4.01999999999973	-4.72701920900456e-16\\
4.01999999999995	-4.72701920899619e-16\\
4.02	-4.72701920899395e-16\\
4.02500000000001	-4.53375148437054e-16\\
4.02500000000006	-4.53375148436839e-16\\
4.02599999999995	-4.49603124913084e-16\\
4.026	-4.49603124912871e-16\\
4.02699999999996	-4.45862404969258e-16\\
4.02799999999992	-4.42153684874015e-16\\
4.02999999999985	-4.34830793356763e-16\\
4.03099999999993	-4.31215904201346e-16\\
4.03099999999999	-4.31215904201141e-16\\
4.03499999999984	-4.17058492786135e-16\\
4.03699999999994	-4.10156227840072e-16\\
4.03699999999999	-4.10156227839877e-16\\
4.03799999999995	-4.06748014188217e-16\\
4.038	-4.06748014188024e-16\\
4.03899999999996	-4.03367966780376e-16\\
4.03999999999991	-4.00015754324191e-16\\
4.04	-4.00015754323889e-16\\
4.04199999999991	-3.93393522755086e-16\\
4.04399999999982	-3.86878723227418e-16\\
4.046	-3.80468801587922e-16\\
4.04600000000006	-3.80468801587741e-16\\
4.04999999999988	-3.6796797367067e-16\\
4.05399999999969	-3.55885826749259e-16\\
4.05999999999994	-3.38506354211919e-16\\
4.06	-3.38506354211758e-16\\
};
\addplot [color=mycolor2,solid,forget plot]
  table[row sep=crcr]{%
0	0.15314\\
3.15544362088405e-30	0.15314\\
0.000656101980281985	0.153143230512962\\
0.00393661188169191	0.153256312778436\\
0.00999999999999994	0.153891071773171\\
0.01	0.153891071773171\\
0.0199999999999999	0.150048203824684\\
0.02	0.150048203824684\\
0.0289999999999998	0.137414337712804\\
0.029	0.137414337712803\\
0.03	0.135470213386942\\
0.0300000000000002	0.135470213386942\\
0.0349999999999996	0.124115011067004\\
0.035	0.124115011067003\\
0.0399999999999993	0.110014119663841\\
0.04	0.110014119663839\\
0.0449999999999993	0.0939630779639858\\
0.0499999999999987	0.0767526455719492\\
0.05	0.0767526455719445\\
0.0500000000000004	0.0767526455719429\\
0.0579999999999996	0.0466980443355424\\
0.058	0.0466980443355407\\
0.0599999999999996	0.0386819498575326\\
0.06	0.0386819498575308\\
0.0619999999999995	0.0306155753047838\\
0.0639999999999991	0.0226526210813104\\
0.0679999999999982	0.00702452540129683\\
0.0699999999999991	-0.000646743531092999\\
0.07	-0.000646743531096385\\
0.0779999999999982	-0.0304497245417899\\
0.0799999999999991	-0.0376945518218964\\
0.08	-0.0376945518218996\\
0.087	-0.0613629561565828\\
0.0870000000000009	-0.0613629561565856\\
0.09	-0.0705722498255595\\
0.0900000000000009	-0.0705722498255621\\
0.0929999999999999	-0.0792329294508874\\
0.095999999999999	-0.0873526353855747\\
0.0999999999999991	-0.0973497076333374\\
0.1	-0.0973497076333395\\
0.104999999999999	-0.108387871323025\\
0.105	-0.108387871323027\\
0.109999999999999	-0.117699946526397\\
0.11	-0.117699946526398\\
0.114999999999999	-0.125308754745586\\
0.115999999999999	-0.126627838237935\\
0.116	-0.126627838237936\\
0.119999999999999	-0.131232943213937\\
0.12	-0.131232943213938\\
0.123999999999999	-0.13476926202664\\
0.127999999999998	-0.137242340899987\\
0.129999999999998	-0.138081442501912\\
0.13	-0.138081442501912\\
0.137999999999998	-0.138771927810782\\
0.139999999999998	-0.138276983003393\\
0.14	-0.138276983003392\\
0.144999999999998	-0.135869686909084\\
0.145	-0.135869686909083\\
0.149999999999998	-0.131785845077204\\
0.15	-0.131785845077202\\
0.154999999999998	-0.126461814037968\\
0.159999999999996	-0.120330910865585\\
0.16	-0.12033091086558\\
0.169999999999996	-0.105586372774747\\
0.17	-0.105586372774741\\
0.173999999999998	-0.0990022340863097\\
0.174	-0.0990022340863068\\
0.174999999999998	-0.0973523335757812\\
0.175	-0.0973523335757782\\
0.176	-0.0957005634817941\\
0.177	-0.0940467619077872\\
0.179000000000001	-0.0907324157456936\\
0.179999999999998	-0.0890715463112649\\
0.18	-0.0890715463112619\\
0.184000000000001	-0.0823996248083219\\
0.188000000000002	-0.0756743350020158\\
0.189999999999998	-0.0722883843683618\\
0.19	-0.0722883843683588\\
0.198000000000002	-0.058558154107107\\
0.199999999999998	-0.0550722607951608\\
0.2	-0.0550722607951577\\
0.202999999999998	-0.0499542634358041\\
0.203	-0.0499542634358012\\
0.205999999999998	-0.045090375635058\\
0.208999999999996	-0.0404763063780602\\
0.209999999999998	-0.0389930887103154\\
0.21	-0.0389930887103128\\
0.215999999999996	-0.0305025648988234\\
0.219999999999998	-0.0251966015169791\\
0.22	-0.0251966015169768\\
0.225999999999996	-0.0177484283157274\\
0.229999999999998	-0.0131121370239185\\
0.23	-0.0131121370239165\\
0.231999999999998	-0.0108899737028716\\
0.232	-0.0108899737028697\\
0.233999999999998	-0.00873062873450069\\
0.235999999999997	-0.00663325550056961\\
0.239999999999993	-0.00262115909906458\\
0.239999999999996	-0.0026211590990612\\
0.24	-0.00262115909905783\\
0.244999999999998	0.00188687770292411\\
0.245	0.00188687770292559\\
0.249999999999998	0.00568827569695383\\
0.25	0.00568827569695508\\
0.254999999999999	0.00879235129348301\\
0.259999999999997	0.0112067117758103\\
0.26	0.0112067117758117\\
0.260999999999996	0.0116103534305937\\
0.261	0.0116103534305951\\
0.262	0.0119926581405718\\
0.263	0.0123536634279823\\
0.265	0.0130119151800953\\
0.269	0.0140742429758504\\
0.269999999999997	0.0142870265310786\\
0.27	0.0142870265310794\\
0.278	0.015231929494095\\
0.279999999999996	0.0152581815612772\\
0.28	0.0152581815612772\\
0.288	0.0148381646770985\\
0.289999999999996	0.0146213487868583\\
0.29	0.0146213487868579\\
0.298	0.0133042875626112\\
0.299999999999996	0.0128619274084787\\
0.3	0.0128619274084778\\
0.308	0.01063490283314\\
0.309999999999996	0.00996265931864272\\
0.31	0.00996265931864148\\
0.314999999999997	0.00820076427699913\\
0.315	0.00820076427699786\\
0.319	0.00675620801608072\\
0.319000000000004	0.00675620801607942\\
0.319999999999996	0.00638990890522511\\
0.32	0.0063899089052238\\
0.321	0.00602147428209788\\
0.321999999999999	0.00565086803499177\\
0.323999999999998	0.00490299515937\\
0.327999999999996	0.00337957699652842\\
0.329999999999996	0.00260343440859324\\
0.33	0.00260343440859186\\
0.337999999999996	-0.000386934584574294\\
0.339999999999996	-0.00109385045098048\\
0.34	-0.00109385045098172\\
0.347999999999996	-0.00376712537690133\\
0.348	-0.0037671253769025\\
0.349999999999996	-0.00439815708383284\\
0.35	-0.00439815708383395\\
0.351999999999996	-0.00501477739830985\\
0.353999999999993	-0.00561722810735759\\
0.357999999999985	-0.00678056001263201\\
0.359999999999996	-0.00734189732970837\\
0.36	-0.00734189732970936\\
0.367999999999985	-0.00927534065693867\\
0.369999999999996	-0.00967033707065848\\
0.37	-0.00967033707065915\\
0.377	-0.0108624199510937\\
0.377000000000004	-0.0108624199510943\\
0.379999999999997	-0.011295123023204\\
0.38	-0.0112951230232045\\
0.382999999999993	-0.0116814835302473\\
0.384999999999997	-0.0119134813252346\\
0.385	-0.011913481325235\\
0.387999999999993	-0.0122233415981324\\
0.389999999999997	-0.0124046089407074\\
0.39	-0.0124046089407077\\
0.392999999999993	-0.0126387283665326\\
0.395999999999986	-0.0128276905077501\\
0.399999999999997	-0.0130096777594896\\
0.4	-0.0130096777594898\\
0.405999999999986	-0.0131344066698936\\
0.406	-0.0131344066698937\\
0.406000000000004	-0.0131344066698937\\
0.41	-0.0131194192205885\\
0.410000000000004	-0.0131194192205884\\
0.414	-0.013025910252926\\
0.417999999999996	-0.0128537331127516\\
0.419999999999997	-0.0127380644190282\\
0.42	-0.012738064419028\\
0.427999999999993	-0.0121511608627946\\
0.429999999999997	-0.0119777204409489\\
0.43	-0.0119777204409485\\
0.435	-0.0114966423981403\\
0.435000000000004	-0.01149664239814\\
0.439999999999997	-0.0109467768969577\\
0.44	-0.0109467768969573\\
0.444999999999993	-0.010355451915085\\
0.449999999999986	-0.00974989383042669\\
0.449999999999993	-0.00974989383042581\\
0.45	-0.00974989383042494\\
0.454999999999997	-0.00912861851892244\\
0.455	-0.00912861851892199\\
0.459999999999997	-0.00849010351173955\\
0.46	-0.00849010351173909\\
0.463999999999997	-0.00797875498394609\\
0.464	-0.00797875498394564\\
0.467999999999997	-0.00748042273240351\\
0.469999999999997	-0.00723589270700806\\
0.47	-0.00723589270700762\\
0.473999999999997	-0.00675562548054035\\
0.477999999999993	-0.00628645622411489\\
0.479999999999997	-0.00605580272715413\\
0.48	-0.00605580272715372\\
0.487999999999993	-0.00515729754952254\\
0.489999999999997	-0.00493825812254068\\
0.49	-0.0049382581225403\\
0.492999999999997	-0.00462164187662262\\
0.493	-0.00462164187662226\\
0.495999999999997	-0.00432559297658477\\
0.498999999999993	-0.00404985024110665\\
0.499999999999997	-0.00396240592092304\\
0.5	-0.00396240592092273\\
0.505999999999993	-0.00348408566948026\\
0.509999999999993	-0.00320878304145653\\
0.51	-0.00320878304145607\\
0.515999999999993	-0.00285520465084399\\
0.519999999999993	-0.00265634789090958\\
0.52	-0.00265634789090925\\
0.521999999999993	-0.00256785689554054\\
0.522	-0.00256785689554024\\
0.523999999999993	-0.00248661028412352\\
0.524999999999993	-0.00244869355699522\\
0.525	-0.00244869355699495\\
0.526999999999993	-0.00237825467839496\\
0.528999999999986	-0.00231498584786251\\
0.529999999999993	-0.00228603233951768\\
0.53	-0.00228603233951748\\
0.533999999999986	-0.00218245323015712\\
0.537999999999972	-0.00209610281135348\\
0.539999999999993	-0.00205934498713197\\
0.54	-0.00205934498713185\\
0.547999999999972	-0.00195477946619444\\
0.549999999999993	-0.00193920328278263\\
0.55	-0.00193920328278258\\
0.550999999999993	-0.00193299477772194\\
0.551	-0.0019329947777219\\
0.551999999999997	-0.00192783848496902\\
0.552999999999993	-0.00192373389857262\\
0.554999999999986	-0.00191867833871959\\
0.558999999999972	-0.00192117689668293\\
0.559999999999993	-0.00192442875915677\\
0.56	-0.0019244287591568\\
0.567999999999972	-0.00197140268441475\\
0.57	-0.001988401838639\\
0.570000000000007	-0.00198840183863906\\
0.577999999999979	-0.0020592072895557\\
0.579999999999993	-0.00207649525762915\\
0.58	-0.00207649525762921\\
0.587999999999972	-0.00214419694775721\\
0.589999999999993	-0.00216079300364885\\
0.59	-0.00216079300364891\\
0.594999999999993	-0.00220177812842115\\
0.595	-0.00220177812842121\\
0.599999999999993	-0.00224212195069527\\
0.6	-0.00224212195069533\\
0.604999999999993	-0.00227749864227557\\
0.608999999999993	-0.00229909690584826\\
0.609	-0.0022990969058483\\
0.609999999999993	-0.00230357020084521\\
0.61	-0.00230357020084524\\
0.610999999999997	-0.00230767391869955\\
0.611999999999993	-0.00231140846167129\\
0.613999999999986	-0.00231777145091609\\
0.617999999999972	-0.00232608101572473\\
0.619999999999993	-0.00232803084947865\\
0.62	-0.00232803084947866\\
0.627999999999972	-0.00232051978163345\\
0.629999999999993	-0.00231477188658906\\
0.63	-0.00231477188658903\\
0.637999999999972	-0.00227623216538157\\
0.637999999999993	-0.00227623216538144\\
0.638	-0.00227623216538139\\
0.639999999999993	-0.00226269131337631\\
0.64	-0.00226269131337626\\
0.641999999999993	-0.00224783108334466\\
0.643999999999986	-0.00223189833618672\\
0.647999999999971	-0.00219678988174738\\
0.649999999999993	-0.0021776004092714\\
0.65	-0.00217760040927134\\
0.657999999999971	-0.00208976029042206\\
0.659999999999993	-0.00206498665594992\\
0.66	-0.00206498665594983\\
0.664999999999993	-0.00199802433862666\\
0.665	-0.00199802433862656\\
0.666999999999993	-0.00196920055372759\\
0.667	-0.00196920055372749\\
0.668999999999993	-0.00193919501446106\\
0.669999999999993	-0.0019237454289288\\
0.67	-0.00192374542892869\\
0.671999999999993	-0.00189246372396009\\
0.673999999999986	-0.00186100751792911\\
0.677999999999972	-0.00179752219965694\\
0.679999999999993	-0.00176546819611356\\
0.68	-0.00176546819611344\\
0.687999999999972	-0.00163488227406933\\
0.689999999999993	-0.00160157931998955\\
0.69	-0.00160157931998944\\
0.695999999999993	-0.00150276358338633\\
0.696	-0.00150276358338622\\
0.699999999999993	-0.00143846754788163\\
0.7	-0.00143846754788152\\
0.703999999999993	-0.00137531848145053\\
0.707999999999986	-0.00131321734327865\\
0.709999999999993	-0.00128252924894784\\
0.71	-0.00128252924894773\\
0.717999999999986	-0.00116191364707716\\
0.719999999999993	-0.00113223482981509\\
0.72	-0.00113223482981499\\
0.724999999999993	-0.00106143079390376\\
0.725	-0.00106143079390366\\
0.729999999999993	-0.000996873375159433\\
0.730000000000001	-0.000996873375159346\\
0.734999999999994	-0.000936717601452401\\
0.735000000000001	-0.000936717601452317\\
0.739999999999994	-0.000879129290255161\\
0.740000000000001	-0.000879129290255081\\
0.744999999999994	-0.000823967307569958\\
0.749999999999987	-0.000771096465700289\\
0.750000000000001	-0.000771096465700144\\
0.753999999999993	-0.00073036215460301\\
0.754	-0.000730362154602938\\
0.757999999999993	-0.000690947265248271\\
0.759999999999993	-0.000671715205610819\\
0.76	-0.000671715205610751\\
0.763999999999993	-0.000635005393658975\\
0.767999999999986	-0.00060114792352203\\
0.77	-0.000585272049199614\\
0.770000000000007	-0.000585272049199559\\
0.777999999999993	-0.000528645554914799\\
0.779999999999993	-0.000516180430556586\\
0.78	-0.000516180430556543\\
0.782999999999993	-0.000498598482522057\\
0.783	-0.000498598482522017\\
0.785999999999993	-0.000482235829801109\\
0.788999999999986	-0.000467078036854374\\
0.79	-0.000462290844231463\\
0.790000000000007	-0.000462290844231429\\
0.795999999999993	-0.000436322274410771\\
0.8	-0.000421602903029317\\
0.800000000000007	-0.000421602903029293\\
0.804999999999993	-0.000405445605427315\\
0.805	-0.000405445605427293\\
0.809999999999987	-0.00039117882537358\\
0.809999999999997	-0.000391178825373552\\
0.810000000000007	-0.000391178825373524\\
0.811999999999993	-0.00038599347258466\\
0.812	-0.000385993472584642\\
0.813999999999987	-0.00038110292454074\\
0.815999999999973	-0.000376505263846724\\
0.819999999999945	-0.000368181508555571\\
0.819999999999987	-0.00036818150855549\\
0.82	-0.000368181508555463\\
0.827999999999944	-0.00035497694563609\\
0.829999999999993	-0.000352385372545459\\
0.830000000000001	-0.00035238537254545\\
0.837999999999945	-0.000344115707971929\\
0.839999999999993	-0.000342523970614159\\
0.84	-0.000342523970614154\\
0.840999999999993	-0.000341798980015613\\
0.841000000000001	-0.000341798980015608\\
0.841999999999997	-0.000341121145212625\\
0.842999999999993	-0.000340490399724983\\
0.844999999999986	-0.000339369934026441\\
0.848999999999972	-0.000337691662864691\\
0.849999999999993	-0.000337389069357229\\
0.85	-0.000337389069357227\\
0.857999999999972	-0.000335259240190865\\
0.859999999999993	-0.000334758414659414\\
0.86	-0.000334758414659412\\
0.867999999999972	-0.000332875747277424\\
0.869999999999993	-0.000332434316465353\\
0.87	-0.000332434316465352\\
0.874999999999994	-0.000331068242922347\\
0.875000000000001	-0.000331068242922344\\
0.879999999999994	-0.000329146222069652\\
0.880000000000001	-0.000329146222069649\\
0.884999999999994	-0.000326663543542365\\
0.889999999999987	-0.000323614122940048\\
0.890000000000001	-0.000323614122940039\\
0.890000000000008	-0.000323614122940034\\
0.899	-0.000316486733966596\\
0.899000000000008	-0.000316486733966589\\
0.9	-0.000315554720433605\\
0.900000000000007	-0.000315554720433599\\
0.901000000000004	-0.000314594375941841\\
0.902	-0.000313605606321269\\
0.903999999999993	-0.0003115424012594\\
0.907999999999979	-0.00030707043771199\\
0.909999999999993	-0.00030465992587264\\
0.910000000000001	-0.000304659925872631\\
0.917999999999972	-0.000293831753429642\\
0.919999999999993	-0.000290822872510042\\
0.920000000000001	-0.000290822872510032\\
0.927999999999972	-0.000278102617137219\\
0.927999999999994	-0.000278102617137184\\
0.928000000000001	-0.000278102617137172\\
0.929999999999994	-0.000274779995718834\\
0.930000000000001	-0.000274779995718822\\
0.931999999999994	-0.000271402704015578\\
0.933999999999987	-0.000267974341572245\\
0.937999999999972	-0.000260959007541334\\
0.939999999999993	-0.000257369285387744\\
0.940000000000001	-0.000257369285387731\\
0.944999999999994	-0.000248144342019191\\
0.945000000000001	-0.000248144342019178\\
0.949999999999994	-0.000238543151774327\\
0.950000000000001	-0.000238543151774313\\
0.954999999999994	-0.000228542184423157\\
0.956999999999994	-0.000224424403949383\\
0.957000000000001	-0.000224424403949368\\
0.959999999999993	-0.000218116930134359\\
0.960000000000001	-0.000218116930134344\\
0.962999999999993	-0.000211787464606918\\
0.965999999999986	-0.000205570352027769\\
0.969999999999993	-0.000197446174189792\\
0.970000000000001	-0.000197446174189778\\
0.975999999999986	-0.000185588360519031\\
0.979999999999993	-0.000177884039190569\\
0.980000000000001	-0.000177884039190555\\
0.985999999999986	-0.000166598677044058\\
0.985999999999993	-0.000166598677044044\\
0.986000000000001	-0.000166598677044031\\
0.989999999999993	-0.000159238640787061\\
0.990000000000001	-0.000159238640787048\\
0.993999999999993	-0.000152203250136343\\
0.997999999999986	-0.000145688781985193\\
0.999999999999993	-0.000142623669357667\\
1	-0.000142623669357657\\
1.00799999999999	-0.000131616565662225\\
1.00999999999999	-0.00012917273208024\\
1.01	-0.000129172732080223\\
1.01499999999999	-0.000123518085292801\\
1.015	-0.000123518085292785\\
1.01999999999999	-0.000118460570614788\\
1.02	-0.000118460570614774\\
1.02499999999999	-0.000113987793199313\\
1.02999999999997	-0.000110088791430314\\
1.03	-0.000110088791430294\\
1.03999999999997	-0.000103479707608394\\
1.04	-0.000103479707608377\\
1.04399999999999	-0.000101175945377768\\
1.044	-0.000101175945377761\\
1.04799999999999	-9.90608062925625e-05\\
1.04999999999999	-9.80729222172685e-05\\
1.05	-9.80729222172616e-05\\
1.05399999999999	-9.6177602974403e-05\\
1.05799999999997	-9.43490965384546e-05\\
1.05999999999999	-9.34589979262636e-05\\
1.06	-9.34589979262574e-05\\
1.06799999999997	-9.00515625303967e-05\\
1.06999999999999	-8.92362735415734e-05\\
1.07	-8.92362735415677e-05\\
1.07299999999999	-8.80394545716419e-05\\
1.073	-8.80394545716363e-05\\
1.07599999999999	-8.68730366506243e-05\\
1.07899999999997	-8.57359907502116e-05\\
1.07999999999999	-8.53633284642103e-05\\
1.08	-8.5363328464205e-05\\
1.08499999999999	-8.35461100915477e-05\\
1.085	-8.35461100915426e-05\\
1.08999999999999	-8.18025023960743e-05\\
1.09	-8.18025023960694e-05\\
1.09499999999999	-8.00618123239189e-05\\
1.09999999999997	-7.82533539751721e-05\\
1.09999999999999	-7.82533539751668e-05\\
1.1	-7.82533539751615e-05\\
1.10199999999999	-7.7510008469497e-05\\
1.102	-7.75100084694916e-05\\
1.10399999999999	-7.675482155282e-05\\
1.10599999999997	-7.59874970617088e-05\\
1.10999999999994	-7.44152271576691e-05\\
1.10999999999999	-7.44152271576516e-05\\
1.11	-7.44152271576459e-05\\
1.11799999999994	-7.11114664205761e-05\\
1.11999999999999	-7.0250474023457e-05\\
1.12	-7.02504740234508e-05\\
1.12799999999994	-6.67858740899437e-05\\
1.12999999999999	-6.59209714122933e-05\\
1.13	-6.59209714122871e-05\\
1.13099999999999	-6.54884739975316e-05\\
1.131	-6.54884739975254e-05\\
1.132	-6.50558893623983e-05\\
1.13299999999999	-6.46231750904298e-05\\
1.13499999999999	-6.37571879771102e-05\\
1.13899999999997	-6.20217906837041e-05\\
1.13999999999999	-6.15869802056827e-05\\
1.14	-6.15869802056765e-05\\
1.14799999999997	-5.80879995338669e-05\\
1.14999999999999	-5.72059883990769e-05\\
1.15	-5.72059883990706e-05\\
1.15499999999999	-5.50513794777781e-05\\
1.155	-5.50513794777721e-05\\
1.15999999999999	-5.30028342394061e-05\\
1.16	-5.30028342394004e-05\\
1.16499999999999	-5.10129318808779e-05\\
1.16999999999997	-4.90343953812653e-05\\
1.16999999999999	-4.90343953812597e-05\\
1.17	-4.9034395381254e-05\\
1.17999999999997	-4.50920404146876e-05\\
1.17999999999999	-4.50920404146819e-05\\
1.18	-4.50920404146763e-05\\
1.18899999999999	-4.15342623684143e-05\\
1.189	-4.15342623684087e-05\\
1.18999999999999	-4.11370988638729e-05\\
1.19	-4.11370988638673e-05\\
1.191	-4.07417576403654e-05\\
1.19199999999999	-4.03505355699611e-05\\
1.19399999999999	-3.9580295913081e-05\\
1.19799999999997	-3.80875858982017e-05\\
1.19999999999999	-3.73645302765287e-05\\
1.2	-3.73645302765236e-05\\
1.20799999999997	-3.4621091388731e-05\\
1.20999999999999	-3.39710830709112e-05\\
1.21	-3.39710830709066e-05\\
1.21799999999997	-3.15228970327698e-05\\
1.218	-3.15228970327618e-05\\
1.21999999999999	-3.09486176065481e-05\\
1.22	-3.0948617606544e-05\\
1.22199999999999	-3.03889859706495e-05\\
1.22399999999997	-2.98437826837512e-05\\
1.22499999999999	-2.9576524655378e-05\\
1.225	-2.95765246553742e-05\\
1.22899999999997	-2.85425094680746e-05\\
1.22999999999999	-2.82926332217404e-05\\
1.23	-2.82926332217369e-05\\
1.23399999999997	-2.7323487331463e-05\\
1.23799999999994	-2.6400420021507e-05\\
1.23999999999999	-2.59557106396736e-05\\
1.24	-2.59557106396705e-05\\
1.24699999999999	-2.44846078433585e-05\\
1.247	-2.44846078433556e-05\\
1.24999999999999	-2.38935603471624e-05\\
1.25	-2.38935603471597e-05\\
1.25299999999999	-2.33254048757642e-05\\
1.25599999999997	-2.27796402106095e-05\\
1.25999999999999	-2.20859547896303e-05\\
1.26	-2.20859547896279e-05\\
1.26599999999997	-2.11368468557044e-05\\
1.26999999999999	-2.05728278320262e-05\\
1.27	-2.05728278320243e-05\\
1.27599999999997	-1.98276669456435e-05\\
1.276	-1.98276669456403e-05\\
1.27999999999999	-1.93970018886283e-05\\
1.28	-1.93970018886269e-05\\
1.28399999999999	-1.90065384414176e-05\\
1.28799999999997	-1.86437787838281e-05\\
1.28999999999999	-1.8472608399611e-05\\
1.29	-1.84726083996098e-05\\
1.295	-1.8073924372266e-05\\
1.29500000000001	-1.80739243722649e-05\\
1.3	-1.77162451166257e-05\\
1.30000000000001	-1.77162451166248e-05\\
1.305	-1.7398694040283e-05\\
1.30500000000001	-1.73986940402822e-05\\
1.31	-1.71204929080031e-05\\
1.31000000000001	-1.71204929080023e-05\\
1.315	-1.68726622586677e-05\\
1.31999999999998	-1.66462970613828e-05\\
1.32	-1.66462970613821e-05\\
1.32999999999997	-1.6182407913925e-05\\
1.33	-1.61824079139237e-05\\
1.33399999999999	-1.59781921743013e-05\\
1.334	-1.59781921743006e-05\\
1.33799999999999	-1.5762842652407e-05\\
1.33999999999999	-1.56508878659748e-05\\
1.34	-1.5650887865974e-05\\
1.34399999999999	-1.54181987659343e-05\\
1.34799999999997	-1.51734975921986e-05\\
1.34999999999999	-1.50465232921688e-05\\
1.35	-1.50465232921679e-05\\
1.35799999999997	-1.45157703747024e-05\\
1.35999999999999	-1.43776793979001e-05\\
1.36	-1.43776793978991e-05\\
1.36299999999999	-1.41662731147417e-05\\
1.363	-1.41662731147407e-05\\
1.36499999999999	-1.40224084739941e-05\\
1.365	-1.40224084739931e-05\\
1.36699999999999	-1.3876136464917e-05\\
1.36899999999997	-1.37273997413389e-05\\
1.36999999999999	-1.36520889270863e-05\\
1.37	-1.36520889270852e-05\\
1.37399999999997	-1.33472884960282e-05\\
1.37799999999995	-1.30375964274968e-05\\
1.37999999999999	-1.28807645570106e-05\\
1.38	-1.28807645570095e-05\\
1.38799999999995	-1.22387569278706e-05\\
1.38999999999999	-1.20742703535102e-05\\
1.39	-1.2074270353509e-05\\
1.39199999999999	-1.19089782731835e-05\\
1.392	-1.19089782731824e-05\\
1.39399999999998	-1.17437322663343e-05\\
1.39599999999997	-1.15784675384878e-05\\
1.39999999999994	-1.12476227181129e-05\\
1.39999999999999	-1.12476227181091e-05\\
1.4	-1.12476227181079e-05\\
1.40799999999994	-1.05828539493579e-05\\
1.41	-1.04156403682309e-05\\
1.41000000000001	-1.04156403682297e-05\\
1.41799999999995	-9.74073318180252e-06\\
1.41999999999999	-9.57016241575112e-06\\
1.42	-9.5701624157499e-06\\
1.42099999999999	-9.48510940411088e-06\\
1.421	-9.48510940410967e-06\\
1.422	-9.40093985731432e-06\\
1.42299999999999	-9.31764552193155e-06\\
1.42499999999998	-9.15364991553894e-06\\
1.42899999999997	-8.83580768595152e-06\\
1.42999999999999	-8.75841594688794e-06\\
1.43	-8.75841594688685e-06\\
1.43499999999999	-8.38349188679117e-06\\
1.435	-8.38349188679013e-06\\
1.43999999999998	-8.02795891160401e-06\\
1.44	-8.02795891160285e-06\\
1.44499999999999	-7.6918743754733e-06\\
1.44999999999997	-7.37534330672069e-06\\
1.44999999999998	-7.37534330671979e-06\\
1.45	-7.37534330671889e-06\\
1.45999999999997	-6.79788465090023e-06\\
1.45999999999998	-6.79788465089938e-06\\
1.46	-6.79788465089855e-06\\
1.46999999999997	-6.28420796862364e-06\\
1.46999999999998	-6.2842079686229e-06\\
1.47	-6.28420796862214e-06\\
1.47899999999998	-5.86737449984043e-06\\
1.479	-5.8673744998398e-06\\
1.47999999999999	-5.82356391729689e-06\\
1.48	-5.82356391729627e-06\\
1.481	-5.78018912457834e-06\\
1.48199999999999	-5.73719690438202e-06\\
1.48399999999999	-5.65234336389853e-06\\
1.48799999999997	-5.48704421281186e-06\\
1.48999999999999	-5.40653379153875e-06\\
1.49	-5.40653379153819e-06\\
1.49799999999997	-5.09802290556219e-06\\
1.49999999999999	-5.02412669001387e-06\\
1.5	-5.02412669001335e-06\\
1.50499999999999	-4.84472233789819e-06\\
1.505	-4.84472233789769e-06\\
1.50799999999998	-4.74059844340479e-06\\
1.508	-4.7405984434043e-06\\
1.50999999999999	-4.67259159220506e-06\\
1.51	-4.67259159220458e-06\\
1.51199999999999	-4.60620380003686e-06\\
1.51399999999997	-4.54193200582875e-06\\
1.51799999999994	-4.41963643965947e-06\\
1.51999999999999	-4.36156471780063e-06\\
1.52	-4.36156471780023e-06\\
1.52799999999994	-4.1354138083054e-06\\
1.52999999999999	-4.07943410167076e-06\\
1.53	-4.07943410167036e-06\\
1.53699999999998	-3.88477286306704e-06\\
1.537	-3.88477286306665e-06\\
1.53999999999999	-3.80178680749949e-06\\
1.54	-3.8017868074991e-06\\
1.54299999999999	-3.71895902663506e-06\\
1.54599999999997	-3.63621645142227e-06\\
1.54999999999999	-3.52589940392205e-06\\
1.55	-3.52589940392166e-06\\
1.55599999999997	-3.36465081405609e-06\\
1.56	-3.26175233456832e-06\\
1.56000000000001	-3.26175233456796e-06\\
1.56599999999999	-3.1139019785524e-06\\
1.566	-3.11390197855206e-06\\
1.57	-3.01944120594047e-06\\
1.57000000000001	-3.01944120594014e-06\\
1.57400000000001	-2.92809175786984e-06\\
1.57499999999999	-2.90572364684587e-06\\
1.575	-2.90572364684556e-06\\
1.579	-2.81806243236287e-06\\
1.57999999999999	-2.79658919644889e-06\\
1.58	-2.79658919644859e-06\\
1.584	-2.71408272039318e-06\\
1.588	-2.63757133604752e-06\\
1.59	-2.60152593969683e-06\\
1.59000000000001	-2.60152593969658e-06\\
1.59499999999998	-2.51412913041771e-06\\
1.595	-2.51412913041746e-06\\
1.59999999999997	-2.42837582822453e-06\\
1.6	-2.42837582822399e-06\\
1.60000000000001	-2.42837582822375e-06\\
1.60499999999998	-2.34405587167826e-06\\
1.60999999999995	-2.26096261446063e-06\\
1.60999999999998	-2.26096261446026e-06\\
1.61	-2.26096261445989e-06\\
1.61999999999994	-2.09764414540558e-06\\
1.62	-2.0976441454046e-06\\
1.62000000000001	-2.09764414540437e-06\\
1.62399999999998	-2.03418756527293e-06\\
1.624	-2.03418756527271e-06\\
1.62799999999997	-1.9731982826703e-06\\
1.63000000000001	-1.94359883578264e-06\\
1.63000000000003	-1.94359883578244e-06\\
1.634	-1.886132304118e-06\\
1.63799999999997	-1.83090085696096e-06\\
1.63999999999999	-1.80409608563959e-06\\
1.64	-1.8040960856394e-06\\
1.64499999999999	-1.7392881786609e-06\\
1.645	-1.73928817866072e-06\\
1.64999999999998	-1.67745862571263e-06\\
1.65	-1.67745862571242e-06\\
1.653	-1.64172717467939e-06\\
1.65300000000001	-1.64172717467923e-06\\
1.65600000000001	-1.60698172692253e-06\\
1.65900000000001	-1.57319163153351e-06\\
1.66	-1.56213539098119e-06\\
1.66000000000002	-1.56213539098104e-06\\
1.66600000000001	-1.49789742039325e-06\\
1.67	-1.45699517814464e-06\\
1.67000000000002	-1.4569951781445e-06\\
1.67600000000001	-1.39968997688897e-06\\
1.67999999999998	-1.36468623029255e-06\\
1.68	-1.36468623029243e-06\\
1.68199999999998	-1.34784840090347e-06\\
1.682	-1.34784840090336e-06\\
1.68399999999998	-1.33107786948138e-06\\
1.68599999999997	-1.31436806081161e-06\\
1.68999999999994	-1.28110442856421e-06\\
1.68999999999998	-1.28110442856383e-06\\
1.69	-1.28110442856372e-06\\
1.69799999999994	-1.21501886846055e-06\\
1.69999999999998	-1.19855169779309e-06\\
1.7	-1.19855169779298e-06\\
1.70799999999994	-1.13270615264987e-06\\
1.70999999999998	-1.11621828157094e-06\\
1.71	-1.11621828157083e-06\\
1.711	-1.10800176251657e-06\\
1.71100000000001	-1.10800176251645e-06\\
1.71200000000001	-1.09985021937416e-06\\
1.71300000000001	-1.09176285283314e-06\\
1.715	-1.0757774851832e-06\\
1.71500000000001	-1.07577748518309e-06\\
1.71700000000001	-1.06003939248359e-06\\
1.71900000000001	-1.04454240459638e-06\\
1.71999999999998	-1.03688242267318e-06\\
1.72	-1.03688242267307e-06\\
1.724	-1.00681512337756e-06\\
1.72799999999999	-9.77628863119831e-07\\
1.72999999999998	-9.63351765227398e-07\\
1.73	-9.63351765227298e-07\\
1.73799999999999	-9.08220568807584e-07\\
1.74	-8.94905050029216e-07\\
1.74000000000001	-8.94905050029122e-07\\
1.74800000000001	-8.45423432646866e-07\\
1.74999999999998	-8.34103079362706e-07\\
1.75	-8.34103079362627e-07\\
1.75799999999999	-7.90179089452211e-07\\
1.75999999999998	-7.79346654904122e-07\\
1.76	-7.79346654904045e-07\\
1.76799999999999	-7.36483722537377e-07\\
1.76899999999998	-7.31169702581359e-07\\
1.769	-7.31169702581284e-07\\
1.76999999999998	-7.25863685036019e-07\\
1.77	-7.25863685035943e-07\\
1.771	-7.20565150087193e-07\\
1.77199999999999	-7.15273578191753e-07\\
1.77399999999998	-7.04709249676107e-07\\
1.77799999999997	-6.83641369082132e-07\\
1.77999999999998	-6.73129556706323e-07\\
1.78	-6.73129556706249e-07\\
1.78499999999998	-6.47467912446302e-07\\
1.785	-6.47467912446231e-07\\
1.78999999999998	-6.22974034316735e-07\\
1.79	-6.22974034316664e-07\\
1.79499999999998	-5.99587893434757e-07\\
1.798	-5.86063971681208e-07\\
1.79800000000001	-5.86063971681145e-07\\
1.8	-5.77252176305366e-07\\
1.80000000000001	-5.77252176305304e-07\\
1.802	-5.68601568826153e-07\\
1.80399999999999	-5.60110663255775e-07\\
1.80799999999996	-5.43594703668865e-07\\
1.81	-5.35563174055897e-07\\
1.81000000000001	-5.35563174055841e-07\\
1.81799999999996	-5.04873917348146e-07\\
1.81999999999998	-4.97545778528657e-07\\
1.82	-4.97545778528606e-07\\
1.82699999999998	-4.72812389832366e-07\\
1.827	-4.72812389832317e-07\\
1.82999999999998	-4.6261313542562e-07\\
1.83	-4.62613135425572e-07\\
1.83299999999999	-4.52641246370177e-07\\
1.83599999999997	-4.42887925632578e-07\\
1.84	-4.3020864893363e-07\\
1.84000000000001	-4.30208648933585e-07\\
1.84599999999999	-4.11822465360447e-07\\
1.85	-3.9994820017622e-07\\
1.85000000000001	-3.99948200176179e-07\\
1.85499999999998	-3.85503668253401e-07\\
1.855	-3.85503668253361e-07\\
1.85599999999998	-3.8266502424376e-07\\
1.856	-3.8266502424372e-07\\
1.85699999999999	-3.79842481433457e-07\\
1.85799999999999	-3.77035763159663e-07\\
1.85999999999998	-3.71468701395167e-07\\
1.86	-3.71468701395117e-07\\
1.86399999999999	-3.60512469061441e-07\\
1.86799999999997	-3.49779143941944e-07\\
1.86999999999999	-3.44490797809161e-07\\
1.87	-3.44490797809124e-07\\
1.87799999999997	-3.24598744015289e-07\\
1.87999999999999	-3.19980501534169e-07\\
1.88	-3.19980501534137e-07\\
1.88499999999998	-3.09035713196465e-07\\
1.885	-3.09035713196435e-07\\
1.88999999999998	-2.98928028982588e-07\\
1.89	-2.98928028982553e-07\\
1.89499999999998	-2.89519334852739e-07\\
1.89999999999996	-2.80673230156768e-07\\
1.89999999999998	-2.80673230156738e-07\\
1.9	-2.80673230156709e-07\\
1.90999999999996	-2.64583396732426e-07\\
1.91	-2.64583396732374e-07\\
1.91000000000001	-2.64583396732352e-07\\
1.91399999999998	-2.58640594949897e-07\\
1.914	-2.58640594949876e-07\\
1.91799999999997	-2.5285533944778e-07\\
1.91999999999999	-2.50018946547335e-07\\
1.92	-2.50018946547315e-07\\
1.92399999999997	-2.44453069985938e-07\\
1.92499999999998	-2.43083076799473e-07\\
1.925	-2.43083076799454e-07\\
1.92899999999997	-2.37684979347253e-07\\
1.92999999999999	-2.36355262466908e-07\\
1.93	-2.36355262466889e-07\\
1.93399999999997	-2.31111714788622e-07\\
1.93799999999994	-2.25982537400944e-07\\
1.93999999999999	-2.23458316883346e-07\\
1.94	-2.23458316883328e-07\\
1.94299999999998	-2.19750345120494e-07\\
1.943	-2.19750345120476e-07\\
1.94599999999998	-2.16158052084892e-07\\
1.94899999999996	-2.12678268605521e-07\\
1.95	-2.11542809512277e-07\\
1.95000000000002	-2.11542809512261e-07\\
1.95599999999998	-2.04711314393199e-07\\
1.95999999999998	-2.0008841789635e-07\\
1.96	-2.00088417896333e-07\\
1.96599999999996	-1.93033095225397e-07\\
1.97	-1.88238136792647e-07\\
1.97000000000001	-1.8823813679263e-07\\
1.97199999999998	-1.85810425800462e-07\\
1.972	-1.85810425800445e-07\\
1.97399999999997	-1.83361293977331e-07\\
1.97599999999994	-1.8088978109603e-07\\
1.97999999999988	-1.75875727165959e-07\\
1.98	-1.75875727165809e-07\\
1.98000000000002	-1.75875727165791e-07\\
1.9879999999999	-1.65941906997073e-07\\
1.99	-1.63514690440074e-07\\
1.99000000000002	-1.63514690440057e-07\\
1.99499999999998	-1.57534622188661e-07\\
1.995	-1.57534622188644e-07\\
1.99999999999997	-1.51668541679637e-07\\
1.99999999999998	-1.51668541679617e-07\\
2	-1.51668541679597e-07\\
2.00099999999997	-1.50509532132948e-07\\
2.001	-1.50509532132915e-07\\
2.00199999999999	-1.49357988038231e-07\\
2.00299999999999	-1.48213796522265e-07\\
2.00499999999998	-1.45947023365456e-07\\
2.00899999999997	-1.4149682065517e-07\\
2.00999999999997	-1.4040099612552e-07\\
2.01	-1.40400996125489e-07\\
2.01799999999997	-1.31858598355349e-07\\
2.01999999999997	-1.29781444248783e-07\\
2.02	-1.29781444248754e-07\\
2.02799999999997	-1.2168217103391e-07\\
2.02999999999997	-1.19705719280349e-07\\
2.03	-1.19705719280321e-07\\
2.03799999999997	-1.12298491979534e-07\\
2.03999999999997	-1.10588212849529e-07\\
2.04	-1.10588212849505e-07\\
2.04799999999997	-1.04293085417701e-07\\
2.04999999999997	-1.02852715724958e-07\\
2.05	-1.02852715724938e-07\\
2.05799999999997	-9.75006450775584e-08\\
2.05899999999997	-9.68735927366414e-08\\
2.059	-9.68735927366237e-08\\
2.05999999999997	-9.62556539200199e-08\\
2.06	-9.62556539200025e-08\\
2.06099999999999	-9.56467680918408e-08\\
2.06199999999999	-9.5046875545577e-08\\
2.06399999999998	-9.38738359915828e-08\\
2.06499999999997	-9.33005740116015e-08\\
2.065	-9.33005740115854e-08\\
2.06899999999998	-9.1094610399996e-08\\
2.06999999999997	-9.05646203133432e-08\\
2.07	-9.05646203133283e-08\\
2.07399999999998	-8.8497237322697e-08\\
2.07799999999997	-8.64988165408296e-08\\
2.07999999999997	-8.55244833791364e-08\\
2.08	-8.55244833791227e-08\\
2.08799999999997	-8.17841586434837e-08\\
2.088	-8.178415864347e-08\\
2.08999999999997	-8.08864963937365e-08\\
2.09	-8.08864963937238e-08\\
2.09199999999997	-8.00030892620572e-08\\
2.09399999999994	-7.91335908519449e-08\\
2.09799999999988	-7.74349619640981e-08\\
2.09999999999997	-7.66051654862678e-08\\
2.1	-7.66051654862561e-08\\
2.10799999999988	-7.34621867475569e-08\\
2.10999999999997	-7.27223032497461e-08\\
2.11	-7.27223032497357e-08\\
2.11699999999997	-7.01359118043542e-08\\
2.117	-7.01359118043436e-08\\
2.11999999999997	-6.90084786995951e-08\\
2.12	-6.90084786995844e-08\\
2.12299999999997	-6.78682262649285e-08\\
2.12599999999994	-6.67141485939064e-08\\
2.12999999999997	-6.5152110935404e-08\\
2.13	-6.51521109353928e-08\\
2.13499999999997	-6.31587170592184e-08\\
2.135	-6.3158717059207e-08\\
2.13999999999997	-6.11153729487431e-08\\
2.14	-6.11153729487314e-08\\
2.14499999999998	-5.90655764122158e-08\\
2.14599999999997	-5.8660219728273e-08\\
2.146	-5.86602197282615e-08\\
2.14999999999997	-5.70528094421023e-08\\
2.15	-5.7052809442091e-08\\
2.15399999999998	-5.54659396014975e-08\\
2.15799999999995	-5.38971214237435e-08\\
2.15999999999997	-5.31187117659023e-08\\
2.16	-5.31187117658912e-08\\
2.16799999999995	-5.00767085310798e-08\\
2.16999999999997	-4.93350460865037e-08\\
2.17	-4.93350460864932e-08\\
2.17499999999997	-4.75106647076705e-08\\
2.175	-4.75106647076602e-08\\
2.17999999999997	-4.57252541013196e-08\\
2.18	-4.57252541013083e-08\\
2.18499999999997	-4.39838928097029e-08\\
2.18999999999994	-4.22917673781084e-08\\
2.18999999999997	-4.22917673780976e-08\\
2.19	-4.22917673780881e-08\\
2.19999999999994	-3.90387467611347e-08\\
2.19999999999997	-3.90387467611244e-08\\
2.2	-3.90387467611154e-08\\
2.20399999999997	-3.77808690538949e-08\\
2.204	-3.77808690538861e-08\\
2.20499999999997	-3.74698790930857e-08\\
2.205	-3.74698790930769e-08\\
2.206	-3.71602197407196e-08\\
2.20699999999999	-3.68518606458321e-08\\
2.20899999999998	-3.62389224633498e-08\\
2.20999999999997	-3.59342833003281e-08\\
2.21	-3.59342833003195e-08\\
2.21399999999998	-3.47532303962502e-08\\
2.21799999999997	-3.36412082875508e-08\\
2.21999999999997	-3.31105345788058e-08\\
2.22	-3.31105345787984e-08\\
2.22799999999997	-3.11519764607153e-08\\
2.22999999999997	-3.07024112634375e-08\\
2.23	-3.07024112634312e-08\\
2.23299999999997	-3.0053460407616e-08\\
2.233	-3.005346040761e-08\\
2.23599999999997	-2.94313374040768e-08\\
2.23899999999994	-2.88354934054765e-08\\
2.23999999999997	-2.86426270073689e-08\\
2.24	-2.86426270073634e-08\\
2.24599999999994	-2.75444767703541e-08\\
2.24999999999997	-2.68673123794119e-08\\
2.25	-2.68673123794073e-08\\
2.25599999999994	-2.59107138025959e-08\\
2.25999999999997	-2.53018006409581e-08\\
2.26	-2.53018006409538e-08\\
2.26199999999997	-2.50056283860713e-08\\
2.262	-2.50056283860671e-08\\
2.26399999999997	-2.47148234869504e-08\\
2.26599999999994	-2.44292719274784e-08\\
2.26999999999988	-2.38734830398353e-08\\
2.26999999999997	-2.38734830398229e-08\\
2.27	-2.3873483039819e-08\\
2.27499999999997	-2.31986194585733e-08\\
2.275	-2.31986194585695e-08\\
2.27999999999997	-2.25374278449599e-08\\
2.28	-2.25374278449562e-08\\
2.28499999999997	-2.18882877717095e-08\\
2.28999999999995	-2.12496083650575e-08\\
2.29	-2.12496083650507e-08\\
2.29099999999997	-2.11229896032781e-08\\
2.291	-2.11229896032745e-08\\
2.29199999999999	-2.0996714190757e-08\\
2.29299999999999	-2.087076973941e-08\\
2.29499999999998	-2.06198243759904e-08\\
2.29899999999997	-2.01213640778561e-08\\
2.29999999999997	-1.99973923854767e-08\\
2.3	-1.99973923854732e-08\\
2.30799999999997	-1.9043034047624e-08\\
2.30999999999997	-1.8815452475335e-08\\
2.31	-1.88154524753318e-08\\
2.31799999999997	-1.794648069634e-08\\
2.31999999999997	-1.77391505165955e-08\\
2.32	-1.77391505165925e-08\\
2.32799999999997	-1.69383337235426e-08\\
2.32999999999997	-1.6744319071297e-08\\
2.33	-1.67443190712943e-08\\
2.33799999999997	-1.5990736025936e-08\\
2.33999999999997	-1.58075898429967e-08\\
2.34	-1.58075898429941e-08\\
2.34499999999997	-1.53557189594839e-08\\
2.345	-1.53557189594813e-08\\
2.34899999999997	-1.49988156417685e-08\\
2.349	-1.4998815641766e-08\\
2.34999999999997	-1.49101547050569e-08\\
2.35	-1.49101547050544e-08\\
2.35099999999999	-1.4821702314004e-08\\
2.35199999999999	-1.47334497990372e-08\\
2.35399999999998	-1.4557509817439e-08\\
2.35799999999996	-1.42076489904189e-08\\
2.35999999999997	-1.40335909714773e-08\\
2.36	-1.40335909714748e-08\\
2.36799999999996	-1.33434189214e-08\\
2.36999999999997	-1.31721672622281e-08\\
2.37	-1.31721672622257e-08\\
2.37799999999996	-1.2490311225873e-08\\
2.378	-1.24903112258702e-08\\
2.37999999999997	-1.23203007069544e-08\\
2.38	-1.23203007069519e-08\\
2.38199999999997	-1.21503382091269e-08\\
2.38399999999994	-1.19803570885372e-08\\
2.38799999999988	-1.16400723804213e-08\\
2.38999999999997	-1.14696353740644e-08\\
2.39	-1.14696353740619e-08\\
2.39799999999988	-1.08132943012622e-08\\
2.39999999999997	-1.06570464716145e-08\\
2.4	-1.06570464716123e-08\\
2.40699999999997	-1.01334965078332e-08\\
2.407	-1.01334965078311e-08\\
2.40999999999997	-9.91978269868879e-09\\
2.41	-9.9197826986868e-09\\
2.41299999999997	-9.71205926303422e-09\\
2.41499999999997	-9.57674009994162e-09\\
2.415	-9.57674009993971e-09\\
2.41799999999997	-9.37837341183913e-09\\
2.41999999999997	-9.24912804760087e-09\\
2.42	-9.24912804759905e-09\\
2.42299999999997	-9.05963196180421e-09\\
2.42599999999994	-8.87523397927092e-09\\
2.42999999999997	-8.63701986088292e-09\\
2.42999999999999	-8.63701986088126e-09\\
2.43599999999994	-8.29333449110274e-09\\
2.43599999999997	-8.29333449110102e-09\\
2.436	-8.29333449109933e-09\\
2.43999999999997	-8.07188734077937e-09\\
2.43999999999999	-8.07188734077782e-09\\
2.44399999999996	-7.85617476063091e-09\\
2.44799999999993	-7.64585843445041e-09\\
2.44999999999999	-7.54262046545015e-09\\
2.45000000000002	-7.5426204654487e-09\\
2.45799999999996	-7.14153100452718e-09\\
2.45999999999997	-7.04402767899343e-09\\
2.46	-7.04402767899205e-09\\
2.46499999999997	-6.80958804106677e-09\\
2.465	-6.80958804106549e-09\\
2.46999999999996	-6.59080123893297e-09\\
2.47	-6.59080123893132e-09\\
2.47499999999997	-6.38017963605352e-09\\
2.47999999999993	-6.17025561394541e-09\\
2.48	-6.1702556139426e-09\\
2.48000000000003	-6.17025561394141e-09\\
2.48499999999997	-5.96051468229027e-09\\
2.485	-5.96051468228908e-09\\
2.48999999999994	-5.75044286275796e-09\\
2.48999999999999	-5.75044286275615e-09\\
2.49000000000003	-5.75044286275433e-09\\
2.49399999999997	-5.58180124510572e-09\\
2.494	-5.58180124510452e-09\\
2.49799999999993	-5.41235390295718e-09\\
2.49999999999997	-5.32724509746643e-09\\
2.5	-5.32724509746521e-09\\
2.50399999999994	-5.1595007673463e-09\\
2.50799999999988	-4.99710918787815e-09\\
2.50999999999997	-4.91784076780849e-09\\
2.51	-4.91784076780737e-09\\
2.51799999999988	-4.61289670873426e-09\\
2.51999999999997	-4.53954365351169e-09\\
2.52	-4.53954365351066e-09\\
2.52299999999997	-4.43181696154967e-09\\
2.523	-4.43181696154866e-09\\
2.52599999999996	-4.32697636218877e-09\\
2.52899999999993	-4.224929363717e-09\\
2.52999999999997	-4.19151886857787e-09\\
2.53	-4.19151886857693e-09\\
2.53599999999993	-3.99718248669507e-09\\
2.53999999999997	-3.8732284782851e-09\\
2.54	-3.87322847828423e-09\\
2.54599999999993	-3.69510654296947e-09\\
2.54999999999997	-3.58124405766897e-09\\
2.55	-3.58124405766818e-09\\
2.55199999999997	-3.52571092877424e-09\\
2.552	-3.52571092877346e-09\\
2.55399999999996	-3.47108072725826e-09\\
2.55499999999997	-3.4440975171927e-09\\
2.555	-3.44409751719194e-09\\
2.55699999999997	-3.39078165503003e-09\\
2.55899999999994	-3.33831581999899e-09\\
2.55999999999997	-3.31239522185945e-09\\
2.56	-3.31239522185872e-09\\
2.56399999999994	-3.21073591335066e-09\\
2.56799999999987	-3.11219467034416e-09\\
2.56999999999997	-3.06404483987443e-09\\
2.57	-3.06404483987375e-09\\
2.57799999999987	-2.88524407180878e-09\\
2.57999999999997	-2.84432842347087e-09\\
2.58	-2.8443284234703e-09\\
2.58099999999997	-2.82442689914827e-09\\
2.581	-2.82442689914771e-09\\
2.58199999999999	-2.80489362923989e-09\\
2.58299999999999	-2.78572669831305e-09\\
2.58499999999998	-2.74848437488904e-09\\
2.58899999999996	-2.67831552252334e-09\\
2.58999999999997	-2.66166235193637e-09\\
2.59	-2.6616623519359e-09\\
2.59799999999997	-2.53710187026083e-09\\
2.59999999999997	-2.50817797495435e-09\\
2.6	-2.50817797495394e-09\\
2.60799999999997	-2.40100700001944e-09\\
2.60999999999997	-2.37629288570651e-09\\
2.61	-2.37629288570616e-09\\
2.61799999999996	-2.28546018087545e-09\\
2.62	-2.26471342760547e-09\\
2.62000000000002	-2.26471342760518e-09\\
2.62499999999997	-2.21602389595769e-09\\
2.625	-2.21602389595742e-09\\
2.62999999999995	-2.17168505826279e-09\\
2.62999999999999	-2.17168505826248e-09\\
2.63000000000002	-2.17168505826217e-09\\
2.63499999999997	-2.12898815044444e-09\\
2.63899999999997	-2.09407051001531e-09\\
2.639	-2.09407051001506e-09\\
2.63999999999997	-2.08522843386654e-09\\
2.64	-2.08522843386628e-09\\
2.641	-2.0763395648682e-09\\
2.64199999999999	-2.06740303174738e-09\\
2.64399999999999	-2.04938346488736e-09\\
2.64799999999997	-2.01273349967945e-09\\
2.64999999999997	-1.99408873163685e-09\\
2.65	-1.99408873163658e-09\\
2.65799999999997	-1.91720616116652e-09\\
2.65999999999997	-1.89737196597236e-09\\
2.66	-1.89737196597208e-09\\
2.66799999999997	-1.8173013964392e-09\\
2.668	-1.81730139643891e-09\\
2.66999999999997	-1.79718311763324e-09\\
2.67	-1.79718311763295e-09\\
2.67199999999998	-1.77700883752361e-09\\
2.67399999999995	-1.75677064560987e-09\\
2.6779999999999	-1.7160707605713e-09\\
2.67999999999997	-1.69559310984288e-09\\
2.68	-1.69559310984259e-09\\
2.6879999999999	-1.61365437524818e-09\\
2.68999999999997	-1.59318566162748e-09\\
2.69	-1.59318566162719e-09\\
2.69499999999997	-1.54195404365513e-09\\
2.695	-1.54195404365484e-09\\
2.69699999999997	-1.52141637299945e-09\\
2.697	-1.52141637299916e-09\\
2.69899999999996	-1.50084090326274e-09\\
2.69999999999997	-1.49053647351744e-09\\
2.7	-1.49053647351715e-09\\
2.70199999999997	-1.469962942423e-09\\
2.70399999999993	-1.44947897560284e-09\\
2.70799999999986	-1.40874764369428e-09\\
2.70999999999997	-1.3884843086538e-09\\
2.71	-1.38848430865351e-09\\
2.71799999999986	-1.30792805561576e-09\\
2.71999999999997	-1.28787378560283e-09\\
2.72	-1.28787378560255e-09\\
2.72599999999997	-1.22778861772034e-09\\
2.726	-1.22778861772005e-09\\
2.72999999999997	-1.18771801916427e-09\\
2.73	-1.18771801916399e-09\\
2.73399999999998	-1.14851586634797e-09\\
2.73799999999995	-1.11107349048925e-09\\
2.73999999999997	-1.09299376156537e-09\\
2.74	-1.09299376156512e-09\\
2.74799999999995	-1.02478774765229e-09\\
2.74999999999997	-1.00873104341549e-09\\
2.75	-1.00873104341526e-09\\
2.75499999999997	-9.70215503026923e-10\\
2.755	-9.7021550302671e-10\\
2.75999999999996	-9.33922475639151e-10\\
2.76	-9.33922475638894e-10\\
2.76499999999997	-8.99763015306507e-10\\
2.765	-8.99763015306281e-10\\
2.76500000000003	-8.99763015306092e-10\\
2.77	-8.67653406061231e-10\\
2.77000000000003	-8.67653406061054e-10\\
2.77499999999999	-8.37269809356801e-10\\
2.77999999999996	-8.08292616656451e-10\\
2.78	-8.08292616656215e-10\\
2.78399999999997	-7.86075636475899e-10\\
2.784	-7.86075636475744e-10\\
2.78799999999996	-7.64678401112527e-10\\
2.78999999999997	-7.54276650686385e-10\\
2.79	-7.54276650686239e-10\\
2.79399999999997	-7.33950032010138e-10\\
2.79799999999993	-7.14168481770912e-10\\
2.79999999999997	-7.0447237949933e-10\\
2.8	-7.04472379499193e-10\\
2.80799999999993	-6.66897653688194e-10\\
2.80999999999997	-6.57787978600437e-10\\
2.81	-6.57787978600308e-10\\
2.81299999999997	-6.44321850768275e-10\\
2.813	-6.44321850768148e-10\\
2.81599999999996	-6.31083241571144e-10\\
2.81899999999993	-6.18060471774781e-10\\
2.81999999999997	-6.13765521871101e-10\\
2.82	-6.13765521870979e-10\\
2.82599999999993	-5.89079510097158e-10\\
2.82999999999997	-5.73725002916078e-10\\
2.83	-5.73725002915972e-10\\
2.83499999999997	-5.55730073334501e-10\\
2.835	-5.55730073334402e-10\\
2.83999999999997	-5.39025472383856e-10\\
2.84	-5.39025472383765e-10\\
2.84199999999997	-5.32662449737619e-10\\
2.842	-5.3266244973753e-10\\
2.84399999999996	-5.26430400524361e-10\\
2.84599999999993	-5.20326881407357e-10\\
2.84999999999985	-5.08495911346259e-10\\
2.85	-5.08495911345832e-10\\
2.85000000000003	-5.0849591134575e-10\\
2.85799999999989	-4.86274307529679e-10\\
2.85999999999997	-4.81006203849977e-10\\
2.86	-4.81006203849903e-10\\
2.86799999999986	-4.60544455576838e-10\\
2.86999999999997	-4.55541887995795e-10\\
2.87	-4.55541887995724e-10\\
2.87099999999997	-4.53056176012854e-10\\
2.871	-4.53056176012784e-10\\
2.87199999999999	-4.50580520147854e-10\\
2.87299999999999	-4.4811467765042e-10\\
2.87499999999998	-4.43211466983318e-10\\
2.87899999999996	-4.33512233753437e-10\\
2.87999999999997	-4.31108368620924e-10\\
2.88	-4.31108368620856e-10\\
2.88799999999997	-4.12142390804831e-10\\
2.88999999999997	-4.07465977989865e-10\\
2.89	-4.07465977989799e-10\\
2.89799999999997	-3.89446722652551e-10\\
2.89999999999997	-3.85134730299006e-10\\
2.9	-3.85134730298945e-10\\
2.90499999999997	-3.74372072640642e-10\\
2.905	-3.7437207264058e-10\\
2.90999999999998	-3.63440760263061e-10\\
2.91000000000001	-3.63440760262998e-10\\
2.91499999999999	-3.52314003096606e-10\\
2.91999999999997	-3.4096453234862e-10\\
2.92	-3.40964532348542e-10\\
2.92899999999997	-3.19885072259964e-10\\
2.92899999999999	-3.19885072259896e-10\\
2.92999999999997	-3.17485577454242e-10\\
2.93	-3.17485577454173e-10\\
2.93099999999999	-3.15086616052151e-10\\
2.93199999999999	-3.1270080638304e-10\\
2.93399999999997	-3.07967708172994e-10\\
2.93799999999994	-2.98649127399725e-10\\
2.94	-2.94059991201173e-10\\
2.94000000000003	-2.94059991201108e-10\\
2.94799999999997	-2.76129359545762e-10\\
2.95	-2.71744392561114e-10\\
2.95000000000003	-2.71744392561052e-10\\
2.95799999999997	-2.54771485580687e-10\\
2.958	-2.54771485580628e-10\\
2.96	-2.50675910802522e-10\\
2.96000000000003	-2.50675910802464e-10\\
2.96200000000003	-2.46636151661437e-10\\
2.96400000000003	-2.42650624113409e-10\\
2.96800000000003	-2.34836034321906e-10\\
2.97	-2.31003908127785e-10\\
2.97000000000003	-2.31003908127731e-10\\
2.97499999999997	-2.2163080318579e-10\\
2.975	-2.21630803185738e-10\\
2.97999999999995	-2.12535422483586e-10\\
2.98	-2.12535422483485e-10\\
2.98499999999995	-2.03927993474582e-10\\
2.98699999999997	-2.00681970088336e-10\\
2.98699999999999	-2.00681970088291e-10\\
2.98999999999997	-1.96019939664457e-10\\
2.99	-1.96019939664414e-10\\
2.99299999999998	-1.91602675355719e-10\\
2.99599999999996	-1.87426280443344e-10\\
2.99999999999997	-1.8222610200507e-10\\
3	-1.82226102005035e-10\\
3.00599999999996	-1.75068142506096e-10\\
3.00999999999997	-1.70655384825394e-10\\
3.01	-1.70655384825364e-10\\
3.01599999999996	-1.64557767045681e-10\\
3.01599999999999	-1.64557767045647e-10\\
3.01600000000002	-1.6455776704562e-10\\
3.01999999999997	-1.60831069957029e-10\\
3.02	-1.60831069957003e-10\\
3.02399999999995	-1.57301711999836e-10\\
3.0279999999999	-1.5389767224842e-10\\
3.02999999999997	-1.52240970790692e-10\\
3.03	-1.52240970790669e-10\\
3.0379999999999	-1.45901265434991e-10\\
3.03999999999997	-1.4438500727574e-10\\
3.04	-1.44385007275719e-10\\
3.04499999999997	-1.40707985246066e-10\\
3.04499999999999	-1.40707985246045e-10\\
3.04999999999996	-1.3718612049681e-10\\
3.05	-1.37186120496782e-10\\
3.05499999999997	-1.33834606884127e-10\\
3.05999999999993	-1.30669055856601e-10\\
3.05999999999997	-1.3066905585658e-10\\
3.06	-1.30669055856558e-10\\
3.06999999999993	-1.24381496211569e-10\\
3.06999999999997	-1.24381496211547e-10\\
3.07	-1.24381496211524e-10\\
3.07399999999997	-1.21781565723582e-10\\
3.07399999999999	-1.21781565723563e-10\\
3.07799999999996	-1.19127045421511e-10\\
3.07999999999997	-1.17778016481e-10\\
3.08	-1.17778016480981e-10\\
3.08399999999997	-1.15033776384914e-10\\
3.08799999999993	-1.12224363213874e-10\\
3.08999999999997	-1.10793843899565e-10\\
3.09	-1.10793843899544e-10\\
3.09799999999993	-1.05097859121997e-10\\
3.09999999999997	-1.03690797307052e-10\\
3.1	-1.03690797307032e-10\\
3.10299999999997	-1.01590657914021e-10\\
3.10299999999999	-1.01590657914002e-10\\
3.10599999999996	-9.95014166311225e-11\\
3.10899999999993	-9.74212303151314e-11\\
3.10999999999997	-9.67295297139426e-11\\
3.11	-9.6729529713923e-11\\
3.11499999999997	-9.33106721281124e-11\\
3.115	-9.33106721280931e-11\\
3.11999999999998	-8.99617653087373e-11\\
3.12000000000001	-8.99617653087184e-11\\
3.12499999999998	-8.66746019490183e-11\\
3.12999999999996	-8.34411260651008e-11\\
3.13000000000001	-8.34411260650683e-11\\
3.13199999999999	-8.21643803651155e-11\\
3.13200000000002	-8.21643803650975e-11\\
3.13400000000001	-8.09012378939449e-11\\
3.136	-7.96512032548095e-11\\
3.13999999999997	-7.718850210907e-11\\
3.14	-7.71885021090527e-11\\
3.14000000000003	-7.71885021090353e-11\\
3.14799999999998	-7.23991512266233e-11\\
3.14999999999998	-7.12274129791639e-11\\
3.15	-7.12274129791473e-11\\
3.15799999999995	-6.66761896726878e-11\\
3.15999999999997	-6.55730292283979e-11\\
3.16	-6.55730292283823e-11\\
3.16099999999997	-6.50263452937866e-11\\
3.16099999999999	-6.50263452937711e-11\\
3.16199999999999	-6.4482854022313e-11\\
3.16299999999998	-6.39425021216175e-11\\
3.16499999999997	-6.28710048945904e-11\\
3.16899999999994	-6.07633726243679e-11\\
3.16999999999997	-6.02435299932207e-11\\
3.17	-6.0243529993206e-11\\
3.17799999999994	-5.61786235759564e-11\\
3.17999999999997	-5.51866383593599e-11\\
3.18	-5.51866383593459e-11\\
3.185	-5.28127408562041e-11\\
3.18500000000003	-5.28127408561911e-11\\
3.18999999999997	-5.06245674515098e-11\\
3.18999999999999	-5.06245674514979e-11\\
3.19499999999993	-4.86167554250972e-11\\
3.19999999999986	-4.67843841510351e-11\\
3.19999999999999	-4.678438415099e-11\\
3.20000000000002	-4.67843841509801e-11\\
3.20999999999989	-4.35511899275112e-11\\
3.21000000000002	-4.35511899274727e-11\\
3.21000000000005	-4.35511899274642e-11\\
3.21899999999997	-4.10679448270629e-11\\
3.21899999999999	-4.10679448270556e-11\\
3.22	-4.08160401971188e-11\\
3.22000000000003	-4.08160401971117e-11\\
3.22100000000004	-4.05678091728686e-11\\
3.22200000000005	-4.03221926468718e-11\\
3.22400000000006	-3.98387070516099e-11\\
3.22800000000009	-3.89020675097027e-11\\
3.23000000000003	-3.84485463242212e-11\\
3.23000000000006	-3.84485463242149e-11\\
3.23800000000012	-3.67290137691329e-11\\
3.23999999999997	-3.63219259741442e-11\\
3.24	-3.63219259741384e-11\\
3.24799999999997	-3.47799667099646e-11\\
3.24799999999999	-3.47799667099594e-11\\
3.24999999999997	-3.44153191943706e-11\\
3.25	-3.44153191943655e-11\\
3.25199999999998	-3.40596910884091e-11\\
3.25399999999996	-3.37139161727704e-11\\
3.25499999999998	-3.35446811482818e-11\\
3.255	-3.3544681148277e-11\\
3.25899999999996	-3.28917063315147e-11\\
3.25999999999997	-3.27343739492348e-11\\
3.26	-3.27343739492304e-11\\
3.26399999999996	-3.21282270655415e-11\\
3.26799999999992	-3.15584675779899e-11\\
3.26999999999997	-3.12869516539733e-11\\
3.27	-3.12869516539695e-11\\
3.27699999999999	-3.03700649485645e-11\\
3.27700000000002	-3.03700649485608e-11\\
3.27999999999997	-2.99875051523858e-11\\
3.28	-2.99875051523822e-11\\
3.28299999999995	-2.96106903056755e-11\\
3.2859999999999	-2.92392879890677e-11\\
3.28999999999997	-2.8751938760565e-11\\
3.29	-2.87519387605616e-11\\
3.2959999999999	-2.80215938385712e-11\\
3.29999999999997	-2.75275469030488e-11\\
3.3	-2.75275469030452e-11\\
3.3059999999999	-2.67738136550483e-11\\
3.30599999999995	-2.67738136550426e-11\\
3.30599999999999	-2.67738136550369e-11\\
3.30999999999997	-2.62617336256493e-11\\
3.31	-2.62617336256456e-11\\
3.31399999999998	-2.57412339406962e-11\\
3.31799999999996	-2.52116867395691e-11\\
3.31999999999997	-2.49432613577354e-11\\
3.32	-2.49432613577315e-11\\
3.32499999999998	-2.42607017722203e-11\\
3.325	-2.42607017722164e-11\\
3.32999999999998	-2.35603760196132e-11\\
3.33000000000001	-2.35603760196091e-11\\
3.33499999999998	-2.284056776732e-11\\
3.33500000000001	-2.28405677673159e-11\\
3.33999999999998	-2.20995129522146e-11\\
3.34000000000001	-2.20995129522104e-11\\
3.34499999999998	-2.13561550583863e-11\\
3.34999999999996	-2.06294319255422e-11\\
3.35000000000001	-2.06294319255347e-11\\
3.35999999999996	-1.92188023084681e-11\\
3.36	-1.92188023084619e-11\\
3.36399999999997	-1.86711101522306e-11\\
3.36399999999999	-1.86711101522267e-11\\
3.36799999999996	-1.81359109882585e-11\\
3.36999999999997	-1.78727332124254e-11\\
3.37	-1.78727332124217e-11\\
3.37399999999997	-1.7354705339603e-11\\
3.37799999999993	-1.68471057882617e-11\\
3.37999999999997	-1.65969672329849e-11\\
3.38	-1.65969672329814e-11\\
3.38799999999993	-1.5624490524753e-11\\
3.38999999999997	-1.53882856596439e-11\\
3.39	-1.53882856596406e-11\\
3.39299999999997	-1.50387851933971e-11\\
3.39299999999999	-1.50387851933938e-11\\
3.39499999999998	-1.48088562328646e-11\\
3.395	-1.48088562328613e-11\\
3.39699999999999	-1.45812790349981e-11\\
3.39899999999997	-1.43559643777905e-11\\
3.39999999999997	-1.42441278631695e-11\\
3.4	-1.42441278631663e-11\\
3.40399999999997	-1.3801998741322e-11\\
3.40799999999993	-1.33676984782305e-11\\
3.40999999999997	-1.31532708498908e-11\\
3.41	-1.31532708498877e-11\\
3.41799999999993	-1.23468274070594e-11\\
3.41999999999997	-1.21598289578157e-11\\
3.42	-1.21598289578131e-11\\
3.42199999999997	-1.19763027025746e-11\\
3.42199999999999	-1.1976302702572e-11\\
3.42399999999996	-1.17939539581156e-11\\
3.42599999999992	-1.16127112234657e-11\\
3.42999999999985	-1.12532599636107e-11\\
3.42999999999997	-1.12532599635996e-11\\
3.43	-1.12532599635971e-11\\
3.43799999999986	-1.05445287171985e-11\\
3.43999999999997	-1.03690592521473e-11\\
3.44	-1.03690592521449e-11\\
3.44799999999986	-9.67197034212084e-12\\
3.44999999999997	-9.49855372689938e-12\\
3.45	-9.49855372689692e-12\\
3.45099999999996	-9.41237932611694e-12\\
3.45099999999999	-9.41237932611451e-12\\
3.45199999999999	-9.32715275430196e-12\\
3.45299999999998	-9.24286565436004e-12\\
3.45499999999996	-9.07707691730687e-12\\
3.45899999999993	-8.75641569149809e-12\\
3.45999999999997	-8.67847863817919e-12\\
3.46	-8.67847863817699e-12\\
3.465	-8.30178202135442e-12\\
3.46500000000003	-8.30178202135234e-12\\
3.46999999999997	-7.94606389936375e-12\\
3.47	-7.94606389936179e-12\\
3.47499999999994	-7.61045248817232e-12\\
3.47999999999988	-7.29412529135568e-12\\
3.47999999999994	-7.29412529135216e-12\\
3.47999999999999	-7.29412529134867e-12\\
3.48999999999987	-6.74676202698775e-12\\
3.48999999999996	-6.7467620269833e-12\\
3.48999999999999	-6.74676202698193e-12\\
3.49999999999987	-6.28108238255635e-12\\
3.49999999999996	-6.28108238255219e-12\\
3.49999999999999	-6.28108238255091e-12\\
3.50899999999999	-5.88697449801625e-12\\
3.50900000000002	-5.88697449801504e-12\\
3.50999999999999	-5.84450180558707e-12\\
3.51000000000002	-5.84450180558586e-12\\
3.51100000000001	-5.80227708211787e-12\\
3.512	-5.76029618630208e-12\\
3.51399999999999	-5.67704944286389e-12\\
3.51799999999996	-5.51330239871449e-12\\
3.51999999999997	-5.43273789594903e-12\\
3.52	-5.43273789594789e-12\\
3.52799999999994	-5.12675371196895e-12\\
3.52999999999997	-5.05469343172973e-12\\
3.53	-5.05469343172872e-12\\
3.53499999999998	-4.88199347124512e-12\\
3.535	-4.88199347124417e-12\\
3.53799999999997	-4.78334756684714e-12\\
3.53799999999999	-4.78334756684623e-12\\
3.53999999999997	-4.71960194109289e-12\\
3.54	-4.71960194109199e-12\\
3.54199999999998	-4.65726093066475e-12\\
3.54399999999996	-4.59611931052526e-12\\
3.54799999999992	-4.47733881935329e-12\\
3.54999999999997	-4.41965337672261e-12\\
3.55	-4.4196533767218e-12\\
3.55799999999992	-4.19975427755904e-12\\
3.56	-4.14738240323028e-12\\
3.56000000000003	-4.14738240322954e-12\\
3.56699999999996	-3.96970437885065e-12\\
3.56699999999999	-3.96970437884994e-12\\
3.56999999999998	-3.89579765989381e-12\\
3.57000000000001	-3.89579765989312e-12\\
3.57299999999999	-3.82314081415196e-12\\
3.57599999999997	-3.75166974502726e-12\\
3.57999999999997	-3.65811069832464e-12\\
3.58	-3.65811069832398e-12\\
3.58599999999997	-3.52077931836244e-12\\
3.58999999999997	-3.43083591917435e-12\\
3.59	-3.43083591917371e-12\\
3.59599999999997	-3.29798439624951e-12\\
3.596	-3.29798439624889e-12\\
3.6	-3.21058985795257e-12\\
3.60000000000003	-3.21058985795195e-12\\
3.60400000000003	-3.12397371909621e-12\\
3.60499999999998	-3.1024253659982e-12\\
3.605	-3.10242536599759e-12\\
3.60900000000001	-3.0165913162159e-12\\
3.61	-2.99521212858456e-12\\
3.61000000000003	-2.99521212858395e-12\\
3.61400000000003	-2.91157677322764e-12\\
3.61800000000004	-2.83150478164585e-12\\
3.61999999999997	-2.79276558775134e-12\\
3.62	-2.7927655877508e-12\\
3.62499999999999	-2.69957845574993e-12\\
3.62500000000002	-2.69957845574941e-12\\
3.62999999999998	-2.61143993886586e-12\\
3.63000000000001	-2.61143993886537e-12\\
3.63499999999997	-2.52770267319305e-12\\
3.63999999999993	-2.44773008393152e-12\\
3.63999999999997	-2.44773008393098e-12\\
3.64	-2.44773008393043e-12\\
3.64999999999993	-2.29830371275204e-12\\
3.64999999999997	-2.29830371275133e-12\\
3.65	-2.29830371275093e-12\\
3.65399999999996	-2.24186112647559e-12\\
3.65399999999999	-2.24186112647519e-12\\
3.65799999999995	-2.18670203552576e-12\\
3.65999999999997	-2.15957668875149e-12\\
3.66	-2.15957668875111e-12\\
3.66399999999996	-2.10618121610029e-12\\
3.66799999999993	-2.05385643619316e-12\\
3.66999999999997	-2.02806984741587e-12\\
3.67	-2.0280698474155e-12\\
3.67499999999998	-1.96461846743619e-12\\
3.67500000000001	-1.96461846743583e-12\\
3.67999999999998	-1.90249323602781e-12\\
3.68000000000001	-1.90249323602746e-12\\
3.68299999999996	-1.86621791613807e-12\\
3.68299999999999	-1.86621791613773e-12\\
3.68599999999995	-1.83118660230184e-12\\
3.68899999999991	-1.79736839142776e-12\\
3.68999999999998	-1.7863600461193e-12\\
3.69000000000001	-1.78636004611898e-12\\
3.69599999999992	-1.72301122731574e-12\\
3.69999999999997	-1.68327609679402e-12\\
3.7	-1.68327609679375e-12\\
3.70599999999992	-1.62653687151272e-12\\
3.71	-1.59021020007978e-12\\
3.71000000000003	-1.59021020007953e-12\\
3.71199999999996	-1.57247517091489e-12\\
3.71199999999999	-1.57247517091464e-12\\
3.71399999999993	-1.5550163721993e-12\\
3.71599999999986	-1.53782695885147e-12\\
3.71999999999973	-1.50422943470876e-12\\
3.71999999999997	-1.50422943470671e-12\\
3.72	-1.50422943470648e-12\\
3.72799999999974	-1.43894586721611e-12\\
3.72999999999997	-1.42287841822669e-12\\
3.73	-1.42287841822646e-12\\
3.73799999999974	-1.35943335690188e-12\\
3.73999999999997	-1.34374718242011e-12\\
3.74	-1.34374718241989e-12\\
3.74099999999996	-1.33592450997468e-12\\
3.74099999999999	-1.33592450997446e-12\\
3.74199999999998	-1.32811228419084e-12\\
3.74299999999998	-1.32030973904822e-12\\
3.74499999999996	-1.3047306326375e-12\\
3.74500000000001	-1.30473063263715e-12\\
3.74899999999997	-1.27365499372016e-12\\
3.74999999999997	-1.26589884427959e-12\\
3.75	-1.26589884427937e-12\\
3.75399999999997	-1.23490248038562e-12\\
3.75799999999994	-1.20391483933303e-12\\
3.75999999999997	-1.18840910311248e-12\\
3.76	-1.18840910311226e-12\\
3.76799999999994	-1.12616478981275e-12\\
3.76999999999996	-1.11051786509164e-12\\
3.76999999999999	-1.11051786509141e-12\\
3.77799999999993	-1.04995351630452e-12\\
3.77999999999996	-1.03544793405671e-12\\
3.77999999999999	-1.03544793405651e-12\\
3.78799999999993	-9.79796963995601e-13\\
3.78999999999996	-9.66449788002072e-13\\
3.78999999999999	-9.66449788001884e-13\\
3.79799999999993	-9.15150941002443e-13\\
3.79899999999996	-9.08962871059808e-13\\
3.79899999999999	-9.08962871059633e-13\\
3.79999999999996	-9.02822592760261e-13\\
3.79999999999999	-9.02822592760087e-13\\
3.80099999999998	-8.9672950456746e-13\\
3.80199999999997	-8.90683009016079e-13\\
3.80399999999995	-8.78727429867746e-13\\
3.80799999999991	-8.55349606863182e-13\\
3.80999999999996	-8.43918197008127e-13\\
3.80999999999999	-8.43918197007966e-13\\
3.81499999999998	-8.15975383097839e-13\\
3.81500000000001	-8.15975383097682e-13\\
3.81999999999999	-7.88840799030302e-13\\
3.82000000000002	-7.8884079903015e-13\\
3.825	-7.62447944208307e-13\\
3.82799999999996	-7.46941282714382e-13\\
3.82799999999999	-7.46941282714236e-13\\
3.82999999999999	-7.36732136444989e-13\\
3.83000000000002	-7.36732136444845e-13\\
3.83200000000002	-7.26621217321988e-13\\
3.83400000000002	-7.16604561263466e-13\\
3.83800000000001	-6.96838365587595e-13\\
3.83999999999997	-6.87081076029482e-13\\
3.84	-6.87081076029344e-13\\
3.84799999999999	-6.50160464939608e-13\\
3.84999999999997	-6.41524234131149e-13\\
3.85	-6.41524234131028e-13\\
3.85699999999996	-6.11815381982768e-13\\
3.85699999999999	-6.11815381982647e-13\\
3.85999999999997	-5.99131020047213e-13\\
3.86	-5.99131020047093e-13\\
3.86299999999998	-5.86459287561072e-13\\
3.86599999999997	-5.73789005770894e-13\\
3.86999999999997	-5.56878233748017e-13\\
3.87	-5.56878233747896e-13\\
3.87599999999997	-5.3142161779762e-13\\
3.87999999999997	-5.14351418679477e-13\\
3.88	-5.14351418679355e-13\\
3.88500000000001	-4.93409839818484e-13\\
3.88500000000003	-4.93409839818367e-13\\
3.88599999999996	-4.89329381498576e-13\\
3.88599999999999	-4.89329381498461e-13\\
3.88699999999998	-4.85283934389823e-13\\
3.88799999999997	-4.8127310196119e-13\\
3.88999999999995	-4.7335371208484e-13\\
3.89	-4.73353712084647e-13\\
3.89399999999996	-4.57913234540109e-13\\
3.89799999999992	-4.42983732659483e-13\\
3.9	-4.35703249727157e-13\\
3.90000000000003	-4.35703249727055e-13\\
3.90799999999995	-4.07923415634119e-13\\
3.91	-4.01311593417117e-13\\
3.91000000000003	-4.01311593417024e-13\\
3.91499999999996	-3.85336460384293e-13\\
3.91499999999999	-3.85336460384205e-13\\
3.91999999999993	-3.70122269912516e-13\\
3.92	-3.7012226991229e-13\\
3.92000000000003	-3.70122269912205e-13\\
3.92499999999997	-3.5563173550312e-13\\
3.9299999999999	-3.41829344619301e-13\\
3.93000000000003	-3.41829344618948e-13\\
3.93000000000006	-3.41829344618872e-13\\
3.93999999999993	-3.17320749675053e-13\\
3.93999999999998	-3.17320749674947e-13\\
3.94	-3.17320749674885e-13\\
3.94399999999996	-3.08551938701094e-13\\
3.94399999999999	-3.08551938701032e-13\\
3.94799999999995	-2.99930586884422e-13\\
3.94999999999998	-2.95670973821325e-13\\
3.95	-2.95670973821265e-13\\
3.95399999999996	-2.87245522843779e-13\\
3.95499999999998	-2.85157491221809e-13\\
3.95500000000001	-2.8515749122175e-13\\
3.95899999999997	-2.76872549279073e-13\\
3.96	-2.7481709491124e-13\\
3.96000000000003	-2.74817094911182e-13\\
3.96399999999999	-2.66652357761793e-13\\
3.96799999999995	-2.58569365010945e-13\\
3.97	-2.54554556514455e-13\\
3.97000000000003	-2.54554556514398e-13\\
3.97299999999996	-2.48636804435287e-13\\
3.97299999999999	-2.48636804435232e-13\\
3.97599999999992	-2.42900198771178e-13\\
3.97899999999986	-2.37339678609551e-13\\
3.97999999999997	-2.35524449763293e-13\\
3.98	-2.35524449763242e-13\\
3.98599999999987	-2.2502263670499e-13\\
3.98999999999997	-2.18379955024267e-13\\
3.99	-2.1837995502422e-13\\
3.99599999999987	-2.08927674346956e-13\\
3.99999999999997	-2.0295290216578e-13\\
4	-2.02952902165738e-13\\
4.00199999999993	-2.00085179746725e-13\\
4.00199999999999	-2.00085179746645e-13\\
4.00399999999992	-1.9732931051357e-13\\
4.00599999999986	-1.94684213965039e-13\\
4.00999999999972	-1.89722233890722e-13\\
4.00999999999995	-1.89722233890456e-13\\
4.01	-1.89722233890388e-13\\
4.01799999999973	-1.80373970631728e-13\\
4.01999999999995	-1.78077473266697e-13\\
4.02	-1.78077473266632e-13\\
4.02500000000001	-1.72397351036583e-13\\
4.02500000000006	-1.72397351036519e-13\\
4.02999999999995	-1.6679343343451e-13\\
4.03	-1.66793433434447e-13\\
4.03099999999999	-1.65680582965713e-13\\
4.03100000000005	-1.6568058296565e-13\\
4.03200000000004	-1.64570121774813e-13\\
4.03300000000003	-1.63461941015748e-13\\
4.035	-1.61251986552836e-13\\
4.03899999999996	-1.56854677492997e-13\\
4.03999999999994	-1.55759429715498e-13\\
4.04	-1.55759429715436e-13\\
4.04799999999991	-1.47297263977629e-13\\
4.04999999999994	-1.45270158350113e-13\\
4.05	-1.45270158350056e-13\\
4.05799999999991	-1.37491629793765e-13\\
4.05999999999994	-1.35625659186116e-13\\
4.06	-1.35625659186064e-13\\
};
\end{axis}
\end{tikzpicture}%}
      \caption{The angular displacement of pendulum $P_1$ as a function of time.
        \texttt{Blue}: $C_1 = 6$ ms, \texttt{Red}: $C_1 = 10$ ms}
      \label{fig:01.5.6_10.1}
    \end{figure}
  \end{minipage}
  \hfill
  \begin{minipage}{0.45\linewidth}
    \begin{figure}[H]\centering
      \scalebox{0.7}{% This file was created by matlab2tikz.
%
%The latest updates can be retrieved from
%  http://www.mathworks.com/matlabcentral/fileexchange/22022-matlab2tikz-matlab2tikz
%where you can also make suggestions and rate matlab2tikz.
%
\definecolor{mycolor1}{rgb}{0.00000,0.44700,0.74100}%
\definecolor{mycolor2}{rgb}{0.85000,0.32500,0.09800}%
%
\begin{tikzpicture}

\begin{axis}[%
width=4.133in,
height=3.26in,
at={(0.693in,0.44in)},
scale only axis,
xmin=0,
xmax=1,
xmajorgrids,
ymin=-0.1,
ymax=0.2,
ymajorgrids,
axis background/.style={fill=white}
]
\addplot [color=mycolor1,solid,forget plot]
  table[row sep=crcr]{%
0	0.15313\\
3.15544362088405e-30	0.15313\\
0.000656101980281985	0.153131614989962\\
0.00393661188169191	0.153188143215565\\
0.00599999999999994	0.153265080494076\\
0.006	0.153265080494076\\
0.012	0.153670560289007\\
0.0120000000000001	0.153670560289007\\
0.018	0.153071664853654\\
0.0180000000000001	0.153071664853654\\
0.0199999999999998	0.152365329512728\\
0.02	0.152365329512728\\
0.026	0.148724274488254\\
0.0260000000000002	0.148724274488254\\
0.0289999999999998	0.146046083995443\\
0.029	0.146046083995443\\
0.0319999999999996	0.142794627452511\\
0.0349999999999991	0.138968470912041\\
0.035	0.13896847091204\\
0.0399999999999996	0.131789348388764\\
0.04	0.131789348388763\\
0.0449999999999996	0.123959478342115\\
0.0459999999999996	0.122314584640995\\
0.046	0.122314584640994\\
0.047	0.120643199324198\\
0.0470000000000004	0.120643199324197\\
0.0490000000000003	0.117220624955326\\
0.0510000000000002	0.113691084393681\\
0.055	0.106308316385691\\
0.0579999999999996	0.100485154924159\\
0.058	0.100485154924158\\
0.0599999999999996	0.0964653994777478\\
0.06	0.0964653994777469\\
0.0619999999999995	0.092334609798713\\
0.0639999999999991	0.0880919761882657\\
0.0659999999999991	0.0837366670740184\\
0.066	0.0837366670740165\\
0.0699999999999991	0.0746845854767947\\
0.07	0.0746845854767927\\
0.0700000000000009	0.0746845854767906\\
0.074	0.0654820973483735\\
0.076	0.0609387524318376\\
0.0760000000000009	0.0609387524318356\\
0.08	0.0519634081808075\\
0.0800000000000009	0.0519634081808055\\
0.0839999999999999	0.0431306724709095\\
0.086	0.0387656144357706\\
0.0860000000000009	0.0387656144357686\\
0.0869999999999991	0.0365955374773486\\
0.087	0.0365955374773467\\
0.0880000000000004	0.0344336199800889\\
0.0890000000000009	0.03227975600762\\
0.0910000000000017	0.0279957668674395\\
0.0929999999999991	0.0237427303784554\\
0.093	0.0237427303784535\\
0.0970000000000017	0.0154879858559104\\
0.0999999999999991	0.00958444869935146\\
0.1	0.00958444869934975\\
0.104000000000002	0.00209147738315415\\
0.104999999999999	0.000285191714757269\\
0.105	0.000285191714755677\\
0.105999999999999	-0.00149449006990914\\
0.106	-0.00149449006991071\\
0.106999999999999	-0.00324765517394124\\
0.107999999999998	-0.00497438950448962\\
0.109999999999997	-0.00834890299548232\\
0.111999999999999	-0.0116186919605986\\
0.112	-0.0116186919606\\
0.115999999999997	-0.0178466394573827\\
0.115999999999998	-0.0178466394573853\\
0.116	-0.0178466394573879\\
0.119999999999997	-0.0236631150271634\\
0.119999999999998	-0.0236631150271658\\
0.12	-0.0236631150271682\\
0.123999999999997	-0.0290726790847185\\
0.125999999999999	-0.031626207097874\\
0.126	-0.0316262070978751\\
0.127999999999998	-0.03407957300888\\
0.128	-0.0340795730088821\\
0.129999999999998	-0.0364271843737179\\
0.131999999999996	-0.0386634280235806\\
0.135999999999993	-0.0428035436794524\\
0.139999999999998	-0.046503166088059\\
0.14	-0.0465031660880606\\
0.144999999999998	-0.0505126366966201\\
0.145	-0.0505126366966214\\
0.145999999999998	-0.0512329261213502\\
0.146	-0.0512329261213514\\
0.146999999999999	-0.0519260994409184\\
0.147999999999998	-0.0525921906227423\\
0.149999999999997	-0.0538432558021337\\
0.151999999999998	-0.0549863668732067\\
0.152	-0.0549863668732077\\
0.155999999999997	-0.0570007720155776\\
0.157999999999998	-0.0578852528495309\\
0.158	-0.0578852528495317\\
0.16	-0.0586881556806868\\
0.160000000000002	-0.0586881556806875\\
0.162000000000002	-0.0594096378799837\\
0.164000000000002	-0.0600498408609338\\
0.166	-0.0606088901058539\\
0.166000000000002	-0.0606088901058543\\
0.170000000000002	-0.0614839498018867\\
0.174	-0.0620355030603464\\
0.174000000000001	-0.0620355030603465\\
0.175	-0.0621228997391115\\
0.175000000000002	-0.0621228997391117\\
0.176000000000001	-0.0621901098124187\\
0.177	-0.0622371365737359\\
0.178999999999998	-0.0622706483888091\\
0.179999999999998	-0.0622571350847035\\
0.18	-0.0622571350847035\\
0.183999999999997	-0.0620804705828559\\
0.186	-0.0619304203383477\\
0.186000000000002	-0.0619304203383475\\
0.189999999999998	-0.0615067368047821\\
0.192	-0.061233020472304\\
0.192000000000002	-0.0612330204723037\\
0.195999999999998	-0.0605615704193962\\
0.199999999999995	-0.0597243116296848\\
0.199999999999997	-0.0597243116296842\\
0.2	-0.0597243116296836\\
0.202999999999998	-0.0589871655511613\\
0.203	-0.0589871655511608\\
0.205999999999998	-0.0581560603911522\\
0.206	-0.0581560603911517\\
0.208999999999998	-0.0572306296136106\\
0.209999999999998	-0.0569011221000455\\
0.21	-0.0569011221000449\\
0.211999999999998	-0.0562104650884198\\
0.212	-0.0562104650884192\\
0.213999999999998	-0.0554939446732612\\
0.215999999999997	-0.0547678599943136\\
0.217999999999998	-0.0540320687361973\\
0.218	-0.0540320687361967\\
0.219999999999998	-0.0532864266819164\\
0.22	-0.0532864266819157\\
0.221999999999998	-0.0525307876837086\\
0.223999999999996	-0.0517650036334558\\
0.225999999999998	-0.0509889244345494\\
0.226	-0.0509889244345487\\
0.229999999999996	-0.0494052700894414\\
0.231999999999998	-0.0485973845409472\\
0.232	-0.0485973845409465\\
0.235999999999996	-0.0469487049222538\\
0.237999999999998	-0.0461075877045604\\
0.238	-0.0461075877045596\\
0.239999999999998	-0.0452638608140846\\
0.24	-0.0452638608140839\\
0.241999999999998	-0.0444261532252999\\
0.243999999999996	-0.0435943007442838\\
0.245	-0.0431805191730594\\
0.245000000000002	-0.0431805191730586\\
0.245999999999998	-0.042768140324533\\
0.246	-0.0427681403245322\\
0.246999999999999	-0.0423571439924296\\
0.247999999999998	-0.0419475100374713\\
0.249999999999997	-0.0411322490361785\\
0.252	-0.0403221975263589\\
0.252000000000003	-0.0403221975263574\\
0.256	-0.0387170888829923\\
0.259999999999997	-0.0371309256194401\\
0.26	-0.0371309256194387\\
0.260999999999996	-0.0367371989641445\\
0.261	-0.0367371989641431\\
0.261999999999998	-0.0363445593118938\\
0.262999999999996	-0.0359529874189366\\
0.264999999999993	-0.0351729702138744\\
0.265999999999997	-0.0347844866804151\\
0.266	-0.0347844866804137\\
0.269999999999993	-0.0332402760675451\\
0.271999999999997	-0.032473740201349\\
0.272	-0.0324737402013477\\
0.275999999999993	-0.0309768515277924\\
0.279999999999986	-0.0295444303379096\\
0.279999999999993	-0.0295444303379072\\
0.28	-0.0295444303379047\\
0.285999999999996	-0.0275142319192832\\
0.286	-0.0275142319192821\\
0.289999999999996	-0.0262381743224773\\
0.29	-0.0262381743224762\\
0.293999999999996	-0.0250228688460333\\
0.295999999999997	-0.0244376985423952\\
0.296	-0.0244376985423942\\
0.297999999999997	-0.023865912771368\\
0.298	-0.023865912771367\\
0.299999999999997	-0.0233059496077691\\
0.3	-0.0233059496077682\\
0.301999999999997	-0.0227576992966336\\
0.303999999999993	-0.0222210543791462\\
0.305999999999997	-0.0216959096708285\\
0.306	-0.0216959096708276\\
0.309999999999993	-0.0206797113972297\\
0.313999999999986	-0.0197083077239614\\
0.314999999999997	-0.0194723659426118\\
0.315	-0.019472365942611\\
0.318999999999997	-0.0185558885254103\\
0.319	-0.0185558885254095\\
0.319999999999996	-0.0183335354663476\\
0.32	-0.0183335354663468\\
0.320999999999998	-0.0181138669354534\\
0.321999999999996	-0.0178968721687158\\
0.323999999999993	-0.0174708615270537\\
0.325999999999996	-0.0170554200416272\\
0.326	-0.0170554200416265\\
0.329999999999993	-0.0162559208851037\\
0.331	-0.0160625270729795\\
0.331000000000004	-0.0160625270729788\\
0.333	-0.0156829269477488\\
0.333000000000004	-0.0156829269477482\\
0.335	-0.0153124997481444\\
0.336999999999996	-0.0149511728689865\\
0.339999999999996	-0.0144260912138601\\
0.34	-0.0144260912138595\\
0.343999999999993	-0.0137571872671349\\
0.345999999999997	-0.0134359608475538\\
0.346	-0.0134359608475532\\
0.347999999999997	-0.0131234669407986\\
0.348	-0.0131234669407981\\
0.349999999999997	-0.0128196442972705\\
0.35	-0.01281964429727\\
0.351999999999997	-0.0125244333665395\\
0.353999999999993	-0.0122377762863185\\
0.354	-0.0122377762863175\\
0.357999999999993	-0.0116869790778428\\
0.359999999999996	-0.0114220006465959\\
0.36	-0.0114220006465954\\
0.363999999999993	-0.0109126244375425\\
0.365999999999996	-0.0106681268200104\\
0.366	-0.01066812682001\\
0.369999999999993	-0.0101992729386525\\
0.373999999999986	-0.00975697004641091\\
0.376999999999997	-0.00944245756938025\\
0.377	-0.00944245756937989\\
0.379999999999997	-0.00914254582941242\\
0.38	-0.00914254582941208\\
0.382999999999996	-0.00885710256496239\\
0.384999999999997	-0.00867478266822029\\
0.385	-0.00867478266821997\\
0.385999999999997	-0.00858600188607707\\
0.386	-0.00858600188607676\\
0.386999999999998	-0.00849880138853904\\
0.387999999999996	-0.0084131769027382\\
0.388999999999997	-0.00832912423305362\\
0.389	-0.00832912423305333\\
0.390999999999997	-0.00816487980833682\\
0.392999999999993	-0.00800519778581065\\
0.394999999999997	-0.00785004686718034\\
0.395	-0.00785004686718007\\
0.398999999999993	-0.00755321758399373\\
0.399999999999997	-0.00748179575418615\\
0.4	-0.0074817957541859\\
0.403999999999993	-0.00720714540431904\\
0.405999999999997	-0.00707639549649106\\
0.406	-0.00707639549649083\\
0.409999999999993	-0.00682791807339543\\
0.411999999999997	-0.00671014185557849\\
0.412	-0.00671014185557829\\
0.415999999999993	-0.00648439614930216\\
0.419999999999986	-0.00626957769808391\\
0.419999999999996	-0.00626957769808335\\
0.42	-0.00626957769808316\\
0.426	-0.00596747249584989\\
0.426000000000004	-0.00596747249584972\\
0.432000000000004	-0.00568904363910817\\
0.432000000000007	-0.00568904363910801\\
0.434999999999997	-0.00555855272545515\\
0.435	-0.005558552725455\\
0.43799999999999	-0.00543379990561904\\
0.439999999999997	-0.00535379191588978\\
0.44	-0.00535379191588964\\
0.44299999999999	-0.00523848156133828\\
0.445999999999979	-0.0051287681435457\\
0.445999999999995	-0.00512876814354515\\
0.446	-0.00512876814354496\\
0.447	-0.00509343233425499\\
0.447000000000004	-0.00509343233425487\\
0.448000000000004	-0.00505858406985652\\
0.449000000000004	-0.00502409443325188\\
0.451000000000004	-0.00495618430091737\\
0.454999999999997	-0.00482459437707987\\
0.455	-0.00482459437707975\\
0.459	-0.00469855938950006\\
0.459999999999997	-0.00466790692656755\\
0.46	-0.00466790692656744\\
0.463999999999997	-0.00454867714962721\\
0.464	-0.0045486771496271\\
0.465999999999997	-0.00449106991554863\\
0.466	-0.00449106991554853\\
0.467999999999996	-0.00443478597566644\\
0.469999999999993	-0.00437981429810855\\
0.471999999999997	-0.00432614410821348\\
0.472	-0.00432614410821338\\
0.473	-0.0042997379678403\\
0.473000000000004	-0.00429973796784021\\
0.474000000000004	-0.00427354165958215\\
0.475000000000004	-0.00424755389979012\\
0.477000000000004	-0.00419619894213898\\
0.479999999999997	-0.00412069911279833\\
0.48	-0.00412069911279824\\
0.484	-0.00402284195400596\\
0.485999999999997	-0.00397509572992924\\
0.486	-0.00397509572992915\\
0.489999999999997	-0.00388192119864539\\
0.49	-0.00388192119864531\\
0.492999999999997	-0.00381403259657121\\
0.493	-0.00381403259657113\\
0.495999999999997	-0.00374781834107435\\
0.498999999999993	-0.00368324923058202\\
0.499	-0.00368324923058187\\
0.499999999999997	-0.00366204063867869\\
0.5	-0.00366204063867862\\
0.500999999999998	-0.00364091823100526\\
0.501999999999997	-0.00361988097247265\\
0.503999999999993	-0.00357805778363198\\
0.505999999999993	-0.00353656287465056\\
0.506	-0.00353656287465042\\
0.507999999999993	-0.00349538811242098\\
0.508	-0.00349538811242084\\
0.509999999999993	-0.00345452542658466\\
0.511999999999986	-0.00341396680789743\\
0.515999999999972	-0.00333373003153357\\
0.519999999999993	-0.00325461487274754\\
0.52	-0.0032546148727474\\
0.521999999999993	-0.00321545848261091\\
0.522	-0.00321545848261077\\
0.523999999999993	-0.00317655930231894\\
0.524999999999993	-0.00315720378110358\\
0.525	-0.00315720378110344\\
0.525999999999993	-0.00313790970692414\\
0.526	-0.00313790970692401\\
0.526999999999998	-0.00311867613436781\\
0.527999999999997	-0.00309950212098513\\
0.529999999999993	-0.00306132901649075\\
0.531999999999993	-0.00302338291136547\\
0.532	-0.00302338291136533\\
0.535999999999993	-0.00294848395299221\\
0.538	-0.00291160190528123\\
0.538000000000007	-0.0029116019052811\\
0.539999999999993	-0.00287508848201567\\
0.54	-0.00287508848201555\\
0.541999999999986	-0.00283893652647114\\
0.543999999999972	-0.00280313895272453\\
0.545999999999993	-0.00276768874431281\\
0.546	-0.00276768874431269\\
0.549999999999972	-0.00269780269691938\\
0.550999999999993	-0.00268053751206524\\
0.551	-0.00268053751206511\\
0.554999999999972	-0.00261227669971175\\
0.556999999999993	-0.00257861455645162\\
0.557	-0.0025786145564515\\
0.559999999999993	-0.00252860911213904\\
0.56	-0.00252860911213893\\
0.562999999999993	-0.0024791062239712\\
0.565999999999986	-0.00243008406036295\\
0.565999999999993	-0.00243008406036284\\
0.566	-0.00243008406036272\\
0.571999999999986	-0.00233339563102444\\
0.571999999999993	-0.00233339563102433\\
0.572	-0.00233339563102421\\
0.577999999999986	-0.00223837324065878\\
0.579999999999993	-0.00220704028678049\\
0.58	-0.00220704028678038\\
0.585999999999986	-0.00211397896728288\\
0.585999999999993	-0.00211397896728277\\
0.586	-0.00211397896728266\\
0.591999999999986	-0.00202219657886334\\
0.591999999999993	-0.00202219657886323\\
0.592	-0.00202219657886312\\
0.594999999999993	-0.00197709382754081\\
0.595	-0.0019770938275407\\
0.597999999999993	-0.00193296945867172\\
0.599999999999993	-0.0019040871313196\\
0.6	-0.0019040871313195\\
0.602999999999993	-0.00186155036532712\\
0.605999999999986	-0.00181994102509126\\
0.606	-0.00181994102509107\\
0.606999999999993	-0.00180627415626971\\
0.607	-0.00180627415626962\\
0.607999999999999	-0.00179270762290004\\
0.608999999999997	-0.00177924076016245\\
0.609000000000004	-0.00177924076016235\\
0.611	-0.0017526034119188\\
0.612999999999997	-0.00172635689053332\\
0.614999999999997	-0.0017004960515865\\
0.615000000000004	-0.00170049605158641\\
0.618999999999997	-0.00164966916666219\\
0.619999999999993	-0.00163712001132626\\
0.62	-0.00163712001132617\\
0.623999999999993	-0.00158753520550816\\
0.625999999999993	-0.00156310139838846\\
0.626	-0.00156310139838837\\
0.629999999999993	-0.00151492702933171\\
0.63	-0.00151492702933163\\
0.633999999999993	-0.00146764558624051\\
0.635999999999993	-0.00144432810065704\\
0.636	-0.00144432810065695\\
0.637999999999993	-0.00142121999801122\\
0.638	-0.00142121999801114\\
0.639999999999993	-0.00139831674905593\\
0.64	-0.00139831674905585\\
0.641999999999993	-0.00137561386466581\\
0.643999999999986	-0.00135310689500291\\
0.645999999999993	-0.00133079142861429\\
0.646	-0.00133079142861422\\
0.649999999999986	-0.00128671754680051\\
0.65	-0.00128671754680036\\
0.650000000000007	-0.00128671754680028\\
0.653999999999993	-0.00124352608126631\\
0.657999999999979	-0.00120135158592858\\
0.659999999999993	-0.00118063533791182\\
0.66	-0.00118063533791175\\
0.664999999999993	-0.00112989427952784\\
0.665	-0.00112989427952777\\
0.665999999999993	-0.00111992200875428\\
0.666	-0.00111992200875421\\
0.666999999999998	-0.00111000723812208\\
0.667000000000006	-0.00111000723812201\\
0.668000000000004	-0.00110014948178418\\
0.669000000000002	-0.0010903482567086\\
0.670999999999998	-0.00107091348204279\\
0.673000000000005	-0.00105169910483379\\
0.673000000000013	-0.00105169910483372\\
0.677000000000005	-0.00101407582451287\\
0.678	-0.00100485171107809\\
0.678000000000007	-0.00100485171107802\\
0.679999999999993	-0.000986618355865464\\
0.68	-0.0009866183558654\\
0.681999999999986	-0.000968668523877228\\
0.683999999999972	-0.000950998696876605\\
0.686	-0.000933605411508941\\
0.686000000000007	-0.000933605411508879\\
0.689999999999979	-0.000899634882709027\\
0.69399999999995	-0.000866730302830468\\
0.695999999999993	-0.000850669649253334\\
0.696	-0.000850669649253277\\
0.699999999999993	-0.000819315875150975\\
0.7	-0.00081931587515092\\
0.703999999999993	-0.000788965076111516\\
0.705999999999993	-0.000774158325460929\\
0.706	-0.000774158325460877\\
0.707999999999993	-0.000759593455176628\\
0.708	-0.000759593455176577\\
0.709999999999993	-0.000745281733509318\\
0.711999999999986	-0.000731234478308297\\
0.713999999999993	-0.000717448936257202\\
0.714	-0.000717448936257153\\
0.717999999999986	-0.000690652234377396\\
0.719999999999993	-0.000677635822290894\\
0.72	-0.000677635822290849\\
0.723999999999986	-0.000652354119068287\\
0.724999999999993	-0.000646188365736588\\
0.725	-0.000646188365736545\\
0.725999999999993	-0.000640083872616837\\
0.726	-0.000640083872616794\\
0.726999999999999	-0.000634040340593207\\
0.727999999999997	-0.000628057473525747\\
0.729999999999993	-0.000616272564570708\\
0.731999999999993	-0.000604726835864012\\
0.732	-0.000604726835863972\\
0.734999999999993	-0.000587851930833831\\
0.735	-0.000587851930833791\\
0.737999999999993	-0.000571502983697621\\
0.74	-0.000560892314416705\\
0.740000000000007	-0.000560892314416668\\
0.743	-0.000545404054311648\\
0.745999999999993	-0.000530423031895794\\
0.746000000000007	-0.000530423031895725\\
0.746999999999993	-0.000525540928906708\\
0.747	-0.000525540928906674\\
0.747999999999999	-0.000520714210543487\\
0.748999999999997	-0.000515942640275208\\
0.750999999999993	-0.000506564011482599\\
0.753999999999993	-0.000492903921019896\\
0.754	-0.000492903921019864\\
0.757999999999993	-0.000475442476622148\\
0.759999999999993	-0.000467030182100566\\
0.76	-0.000467030182100537\\
0.763999999999993	-0.00045074052161536\\
0.766	-0.000442836543384021\\
0.766000000000007	-0.000442836543383993\\
0.77	-0.000427502544636809\\
0.770000000000007	-0.000427502544636782\\
0.774	-0.000412790522106836\\
0.776	-0.000405664128438062\\
0.776000000000007	-0.000405664128438037\\
0.779999999999993	-0.00039186359205892\\
0.78	-0.000391863592058896\\
0.782999999999993	-0.000381903598373912\\
0.783	-0.000381903598373889\\
0.785999999999993	-0.000372273346438775\\
0.786000000000001	-0.000372273346438752\\
0.788999999999994	-0.000362968589154366\\
0.791999999999987	-0.000353985222971942\\
0.792	-0.000353985222971901\\
0.792000000000008	-0.00035398522297188\\
0.797999999999994	-0.000336911043278193\\
0.799999999999993	-0.000331471725798996\\
0.8	-0.000331471725798977\\
0.804999999999993	-0.000318414712141896\\
0.805000000000001	-0.000318414712141878\\
0.805999999999993	-0.0003158950735722\\
0.806	-0.000315895073572183\\
0.806999999999994	-0.000313405731496958\\
0.807999999999987	-0.000310946563932357\\
0.809999999999973	-0.000306118271808874\\
0.811999999999993	-0.000301409250837741\\
0.812	-0.000301409250837724\\
0.815999999999973	-0.000292345353633015\\
0.817999999999993	-0.00028798870084009\\
0.818000000000001	-0.000287988700840075\\
0.819999999999993	-0.000283723658083559\\
0.82	-0.000283723658083544\\
0.821999999999993	-0.000279525281738471\\
0.823999999999986	-0.000275392748906758\\
0.825999999999993	-0.000271325249596005\\
0.826	-0.000271325249595991\\
0.829999999999986	-0.000263382175170138\\
0.833999999999972	-0.000255689830680818\\
0.839999999999993	-0.0002446082918977\\
0.84	-0.000244608291897687\\
0.840999999999993	-0.000242813514315119\\
0.841000000000001	-0.000242813514315106\\
0.841999999999994	-0.000241033398545753\\
0.842999999999987	-0.000239267857343117\\
0.844999999999973	-0.000235780153300426\\
0.845999999999993	-0.000234057819560533\\
0.846	-0.000234057819560521\\
0.849999999999973	-0.000227309978716701\\
0.851999999999993	-0.000224019799905309\\
0.852	-0.000224019799905297\\
0.855999999999973	-0.000217612731891902\\
0.857999999999993	-0.000214496844502343\\
0.858	-0.000214496844502332\\
0.86	-0.000211438570089865\\
0.860000000000007	-0.000211438570089854\\
0.862000000000007	-0.000208437309224881\\
0.864000000000007	-0.000205492473648578\\
0.866	-0.000202603486161819\\
0.866000000000007	-0.000202603486161809\\
0.87	-0.000196990801298907\\
0.870000000000007	-0.000196990801298898\\
0.874	-0.000191594854001679\\
0.874999999999994	-0.000190279232949771\\
0.875000000000001	-0.000190279232949762\\
0.876	-0.000188976828297866\\
0.876000000000007	-0.000188976828297856\\
0.877000000000007	-0.000187684412002462\\
0.878000000000006	-0.000186398756507051\\
0.879999999999998	-0.000183847476254879\\
0.880000000000006	-0.00018384747625487\\
0.882000000000005	-0.000181322487482216\\
0.884000000000004	-0.000178823295283206\\
0.886000000000005	-0.000176349409806542\\
0.886000000000013	-0.000176349409806533\\
0.888000000000007	-0.000173900346164411\\
0.888000000000014	-0.000173900346164402\\
0.890000000000009	-0.000171475624334087\\
0.892000000000004	-0.000169074769060716\\
0.895999999999993	-0.000164342780469229\\
0.898999999999993	-0.000160852971891248\\
0.899000000000001	-0.00016085297189124\\
0.899999999999993	-0.000159700670206876\\
0.9	-0.000159700670206867\\
0.900999999999994	-0.000158553758126382\\
0.901999999999987	-0.000157412179438012\\
0.903999999999973	-0.000155144798745878\\
0.905999999999993	-0.000152898083715084\\
0.906	-0.000152898083715076\\
0.909999999999973	-0.000148464893155957\\
0.909999999999987	-0.000148464893155942\\
0.910000000000001	-0.000148464893155927\\
0.910999999999993	-0.000147368828266066\\
0.911000000000001	-0.000147368828266059\\
0.911999999999994	-0.00014627788771615\\
0.912999999999987	-0.000145192357059511\\
0.914999999999973	-0.000143037312923647\\
0.916999999999993	-0.000140903273461481\\
0.917000000000001	-0.000140903273461474\\
0.919999999999993	-0.00013774068415686\\
0.92	-0.000137740684156853\\
0.922999999999993	-0.000134623020437661\\
0.925999999999986	-0.000131548907363069\\
0.925999999999993	-0.000131548907363062\\
0.926	-0.000131548907363055\\
0.927999999999993	-0.000129523023169475\\
0.928000000000001	-0.000129523023169468\\
0.929999999999994	-0.000127515494816387\\
0.931999999999987	-0.000125525928812399\\
0.933999999999994	-0.000123553935194896\\
0.934000000000001	-0.000123553935194889\\
0.937999999999987	-0.000119676323222335\\
0.939999999999993	-0.000117773744853171\\
0.940000000000001	-0.000117773744853165\\
0.943999999999987	-0.000114039178723854\\
0.944999999999994	-0.000113119976282754\\
0.945000000000001	-0.000113119976282747\\
0.945999999999993	-0.000112206458973796\\
0.946000000000001	-0.000112206458973789\\
0.946999999999994	-0.000111298582035277\\
0.947999999999987	-0.000110396300980206\\
0.949999999999973	-0.000108608349944615\\
0.952	-0.000106842255421689\\
0.952000000000008	-0.000106842255421683\\
0.95599999999998	-0.000103374255495249\\
0.956999999999994	-0.000102520380047639\\
0.957000000000001	-0.000102520380047633\\
0.96	-9.99895821087766e-05\\
0.960000000000008	-9.99895821087707e-05\\
0.963000000000007	-9.75041613143291e-05\\
0.966000000000007	-9.5063021552972e-05\\
0.966000000000014	-9.50630215529663e-05\\
0.969000000000007	-9.26650862343915e-05\\
0.969000000000014	-9.26650862343859e-05\\
0.972000000000007	-9.03149750231814e-05\\
0.975	-8.80173286505207e-05\\
0.979999999999994	-8.4301764671992e-05\\
0.980000000000001	-8.43017646719868e-05\\
0.985999999999987	-8.00253586065973e-05\\
0.986000000000001	-8.00253586065876e-05\\
0.991999999999987	-7.59406814815088e-05\\
0.992000000000001	-7.59406814814995e-05\\
0.997999999999987	-7.20486910896441e-05\\
0.998000000000001	-7.20486910896352e-05\\
0.999999999999993	-7.07947977279235e-05\\
1	-7.07947977279191e-05\\
1.00199999999999	-6.95622096195735e-05\\
1.00399999999999	-6.83506851673719e-05\\
1.00599999999999	-6.71599869078441e-05\\
1.006	-6.71599869078357e-05\\
1.00999999999999	-6.48401394844127e-05\\
1.01399999999997	-6.26008484692461e-05\\
1.01499999999999	-6.20534055317706e-05\\
1.015	-6.20534055317629e-05\\
1.01999999999999	-5.93891314832457e-05\\
1.02	-5.93891314832383e-05\\
1.02499999999999	-5.68440533010911e-05\\
1.02599999999999	-5.6349064251296e-05\\
1.026	-5.6349064251289e-05\\
1.02699999999999	-5.5858693604283e-05\\
1.027	-5.58586936042761e-05\\
1.02799999999999	-5.53728531834919e-05\\
1.02899999999999	-5.48914550322382e-05\\
1.03099999999997	-5.39418914004163e-05\\
1.03299999999999	-5.30098165906625e-05\\
1.033	-5.30098165906559e-05\\
1.03699999999997	-5.11974060740385e-05\\
1.04	-4.98826721685848e-05\\
1.04000000000001	-4.98826721685787e-05\\
1.04399999999999	-4.81880232058742e-05\\
1.044	-4.81880232058683e-05\\
1.046	-4.73653225225227e-05\\
1.04600000000001	-4.73653225225169e-05\\
1.04800000000001	-4.65588216338842e-05\\
1.05	-4.57683624626513e-05\\
1.05000000000001	-4.57683624626457e-05\\
1.05200000000001	-4.49937900762732e-05\\
1.05200000000002	-4.49937900762678e-05\\
1.05400000000002	-4.42348729009988e-05\\
1.05600000000002	-4.3491382431542e-05\\
1.05800000000001	-4.27631729408992e-05\\
1.05800000000002	-4.27631729408941e-05\\
1.05999999999999	-4.20501016981892e-05\\
1.06	-4.20501016981842e-05\\
1.06199999999996	-4.13520289396309e-05\\
1.06399999999992	-4.06688178402438e-05\\
1.06599999999999	-4.00003344879761e-05\\
1.066	-4.00003344879714e-05\\
1.06999999999992	-3.87070297887418e-05\\
1.07299999999999	-3.77747578454953e-05\\
1.073	-3.77747578454909e-05\\
1.07699999999992	-3.65812223238851e-05\\
1.07899999999999	-3.60054022074721e-05\\
1.079	-3.60054022074681e-05\\
1.07999999999999	-3.57221447570039e-05\\
1.08	-3.57221447569999e-05\\
1.08099999999999	-3.54412578530156e-05\\
1.08199999999999	-3.51627277317338e-05\\
1.08399999999997	-3.46126833599653e-05\\
1.08499999999999	-3.4341142156937e-05\\
1.085	-3.43411421569332e-05\\
1.08599999999999	-3.40719038307299e-05\\
1.086	-3.40719038307261e-05\\
1.08699999999999	-3.38049551888542e-05\\
1.08799999999999	-3.35402831505268e-05\\
1.08999999999997	-3.30177171194932e-05\\
1.09199999999999	-3.25041033126607e-05\\
1.092	-3.25041033126571e-05\\
1.09599999999997	-3.15033314284924e-05\\
1.09999999999995	-3.05371828412418e-05\\
1.09999999999997	-3.05371828412353e-05\\
1.1	-3.05371828412287e-05\\
1.10199999999999	-3.00668545108411e-05\\
1.102	-3.00668545108378e-05\\
1.10399999999999	-2.9604900033819e-05\\
1.10599999999997	-2.91512288582763e-05\\
1.10599999999999	-2.9151228858273e-05\\
1.106	-2.91512288582698e-05\\
1.10999999999997	-2.82683823337324e-05\\
1.11199999999999	-2.78390339436935e-05\\
1.112	-2.78390339436904e-05\\
1.11599999999997	-2.70045342993367e-05\\
1.11999999999994	-2.62020670142234e-05\\
1.12	-2.62020670142121e-05\\
1.12000000000001	-2.62020670142093e-05\\
1.126	-2.505705718079e-05\\
1.12600000000001	-2.50570571807874e-05\\
1.13099999999999	-2.41553998898001e-05\\
1.131	-2.41553998897976e-05\\
1.13599999999997	-2.33003044539298e-05\\
1.13699999999999	-2.31347796109702e-05\\
1.137	-2.31347796109679e-05\\
1.138	-2.29706780724201e-05\\
1.13800000000001	-2.29706780724178e-05\\
1.13900000000001	-2.28076028209267e-05\\
1.14	-2.26455458654222e-05\\
1.14000000000001	-2.26455458654199e-05\\
1.14100000000001	-2.2484499265074e-05\\
1.14200000000001	-2.23244551285734e-05\\
1.14400000000001	-2.20073429270659e-05\\
1.146	-2.16941471056864e-05\\
1.14600000000001	-2.16941471056842e-05\\
1.15000000000001	-2.10792598103952e-05\\
1.15400000000001	-2.04793111393561e-05\\
1.15499999999999	-2.0331602754049e-05\\
1.155	-2.03316027540469e-05\\
1.15999999999999	-1.96063721793395e-05\\
1.16	-1.96063721793375e-05\\
1.16499999999999	-1.89026795963069e-05\\
1.16599999999999	-1.87644493195945e-05\\
1.166	-1.87644493195926e-05\\
1.17099999999999	-1.80854665808059e-05\\
1.17199999999999	-1.79520571834687e-05\\
1.172	-1.79520571834668e-05\\
1.173	-1.78196115072008e-05\\
1.17300000000001	-1.7819611507199e-05\\
1.17400000000001	-1.76883064102212e-05\\
1.17500000000001	-1.75581354582677e-05\\
1.17700000000001	-1.73011705310835e-05\\
1.17999999999999	-1.69240715573422e-05\\
1.18	-1.69240715573404e-05\\
1.184	-1.6436599074425e-05\\
1.18599999999999	-1.61993234522913e-05\\
1.186	-1.61993234522896e-05\\
1.18899999999999	-1.5851354695347e-05\\
1.189	-1.58513546953454e-05\\
1.18999999999999	-1.57374609041885e-05\\
1.19	-1.57374609041868e-05\\
1.19099999999999	-1.56246056662662e-05\\
1.19199999999999	-1.55127834516205e-05\\
1.19399999999997	-1.52922162252599e-05\\
1.19499999999999	-1.51834604056034e-05\\
1.195	-1.51834604056019e-05\\
1.19899999999997	-1.47547700599766e-05\\
1.19999999999999	-1.4648921426791e-05\\
1.2	-1.46489214267895e-05\\
1.20399999999997	-1.42306671027404e-05\\
1.20599999999999	-1.40245524215676e-05\\
1.206	-1.40245524215661e-05\\
1.20699999999999	-1.39222304626545e-05\\
1.207	-1.39222304626531e-05\\
1.20799999999999	-1.38203920652582e-05\\
1.20899999999999	-1.37190322391849e-05\\
1.21099999999997	-1.35177284576024e-05\\
1.21499999999995	-1.31206467589369e-05\\
1.21799999999999	-1.28275175591633e-05\\
1.218	-1.2827517559162e-05\\
1.21999999999999	-1.26342556732136e-05\\
1.22	-1.26342556732123e-05\\
1.22199999999999	-1.24426756123699e-05\\
1.22399999999997	-1.22527398234283e-05\\
1.22499999999999	-1.2158376884686e-05\\
1.225	-1.21583768846847e-05\\
1.22599999999999	-1.20644110781907e-05\\
1.226	-1.20644110781894e-05\\
1.22699999999999	-1.19708377996829e-05\\
1.22799999999999	-1.18776524639688e-05\\
1.22999999999997	-1.16924273753032e-05\\
1.23	-1.16924273753007e-05\\
1.23399999999997	-1.13273556843818e-05\\
1.236	-1.11476682242628e-05\\
1.23600000000001	-1.11476682242615e-05\\
1.23999999999999	-1.07938139717061e-05\\
1.24	-1.07938139717049e-05\\
1.24399999999997	-1.04470900073049e-05\\
1.24599999999999	-1.02763167176051e-05\\
1.246	-1.02763167176038e-05\\
1.247	-1.01915625468847e-05\\
1.24700000000001	-1.01915625468835e-05\\
1.24800000000001	-1.01072244815172e-05\\
1.24900000000001	-1.00232983888436e-05\\
1.25100000000001	-9.85666569197846e-06\\
1.253	-9.69163179563864e-06\\
1.25300000000001	-9.69163179563747e-06\\
1.25700000000001	-9.3676450052173e-06\\
1.25999999999999	-9.13046943887784e-06\\
1.26	-9.13046943887673e-06\\
1.264	-8.82178918179261e-06\\
1.266	-8.67061822590467e-06\\
1.26600000000001	-8.6706182259036e-06\\
1.27000000000001	-8.37446650407949e-06\\
1.272	-8.22942769125908e-06\\
1.27200000000001	-8.22942769125805e-06\\
1.276	-7.94528201625836e-06\\
1.27600000000001	-7.94528201625736e-06\\
1.27999999999999	-7.66886059805219e-06\\
1.28000000000001	-7.66886059805122e-06\\
1.28399999999999	-7.39994670808966e-06\\
1.28599999999999	-7.26823897252217e-06\\
1.28600000000001	-7.26823897252124e-06\\
1.288	-7.13832950393264e-06\\
1.28800000000001	-7.13832950393173e-06\\
1.29	-7.01037836194254e-06\\
1.29199999999999	-6.8845459898406e-06\\
1.29499999999999	-6.69971628604371e-06\\
1.295	-6.69971628604284e-06\\
1.29899999999998	-6.4604656111749e-06\\
1.29999999999999	-6.4019170575923e-06\\
1.3	-6.40191705759147e-06\\
1.30399999999998	-6.17269323372248e-06\\
1.30499999999999	-6.11661583531408e-06\\
1.305	-6.11661583531329e-06\\
1.30599999999999	-6.06102432834729e-06\\
1.306	-6.06102432834651e-06\\
1.30699999999999	-6.00591598887775e-06\\
1.30799999999999	-5.95128811653486e-06\\
1.30999999999997	-5.84346308953184e-06\\
1.31199999999999	-5.7375281132148e-06\\
1.312	-5.73752811321405e-06\\
1.31299999999999	-5.68528172104826e-06\\
1.313	-5.68528172104752e-06\\
1.31399999999999	-5.63353774481781e-06\\
1.31499999999999	-5.58229364901029e-06\\
1.31699999999997	-5.48129507914927e-06\\
1.32	-5.333484342734e-06\\
1.32000000000001	-5.33348434273331e-06\\
1.32399999999999	-5.14318508225873e-06\\
1.326	-5.05089983583614e-06\\
1.32600000000001	-5.05089983583549e-06\\
1.32999999999999	-4.87196766631394e-06\\
1.33	-4.87196766631331e-06\\
1.33399999999997	-4.70043696349007e-06\\
1.334	-4.70043696348899e-06\\
1.33799999999997	-4.5361732385068e-06\\
1.34	-4.45672602563384e-06\\
1.34000000000001	-4.45672602563328e-06\\
1.34399999999999	-4.30251287421099e-06\\
1.346	-4.22756417611948e-06\\
1.34600000000001	-4.22756417611896e-06\\
1.348	-4.15403438256845e-06\\
1.34800000000002	-4.15403438256793e-06\\
1.35	-4.08190908153086e-06\\
1.35199999999999	-4.0111741361671e-06\\
1.35599999999997	-3.87382012515426e-06\\
1.35999999999999	-3.74186465546198e-06\\
1.36	-3.74186465546152e-06\\
1.36299999999999	-3.64637845226643e-06\\
1.363	-3.64637845226598e-06\\
1.36499999999999	-3.5843548015264e-06\\
1.365	-3.58435480152596e-06\\
1.36599999999999	-3.55382853857513e-06\\
1.366	-3.5538285385747e-06\\
1.36699999999999	-3.52362398423309e-06\\
1.36799999999999	-3.4937396584704e-06\\
1.36999999999997	-3.43492585094958e-06\\
1.37199999999999	-3.3773755882781e-06\\
1.372	-3.3773755882777e-06\\
1.37599999999997	-3.26595316195702e-06\\
1.378	-3.21204224464791e-06\\
1.37800000000001	-3.21204224464753e-06\\
1.37999999999999	-3.15931735747209e-06\\
1.38	-3.15931735747172e-06\\
1.38199999999997	-3.10776816621658e-06\\
1.38399999999994	-3.05738456703884e-06\\
1.38599999999999	-3.00815668455678e-06\\
1.386	-3.00815668455643e-06\\
1.38999999999994	-2.91312969919946e-06\\
1.39199999999999	-2.86731197066199e-06\\
1.392	-2.86731197066167e-06\\
1.39599999999994	-2.77902313813485e-06\\
1.39799999999999	-2.73653472918714e-06\\
1.398	-2.73653472918684e-06\\
1.39999999999999	-2.69492703274025e-06\\
1.4	-2.69492703273996e-06\\
1.40199999999999	-2.6539797769404e-06\\
1.40399999999997	-2.61368493596654e-06\\
1.40599999999999	-2.57403461187361e-06\\
1.406	-2.57403461187333e-06\\
1.40999999999997	-2.49663655290851e-06\\
1.412	-2.45887364771557e-06\\
1.41200000000001	-2.4588736477153e-06\\
1.41599999999999	-2.38518307631144e-06\\
1.41999999999996	-2.31389154170394e-06\\
1.41999999999998	-2.31389154170356e-06\\
1.42	-2.31389154170319e-06\\
1.42099999999999	-2.2964369163176e-06\\
1.421	-2.29643691631735e-06\\
1.42199999999999	-2.27912787316759e-06\\
1.42299999999999	-2.26196356390797e-06\\
1.42499999999997	-2.22806578989156e-06\\
1.42599999999999	-2.21133066412117e-06\\
1.426	-2.21133066412093e-06\\
1.42999999999997	-2.14579618269136e-06\\
1.43199999999999	-2.11386133961809e-06\\
1.432	-2.11386133961787e-06\\
1.43499999999999	-2.06704218373801e-06\\
1.435	-2.06704218373779e-06\\
1.43799999999998	-2.02155265098667e-06\\
1.43999999999999	-1.99195504449106e-06\\
1.44	-1.99195504449085e-06\\
1.44299999999999	-1.94863725445236e-06\\
1.44599999999997	-1.9065968690407e-06\\
1.44599999999999	-1.90659686904048e-06\\
1.446	-1.90659686904027e-06\\
1.44699999999999	-1.89286405427437e-06\\
1.447	-1.89286405427417e-06\\
1.44799999999999	-1.87927043587937e-06\\
1.44899999999999	-1.86581534770759e-06\\
1.44999999999999	-1.8524981304562e-06\\
1.45	-1.85249813045601e-06\\
1.45199999999999	-1.82627470525601e-06\\
1.45399999999997	-1.80059502040327e-06\\
1.45599999999999	-1.77545404258007e-06\\
1.456	-1.77545404257989e-06\\
1.45999999999997	-1.72626273934275e-06\\
1.46	-1.72626273934241e-06\\
1.46000000000001	-1.72626273934223e-06\\
1.46399999999999	-1.67815636448379e-06\\
1.466	-1.65449820593373e-06\\
1.46600000000001	-1.65449820593357e-06\\
1.46999999999999	-1.60794876037299e-06\\
1.47	-1.60794876037282e-06\\
1.47399999999997	-1.56239147897107e-06\\
1.47599999999999	-1.53997370793431e-06\\
1.476	-1.53997370793415e-06\\
1.47899999999998	-1.50678576380868e-06\\
1.479	-1.50678576380852e-06\\
1.47999999999999	-1.49583793460443e-06\\
1.48	-1.49583793460428e-06\\
1.48099999999999	-1.4849466184123e-06\\
1.48199999999999	-1.4741112815527e-06\\
1.48399999999997	-1.45260642481291e-06\\
1.48599999999999	-1.43131914934989e-06\\
1.486	-1.43131914934974e-06\\
1.48999999999997	-1.38938069465798e-06\\
1.491	-1.37902560040018e-06\\
1.49100000000001	-1.37902560040003e-06\\
1.49499999999999	-1.33813803681195e-06\\
1.49899999999996	-1.2980910088061e-06\\
1.49999999999999	-1.28820689169125e-06\\
1.5	-1.28820689169111e-06\\
1.50499999999999	-1.23952795957562e-06\\
1.505	-1.23952795957548e-06\\
1.50599999999999	-1.22993716448838e-06\\
1.506	-1.22993716448824e-06\\
1.50699999999999	-1.2203935978135e-06\\
1.50799999999999	-1.21089679189402e-06\\
1.508	-1.21089679189388e-06\\
1.50999999999999	-1.19204160320765e-06\\
1.51199999999997	-1.17336790275205e-06\\
1.51399999999999	-1.15487203040965e-06\\
1.514	-1.15487203040952e-06\\
1.51799999999997	-1.11852615852831e-06\\
1.518	-1.11852615852807e-06\\
1.51999999999999	-1.10070074731822e-06\\
1.52	-1.1007007473181e-06\\
1.52199999999999	-1.08310234577188e-06\\
1.52399999999997	-1.06572750453432e-06\\
1.52599999999999	-1.04857281806936e-06\\
1.526	-1.04857281806924e-06\\
1.52999999999997	-1.01491050251353e-06\\
1.53399999999994	-9.82089006112928e-07\\
1.537	-9.58009166376067e-07\\
1.53700000000001	-9.58009166375954e-07\\
1.53999999999999	-9.34377179497663e-07\\
1.54	-9.34377179497552e-07\\
1.54299999999997	-9.11182623717409e-07\\
1.54599999999994	-8.88415269517888e-07\\
1.54599999999997	-8.8841526951768e-07\\
1.546	-8.88415269517469e-07\\
1.549	-8.66065076050485e-07\\
1.54900000000001	-8.6606507605038e-07\\
1.55200000000001	-8.44169805071282e-07\\
1.55500000000001	-8.2276741873344e-07\\
1.55500000000003	-8.2276741873334e-07\\
1.55999999999999	-7.88166534418296e-07\\
1.56	-7.88166534418199e-07\\
1.56499999999996	-7.5486625256255e-07\\
1.56599999999998	-7.48358650859211e-07\\
1.566	-7.48358650859118e-07\\
1.57099999999996	-7.16565376563523e-07\\
1.57199999999998	-7.10353487514413e-07\\
1.572	-7.10353487514325e-07\\
1.57499999999999	-6.92022111635327e-07\\
1.575	-6.92022111635241e-07\\
1.57799999999999	-6.74149041057575e-07\\
1.57999999999999	-6.6248437400273e-07\\
1.58	-6.62484374002648e-07\\
1.58299999999999	-6.45357727541423e-07\\
1.58599999999997	-6.28668806610422e-07\\
1.586	-6.2866880661027e-07\\
1.58699999999999	-6.23201827672381e-07\\
1.587	-6.23201827672304e-07\\
1.58799999999999	-6.17782397695095e-07\\
1.58899999999999	-6.12410251102481e-07\\
1.59099999999997	-6.01806757430007e-07\\
1.59499999999995	-5.81155741959902e-07\\
1.59499999999997	-5.81155741959766e-07\\
1.595	-5.81155741959629e-07\\
1.59999999999999	-5.5636368732e-07\\
1.6	-5.56363687319931e-07\\
1.60499999999999	-5.32678587721969e-07\\
1.60599999999999	-5.28071825399014e-07\\
1.606	-5.28071825398949e-07\\
1.60699999999998	-5.23507950624845e-07\\
1.607	-5.23507950624781e-07\\
1.60799999999999	-5.18986046486511e-07\\
1.60899999999999	-5.14505198108044e-07\\
1.60999999999998	-5.10065185926705e-07\\
1.61	-5.10065185926642e-07\\
1.61199999999999	-5.01306801900782e-07\\
1.61399999999997	-4.92709177737343e-07\\
1.61599999999999	-4.84270628268706e-07\\
1.616	-4.84270628268647e-07\\
1.61999999999997	-4.67864168369017e-07\\
1.61999999999999	-4.67864168368959e-07\\
1.62	-4.67864168368901e-07\\
1.62399999999997	-4.52074558690693e-07\\
1.624	-4.52074558690597e-07\\
1.62599999999999	-4.44407185347908e-07\\
1.626	-4.44407185347854e-07\\
1.62799999999999	-4.36889419365828e-07\\
1.62999999999998	-4.29519787233023e-07\\
1.63199999999999	-4.22296844473021e-07\\
1.632	-4.22296844472971e-07\\
1.63599999999998	-4.08282867872059e-07\\
1.63999999999995	-3.94833977063589e-07\\
1.63999999999998	-3.9483397706351e-07\\
1.64	-3.94833977063431e-07\\
1.645	-3.7880149278047e-07\\
1.64500000000001	-3.78801492780426e-07\\
1.64599999999999	-3.75697229345322e-07\\
1.646	-3.75697229345278e-07\\
1.64699999999999	-3.72626684913019e-07\\
1.64799999999999	-3.6958970901955e-07\\
1.64999999999997	-3.63615869236855e-07\\
1.65199999999999	-3.57774539101445e-07\\
1.652	-3.57774539101404e-07\\
1.65299999999998	-3.54903206356646e-07\\
1.653	-3.54903206356605e-07\\
1.65399999999999	-3.5206457369935e-07\\
1.65499999999997	-3.49258502033321e-07\\
1.65699999999995	-3.43743493271411e-07\\
1.65899999999998	-3.38357099107699e-07\\
1.659	-3.38357099107662e-07\\
1.65999999999999	-3.35706895155976e-07\\
1.66	-3.35706895155938e-07\\
1.66099999999999	-3.33078637780421e-07\\
1.66199999999998	-3.30472198193459e-07\\
1.66399999999995	-3.25324262583774e-07\\
1.66599999999999	-3.20262079311505e-07\\
1.666	-3.20262079311469e-07\\
1.66999999999995	-3.10391017593686e-07\\
1.67399999999991	-3.00851273618371e-07\\
1.67999999999998	-2.8714658623336e-07\\
1.68	-2.87146586233329e-07\\
1.68199999999998	-2.8273607409928e-07\\
1.682	-2.82736074099249e-07\\
1.68399999999998	-2.78402966953575e-07\\
1.68599999999997	-2.74146415424678e-07\\
1.68599999999999	-2.74146415424646e-07\\
1.686	-2.74146415424616e-07\\
1.68999999999997	-2.65859656867926e-07\\
1.69199999999999	-2.6182782560612e-07\\
1.692	-2.61827825606092e-07\\
1.69599999999997	-2.53989063491912e-07\\
1.69799999999999	-2.50182035085901e-07\\
1.698	-2.50182035085874e-07\\
1.69999999999999	-2.46448908652388e-07\\
1.7	-2.46448908652361e-07\\
1.70199999999999	-2.4278895248907e-07\\
1.70399999999998	-2.39201449230298e-07\\
1.70599999999999	-2.35685695711517e-07\\
1.706	-2.35685695711493e-07\\
1.70999999999998	-2.28866695438398e-07\\
1.71099999999998	-2.27205729073886e-07\\
1.711	-2.27205729073862e-07\\
1.71499999999997	-2.20734556705414e-07\\
1.715	-2.20734556705374e-07\\
1.71699999999998	-2.17601479500977e-07\\
1.717	-2.17601479500955e-07\\
1.71899999999998	-2.14521553882354e-07\\
1.71999999999999	-2.12995947288766e-07\\
1.72	-2.12995947288744e-07\\
1.72199999999999	-2.09973072754497e-07\\
1.72399999999997	-2.06987490116208e-07\\
1.72599999999999	-2.04038614188135e-07\\
1.726	-2.04038614188114e-07\\
1.72899999999998	-1.99682862971758e-07\\
1.729	-1.99682862971738e-07\\
1.73199999999998	-1.9540648208593e-07\\
1.73499999999997	-1.91207585576567e-07\\
1.73999999999998	-1.84376597821633e-07\\
1.74	-1.84376597821614e-07\\
1.74599999999997	-1.76444984375117e-07\\
1.74599999999998	-1.76444984375096e-07\\
1.746	-1.76444984375074e-07\\
1.74999999999998	-1.7131170700702e-07\\
1.75	-1.71311707007002e-07\\
1.75199999999998	-1.68789889596657e-07\\
1.752	-1.68789889596639e-07\\
1.75399999999998	-1.66304323092843e-07\\
1.75599999999997	-1.63861551359575e-07\\
1.75799999999998	-1.61461095604116e-07\\
1.758	-1.61461095604099e-07\\
1.75999999999999	-1.59102485331164e-07\\
1.76	-1.59102485331148e-07\\
1.76199999999999	-1.56785258247086e-07\\
1.76399999999998	-1.54508960166368e-07\\
1.76599999999999	-1.52273144925767e-07\\
1.766	-1.52273144925751e-07\\
1.76899999999998	-1.48994370921664e-07\\
1.769	-1.48994370921648e-07\\
1.77199999999998	-1.45804253148329e-07\\
1.77499999999996	-1.42701384711502e-07\\
1.77499999999998	-1.42701384711485e-07\\
1.775	-1.42701384711467e-07\\
1.78	-1.37665426175532e-07\\
1.78000000000002	-1.37665426175517e-07\\
1.78499999999998	-1.3275247359739e-07\\
1.785	-1.32752473597376e-07\\
1.78600000000001	-1.31784111555373e-07\\
1.78600000000003	-1.31784111555359e-07\\
1.78700000000005	-1.308203810859e-07\\
1.78800000000006	-1.29861234963934e-07\\
1.79000000000009	-1.27956507991403e-07\\
1.79200000000003	-1.26069557304039e-07\\
1.79200000000004	-1.26069557304025e-07\\
1.79600000000011	-1.22347508808765e-07\\
1.79799999999998	-1.20511681465223e-07\\
1.798	-1.2051168146521e-07\\
1.8	-1.1869217119805e-07\\
1.80000000000002	-1.18692171198038e-07\\
1.80200000000002	-1.16888621378719e-07\\
1.80400000000002	-1.15100678504502e-07\\
1.806	-1.1332799213172e-07\\
1.80600000000002	-1.13327992131707e-07\\
1.80999999999998	-1.09827002009612e-07\\
1.81	-1.09827002009599e-07\\
1.81399999999997	-1.06391543839114e-07\\
1.81799999999994	-1.0302756182855e-07\\
1.82	-1.01371547996724e-07\\
1.82000000000001	-1.01371547996712e-07\\
1.826	-9.65035300846211e-08\\
1.82600000000001	-9.65035300846097e-08\\
1.827	-9.570636256889e-08\\
1.82700000000001	-9.57063625688787e-08\\
1.828	-9.4913138466805e-08\\
1.82899999999999	-9.41238189069571e-08\\
1.83099999999996	-9.25567388961018e-08\\
1.833	-9.10048153825517e-08\\
1.83300000000001	-9.10048153825407e-08\\
1.83699999999996	-8.79584212989152e-08\\
1.838	-8.72099487175452e-08\\
1.83800000000001	-8.72099487175346e-08\\
1.83999999999999	-8.57284970573063e-08\\
1.84	-8.57284970572959e-08\\
1.84199999999998	-8.4267461847356e-08\\
1.84399999999995	-8.28265567191319e-08\\
1.84599999999999	-8.14054992496207e-08\\
1.846	-8.14054992496107e-08\\
1.84999999999995	-7.8621817000051e-08\\
1.8539999999999	-7.59142325481688e-08\\
1.85499999999998	-7.52489766459447e-08\\
1.855	-7.52489766459353e-08\\
1.85599999999998	-7.45883112950838e-08\\
1.856	-7.45883112950744e-08\\
1.85699999999999	-7.39322041301608e-08\\
1.85799999999997	-7.32806229943321e-08\\
1.85999999999995	-7.19909113197179e-08\\
1.85999999999998	-7.19909113197016e-08\\
1.86	-7.19909113196851e-08\\
1.86399999999995	-6.9464410172314e-08\\
1.86599999999999	-6.82271254963768e-08\\
1.866	-6.8227125496368e-08\\
1.86799999999998	-6.70068269449261e-08\\
1.868	-6.70068269449175e-08\\
1.86999999999998	-6.58049953999871e-08\\
1.87199999999996	-6.46231153615943e-08\\
1.87299999999998	-6.40395846139472e-08\\
1.873	-6.40395846139389e-08\\
1.87699999999996	-6.17541923890512e-08\\
1.87999999999999	-6.00905349373025e-08\\
1.88	-6.00905349372948e-08\\
1.88399999999997	-5.79381123220036e-08\\
1.88499999999998	-5.74115778577305e-08\\
1.885	-5.7411577857723e-08\\
1.88599999999999	-5.68896199298675e-08\\
1.886	-5.68896199298601e-08\\
1.88699999999999	-5.63722129628449e-08\\
1.88799999999998	-5.58593316031364e-08\\
1.88999999999996	-5.48470454010066e-08\\
1.88999999999998	-5.48470454009962e-08\\
1.89	-5.48470454009858e-08\\
1.89199999999999	-5.38525629114445e-08\\
1.892	-5.38525629114375e-08\\
1.89399999999999	-5.28763777946159e-08\\
1.89599999999998	-5.19189872949678e-08\\
1.89799999999999	-5.09802037602512e-08\\
1.898	-5.09802037602446e-08\\
1.89999999999999	-5.00598431865559e-08\\
1.9	-5.00598431865495e-08\\
1.90199999999999	-4.91577251808642e-08\\
1.90399999999997	-4.82736729245355e-08\\
1.90599999999999	-4.74075131398962e-08\\
1.906	-4.74075131398902e-08\\
1.90999999999997	-4.57281953820331e-08\\
1.91399999999994	-4.41184552355144e-08\\
1.91399999999997	-4.41184552355037e-08\\
1.914	-4.4118455235493e-08\\
1.91999999999998	-4.18315568172844e-08\\
1.92	-4.18315568172792e-08\\
1.92499999999998	-4.00312896122251e-08\\
1.925	-4.00312896122201e-08\\
1.92599999999998	-3.9681347303424e-08\\
1.926	-3.96813473034191e-08\\
1.92699999999999	-3.93347349788759e-08\\
1.92799999999997	-3.89914356537493e-08\\
1.92999999999995	-3.83147088763167e-08\\
1.93199999999998	-3.76510343300422e-08\\
1.932	-3.76510343300375e-08\\
1.93599999999995	-3.63623241367501e-08\\
1.9399999999999	-3.51242946585884e-08\\
1.93999999999999	-3.51242946585623e-08\\
1.94	-3.5124294658558e-08\\
1.94299999999998	-3.42284463599272e-08\\
1.943	-3.4228446359923e-08\\
1.94599999999998	-3.33601640875806e-08\\
1.946	-3.33601640875746e-08\\
1.94899999999998	-3.25190649014224e-08\\
1.95199999999997	-3.17047778630876e-08\\
1.95199999999998	-3.1704777863083e-08\\
1.952	-3.17047778630782e-08\\
1.95799999999997	-3.01537344030671e-08\\
1.95999999999998	-2.96590999767079e-08\\
1.96	-2.96590999767044e-08\\
1.96599999999996	-2.82409794610751e-08\\
1.96599999999998	-2.82409794610711e-08\\
1.966	-2.82409794610671e-08\\
1.97199999999996	-2.69196096332222e-08\\
1.97199999999998	-2.69196096332185e-08\\
1.972	-2.69196096332148e-08\\
1.97799999999996	-2.56926592541164e-08\\
1.97799999999998	-2.56926592541129e-08\\
1.978	-2.56926592541095e-08\\
1.98	-2.53022877167275e-08\\
1.98000000000002	-2.53022877167247e-08\\
1.98200000000002	-2.49181062335062e-08\\
1.98400000000002	-2.45400395017882e-08\\
1.986	-2.41680134190381e-08\\
1.98600000000002	-2.41680134190354e-08\\
1.99000000000002	-2.34417926981472e-08\\
1.99400000000003	-2.27388746763394e-08\\
1.995	-2.25667212356414e-08\\
1.99500000000001	-2.2566721235639e-08\\
1.99999999999999	-2.17269895028307e-08\\
2	-2.17269895028283e-08\\
2.00099999999997	-2.15631926451752e-08\\
2.001	-2.15631926451706e-08\\
2.00199999999999	-2.14007600655816e-08\\
2.00299999999997	-2.12396838044754e-08\\
2.00499999999995	-2.09215687327087e-08\\
2.00599999999997	-2.07645143341947e-08\\
2.006	-2.07645143341902e-08\\
2.00999999999995	-2.01494721786688e-08\\
2.01199999999997	-1.98497511074405e-08\\
2.012	-1.98497511074363e-08\\
2.01299999999998	-1.97018778116177e-08\\
2.01300000000001	-1.97018778116135e-08\\
2.014	-1.95554038006506e-08\\
2.01499999999999	-1.94103218971492e-08\\
2.01699999999996	-1.91243060442274e-08\\
2.01999999999997	-1.8705547543061e-08\\
2.02	-1.87055475430571e-08\\
2.02399999999995	-1.81660788469888e-08\\
2.02599999999997	-1.79043149499689e-08\\
2.026	-1.79043149499652e-08\\
2.02999999999995	-1.73964715158294e-08\\
2.03	-1.73964715158239e-08\\
2.03399999999995	-1.69092104518742e-08\\
2.03599999999997	-1.66731782996926e-08\\
2.036	-1.66731782996893e-08\\
2.03999999999995	-1.62113389343112e-08\\
2.04	-1.62113389343052e-08\\
2.04399999999995	-1.57596717406334e-08\\
2.04599999999997	-1.55375417384341e-08\\
2.046	-1.5537541738431e-08\\
2.04799999999997	-1.53178225881549e-08\\
2.048	-1.53178225881518e-08\\
2.04999999999996	-1.5100471224279e-08\\
2.05199999999993	-1.48854450451002e-08\\
2.05599999999987	-1.44622001050315e-08\\
2.05899999999997	-1.41505597784908e-08\\
2.059	-1.41505597784879e-08\\
2.05999999999997	-1.40477559166967e-08\\
2.06	-1.40477559166937e-08\\
2.06099999999999	-1.39454817618426e-08\\
2.06199999999998	-1.38437323013294e-08\\
2.06399999999995	-1.3641787545807e-08\\
2.06499999999997	-1.35415823553696e-08\\
2.065	-1.35415823553667e-08\\
2.06599999999997	-1.34418820681208e-08\\
2.066	-1.3441882068118e-08\\
2.06699999999999	-1.33426817988148e-08\\
2.06799999999998	-1.32439766865307e-08\\
2.06999999999995	-1.30480326107648e-08\\
2.07099999999997	-1.29507840458923e-08\\
2.071	-1.29507840458895e-08\\
2.07499999999995	-1.25667888675737e-08\\
2.07699999999997	-1.23777679812538e-08\\
2.077	-1.23777679812511e-08\\
2.07999999999997	-1.20978531934828e-08\\
2.08	-1.20978531934801e-08\\
2.08299999999998	-1.18221686661598e-08\\
2.08599999999995	-1.15505928178903e-08\\
2.086	-1.1550592817886e-08\\
2.08799999999997	-1.13717656326952e-08\\
2.088	-1.13717656326927e-08\\
2.08999999999997	-1.11946762101953e-08\\
2.09199999999993	-1.10192898394639e-08\\
2.09399999999997	-1.08455721440888e-08\\
2.094	-1.08455721440863e-08\\
2.09799999999993	-1.05042011474244e-08\\
2.09999999999997	-1.03367794824425e-08\\
2.1	-1.03367794824401e-08\\
2.10399999999993	-1.00082997373021e-08\\
2.10599999999997	-9.84717727383331e-09\\
2.106	-9.84717727383104e-09\\
2.10999999999993	-9.53100925676381e-09\\
2.112	-9.37590173299745e-09\\
2.11200000000003	-9.37590173299526e-09\\
2.11599999999996	-9.0714876416268e-09\\
2.11699999999997	-8.99657086389484e-09\\
2.117	-8.99657086389271e-09\\
2.11999999999997	-8.7746098902634e-09\\
2.12	-8.77460989026132e-09\\
2.12299999999998	-8.55675705892501e-09\\
2.12599999999995	-8.34291629321021e-09\\
2.126	-8.3429162932068e-09\\
2.12899999999997	-8.13299328515979e-09\\
2.129	-8.13299328515782e-09\\
2.13199999999996	-7.92734331050279e-09\\
2.13499999999993	-7.72632352728159e-09\\
2.13499999999997	-7.72632352727935e-09\\
2.135	-7.72632352727712e-09\\
2.13999999999997	-7.40133999134691e-09\\
2.14	-7.4013399913451e-09\\
2.14499999999998	-7.08857451889077e-09\\
2.14599999999997	-7.02745361420819e-09\\
2.146	-7.02745361420646e-09\\
2.15099999999997	-6.72884530835135e-09\\
2.15199999999997	-6.67050240601952e-09\\
2.152	-6.67050240601787e-09\\
2.15299999999997	-6.61262934049074e-09\\
2.15299999999999	-6.6126293404891e-09\\
2.15399999999998	-6.55524024944396e-09\\
2.15499999999997	-6.49833232067931e-09\\
2.15699999999995	-6.38594881943226e-09\\
2.15999999999997	-6.22091333110968e-09\\
2.16	-6.22091333110813e-09\\
2.16399999999995	-6.00736083516805e-09\\
2.16599999999997	-5.90332089117342e-09\\
2.166	-5.90332089117196e-09\\
2.16999999999995	-5.70061164661789e-09\\
2.17	-5.70061164661556e-09\\
2.17399999999995	-5.50493139749837e-09\\
2.17499999999997	-5.45709151705487e-09\\
2.175	-5.45709151705351e-09\\
2.17899999999995	-5.26998239498095e-09\\
2.17999999999997	-5.22425625387275e-09\\
2.18	-5.22425625387146e-09\\
2.18399999999995	-5.04548902676957e-09\\
2.18599999999997	-4.95855722717335e-09\\
2.186	-4.95855722717213e-09\\
2.187	-4.91569677406088e-09\\
2.18700000000002	-4.91569677405967e-09\\
2.18800000000002	-4.87323061084445e-09\\
2.18800000000005	-4.87323061084325e-09\\
2.18900000000004	-4.83115012216696e-09\\
2.19000000000004	-4.78945324607722e-09\\
2.19200000000003	-4.70720217773331e-09\\
2.19600000000001	-4.54721474042109e-09\\
2.2	-4.3931428588351e-09\\
2.20000000000003	-4.39314285883403e-09\\
2.20399999999997	-4.24486573457977e-09\\
2.204	-4.24486573457874e-09\\
2.20499999999997	-4.20868816236106e-09\\
2.205	-4.20868816236003e-09\\
2.20599999999999	-4.17286370200183e-09\\
2.20600000000003	-4.17286370200033e-09\\
2.20700000000002	-4.13739059806238e-09\\
2.20800000000001	-4.1022671123555e-09\\
2.20999999999998	-4.03306212845063e-09\\
2.21200000000003	-3.96523518583935e-09\\
2.21200000000006	-3.9652351858384e-09\\
2.21600000000001	-3.83363827357977e-09\\
2.21800000000003	-3.76983645035347e-09\\
2.21800000000006	-3.76983645035257e-09\\
2.21999999999997	-3.70734895525977e-09\\
2.22	-3.70734895525889e-09\\
2.22199999999992	-3.64616354058833e-09\\
2.22399999999983	-3.58626821375995e-09\\
2.226	-3.5276512350512e-09\\
2.22600000000003	-3.52765123505038e-09\\
2.22999999999986	-3.41420661407492e-09\\
2.23299999999997	-3.33239516206389e-09\\
2.233	-3.33239516206313e-09\\
2.23699999999983	-3.22760731251082e-09\\
2.23899999999997	-3.1770305369125e-09\\
2.239	-3.17703053691179e-09\\
2.24	-3.15214595769945e-09\\
2.24000000000003	-3.15214595769874e-09\\
2.24100000000003	-3.12746748377689e-09\\
2.24200000000003	-3.10299390587212e-09\\
2.24400000000004	-3.05465665125412e-09\\
2.246	-3.00712471955739e-09\\
2.24600000000003	-3.00712471955672e-09\\
2.25000000000004	-2.91443971545888e-09\\
2.252	-2.86926847643928e-09\\
2.25200000000003	-2.86926847643864e-09\\
2.25600000000004	-2.78122425430432e-09\\
2.25999999999997	-2.69618709682006e-09\\
2.26	-2.69618709681947e-09\\
2.26199999999997	-2.65477524117238e-09\\
2.262	-2.65477524117179e-09\\
2.26399999999996	-2.61409033046825e-09\\
2.26599999999993	-2.57412439030736e-09\\
2.26599999999997	-2.57412439030666e-09\\
2.266	-2.57412439030596e-09\\
2.26999999999994	-2.49631822718897e-09\\
2.272	-2.45846275394815e-09\\
2.27200000000003	-2.45846275394762e-09\\
2.27499999999997	-2.40299836655627e-09\\
2.275	-2.40299836655575e-09\\
2.27799999999994	-2.34911905792272e-09\\
2.27999999999997	-2.31406835553695e-09\\
2.28	-2.31406835553646e-09\\
2.28299999999994	-2.26277838160786e-09\\
2.28599999999989	-2.21301164714088e-09\\
2.28599999999997	-2.21301164713945e-09\\
2.286	-2.21301164713898e-09\\
2.28699999999997	-2.19675742060556e-09\\
2.287	-2.1967574206051e-09\\
2.28799999999999	-2.18066920240034e-09\\
2.28899999999997	-2.16474620412977e-09\\
2.29099999999995	-2.13339275452833e-09\\
2.291	-2.13339275452761e-09\\
2.29499999999995	-2.07263470216078e-09\\
2.29699999999997	-2.04321819060171e-09\\
2.297	-2.04321819060129e-09\\
2.29999999999997	-1.99997676668582e-09\\
2.3	-1.99997676668541e-09\\
2.30299999999998	-1.95753560360859e-09\\
2.30599999999995	-1.91587598412145e-09\\
2.306	-1.91587598412081e-09\\
2.30999999999997	-1.86151385276234e-09\\
2.31	-1.86151385276196e-09\\
2.31399999999997	-1.80846588606633e-09\\
2.31599999999997	-1.78242167746164e-09\\
2.316	-1.78242167746127e-09\\
2.31999999999997	-1.73126728506668e-09\\
2.32	-1.73126728506632e-09\\
2.32399999999997	-1.68132493726019e-09\\
2.32599999999997	-1.65679600710987e-09\\
2.326	-1.65679600710953e-09\\
2.32999999999997	-1.60859859469415e-09\\
2.33199999999997	-1.58492066554941e-09\\
2.332	-1.58492066554908e-09\\
2.33599999999997	-1.53864721584874e-09\\
2.33999999999994	-1.49396290685296e-09\\
2.34	-1.49396290685225e-09\\
2.34000000000003	-1.49396290685194e-09\\
2.34499999999997	-1.44028899244637e-09\\
2.345	-1.44028899244607e-09\\
2.34600000000003	-1.42983977710715e-09\\
2.34600000000006	-1.42983977710685e-09\\
2.34700000000009	-1.41948454541527e-09\\
2.34800000000012	-1.40922278993983e-09\\
2.34899999999997	-1.39905400785384e-09\\
2.349	-1.39905400785355e-09\\
2.35100000000006	-1.37899337528491e-09\\
2.35300000000011	-1.35929871576225e-09\\
2.35499999999997	-1.33996616905825e-09\\
2.355	-1.33996616905798e-09\\
2.35800000000003	-1.31145318556236e-09\\
2.35800000000006	-1.3114531855621e-09\\
2.35999999999997	-1.29268083131363e-09\\
2.36	-1.29268083131337e-09\\
2.36199999999992	-1.27409322013992e-09\\
2.36399999999983	-1.25568670880826e-09\\
2.36599999999997	-1.23745768956927e-09\\
2.366	-1.23745768956901e-09\\
2.36999999999983	-1.20151786974349e-09\\
2.37399999999966	-1.16624558198903e-09\\
2.37799999999997	-1.13161317065069e-09\\
2.378	-1.13161317065044e-09\\
2.37999999999997	-1.11452841420626e-09\\
2.38	-1.11452841420602e-09\\
2.38199999999997	-1.09759348264727e-09\\
2.38399999999995	-1.08080505641673e-09\\
2.38599999999997	-1.06415984491784e-09\\
2.386	-1.06415984491761e-09\\
2.38999999999995	-1.03128604354357e-09\\
2.39	-1.03128604354319e-09\\
2.39399999999995	-9.99027452896263e-10\\
2.396	-9.83151367853786e-10\\
2.39600000000002	-9.83151367853561e-10\\
2.39999999999997	-9.51890058947727e-10\\
2.4	-9.51890058947507e-10\\
2.40399999999995	-9.21262859377403e-10\\
2.40599999999997	-9.06179527525357e-10\\
2.406	-9.06179527525144e-10\\
2.40699999999997	-8.98694130278034e-10\\
2.407	-8.98694130277822e-10\\
2.40799999999998	-8.9124575583242e-10\\
2.40899999999997	-8.83834039210157e-10\\
2.41099999999995	-8.69119128511101e-10\\
2.41299999999997	-8.54546514001179e-10\\
2.413	-8.54546514000972e-10\\
2.41499999999997	-8.40144325727746e-10\\
2.415	-8.40144325727543e-10\\
2.41699999999997	-8.25940727141565e-10\\
2.41899999999995	-8.11932934282373e-10\\
2.41999999999997	-8.05001605517956e-10\\
2.42	-8.0500160551776e-10\\
2.42399999999995	-7.77752177356834e-10\\
2.426	-7.64408330174143e-10\\
2.42600000000003	-7.64408330173954e-10\\
2.427	-7.57805473075876e-10\\
2.42700000000003	-7.57805473075689e-10\\
2.42800000000002	-7.51248228174591e-10\\
2.429	-7.44736274158145e-10\\
2.43099999999998	-7.31846964634642e-10\\
2.43499999999993	-7.06597943149877e-10\\
2.43599999999997	-7.00394213140809e-10\\
2.436	-7.00394213140633e-10\\
2.43999999999997	-6.76004274925558e-10\\
2.44	-6.76004274925388e-10\\
2.44399999999998	-6.52280040411578e-10\\
2.446	-6.40661729127687e-10\\
2.44600000000003	-6.40661729127523e-10\\
2.448	-6.2920290829421e-10\\
2.44800000000002	-6.29202908294048e-10\\
2.44999999999999	-6.1791748794341e-10\\
2.45000000000002	-6.17917487943251e-10\\
2.45199999999998	-6.06819412079602e-10\\
2.45399999999995	-5.95906505444543e-10\\
2.45600000000002	-5.85176629065746e-10\\
2.45600000000005	-5.85176629065595e-10\\
2.45999999999998	-5.64257590217451e-10\\
2.46000000000001	-5.64257590217305e-10\\
2.46399999999994	-5.44045893928809e-10\\
2.46499999999997	-5.39101623049226e-10\\
2.465	-5.39101623049086e-10\\
2.46599999999998	-5.34200325724678e-10\\
2.46600000000001	-5.3420032572454e-10\\
2.46699999999999	-5.29341761795003e-10\\
2.46799999999998	-5.24525693185172e-10\\
2.46999999999996	-5.15020100042042e-10\\
2.47199999999998	-5.05681683162023e-10\\
2.47200000000001	-5.05681683161891e-10\\
2.47599999999996	-4.87524963431531e-10\\
2.47999999999991	-4.70067172426743e-10\\
2.47999999999997	-4.70067172426467e-10\\
2.48	-4.70067172426346e-10\\
2.48499999999997	-4.49207000838113e-10\\
2.485	-4.49207000837998e-10\\
2.48599999999997	-4.45161182251697e-10\\
2.486	-4.45161182251583e-10\\
2.48699999999999	-4.41156968210199e-10\\
2.48799999999998	-4.37194162497634e-10\\
2.48999999999995	-4.29392001365374e-10\\
2.49199999999997	-4.21753169600935e-10\\
2.492	-4.21753169600827e-10\\
2.494	-4.14276169974353e-10\\
2.49400000000002	-4.14276169974248e-10\\
2.49600000000002	-4.06959536974705e-10\\
2.49800000000001	-3.99801836513459e-10\\
2.49999999999997	-3.92801665653694e-10\\
2.5	-3.92801665653595e-10\\
2.50399999999999	-3.79214213353523e-10\\
2.50599999999997	-3.7261070893629e-10\\
2.506	-3.72610708936197e-10\\
2.50999999999999	-3.59777671791511e-10\\
2.51399999999998	-3.47434919367978e-10\\
2.51999999999997	-3.29819008564371e-10\\
2.52	-3.2981900856429e-10\\
2.52299999999997	-3.21406813448657e-10\\
2.523	-3.21406813448579e-10\\
2.52599999999996	-3.132534735627e-10\\
2.526	-3.13253473562602e-10\\
2.52899999999997	-3.0535539302427e-10\\
2.53199999999993	-2.97709088651961e-10\\
2.53199999999997	-2.9770908865188e-10\\
2.532	-2.97709088651798e-10\\
2.53799999999993	-2.83144521197966e-10\\
2.53799999999997	-2.83144521197889e-10\\
2.538	-2.83144521197813e-10\\
2.53999999999997	-2.78499822431848e-10\\
2.54	-2.78499822431783e-10\\
2.54199999999998	-2.73958667841102e-10\\
2.54399999999995	-2.69520167322178e-10\\
2.54599999999997	-2.65183450911658e-10\\
2.546	-2.65183450911598e-10\\
2.54999999999995	-2.56811990173398e-10\\
2.55199999999997	-2.52775605006605e-10\\
2.552	-2.52775605006549e-10\\
2.55499999999997	-2.46905476792075e-10\\
2.555	-2.46905476792021e-10\\
2.55799999999998	-2.41254394060164e-10\\
2.55800000000001	-2.41254394060112e-10\\
2.55999999999997	-2.37588763523654e-10\\
2.56	-2.37588763523603e-10\\
2.56199999999997	-2.33981259811801e-10\\
2.56399999999994	-2.30431175841995e-10\\
2.56599999999997	-2.26937815783919e-10\\
2.566	-2.2693781578387e-10\\
2.56999999999994	-2.20118539560074e-10\\
2.56999999999997	-2.20118539560019e-10\\
2.57	-2.20118539559963e-10\\
2.57399999999994	-2.13518084475836e-10\\
2.57799999999988	-2.07131275436497e-10\\
2.57999999999997	-2.04016416830315e-10\\
2.58	-2.04016416830271e-10\\
2.58099999999997	-2.02478355246968e-10\\
2.581	-2.02478355246924e-10\\
2.58199999999998	-2.00953104792171e-10\\
2.58299999999997	-1.99440590710864e-10\\
2.58499999999995	-1.96453475835079e-10\\
2.58599999999997	-1.94978728669949e-10\\
2.586	-1.94978728669907e-10\\
2.58999999999995	-1.89203462789397e-10\\
2.59	-1.89203462789331e-10\\
2.59199999999997	-1.8638907515443e-10\\
2.592	-1.8638907515439e-10\\
2.59399999999997	-1.83625148761516e-10\\
2.59599999999995	-1.8091351158901e-10\\
2.59799999999997	-1.78253632145454e-10\\
2.598	-1.78253632145417e-10\\
2.59999999999997	-1.75644989087918e-10\\
2.6	-1.75644989087881e-10\\
2.60199999999997	-1.73087071115691e-10\\
2.60399999999995	-1.70579376866916e-10\\
2.60599999999997	-1.68121414823735e-10\\
2.606	-1.68121414823701e-10\\
2.60999999999995	-1.63352769942347e-10\\
2.61	-1.63352769942288e-10\\
2.61399999999994	-1.58777397607867e-10\\
2.61599999999997	-1.56561061763077e-10\\
2.616	-1.56561061763046e-10\\
2.61999999999994	-1.52224403568707e-10\\
2.61999999999997	-1.52224403568674e-10\\
2.62	-1.52224403568641e-10\\
2.62399999999995	-1.47983262659608e-10\\
2.62499999999997	-1.46937510718826e-10\\
2.625	-1.46937510718797e-10\\
2.62599999999997	-1.45897469183968e-10\\
2.626	-1.45897469183939e-10\\
2.62699999999999	-1.44863087093697e-10\\
2.62799999999998	-1.43834313762169e-10\\
2.62999999999995	-1.41793392007215e-10\\
2.63199999999997	-1.3977430389051e-10\\
2.632	-1.39774303890481e-10\\
2.63599999999995	-1.35800049789669e-10\\
2.63899999999997	-1.32873759498341e-10\\
2.639	-1.32873759498314e-10\\
2.63999999999997	-1.31908435437998e-10\\
2.64	-1.31908435437971e-10\\
2.64099999999999	-1.30948085356726e-10\\
2.64199999999998	-1.29992662196324e-10\\
2.64399999999995	-1.28096409613474e-10\\
2.64599999999997	-1.26219306016573e-10\\
2.646	-1.26219306016546e-10\\
2.64999999999995	-1.22521077794882e-10\\
2.65099999999997	-1.21607918423308e-10\\
2.651	-1.21607918423282e-10\\
2.65499999999995	-1.18002221972096e-10\\
2.6589999999999	-1.14470596266435e-10\\
2.65999999999997	-1.13598937650055e-10\\
2.66	-1.1359893765003e-10\\
2.66599999999997	-1.0846018701723e-10\\
2.666	-1.08460187017206e-10\\
2.66799999999997	-1.0678100744891e-10\\
2.668	-1.06781007448886e-10\\
2.66999999999996	-1.05118145273492e-10\\
2.67199999999993	-1.03471274556863e-10\\
2.674	-1.01840072505916e-10\\
2.67400000000002	-1.01840072505893e-10\\
2.67799999999996	-9.86346122859581e-11\\
2.67799999999999	-9.86346122859297e-11\\
2.67800000000003	-9.86346122859012e-11\\
2.67999999999997	-9.70625291353826e-11\\
2.68	-9.70625291353604e-11\\
2.68199999999995	-9.55104651203672e-11\\
2.68399999999989	-9.39781160303321e-11\\
2.68599999999997	-9.24651815188273e-11\\
2.686	-9.24651815188059e-11\\
2.68999999999989	-8.94963738240988e-11\\
2.69399999999978	-8.66017143377782e-11\\
2.69499999999997	-8.58893678996115e-11\\
2.695	-8.58893678995913e-11\\
2.69699999999997	-8.44780120371611e-11\\
2.697	-8.44780120371412e-11\\
2.69899999999996	-8.30842087250671e-11\\
2.69999999999997	-8.23938037625289e-11\\
2.7	-8.23938037625094e-11\\
2.70199999999997	-8.10258180730562e-11\\
2.70399999999993	-7.96747082713699e-11\\
2.70599999999997	-7.83402095346764e-11\\
2.706	-7.83402095346575e-11\\
2.70899999999997	-7.63690358812078e-11\\
2.709	-7.63690358811893e-11\\
2.71199999999996	-7.44379854343939e-11\\
2.71299999999997	-7.38040050030584e-11\\
2.713	-7.38040050030405e-11\\
2.71599999999997	-7.19307379400007e-11\\
2.71899999999993	-7.00998421219007e-11\\
2.71999999999997	-6.94988192540647e-11\\
2.72	-6.94988192540476e-11\\
2.72599999999993	-6.59880275171684e-11\\
2.726	-6.59880275171329e-11\\
2.72600000000002	-6.59880275171166e-11\\
2.72999999999997	-6.3736214360205e-11\\
2.73	-6.37362143601892e-11\\
2.73199999999999	-6.26362549594678e-11\\
2.73200000000002	-6.26362549594523e-11\\
2.73400000000002	-6.15539426691592e-11\\
2.73600000000001	-6.04897029300394e-11\\
2.73799999999999	-5.94433271470043e-11\\
2.73800000000002	-5.94433271469896e-11\\
2.73999999999997	-5.84146102277692e-11\\
2.74	-5.84146102277547e-11\\
2.74199999999995	-5.74033505411959e-11\\
2.7439999999999	-5.6409349876465e-11\\
2.74599999999997	-5.54324134055397e-11\\
2.746	-5.54324134055259e-11\\
2.7499999999999	-5.35289704250801e-11\\
2.7539999999998	-5.16915292023018e-11\\
2.75499999999997	-5.12423117193521e-11\\
2.755	-5.12423117193394e-11\\
2.75999999999997	-4.90559830973219e-11\\
2.76	-4.90559830973097e-11\\
2.76499999999998	-4.69673060585418e-11\\
2.76500000000001	-4.69673060585302e-11\\
2.76599999999997	-4.65610615858433e-11\\
2.766	-4.65610615858318e-11\\
2.76699999999999	-4.61586005422954e-11\\
2.76700000000002	-4.61586005422841e-11\\
2.76800000000001	-4.57598418596641e-11\\
2.76899999999999	-4.53647046520195e-11\\
2.77099999999997	-4.45852173909519e-11\\
2.77299999999999	-4.38199859765541e-11\\
2.77300000000002	-4.38199859765433e-11\\
2.77699999999997	-4.23316935039945e-11\\
2.77999999999997	-4.12518002455e-11\\
2.78	-4.125180024549e-11\\
2.78399999999995	-3.98594718766973e-11\\
2.784	-3.98594718766831e-11\\
2.786	-3.91833696361644e-11\\
2.78600000000003	-3.91833696361549e-11\\
2.78800000000004	-3.85204644638442e-11\\
2.79000000000004	-3.7870626427743e-11\\
2.792	-3.72337281571029e-11\\
2.79200000000003	-3.72337281570939e-11\\
2.79600000000004	-3.59980266101386e-11\\
2.79999999999997	-3.48121634897285e-11\\
2.8	-3.48121634897202e-11\\
2.80400000000001	-3.36752090183786e-11\\
2.80599999999997	-3.31247924750994e-11\\
2.806	-3.31247924750917e-11\\
2.81000000000001	-3.20595413366134e-11\\
2.81199999999997	-3.15444979480709e-11\\
2.812	-3.15444979480637e-11\\
2.81299999999997	-3.12913272580302e-11\\
2.813	-3.12913272580231e-11\\
2.81399999999998	-3.10410406498908e-11\\
2.81499999999997	-3.07936258593316e-11\\
2.81699999999995	-3.03073633775614e-11\\
2.81899999999997	-2.98324445033773e-11\\
2.819	-2.98324445033706e-11\\
2.81999999999997	-2.95987768408143e-11\\
2.82	-2.95987768408077e-11\\
2.82099999999999	-2.9367044519258e-11\\
2.82199999999998	-2.91372361835686e-11\\
2.82399999999995	-2.8683346520877e-11\\
2.82599999999997	-2.82370188885987e-11\\
2.826	-2.82370188885924e-11\\
2.82999999999995	-2.73667012552836e-11\\
2.83399999999991	-2.65256009097085e-11\\
2.83499999999997	-2.63198132451408e-11\\
2.835	-2.6319813245135e-11\\
2.83999999999997	-2.53172963516026e-11\\
2.84	-2.5317296351597e-11\\
2.84199999999997	-2.49284365997138e-11\\
2.842	-2.49284365997083e-11\\
2.84399999999996	-2.45464029050013e-11\\
2.84599999999992	-2.41711203868222e-11\\
2.846	-2.41711203868079e-11\\
2.84600000000003	-2.41711203868026e-11\\
2.84999999999996	-2.34405159625682e-11\\
2.852	-2.30850508554508e-11\\
2.85200000000003	-2.30850508554457e-11\\
2.853	-2.29097784263452e-11\\
2.85300000000003	-2.29097784263402e-11\\
2.85400000000002	-2.27361766550507e-11\\
2.85500000000001	-2.25642370348889e-11\\
2.85699999999998	-2.22253106286781e-11\\
2.85999999999997	-2.17291791508379e-11\\
2.86	-2.17291791508333e-11\\
2.86399999999995	-2.10902136703638e-11\\
2.86599999999997	-2.07802510346248e-11\\
2.866	-2.07802510346205e-11\\
2.86999999999995	-2.01790621927587e-11\\
2.87	-2.01790621927515e-11\\
2.87099999999997	-2.00326253336902e-11\\
2.871	-2.00326253336861e-11\\
2.87199999999998	-1.98877181513316e-11\\
2.87299999999997	-1.97443335450681e-11\\
2.87499999999995	-1.94621040315792e-11\\
2.87699999999997	-1.9185881475139e-11\\
2.87699999999999	-1.91858814751352e-11\\
2.87999999999997	-1.87798423171719e-11\\
2.88	-1.8779842317168e-11\\
2.88299999999998	-1.83813176899147e-11\\
2.88599999999996	-1.79901318374748e-11\\
2.886	-1.79901318374693e-11\\
2.888	-1.77333330394983e-11\\
2.88800000000003	-1.77333330394947e-11\\
2.89000000000003	-1.74796688467695e-11\\
2.89200000000003	-1.7229089539088e-11\\
2.89600000000002	-1.67369897164925e-11\\
2.89999999999997	-1.62566477350344e-11\\
2.9	-1.62566477350311e-11\\
2.90499999999997	-1.56721818402349e-11\\
2.905	-1.56721818402316e-11\\
2.90599999999997	-1.55573593362218e-11\\
2.90599999999999	-1.55573593362185e-11\\
2.90699999999998	-1.54432138507151e-11\\
2.90799999999997	-1.5329739790323e-11\\
2.90999999999995	-1.51047837364932e-11\\
2.91199999999997	-1.48824470824713e-11\\
2.91199999999999	-1.48824470824682e-11\\
2.91599999999995	-1.44479376816201e-11\\
2.91799999999997	-1.42362996181168e-11\\
2.91799999999999	-1.42362996181138e-11\\
2.91999999999997	-1.40283504266273e-11\\
2.92	-1.40283504266244e-11\\
2.92199999999998	-1.38240493485795e-11\\
2.92399999999996	-1.36233563401696e-11\\
2.926	-1.34262320647849e-11\\
2.92600000000003	-1.34262320647821e-11\\
2.92899999999997	-1.31371526975295e-11\\
2.92899999999999	-1.31371526975268e-11\\
2.93199999999993	-1.28558887201996e-11\\
2.93499999999987	-1.25823160907694e-11\\
2.93499999999997	-1.25823160907606e-11\\
2.935	-1.2582316090758e-11\\
2.93999999999997	-1.21383054085953e-11\\
2.94	-1.21383054085928e-11\\
2.94499999999998	-1.17051378374286e-11\\
2.94599999999997	-1.16197585696554e-11\\
2.946	-1.1619758569653e-11\\
2.95099999999998	-1.11989033510574e-11\\
2.95199999999997	-1.11159116605299e-11\\
2.952	-1.11159116605276e-11\\
2.95699999999998	-1.07066262967569e-11\\
2.95799999999999	-1.06258757646096e-11\\
2.95800000000002	-1.06258757646073e-11\\
2.95999999999997	-1.04654494788889e-11\\
2.96	-1.04654494788866e-11\\
2.96199999999995	-1.03064300573495e-11\\
2.9639999999999	-1.01487863313495e-11\\
2.96599999999997	-9.99248740210029e-12\\
2.966	-9.99248740209808e-12\\
2.9699999999999	-9.68380165162971e-12\\
2.96999999999999	-9.68380165162281e-12\\
2.97000000000002	-9.68380165162063e-12\\
2.97399999999992	-9.38089276145319e-12\\
2.97499999999997	-9.30615929125593e-12\\
2.975	-9.30615929125381e-12\\
2.9789999999999	-9.01109055694533e-12\\
2.97999999999997	-8.93827158689449e-12\\
2.98	-8.93827158689242e-12\\
2.9839999999999	-8.6506815123428e-12\\
2.986	-8.50904869864449e-12\\
2.98600000000003	-8.50904869864249e-12\\
2.98699999999997	-8.43876065658802e-12\\
2.98699999999999	-8.43876065658603e-12\\
2.98799999999998	-8.36882025989018e-12\\
2.98899999999997	-8.2992240813984e-12\\
2.99099999999995	-8.16105075482004e-12\\
2.99299999999999	-8.02421359434202e-12\\
2.99300000000002	-8.02421359434009e-12\\
2.99699999999997	-7.75560464782525e-12\\
2.99999999999997	-7.55898584961571e-12\\
3	-7.55898584961387e-12\\
3.00399999999995	-7.30311313697392e-12\\
3.00599999999997	-7.17781412614375e-12\\
3.006	-7.17781412614198e-12\\
3.00999999999995	-6.93236799595898e-12\\
3.01	-6.93236799595605e-12\\
3.01399999999995	-6.69363124445467e-12\\
3.01599999999997	-6.57672018985643e-12\\
3.01599999999999	-6.57672018985478e-12\\
3.01999999999995	-6.34769814408939e-12\\
3.02	-6.34769814408623e-12\\
3.02399999999995	-6.12492706589027e-12\\
3.02599999999997	-6.01583086939761e-12\\
3.026	-6.01583086939607e-12\\
3.02799999999997	-5.90823229118074e-12\\
3.02799999999999	-5.90823229117922e-12\\
3.02999999999996	-5.8022619464486e-12\\
3.03199999999992	-5.69805076940972e-12\\
3.03599999999985	-5.4948245563101e-12\\
3.03999999999997	-5.29839430992907e-12\\
3.04	-5.2983943099277e-12\\
3.04499999999997	-5.06217920525298e-12\\
3.04499999999999	-5.06217920525167e-12\\
3.04599999999997	-5.01615591103689e-12\\
3.046	-5.01615591103559e-12\\
3.04699999999998	-4.97053388410342e-12\\
3.04799999999996	-4.92531088886532e-12\\
3.04999999999992	-4.8360531491721e-12\\
3.05199999999997	-4.74836519709039e-12\\
3.052	-4.74836519708916e-12\\
3.05599999999992	-4.57787317217652e-12\\
3.05799999999997	-4.49509641932757e-12\\
3.058	-4.4950964193264e-12\\
3.05999999999997	-4.41394409972527e-12\\
3.06	-4.41394409972413e-12\\
3.06199999999997	-4.3344003073063e-12\\
3.06399999999995	-4.25644945117825e-12\\
3.066	-4.1800762526682e-12\\
3.06600000000003	-4.18007625266713e-12\\
3.06999999999997	-4.03200325750283e-12\\
3.07399999999992	-3.89006522494165e-12\\
3.07399999999996	-3.89006522494033e-12\\
3.07399999999999	-3.89006522493902e-12\\
3.07999999999997	-3.68841908442741e-12\\
3.07999999999999	-3.68841908442649e-12\\
3.08599999999997	-3.49882547894769e-12\\
3.08599999999999	-3.49882547894682e-12\\
3.09199999999997	-3.3198038282741e-12\\
3.09199999999999	-3.31980382827327e-12\\
3.09799999999997	-3.15103829179038e-12\\
3.09999999999997	-3.09701027590395e-12\\
3.1	-3.09701027590319e-12\\
3.10299999999997	-3.01801953502208e-12\\
3.10299999999999	-3.01801953502134e-12\\
3.10599999999996	-2.94145944854362e-12\\
3.106	-2.9414594485425e-12\\
3.10600000000003	-2.94145944854179e-12\\
3.10899999999999	-2.86729625194502e-12\\
3.11199999999996	-2.7954972380575e-12\\
3.11199999999999	-2.79549723805663e-12\\
3.11200000000003	-2.79549723805577e-12\\
3.11499999999997	-2.72599809685364e-12\\
3.115	-2.72599809685299e-12\\
3.11799999999995	-2.65873553332895e-12\\
3.11999999999997	-2.61512167829872e-12\\
3.12	-2.61512167829811e-12\\
3.12299999999995	-2.55152130524519e-12\\
3.12599999999989	-2.49008056252698e-12\\
3.12599999999995	-2.49008056252586e-12\\
3.126	-2.49008056252473e-12\\
3.127	-2.47007552219259e-12\\
3.12700000000003	-2.47007552219202e-12\\
3.12800000000003	-2.45030644217331e-12\\
3.12900000000003	-2.43077235369364e-12\\
3.13100000000003	-2.3924053341222e-12\\
3.13199999999997	-2.37357052302229e-12\\
3.13199999999999	-2.37357052302176e-12\\
3.13599999999999	-2.30053453391936e-12\\
3.13799999999997	-2.26538600604482e-12\\
3.13799999999999	-2.26538600604433e-12\\
3.13999999999997	-2.23096562398205e-12\\
3.14	-2.23096562398156e-12\\
3.14199999999998	-2.19709105438571e-12\\
3.14399999999996	-2.16375565770895e-12\\
3.14599999999997	-2.13095290008445e-12\\
3.146	-2.13095290008399e-12\\
3.14999999999996	-2.06691968746139e-12\\
3.15	-2.06691968746074e-12\\
3.15399999999996	-2.00494121117471e-12\\
3.15599999999997	-1.97470725153823e-12\\
3.156	-1.97470725153781e-12\\
3.15999999999996	-1.91572025843586e-12\\
3.16	-1.91572025843527e-12\\
3.16099999999997	-1.90127781252388e-12\\
3.16099999999999	-1.90127781252347e-12\\
3.16199999999996	-1.88695566329608e-12\\
3.16299999999992	-1.87275310895104e-12\\
3.16499999999985	-1.84470400702205e-12\\
3.16599999999997	-1.8308560850115e-12\\
3.166	-1.83085608501111e-12\\
3.16999999999986	-1.77662615712271e-12\\
3.17199999999997	-1.75019896850808e-12\\
3.172	-1.75019896850771e-12\\
3.17599999999986	-1.69878325375639e-12\\
3.17999999999972	-1.64931166116149e-12\\
3.17999999999997	-1.64931166115842e-12\\
3.18	-1.64931166115807e-12\\
3.18499999999997	-1.59014714295145e-12\\
3.185	-1.59014714295112e-12\\
3.18599999999997	-1.57866506083961e-12\\
3.186	-1.57866506083929e-12\\
3.18699999999997	-1.56729859313016e-12\\
3.18799999999994	-1.55604718283906e-12\\
3.18999999999989	-1.53388733483639e-12\\
3.18999999999994	-1.5338873348358e-12\\
3.18999999999999	-1.53388733483521e-12\\
3.19399999999988	-1.49092444414231e-12\\
3.19599999999999	-1.47011298057184e-12\\
3.19600000000002	-1.47011298057154e-12\\
3.198	-1.44963820941492e-12\\
3.19800000000003	-1.44963820941463e-12\\
3.19999999999998	-1.42939162335557e-12\\
3.20000000000001	-1.42939162335529e-12\\
3.20199999999996	-1.4093692539842e-12\\
3.20399999999991	-1.38956717685257e-12\\
3.20599999999998	-1.3699815106759e-12\\
3.20600000000001	-1.36998151067562e-12\\
3.20999999999991	-1.33144409750937e-12\\
3.21399999999982	-1.29372679916477e-12\\
3.21899999999997	-1.24768847019449e-12\\
3.21899999999999	-1.24768847019424e-12\\
3.21999999999997	-1.2386240472449e-12\\
3.22	-1.23862404724464e-12\\
3.22099999999998	-1.22960633023258e-12\\
3.22199999999996	-1.22063487718652e-12\\
3.22399999999992	-1.20282900682581e-12\\
3.22599999999997	-1.18520294614256e-12\\
3.226	-1.18520294614231e-12\\
3.22999999999992	-1.15047646937924e-12\\
3.23099999999999	-1.141901874859e-12\\
3.23100000000002	-1.14190187485876e-12\\
3.23499999999994	-1.1080442748024e-12\\
3.23699999999999	-1.09137793497014e-12\\
3.23700000000002	-1.0913779349699e-12\\
3.23999999999997	-1.06669730088544e-12\\
3.24	-1.06669730088521e-12\\
3.24299999999996	-1.04238965486689e-12\\
3.24599999999991	-1.01844427684422e-12\\
3.24599999999997	-1.01844427684369e-12\\
3.246	-1.01844427684346e-12\\
3.24799999999997	-1.0026767300233e-12\\
3.24799999999999	-1.00267673002307e-12\\
3.24999999999996	-9.87062404126876e-13\\
3.25199999999992	-9.71598238623281e-13\\
3.25399999999997	-9.56281202475678e-13\\
3.25399999999999	-9.56281202475462e-13\\
3.25499999999998	-9.48683499302722e-13\\
3.255	-9.48683499302507e-13\\
3.25599999999998	-9.41134616569704e-13\\
3.25699999999997	-9.33634184373027e-13\\
3.25899999999993	-9.1877720385342e-13\\
3.25999999999997	-9.11419927528205e-13\\
3.26	-9.11419927527996e-13\\
3.26399999999993	-8.82457196716489e-13\\
3.26599999999997	-8.68250696761139e-13\\
3.266	-8.68250696760939e-13\\
3.267	-8.61214939072229e-13\\
3.26700000000003	-8.6121493907203e-13\\
3.26800000000003	-8.54223715845782e-13\\
3.26900000000003	-8.47276684504138e-13\\
3.27100000000003	-8.33513837999691e-13\\
3.27500000000003	-8.06503612785155e-13\\
3.27699999999997	-7.9325093978526e-13\\
3.27699999999999	-7.93250939785073e-13\\
3.27999999999997	-7.73680162750712e-13\\
3.28	-7.73680162750528e-13\\
3.28299999999998	-7.54471604941536e-13\\
3.28599999999996	-7.35616794788954e-13\\
3.286	-7.35616794788708e-13\\
3.28899999999999	-7.17107416952931e-13\\
3.28900000000002	-7.17107416952757e-13\\
3.29	-7.1101735941356e-13\\
3.29000000000003	-7.11017359413387e-13\\
3.29100000000001	-7.04973251618005e-13\\
3.29199999999999	-6.98974797402188e-13\\
3.29399999999996	-6.87113676230055e-13\\
3.296	-6.75431671071039e-13\\
3.29600000000003	-6.75431671070874e-13\\
3.29999999999996	-6.52595884650585e-13\\
3.3	-6.52595884650338e-13\\
3.30000000000003	-6.52595884650178e-13\\
3.30399999999996	-6.3044953370589e-13\\
3.30599999999997	-6.19629449579047e-13\\
3.30599999999999	-6.19629449578895e-13\\
3.30999999999992	-5.98484859774678e-13\\
3.31199999999999	-5.8815620967493e-13\\
3.31200000000002	-5.88156209674785e-13\\
3.31599999999995	-5.68000025083738e-13\\
3.31999999999988	-5.48514845707144e-13\\
3.32	-5.48514845706538e-13\\
3.32000000000003	-5.48514845706402e-13\\
3.32499999999998	-5.25078747394788e-13\\
3.325	-5.25078747394657e-13\\
3.326	-5.20511932882717e-13\\
3.32600000000003	-5.20511932882588e-13\\
3.32700000000003	-5.1598472671187e-13\\
3.32800000000003	-5.11496907038435e-13\\
3.33000000000002	-5.02638549486335e-13\\
3.332	-4.93935123954123e-13\\
3.33200000000003	-4.93935123954e-13\\
3.33499999999999	-4.81166759776665e-13\\
3.33500000000002	-4.81166759776546e-13\\
3.33799999999998	-4.68737530369277e-13\\
3.33999999999997	-4.60637072643467e-13\\
3.34	-4.60637072643353e-13\\
3.34299999999997	-4.48760959518755e-13\\
3.34599999999993	-4.37209689678633e-13\\
3.34599999999997	-4.37209689678492e-13\\
3.346	-4.37209689678352e-13\\
3.34699999999999	-4.33430569199851e-13\\
3.34700000000002	-4.33430569199744e-13\\
3.34800000000001	-4.29686214014326e-13\\
3.349	-4.25975864581989e-13\\
3.35099999999999	-4.18656457402482e-13\\
3.35499999999995	-4.0441782308688e-13\\
3.35999999999997	-3.87355577165823e-13\\
3.36	-3.87355577165728e-13\\
3.36399999999997	-3.74281574790268e-13\\
3.36399999999999	-3.74281574790177e-13\\
3.36599999999997	-3.67932955903292e-13\\
3.366	-3.67932955903203e-13\\
3.36799999999998	-3.6170825786299e-13\\
3.36999999999996	-3.55606260604363e-13\\
3.37199999999997	-3.49625768112222e-13\\
3.372	-3.49625768112138e-13\\
3.37599999999996	-3.38022496189186e-13\\
3.37799999999997	-3.32396908482526e-13\\
3.378	-3.32396908482447e-13\\
3.37999999999997	-3.26887208460771e-13\\
3.38	-3.26887208460694e-13\\
3.38199999999997	-3.21492316208679e-13\\
3.38399999999995	-3.1621117430639e-13\\
3.38599999999997	-3.11042747629482e-13\\
3.386	-3.11042747629409e-13\\
3.38999999999995	-3.01040009751384e-13\\
3.39299999999997	-2.93826458140032e-13\\
3.39299999999999	-2.93826458139964e-13\\
3.39499999999998	-2.89153028123512e-13\\
3.395	-2.89153028123446e-13\\
3.39699999999999	-2.84587009855958e-13\\
3.39899999999997	-2.80127508380005e-13\\
3.399	-2.80127508379942e-13\\
3.39999999999997	-2.77933362507385e-13\\
3.4	-2.77933362507323e-13\\
3.40099999999998	-2.75757389539134e-13\\
3.40199999999996	-2.73599482850152e-13\\
3.40399999999991	-2.6933744623844e-13\\
3.40599999999997	-2.65146417296355e-13\\
3.406	-2.65146417296296e-13\\
3.40999999999991	-2.56974110372578e-13\\
3.41099999999997	-2.54974146530968e-13\\
3.411	-2.54974146530911e-13\\
3.41499999999991	-2.47143802649475e-13\\
3.41899999999982	-2.3958010242293e-13\\
3.41999999999997	-2.377301409291e-13\\
3.42	-2.37730140929048e-13\\
3.42199999999997	-2.34078736883186e-13\\
3.42199999999999	-2.34078736883135e-13\\
3.42399999999996	-2.304914297469e-13\\
3.42599999999992	-2.26967516339711e-13\\
3.426	-2.26967516339565e-13\\
3.42600000000003	-2.26967516339516e-13\\
3.42999999999996	-2.20107120209086e-13\\
3.43	-2.20107120209004e-13\\
3.432	-2.16769292823825e-13\\
3.43200000000003	-2.16769292823778e-13\\
3.43400000000003	-2.13493354232011e-13\\
3.43600000000003	-2.10279846981791e-13\\
3.438	-2.07128141213328e-13\\
3.43800000000003	-2.07128141213284e-13\\
3.43999999999997	-2.04037619184366e-13\\
3.44	-2.04037619184323e-13\\
3.44199999999995	-2.01007675144722e-13\\
3.44399999999989	-1.98037715213592e-13\\
3.44599999999997	-1.95127157267125e-13\\
3.446	-1.95127157267084e-13\\
3.44999999999989	-1.89481976955495e-13\\
3.45099999999996	-1.88106930681509e-13\\
3.45099999999999	-1.8810693068147e-13\\
3.45499999999988	-1.82749719397263e-13\\
3.45699999999996	-1.80155981707531e-13\\
3.45699999999999	-1.80155981707494e-13\\
3.45999999999997	-1.76343262411956e-13\\
3.46	-1.7634326241192e-13\\
3.46299999999998	-1.72601104764111e-13\\
3.46499999999998	-1.70144717666367e-13\\
3.465	-1.70144717666332e-13\\
3.46599999999997	-1.68927858410524e-13\\
3.466	-1.68927858410489e-13\\
3.46699999999997	-1.67718475725644e-13\\
3.46799999999994	-1.66516510351418e-13\\
3.46999999999988	-1.64134596309215e-13\\
3.47199999999997	-1.6178164941937e-13\\
3.472	-1.61781649419336e-13\\
3.47599999999988	-1.57160817952207e-13\\
3.47999999999976	-1.52650392979641e-13\\
3.47999999999996	-1.52650392979412e-13\\
3.47999999999999	-1.52650392979381e-13\\
3.48599999999996	-1.46084054779771e-13\\
3.48599999999999	-1.4608405477974e-13\\
3.49199999999996	-1.39746609218652e-13\\
3.49199999999999	-1.39746609218622e-13\\
3.49799999999996	-1.33679265825626e-13\\
3.5	-1.31726616884913e-13\\
3.50000000000003	-1.31726616884886e-13\\
3.506	-1.26072708008889e-13\\
3.50600000000003	-1.26072708008863e-13\\
3.50899999999999	-1.2335824434972e-13\\
3.50900000000002	-1.23358244349694e-13\\
3.51199999999998	-1.20717167431565e-13\\
3.51499999999995	-1.18148312487092e-13\\
3.51499999999998	-1.1814831248706e-13\\
3.51500000000002	-1.18148312487028e-13\\
3.518	-1.15634247311025e-13\\
3.51800000000003	-1.15634247311001e-13\\
3.51999999999997	-1.13979039258338e-13\\
3.52	-1.13979039258315e-13\\
3.52199999999995	-1.12340120548718e-13\\
3.52399999999989	-1.10717169948701e-13\\
3.52599999999997	-1.09109869353589e-13\\
3.526	-1.09109869353566e-13\\
3.52999999999989	-1.05940961045359e-13\\
3.53399999999978	-1.02830911036443e-13\\
3.53499999999998	-1.02062308393732e-13\\
3.535	-1.0206230839371e-13\\
3.53799999999997	-9.97772808476234e-14\\
3.53799999999999	-9.97772808476019e-14\\
3.53999999999997	-9.82708733568416e-14\\
3.54	-9.82708733568203e-14\\
3.54199999999998	-9.67776763583061e-14\\
3.54399999999996	-9.5297397180491e-14\\
3.54599999999997	-9.38297456829367e-14\\
3.546	-9.38297456829159e-14\\
3.54999999999996	-9.09311774946338e-14\\
3.54999999999999	-9.09311774946127e-14\\
3.553	-8.87923015406012e-14\\
3.55300000000003	-8.87923015405811e-14\\
3.55600000000004	-8.66870187720262e-14\\
3.55900000000005	-8.46144007247041e-14\\
3.55999999999997	-8.39306284738931e-14\\
3.56	-8.39306284738737e-14\\
3.56599999999999	-7.99002130614967e-14\\
3.56600000000002	-7.99002130614779e-14\\
3.56699999999996	-7.92402062992367e-14\\
3.56699999999999	-7.9240206299218e-14\\
3.56799999999996	-7.85834639392896e-14\\
3.56899999999992	-7.79299537984646e-14\\
3.56999999999998	-7.72796438545315e-14\\
3.57	-7.7279643854513e-14\\
3.57199999999993	-7.59884972517226e-14\\
3.57299999999996	-7.53475973262711e-14\\
3.57299999999999	-7.53475973262529e-14\\
3.57499999999992	-7.40777193373298e-14\\
3.57699999999985	-7.28253515077975e-14\\
3.57899999999996	-7.15902483686228e-14\\
3.57899999999999	-7.15902483686054e-14\\
3.57999999999997	-7.09790952759618e-14\\
3.58	-7.09790952759445e-14\\
3.58099999999998	-7.03721678368933e-14\\
3.58199999999997	-6.97694363113005e-14\\
3.58399999999993	-6.85764430687827e-14\\
3.58599999999997	-6.73998817170701e-14\\
3.586	-6.73998817170535e-14\\
3.58999999999993	-6.50951354265461e-14\\
3.59399999999986	-6.28533903997532e-14\\
3.59599999999996	-6.17555921885506e-14\\
3.59599999999999	-6.17555921885352e-14\\
3.59999999999997	-5.96050685121926e-14\\
3.6	-5.96050685121776e-14\\
3.60399999999998	-5.75132416300556e-14\\
3.60499999999998	-5.69992630875212e-14\\
3.605	-5.69992630875066e-14\\
3.60599999999997	-5.64888251834409e-14\\
3.606	-5.64888251834264e-14\\
3.60699999999997	-5.59819029067255e-14\\
3.60799999999994	-5.54784714176222e-14\\
3.60799999999999	-5.54784714175963e-14\\
3.60999999999993	-5.44834068114571e-14\\
3.61199999999987	-5.35048608415422e-14\\
3.61399999999996	-5.25426417089934e-14\\
3.61399999999999	-5.25426417089799e-14\\
3.61799999999987	-5.06664327306191e-14\\
3.61999999999997	-4.97520751404169e-14\\
3.62	-4.9752075140404e-14\\
3.62399999999988	-4.79699576405413e-14\\
3.62499999999996	-4.75340084912547e-14\\
3.62499999999999	-4.75340084912423e-14\\
3.62599999999997	-4.71018484285682e-14\\
3.626	-4.7101848428556e-14\\
3.62699999999998	-4.66734562769173e-14\\
3.62799999999997	-4.62488110446081e-14\\
3.62999999999993	-4.54106782897531e-14\\
3.63199999999997	-4.45872858867035e-14\\
3.632	-4.45872858866919e-14\\
3.63599999999993	-4.29863608493592e-14\\
3.63999999999985	-4.14470621278111e-14\\
3.63999999999997	-4.1447062127765e-14\\
3.64	-4.14470621277543e-14\\
3.64599999999997	-3.92510362891824e-14\\
3.646	-3.92510362891723e-14\\
3.65199999999997	-3.71870909114119e-14\\
3.652	-3.71870909114025e-14\\
3.65399999999996	-3.65278243738834e-14\\
3.65399999999999	-3.65278243738742e-14\\
3.65599999999995	-3.5882697769789e-14\\
3.65799999999992	-3.52515846490476e-14\\
3.65999999999996	-3.46343613110151e-14\\
3.65999999999999	-3.46343613110064e-14\\
3.66399999999992	-3.34363201442009e-14\\
3.66599999999996	-3.28540718960251e-14\\
3.66599999999999	-3.28540718960169e-14\\
3.66999999999992	-3.17225494368373e-14\\
3.67399999999984	-3.06342566306588e-14\\
3.67499999999998	-3.03688373308038e-14\\
3.67500000000001	-3.03688373307963e-14\\
3.67999999999997	-2.90810155675151e-14\\
3.68	-2.9081015567508e-14\\
3.68299999999996	-2.833929024112e-14\\
3.68299999999999	-2.83392902411131e-14\\
3.68599999999995	-2.76203888282483e-14\\
3.686	-2.76203888282368e-14\\
3.68899999999996	-2.69239942808392e-14\\
3.69199999999993	-2.62497994776264e-14\\
3.69199999999997	-2.62497994776156e-14\\
3.692	-2.62497994776094e-14\\
3.69299999999998	-2.60299154232021e-14\\
3.69300000000001	-2.60299154231958e-14\\
3.69399999999998	-2.58123859995534e-14\\
3.69499999999995	-2.55972005467561e-14\\
3.6969999999999	-2.51738194915144e-14\\
3.69999999999997	-2.45560677918906e-14\\
3.7	-2.45560677918848e-14\\
3.7039999999999	-2.3764307876104e-14\\
3.70599999999997	-2.33819281238275e-14\\
3.706	-2.33819281238222e-14\\
3.7099999999999	-2.26437942786869e-14\\
3.70999999999998	-2.26437942786725e-14\\
3.71000000000001	-2.26437942786674e-14\\
3.71199999999996	-2.22878955086735e-14\\
3.71199999999999	-2.22878955086685e-14\\
3.71399999999995	-2.19406819257382e-14\\
3.71599999999991	-2.16020854746506e-14\\
3.71799999999996	-2.12720397891771e-14\\
3.71799999999999	-2.12720397891725e-14\\
3.71999999999997	-2.09488314138285e-14\\
3.72	-2.0948831413824e-14\\
3.72199999999998	-2.06307482331831e-14\\
3.72399999999997	-2.03177279017053e-14\\
3.72599999999997	-2.00097090661947e-14\\
3.726	-2.00097090661904e-14\\
3.728	-1.97066313541854e-14\\
3.72800000000003	-1.97066313541811e-14\\
3.73000000000004	-1.94084353616775e-14\\
3.73200000000004	-1.91150626411012e-14\\
3.73600000000004	-1.85425579418699e-14\\
3.74	-1.79886683773474e-14\\
3.74000000000003	-1.79886683773435e-14\\
3.74099999999996	-1.7853053394309e-14\\
3.74099999999999	-1.78530533943051e-14\\
3.74199999999996	-1.77185680019975e-14\\
3.74299999999992	-1.75852056090617e-14\\
3.74499999999985	-1.73218237368769e-14\\
3.745	-1.73218237368569e-14\\
3.74500000000003	-1.73218237368531e-14\\
3.746	-1.71917913517187e-14\\
3.74600000000003	-1.71917913517151e-14\\
3.747	-1.70628561539129e-14\\
3.74799999999997	-1.69350118254708e-14\\
3.74999999999991	-1.66825707722222e-14\\
3.752	-1.64344187125295e-14\\
3.75200000000003	-1.6434418712526e-14\\
3.75599999999991	-1.59516236749897e-14\\
3.758	-1.57170950130466e-14\\
3.75800000000003	-1.57170950130433e-14\\
3.75999999999997	-1.54870839998064e-14\\
3.76	-1.54870839998032e-14\\
3.76199999999995	-1.52615455525109e-14\\
3.76399999999989	-1.50404354647483e-14\\
3.76599999999997	-1.48237103980698e-14\\
3.766	-1.48237103980668e-14\\
3.76999999999989	-1.44032462646985e-14\\
3.76999999999996	-1.4403246264691e-14\\
3.76999999999999	-1.44032462646881e-14\\
3.77399999999988	-1.39998234868054e-14\\
3.77599999999999	-1.380440324591e-14\\
3.77600000000002	-1.38044032459072e-14\\
3.77999999999991	-1.34220285315939e-14\\
3.77999999999996	-1.34220285315895e-14\\
3.78	-1.3422028531585e-14\\
3.78399999999989	-1.3048075834305e-14\\
3.78599999999997	-1.28641658659008e-14\\
3.786	-1.28641658658982e-14\\
3.78999999999989	-1.25022984483233e-14\\
3.79199999999997	-1.23242700717147e-14\\
3.792	-1.23242700717122e-14\\
3.79599999999989	-1.19738495058769e-14\\
3.79899999999996	-1.17158307017958e-14\\
3.79899999999999	-1.17158307017934e-14\\
3.79999999999997	-1.16307155103961e-14\\
3.8	-1.16307155103936e-14\\
3.80099999999999	-1.15460388882036e-14\\
3.80199999999997	-1.1461796685971e-14\\
3.80399999999993	-1.12945990509233e-14\\
3.80599999999997	-1.11290898338637e-14\\
3.806	-1.11290898338614e-14\\
3.80999999999993	-1.08030072175168e-14\\
3.81099999999996	-1.07224915281335e-14\\
3.81099999999999	-1.07224915281312e-14\\
3.81499999999992	-1.04045676888103e-14\\
3.81499999999998	-1.04045676888059e-14\\
3.81500000000001	-1.04045676888036e-14\\
3.81899999999993	-1.0093174867736e-14\\
3.81999999999997	-1.00163184131473e-14\\
3.82	-1.00163184131451e-14\\
3.82399999999993	-9.71274661367409e-15\\
3.826	-9.563221127408e-15\\
3.82600000000003	-9.56322112740588e-15\\
3.82700000000001	-9.48901061764938e-15\\
3.82700000000003	-9.48901061764728e-15\\
3.82799999999998	-9.41516340912175e-15\\
3.82800000000001	-9.41516340911966e-15\\
3.82899999999997	-9.34167588322551e-15\\
3.82999999999993	-9.26854443906041e-15\\
3.83199999999986	-9.12333547936829e-15\\
3.83399999999998	-8.9795080685137e-15\\
3.83400000000001	-8.97950806851167e-15\\
3.83799999999985	-8.69687414224253e-15\\
3.84	-8.55825940375652e-15\\
3.84000000000003	-8.55825940375456e-15\\
3.84399999999988	-8.28629852652514e-15\\
3.846	-8.15289908236819e-15\\
3.84600000000003	-8.15289908236631e-15\\
3.84999999999988	-7.89113144466315e-15\\
3.84999999999998	-7.89113144465669e-15\\
3.85000000000001	-7.89113144465485e-15\\
3.85399999999985	-7.63590165374497e-15\\
3.85599999999998	-7.51067571944e-15\\
3.85600000000001	-7.51067571943823e-15\\
3.85699999999996	-7.44864920347337e-15\\
3.85699999999999	-7.44864920347161e-15\\
3.85799999999995	-7.38700959624349e-15\\
3.85899999999992	-7.32575387734175e-15\\
3.85999999999997	-7.26487904521555e-15\\
3.86	-7.26487904521383e-15\\
3.86199999999993	-7.14426012833152e-15\\
3.86399999999985	-7.02512920389251e-15\\
3.86599999999997	-6.90746292176486e-15\\
3.866	-6.9074629217632e-15\\
3.86899999999996	-6.73365933031322e-15\\
3.86899999999999	-6.73365933031159e-15\\
3.87199999999995	-6.56339350943366e-15\\
3.87499999999991	-6.39696116990132e-15\\
3.87999999999997	-6.12789417999429e-15\\
3.88	-6.12789417999279e-15\\
3.88499999999998	-5.86894290397029e-15\\
3.88500000000001	-5.86894290396885e-15\\
3.88599999999999	-5.81833840285625e-15\\
3.88600000000002	-5.81833840285482e-15\\
3.88700000000001	-5.76812332246034e-15\\
3.88799999999999	-5.71829520212548e-15\\
3.88999999999996	-5.61979009412465e-15\\
3.89199999999999	-5.52280377146847e-15\\
3.89200000000002	-5.5228037714671e-15\\
3.89599999999996	-5.33353661615564e-15\\
3.89799999999999	-5.24127490318989e-15\\
3.89800000000002	-5.24127490318859e-15\\
3.89999999999997	-5.15057021903449e-15\\
3.9	-5.15057021903321e-15\\
3.90199999999996	-5.06140478533817e-15\\
3.90399999999991	-4.97376112533022e-15\\
3.90599999999997	-4.88762206050934e-15\\
3.906	-4.88762206050813e-15\\
3.90999999999991	-4.71979047420756e-15\\
3.91399999999982	-4.5577784378095e-15\\
3.91499999999996	-4.51816974827689e-15\\
3.91499999999999	-4.51816974827577e-15\\
3.91999999999997	-4.3253953917768e-15\\
3.92	-4.32539539177573e-15\\
3.92499999999999	-4.14123123426453e-15\\
3.92599999999997	-4.10541158897463e-15\\
3.926	-4.10541158897361e-15\\
3.92699999999996	-4.06992553858963e-15\\
3.92699999999999	-4.06992553858863e-15\\
3.92799999999995	-4.03476593523774e-15\\
3.92899999999992	-3.99992564689076e-15\\
3.93099999999984	-3.93119620204696e-15\\
3.93299999999996	-3.86372373281414e-15\\
3.93299999999999	-3.86372373281319e-15\\
3.93699999999984	-3.73249706594185e-15\\
3.93999999999997	-3.63728003351394e-15\\
3.94	-3.63728003351305e-15\\
3.94399999999985	-3.51451477489942e-15\\
3.94399999999996	-3.51451477489599e-15\\
3.94399999999999	-3.51451477489514e-15\\
3.94599999999997	-3.45490106012126e-15\\
3.946	-3.45490106012042e-15\\
3.94799999999999	-3.39645096567762e-15\\
3.94999999999997	-3.3391530351195e-15\\
3.95199999999997	-3.28299603782984e-15\\
3.952	-3.28299603782905e-15\\
3.95499999999998	-3.20086449267831e-15\\
3.95500000000001	-3.20086449267755e-15\\
3.95799999999998	-3.12121654816652e-15\\
3.95999999999998	-3.06948030581747e-15\\
3.96	-3.06948030581675e-15\\
3.96299999999998	-2.9938941716905e-15\\
3.96599999999995	-2.92070036092197e-15\\
3.966	-2.92070036092077e-15\\
3.96600000000003	-2.92070036092009e-15\\
3.96700000000001	-2.89682844199341e-15\\
3.96700000000003	-2.89682844199273e-15\\
3.96800000000001	-2.87321757195077e-15\\
3.96899999999998	-2.84986659375462e-15\\
3.97099999999993	-2.80393974876799e-15\\
3.97299999999996	-2.75903890519198e-15\\
3.97299999999999	-2.75903890519135e-15\\
3.97699999999989	-2.67228021897267e-15\\
3.97899999999996	-2.63040537128516e-15\\
3.97899999999999	-2.63040537128457e-15\\
3.97999999999997	-2.60980227817279e-15\\
3.98	-2.6098022781722e-15\\
3.98099999999999	-2.58936982915896e-15\\
3.98199999999997	-2.56910702303091e-15\\
3.98399999999994	-2.52908637616805e-15\\
3.98599999999997	-2.48973249338103e-15\\
3.986	-2.48973249338048e-15\\
3.98999999999994	-2.41299429540171e-15\\
3.98999999999997	-2.41299429540103e-15\\
3.99000000000001	-2.41299429540037e-15\\
3.99399999999994	-2.33883226238571e-15\\
3.99599999999998	-2.30269905918614e-15\\
3.99600000000001	-2.30269905918563e-15\\
3.99999999999994	-2.23229286699979e-15\\
3.99999999999997	-2.23229286699922e-15\\
4	-2.23229286699867e-15\\
4.00199999999993	-2.19800607812475e-15\\
4.00199999999999	-2.19800607812378e-15\\
4.00399999999992	-2.16432116057834e-15\\
4.00599999999986	-2.1312315119864e-15\\
4.00599999999993	-2.13123151198521e-15\\
4.006	-2.131231511984e-15\\
4.00999999999987	-2.06681219435629e-15\\
4.01199999999995	-2.03546989890434e-15\\
4.012	-2.03546989890346e-15\\
4.01599999999987	-1.97453381523466e-15\\
4.01999999999973	-1.91591911619815e-15\\
4.01999999999995	-1.9159191161951e-15\\
4.02	-1.91591911619428e-15\\
4.02500000000001	-1.84584534691078e-15\\
4.02500000000006	-1.84584534691e-15\\
4.02599999999995	-1.83224962131657e-15\\
4.026	-1.8322496213158e-15\\
4.02699999999996	-1.81879200145041e-15\\
4.02799999999992	-1.80547182785829e-15\\
4.02999999999985	-1.77924121543208e-15\\
4.03099999999993	-1.76632949127905e-15\\
4.03099999999999	-1.76632949127832e-15\\
4.03499999999984	-1.71602512289913e-15\\
4.03699999999994	-1.69166985123904e-15\\
4.03699999999999	-1.69166985123835e-15\\
4.03799999999995	-1.67966141511966e-15\\
4.038	-1.67966141511898e-15\\
4.03899999999996	-1.66772776201919e-15\\
4.03999999999991	-1.65586830717578e-15\\
4.04	-1.65586830717471e-15\\
4.04199999999991	-1.63236967140167e-15\\
4.04399999999982	-1.60916090199095e-15\\
4.046	-1.5862374499312e-15\\
4.04600000000006	-1.58623744993055e-15\\
4.04999999999988	-1.54122858071832e-15\\
4.05399999999969	-1.49730777448647e-15\\
4.05999999999994	-1.43339158073021e-15\\
4.06	-1.43339158072961e-15\\
};
\addplot [color=mycolor2,solid,forget plot]
  table[row sep=crcr]{%
0	0.15313\\
3.15544362088405e-30	0.15313\\
0.000656101980281985	0.153131614989962\\
0.00393661188169191	0.153188143215565\\
0.00999999999999994	0.153505321610126\\
0.01	0.153505321610126\\
0.0199999999999999	0.154633126647887\\
0.02	0.154633126647887\\
0.0289999999999998	0.153421137821655\\
0.029	0.153421137821655\\
0.03	0.152969434437754\\
0.0300000000000002	0.152969434437754\\
0.0349999999999996	0.14975837658894\\
0.035	0.14975837658894\\
0.0399999999999993	0.144956580293631\\
0.04	0.14495658029363\\
0.0449999999999993	0.138999841807976\\
0.0499999999999987	0.132322542411715\\
0.05	0.132322542411713\\
0.0500000000000004	0.132322542411713\\
0.0579999999999996	0.120119344103204\\
0.058	0.120119344103203\\
0.0599999999999996	0.116772645889942\\
0.06	0.116772645889941\\
0.0619999999999995	0.113306311255493\\
0.0639999999999991	0.10971966074666\\
0.0679999999999982	0.102182576405751\\
0.0699999999999991	0.0982306652303514\\
0.07	0.0982306652303496\\
0.0779999999999982	0.0811822039884975\\
0.0779999999999991	0.0811822039884955\\
0.078	0.0811822039884935\\
0.0799999999999991	0.0766747750397316\\
0.08	0.0766747750397296\\
0.0819999999999991	0.0721779853201548\\
0.0839999999999982	0.0676909533788196\\
0.0869999999999991	0.0609767779339659\\
0.087	0.060976777933964\\
0.0899999999999991	0.0542796180244169\\
0.09	0.0542796180244149\\
0.0929999999999991	0.0475965200953159\\
0.0959999999999982	0.0409245367909504\\
0.0999999999999991	0.0320407632769474\\
0.1	0.0320407632769454\\
0.104999999999999	0.021256909365745\\
0.105	0.0212569093657432\\
0.109999999999999	0.0110937472400841\\
0.11	0.0110937472400824\\
0.114999999999999	0.00153882575052224\\
0.116	-0.000300212751249341\\
0.116000000000001	-0.000300212751250965\\
0.12	-0.00741956108284555\\
0.120000000000001	-0.00741956108284709\\
0.124	-0.0141641650924184\\
0.127999999999999	-0.0205393128972452\\
0.129999999999999	-0.0235899139914064\\
0.130000000000001	-0.0235899139914091\\
0.135999999999999	-0.0322009479817859\\
0.136000000000001	-0.0322009479817883\\
0.139999999999998	-0.0374617592732997\\
0.14	-0.0374617592733019\\
0.143999999999997	-0.0423017423766883\\
0.144999999999998	-0.0434464356597455\\
0.145	-0.0434464356597475\\
0.148999999999997	-0.0477656815918623\\
0.149999999999998	-0.0487808758306323\\
0.15	-0.0487808758306341\\
0.153999999999997	-0.0525846762078988\\
0.157999999999995	-0.0559795055490512\\
0.159999999999998	-0.0575244003628621\\
0.16	-0.0575244003628634\\
0.167999999999995	-0.0626813071469279\\
0.17	-0.0637153351896217\\
0.170000000000002	-0.0637153351896226\\
0.173999999999998	-0.0654786415475603\\
0.174	-0.065478641547561\\
0.174999999999998	-0.0658561064227037\\
0.175	-0.0658561064227043\\
0.176	-0.0662082654683606\\
0.177	-0.0665351359408366\\
0.179000000000001	-0.0671130739947607\\
0.179999999999998	-0.0673641698955957\\
0.18	-0.0673641698955961\\
0.184000000000001	-0.0681163386563733\\
0.188000000000002	-0.068465361489994\\
0.189999999999998	-0.0684887910453806\\
0.19	-0.0684887910453806\\
0.193999999999998	-0.0682335093779742\\
0.194	-0.068233509377974\\
0.197999999999998	-0.0676618905053519\\
0.199999999999998	-0.067289840431232\\
0.2	-0.0672898404312317\\
0.202999999999998	-0.0666237633867744\\
0.203	-0.066623763386774\\
0.205999999999998	-0.065827818675382\\
0.208999999999997	-0.0649016552724822\\
0.209999999999998	-0.0645639277109202\\
0.21	-0.0645639277109196\\
0.215999999999997	-0.0622317990176866\\
0.22	-0.0603844073330849\\
0.220000000000002	-0.0603844073330841\\
0.225999999999998	-0.0573535881896798\\
0.23	-0.0552388367632122\\
0.230000000000002	-0.0552388367632112\\
0.231999999999998	-0.054152583272455\\
0.232	-0.054152583272454\\
0.233999999999996	-0.0530467953200896\\
0.235999999999993	-0.0519212561636086\\
0.239999999999986	-0.0496100378866324\\
0.24	-0.0496100378866242\\
0.240000000000002	-0.0496100378866231\\
0.244999999999998	-0.0466059013476809\\
0.245	-0.0466059013476798\\
0.249999999999997	-0.0434704183131218\\
0.249999999999999	-0.0434704183131201\\
0.250000000000002	-0.0434704183131185\\
0.251999999999996	-0.0421785931221946\\
0.252	-0.0421785931221923\\
0.253999999999995	-0.0408776958240218\\
0.255999999999989	-0.0395802807330569\\
0.259999999999979	-0.0369948806153816\\
0.259999999999995	-0.0369948806153714\\
0.26	-0.036994880615368\\
0.260999999999997	-0.0363503103724911\\
0.261	-0.0363503103724888\\
0.262	-0.035706388834962\\
0.263	-0.0350630844434952\\
0.265	-0.033778201039262\\
0.269	-0.0312144548498167\\
0.269999999999996	-0.0305745893556221\\
0.27	-0.0305745893556199\\
0.278	-0.025466199863606\\
0.279999999999996	-0.0241908815627149\\
0.28	-0.0241908815627126\\
0.288	-0.0193397003059672\\
0.289999999999996	-0.018203890310217\\
0.29	-0.018203890310215\\
0.298	-0.0139619118989523\\
0.299999999999996	-0.0129756933729801\\
0.3	-0.0129756933729784\\
0.308	-0.00927737335173871\\
0.309999999999996	-0.00841072954443441\\
0.31	-0.00841072954443289\\
0.314999999999997	-0.0063436318507457\\
0.315	-0.00634363185074428\\
0.319	-0.00479098079271896\\
0.319000000000004	-0.00479098079271763\\
0.319999999999996	-0.00441668878355274\\
0.32	-0.00441668878355142\\
0.321	-0.0040479079892336\\
0.322	-0.00368462033907317\\
0.324	-0.0029744535348474\\
0.328	-0.00161927233280755\\
0.329999999999996	-0.000973992304732705\\
0.33	-0.000973992304731578\\
0.338	0.00139462636162467\\
0.339000000000004	0.00166703773321886\\
0.339000000000007	0.00166703773321982\\
0.339999999999996	0.00193413670471284\\
0.34	0.00193413670471378\\
0.340999999999996	0.00219583707151154\\
0.341999999999993	0.00245215166026715\\
0.343999999999986	0.00294867347822746\\
0.347999999999972	0.00387762278717002\\
0.348	0.00387762278717632\\
0.348000000000004	0.00387762278717711\\
0.349999999999996	0.00431023236252524\\
0.35	0.00431023236252599\\
0.351999999999993	0.00472171298242351\\
0.353999999999986	0.00511214530601302\\
0.357999999999972	0.00583016705766123\\
0.359999999999997	0.00615789722608327\\
0.36	0.00615789722608384\\
0.367999999999972	0.00728068066986741\\
0.369999999999997	0.00751580136352198\\
0.37	0.00751580136352238\\
0.376999999999997	0.0081962003306901\\
0.377	0.00819620033069039\\
0.379999999999997	0.00842021403749647\\
0.38	0.00842021403749671\\
0.382999999999997	0.0086038275331808\\
0.384999999999997	0.00870383269055749\\
0.385	0.00870383269055765\\
0.387999999999997	0.00882028367112367\\
0.39	0.00887556867862432\\
0.390000000000004	0.00887556867862441\\
0.393	0.00892499981108655\\
0.395999999999997	0.00893425335379649\\
0.397	0.00892841073051424\\
0.397000000000004	0.00892841073051421\\
0.399999999999997	0.00889231189782712\\
0.4	0.00889231189782707\\
0.402999999999993	0.00883244997578861\\
0.405999999999986	0.00874879856446565\\
0.405999999999997	0.00874879856446531\\
0.406	0.00874879856446519\\
0.409999999999997	0.00860019220063898\\
0.41	0.00860019220063883\\
0.413999999999997	0.00840911010059374\\
0.417999999999993	0.00817540244614365\\
0.419999999999997	0.0080425080817735\\
0.42	0.00804250808177325\\
0.427999999999993	0.00746024944809923\\
0.429999999999997	0.00730542647395913\\
0.43	0.00730542647395886\\
0.434999999999997	0.0069018380545573\\
0.435	0.00690183805455701\\
0.439999999999997	0.00647423490447306\\
0.44	0.00647423490447275\\
0.444999999999997	0.00602209315428838\\
0.449999999999993	0.00554485887381604\\
0.45	0.00554485887381536\\
0.450000000000004	0.00554485887381501\\
0.454999999999997	0.00504194738033945\\
0.455	0.00504194738033908\\
0.459999999999993	0.00452878709595633\\
0.46	0.00452878709595561\\
0.463999999999997	0.00412200909747846\\
0.464	0.0041220090974781\\
0.467999999999997	0.00371821895045717\\
0.469999999999997	0.00351734530130976\\
0.47	0.0035173453013094\\
0.473999999999997	0.00311744398515352\\
0.477999999999993	0.00271974288983303\\
0.479999999999997	0.00252161991827851\\
0.48	0.00252161991827816\\
0.487999999999993	0.00176906322921124\\
0.489999999999997	0.00159297228841402\\
0.49	0.00159297228841371\\
0.492999999999997	0.00133772944864621\\
0.493	0.00133772944864591\\
0.495999999999997	0.0010930591872577\\
0.498999999999993	0.000858853596700756\\
0.499999999999997	0.000783092564210614\\
0.5	0.000783092564210347\\
0.505999999999993	0.00035249817731223\\
0.509999999999993	8.80107191641623e-05\\
0.51	8.80107191637084e-05\\
0.512999999999993	-9.86520603868523e-05\\
0.513	-9.86520603872826e-05\\
0.515999999999993	-0.000275630403669611\\
0.518999999999986	-0.000443257033022569\\
0.519999999999993	-0.000497067313238426\\
0.52	-0.000497067313238805\\
0.521999999999993	-0.000601605875257211\\
0.522	-0.000601605875257575\\
0.523999999999993	-0.000702052096524967\\
0.524999999999993	-0.000750746763239926\\
0.525	-0.000750746763240269\\
0.526999999999993	-0.000845091139182644\\
0.528999999999986	-0.000935390898807118\\
0.529999999999993	-0.000979029609364891\\
0.53	-0.000979029609365198\\
0.533999999999986	-0.00114355930990597\\
0.537999999999972	-0.00129214376675162\\
0.539999999999993	-0.00136049342133831\\
0.54	-0.00136049342133855\\
0.547999999999972	-0.00159126458725587\\
0.549999999999993	-0.00163814999004629\\
0.55	-0.00163814999004645\\
0.550999999999993	-0.00165997840960948\\
0.551	-0.00165997840960963\\
0.551999999999993	-0.00168073208478697\\
0.552999999999987	-0.0017004120328283\\
0.554999999999973	-0.00173655455222289\\
0.558999999999945	-0.00179599377605356\\
0.559999999999993	-0.00180818169465182\\
0.56	-0.0018081816946519\\
0.567999999999945	-0.0018672948741246\\
0.569999999999993	-0.00187142203084885\\
0.57	-0.00187142203084886\\
0.570999999999993	-0.00187188899666083\\
0.571	-0.00187188899666083\\
0.571999999999994	-0.00187154864393448\\
0.572999999999987	-0.00187065799803123\\
0.574999999999973	-0.00186722562516012\\
0.578999999999945	-0.00185375363405945\\
0.579999999999993	-0.00184900820348367\\
0.58	-0.00184900820348364\\
0.587999999999945	-0.00179116923333827\\
0.589999999999993	-0.00177117527363048\\
0.59	-0.0017711752736304\\
0.594999999999993	-0.00171146441039013\\
0.595	-0.00171146441039004\\
0.599999999999993	-0.00163779874635414\\
0.6	-0.00163779874635403\\
0.604999999999993	-0.00155482170220468\\
0.608999999999993	-0.00148507599380066\\
0.609	-0.00148507599380054\\
0.61	-0.00146716529106077\\
0.610000000000007	-0.00146716529106064\\
0.611000000000007	-0.00144906313683881\\
0.612000000000007	-0.00143076864409049\\
0.614000000000006	-0.00139359904781055\\
0.618000000000005	-0.00131691171295292\\
0.619999999999993	-0.00127737894288338\\
0.62	-0.00127737894288324\\
0.627999999999998	-0.00111886322918495\\
0.629999999999993	-0.00107958202193794\\
0.63	-0.0010795820219378\\
0.637999999999998	-0.000923617064250694\\
0.638000000000005	-0.000923617064250556\\
0.639999999999993	-0.000884877580222641\\
0.64	-0.000884877580222504\\
0.641999999999989	-0.000846223584755479\\
0.643999999999977	-0.000807647500977641\\
0.647999999999954	-0.000730698838134866\\
0.649999999999993	-0.000692311176328506\\
0.65	-0.000692311176328369\\
0.657999999999954	-0.000539162865472987\\
0.657999999999992	-0.000539162865472252\\
0.658000000000005	-0.000539162865472008\\
0.659999999999993	-0.000501276878966948\\
0.66	-0.000501276878966814\\
0.661999999999989	-0.000464077253026777\\
0.663999999999977	-0.000427556695814266\\
0.664999999999993	-0.000409548825882352\\
0.665	-0.000409548825882225\\
0.666999999999993	-0.000374033491422775\\
0.667	-0.00037403349142265\\
0.668999999999993	-0.000339179576849402\\
0.669999999999993	-0.000321998512145828\\
0.67	-0.000321998512145706\\
0.671999999999993	-0.000288123958488455\\
0.673999999999986	-0.000254893986149715\\
0.677999999999971	-0.000190341857767758\\
0.679999999999993	-0.000159007048816407\\
0.68	-0.000159007048816297\\
0.687999999999971	-4.29823066736993e-05\\
0.689999999999993	-1.64751036363088e-05\\
0.69	-1.64751036362164e-05\\
0.695999999999993	5.71328625549299e-05\\
0.696	5.71328625550118e-05\\
0.699999999999993	0.000101324416735105\\
0.7	0.000101324416735181\\
0.703999999999993	0.000141652500755774\\
0.707999999999986	0.000178148733990258\\
0.709999999999993	0.00019496896574381\\
0.71	0.000194968965743868\\
0.716	0.000239757126250045\\
0.716000000000007	0.000239757126250093\\
0.719999999999993	0.000265227313110937\\
0.72	0.000265227313110979\\
0.723999999999986	0.000287581986330444\\
0.724999999999993	0.000292685963081671\\
0.725000000000001	0.000292685963081707\\
0.728999999999986	0.000311170607035509\\
0.729999999999993	0.000315310084454789\\
0.73	0.000315310084454818\\
0.733999999999986	0.000329947345143448\\
0.734999999999993	0.000333127394608294\\
0.735	0.000333127394608316\\
0.738999999999986	0.000343935204506619\\
0.739999999999993	0.000346159721975314\\
0.74	0.000346159721975329\\
0.743999999999986	0.000353525091717178\\
0.747999999999972	0.000358591717285046\\
0.749999999999993	0.000360264312770394\\
0.75	0.000360264312770399\\
0.753999999999993	0.000361889708221085\\
0.754	0.000361889708221086\\
0.757999999999993	0.000361222917688959\\
0.759999999999993	0.000360029859545878\\
0.76	0.000360029859545872\\
0.763999999999993	0.000355923248241808\\
0.767999999999986	0.000349519773041307\\
0.769999999999993	0.000345455212940812\\
0.77	0.000345455212940796\\
0.773999999999993	0.000335596464305072\\
0.774000000000001	0.000335596464305053\\
0.777999999999994	0.000324108495114803\\
0.779999999999993	0.000318007109925459\\
0.78	0.000318007109925437\\
0.782999999999993	0.000308404934350245\\
0.783	0.000308404934350221\\
0.785999999999993	0.000298258859566388\\
0.788999999999986	0.000287564410989037\\
0.79	0.000283876925761341\\
0.790000000000007	0.000283876925761315\\
0.795999999999993	0.000260451182375555\\
0.8	0.000243580934784975\\
0.800000000000007	0.000243580934784944\\
0.804999999999993	0.000221998327731054\\
0.805	0.000221998327731024\\
0.809999999999987	0.000200675108573026\\
0.809999999999997	0.000200675108572982\\
0.810000000000007	0.000200675108572938\\
0.811999999999993	0.000192212587868439\\
0.812	0.000192212587868409\\
0.813999999999987	0.000183785727583908\\
0.815999999999973	0.000175392876010764\\
0.819999999999945	0.000158702625238849\\
0.819999999999987	0.000158702625238675\\
0.82	0.000158702625238617\\
0.827999999999944	0.000125658254151261\\
0.829999999999993	0.000117457735527058\\
0.830000000000001	0.000117457735527029\\
0.831999999999994	0.000109278225116423\\
0.832000000000001	0.000109278225116394\\
0.833999999999994	0.000101171711119848\\
0.835999999999987	9.31901958076729e-05\\
0.839999999999973	7.75959278198192e-05\\
0.839999999999987	7.75959278197659e-05\\
0.84	7.75959278197126e-05\\
0.840999999999993	7.37730698007684e-05\\
0.841000000000001	7.37730698007414e-05\\
0.841999999999994	6.99801185746754e-05\\
0.842999999999987	6.62168882426209e-05\\
0.844999999999973	5.87788541116846e-05\\
0.848999999999945	4.42514194889586e-05\\
0.849999999999993	4.06911526263444e-05\\
0.85	4.06911526263192e-05\\
0.857999999999944	1.3212939114302e-05\\
0.859999999999993	6.61609394092825e-06\\
0.86	6.616093940905e-06\\
0.867999999999944	-1.75889111928683e-05\\
0.869999999999993	-2.30298701204867e-05\\
0.87	-2.30298701205056e-05\\
0.874999999999993	-3.55761162380169e-05\\
0.875000000000001	-3.55761162380336e-05\\
0.879999999999994	-4.66256500617711e-05\\
0.880000000000001	-4.66256500617857e-05\\
0.884999999999994	-5.64335494543234e-05\\
0.889999999999987	-6.52533710112184e-05\\
0.890000000000001	-6.52533710112419e-05\\
0.898999999999994	-7.86728198029097e-05\\
0.899000000000001	-7.8672819802919e-05\\
0.899999999999993	-7.9970804899513e-05\\
0.9	-7.99708048995221e-05\\
0.900999999999993	-8.12304240605533e-05\\
0.901999999999986	-8.24517390330102e-05\\
0.903999999999972	-8.4779693916931e-05\\
0.907999999999944	-8.89784611874238e-05\\
0.909999999999993	-9.08500965806667e-05\\
0.91	-9.08500965806731e-05\\
0.917999999999944	-9.68265304824112e-05\\
0.918999999999994	-9.74043092332666e-05\\
0.919000000000001	-9.74043092332706e-05\\
0.919999999999993	-9.79487953765544e-05\\
0.92	-9.79487953765582e-05\\
0.920999999999993	-9.84642346442333e-05\\
0.921999999999986	-9.89506523009617e-05\\
0.923999999999972	-9.98365166989302e-05\\
0.927999999999944	-0.000101260939743647\\
0.927999999999987	-0.000101260939743659\\
0.928000000000001	-0.000101260939743663\\
0.929999999999993	-0.000101799777595601\\
0.93	-0.000101799777595602\\
0.931999999999993	-0.000102223181343474\\
0.933999999999986	-0.00010253123399108\\
0.937999999999972	-0.000102801504905289\\
0.939999999999993	-0.000102763776151242\\
0.940000000000001	-0.000102763776151242\\
0.945000000000001	-0.000102297117257328\\
0.945000000000008	-0.000102297117257327\\
0.949999999999993	-0.00010137340727486\\
0.950000000000001	-0.000101373407274859\\
0.954999999999986	-9.9991514549674e-05\\
0.956999999999994	-9.93101059645939e-05\\
0.957000000000001	-9.93101059645914e-05\\
0.96	-9.81497460884982e-05\\
0.960000000000008	-9.81497460884953e-05\\
0.963000000000007	-9.68230295148333e-05\\
0.966000000000007	-9.53293711394087e-05\\
0.97	-9.30769902074456e-05\\
0.970000000000008	-9.30769902074413e-05\\
0.976000000000007	-8.91358071094371e-05\\
0.976999999999994	-8.84129010303857e-05\\
0.977000000000001	-8.84129010303805e-05\\
0.979999999999993	-8.61850033043572e-05\\
0.980000000000001	-8.61850033043519e-05\\
0.982999999999993	-8.38949329104704e-05\\
0.985999999999986	-8.15416798840167e-05\\
0.985999999999993	-8.1541679884011e-05\\
0.986000000000001	-8.15416798840052e-05\\
0.989999999999993	-7.83039272440832e-05\\
0.990000000000001	-7.83039272440774e-05\\
0.993999999999993	-7.49494630832039e-05\\
0.997999999999986	-7.14756573250192e-05\\
0.999999999999993	-6.96931538977346e-05\\
1	-6.96931538977282e-05\\
1.00799999999999	-6.26151671080816e-05\\
1.00999999999999	-6.08797032664715e-05\\
1.01	-6.08797032664592e-05\\
1.01499999999999	-5.65968697565506e-05\\
1.015	-5.65968697565386e-05\\
1.01999999999999	-5.23896018117568e-05\\
1.02	-5.2389601811745e-05\\
1.02499999999999	-4.82527449740554e-05\\
1.02999999999997	-4.4181231076488e-05\\
1.03	-4.41812310764652e-05\\
1.03499999999999	-4.01700719013518e-05\\
1.035	-4.01700719013405e-05\\
1.03999999999999	-3.62684475691184e-05\\
1.04	-3.62684475691075e-05\\
1.04399999999999	-3.32617421224035e-05\\
1.044	-3.3261742122393e-05\\
1.04799999999999	-3.03543387881424e-05\\
1.04999999999999	-2.89371600594872e-05\\
1.05	-2.89371600594772e-05\\
1.05399999999999	-2.6174459570689e-05\\
1.05799999999997	-2.3505504400891e-05\\
1.05999999999999	-2.22055245280643e-05\\
1.06	-2.22055245280551e-05\\
1.06799999999997	-1.73794045543739e-05\\
1.06999999999999	-1.6274496497497e-05\\
1.07	-1.62744964974893e-05\\
1.07299999999999	-1.4692450908133e-05\\
1.073	-1.46924509081257e-05\\
1.07599999999999	-1.32001679003735e-05\\
1.07899999999997	-1.1796989327222e-05\\
1.07999999999999	-1.13489560356659e-05\\
1.08	-1.13489560356596e-05\\
1.08499999999999	-9.25550920768639e-06\\
1.085	-9.25550920768079e-06\\
1.08999999999999	-7.40475816999422e-06\\
1.09	-7.40475816998923e-06\\
1.09299999999999	-6.40984600255705e-06\\
1.093	-6.40984600255254e-06\\
1.09599999999999	-5.48838727698545e-06\\
1.09899999999997	-4.62731926157693e-06\\
1.09999999999999	-4.35365034749533e-06\\
1.1	-4.35365034749149e-06\\
1.10199999999999	-3.82626220748261e-06\\
1.102	-3.82626220747896e-06\\
1.10399999999999	-3.32538805611812e-06\\
1.10599999999997	-2.85092971763011e-06\\
1.10999999999994	-1.98089365723358e-06\\
1.10999999999999	-1.98089365722464e-06\\
1.11	-1.98089365722173e-06\\
1.11799999999994	-5.5406423745634e-07\\
1.11999999999999	-2.62200488842525e-07\\
1.12	-2.62200488840543e-07\\
1.12799999999994	6.76533478320239e-07\\
1.12999999999999	8.56074829204535e-07\\
1.13	8.56074829205733e-07\\
1.13099999999999	9.37600973072676e-07\\
1.131	9.37600973073795e-07\\
1.13199999999999	1.01363617794279e-06\\
1.13299999999999	1.0841841708413e-06\\
1.13499999999997	1.20883207948458e-06\\
1.13899999999994	1.39241546034296e-06\\
1.13999999999999	1.42463472509868e-06\\
1.14	1.4246347250991e-06\\
1.14799999999994	1.48565058165763e-06\\
1.14999999999999	1.44626628008323e-06\\
1.15	1.44626628008287e-06\\
1.15099999999999	1.41837565880487e-06\\
1.151	1.41837565880444e-06\\
1.15199999999999	1.3867022773953e-06\\
1.15299999999999	1.35292920358639e-06\\
1.15499999999997	1.27907725625326e-06\\
1.155	1.27907725625216e-06\\
1.15899999999997	1.10609733402214e-06\\
1.15999999999999	1.05757430409747e-06\\
1.16	1.05757430409677e-06\\
1.16399999999997	8.42298654923144e-07\\
1.16799999999994	5.92999029280109e-07\\
1.16999999999999	4.55530187435144e-07\\
1.17	4.55530187434136e-07\\
1.17799999999994	-1.80453733562044e-07\\
1.17999999999999	-3.61132671177989e-07\\
1.18	-3.61132671179304e-07\\
1.18799999999994	-1.09879527629636e-06\\
1.18899999999999	-1.19071085292659e-06\\
1.189	-1.19071085292789e-06\\
1.18999999999999	-1.28257651514401e-06\\
1.19	-1.28257651514532e-06\\
1.19099999999999	-1.37439676339531e-06\\
1.19199999999999	-1.46617609788438e-06\\
1.19399999999997	-1.64963001261835e-06\\
1.19799999999994	-2.01624464678567e-06\\
1.19999999999999	-2.19947722650923e-06\\
1.2	-2.19947722651054e-06\\
1.20799999999994	-2.9024888549027e-06\\
1.20999999999999	-3.06904364077629e-06\\
1.21	-3.06904364077746e-06\\
1.21799999999994	-3.69944980899344e-06\\
1.21799999999997	-3.69944980899553e-06\\
1.218	-3.69944980899762e-06\\
1.21999999999999	-3.84825256128828e-06\\
1.22	-3.84825256128933e-06\\
1.22199999999999	-3.99359515538748e-06\\
1.22399999999997	-4.13550608128081e-06\\
1.22499999999999	-4.2051833870278e-06\\
1.225	-4.20518338702879e-06\\
1.22899999999997	-4.4754507430482e-06\\
1.22999999999999	-4.54092366930462e-06\\
1.23	-4.54092366930555e-06\\
1.23399999999997	-4.79453597266536e-06\\
1.23799999999995	-5.03504890921509e-06\\
1.23799999999997	-5.03504890921666e-06\\
1.238	-5.03504890921822e-06\\
1.23999999999999	-5.14945961507694e-06\\
1.24	-5.14945961507773e-06\\
1.24199999999999	-5.25867954708842e-06\\
1.24399999999997	-5.36273011469777e-06\\
1.24699999999999	-5.50915774423749e-06\\
1.247	-5.50915774423816e-06\\
1.24999999999999	-5.64406453134481e-06\\
1.25	-5.64406453134542e-06\\
1.25299999999999	-5.76750997207405e-06\\
1.25599999999997	-5.87954850789479e-06\\
1.25999999999999	-6.01127382787093e-06\\
1.26	-6.01127382787136e-06\\
1.26599999999997	-6.1705282962604e-06\\
1.26999999999999	-6.25095447543705e-06\\
1.27	-6.2509544754373e-06\\
1.27599999999997	-6.33313620029345e-06\\
1.276	-6.33313620029371e-06\\
1.28	-6.3623482996826e-06\\
1.28000000000001	-6.36234829968267e-06\\
1.28400000000002	-6.37112888452449e-06\\
1.28800000000002	-6.35948483939396e-06\\
1.29	-6.34600135333367e-06\\
1.29000000000001	-6.34600135333355e-06\\
1.29499999999999	-6.28992449972502e-06\\
1.295	-6.28992449972481e-06\\
1.29599999999999	-6.2748703198453e-06\\
1.296	-6.27487031984507e-06\\
1.29699999999999	-6.25876327065578e-06\\
1.29799999999999	-6.24183087144015e-06\\
1.29999999999997	-6.2054866636831e-06\\
1.3	-6.20548666368257e-06\\
1.30399999999997	-6.12285482664599e-06\\
1.30499999999999	-6.10011947573877e-06\\
1.305	-6.10011947573844e-06\\
1.30899999999997	-6.00083508307987e-06\\
1.30999999999999	-5.97392215645516e-06\\
1.31	-5.97392215645477e-06\\
1.31399999999997	-5.85786357237691e-06\\
1.31799999999994	-5.72828425311697e-06\\
1.31999999999999	-5.65839298417165e-06\\
1.32	-5.65839298417114e-06\\
1.32799999999994	-5.36284082318505e-06\\
1.32999999999999	-5.28603026386265e-06\\
1.33	-5.2860302638621e-06\\
1.33399999999999	-5.12879754221041e-06\\
1.334	-5.12879754220984e-06\\
1.33799999999999	-4.96664622386803e-06\\
1.33999999999999	-4.88368651402257e-06\\
1.34	-4.88368651402198e-06\\
1.34399999999999	-4.71391768815429e-06\\
1.34799999999997	-4.53890497709274e-06\\
1.35	-4.4493894459994e-06\\
1.35000000000001	-4.44938944599876e-06\\
1.354	-4.26625230570814e-06\\
1.35400000000001	-4.26625230570748e-06\\
1.358	-4.08058893535908e-06\\
1.35999999999999	-3.98791515149557e-06\\
1.36	-3.98791515149491e-06\\
1.36299999999999	-3.84905075750649e-06\\
1.363	-3.84905075750583e-06\\
1.36499999999999	-3.75654555418067e-06\\
1.365	-3.75654555418001e-06\\
1.36699999999999	-3.6640760436364e-06\\
1.36899999999997	-3.57162410154498e-06\\
1.36999999999999	-3.52539905581486e-06\\
1.37	-3.52539905581421e-06\\
1.37399999999997	-3.34045218885812e-06\\
1.37799999999995	-3.15532186147518e-06\\
1.37999999999999	-3.06264254762535e-06\\
1.38	-3.06264254762469e-06\\
1.38799999999994	-2.7060455882659e-06\\
1.38999999999999	-2.6212954073089e-06\\
1.39	-2.6212954073083e-06\\
1.39199999999999	-2.53827111357534e-06\\
1.392	-2.53827111357475e-06\\
1.39399999999998	-2.45695643529443e-06\\
1.39599999999997	-2.37733543321935e-06\\
1.39999999999994	-2.22311236270801e-06\\
1.39999999999999	-2.22311236270619e-06\\
1.4	-2.22311236270566e-06\\
1.40799999999994	-1.9343255443589e-06\\
1.41	-1.86614152021351e-06\\
1.41000000000001	-1.86614152021303e-06\\
1.41200000000001	-1.79953537070737e-06\\
1.41200000000003	-1.7995353707069e-06\\
1.41400000000003	-1.73441686130285e-06\\
1.41600000000003	-1.67069604696641e-06\\
1.41999999999999	-1.54739781706878e-06\\
1.42	-1.54739781706835e-06\\
1.42099999999999	-1.51742763234977e-06\\
1.421	-1.51742763234934e-06\\
1.42199999999999	-1.48779623462973e-06\\
1.42299999999999	-1.45850217193656e-06\\
1.42499999999997	-1.40092032642897e-06\\
1.42899999999994	-1.28974259636667e-06\\
1.42999999999999	-1.26277060381741e-06\\
1.43	-1.26277060381703e-06\\
1.43499999999999	-1.13277939823493e-06\\
1.435	-1.13277939823458e-06\\
1.43999999999998	-1.01078745577652e-06\\
1.44	-1.01078745577613e-06\\
1.44499999999999	-8.98533596392363e-07\\
1.44999999999997	-7.97768571382064e-07\\
1.44999999999998	-7.97768571381781e-07\\
1.45	-7.97768571381499e-07\\
1.45999999999997	-6.3022514976888e-07\\
1.45999999999999	-6.30225149768629e-07\\
1.46	-6.30225149768422e-07\\
1.46999999999997	-4.9679417040096e-07\\
1.46999999999998	-4.9679417040077e-07\\
1.47	-4.96794170400578e-07\\
1.47899999999998	-3.96316324462389e-07\\
1.479	-3.96316324462245e-07\\
1.47999999999999	-3.86279832212133e-07\\
1.48	-3.86279832211992e-07\\
1.48099999999999	-3.76466998722287e-07\\
1.48199999999999	-3.66877342974588e-07\\
1.48399999999997	-3.48365696037748e-07\\
1.48799999999994	-3.14000589594853e-07\\
1.49	-2.98140394143445e-07\\
1.49000000000001	-2.98140394143335e-07\\
1.49799999999996	-2.43444799507422e-07\\
1.49900000000001	-2.37586111879847e-07\\
1.49900000000003	-2.37586111879765e-07\\
1.49999999999999	-2.31907202833836e-07\\
1.5	-2.31907202833756e-07\\
1.50099999999999	-2.26371200185386e-07\\
1.50199999999999	-2.20977832597605e-07\\
1.50399999999997	-2.10617952475648e-07\\
1.50499999999999	-2.0565093230025e-07\\
1.505	-2.0565093230018e-07\\
1.50799999999998	-1.91598651511913e-07\\
1.508	-1.9159865151185e-07\\
1.50999999999999	-1.82935502833228e-07\\
1.51	-1.82935502833169e-07\\
1.51199999999999	-1.74834387851358e-07\\
1.51399999999998	-1.67293718720845e-07\\
1.51799999999995	-1.53887915592413e-07\\
1.51999999999999	-1.48020153900813e-07\\
1.52	-1.48020153900774e-07\\
1.52799999999995	-1.28487163241855e-07\\
1.52999999999999	-1.24483419875126e-07\\
1.53	-1.24483419875099e-07\\
1.53699999999998	-1.13223053106369e-07\\
1.537	-1.1322305310635e-07\\
1.53999999999999	-1.09703342333946e-07\\
1.54	-1.09703342333931e-07\\
1.54299999999999	-1.0696491965598e-07\\
1.54599999999997	-1.05006577338074e-07\\
1.54999999999999	-1.03607469333625e-07\\
1.55	-1.03607469333622e-07\\
1.55599999999997	-1.04104167722877e-07\\
1.55699999999998	-1.04489754089401e-07\\
1.557	-1.04489754089407e-07\\
1.55999999999999	-1.05918564554988e-07\\
1.56	-1.05918564554995e-07\\
1.56299999999999	-1.07632285313732e-07\\
1.56599999999998	-1.09631672145755e-07\\
1.566	-1.09631672145771e-07\\
1.56999999999999	-1.12743432932461e-07\\
1.57	-1.12743432932473e-07\\
1.57399999999999	-1.1636706643128e-07\\
1.57499999999999	-1.17353279012601e-07\\
1.575	-1.17353279012615e-07\\
1.57899999999999	-1.21620795960733e-07\\
1.57999999999999	-1.22768603240317e-07\\
1.58	-1.22768603240333e-07\\
1.58399999999999	-1.27386149770507e-07\\
1.58799999999998	-1.31928885437143e-07\\
1.58999999999999	-1.34173314487438e-07\\
1.59	-1.34173314487454e-07\\
1.59499999999998	-1.39709340738233e-07\\
1.595	-1.39709340738249e-07\\
1.59999999999998	-1.45143568131888e-07\\
1.6	-1.45143568131907e-07\\
1.60499999999998	-1.50482654313671e-07\\
1.60999999999997	-1.55733140331824e-07\\
1.60999999999998	-1.5573314033184e-07\\
1.61	-1.55733140331856e-07\\
1.61499999999998	-1.60901458822611e-07\\
1.615	-1.60901458822626e-07\\
1.61999999999998	-1.65769076333418e-07\\
1.62	-1.65769076333434e-07\\
1.624	-1.69288791799089e-07\\
1.62400000000001	-1.69288791799101e-07\\
1.62800000000001	-1.72478733572001e-07\\
1.62999999999999	-1.73950830956978e-07\\
1.63	-1.73950830956988e-07\\
1.634	-1.76650721455137e-07\\
1.638	-1.79026610391697e-07\\
1.63999999999999	-1.80093646306597e-07\\
1.64	-1.80093646306605e-07\\
1.64499999999998	-1.8232270444001e-07\\
1.645	-1.82322704440016e-07\\
1.64999999999998	-1.83877353584522e-07\\
1.65	-1.83877353584526e-07\\
1.65299999999998	-1.84487253395652e-07\\
1.653	-1.84487253395655e-07\\
1.65599999999998	-1.84855336441705e-07\\
1.65899999999997	-1.84981765054762e-07\\
1.65999999999999	-1.84970219647453e-07\\
1.66	-1.84970219647453e-07\\
1.66599999999997	-1.84337111607414e-07\\
1.67	-1.83377601747931e-07\\
1.67000000000002	-1.83377601747926e-07\\
1.673	-1.82375324610327e-07\\
1.67300000000001	-1.82375324610322e-07\\
1.676	-1.81165712069055e-07\\
1.67899999999998	-1.79783643418954e-07\\
1.67999999999998	-1.7928452713522e-07\\
1.68	-1.79284527135213e-07\\
1.68199999999998	-1.78228509139839e-07\\
1.682	-1.78228509139831e-07\\
1.68399999999998	-1.77095269852024e-07\\
1.68599999999997	-1.75884587149885e-07\\
1.68999999999994	-1.73229927086151e-07\\
1.68999999999998	-1.73229927086119e-07\\
1.69	-1.73229927086109e-07\\
1.69799999999994	-1.66979821909786e-07\\
1.69999999999998	-1.65219534680676e-07\\
1.7	-1.65219534680663e-07\\
1.70799999999994	-1.57776084664207e-07\\
1.70999999999998	-1.55837760524645e-07\\
1.71	-1.55837760524631e-07\\
1.711	-1.54856721705546e-07\\
1.71100000000001	-1.54856721705532e-07\\
1.71200000000001	-1.53867701096435e-07\\
1.71300000000001	-1.52870650225118e-07\\
1.715	-1.50852261877192e-07\\
1.71500000000001	-1.50852261877177e-07\\
1.71700000000001	-1.48801161051131e-07\\
1.71900000000001	-1.46716945724651e-07\\
1.71999999999999	-1.4566229271829e-07\\
1.72	-1.45662292718275e-07\\
1.724	-1.41358834412461e-07\\
1.72799999999999	-1.36917076449065e-07\\
1.72999999999999	-1.34643251072792e-07\\
1.73	-1.34643251072776e-07\\
1.731	-1.3349290795556e-07\\
1.73100000000001	-1.33492907955543e-07\\
1.73200000000001	-1.32337181458921e-07\\
1.73300000000001	-1.31179660180316e-07\\
1.735	-1.28859006311337e-07\\
1.73899999999999	-1.24193659328994e-07\\
1.73999999999998	-1.23021981162675e-07\\
1.74	-1.23021981162659e-07\\
1.74799999999998	-1.13562665176104e-07\\
1.74999999999998	-1.11171852115688e-07\\
1.75	-1.11171852115671e-07\\
1.75799999999998	-1.01490603198705e-07\\
1.75999999999998	-9.90384198341185e-08\\
1.76	-9.9038419834101e-08\\
1.76799999999998	-8.95404974600331e-08\\
1.76899999999998	-8.83999909393547e-08\\
1.769	-8.83999909393386e-08\\
1.76999999999998	-8.72696804720876e-08\\
1.77	-8.72696804720716e-08\\
1.77099999999998	-8.61495106855157e-08\\
1.77199999999997	-8.50394266781538e-08\\
1.77399999999994	-8.2849298915198e-08\\
1.77799999999989	-7.85877121746785e-08\\
1.77999999999998	-7.65154178804714e-08\\
1.78	-7.65154178804568e-08\\
1.78499999999998	-7.14806909976541e-08\\
1.785	-7.14806909976401e-08\\
1.78999999999998	-6.66374359191699e-08\\
1.79	-6.66374359191564e-08\\
1.79499999999998	-6.197971901615e-08\\
1.798	-5.92717641358894e-08\\
1.79800000000001	-5.92717641358767e-08\\
1.79999999999999	-5.75018339961457e-08\\
1.8	-5.75018339961332e-08\\
1.80199999999998	-5.57597968155233e-08\\
1.80399999999995	-5.40453111454558e-08\\
1.8079999999999	-5.06976554925649e-08\\
1.80999999999999	-4.90638293335076e-08\\
1.81	-4.90638293334961e-08\\
1.8179999999999	-4.27877900178846e-08\\
1.818	-4.27877900178101e-08\\
1.81800000000001	-4.27877900177993e-08\\
1.81999999999998	-4.12856094182071e-08\\
1.82	-4.12856094181965e-08\\
1.82199999999997	-3.9815244511936e-08\\
1.82399999999994	-3.83764070790161e-08\\
1.82699999999998	-3.62766497061462e-08\\
1.827	-3.62766497061364e-08\\
1.82999999999998	-3.42462700189901e-08\\
1.83	-3.42462700189806e-08\\
1.83299999999999	-3.22843725875455e-08\\
1.83599999999997	-3.03900921828571e-08\\
1.83999999999998	-2.79681273471839e-08\\
1.84	-2.79681273471755e-08\\
1.84599999999997	-2.45867608699581e-08\\
1.84999999999998	-2.25123684931811e-08\\
1.85	-2.2512368493174e-08\\
1.85499999999998	-2.01188400604365e-08\\
1.855	-2.01188400604301e-08\\
1.85599999999998	-1.96664914964406e-08\\
1.856	-1.96664914964342e-08\\
1.85699999999998	-1.92228759199655e-08\\
1.85799999999997	-1.87879715927307e-08\\
1.85999999999994	-1.79442118701757e-08\\
1.85999999999999	-1.7944211870156e-08\\
1.86	-1.79442118701501e-08\\
1.86399999999994	-1.63603182307745e-08\\
1.86799999999988	-1.49135687245113e-08\\
1.86999999999999	-1.42412643576148e-08\\
1.87	-1.42412643576101e-08\\
1.876	-1.24270714305463e-08\\
1.87600000000001	-1.24270714305423e-08\\
1.87999999999999	-1.13672349652606e-08\\
1.88	-1.13672349652571e-08\\
1.88399999999998	-1.04043494973313e-08\\
1.88499999999998	-1.01786865906946e-08\\
1.885	-1.01786865906914e-08\\
1.88899999999997	-9.33593706692618e-09\\
1.89	-9.14017359287341e-09\\
1.89000000000001	-9.14017359287067e-09\\
1.89399999999999	-8.41652753970751e-09\\
1.89799999999996	-7.78751897596076e-09\\
1.89999999999999	-7.50834674272613e-09\\
1.9	-7.50834674272423e-09\\
1.90799999999995	-6.57693036177996e-09\\
1.90999999999999	-6.38710124176791e-09\\
1.91	-6.38710124176663e-09\\
1.914	-6.05880073040123e-09\\
1.91400000000001	-6.05880073040019e-09\\
1.91800000000001	-5.79876827861214e-09\\
1.91999999999999	-5.69428677450082e-09\\
1.92	-5.69428677450014e-09\\
1.924	-5.5362908126955e-09\\
1.92499999999998	-5.50739836329102e-09\\
1.925	-5.50739836329064e-09\\
1.92899999999999	-5.43421225626421e-09\\
1.92999999999999	-5.42650716944095e-09\\
1.93	-5.42650716944087e-09\\
1.934	-5.43804125505915e-09\\
1.93400000000001	-5.43804125505931e-09\\
1.93800000000001	-5.49545090128984e-09\\
1.93999999999999	-5.53315772120561e-09\\
1.94	-5.5331577212059e-09\\
1.94299999999998	-5.60099081995362e-09\\
1.943	-5.60099081995398e-09\\
1.94599999999998	-5.68237839495383e-09\\
1.94899999999996	-5.77735633945086e-09\\
1.94999999999999	-5.81204280320595e-09\\
1.95	-5.81204280320645e-09\\
1.95599999999997	-6.0520695302492e-09\\
1.95999999999998	-6.24262372456705e-09\\
1.96	-6.24262372456777e-09\\
1.96599999999996	-6.54355680601461e-09\\
1.97	-6.74066554893863e-09\\
1.97000000000001	-6.74066554893932e-09\\
1.97199999999998	-6.83822408161124e-09\\
1.972	-6.83822408161193e-09\\
1.97399999999997	-6.93514425979036e-09\\
1.97599999999994	-7.03144508058978e-09\\
1.97999999999988	-7.22226403343146e-09\\
1.98	-7.2222640334371e-09\\
1.98000000000002	-7.22226403343777e-09\\
1.9879999999999	-7.59729248199696e-09\\
1.99	-7.68977997517223e-09\\
1.99000000000002	-7.68977997517289e-09\\
1.99199999999998	-7.78179602196335e-09\\
1.992	-7.781796021964e-09\\
1.99399999999997	-7.87177873495097e-09\\
1.995	-7.91542066031057e-09\\
1.99500000000001	-7.91542066031119e-09\\
1.99699999999998	-8.0000163411671e-09\\
1.99899999999995	-8.08104153993642e-09\\
1.99999999999999	-8.12022019870419e-09\\
2	-8.12022019870474e-09\\
2.00099999999997	-8.15851213789408e-09\\
2.001	-8.15851213789516e-09\\
2.00199999999997	-8.19591923378523e-09\\
2.00299999999995	-8.23244331936913e-09\\
2.00499999999989	-8.30284957518835e-09\\
2.00899999999979	-8.43314182107459e-09\\
2.00999999999997	-8.46353253394796e-09\\
2.01	-8.46353253394881e-09\\
2.01799999999979	-8.67545808918363e-09\\
2.01999999999997	-8.71981997798883e-09\\
2.02	-8.71981997798944e-09\\
2.02799999999979	-8.84983100586474e-09\\
2.02999999999997	-8.8696804738648e-09\\
2.03	-8.86968047386505e-09\\
2.03799999999979	-8.89858192225543e-09\\
2.03999999999997	-8.89319026018961e-09\\
2.04	-8.8931902601895e-09\\
2.04799999999979	-8.83341279999698e-09\\
2.05	-8.80966896079454e-09\\
2.05000000000003	-8.80966896079418e-09\\
2.05799999999981	-8.67935606750398e-09\\
2.05899999999997	-8.65907849213657e-09\\
2.059	-8.65907849213598e-09\\
2.05999999999997	-8.63791153311926e-09\\
2.06	-8.63791153311865e-09\\
2.06099999999998	-8.61585415344539e-09\\
2.06199999999995	-8.59290527213713e-09\\
2.0639999999999	-8.54432846287758e-09\\
2.06499999999997	-8.51869815463033e-09\\
2.065	-8.51869815462958e-09\\
2.0689999999999	-8.40720108069898e-09\\
2.06999999999997	-8.37707602274301e-09\\
2.07	-8.37707602274214e-09\\
2.0739999999999	-8.24752834726524e-09\\
2.0779999999998	-8.10342698563718e-09\\
2.07899999999997	-8.06511464318357e-09\\
2.079	-8.06511464318247e-09\\
2.07999999999997	-8.02609369106118e-09\\
2.08	-8.02609369106007e-09\\
2.08099999999998	-7.98657209600803e-09\\
2.08199999999995	-7.946547921075e-09\\
2.08399999999991	-7.86498396207298e-09\\
2.08799999999981	-7.69573713112357e-09\\
2.088	-7.69573713111541e-09\\
2.08800000000003	-7.69573713111417e-09\\
2.08999999999997	-7.60802108508809e-09\\
2.09	-7.60802108508682e-09\\
2.09199999999995	-7.51822049883055e-09\\
2.0939999999999	-7.42631777001138e-09\\
2.09799999999979	-7.23613341620939e-09\\
2.09999999999997	-7.13781451316666e-09\\
2.1	-7.13781451316525e-09\\
2.10799999999979	-6.73839905826255e-09\\
2.10999999999997	-6.63790127714492e-09\\
2.11	-6.63790127714349e-09\\
2.11699999999997	-6.28369129171821e-09\\
2.117	-6.28369129171677e-09\\
2.11999999999997	-6.13056305712551e-09\\
2.12	-6.13056305712405e-09\\
2.12299999999998	-5.97654307892396e-09\\
2.12599999999995	-5.82156343072524e-09\\
2.13	-5.61331288147189e-09\\
2.13000000000003	-5.6133128814704e-09\\
2.13499999999997	-5.35018068312386e-09\\
2.135	-5.35018068312235e-09\\
2.137	-5.24398467038055e-09\\
2.13700000000002	-5.24398467037904e-09\\
2.13900000000002	-5.13786893007249e-09\\
2.14	-5.08507854456881e-09\\
2.14000000000003	-5.08507854456731e-09\\
2.14200000000003	-4.98001980903943e-09\\
2.14400000000002	-4.87563997509583e-09\\
2.14599999999997	-4.77191858389025e-09\\
2.146	-4.77191858388879e-09\\
2.14999999999999	-4.56636993618967e-09\\
2.15000000000003	-4.56636993618778e-09\\
2.15400000000002	-4.36321270520412e-09\\
2.15800000000002	-4.16228760526055e-09\\
2.15999999999997	-4.06261280812488e-09\\
2.16	-4.06261280812347e-09\\
2.16799999999999	-3.67965964592755e-09\\
2.16999999999997	-3.5884481852551e-09\\
2.17	-3.58844818525382e-09\\
2.17499999999997	-3.36814449287551e-09\\
2.175	-3.36814449287428e-09\\
2.17999999999997	-3.15866215319304e-09\\
2.18	-3.15866215319165e-09\\
2.18499999999997	-2.95974452253307e-09\\
2.18999999999994	-2.77114790192347e-09\\
2.18999999999997	-2.77114790192236e-09\\
2.19	-2.77114790192126e-09\\
2.19499999999997	-2.59264123184379e-09\\
2.195	-2.59264123184281e-09\\
2.19999999999996	-2.42375741152347e-09\\
2.2	-2.42375741152235e-09\\
2.20399999999997	-2.29526052341299e-09\\
2.204	-2.2952605234121e-09\\
2.20499999999997	-2.26404111356492e-09\\
2.205	-2.26404111356404e-09\\
2.20599999999997	-2.23318054257863e-09\\
2.20699999999995	-2.20267729825124e-09\\
2.20899999999989	-2.14273682834608e-09\\
2.20999999999997	-2.11329666564264e-09\\
2.21	-2.11329666564181e-09\\
2.21399999999989	-1.99903637035538e-09\\
2.21799999999979	-1.89031007076117e-09\\
2.21999999999997	-1.83799536119601e-09\\
2.22	-1.83799536119528e-09\\
2.22799999999979	-1.64485683845248e-09\\
2.22999999999997	-1.60072368877682e-09\\
2.23	-1.60072368877621e-09\\
2.23299999999997	-1.53760160983678e-09\\
2.233	-1.5376016098362e-09\\
2.23599999999997	-1.47814808220684e-09\\
2.23899999999994	-1.42233688487599e-09\\
2.24	-1.40453825881486e-09\\
2.24000000000003	-1.40453825881436e-09\\
2.24599999999997	-1.30614009984935e-09\\
2.25	-1.24847737132973e-09\\
2.25000000000003	-1.24847737132935e-09\\
2.25299999999997	-1.20936394495035e-09\\
2.253	-1.20936394494999e-09\\
2.25599999999994	-1.17298613992152e-09\\
2.25899999999988	-1.13853974192294e-09\\
2.26	-1.12748416068537e-09\\
2.26000000000003	-1.12748416068506e-09\\
2.26199999999997	-1.10600955937873e-09\\
2.262	-1.10600955937843e-09\\
2.26399999999994	-1.08538021615823e-09\\
2.26599999999988	-1.06559208749543e-09\\
2.26999999999977	-1.0285241238169e-09\\
2.27	-1.02852412381486e-09\\
2.27000000000003	-1.02852412381461e-09\\
2.27499999999997	-9.86855387042123e-10\\
2.275	-9.86855387041901e-10\\
2.27999999999994	-9.50323987234743e-10\\
2.28	-9.50323987234357e-10\\
2.28499999999995	-9.18312723832458e-10\\
2.28999999999989	-8.90209934374831e-10\\
2.29	-8.90209934374262e-10\\
2.29000000000003	-8.90209934374113e-10\\
2.29099999999997	-8.8505527964348e-10\\
2.291	-8.85055279643336e-10\\
2.29199999999997	-8.80055319074056e-10\\
2.29299999999994	-8.75209807521751e-10\\
2.29499999999989	-8.65981189300331e-10\\
2.29899999999979	-8.49367399754311e-10\\
2.29999999999997	-8.45596805591199e-10\\
2.3	-8.45596805591094e-10\\
2.30799999999979	-8.20917482252762e-10\\
2.31	-8.16265907580453e-10\\
2.31000000000003	-8.16265907580391e-10\\
2.31099999999999	-8.14167160377177e-10\\
2.31100000000002	-8.1416716037712e-10\\
2.31199999999999	-8.12160483730895e-10\\
2.31299999999996	-8.10186629319306e-10\\
2.31499999999989	-8.0633700192655e-10\\
2.31899999999976	-7.9902746548877e-10\\
2.31999999999997	-7.97280748036335e-10\\
2.32	-7.97280748036285e-10\\
2.32799999999974	-7.84455500781731e-10\\
2.32999999999997	-7.81565348503442e-10\\
2.33	-7.81565348503402e-10\\
2.33799999999974	-7.71252380567427e-10\\
2.33999999999997	-7.68983522116406e-10\\
2.34	-7.68983522116374e-10\\
2.345	-7.63373375496646e-10\\
2.34500000000003	-7.63373375496614e-10\\
2.34899999999997	-7.58750199368782e-10\\
2.349	-7.58750199368749e-10\\
2.34999999999997	-7.57575178297802e-10\\
2.35	-7.57575178297768e-10\\
2.35099999999997	-7.56392351827789e-10\\
2.35199999999994	-7.55201661999016e-10\\
2.35399999999988	-7.52796458500768e-10\\
2.35799999999976	-7.47889097892229e-10\\
2.35999999999997	-7.45385978888841e-10\\
2.36	-7.45385978888806e-10\\
2.36799999999976	-7.34208243940452e-10\\
2.36999999999997	-7.31068260580098e-10\\
2.37	-7.31068260580053e-10\\
2.37799999999976	-7.17107662273996e-10\\
2.378	-7.17107662273559e-10\\
2.37800000000002	-7.17107662273506e-10\\
2.37999999999997	-7.1326392606355e-10\\
2.38	-7.13263926063494e-10\\
2.38199999999995	-7.09277290202743e-10\\
2.38399999999989	-7.05146973251639e-10\\
2.38799999999979	-6.96452029536267e-10\\
2.38999999999997	-6.91885698476349e-10\\
2.39	-6.91885698476283e-10\\
2.39799999999979	-6.72140091779193e-10\\
2.398	-6.72140091778647e-10\\
2.39800000000002	-6.72140091778573e-10\\
2.39999999999997	-6.66840877063099e-10\\
2.4	-6.66840877063023e-10\\
2.40199999999994	-6.61413854361904e-10\\
2.40399999999988	-6.55857959893151e-10\\
2.40699999999997	-6.47280094535823e-10\\
2.407	-6.47280094535741e-10\\
2.40999999999997	-6.38406028564223e-10\\
2.41	-6.38406028564137e-10\\
2.41299999999997	-6.29231848372742e-10\\
2.41499999999997	-6.22947005615683e-10\\
2.415	-6.22947005615592e-10\\
2.41799999999997	-6.1326357786174e-10\\
2.41999999999997	-6.06635338349725e-10\\
2.42	-6.0663533834963e-10\\
2.42299999999997	-5.96548548410694e-10\\
2.42599999999994	-5.86378681997985e-10\\
2.42999999999997	-5.72681872515567e-10\\
2.42999999999999	-5.72681872515469e-10\\
2.43599999999994	-5.51821251141923e-10\\
2.43599999999997	-5.51821251141816e-10\\
2.436	-5.5182125114171e-10\\
2.43999999999997	-5.37687997229574e-10\\
2.44	-5.37687997229473e-10\\
2.44399999999998	-5.23360897921508e-10\\
2.44799999999996	-5.08828720016225e-10\\
2.44999999999997	-5.01482172394479e-10\\
2.45	-5.01482172394375e-10\\
2.45599999999999	-4.79103382632827e-10\\
2.45600000000002	-4.79103382632719e-10\\
2.45999999999997	-4.64031673975844e-10\\
2.46	-4.64031673975737e-10\\
2.46399999999995	-4.48997883329356e-10\\
2.46499999999997	-4.45243978508672e-10\\
2.465	-4.45243978508566e-10\\
2.46899999999995	-4.3024101211144e-10\\
2.46999999999997	-4.26492514795603e-10\\
2.47	-4.26492514795497e-10\\
2.47399999999995	-4.11501992137555e-10\\
2.4779999999999	-3.96508201504441e-10\\
2.47999999999997	-3.89006407222263e-10\\
2.48	-3.89006407222157e-10\\
2.48499999999997	-3.70577402053518e-10\\
2.485	-3.70577402053415e-10\\
2.48999999999997	-3.52796280637683e-10\\
2.49	-3.52796280637577e-10\\
2.49399999999997	-3.39023194812885e-10\\
2.494	-3.39023194812789e-10\\
2.49799999999996	-3.25639999158652e-10\\
2.49999999999997	-3.19091319327772e-10\\
2.5	-3.1909131932768e-10\\
2.50399999999997	-3.06273377613631e-10\\
2.50799999999994	-2.9381964853178e-10\\
2.50999999999997	-2.8772630132853e-10\\
2.51	-2.87726301328444e-10\\
2.51399999999997	-2.75800670027453e-10\\
2.514	-2.7580067002737e-10\\
2.51799999999996	-2.64217159744458e-10\\
2.51999999999997	-2.58551526603449e-10\\
2.52	-2.58551526603369e-10\\
2.523	-2.50207600348613e-10\\
2.52300000000002	-2.50207600348535e-10\\
2.52600000000002	-2.42045837337729e-10\\
2.52900000000001	-2.34062638099511e-10\\
2.52999999999997	-2.31440641756315e-10\\
2.53	-2.31440641756241e-10\\
2.53599999999999	-2.1610995534186e-10\\
2.53999999999997	-2.06262549484042e-10\\
2.54	-2.06262549483973e-10\\
2.54599999999999	-1.92278414912268e-10\\
2.54999999999997	-1.83579283731091e-10\\
2.55	-1.83579283731031e-10\\
2.55199999999997	-1.7941421157646e-10\\
2.552	-1.79414211576402e-10\\
2.55399999999996	-1.75371038567265e-10\\
2.55499999999997	-1.73394915656862e-10\\
2.555	-1.73394915656806e-10\\
2.55699999999997	-1.69533112868027e-10\\
2.55899999999994	-1.6579127247684e-10\\
2.55999999999997	-1.63965108072526e-10\\
2.56	-1.63965108072475e-10\\
2.56399999999994	-1.56956751441789e-10\\
2.56799999999987	-1.50418407402933e-10\\
2.56999999999997	-1.47323873772532e-10\\
2.57	-1.47323873772489e-10\\
2.57199999999997	-1.44344949461336e-10\\
2.572	-1.44344949461294e-10\\
2.57399999999996	-1.41457126898646e-10\\
2.57599999999993	-1.38635916272592e-10\\
2.57999999999986	-1.3319113191282e-10\\
2.57999999999997	-1.33191131912673e-10\\
2.58	-1.33191131912635e-10\\
2.58099999999997	-1.31870719910768e-10\\
2.581	-1.31870719910731e-10\\
2.58199999999996	-1.30566490968613e-10\\
2.58299999999993	-1.29278381163306e-10\\
2.58499999999987	-1.26750267279321e-10\\
2.58899999999974	-1.21884744519207e-10\\
2.58999999999997	-1.20707745209955e-10\\
2.59	-1.20707745209922e-10\\
2.59799999999975	-1.11849964530719e-10\\
2.59999999999997	-1.09788596327074e-10\\
2.6	-1.09788596327045e-10\\
2.60799999999975	-1.02214965755642e-10\\
2.60999999999997	-1.00492140559324e-10\\
2.61	-1.004921405593e-10\\
2.61799999999974	-9.42730376338613e-11\\
2.61999999999997	-9.28847876376042e-11\\
2.62	-9.2884787637585e-11\\
2.62499999999997	-8.95708139484621e-11\\
2.625	-8.95708139484437e-11\\
2.62999999999998	-8.6402128505837e-11\\
2.63000000000001	-8.64021285058194e-11\\
2.63499999999998	-8.33748492499063e-11\\
2.63899999999997	-8.10523402844032e-11\\
2.639	-8.1052340284387e-11\\
2.63999999999997	-8.04852673830563e-11\\
2.64	-8.04852673830403e-11\\
2.64099999999998	-7.99235602487572e-11\\
2.64199999999996	-7.93671913565217e-11\\
2.64399999999991	-7.82703595096847e-11\\
2.64799999999982	-7.61395725466454e-11\\
2.64999999999997	-7.51051997733065e-11\\
2.65	-7.5105199773292e-11\\
2.65799999999982	-7.11698292694209e-11\\
2.65899999999997	-7.07002477859909e-11\\
2.659	-7.07002477859776e-11\\
2.66	-7.0234458608791e-11\\
2.66000000000003	-7.02344586087778e-11\\
2.66100000000003	-6.97713449270485e-11\\
2.66200000000004	-6.93108840428852e-11\\
2.66400000000005	-6.83978305454371e-11\\
2.66799999999997	-6.66025729373778e-11\\
2.668	-6.66025729373652e-11\\
2.67	-6.57200169498854e-11\\
2.67000000000003	-6.57200169498729e-11\\
2.67200000000003	-6.48472782099789e-11\\
2.67400000000004	-6.39841856574965e-11\\
2.67800000000005	-6.22862642966853e-11\\
2.67999999999997	-6.14511026778576e-11\\
2.68	-6.14511026778458e-11\\
2.68800000000002	-5.82099459213365e-11\\
2.68999999999997	-5.74244408295731e-11\\
2.69	-5.7424440829562e-11\\
2.69499999999997	-5.5502358348717e-11\\
2.695	-5.55023583487062e-11\\
2.69699999999997	-5.47498018827195e-11\\
2.697	-5.47498018827088e-11\\
2.69899999999996	-5.4006324014263e-11\\
2.69999999999997	-5.36379439460186e-11\\
2.7	-5.36379439460082e-11\\
2.70199999999997	-5.29078112872098e-11\\
2.70399999999993	-5.21863961883396e-11\\
2.70799999999987	-5.07691547526691e-11\\
2.71	-5.00730506214906e-11\\
2.71000000000003	-5.00730506214808e-11\\
2.71699999999999	-4.77000804432976e-11\\
2.71700000000002	-4.77000804432882e-11\\
2.72	-4.67109113776037e-11\\
2.72000000000003	-4.67109113775944e-11\\
2.72300000000001	-4.57358755665811e-11\\
2.72599999999999	-4.47745429961321e-11\\
2.72600000000002	-4.4774542996123e-11\\
2.73	-4.35133504565056e-11\\
2.73000000000003	-4.35133504564967e-11\\
2.73400000000002	-4.22747773583748e-11\\
2.738	-4.10578525948603e-11\\
2.74	-4.045720952113e-11\\
2.74000000000003	-4.04572095211215e-11\\
2.748	-3.81342636547448e-11\\
2.75	-3.75747537730609e-11\\
2.75000000000003	-3.7574753773053e-11\\
2.75499999999997	-3.62119192997308e-11\\
2.755	-3.62119192997232e-11\\
2.75999999999993	-3.48990998095095e-11\\
2.75999999999997	-3.48990998095006e-11\\
2.76	-3.48990998094917e-11\\
2.76499999999994	-3.36346869247027e-11\\
2.76499999999997	-3.3634686924695e-11\\
2.765	-3.36346869246872e-11\\
2.76999999999994	-3.24171315814151e-11\\
2.76999999999997	-3.24171315814076e-11\\
2.77	-3.24171315814001e-11\\
2.77499999999994	-3.12449421220853e-11\\
2.775	-3.12449421220724e-11\\
2.77999999999993	-3.01104167764235e-11\\
2.77999999999997	-3.01104167764156e-11\\
2.78	-3.01104167764079e-11\\
2.78399999999997	-2.92244680781259e-11\\
2.784	-2.92244680781196e-11\\
2.78799999999996	-2.83570294766106e-11\\
2.78999999999997	-2.79300383731651e-11\\
2.79	-2.79300383731591e-11\\
2.79399999999997	-2.70890940875259e-11\\
2.79799999999993	-2.62649856558159e-11\\
2.79999999999997	-2.58590424576445e-11\\
2.8	-2.58590424576387e-11\\
2.80799999999993	-2.43004732303696e-11\\
2.80999999999997	-2.39283939891962e-11\\
2.81	-2.3928393989191e-11\\
2.81299999999997	-2.33831476800482e-11\\
2.813	-2.33831476800431e-11\\
2.81599999999996	-2.28531352002142e-11\\
2.81899999999993	-2.23381227974411e-11\\
2.81999999999997	-2.21697459008016e-11\\
2.82	-2.21697459007969e-11\\
2.82599999999993	-2.119349716317e-11\\
2.83	-2.05744773076518e-11\\
2.83000000000003	-2.05744773076475e-11\\
2.83299999999999	-2.0126566364595e-11\\
2.83300000000002	-2.01265663645908e-11\\
2.835	-1.98342705731522e-11\\
2.83500000000003	-1.98342705731481e-11\\
2.83700000000001	-1.95453180762722e-11\\
2.83899999999999	-1.92596522383142e-11\\
2.84	-1.91180342941144e-11\\
2.84000000000003	-1.91180342941104e-11\\
2.84199999999997	-1.88371936587488e-11\\
2.842	-1.88371936587448e-11\\
2.84399999999993	-1.85595008878352e-11\\
2.84599999999987	-1.82849015525184e-11\\
2.84999999999974	-1.77447684950446e-11\\
2.85	-1.77447684950099e-11\\
2.85000000000003	-1.77447684950061e-11\\
2.85799999999978	-1.6699294759783e-11\\
2.85999999999997	-1.64448750834372e-11\\
2.86	-1.64448750834336e-11\\
2.86799999999975	-1.54723589215127e-11\\
2.86999999999997	-1.52414609397052e-11\\
2.87	-1.52414609397019e-11\\
2.87099999999997	-1.51278149215229e-11\\
2.871	-1.51278149215197e-11\\
2.87199999999996	-1.50153634416087e-11\\
2.87299999999993	-1.49041009885052e-11\\
2.87499999999986	-1.46851214131863e-11\\
2.87899999999973	-1.42611945723838e-11\\
2.87999999999997	-1.41581058824103e-11\\
2.88	-1.41581058824074e-11\\
2.88799999999974	-1.33742765562098e-11\\
2.88999999999997	-1.3189499302409e-11\\
2.89	-1.31894993024064e-11\\
2.89099999999997	-1.30987622646033e-11\\
2.89099999999999	-1.30987622646008e-11\\
2.89199999999996	-1.30088021121309e-11\\
2.89299999999993	-1.29192962071742e-11\\
2.89499999999986	-1.27416296186142e-11\\
2.89899999999973	-1.23915559010911e-11\\
2.89999999999997	-1.23051081446577e-11\\
2.9	-1.23051081446552e-11\\
2.90499999999997	-1.18790818565365e-11\\
2.905	-1.18790818565341e-11\\
2.90999999999998	-1.1463028823498e-11\\
2.91000000000001	-1.14630288234956e-11\\
2.91499999999999	-1.10564393256473e-11\\
2.91999999999997	-1.0658815240005e-11\\
2.92	-1.06588152400023e-11\\
2.92899999999997	-9.98206172895508e-12\\
2.92899999999999	-9.98206172895304e-12\\
2.92999999999997	-9.91065685946008e-12\\
2.93	-9.91065685945806e-12\\
2.93099999999998	-9.83999688459818e-12\\
2.93199999999996	-9.7700783407847e-12\\
2.93399999999992	-9.63245187836882e-12\\
2.93799999999983	-9.36594779472773e-12\\
2.93999999999997	-9.23701793583078e-12\\
2.94	-9.23701793582896e-12\\
2.94799999999983	-8.73541913644989e-12\\
2.94999999999997	-8.6125449942481e-12\\
2.95	-8.61254499424636e-12\\
2.95799999999983	-8.13042803162266e-12\\
2.95799999999997	-8.13042803161459e-12\\
2.95799999999999	-8.13042803161291e-12\\
2.95999999999997	-8.01212557061052e-12\\
2.96	-8.01212557060884e-12\\
2.96199999999998	-7.89466719088311e-12\\
2.96399999999996	-7.77802986845263e-12\\
2.96799999999992	-7.54712710677254e-12\\
2.96999999999997	-7.43281640815155e-12\\
2.97	-7.43281640814993e-12\\
2.97499999999997	-7.15018708595127e-12\\
2.975	-7.15018708594968e-12\\
2.97799999999997	-6.98265701812317e-12\\
2.97799999999999	-6.98265701812159e-12\\
2.97999999999997	-6.87187669117859e-12\\
2.98	-6.87187669117702e-12\\
2.98199999999998	-6.76191486899899e-12\\
2.98399999999996	-6.65274999870539e-12\\
2.98699999999999	-6.49045021449203e-12\\
2.98700000000002	-6.4904502144905e-12\\
2.99	-6.32982388831883e-12\\
2.99000000000003	-6.32982388831731e-12\\
2.99300000000001	-6.17080018140021e-12\\
2.99599999999999	-6.01330896167584e-12\\
2.99999999999997	-5.80558458313325e-12\\
3	-5.80558458313179e-12\\
3.00599999999996	-5.50271053821892e-12\\
3.00999999999997	-5.30823465638378e-12\\
3.01	-5.30823465638242e-12\\
3.01599999999996	-5.02729983304125e-12\\
3.01599999999999	-5.02729983303965e-12\\
3.01600000000002	-5.02729983303834e-12\\
3.01999999999997	-4.84698180030114e-12\\
3.02	-4.84698180029988e-12\\
3.02399999999995	-4.6720756727381e-12\\
3.0279999999999	-4.50244431467254e-12\\
3.03	-4.41956495186024e-12\\
3.03000000000003	-4.41956495185907e-12\\
3.03600000000002	-4.17847793779118e-12\\
3.03600000000005	-4.17847793779007e-12\\
3.03999999999997	-4.0238496450987e-12\\
3.04	-4.02384964509762e-12\\
3.04399999999993	-3.87390934097317e-12\\
3.04499999999997	-3.83714290697643e-12\\
3.04499999999999	-3.83714290697539e-12\\
3.04899999999992	-3.69289769015622e-12\\
3.04999999999997	-3.65753268065879e-12\\
3.05	-3.65753268065779e-12\\
3.05399999999993	-3.51880583271435e-12\\
3.05799999999985	-3.38437099284798e-12\\
3.05999999999997	-3.31873000707718e-12\\
3.06	-3.31873000707625e-12\\
3.06799999999985	-3.07009124933989e-12\\
3.06999999999997	-3.01158397335542e-12\\
3.07	-3.0115839733546e-12\\
3.07399999999997	-2.89887104545984e-12\\
3.07399999999999	-2.89887104545906e-12\\
3.07799999999996	-2.79182044041767e-12\\
3.08	-2.74039210771212e-12\\
3.08000000000003	-2.7403921077114e-12\\
3.08399999999999	-2.64167897951589e-12\\
3.08799999999996	-2.54842652106248e-12\\
3.09	-2.50382502827727e-12\\
3.09000000000003	-2.50382502827665e-12\\
3.09399999999997	-2.41862780558745e-12\\
3.09399999999999	-2.41862780558687e-12\\
3.09799999999993	-2.33771323731456e-12\\
3.09999999999997	-2.29846584891111e-12\\
3.1	-2.29846584891055e-12\\
3.10299999999997	-2.24108544478981e-12\\
3.10299999999999	-2.24108544478928e-12\\
3.10599999999996	-2.18547141918411e-12\\
3.10899999999992	-2.13159924539665e-12\\
3.10999999999997	-2.11402480390739e-12\\
3.11	-2.11402480390689e-12\\
3.11499999999997	-2.02898617686842e-12\\
3.115	-2.02898617686795e-12\\
3.11999999999998	-1.94859462926735e-12\\
3.12	-1.94859462926691e-12\\
3.12499999999998	-1.87306573762659e-12\\
3.12999999999995	-1.8026210358682e-12\\
3.13	-1.80262103586749e-12\\
3.13000000000003	-1.8026210358671e-12\\
3.13199999999997	-1.77584718850467e-12\\
3.13199999999999	-1.7758471885043e-12\\
3.13399999999993	-1.74986768765379e-12\\
3.13599999999986	-1.72467744022089e-12\\
3.13999999999973	-1.67664510986037e-12\\
3.13999999999997	-1.6766451098575e-12\\
3.14	-1.67664510985717e-12\\
3.14799999999974	-1.58984449407669e-12\\
3.14999999999997	-1.57004931873733e-12\\
3.15	-1.57004931873706e-12\\
3.15199999999997	-1.55100815186435e-12\\
3.15199999999999	-1.55100815186408e-12\\
3.15399999999996	-1.53248709376087e-12\\
3.15599999999992	-1.51425234605027e-12\\
3.15999999999985	-1.47862754094865e-12\\
3.15999999999997	-1.47862754094754e-12\\
3.16	-1.47862754094729e-12\\
3.16099999999997	-1.46989476602294e-12\\
3.16099999999999	-1.4698947660227e-12\\
3.16199999999996	-1.46123050088244e-12\\
3.16299999999993	-1.45263432087621e-12\\
3.16499999999986	-1.43564453472322e-12\\
3.16899999999973	-1.40246367158533e-12\\
3.16999999999997	-1.39433247766224e-12\\
3.17	-1.39433247766201e-12\\
3.17799999999974	-1.33158317716709e-12\\
3.17999999999997	-1.31652074635747e-12\\
3.18	-1.31652074635726e-12\\
3.18499999999997	-1.28005385459413e-12\\
3.185	-1.28005385459393e-12\\
3.18999999999998	-1.2453235176218e-12\\
3.19	-1.2453235176216e-12\\
3.19499999999998	-1.21228718616882e-12\\
3.19999999999995	-1.1809043865979e-12\\
3.2	-1.18090438659757e-12\\
3.20999999999995	-1.12008207770571e-12\\
3.21	-1.12008207770537e-12\\
3.21899999999997	-1.06572081402261e-12\\
3.21899999999999	-1.06572081402244e-12\\
3.21999999999997	-1.05969294708663e-12\\
3.22	-1.05969294708646e-12\\
3.22099999999998	-1.0536664512016e-12\\
3.22199999999996	-1.0476410309714e-12\\
3.22399999999992	-1.03559223652601e-12\\
3.22799999999984	-1.01149456638412e-12\\
3.22999999999997	-9.99440967292531e-13\\
3.23	-9.99440967292359e-13\\
3.23799999999984	-9.51139667835712e-13\\
3.23899999999997	-9.45087347845734e-13\\
3.23899999999999	-9.45087347845562e-13\\
3.23999999999997	-9.39034706433486e-13\\
3.24	-9.39034706433314e-13\\
3.24099999999998	-9.32985370404401e-13\\
3.24199999999996	-9.26939043274375e-13\\
3.24399999999992	-9.14854230758944e-13\\
3.24799999999984	-8.90707684673352e-13\\
3.24799999999997	-8.90707684672605e-13\\
3.24799999999999	-8.90707684672433e-13\\
3.24999999999997	-8.78641218127691e-13\\
3.25	-8.7864121812752e-13\\
3.25199999999998	-8.66576135872231e-13\\
3.25399999999996	-8.54510072937357e-13\\
3.25499999999998	-8.48475934671262e-13\\
3.255	-8.4847593467109e-13\\
3.25899999999996	-8.24325103166412e-13\\
3.25999999999997	-8.18282346431815e-13\\
3.26	-8.18282346431643e-13\\
3.26399999999996	-7.94309527844283e-13\\
3.26799999999992	-7.70730699665185e-13\\
3.26999999999997	-7.59083242946496e-13\\
3.27	-7.59083242946332e-13\\
3.27699999999999	-7.19023581964541e-13\\
3.27700000000002	-7.1902358196438e-13\\
3.28	-7.02174744974287e-13\\
3.28000000000003	-7.02174744974128e-13\\
3.28300000000001	-6.85506888448146e-13\\
3.28599999999999	-6.69012661465008e-13\\
3.28999999999998	-6.47277887084846e-13\\
3.29	-6.47277887084693e-13\\
3.29599999999996	-6.15193041698289e-13\\
3.29699999999999	-6.09902550348582e-13\\
3.29700000000002	-6.09902550348431e-13\\
3.29999999999997	-5.94176366059448e-13\\
3.3	-5.941763660593e-13\\
3.30299999999995	-5.78689135927983e-13\\
3.30599999999991	-5.63434029781097e-13\\
3.30599999999995	-5.63434029780873e-13\\
3.30599999999999	-5.63434029780649e-13\\
3.31	-5.43443356136929e-13\\
3.31000000000003	-5.43443356136788e-13\\
3.31400000000004	-5.23837724686929e-13\\
3.31800000000005	-5.04601763577628e-13\\
3.31999999999997	-4.95117685144386e-13\\
3.32	-4.95117685144252e-13\\
3.32499999999998	-4.72054464932063e-13\\
3.325	-4.72054464931935e-13\\
3.32999999999998	-4.50048077013678e-13\\
3.33000000000001	-4.50048077013556e-13\\
3.33499999999998	-4.29071560644823e-13\\
3.33500000000001	-4.29071560644707e-13\\
3.33999999999998	-4.09099216969682e-13\\
3.34000000000001	-4.09099216969572e-13\\
3.34499999999998	-3.90106577369569e-13\\
3.34999999999996	-3.72070373485909e-13\\
3.35000000000001	-3.72070373485722e-13\\
3.35499999999998	-3.54968508229181e-13\\
3.35500000000001	-3.54968508229086e-13\\
3.35999999999998	-3.38686862604296e-13\\
3.36000000000001	-3.38686862604205e-13\\
3.36399999999999	-3.26171589145044e-13\\
3.36400000000002	-3.26171589144956e-13\\
3.36800000000001	-3.14099033008136e-13\\
3.36999999999998	-3.0822580260644e-13\\
3.37000000000001	-3.08225802606358e-13\\
3.37399999999999	-2.96799681066666e-13\\
3.37799999999998	-2.85793247765051e-13\\
3.37999999999997	-2.80444704701789e-13\\
3.38	-2.80444704701714e-13\\
3.38799999999998	-2.60316098716094e-13\\
3.38999999999997	-2.55611568079499e-13\\
3.39	-2.55611568079433e-13\\
3.39299999999997	-2.48796676799765e-13\\
3.39299999999999	-2.48796676799701e-13\\
3.395	-2.44413387155417e-13\\
3.39500000000003	-2.44413387155355e-13\\
3.39700000000004	-2.40157068442025e-13\\
3.39900000000005	-2.36026886407075e-13\\
3.39999999999997	-2.34008842703082e-13\\
3.4	-2.34008842703025e-13\\
3.40400000000002	-2.2624802716638e-13\\
3.40800000000004	-2.18980877665631e-13\\
3.40999999999997	-2.15530632059631e-13\\
3.41	-2.15530632059583e-13\\
3.41299999999996	-2.10582514742852e-13\\
3.41299999999999	-2.10582514742807e-13\\
3.41599999999996	-2.0584043890365e-13\\
3.41899999999992	-2.01237584062374e-13\\
3.41999999999997	-1.9973388450433e-13\\
3.42	-1.99733884504288e-13\\
3.42199999999997	-1.96771920264999e-13\\
3.42199999999999	-1.96771920264957e-13\\
3.42399999999996	-1.93870053317123e-13\\
3.42599999999992	-1.91027714871167e-13\\
3.42999999999985	-1.85519406613776e-13\\
3.42999999999997	-1.85519406613602e-13\\
3.43	-1.85519406613564e-13\\
3.43799999999985	-1.75193387896375e-13\\
3.43999999999997	-1.72752789837843e-13\\
3.44	-1.72752789837809e-13\\
3.44799999999985	-1.63586862631611e-13\\
3.44999999999997	-1.61445200127268e-13\\
3.45	-1.61445200127238e-13\\
3.45099999999999	-1.60396546155083e-13\\
3.45100000000002	-1.60396546155054e-13\\
3.45200000000001	-1.59362608291644e-13\\
3.453	-1.58343335861402e-13\\
3.45499999999999	-1.5634858823837e-13\\
3.45899999999995	-1.52532928478764e-13\\
3.46	-1.51614955560505e-13\\
3.46000000000003	-1.51614955560479e-13\\
3.46499999999998	-1.47238502779404e-13\\
3.465	-1.4723850277938e-13\\
3.46999999999995	-1.43213868248431e-13\\
3.47	-1.4321386824839e-13\\
3.47000000000003	-1.43213868248368e-13\\
3.47099999999999	-1.42450718110256e-13\\
3.47100000000002	-1.42450718110234e-13\\
3.47199999999999	-1.41696435335632e-13\\
3.47299999999996	-1.40946013577412e-13\\
3.47499999999989	-1.39456606214246e-13\\
3.47899999999976	-1.36522518223525e-13\\
3.47999999999996	-1.35798104357712e-13\\
3.47999999999999	-1.35798104357692e-13\\
3.48799999999973	-1.30128440011399e-13\\
3.48999999999997	-1.28744653962377e-13\\
3.48999999999999	-1.28744653962358e-13\\
3.49799999999973	-1.23335891854934e-13\\
3.49999999999999	-1.22013971666649e-13\\
3.50000000000002	-1.2201397166663e-13\\
3.50799999999976	-1.16900347328858e-13\\
3.50899999999999	-1.1628191604106e-13\\
3.50900000000002	-1.16281916041043e-13\\
3.50999999999999	-1.15667998380342e-13\\
3.51000000000002	-1.15667998380325e-13\\
3.51099999999999	-1.15058564271332e-13\\
3.51199999999996	-1.14453583844692e-13\\
3.51399999999991	-1.13256865678601e-13\\
3.51799999999979	-1.10915583635259e-13\\
3.51999999999997	-1.09770560841717e-13\\
3.52	-1.097705608417e-13\\
3.52799999999977	-1.05226588574136e-13\\
3.52999999999997	-1.04090421988884e-13\\
3.53	-1.04090421988868e-13\\
3.53499999999998	-1.0124726622607e-13\\
3.535	-1.01247266226053e-13\\
3.53799999999997	-9.95383797396248e-14\\
3.53799999999999	-9.95383797396086e-14\\
3.53999999999997	-9.83974099467899e-14\\
3.54	-9.83974099467736e-14\\
3.54199999999998	-9.72548095798555e-14\\
3.54399999999996	-9.61103546853944e-14\\
3.54799999999993	-9.38149836390899e-14\\
3.54999999999997	-9.26636175706676e-14\\
3.55	-9.26636175706513e-14\\
3.55799999999993	-8.80283441254689e-14\\
3.55799999999996	-8.80283441254492e-14\\
3.55799999999999	-8.80283441254292e-14\\
3.56	-8.68630413646524e-14\\
3.56000000000003	-8.68630413646358e-14\\
3.56200000000004	-8.56980544032487e-14\\
3.56400000000005	-8.45331548832336e-14\\
3.56699999999996	-8.278547009037e-14\\
3.56699999999999	-8.27854700903534e-14\\
3.56999999999998	-8.10366975476699e-14\\
3.57	-8.10366975476534e-14\\
3.57299999999999	-7.92860660184684e-14\\
3.57599999999997	-7.75328034455906e-14\\
3.57999999999997	-7.51896906422172e-14\\
3.58	-7.51896906422005e-14\\
3.58599999999997	-7.17193774214219e-14\\
3.58999999999997	-6.94593025092553e-14\\
3.59	-6.94593025092394e-14\\
3.59599999999997	-6.61449606649467e-14\\
3.596	-6.61449606649312e-14\\
3.59999999999997	-6.39833866722024e-14\\
3.6	-6.39833866721872e-14\\
3.60399999999998	-6.18582208895072e-14\\
3.60499999999998	-6.1332422354761e-14\\
3.605	-6.13324223547461e-14\\
3.60899999999998	-5.92504269022408e-14\\
3.60999999999998	-5.87351001876604e-14\\
3.61	-5.87351001876458e-14\\
3.61399999999998	-5.66937243605164e-14\\
3.61599999999997	-5.56846453022969e-14\\
3.616	-5.56846453022826e-14\\
3.61999999999997	-5.36957184856008e-14\\
3.62	-5.36957184855868e-14\\
3.62399999999998	-5.1749138894093e-14\\
3.62499999999996	-5.12689311836847e-14\\
3.62499999999999	-5.12689311836711e-14\\
3.62899999999997	-4.93731431560622e-14\\
3.62999999999997	-4.89053407348684e-14\\
3.63	-4.89053407348551e-14\\
3.63399999999998	-4.70580216937147e-14\\
3.63799999999996	-4.52478455834767e-14\\
3.63999999999997	-4.43562414850055e-14\\
3.64	-4.43562414849929e-14\\
3.64799999999996	-4.09484368032976e-14\\
3.65	-4.01398510414467e-14\\
3.65000000000003	-4.01398510414353e-14\\
3.65399999999999	-3.85735967576355e-14\\
3.65400000000002	-3.85735967576246e-14\\
3.65799999999998	-3.70742143129458e-14\\
3.65999999999997	-3.63492306275563e-14\\
3.66	-3.63492306275462e-14\\
3.66399999999997	-3.49479678092178e-14\\
3.66799999999993	-3.36107341327484e-14\\
3.66999999999997	-3.29657986306557e-14\\
3.67	-3.29657986306467e-14\\
3.67399999999999	-3.17226582535347e-14\\
3.67400000000002	-3.17226582535261e-14\\
3.67499999999998	-3.14209850475682e-14\\
3.67500000000001	-3.14209850475597e-14\\
3.67599999999997	-3.11220714937663e-14\\
3.67699999999994	-3.08259029450339e-14\\
3.67899999999988	-3.02417429472744e-14\\
3.67999999999997	-2.99537228739846e-14\\
3.68	-2.99537228739764e-14\\
3.68299999999996	-2.91057333949796e-14\\
3.68299999999999	-2.91057333949717e-14\\
3.68599999999995	-2.82815605309508e-14\\
3.68899999999992	-2.74808408081505e-14\\
3.69	-2.72190847121041e-14\\
3.69000000000003	-2.72190847120967e-14\\
3.69599999999995	-2.57017368703901e-14\\
3.69999999999997	-2.47399255950896e-14\\
3.7	-2.47399255950829e-14\\
3.70599999999993	-2.33838201420579e-14\\
3.70999999999997	-2.25426192169134e-14\\
3.71	-2.25426192169076e-14\\
3.71199999999999	-2.21406285663209e-14\\
3.71200000000002	-2.21406285663153e-14\\
3.71400000000001	-2.17509385957753e-14\\
3.716	-2.13734729227774e-14\\
3.71999999999998	-2.06549209129948e-14\\
3.72000000000001	-2.06549209129899e-14\\
3.72799999999997	-1.93614119348742e-14\\
3.73000000000001	-1.90675771818961e-14\\
3.73000000000004	-1.9067577181892e-14\\
3.73200000000002	-1.87854407793811e-14\\
3.73200000000005	-1.87854407793771e-14\\
3.73400000000003	-1.8511871126943e-14\\
3.73600000000002	-1.82437382939931e-14\\
3.73999999999999	-1.77235739223079e-14\\
3.74000000000002	-1.77235739223043e-14\\
3.74099999999999	-1.75968499985432e-14\\
3.74100000000002	-1.75968499985396e-14\\
3.742	-1.7471440428247e-14\\
3.74299999999997	-1.73473390648628e-14\\
3.74499999999993	-1.71030366986578e-14\\
3.745	-1.71030366986485e-14\\
3.74500000000003	-1.71030366986451e-14\\
3.74899999999994	-1.66298671439814e-14\\
3.75000000000001	-1.6514756513655e-14\\
3.75000000000004	-1.65147565136517e-14\\
3.75399999999995	-1.6066871805887e-14\\
3.75799999999986	-1.56388184369852e-14\\
3.76	-1.54321230213357e-14\\
3.76000000000003	-1.54321230213328e-14\\
3.76799999999985	-1.46571193149398e-14\\
3.76999999999996	-1.44763627569683e-14\\
3.76999999999999	-1.44763627569657e-14\\
3.77799999999981	-1.38042509700107e-14\\
3.77999999999996	-1.36487869620697e-14\\
3.77999999999999	-1.36487869620676e-14\\
3.78799999999981	-1.30459655179011e-14\\
3.78999999999996	-1.28979773919453e-14\\
3.78999999999999	-1.28979773919432e-14\\
3.79799999999981	-1.23160235083557e-14\\
3.79899999999996	-1.22443452539914e-14\\
3.79899999999999	-1.22443452539894e-14\\
3.79999999999997	-1.21728921104518e-14\\
3.8	-1.21728921104497e-14\\
3.80099999999999	-1.21016605772905e-14\\
3.80199999999997	-1.20306471633623e-14\\
3.80399999999993	-1.18892607850304e-14\\
3.80799999999986	-1.1608953048753e-14\\
3.80999999999997	-1.14699767475403e-14\\
3.81	-1.14699767475383e-14\\
3.81499999999998	-1.11257522257637e-14\\
3.81500000000001	-1.11257522257617e-14\\
3.81899999999996	-1.08534583650381e-14\\
3.81899999999999	-1.08534583650362e-14\\
3.81999999999996	-1.07857626871095e-14\\
3.81999999999999	-1.07857626871076e-14\\
3.82099999999996	-1.07181748003756e-14\\
3.82199999999993	-1.06506913929571e-14\\
3.82399999999988	-1.05160247942156e-14\\
3.82799999999976	-1.02478001767703e-14\\
3.82799999999996	-1.02478001767566e-14\\
3.82799999999999	-1.02478001767547e-14\\
3.82999999999996	-1.01141895832226e-14\\
3.82999999999999	-1.01141895832207e-14\\
3.83199999999996	-9.98087853097389e-15\\
3.83399999999993	-9.84784088867086e-15\\
3.83799999999988	-9.58248157915955e-15\\
3.83999999999996	-9.45010789875122e-15\\
3.83999999999999	-9.45010789874934e-15\\
3.84799999999988	-8.92990189270999e-15\\
3.84999999999996	-8.80253586625342e-15\\
3.84999999999999	-8.80253586625162e-15\\
3.85699999999996	-8.36465636589612e-15\\
3.85699999999999	-8.36465636589437e-15\\
3.85999999999997	-8.18057264873354e-15\\
3.86	-8.18057264873181e-15\\
3.86299999999999	-7.99851857419281e-15\\
3.86599999999997	-7.81841385209582e-15\\
3.86999999999997	-7.58116938284241e-15\\
3.87	-7.58116938284073e-15\\
3.87599999999997	-7.23112971146507e-15\\
3.87699999999996	-7.17343281039885e-15\\
3.87699999999999	-7.17343281039721e-15\\
3.87999999999996	-7.00180826876802e-15\\
3.87999999999999	-7.00180826876641e-15\\
3.88299999999996	-6.83253446960107e-15\\
3.88499999999998	-6.72095433076203e-15\\
3.88500000000001	-6.72095433076045e-15\\
3.88599999999996	-6.66553675979805e-15\\
3.88599999999999	-6.66553675979648e-15\\
3.88699999999996	-6.61036389648823e-15\\
3.88799999999993	-6.5554330373375e-15\\
3.88999999999986	-6.44628657674261e-15\\
3.88999999999996	-6.4462865767372e-15\\
3.88999999999999	-6.44628657673566e-15\\
3.89399999999986	-6.23078005234936e-15\\
3.89799999999973	-6.0188482184759e-15\\
3.89999999999996	-5.9141707763168e-15\\
3.89999999999999	-5.91417077631532e-15\\
3.90799999999973	-5.51099085918341e-15\\
3.90999999999996	-5.41444300180267e-15\\
3.90999999999999	-5.41444300180131e-15\\
3.91499999999996	-5.18029747856448e-15\\
3.91499999999999	-5.18029747856318e-15\\
3.91999999999997	-4.95624415164757e-15\\
3.92	-4.95624415164595e-15\\
3.92499999999998	-4.74200852602223e-15\\
3.92999999999995	-4.53732813625725e-15\\
3.93	-4.53732813625515e-15\\
3.93499999999996	-4.34195221782897e-15\\
3.93499999999999	-4.34195221782788e-15\\
3.93999999999995	-4.15474796858554e-15\\
3.93999999999999	-4.15474796858417e-15\\
3.94399999999996	-4.01007046579708e-15\\
3.94399999999999	-4.01007046579607e-15\\
3.94799999999997	-3.86979063406472e-15\\
3.94999999999996	-3.80126534166682e-15\\
3.94999999999999	-3.80126534166586e-15\\
3.95399999999996	-3.6673768454759e-15\\
3.95499999999998	-3.63455390362956e-15\\
3.95500000000001	-3.63455390362863e-15\\
3.95899999999998	-3.50581061324929e-15\\
3.95999999999999	-3.4742540243245e-15\\
3.96000000000002	-3.47425402432361e-15\\
3.96399999999999	-3.35140279055303e-15\\
3.96799999999997	-3.23424195345618e-15\\
3.97000000000002	-3.17776655420574e-15\\
3.97000000000005	-3.17776655420495e-15\\
3.97299999999996	-3.09565350115056e-15\\
3.97299999999999	-3.0956535011498e-15\\
3.97599999999991	-3.0166284164241e-15\\
3.97899999999983	-2.94065644741392e-15\\
3.97999999999997	-2.91600509832809e-15\\
3.98	-2.9160050983274e-15\\
3.98599999999984	-2.77507603277661e-15\\
3.98999999999997	-2.68768650321334e-15\\
3.99	-2.68768650321273e-15\\
3.99299999999996	-2.62554076860017e-15\\
3.99299999999999	-2.62554076859959e-15\\
3.99599999999995	-2.5656165765489e-15\\
3.99899999999991	-2.50722840189015e-15\\
3.99999999999997	-2.48810254418684e-15\\
4	-2.4881025441863e-15\\
4.00199999999993	-2.45035049423439e-15\\
4.00199999999999	-2.45035049423333e-15\\
4.00399999999992	-2.41325851240015e-15\\
4.00599999999986	-2.37681932835731e-15\\
4.00999999999972	-2.30587091140678e-15\\
4.00999999999995	-2.30587091140286e-15\\
4.01	-2.30587091140187e-15\\
4.01799999999973	-2.17150179692448e-15\\
4.01999999999995	-2.13943920734999e-15\\
4.02	-2.13943920734909e-15\\
4.02500000000001	-2.06240260588939e-15\\
4.02500000000006	-2.06240260588854e-15\\
4.02999999999995	-1.99004078607611e-15\\
4.03	-1.99004078607531e-15\\
4.03099999999999	-1.97612149485718e-15\\
4.03100000000005	-1.9761214948564e-15\\
4.03200000000004	-1.96238494879604e-15\\
4.03300000000003	-1.94883047476899e-15\\
4.035	-1.92226509501615e-15\\
4.03899999999996	-1.87129057012396e-15\\
4.03999999999994	-1.85899249893376e-15\\
4.04	-1.85899249893306e-15\\
4.04799999999991	-1.76693133386828e-15\\
4.04999999999994	-1.74565194654486e-15\\
4.05	-1.74565194654426e-15\\
4.05099999999999	-1.73526968961211e-15\\
4.05100000000005	-1.73526968961153e-15\\
4.05200000000004	-1.72500171128585e-15\\
4.05300000000002	-1.71479084345032e-15\\
4.055	-1.69453844068539e-15\\
4.05899999999996	-1.65469713080699e-15\\
4.05999999999994	-1.64487218067451e-15\\
4.06	-1.64487218067395e-15\\
};
\end{axis}
\end{tikzpicture}%}
      \caption{The angular displacement of pendulum $P_2$ as a function of time.
        \texttt{Blue}: $C_2 = 6$ ms, \texttt{Red}: $C_2 = 10$ ms}
      \label{fig:01.5.6_10.2}
    \end{figure}
  \end{minipage}
\end{minipage}
}\\


Figure \ref{fig:01.5.3} shows the schedule calculated for each pendulum with all
jobs having execution time $C_i = 10$ ms.  Figure \ref{fig:01.5.4} illustrates
that the schedule is not feasible by ploting the overall usage of the CPU over
the length of a schedule period, which is at all times $100\%$, indicative of
the excessive processing load demanded.


\noindent\makebox[\textwidth][c]{%
\begin{minipage}{\linewidth}
  \begin{minipage}{0.45\linewidth}
    \begin{figure}[H]\centering
      \scalebox{0.7}{% This file was created by matlab2tikz.
%
%The latest updates can be retrieved from
%  http://www.mathworks.com/matlabcentral/fileexchange/22022-matlab2tikz-matlab2tikz
%where you can also make suggestions and rate matlab2tikz.
%
\definecolor{mycolor1}{rgb}{0.00000,0.44700,0.74100}%
\definecolor{mycolor2}{rgb}{0.92900,0.69400,0.12500}%
\definecolor{mycolor3}{rgb}{0.85000,0.32500,0.09800}%
%
\begin{tikzpicture}

\begin{axis}[%
width=4.133in,
height=0.863in,
at={(0.693in,2.837in)},
scale only axis,
xmin=0,
xmax=2000,
xmajorgrids,
ymin=0,
ymax=1,
ymajorgrids,
axis background/.style={fill=white}
]
\pgfplotsset{max space between ticks=50}
\addplot [color=mycolor1,solid,forget plot]
  table[row sep=crcr]{%
1	0\\
2	1\\
3	1\\
4	1\\
5	1\\
6	0\\
7	0\\
8	1\\
9	1\\
10	1\\
11	1\\
12	0\\
13	0\\
14	0\\
15	0\\
16	1\\
17	1\\
18	1\\
19	1\\
20	0\\
21	0\\
22	0\\
23	0\\
24	1\\
25	1\\
26	1\\
27	1\\
28	1\\
29	0\\
30	0\\
31	0\\
32	0\\
33	0\\
34	1\\
35	1\\
36	1\\
37	1\\
38	1\\
39	1\\
40	0\\
41	0\\
42	0\\
43	0\\
44	1\\
45	1\\
46	1\\
47	1\\
48	0\\
49	0\\
50	0\\
51	0\\
52	0\\
53	1\\
54	1\\
55	1\\
56	1\\
57	0\\
58	0\\
59	0\\
60	0\\
61	1\\
62	1\\
63	1\\
64	1\\
65	1\\
66	1\\
67	0\\
68	0\\
69	0\\
70	0\\
71	1\\
72	1\\
73	1\\
74	0\\
75	0\\
76	0\\
77	0\\
78	0\\
79	0\\
80	0\\
81	0\\
82	0\\
83	1\\
84	1\\
85	1\\
86	1\\
87	0\\
88	0\\
89	0\\
90	0\\
91	0\\
92	1\\
93	1\\
94	1\\
95	1\\
96	1\\
97	1\\
98	0\\
99	0\\
100	0\\
101	1\\
102	1\\
103	1\\
104	0\\
105	0\\
106	0\\
107	0\\
108	0\\
109	0\\
110	0\\
111	1\\
112	1\\
113	1\\
114	1\\
115	1\\
116	0\\
117	0\\
118	0\\
119	0\\
120	0\\
121	0\\
122	0\\
123	1\\
124	1\\
125	1\\
126	1\\
127	1\\
128	1\\
129	1\\
130	1\\
131	0\\
132	0\\
133	0\\
134	1\\
135	1\\
136	1\\
137	0\\
138	0\\
139	0\\
140	1\\
141	1\\
142	1\\
143	0\\
144	0\\
145	0\\
146	0\\
147	0\\
148	0\\
149	1\\
150	1\\
151	1\\
152	1\\
153	1\\
154	1\\
155	0\\
156	0\\
157	0\\
158	0\\
159	0\\
160	1\\
161	1\\
162	1\\
163	1\\
164	1\\
165	1\\
166	1\\
167	1\\
168	0\\
169	0\\
170	0\\
171	0\\
172	0\\
173	1\\
174	1\\
175	1\\
176	0\\
177	0\\
178	0\\
179	0\\
180	1\\
181	1\\
182	1\\
183	1\\
184	1\\
185	1\\
186	0\\
187	0\\
188	0\\
189	0\\
190	0\\
191	0\\
192	1\\
193	1\\
194	1\\
195	1\\
196	1\\
197	1\\
198	0\\
199	0\\
200	0\\
201	0\\
202	1\\
203	1\\
204	1\\
205	0\\
206	0\\
207	0\\
208	0\\
209	1\\
210	1\\
211	1\\
212	1\\
213	0\\
214	0\\
215	0\\
216	0\\
217	1\\
218	1\\
219	1\\
220	1\\
221	1\\
222	0\\
223	0\\
224	0\\
225	0\\
226	1\\
227	1\\
228	1\\
229	0\\
230	0\\
231	0\\
232	0\\
233	0\\
234	0\\
235	1\\
236	1\\
237	1\\
238	0\\
239	0\\
240	0\\
241	0\\
242	0\\
243	0\\
244	1\\
245	1\\
246	1\\
247	1\\
248	1\\
249	1\\
250	1\\
251	1\\
252	1\\
253	0\\
254	0\\
255	0\\
256	0\\
257	1\\
258	1\\
259	1\\
260	0\\
261	0\\
262	0\\
263	0\\
264	0\\
265	0\\
266	0\\
267	0\\
268	1\\
269	1\\
270	1\\
271	0\\
272	0\\
273	0\\
274	0\\
275	0\\
276	0\\
277	0\\
278	0\\
279	1\\
280	1\\
281	1\\
282	0\\
283	0\\
284	0\\
285	0\\
286	1\\
287	1\\
288	1\\
289	1\\
290	1\\
291	0\\
292	0\\
293	0\\
294	0\\
295	0\\
296	0\\
297	1\\
298	1\\
299	1\\
300	0\\
301	0\\
302	0\\
303	0\\
304	1\\
305	1\\
306	1\\
307	1\\
308	1\\
309	0\\
310	0\\
311	0\\
312	0\\
313	0\\
314	1\\
315	1\\
316	1\\
317	1\\
318	1\\
319	1\\
320	1\\
321	1\\
322	1\\
323	0\\
324	0\\
325	0\\
326	0\\
327	0\\
328	1\\
329	1\\
330	1\\
331	0\\
332	0\\
333	0\\
334	0\\
335	1\\
336	1\\
337	1\\
338	1\\
339	0\\
340	0\\
341	0\\
342	0\\
343	1\\
344	1\\
345	1\\
346	1\\
347	1\\
348	1\\
349	0\\
350	0\\
351	0\\
352	0\\
353	0\\
354	0\\
355	1\\
356	1\\
357	1\\
358	1\\
359	0\\
360	0\\
361	0\\
362	0\\
363	0\\
364	1\\
365	1\\
366	1\\
367	1\\
368	0\\
369	0\\
370	0\\
371	0\\
372	0\\
373	1\\
374	1\\
375	1\\
376	1\\
377	1\\
378	1\\
379	0\\
380	0\\
381	0\\
382	1\\
383	1\\
384	1\\
385	1\\
386	1\\
387	0\\
388	0\\
389	0\\
390	0\\
391	0\\
392	0\\
393	0\\
394	1\\
395	1\\
396	1\\
397	0\\
398	0\\
399	0\\
400	0\\
401	0\\
402	0\\
403	0\\
404	1\\
405	1\\
406	1\\
407	1\\
408	1\\
409	1\\
410	1\\
411	1\\
412	0\\
413	0\\
414	0\\
415	1\\
416	1\\
417	1\\
418	0\\
419	0\\
420	0\\
421	0\\
422	1\\
423	1\\
424	1\\
425	0\\
426	0\\
427	0\\
428	0\\
429	1\\
430	1\\
431	1\\
432	1\\
433	1\\
434	1\\
435	0\\
436	0\\
437	0\\
438	0\\
439	0\\
440	1\\
441	1\\
442	1\\
443	1\\
444	1\\
445	1\\
446	1\\
447	1\\
448	0\\
449	0\\
450	0\\
451	0\\
452	0\\
453	1\\
454	1\\
455	1\\
456	1\\
457	0\\
458	0\\
459	0\\
460	0\\
461	0\\
462	1\\
463	1\\
464	1\\
465	1\\
466	0\\
467	0\\
468	0\\
469	0\\
470	0\\
471	1\\
472	1\\
473	1\\
474	1\\
475	1\\
476	1\\
477	0\\
478	0\\
479	0\\
480	0\\
481	1\\
482	1\\
483	1\\
484	0\\
485	0\\
486	0\\
487	0\\
488	1\\
489	1\\
490	1\\
491	0\\
492	0\\
493	0\\
494	0\\
495	1\\
496	1\\
497	1\\
498	1\\
499	1\\
500	0\\
501	0\\
502	0\\
503	0\\
504	1\\
505	1\\
506	1\\
507	0\\
508	0\\
509	0\\
510	0\\
511	0\\
512	0\\
513	1\\
514	1\\
515	1\\
516	1\\
517	0\\
518	0\\
519	0\\
520	0\\
521	0\\
522	0\\
523	1\\
524	1\\
525	1\\
526	1\\
527	1\\
528	1\\
529	1\\
530	0\\
531	0\\
532	0\\
533	1\\
534	1\\
535	1\\
536	0\\
537	0\\
538	0\\
539	0\\
540	0\\
541	0\\
542	0\\
543	0\\
544	1\\
545	1\\
546	1\\
547	0\\
548	0\\
549	0\\
550	0\\
551	0\\
552	0\\
553	0\\
554	0\\
555	0\\
556	1\\
557	1\\
558	1\\
559	1\\
560	0\\
561	0\\
562	0\\
563	1\\
564	1\\
565	1\\
566	1\\
567	1\\
568	0\\
569	0\\
570	0\\
571	0\\
572	0\\
573	0\\
574	1\\
575	1\\
576	1\\
577	0\\
578	0\\
579	0\\
580	0\\
581	0\\
582	1\\
583	1\\
584	1\\
585	1\\
586	1\\
587	1\\
588	1\\
589	0\\
590	0\\
591	0\\
592	0\\
593	0\\
594	0\\
595	1\\
596	1\\
597	1\\
598	1\\
599	1\\
600	1\\
601	0\\
602	0\\
603	0\\
604	0\\
605	1\\
606	1\\
607	1\\
608	0\\
609	0\\
610	0\\
611	0\\
612	1\\
613	1\\
614	1\\
615	1\\
616	0\\
617	0\\
618	0\\
619	0\\
620	0\\
621	0\\
622	0\\
623	0\\
624	1\\
625	1\\
626	1\\
627	1\\
628	1\\
629	1\\
630	0\\
631	0\\
632	0\\
633	0\\
634	1\\
635	1\\
636	1\\
637	0\\
638	0\\
639	0\\
640	0\\
641	0\\
642	1\\
643	1\\
644	1\\
645	1\\
646	0\\
647	0\\
648	0\\
649	0\\
650	0\\
651	1\\
652	1\\
653	1\\
654	1\\
655	1\\
656	1\\
657	1\\
658	1\\
659	0\\
660	0\\
661	0\\
662	0\\
663	1\\
664	1\\
665	1\\
666	0\\
667	0\\
668	0\\
669	0\\
670	0\\
671	0\\
672	0\\
673	1\\
674	1\\
675	1\\
676	0\\
677	0\\
678	0\\
679	0\\
680	0\\
681	0\\
682	1\\
683	1\\
684	1\\
685	1\\
686	1\\
687	1\\
688	1\\
689	1\\
690	0\\
691	0\\
692	0\\
693	0\\
694	1\\
695	1\\
696	1\\
697	1\\
698	0\\
699	0\\
700	0\\
701	1\\
702	1\\
703	1\\
704	0\\
705	0\\
706	0\\
707	0\\
708	1\\
709	1\\
710	1\\
711	1\\
712	1\\
713	1\\
714	0\\
715	0\\
716	0\\
717	0\\
718	0\\
719	1\\
720	1\\
721	1\\
722	1\\
723	1\\
724	1\\
725	1\\
726	1\\
727	1\\
728	0\\
729	0\\
730	0\\
731	0\\
732	0\\
733	1\\
734	1\\
735	1\\
736	0\\
737	0\\
738	0\\
739	0\\
740	1\\
741	1\\
742	1\\
743	1\\
744	0\\
745	0\\
746	0\\
747	0\\
748	0\\
749	1\\
750	1\\
751	1\\
752	1\\
753	1\\
754	0\\
755	0\\
756	0\\
757	0\\
758	0\\
759	0\\
760	1\\
761	1\\
762	1\\
763	1\\
764	0\\
765	0\\
766	0\\
767	0\\
768	1\\
769	1\\
770	1\\
771	1\\
772	0\\
773	0\\
774	0\\
775	0\\
776	1\\
777	1\\
778	1\\
779	1\\
780	1\\
781	0\\
782	0\\
783	0\\
784	0\\
785	1\\
786	1\\
787	1\\
788	1\\
789	0\\
790	0\\
791	0\\
792	0\\
793	0\\
794	0\\
795	1\\
796	1\\
797	1\\
798	0\\
799	0\\
800	0\\
801	0\\
802	0\\
803	0\\
804	1\\
805	1\\
806	1\\
807	1\\
808	1\\
809	1\\
810	1\\
811	0\\
812	0\\
813	0\\
814	1\\
815	1\\
816	1\\
817	0\\
818	0\\
819	0\\
820	0\\
821	0\\
822	0\\
823	0\\
824	0\\
825	0\\
826	0\\
827	1\\
828	1\\
829	1\\
830	1\\
831	0\\
832	0\\
833	0\\
834	0\\
835	0\\
836	0\\
837	0\\
838	0\\
839	1\\
840	1\\
841	1\\
842	0\\
843	0\\
844	0\\
845	1\\
846	1\\
847	1\\
848	1\\
849	1\\
850	0\\
851	0\\
852	0\\
853	0\\
854	0\\
855	0\\
856	1\\
857	1\\
858	1\\
859	1\\
860	0\\
861	0\\
862	0\\
863	0\\
864	0\\
865	1\\
866	1\\
867	1\\
868	1\\
869	1\\
870	0\\
871	0\\
872	0\\
873	0\\
874	0\\
875	1\\
876	1\\
877	1\\
878	1\\
879	1\\
880	1\\
881	0\\
882	0\\
883	0\\
884	0\\
885	1\\
886	1\\
887	1\\
888	0\\
889	0\\
890	0\\
891	0\\
892	0\\
893	0\\
894	0\\
895	0\\
896	0\\
897	1\\
898	1\\
899	1\\
900	1\\
901	0\\
902	0\\
903	0\\
904	0\\
905	1\\
906	1\\
907	1\\
908	1\\
909	1\\
910	1\\
911	0\\
912	0\\
913	0\\
914	0\\
915	1\\
916	1\\
917	1\\
918	0\\
919	0\\
920	0\\
921	0\\
922	0\\
923	1\\
924	1\\
925	1\\
926	1\\
927	1\\
928	1\\
929	0\\
930	0\\
931	0\\
932	0\\
933	0\\
934	1\\
935	1\\
936	1\\
937	1\\
938	1\\
939	1\\
940	0\\
941	0\\
942	0\\
943	1\\
944	1\\
945	1\\
946	0\\
947	0\\
948	0\\
949	0\\
950	0\\
951	0\\
952	0\\
953	1\\
954	1\\
955	1\\
956	0\\
957	0\\
958	0\\
959	0\\
960	0\\
961	0\\
962	0\\
963	0\\
964	0\\
965	1\\
966	1\\
967	1\\
968	1\\
969	1\\
970	1\\
971	1\\
972	1\\
973	0\\
974	0\\
975	0\\
976	1\\
977	1\\
978	1\\
979	0\\
980	0\\
981	0\\
982	1\\
983	1\\
984	1\\
985	0\\
986	0\\
987	0\\
988	0\\
989	0\\
990	1\\
991	1\\
992	1\\
993	1\\
994	1\\
995	1\\
996	1\\
997	1\\
998	0\\
999	0\\
1000	0\\
1001	0\\
1002	0\\
1003	0\\
1004	1\\
1005	1\\
1006	1\\
1007	1\\
1008	1\\
1009	1\\
1010	1\\
1011	1\\
1012	0\\
1013	0\\
1014	0\\
1015	0\\
1016	0\\
1017	1\\
1018	1\\
1019	1\\
1020	0\\
1021	0\\
1022	0\\
1023	0\\
1024	1\\
1025	1\\
1026	1\\
1027	1\\
1028	0\\
1029	0\\
1030	0\\
1031	0\\
1032	0\\
1033	0\\
1034	0\\
1035	1\\
1036	1\\
1037	1\\
1038	1\\
1039	1\\
1040	1\\
1041	0\\
1042	0\\
1043	0\\
1044	0\\
1045	1\\
1046	1\\
1047	1\\
1048	0\\
1049	0\\
1050	0\\
1051	0\\
1052	1\\
1053	1\\
1054	1\\
1055	1\\
1056	0\\
1057	0\\
1058	0\\
1059	0\\
1060	1\\
1061	1\\
1062	1\\
1063	1\\
1064	1\\
1065	1\\
1066	1\\
1067	1\\
1068	1\\
1069	0\\
1070	0\\
1071	0\\
1072	0\\
1073	1\\
1074	1\\
1075	1\\
1076	0\\
1077	0\\
1078	0\\
1079	0\\
1080	0\\
1081	0\\
1082	1\\
1083	1\\
1084	1\\
1085	0\\
1086	0\\
1087	0\\
1088	0\\
1089	0\\
1090	0\\
1091	1\\
1092	1\\
1093	1\\
1094	1\\
1095	1\\
1096	1\\
1097	1\\
1098	0\\
1099	0\\
1100	0\\
1101	0\\
1102	1\\
1103	1\\
1104	1\\
1105	1\\
1106	0\\
1107	0\\
1108	0\\
1109	0\\
1110	0\\
1111	0\\
1112	0\\
1113	0\\
1114	1\\
1115	1\\
1116	1\\
1117	0\\
1118	0\\
1119	0\\
1120	0\\
1121	0\\
1122	0\\
1123	0\\
1124	0\\
1125	1\\
1126	1\\
1127	1\\
1128	0\\
1129	0\\
1130	0\\
1131	1\\
1132	1\\
1133	1\\
1134	1\\
1135	1\\
1136	1\\
1137	0\\
1138	0\\
1139	0\\
1140	0\\
1141	0\\
1142	0\\
1143	1\\
1144	1\\
1145	1\\
1146	0\\
1147	0\\
1148	0\\
1149	0\\
1150	0\\
1151	1\\
1152	1\\
1153	1\\
1154	1\\
1155	1\\
1156	0\\
1157	0\\
1158	0\\
1159	0\\
1160	0\\
1161	1\\
1162	1\\
1163	1\\
1164	1\\
1165	1\\
1166	1\\
1167	0\\
1168	0\\
1169	0\\
1170	0\\
1171	0\\
1172	0\\
1173	1\\
1174	1\\
1175	1\\
1176	1\\
1177	0\\
1178	0\\
1179	0\\
1180	0\\
1181	0\\
1182	1\\
1183	1\\
1184	1\\
1185	1\\
1186	0\\
1187	0\\
1188	0\\
1189	0\\
1190	1\\
1191	1\\
1192	1\\
1193	1\\
1194	1\\
1195	1\\
1196	0\\
1197	0\\
1198	0\\
1199	0\\
1200	1\\
1201	1\\
1202	1\\
1203	1\\
1204	0\\
1205	0\\
1206	0\\
1207	0\\
1208	0\\
1209	1\\
1210	1\\
1211	1\\
1212	1\\
1213	0\\
1214	0\\
1215	0\\
1216	0\\
1217	0\\
1218	1\\
1219	1\\
1220	1\\
1221	1\\
1222	1\\
1223	1\\
1224	0\\
1225	0\\
1226	0\\
1227	1\\
1228	1\\
1229	1\\
1230	0\\
1231	0\\
1232	0\\
1233	0\\
1234	0\\
1235	0\\
1236	0\\
1237	0\\
1238	0\\
1239	1\\
1240	1\\
1241	1\\
1242	1\\
1243	0\\
1244	0\\
1245	0\\
1246	0\\
1247	0\\
1248	0\\
1249	0\\
1250	1\\
1251	1\\
1252	1\\
1253	1\\
1254	1\\
1255	1\\
1256	1\\
1257	1\\
1258	0\\
1259	0\\
1260	0\\
1261	1\\
1262	1\\
1263	1\\
1264	0\\
1265	0\\
1266	0\\
1267	1\\
1268	1\\
1269	1\\
1270	1\\
1271	0\\
1272	0\\
1273	0\\
1274	0\\
1275	0\\
1276	1\\
1277	1\\
1278	1\\
1279	1\\
1280	1\\
1281	1\\
1282	0\\
1283	0\\
1284	0\\
1285	0\\
1286	0\\
1287	1\\
1288	1\\
1289	1\\
1290	1\\
1291	1\\
1292	1\\
1293	1\\
1294	0\\
1295	0\\
1296	0\\
1297	0\\
1298	0\\
1299	1\\
1300	1\\
1301	1\\
1302	0\\
1303	0\\
1304	0\\
1305	0\\
1306	0\\
1307	0\\
1308	0\\
1309	1\\
1310	1\\
1311	1\\
1312	1\\
1313	1\\
1314	0\\
1315	0\\
1316	0\\
1317	0\\
1318	1\\
1319	1\\
1320	1\\
1321	1\\
1322	1\\
1323	0\\
1324	0\\
1325	0\\
1326	0\\
1327	1\\
1328	1\\
1329	1\\
1330	0\\
1331	0\\
1332	0\\
1333	0\\
1334	0\\
1335	1\\
1336	1\\
1337	1\\
1338	1\\
1339	1\\
1340	1\\
1341	0\\
1342	0\\
1343	0\\
1344	0\\
1345	0\\
1346	1\\
1347	1\\
1348	1\\
1349	1\\
1350	1\\
1351	0\\
1352	0\\
1353	0\\
1354	0\\
1355	1\\
1356	1\\
1357	1\\
1358	0\\
1359	0\\
1360	0\\
1361	0\\
1362	0\\
1363	0\\
1364	1\\
1365	1\\
1366	1\\
1367	0\\
1368	0\\
1369	0\\
1370	0\\
1371	0\\
1372	0\\
1373	0\\
1374	0\\
1375	1\\
1376	1\\
1377	1\\
1378	1\\
1379	1\\
1380	1\\
1381	1\\
1382	0\\
1383	0\\
1384	0\\
1385	1\\
1386	1\\
1387	1\\
1388	0\\
1389	0\\
1390	0\\
1391	0\\
1392	0\\
1393	0\\
1394	0\\
1395	0\\
1396	1\\
1397	1\\
1398	1\\
1399	0\\
1400	0\\
1401	0\\
1402	0\\
1403	0\\
1404	0\\
1405	0\\
1406	0\\
1407	1\\
1408	1\\
1409	1\\
1410	1\\
1411	0\\
1412	0\\
1413	0\\
1414	1\\
1415	1\\
1416	1\\
1417	1\\
1418	0\\
1419	0\\
1420	0\\
1421	0\\
1422	0\\
1423	0\\
1424	1\\
1425	1\\
1426	1\\
1427	0\\
1428	0\\
1429	0\\
1430	0\\
1431	0\\
1432	1\\
1433	1\\
1434	1\\
1435	1\\
1436	1\\
1437	0\\
1438	0\\
1439	0\\
1440	0\\
1441	0\\
1442	0\\
1443	1\\
1444	1\\
1445	1\\
1446	1\\
1447	1\\
1448	1\\
1449	0\\
1450	0\\
1451	0\\
1452	0\\
1453	1\\
1454	1\\
1455	1\\
1456	0\\
1457	0\\
1458	0\\
1459	0\\
1460	0\\
1461	1\\
1462	1\\
1463	1\\
1464	1\\
1465	0\\
1466	0\\
1467	0\\
1468	0\\
1469	1\\
1470	1\\
1471	1\\
1472	1\\
1473	1\\
1474	1\\
1475	0\\
1476	0\\
1477	0\\
1478	0\\
1479	1\\
1480	1\\
1481	1\\
1482	0\\
1483	0\\
1484	0\\
1485	0\\
1486	0\\
1487	1\\
1488	1\\
1489	1\\
1490	1\\
1491	0\\
1492	0\\
1493	0\\
1494	0\\
1495	0\\
1496	1\\
1497	1\\
1498	1\\
1499	1\\
1500	1\\
1501	1\\
1502	0\\
1503	0\\
1504	0\\
1505	0\\
1506	1\\
1507	1\\
1508	1\\
1509	1\\
1510	0\\
1511	0\\
1512	0\\
1513	0\\
1514	0\\
1515	0\\
1516	0\\
1517	1\\
1518	1\\
1519	1\\
1520	0\\
1521	0\\
1522	0\\
1523	0\\
1524	0\\
1525	0\\
1526	0\\
1527	1\\
1528	1\\
1529	1\\
1530	1\\
1531	1\\
1532	1\\
1533	1\\
1534	1\\
1535	0\\
1536	0\\
1537	0\\
1538	1\\
1539	1\\
1540	1\\
1541	1\\
1542	0\\
1543	0\\
1544	0\\
1545	1\\
1546	1\\
1547	0\\
1548	0\\
1549	0\\
1550	0\\
1551	1\\
1552	1\\
1553	1\\
1554	1\\
1555	1\\
1556	1\\
1557	0\\
1558	0\\
1559	0\\
1560	0\\
1561	0\\
1562	1\\
1563	1\\
1564	1\\
1565	1\\
1566	1\\
1567	1\\
1568	1\\
1569	1\\
1570	0\\
1571	0\\
1572	0\\
1573	0\\
1574	0\\
1575	0\\
1576	0\\
1577	1\\
1578	1\\
1579	1\\
1580	1\\
1581	0\\
1582	0\\
1583	0\\
1584	0\\
1585	1\\
1586	1\\
1587	1\\
1588	1\\
1589	0\\
1590	0\\
1591	0\\
1592	0\\
1593	0\\
1594	1\\
1595	1\\
1596	1\\
1597	1\\
1598	1\\
1599	1\\
1600	0\\
1601	0\\
1602	0\\
1603	0\\
1604	1\\
1605	1\\
1606	1\\
1607	1\\
1608	0\\
1609	0\\
1610	0\\
1611	0\\
1612	1\\
1613	1\\
1614	1\\
1615	0\\
1616	0\\
1617	0\\
1618	0\\
1619	1\\
1620	1\\
1621	1\\
1622	1\\
1623	1\\
1624	0\\
1625	0\\
1626	0\\
1627	0\\
1628	1\\
1629	1\\
1630	1\\
1631	0\\
1632	0\\
1633	0\\
1634	0\\
1635	0\\
1636	0\\
1637	0\\
1638	0\\
1639	1\\
1640	1\\
1641	1\\
1642	1\\
1643	0\\
1644	0\\
1645	0\\
1646	0\\
1647	0\\
1648	0\\
1649	1\\
1650	1\\
1651	1\\
1652	1\\
1653	1\\
1654	1\\
1655	1\\
1656	0\\
1657	0\\
1658	0\\
1659	1\\
1660	1\\
1661	1\\
1662	0\\
1663	0\\
1664	0\\
1665	0\\
1666	0\\
1667	0\\
1668	0\\
1669	0\\
1670	1\\
1671	1\\
1672	1\\
1673	1\\
1674	1\\
1675	0\\
1676	0\\
1677	0\\
1678	0\\
1679	0\\
1680	0\\
1681	0\\
1682	0\\
1683	1\\
1684	1\\
1685	1\\
1686	0\\
1687	0\\
1688	0\\
1689	1\\
1690	1\\
1691	1\\
1692	1\\
1693	1\\
1694	0\\
1695	0\\
1696	0\\
1697	0\\
1698	0\\
1699	0\\
1700	1\\
1701	1\\
1702	1\\
1703	0\\
1704	0\\
1705	0\\
1706	0\\
1707	0\\
1708	0\\
1709	1\\
1710	1\\
1711	1\\
1712	1\\
1713	1\\
1714	0\\
1715	0\\
1716	0\\
1717	0\\
1718	0\\
1719	1\\
1720	1\\
1721	1\\
1722	1\\
1723	1\\
1724	1\\
1725	0\\
1726	0\\
1727	0\\
1728	0\\
1729	1\\
1730	1\\
1731	1\\
1732	0\\
1733	0\\
1734	0\\
1735	0\\
1736	1\\
1737	1\\
1738	1\\
1739	1\\
1740	1\\
1741	1\\
1742	0\\
1743	0\\
1744	0\\
1745	0\\
1746	0\\
1747	1\\
1748	1\\
1749	1\\
1750	1\\
1751	1\\
1752	1\\
1753	0\\
1754	0\\
1755	0\\
1756	0\\
1757	1\\
1758	1\\
1759	1\\
1760	0\\
1761	0\\
1762	0\\
1763	0\\
1764	0\\
1765	1\\
1766	1\\
1767	1\\
1768	1\\
1769	0\\
1770	0\\
1771	0\\
1772	0\\
1773	0\\
1774	0\\
1775	0\\
1776	0\\
1777	0\\
1778	1\\
1779	1\\
1780	1\\
1781	1\\
1782	1\\
1783	1\\
1784	0\\
1785	0\\
1786	0\\
1787	1\\
1788	1\\
1789	1\\
1790	0\\
1791	0\\
1792	0\\
1793	0\\
1794	0\\
1795	0\\
1796	1\\
1797	1\\
1798	1\\
1799	0\\
1800	0\\
1801	0\\
1802	0\\
1803	0\\
1804	0\\
1805	1\\
1806	1\\
1807	1\\
1808	1\\
1809	1\\
1810	1\\
1811	1\\
1812	1\\
1813	1\\
1814	1\\
1815	0\\
1816	0\\
1817	0\\
1818	0\\
1819	1\\
1820	1\\
1821	1\\
1822	0\\
1823	0\\
1824	0\\
1825	1\\
1826	1\\
1827	1\\
1828	0\\
1829	0\\
1830	0\\
1831	0\\
1832	0\\
1833	1\\
1834	1\\
1835	1\\
1836	1\\
1837	1\\
1838	1\\
1839	0\\
1840	0\\
1841	0\\
1842	0\\
1843	0\\
1844	0\\
1845	1\\
1846	1\\
1847	1\\
1848	1\\
1849	1\\
1850	1\\
1851	1\\
1852	1\\
1853	0\\
1854	0\\
1855	0\\
1856	0\\
1857	0\\
1858	1\\
1859	1\\
1860	1\\
1861	0\\
1862	0\\
1863	0\\
1864	0\\
1865	1\\
1866	1\\
1867	1\\
1868	1\\
1869	0\\
1870	0\\
1871	0\\
1872	0\\
1873	0\\
1874	1\\
1875	1\\
1876	1\\
1877	1\\
1878	1\\
1879	1\\
1880	1\\
1881	1\\
1882	1\\
1883	1\\
1884	0\\
1885	0\\
1886	0\\
1887	0\\
1888	1\\
1889	1\\
1890	1\\
1891	0\\
1892	0\\
1893	0\\
1894	0\\
1895	1\\
1896	1\\
1897	1\\
1898	0\\
1899	0\\
1900	0\\
1901	0\\
1902	1\\
1903	1\\
1904	1\\
1905	1\\
1906	1\\
1907	0\\
1908	0\\
1909	0\\
1910	0\\
1911	0\\
1912	0\\
1913	1\\
1914	1\\
1915	1\\
1916	1\\
1917	0\\
1918	0\\
1919	0\\
1920	0\\
1921	0\\
1922	0\\
1923	1\\
1924	1\\
1925	1\\
1926	0\\
1927	0\\
1928	0\\
1929	0\\
1930	0\\
1931	0\\
1932	1\\
1933	1\\
1934	1\\
1935	1\\
1936	1\\
1937	1\\
1938	1\\
1939	0\\
1940	0\\
1941	0\\
1942	1\\
1943	1\\
1944	1\\
1945	1\\
1946	0\\
1947	0\\
1948	0\\
1949	0\\
1950	0\\
1951	0\\
1952	0\\
1953	0\\
1954	1\\
1955	1\\
1956	1\\
1957	0\\
1958	0\\
1959	0\\
1960	0\\
1961	0\\
1962	0\\
1963	0\\
1964	0\\
1965	1\\
};
\end{axis}

\begin{axis}[%
width=4.133in,
height=0.863in,
at={(0.693in,0.44in)},
scale only axis,
xmin=0,
xmax=2000,
xmajorgrids,
ymin=0,
ymax=1,
ymajorgrids,
axis background/.style={fill=white}
]
\pgfplotsset{max space between ticks=50}
\addplot [color=mycolor2,solid,forget plot]
  table[row sep=crcr]{%
1	0\\
2	0.5\\
3	0.5\\
4	0.5\\
5	0.5\\
6	0.5\\
7	0.5\\
8	0.5\\
9	0.5\\
10	0.5\\
11	0.5\\
12	0.5\\
13	0.5\\
14	0.5\\
15	0.5\\
16	0.5\\
17	0.5\\
18	0.5\\
19	0.5\\
20	1\\
21	1\\
22	0.5\\
23	0.5\\
24	0.5\\
25	0.5\\
26	0.5\\
27	0.5\\
28	0.5\\
29	0.5\\
30	0.5\\
31	0.5\\
32	1\\
33	1\\
34	0.5\\
35	0.5\\
36	0.5\\
37	0.5\\
38	0.5\\
39	0.5\\
40	0.5\\
41	0.5\\
42	0.5\\
43	0.5\\
44	0.5\\
45	0.5\\
46	0.5\\
47	0.5\\
48	1\\
49	1\\
50	1\\
51	0.5\\
52	0.5\\
53	0.5\\
54	0.5\\
55	0.5\\
56	0.5\\
57	0.5\\
58	0.5\\
59	1\\
60	1\\
61	0.5\\
62	0.5\\
63	0.5\\
64	0.5\\
65	0.5\\
66	0.5\\
67	0.5\\
68	0.5\\
69	0.5\\
70	0.5\\
71	0.5\\
72	0.5\\
73	0.5\\
74	1\\
75	1\\
76	0.5\\
77	0.5\\
78	0.5\\
79	0.5\\
80	0.5\\
81	0.5\\
82	0.5\\
83	0.5\\
84	0.5\\
85	0.5\\
86	0.5\\
87	0.5\\
88	0.5\\
89	1\\
90	1\\
91	1\\
92	0.5\\
93	0.5\\
94	0.5\\
95	0.5\\
96	0.5\\
97	0.5\\
98	0.5\\
99	0.5\\
100	0.5\\
101	0.5\\
102	0.5\\
103	0.5\\
104	1\\
105	1\\
106	0.5\\
107	0.5\\
108	0.5\\
109	0.5\\
110	0.5\\
111	0.5\\
112	0.5\\
113	0.5\\
114	0.5\\
115	0.5\\
116	0.5\\
117	0.5\\
118	1\\
119	1\\
120	1\\
121	1\\
122	1\\
123	0.5\\
124	0.5\\
125	0.5\\
126	0.5\\
127	0.5\\
128	0.5\\
129	0.5\\
130	0.5\\
131	0.5\\
132	0.5\\
133	0.5\\
134	0.5\\
135	0.5\\
136	0.5\\
137	0.5\\
138	0.5\\
139	0.5\\
140	0.5\\
141	0.5\\
142	0.5\\
143	1\\
144	1\\
145	1\\
146	1\\
147	0.5\\
148	0.5\\
149	0.5\\
150	0.5\\
151	0.5\\
152	0.5\\
153	0.5\\
154	0.5\\
155	0.5\\
156	0.5\\
157	0.5\\
158	1\\
159	1\\
160	0.5\\
161	0.5\\
162	0.5\\
163	0.5\\
164	0.5\\
165	0.5\\
166	0.5\\
167	0.5\\
168	0.5\\
169	0.5\\
170	0.5\\
171	0.5\\
172	0.5\\
173	0.5\\
174	0.5\\
175	0.5\\
176	1\\
177	1\\
178	0.5\\
179	0.5\\
180	0.5\\
181	0.5\\
182	0.5\\
183	0.5\\
184	0.5\\
185	0.5\\
186	0.5\\
187	0.5\\
188	0.5\\
189	0.5\\
190	1\\
191	1\\
192	0.5\\
193	0.5\\
194	0.5\\
195	0.5\\
196	0.5\\
197	0.5\\
198	0.5\\
199	0.5\\
200	0.5\\
201	0.5\\
202	0.5\\
203	0.5\\
204	0.5\\
205	1\\
206	1\\
207	0.5\\
208	0.5\\
209	0.5\\
210	0.5\\
211	0.5\\
212	0.5\\
213	0.5\\
214	0.5\\
215	1\\
216	1\\
217	0.5\\
218	0.5\\
219	0.5\\
220	0.5\\
221	0.5\\
222	0.5\\
223	0.5\\
224	0.5\\
225	0.5\\
226	0.5\\
227	0.5\\
228	0.5\\
229	1\\
230	1\\
231	0.5\\
232	0.5\\
233	0.5\\
234	0.5\\
235	0.5\\
236	0.5\\
237	0.5\\
238	0.5\\
239	0.5\\
240	1\\
241	1\\
242	1\\
243	1\\
244	0.5\\
245	0.5\\
246	0.5\\
247	0.5\\
248	0.5\\
249	0.5\\
250	0.5\\
251	0.5\\
252	0.5\\
253	0.5\\
254	0.5\\
255	0.5\\
256	0.5\\
257	0.5\\
258	0.5\\
259	0.5\\
260	1\\
261	1\\
262	0.5\\
263	0.5\\
264	0.5\\
265	0.5\\
266	0.5\\
267	0.5\\
268	0.5\\
269	0.5\\
270	0.5\\
271	0.5\\
272	0.5\\
273	1\\
274	1\\
275	1\\
276	1\\
277	1\\
278	1\\
279	0.5\\
280	0.5\\
281	0.5\\
282	0.5\\
283	0.5\\
284	0.5\\
285	0.5\\
286	0.5\\
287	0.5\\
288	0.5\\
289	0.5\\
290	0.5\\
291	0.5\\
292	0.5\\
293	0.5\\
294	0.5\\
295	0.5\\
296	0.5\\
297	0.5\\
298	0.5\\
299	0.5\\
300	1\\
301	1\\
302	0.5\\
303	0.5\\
304	0.5\\
305	0.5\\
306	0.5\\
307	0.5\\
308	0.5\\
309	0.5\\
310	0.5\\
311	0.5\\
312	1\\
313	1\\
314	0.5\\
315	0.5\\
316	0.5\\
317	0.5\\
318	0.5\\
319	0.5\\
320	0.5\\
321	0.5\\
322	0.5\\
323	0.5\\
324	0.5\\
325	0.5\\
326	0.5\\
327	0.5\\
328	0.5\\
329	0.5\\
330	0.5\\
331	1\\
332	1\\
333	0.5\\
334	0.5\\
335	0.5\\
336	0.5\\
337	0.5\\
338	0.5\\
339	0.5\\
340	0.5\\
341	1\\
342	1\\
343	0.5\\
344	0.5\\
345	0.5\\
346	0.5\\
347	0.5\\
348	0.5\\
349	0.5\\
350	0.5\\
351	0.5\\
352	0.5\\
353	0.5\\
354	0.5\\
355	0.5\\
356	0.5\\
357	0.5\\
358	0.5\\
359	1\\
360	1\\
361	0.5\\
362	0.5\\
363	0.5\\
364	0.5\\
365	0.5\\
366	0.5\\
367	0.5\\
368	0.5\\
369	0.5\\
370	1\\
371	1\\
372	1\\
373	0.5\\
374	0.5\\
375	0.5\\
376	0.5\\
377	0.5\\
378	0.5\\
379	0.5\\
380	0.5\\
381	0.5\\
382	0.5\\
383	0.5\\
384	0.5\\
385	0.5\\
386	0.5\\
387	1\\
388	1\\
389	0.5\\
390	0.5\\
391	0.5\\
392	0.5\\
393	0.5\\
394	0.5\\
395	0.5\\
396	0.5\\
397	0.5\\
398	0.5\\
399	1\\
400	1\\
401	1\\
402	1\\
403	1\\
404	0.5\\
405	0.5\\
406	0.5\\
407	0.5\\
408	0.5\\
409	0.5\\
410	0.5\\
411	0.5\\
412	0.5\\
413	0.5\\
414	0.5\\
415	0.5\\
416	0.5\\
417	0.5\\
418	0.5\\
419	0.5\\
420	0.5\\
421	0.5\\
422	0.5\\
423	0.5\\
424	0.5\\
425	1\\
426	1\\
427	0.5\\
428	0.5\\
429	0.5\\
430	0.5\\
431	0.5\\
432	0.5\\
433	0.5\\
434	0.5\\
435	0.5\\
436	0.5\\
437	0.5\\
438	1\\
439	1\\
440	0.5\\
441	0.5\\
442	0.5\\
443	0.5\\
444	0.5\\
445	0.5\\
446	0.5\\
447	0.5\\
448	0.5\\
449	0.5\\
450	0.5\\
451	0.5\\
452	0.5\\
453	0.5\\
454	0.5\\
455	0.5\\
456	0.5\\
457	1\\
458	1\\
459	1\\
460	0.5\\
461	0.5\\
462	0.5\\
463	0.5\\
464	0.5\\
465	0.5\\
466	0.5\\
467	0.5\\
468	0.5\\
469	1\\
470	1\\
471	0.5\\
472	0.5\\
473	0.5\\
474	0.5\\
475	0.5\\
476	0.5\\
477	0.5\\
478	0.5\\
479	0.5\\
480	0.5\\
481	0.5\\
482	0.5\\
483	0.5\\
484	1\\
485	1\\
486	0.5\\
487	0.5\\
488	0.5\\
489	0.5\\
490	0.5\\
491	0.5\\
492	0.5\\
493	1\\
494	1\\
495	0.5\\
496	0.5\\
497	0.5\\
498	0.5\\
499	0.5\\
500	0.5\\
501	0.5\\
502	0.5\\
503	0.5\\
504	0.5\\
505	0.5\\
506	0.5\\
507	1\\
508	1\\
509	0.5\\
510	0.5\\
511	0.5\\
512	0.5\\
513	0.5\\
514	0.5\\
515	0.5\\
516	0.5\\
517	0.5\\
518	0.5\\
519	1\\
520	1\\
521	1\\
522	1\\
523	0.5\\
524	0.5\\
525	0.5\\
526	0.5\\
527	0.5\\
528	0.5\\
529	0.5\\
530	0.5\\
531	0.5\\
532	0.5\\
533	0.5\\
534	0.5\\
535	0.5\\
536	1\\
537	1\\
538	0.5\\
539	0.5\\
540	0.5\\
541	0.5\\
542	0.5\\
543	0.5\\
544	0.5\\
545	0.5\\
546	0.5\\
547	0.5\\
548	0.5\\
549	1\\
550	1\\
551	1\\
552	1\\
553	1\\
554	1\\
555	1\\
556	0.5\\
557	0.5\\
558	0.5\\
559	0.5\\
560	0.5\\
561	0.5\\
562	0.5\\
563	0.5\\
564	0.5\\
565	0.5\\
566	0.5\\
567	0.5\\
568	0.5\\
569	0.5\\
570	0.5\\
571	0.5\\
572	0.5\\
573	0.5\\
574	0.5\\
575	0.5\\
576	0.5\\
577	1\\
578	1\\
579	1\\
580	0.5\\
581	0.5\\
582	0.5\\
583	0.5\\
584	0.5\\
585	0.5\\
586	0.5\\
587	0.5\\
588	0.5\\
589	0.5\\
590	0.5\\
591	0.5\\
592	0.5\\
593	1\\
594	1\\
595	0.5\\
596	0.5\\
597	0.5\\
598	0.5\\
599	0.5\\
600	0.5\\
601	0.5\\
602	0.5\\
603	0.5\\
604	0.5\\
605	0.5\\
606	0.5\\
607	0.5\\
608	1\\
609	1\\
610	0.5\\
611	0.5\\
612	0.5\\
613	0.5\\
614	0.5\\
615	0.5\\
616	0.5\\
617	0.5\\
618	0.5\\
619	0.5\\
620	1\\
621	1\\
622	1\\
623	1\\
624	0.5\\
625	0.5\\
626	0.5\\
627	0.5\\
628	0.5\\
629	0.5\\
630	0.5\\
631	0.5\\
632	0.5\\
633	0.5\\
634	0.5\\
635	0.5\\
636	0.5\\
637	1\\
638	1\\
639	0.5\\
640	0.5\\
641	0.5\\
642	0.5\\
643	0.5\\
644	0.5\\
645	0.5\\
646	0.5\\
647	0.5\\
648	1\\
649	1\\
650	1\\
651	0.5\\
652	0.5\\
653	0.5\\
654	0.5\\
655	0.5\\
656	0.5\\
657	0.5\\
658	0.5\\
659	0.5\\
660	0.5\\
661	0.5\\
662	0.5\\
663	0.5\\
664	0.5\\
665	0.5\\
666	1\\
667	1\\
668	0.5\\
669	0.5\\
670	0.5\\
671	0.5\\
672	0.5\\
673	0.5\\
674	0.5\\
675	0.5\\
676	0.5\\
677	0.5\\
678	1\\
679	1\\
680	1\\
681	1\\
682	0.5\\
683	0.5\\
684	0.5\\
685	0.5\\
686	0.5\\
687	0.5\\
688	0.5\\
689	0.5\\
690	0.5\\
691	0.5\\
692	0.5\\
693	0.5\\
694	0.5\\
695	0.5\\
696	0.5\\
697	0.5\\
698	0.5\\
699	0.5\\
700	0.5\\
701	0.5\\
702	0.5\\
703	0.5\\
704	1\\
705	1\\
706	0.5\\
707	0.5\\
708	0.5\\
709	0.5\\
710	0.5\\
711	0.5\\
712	0.5\\
713	0.5\\
714	0.5\\
715	0.5\\
716	0.5\\
717	1\\
718	1\\
719	0.5\\
720	0.5\\
721	0.5\\
722	0.5\\
723	0.5\\
724	0.5\\
725	0.5\\
726	0.5\\
727	0.5\\
728	0.5\\
729	0.5\\
730	0.5\\
731	0.5\\
732	0.5\\
733	0.5\\
734	0.5\\
735	0.5\\
736	1\\
737	1\\
738	0.5\\
739	0.5\\
740	0.5\\
741	0.5\\
742	0.5\\
743	0.5\\
744	0.5\\
745	0.5\\
746	0.5\\
747	1\\
748	1\\
749	0.5\\
750	0.5\\
751	0.5\\
752	0.5\\
753	0.5\\
754	0.5\\
755	0.5\\
756	0.5\\
757	0.5\\
758	0.5\\
759	0.5\\
760	0.5\\
761	0.5\\
762	0.5\\
763	0.5\\
764	1\\
765	1\\
766	0.5\\
767	0.5\\
768	0.5\\
769	0.5\\
770	0.5\\
771	0.5\\
772	0.5\\
773	0.5\\
774	1\\
775	1\\
776	0.5\\
777	0.5\\
778	0.5\\
779	0.5\\
780	0.5\\
781	0.5\\
782	0.5\\
783	0.5\\
784	0.5\\
785	0.5\\
786	0.5\\
787	0.5\\
788	0.5\\
789	1\\
790	1\\
791	0.5\\
792	0.5\\
793	0.5\\
794	0.5\\
795	0.5\\
796	0.5\\
797	0.5\\
798	0.5\\
799	0.5\\
800	1\\
801	1\\
802	1\\
803	1\\
804	0.5\\
805	0.5\\
806	0.5\\
807	0.5\\
808	0.5\\
809	0.5\\
810	0.5\\
811	0.5\\
812	0.5\\
813	0.5\\
814	0.5\\
815	0.5\\
816	0.5\\
817	1\\
818	1\\
819	0.5\\
820	0.5\\
821	0.5\\
822	0.5\\
823	0.5\\
824	0.5\\
825	0.5\\
826	0.5\\
827	0.5\\
828	0.5\\
829	0.5\\
830	0.5\\
831	0.5\\
832	0.5\\
833	1\\
834	1\\
835	1\\
836	1\\
837	1\\
838	1\\
839	0.5\\
840	0.5\\
841	0.5\\
842	0.5\\
843	0.5\\
844	0.5\\
845	0.5\\
846	0.5\\
847	0.5\\
848	0.5\\
849	0.5\\
850	0.5\\
851	0.5\\
852	0.5\\
853	0.5\\
854	0.5\\
855	0.5\\
856	0.5\\
857	0.5\\
858	0.5\\
859	0.5\\
860	1\\
861	1\\
862	1\\
863	0.5\\
864	0.5\\
865	0.5\\
866	0.5\\
867	0.5\\
868	0.5\\
869	0.5\\
870	0.5\\
871	0.5\\
872	0.5\\
873	1\\
874	1\\
875	0.5\\
876	0.5\\
877	0.5\\
878	0.5\\
879	0.5\\
880	0.5\\
881	0.5\\
882	0.5\\
883	0.5\\
884	0.5\\
885	0.5\\
886	0.5\\
887	0.5\\
888	1\\
889	1\\
890	1\\
891	1\\
892	0.5\\
893	0.5\\
894	0.5\\
895	0.5\\
896	0.5\\
897	0.5\\
898	0.5\\
899	0.5\\
900	0.5\\
901	0.5\\
902	0.5\\
903	1\\
904	1\\
905	0.5\\
906	0.5\\
907	0.5\\
908	0.5\\
909	0.5\\
910	0.5\\
911	0.5\\
912	0.5\\
913	0.5\\
914	0.5\\
915	0.5\\
916	0.5\\
917	0.5\\
918	1\\
919	1\\
920	0.5\\
921	0.5\\
922	0.5\\
923	0.5\\
924	0.5\\
925	0.5\\
926	0.5\\
927	0.5\\
928	0.5\\
929	0.5\\
930	0.5\\
931	1\\
932	1\\
933	1\\
934	0.5\\
935	0.5\\
936	0.5\\
937	0.5\\
938	0.5\\
939	0.5\\
940	0.5\\
941	0.5\\
942	0.5\\
943	0.5\\
944	0.5\\
945	0.5\\
946	1\\
947	1\\
948	0.5\\
949	0.5\\
950	0.5\\
951	0.5\\
952	0.5\\
953	0.5\\
954	0.5\\
955	0.5\\
956	0.5\\
957	0.5\\
958	1\\
959	1\\
960	1\\
961	1\\
962	1\\
963	1\\
964	1\\
965	0.5\\
966	0.5\\
967	0.5\\
968	0.5\\
969	0.5\\
970	0.5\\
971	0.5\\
972	0.5\\
973	0.5\\
974	0.5\\
975	0.5\\
976	0.5\\
977	0.5\\
978	0.5\\
979	0.5\\
980	0.5\\
981	0.5\\
982	0.5\\
983	0.5\\
984	0.5\\
985	1\\
986	1\\
987	1\\
988	0.5\\
989	0.5\\
990	0.5\\
991	0.5\\
992	0.5\\
993	0.5\\
994	0.5\\
995	0.5\\
996	0.5\\
997	0.5\\
998	0.5\\
999	0.5\\
1000	0.5\\
1001	0.5\\
1002	1\\
1003	1\\
1004	0.5\\
1005	0.5\\
1006	0.5\\
1007	0.5\\
1008	0.5\\
1009	0.5\\
1010	0.5\\
1011	0.5\\
1012	0.5\\
1013	0.5\\
1014	0.5\\
1015	0.5\\
1016	0.5\\
1017	0.5\\
1018	0.5\\
1019	0.5\\
1020	1\\
1021	1\\
1022	0.5\\
1023	0.5\\
1024	0.5\\
1025	0.5\\
1026	0.5\\
1027	0.5\\
1028	0.5\\
1029	0.5\\
1030	0.5\\
1031	0.5\\
1032	1\\
1033	1\\
1034	1\\
1035	0.5\\
1036	0.5\\
1037	0.5\\
1038	0.5\\
1039	0.5\\
1040	0.5\\
1041	0.5\\
1042	0.5\\
1043	0.5\\
1044	0.5\\
1045	0.5\\
1046	0.5\\
1047	0.5\\
1048	1\\
1049	1\\
1050	0.5\\
1051	0.5\\
1052	0.5\\
1053	0.5\\
1054	0.5\\
1055	0.5\\
1056	0.5\\
1057	0.5\\
1058	1\\
1059	1\\
1060	0.5\\
1061	0.5\\
1062	0.5\\
1063	0.5\\
1064	0.5\\
1065	0.5\\
1066	0.5\\
1067	0.5\\
1068	0.5\\
1069	0.5\\
1070	0.5\\
1071	0.5\\
1072	0.5\\
1073	0.5\\
1074	0.5\\
1075	0.5\\
1076	1\\
1077	1\\
1078	0.5\\
1079	0.5\\
1080	0.5\\
1081	0.5\\
1082	0.5\\
1083	0.5\\
1084	0.5\\
1085	0.5\\
1086	0.5\\
1087	1\\
1088	1\\
1089	1\\
1090	1\\
1091	0.5\\
1092	0.5\\
1093	0.5\\
1094	0.5\\
1095	0.5\\
1096	0.5\\
1097	0.5\\
1098	0.5\\
1099	0.5\\
1100	0.5\\
1101	0.5\\
1102	0.5\\
1103	0.5\\
1104	0.5\\
1105	0.5\\
1106	1\\
1107	1\\
1108	0.5\\
1109	0.5\\
1110	0.5\\
1111	0.5\\
1112	0.5\\
1113	0.5\\
1114	0.5\\
1115	0.5\\
1116	0.5\\
1117	0.5\\
1118	0.5\\
1119	1\\
1120	1\\
1121	1\\
1122	1\\
1123	1\\
1124	1\\
1125	0.5\\
1126	0.5\\
1127	0.5\\
1128	0.5\\
1129	0.5\\
1130	0.5\\
1131	0.5\\
1132	0.5\\
1133	0.5\\
1134	0.5\\
1135	0.5\\
1136	0.5\\
1137	0.5\\
1138	0.5\\
1139	0.5\\
1140	0.5\\
1141	0.5\\
1142	0.5\\
1143	0.5\\
1144	0.5\\
1145	0.5\\
1146	1\\
1147	1\\
1148	1\\
1149	0.5\\
1150	0.5\\
1151	0.5\\
1152	0.5\\
1153	0.5\\
1154	0.5\\
1155	0.5\\
1156	0.5\\
1157	0.5\\
1158	0.5\\
1159	1\\
1160	1\\
1161	0.5\\
1162	0.5\\
1163	0.5\\
1164	0.5\\
1165	0.5\\
1166	0.5\\
1167	0.5\\
1168	0.5\\
1169	0.5\\
1170	0.5\\
1171	0.5\\
1172	0.5\\
1173	0.5\\
1174	0.5\\
1175	0.5\\
1176	0.5\\
1177	1\\
1178	1\\
1179	1\\
1180	0.5\\
1181	0.5\\
1182	0.5\\
1183	0.5\\
1184	0.5\\
1185	0.5\\
1186	0.5\\
1187	0.5\\
1188	1\\
1189	1\\
1190	0.5\\
1191	0.5\\
1192	0.5\\
1193	0.5\\
1194	0.5\\
1195	0.5\\
1196	0.5\\
1197	0.5\\
1198	0.5\\
1199	0.5\\
1200	0.5\\
1201	0.5\\
1202	0.5\\
1203	0.5\\
1204	1\\
1205	1\\
1206	0.5\\
1207	0.5\\
1208	0.5\\
1209	0.5\\
1210	0.5\\
1211	0.5\\
1212	0.5\\
1213	0.5\\
1214	0.5\\
1215	1\\
1216	1\\
1217	1\\
1218	0.5\\
1219	0.5\\
1220	0.5\\
1221	0.5\\
1222	0.5\\
1223	0.5\\
1224	0.5\\
1225	0.5\\
1226	0.5\\
1227	0.5\\
1228	0.5\\
1229	0.5\\
1230	1\\
1231	1\\
1232	0.5\\
1233	0.5\\
1234	0.5\\
1235	0.5\\
1236	0.5\\
1237	0.5\\
1238	0.5\\
1239	0.5\\
1240	0.5\\
1241	0.5\\
1242	0.5\\
1243	0.5\\
1244	0.5\\
1245	1\\
1246	1\\
1247	1\\
1248	1\\
1249	1\\
1250	0.5\\
1251	0.5\\
1252	0.5\\
1253	0.5\\
1254	0.5\\
1255	0.5\\
1256	0.5\\
1257	0.5\\
1258	0.5\\
1259	0.5\\
1260	0.5\\
1261	0.5\\
1262	0.5\\
1263	0.5\\
1264	0.5\\
1265	0.5\\
1266	0.5\\
1267	0.5\\
1268	0.5\\
1269	0.5\\
1270	0.5\\
1271	1\\
1272	1\\
1273	1\\
1274	0.5\\
1275	0.5\\
1276	0.5\\
1277	0.5\\
1278	0.5\\
1279	0.5\\
1280	0.5\\
1281	0.5\\
1282	0.5\\
1283	0.5\\
1284	0.5\\
1285	1\\
1286	1\\
1287	0.5\\
1288	0.5\\
1289	0.5\\
1290	0.5\\
1291	0.5\\
1292	0.5\\
1293	0.5\\
1294	0.5\\
1295	0.5\\
1296	0.5\\
1297	0.5\\
1298	0.5\\
1299	0.5\\
1300	0.5\\
1301	0.5\\
1302	1\\
1303	1\\
1304	1\\
1305	1\\
1306	0.5\\
1307	0.5\\
1308	0.5\\
1309	0.5\\
1310	0.5\\
1311	0.5\\
1312	0.5\\
1313	0.5\\
1314	0.5\\
1315	0.5\\
1316	1\\
1317	1\\
1318	0.5\\
1319	0.5\\
1320	0.5\\
1321	0.5\\
1322	0.5\\
1323	0.5\\
1324	0.5\\
1325	0.5\\
1326	0.5\\
1327	0.5\\
1328	0.5\\
1329	0.5\\
1330	1\\
1331	1\\
1332	0.5\\
1333	0.5\\
1334	0.5\\
1335	0.5\\
1336	0.5\\
1337	0.5\\
1338	0.5\\
1339	0.5\\
1340	0.5\\
1341	0.5\\
1342	0.5\\
1343	1\\
1344	1\\
1345	1\\
1346	0.5\\
1347	0.5\\
1348	0.5\\
1349	0.5\\
1350	0.5\\
1351	0.5\\
1352	0.5\\
1353	0.5\\
1354	0.5\\
1355	0.5\\
1356	0.5\\
1357	0.5\\
1358	1\\
1359	1\\
1360	0.5\\
1361	0.5\\
1362	0.5\\
1363	0.5\\
1364	0.5\\
1365	0.5\\
1366	0.5\\
1367	0.5\\
1368	0.5\\
1369	1\\
1370	1\\
1371	1\\
1372	1\\
1373	1\\
1374	1\\
1375	0.5\\
1376	0.5\\
1377	0.5\\
1378	0.5\\
1379	0.5\\
1380	0.5\\
1381	0.5\\
1382	0.5\\
1383	0.5\\
1384	0.5\\
1385	0.5\\
1386	0.5\\
1387	0.5\\
1388	1\\
1389	1\\
1390	0.5\\
1391	0.5\\
1392	0.5\\
1393	0.5\\
1394	0.5\\
1395	0.5\\
1396	0.5\\
1397	0.5\\
1398	0.5\\
1399	0.5\\
1400	0.5\\
1401	1\\
1402	1\\
1403	1\\
1404	1\\
1405	1\\
1406	1\\
1407	0.5\\
1408	0.5\\
1409	0.5\\
1410	0.5\\
1411	0.5\\
1412	0.5\\
1413	0.5\\
1414	0.5\\
1415	0.5\\
1416	0.5\\
1417	0.5\\
1418	0.5\\
1419	0.5\\
1420	0.5\\
1421	0.5\\
1422	0.5\\
1423	0.5\\
1424	0.5\\
1425	0.5\\
1426	0.5\\
1427	1\\
1428	1\\
1429	1\\
1430	0.5\\
1431	0.5\\
1432	0.5\\
1433	0.5\\
1434	0.5\\
1435	0.5\\
1436	0.5\\
1437	0.5\\
1438	0.5\\
1439	0.5\\
1440	0.5\\
1441	1\\
1442	1\\
1443	0.5\\
1444	0.5\\
1445	0.5\\
1446	0.5\\
1447	0.5\\
1448	0.5\\
1449	0.5\\
1450	0.5\\
1451	0.5\\
1452	0.5\\
1453	0.5\\
1454	0.5\\
1455	0.5\\
1456	1\\
1457	1\\
1458	1\\
1459	0.5\\
1460	0.5\\
1461	0.5\\
1462	0.5\\
1463	0.5\\
1464	0.5\\
1465	0.5\\
1466	0.5\\
1467	1\\
1468	1\\
1469	0.5\\
1470	0.5\\
1471	0.5\\
1472	0.5\\
1473	0.5\\
1474	0.5\\
1475	0.5\\
1476	0.5\\
1477	0.5\\
1478	0.5\\
1479	0.5\\
1480	0.5\\
1481	0.5\\
1482	1\\
1483	1\\
1484	0.5\\
1485	0.5\\
1486	0.5\\
1487	0.5\\
1488	0.5\\
1489	0.5\\
1490	0.5\\
1491	0.5\\
1492	0.5\\
1493	1\\
1494	1\\
1495	1\\
1496	0.5\\
1497	0.5\\
1498	0.5\\
1499	0.5\\
1500	0.5\\
1501	0.5\\
1502	0.5\\
1503	0.5\\
1504	0.5\\
1505	0.5\\
1506	0.5\\
1507	0.5\\
1508	0.5\\
1509	0.5\\
1510	1\\
1511	1\\
1512	0.5\\
1513	0.5\\
1514	0.5\\
1515	0.5\\
1516	0.5\\
1517	0.5\\
1518	0.5\\
1519	0.5\\
1520	0.5\\
1521	0.5\\
1522	1\\
1523	1\\
1524	1\\
1525	1\\
1526	1\\
1527	0.5\\
1528	0.5\\
1529	0.5\\
1530	0.5\\
1531	0.5\\
1532	0.5\\
1533	0.5\\
1534	0.5\\
1535	0.5\\
1536	0.5\\
1537	0.5\\
1538	0.5\\
1539	0.5\\
1540	0.5\\
1541	0.5\\
1542	0.5\\
1543	0.5\\
1544	0.5\\
1545	0.5\\
1546	0.5\\
1547	1\\
1548	1\\
1549	0.5\\
1550	0.5\\
1551	0.5\\
1552	0.5\\
1553	0.5\\
1554	0.5\\
1555	0.5\\
1556	0.5\\
1557	0.5\\
1558	0.5\\
1559	0.5\\
1560	1\\
1561	1\\
1562	0.5\\
1563	0.5\\
1564	0.5\\
1565	0.5\\
1566	0.5\\
1567	0.5\\
1568	0.5\\
1569	0.5\\
1570	0.5\\
1571	0.5\\
1572	0.5\\
1573	0.5\\
1574	0.5\\
1575	0.5\\
1576	0.5\\
1577	0.5\\
1578	0.5\\
1579	0.5\\
1580	0.5\\
1581	1\\
1582	1\\
1583	0.5\\
1584	0.5\\
1585	0.5\\
1586	0.5\\
1587	0.5\\
1588	0.5\\
1589	0.5\\
1590	0.5\\
1591	0.5\\
1592	1\\
1593	1\\
1594	0.5\\
1595	0.5\\
1596	0.5\\
1597	0.5\\
1598	0.5\\
1599	0.5\\
1600	0.5\\
1601	0.5\\
1602	0.5\\
1603	0.5\\
1604	0.5\\
1605	0.5\\
1606	0.5\\
1607	0.5\\
1608	1\\
1609	1\\
1610	0.5\\
1611	0.5\\
1612	0.5\\
1613	0.5\\
1614	0.5\\
1615	0.5\\
1616	0.5\\
1617	1\\
1618	1\\
1619	0.5\\
1620	0.5\\
1621	0.5\\
1622	0.5\\
1623	0.5\\
1624	0.5\\
1625	0.5\\
1626	0.5\\
1627	0.5\\
1628	0.5\\
1629	0.5\\
1630	0.5\\
1631	1\\
1632	1\\
1633	0.5\\
1634	0.5\\
1635	0.5\\
1636	0.5\\
1637	0.5\\
1638	0.5\\
1639	0.5\\
1640	0.5\\
1641	0.5\\
1642	0.5\\
1643	0.5\\
1644	0.5\\
1645	1\\
1646	1\\
1647	1\\
1648	1\\
1649	0.5\\
1650	0.5\\
1651	0.5\\
1652	0.5\\
1653	0.5\\
1654	0.5\\
1655	0.5\\
1656	0.5\\
1657	0.5\\
1658	0.5\\
1659	0.5\\
1660	0.5\\
1661	0.5\\
1662	1\\
1663	1\\
1664	0.5\\
1665	0.5\\
1666	0.5\\
1667	0.5\\
1668	0.5\\
1669	0.5\\
1670	0.5\\
1671	0.5\\
1672	0.5\\
1673	0.5\\
1674	0.5\\
1675	0.5\\
1676	0.5\\
1677	1\\
1678	1\\
1679	1\\
1680	1\\
1681	1\\
1682	1\\
1683	0.5\\
1684	0.5\\
1685	0.5\\
1686	0.5\\
1687	0.5\\
1688	0.5\\
1689	0.5\\
1690	0.5\\
1691	0.5\\
1692	0.5\\
1693	0.5\\
1694	0.5\\
1695	0.5\\
1696	0.5\\
1697	0.5\\
1698	0.5\\
1699	0.5\\
1700	0.5\\
1701	0.5\\
1702	0.5\\
1703	1\\
1704	1\\
1705	1\\
1706	1\\
1707	0.5\\
1708	0.5\\
1709	0.5\\
1710	0.5\\
1711	0.5\\
1712	0.5\\
1713	0.5\\
1714	0.5\\
1715	0.5\\
1716	0.5\\
1717	1\\
1718	1\\
1719	0.5\\
1720	0.5\\
1721	0.5\\
1722	0.5\\
1723	0.5\\
1724	0.5\\
1725	0.5\\
1726	0.5\\
1727	0.5\\
1728	0.5\\
1729	0.5\\
1730	0.5\\
1731	0.5\\
1732	1\\
1733	1\\
1734	0.5\\
1735	0.5\\
1736	0.5\\
1737	0.5\\
1738	0.5\\
1739	0.5\\
1740	0.5\\
1741	0.5\\
1742	0.5\\
1743	0.5\\
1744	0.5\\
1745	1\\
1746	1\\
1747	0.5\\
1748	0.5\\
1749	0.5\\
1750	0.5\\
1751	0.5\\
1752	0.5\\
1753	0.5\\
1754	0.5\\
1755	0.5\\
1756	0.5\\
1757	0.5\\
1758	0.5\\
1759	0.5\\
1760	1\\
1761	1\\
1762	0.5\\
1763	0.5\\
1764	0.5\\
1765	0.5\\
1766	0.5\\
1767	0.5\\
1768	0.5\\
1769	0.5\\
1770	0.5\\
1771	1\\
1772	1\\
1773	1\\
1774	1\\
1775	1\\
1776	1\\
1777	1\\
1778	0.5\\
1779	0.5\\
1780	0.5\\
1781	0.5\\
1782	0.5\\
1783	0.5\\
1784	0.5\\
1785	0.5\\
1786	0.5\\
1787	0.5\\
1788	0.5\\
1789	0.5\\
1790	1\\
1791	1\\
1792	0.5\\
1793	0.5\\
1794	0.5\\
1795	0.5\\
1796	0.5\\
1797	0.5\\
1798	0.5\\
1799	0.5\\
1800	0.5\\
1801	1\\
1802	1\\
1803	1\\
1804	1\\
1805	0.5\\
1806	0.5\\
1807	0.5\\
1808	0.5\\
1809	0.5\\
1810	0.5\\
1811	0.5\\
1812	0.5\\
1813	0.5\\
1814	0.5\\
1815	0.5\\
1816	0.5\\
1817	0.5\\
1818	0.5\\
1819	0.5\\
1820	0.5\\
1821	0.5\\
1822	0.5\\
1823	0.5\\
1824	0.5\\
1825	0.5\\
1826	0.5\\
1827	0.5\\
1828	1\\
1829	1\\
1830	1\\
1831	0.5\\
1832	0.5\\
1833	0.5\\
1834	0.5\\
1835	0.5\\
1836	0.5\\
1837	0.5\\
1838	0.5\\
1839	0.5\\
1840	0.5\\
1841	0.5\\
1842	0.5\\
1843	1\\
1844	1\\
1845	0.5\\
1846	0.5\\
1847	0.5\\
1848	0.5\\
1849	0.5\\
1850	0.5\\
1851	0.5\\
1852	0.5\\
1853	0.5\\
1854	0.5\\
1855	0.5\\
1856	0.5\\
1857	0.5\\
1858	0.5\\
1859	0.5\\
1860	0.5\\
1861	1\\
1862	1\\
1863	0.5\\
1864	0.5\\
1865	0.5\\
1866	0.5\\
1867	0.5\\
1868	0.5\\
1869	0.5\\
1870	0.5\\
1871	0.5\\
1872	1\\
1873	1\\
1874	0.5\\
1875	0.5\\
1876	0.5\\
1877	0.5\\
1878	0.5\\
1879	0.5\\
1880	0.5\\
1881	0.5\\
1882	0.5\\
1883	0.5\\
1884	0.5\\
1885	0.5\\
1886	0.5\\
1887	0.5\\
1888	0.5\\
1889	0.5\\
1890	0.5\\
1891	1\\
1892	1\\
1893	0.5\\
1894	0.5\\
1895	0.5\\
1896	0.5\\
1897	0.5\\
1898	0.5\\
1899	0.5\\
1900	1\\
1901	1\\
1902	0.5\\
1903	0.5\\
1904	0.5\\
1905	0.5\\
1906	0.5\\
1907	0.5\\
1908	0.5\\
1909	0.5\\
1910	0.5\\
1911	0.5\\
1912	0.5\\
1913	0.5\\
1914	0.5\\
1915	0.5\\
1916	0.5\\
1917	1\\
1918	1\\
1919	0.5\\
1920	0.5\\
1921	0.5\\
1922	0.5\\
1923	0.5\\
1924	0.5\\
1925	0.5\\
1926	0.5\\
1927	0.5\\
1928	1\\
1929	1\\
1930	1\\
1931	1\\
1932	0.5\\
1933	0.5\\
1934	0.5\\
1935	0.5\\
1936	0.5\\
1937	0.5\\
1938	0.5\\
1939	0.5\\
1940	0.5\\
1941	0.5\\
1942	0.5\\
1943	0.5\\
1944	0.5\\
1945	0.5\\
1946	1\\
1947	1\\
1948	0.5\\
1949	0.5\\
1950	0.5\\
1951	0.5\\
1952	0.5\\
1953	0.5\\
1954	0.5\\
1955	0.5\\
1956	0.5\\
1957	0.5\\
1958	0.5\\
1959	1\\
1960	1\\
1961	1\\
1962	1\\
1963	1\\
1964	1\\
1965	0.5\\
};
\end{axis}

\begin{axis}[%
width=4.133in,
height=0.863in,
at={(0.693in,1.639in)},
scale only axis,
xmin=0,
xmax=2000,
xmajorgrids,
ymin=0,
ymax=1,
ymajorgrids,
axis background/.style={fill=white}
]
\pgfplotsset{max space between ticks=50}
\addplot [color=mycolor3,solid,forget plot]
  table[row sep=crcr]{%
1	0\\
2	0.5\\
3	0.5\\
4	0.5\\
5	0.5\\
6	1\\
7	1\\
8	0\\
9	0\\
10	0.5\\
11	0.5\\
12	1\\
13	1\\
14	1\\
15	1\\
16	0\\
17	0\\
18	0\\
19	0\\
20	0\\
21	0\\
22	1\\
23	1\\
24	0.5\\
25	0.5\\
26	0.5\\
27	0.5\\
28	0.5\\
29	1\\
30	1\\
31	1\\
32	0\\
33	0\\
34	0\\
35	0\\
36	0\\
37	0\\
38	0.5\\
39	0.5\\
40	1\\
41	1\\
42	1\\
43	1\\
44	0\\
45	0\\
46	0\\
47	0\\
48	0\\
49	0\\
50	0\\
51	1\\
52	1\\
53	0.5\\
54	0.5\\
55	0.5\\
56	0.5\\
57	1\\
58	1\\
59	0\\
60	0\\
61	0\\
62	0\\
63	0\\
64	0.5\\
65	0.5\\
66	0.5\\
67	1\\
68	1\\
69	1\\
70	1\\
71	0\\
72	0\\
73	0\\
74	0\\
75	0\\
76	1\\
77	1\\
78	1\\
79	1\\
80	1\\
81	1\\
82	1\\
83	0.5\\
84	0.5\\
85	0.5\\
86	0.5\\
87	1\\
88	1\\
89	0\\
90	0\\
91	0\\
92	0\\
93	0\\
94	0.5\\
95	0.5\\
96	0.5\\
97	0.5\\
98	1\\
99	1\\
100	1\\
101	0\\
102	0\\
103	0\\
104	0\\
105	0\\
106	1\\
107	1\\
108	1\\
109	1\\
110	1\\
111	0.5\\
112	0.5\\
113	0.5\\
114	0.5\\
115	0.5\\
116	1\\
117	1\\
118	0\\
119	0\\
120	0\\
121	0\\
122	0\\
123	0\\
124	0\\
125	0.5\\
126	0.5\\
127	0.5\\
128	0.5\\
129	0.5\\
130	0.5\\
131	1\\
132	1\\
133	1\\
134	0\\
135	0\\
136	0\\
137	1\\
138	1\\
139	1\\
140	0\\
141	0\\
142	0\\
143	0\\
144	0\\
145	0\\
146	0\\
147	1\\
148	1\\
149	0.5\\
150	0.5\\
151	0.5\\
152	0.5\\
153	0.5\\
154	0.5\\
155	1\\
156	1\\
157	1\\
158	0\\
159	0\\
160	0\\
161	0\\
162	0\\
163	0\\
164	0\\
165	0\\
166	0.5\\
167	0.5\\
168	1\\
169	1\\
170	1\\
171	1\\
172	1\\
173	0\\
174	0\\
175	0\\
176	0\\
177	0\\
178	1\\
179	1\\
180	0.5\\
181	0.5\\
182	0.5\\
183	0.5\\
184	0.5\\
185	0.5\\
186	1\\
187	1\\
188	1\\
189	1\\
190	0\\
191	0\\
192	0\\
193	0\\
194	0\\
195	0\\
196	0.5\\
197	0.5\\
198	1\\
199	1\\
200	1\\
201	1\\
202	0\\
203	0\\
204	0\\
205	0\\
206	0\\
207	1\\
208	1\\
209	0.5\\
210	0.5\\
211	0.5\\
212	0.5\\
213	1\\
214	1\\
215	0\\
216	0\\
217	0\\
218	0\\
219	0.5\\
220	0.5\\
221	0.5\\
222	1\\
223	1\\
224	1\\
225	1\\
226	0\\
227	0\\
228	0\\
229	0\\
230	0\\
231	1\\
232	1\\
233	1\\
234	1\\
235	0.5\\
236	0.5\\
237	0.5\\
238	1\\
239	1\\
240	0\\
241	0\\
242	0\\
243	0\\
244	0\\
245	0\\
246	0.5\\
247	0.5\\
248	0.5\\
249	0.5\\
250	0.5\\
251	0.5\\
252	0.5\\
253	1\\
254	1\\
255	1\\
256	1\\
257	0\\
258	0\\
259	0\\
260	0\\
261	0\\
262	1\\
263	1\\
264	1\\
265	1\\
266	1\\
267	1\\
268	0.5\\
269	0.5\\
270	0.5\\
271	1\\
272	1\\
273	0\\
274	0\\
275	0\\
276	0\\
277	0\\
278	0\\
279	0.5\\
280	0.5\\
281	0.5\\
282	1\\
283	1\\
284	1\\
285	1\\
286	0\\
287	0\\
288	0\\
289	0.5\\
290	0.5\\
291	1\\
292	1\\
293	1\\
294	1\\
295	1\\
296	1\\
297	0\\
298	0\\
299	0\\
300	0\\
301	0\\
302	1\\
303	1\\
304	0.5\\
305	0.5\\
306	0.5\\
307	0.5\\
308	0.5\\
309	1\\
310	1\\
311	1\\
312	0\\
313	0\\
314	0\\
315	0\\
316	0\\
317	0\\
318	0\\
319	0\\
320	0.5\\
321	0.5\\
322	0.5\\
323	1\\
324	1\\
325	1\\
326	1\\
327	1\\
328	0\\
329	0\\
330	0\\
331	0\\
332	0\\
333	1\\
334	1\\
335	0.5\\
336	0.5\\
337	0.5\\
338	0.5\\
339	1\\
340	1\\
341	0\\
342	0\\
343	0\\
344	0\\
345	0\\
346	0.5\\
347	0.5\\
348	0.5\\
349	1\\
350	1\\
351	1\\
352	1\\
353	1\\
354	1\\
355	0\\
356	0\\
357	0\\
358	0\\
359	0\\
360	0\\
361	1\\
362	1\\
363	1\\
364	0.5\\
365	0.5\\
366	0.5\\
367	0.5\\
368	1\\
369	1\\
370	0\\
371	0\\
372	0\\
373	0\\
374	0\\
375	0.5\\
376	0.5\\
377	0.5\\
378	0.5\\
379	1\\
380	1\\
381	1\\
382	0\\
383	0\\
384	0\\
385	0\\
386	0\\
387	0\\
388	0\\
389	1\\
390	1\\
391	1\\
392	1\\
393	1\\
394	0.5\\
395	0.5\\
396	0.5\\
397	1\\
398	1\\
399	0\\
400	0\\
401	0\\
402	0\\
403	0\\
404	0\\
405	0\\
406	0.5\\
407	0.5\\
408	0.5\\
409	0.5\\
410	0.5\\
411	0.5\\
412	1\\
413	1\\
414	1\\
415	0\\
416	0\\
417	0\\
418	1\\
419	1\\
420	1\\
421	1\\
422	0\\
423	0\\
424	0\\
425	0\\
426	0\\
427	1\\
428	1\\
429	0.5\\
430	0.5\\
431	0.5\\
432	0.5\\
433	0.5\\
434	0.5\\
435	1\\
436	1\\
437	1\\
438	0\\
439	0\\
440	0\\
441	0\\
442	0\\
443	0\\
444	0\\
445	0\\
446	0.5\\
447	0.5\\
448	1\\
449	1\\
450	1\\
451	1\\
452	1\\
453	0\\
454	0\\
455	0\\
456	0\\
457	0\\
458	0\\
459	0\\
460	1\\
461	1\\
462	0.5\\
463	0.5\\
464	0.5\\
465	0.5\\
466	1\\
467	1\\
468	1\\
469	0\\
470	0\\
471	0\\
472	0\\
473	0\\
474	0\\
475	0.5\\
476	0.5\\
477	1\\
478	1\\
479	1\\
480	1\\
481	0\\
482	0\\
483	0\\
484	0\\
485	0\\
486	1\\
487	1\\
488	0.5\\
489	0.5\\
490	0.5\\
491	1\\
492	1\\
493	0\\
494	0\\
495	0\\
496	0\\
497	0.5\\
498	0.5\\
499	0.5\\
500	1\\
501	1\\
502	1\\
503	1\\
504	0\\
505	0\\
506	0\\
507	0\\
508	0\\
509	1\\
510	1\\
511	1\\
512	1\\
513	0.5\\
514	0.5\\
515	0.5\\
516	0.5\\
517	1\\
518	1\\
519	0\\
520	0\\
521	0\\
522	0\\
523	0\\
524	0\\
525	0.5\\
526	0.5\\
527	0.5\\
528	0.5\\
529	0.5\\
530	1\\
531	1\\
532	1\\
533	0\\
534	0\\
535	0\\
536	0\\
537	0\\
538	1\\
539	1\\
540	1\\
541	1\\
542	1\\
543	1\\
544	0.5\\
545	0.5\\
546	0.5\\
547	1\\
548	1\\
549	0\\
550	0\\
551	0\\
552	0\\
553	0\\
554	0\\
555	0\\
556	0.5\\
557	0.5\\
558	0.5\\
559	0.5\\
560	1\\
561	1\\
562	1\\
563	0\\
564	0\\
565	0\\
566	0.5\\
567	0.5\\
568	1\\
569	1\\
570	1\\
571	1\\
572	1\\
573	1\\
574	0\\
575	0\\
576	0\\
577	0\\
578	0\\
579	0\\
580	1\\
581	1\\
582	0.5\\
583	0.5\\
584	0.5\\
585	0.5\\
586	0.5\\
587	0.5\\
588	0.5\\
589	1\\
590	1\\
591	1\\
592	1\\
593	0\\
594	0\\
595	0\\
596	0\\
597	0\\
598	0\\
599	0.5\\
600	0.5\\
601	1\\
602	1\\
603	1\\
604	1\\
605	0\\
606	0\\
607	0\\
608	0\\
609	0\\
610	1\\
611	1\\
612	0.5\\
613	0.5\\
614	0.5\\
615	0.5\\
616	1\\
617	1\\
618	1\\
619	1\\
620	0\\
621	0\\
622	0\\
623	0\\
624	0\\
625	0\\
626	0\\
627	0.5\\
628	0.5\\
629	0.5\\
630	1\\
631	1\\
632	1\\
633	1\\
634	0\\
635	0\\
636	0\\
637	0\\
638	0\\
639	1\\
640	1\\
641	1\\
642	0.5\\
643	0.5\\
644	0.5\\
645	0.5\\
646	1\\
647	1\\
648	0\\
649	0\\
650	0\\
651	0\\
652	0\\
653	0.5\\
654	0.5\\
655	0.5\\
656	0.5\\
657	0.5\\
658	0.5\\
659	1\\
660	1\\
661	1\\
662	1\\
663	0\\
664	0\\
665	0\\
666	0\\
667	0\\
668	1\\
669	1\\
670	1\\
671	1\\
672	1\\
673	0.5\\
674	0.5\\
675	0.5\\
676	1\\
677	1\\
678	0\\
679	0\\
680	0\\
681	0\\
682	0\\
683	0\\
684	0.5\\
685	0.5\\
686	0.5\\
687	0.5\\
688	0.5\\
689	0.5\\
690	1\\
691	1\\
692	1\\
693	1\\
694	0\\
695	0\\
696	0\\
697	0\\
698	1\\
699	1\\
700	1\\
701	0\\
702	0\\
703	0\\
704	0\\
705	0\\
706	1\\
707	1\\
708	0.5\\
709	0.5\\
710	0.5\\
711	0.5\\
712	0.5\\
713	0.5\\
714	1\\
715	1\\
716	1\\
717	0\\
718	0\\
719	0\\
720	0\\
721	0\\
722	0\\
723	0\\
724	0\\
725	0\\
726	0.5\\
727	0.5\\
728	1\\
729	1\\
730	1\\
731	1\\
732	1\\
733	0\\
734	0\\
735	0\\
736	0\\
737	0\\
738	1\\
739	1\\
740	0.5\\
741	0.5\\
742	0.5\\
743	0.5\\
744	1\\
745	1\\
746	1\\
747	0\\
748	0\\
749	0\\
750	0\\
751	0\\
752	0.5\\
753	0.5\\
754	1\\
755	1\\
756	1\\
757	1\\
758	1\\
759	1\\
760	0\\
761	0\\
762	0\\
763	0\\
764	0\\
765	0\\
766	1\\
767	1\\
768	0.5\\
769	0.5\\
770	0.5\\
771	0.5\\
772	1\\
773	1\\
774	0\\
775	0\\
776	0\\
777	0\\
778	0.5\\
779	0.5\\
780	0.5\\
781	1\\
782	1\\
783	1\\
784	1\\
785	0\\
786	0\\
787	0\\
788	0\\
789	0\\
790	0\\
791	1\\
792	1\\
793	1\\
794	1\\
795	0.5\\
796	0.5\\
797	0.5\\
798	1\\
799	1\\
800	0\\
801	0\\
802	0\\
803	0\\
804	0\\
805	0\\
806	0.5\\
807	0.5\\
808	0.5\\
809	0.5\\
810	0.5\\
811	1\\
812	1\\
813	1\\
814	0\\
815	0\\
816	0\\
817	0\\
818	0\\
819	1\\
820	1\\
821	1\\
822	1\\
823	1\\
824	1\\
825	1\\
826	1\\
827	0.5\\
828	0.5\\
829	0.5\\
830	0.5\\
831	1\\
832	1\\
833	0\\
834	0\\
835	0\\
836	0\\
837	0\\
838	0\\
839	0.5\\
840	0.5\\
841	0.5\\
842	1\\
843	1\\
844	1\\
845	0\\
846	0\\
847	0\\
848	0.5\\
849	0.5\\
850	1\\
851	1\\
852	1\\
853	1\\
854	1\\
855	1\\
856	0\\
857	0\\
858	0\\
859	0\\
860	0\\
861	0\\
862	0\\
863	1\\
864	1\\
865	0.5\\
866	0.5\\
867	0.5\\
868	0.5\\
869	0.5\\
870	1\\
871	1\\
872	1\\
873	0\\
874	0\\
875	0\\
876	0\\
877	0\\
878	0\\
879	0.5\\
880	0.5\\
881	1\\
882	1\\
883	1\\
884	1\\
885	0\\
886	0\\
887	0\\
888	0\\
889	0\\
890	0\\
891	0\\
892	1\\
893	1\\
894	1\\
895	1\\
896	1\\
897	0.5\\
898	0.5\\
899	0.5\\
900	0.5\\
901	1\\
902	1\\
903	0\\
904	0\\
905	0\\
906	0\\
907	0\\
908	0.5\\
909	0.5\\
910	0.5\\
911	1\\
912	1\\
913	1\\
914	1\\
915	0\\
916	0\\
917	0\\
918	0\\
919	0\\
920	1\\
921	1\\
922	1\\
923	0.5\\
924	0.5\\
925	0.5\\
926	0.5\\
927	0.5\\
928	0.5\\
929	1\\
930	1\\
931	0\\
932	0\\
933	0\\
934	0\\
935	0\\
936	0.5\\
937	0.5\\
938	0.5\\
939	0.5\\
940	1\\
941	1\\
942	1\\
943	0\\
944	0\\
945	0\\
946	0\\
947	0\\
948	1\\
949	1\\
950	1\\
951	1\\
952	1\\
953	0.5\\
954	0.5\\
955	0.5\\
956	1\\
957	1\\
958	0\\
959	0\\
960	0\\
961	0\\
962	0\\
963	0\\
964	0\\
965	0\\
966	0\\
967	0.5\\
968	0.5\\
969	0.5\\
970	0.5\\
971	0.5\\
972	0.5\\
973	1\\
974	1\\
975	1\\
976	0\\
977	0\\
978	0\\
979	1\\
980	1\\
981	1\\
982	0\\
983	0\\
984	0\\
985	0\\
986	0\\
987	0\\
988	1\\
989	1\\
990	0.5\\
991	0.5\\
992	0.5\\
993	0.5\\
994	0.5\\
995	0.5\\
996	0.5\\
997	0.5\\
998	1\\
999	1\\
1000	1\\
1001	1\\
1002	0\\
1003	0\\
1004	0\\
1005	0\\
1006	0\\
1007	0\\
1008	0\\
1009	0\\
1010	0.5\\
1011	0.5\\
1012	1\\
1013	1\\
1014	1\\
1015	1\\
1016	1\\
1017	0\\
1018	0\\
1019	0\\
1020	0\\
1021	0\\
1022	1\\
1023	1\\
1024	0.5\\
1025	0.5\\
1026	0.5\\
1027	0.5\\
1028	1\\
1029	1\\
1030	1\\
1031	1\\
1032	0\\
1033	0\\
1034	0\\
1035	0\\
1036	0\\
1037	0\\
1038	0\\
1039	0.5\\
1040	0.5\\
1041	1\\
1042	1\\
1043	1\\
1044	1\\
1045	0\\
1046	0\\
1047	0\\
1048	0\\
1049	0\\
1050	1\\
1051	1\\
1052	0.5\\
1053	0.5\\
1054	0.5\\
1055	0.5\\
1056	1\\
1057	1\\
1058	0\\
1059	0\\
1060	0\\
1061	0\\
1062	0.5\\
1063	0.5\\
1064	0.5\\
1065	0.5\\
1066	0.5\\
1067	0.5\\
1068	0.5\\
1069	1\\
1070	1\\
1071	1\\
1072	1\\
1073	0\\
1074	0\\
1075	0\\
1076	0\\
1077	0\\
1078	1\\
1079	1\\
1080	1\\
1081	1\\
1082	0.5\\
1083	0.5\\
1084	0.5\\
1085	1\\
1086	1\\
1087	0\\
1088	0\\
1089	0\\
1090	0\\
1091	0\\
1092	0\\
1093	0.5\\
1094	0.5\\
1095	0.5\\
1096	0.5\\
1097	0.5\\
1098	1\\
1099	1\\
1100	1\\
1101	1\\
1102	0\\
1103	0\\
1104	0\\
1105	0\\
1106	0\\
1107	0\\
1108	1\\
1109	1\\
1110	1\\
1111	1\\
1112	1\\
1113	1\\
1114	0.5\\
1115	0.5\\
1116	0.5\\
1117	1\\
1118	1\\
1119	0\\
1120	0\\
1121	0\\
1122	0\\
1123	0\\
1124	0\\
1125	0.5\\
1126	0.5\\
1127	0.5\\
1128	1\\
1129	1\\
1130	1\\
1131	0\\
1132	0\\
1133	0\\
1134	0\\
1135	0.5\\
1136	0.5\\
1137	1\\
1138	1\\
1139	1\\
1140	1\\
1141	1\\
1142	1\\
1143	0\\
1144	0\\
1145	0\\
1146	0\\
1147	0\\
1148	0\\
1149	1\\
1150	1\\
1151	0.5\\
1152	0.5\\
1153	0.5\\
1154	0.5\\
1155	0.5\\
1156	1\\
1157	1\\
1158	1\\
1159	0\\
1160	0\\
1161	0\\
1162	0\\
1163	0\\
1164	0\\
1165	0.5\\
1166	0.5\\
1167	1\\
1168	1\\
1169	1\\
1170	1\\
1171	1\\
1172	1\\
1173	0\\
1174	0\\
1175	0\\
1176	0\\
1177	0\\
1178	0\\
1179	0\\
1180	1\\
1181	1\\
1182	0.5\\
1183	0.5\\
1184	0.5\\
1185	0.5\\
1186	1\\
1187	1\\
1188	0\\
1189	0\\
1190	0\\
1191	0\\
1192	0\\
1193	0.5\\
1194	0.5\\
1195	0.5\\
1196	1\\
1197	1\\
1198	1\\
1199	1\\
1200	0\\
1201	0\\
1202	0\\
1203	0\\
1204	0\\
1205	0\\
1206	1\\
1207	1\\
1208	1\\
1209	0.5\\
1210	0.5\\
1211	0.5\\
1212	0.5\\
1213	1\\
1214	1\\
1215	0\\
1216	0\\
1217	0\\
1218	0\\
1219	0\\
1220	0.5\\
1221	0.5\\
1222	0.5\\
1223	0.5\\
1224	1\\
1225	1\\
1226	1\\
1227	0\\
1228	0\\
1229	0\\
1230	0\\
1231	0\\
1232	1\\
1233	1\\
1234	1\\
1235	1\\
1236	1\\
1237	1\\
1238	1\\
1239	0.5\\
1240	0.5\\
1241	0.5\\
1242	0.5\\
1243	1\\
1244	1\\
1245	0\\
1246	0\\
1247	0\\
1248	0\\
1249	0\\
1250	0\\
1251	0\\
1252	0.5\\
1253	0.5\\
1254	0.5\\
1255	0.5\\
1256	0.5\\
1257	0.5\\
1258	1\\
1259	1\\
1260	1\\
1261	0\\
1262	0\\
1263	0\\
1264	1\\
1265	1\\
1266	1\\
1267	0\\
1268	0\\
1269	0\\
1270	0\\
1271	0\\
1272	0\\
1273	0\\
1274	1\\
1275	1\\
1276	0.5\\
1277	0.5\\
1278	0.5\\
1279	0.5\\
1280	0.5\\
1281	0.5\\
1282	1\\
1283	1\\
1284	1\\
1285	0\\
1286	0\\
1287	0\\
1288	0\\
1289	0\\
1290	0\\
1291	0\\
1292	0.5\\
1293	0.5\\
1294	1\\
1295	1\\
1296	1\\
1297	1\\
1298	1\\
1299	0\\
1300	0\\
1301	0\\
1302	0\\
1303	0\\
1304	0\\
1305	0\\
1306	1\\
1307	1\\
1308	1\\
1309	0.5\\
1310	0.5\\
1311	0.5\\
1312	0.5\\
1313	0.5\\
1314	1\\
1315	1\\
1316	0\\
1317	0\\
1318	0\\
1319	0\\
1320	0\\
1321	0.5\\
1322	0.5\\
1323	1\\
1324	1\\
1325	1\\
1326	1\\
1327	0\\
1328	0\\
1329	0\\
1330	0\\
1331	0\\
1332	1\\
1333	1\\
1334	1\\
1335	0.5\\
1336	0.5\\
1337	0.5\\
1338	0.5\\
1339	0.5\\
1340	0.5\\
1341	1\\
1342	1\\
1343	0\\
1344	0\\
1345	0\\
1346	0\\
1347	0\\
1348	0.5\\
1349	0.5\\
1350	0.5\\
1351	1\\
1352	1\\
1353	1\\
1354	1\\
1355	0\\
1356	0\\
1357	0\\
1358	0\\
1359	0\\
1360	1\\
1361	1\\
1362	1\\
1363	1\\
1364	0.5\\
1365	0.5\\
1366	0.5\\
1367	1\\
1368	1\\
1369	0\\
1370	0\\
1371	0\\
1372	0\\
1373	0\\
1374	0\\
1375	0\\
1376	0\\
1377	0.5\\
1378	0.5\\
1379	0.5\\
1380	0.5\\
1381	0.5\\
1382	1\\
1383	1\\
1384	1\\
1385	0\\
1386	0\\
1387	0\\
1388	0\\
1389	0\\
1390	1\\
1391	1\\
1392	1\\
1393	1\\
1394	1\\
1395	1\\
1396	0.5\\
1397	0.5\\
1398	0.5\\
1399	1\\
1400	1\\
1401	0\\
1402	0\\
1403	0\\
1404	0\\
1405	0\\
1406	0\\
1407	0.5\\
1408	0.5\\
1409	0.5\\
1410	0.5\\
1411	1\\
1412	1\\
1413	1\\
1414	0\\
1415	0\\
1416	0.5\\
1417	0.5\\
1418	1\\
1419	1\\
1420	1\\
1421	1\\
1422	1\\
1423	1\\
1424	0\\
1425	0\\
1426	0\\
1427	0\\
1428	0\\
1429	0\\
1430	1\\
1431	1\\
1432	0.5\\
1433	0.5\\
1434	0.5\\
1435	0.5\\
1436	0.5\\
1437	1\\
1438	1\\
1439	1\\
1440	1\\
1441	0\\
1442	0\\
1443	0\\
1444	0\\
1445	0\\
1446	0\\
1447	0.5\\
1448	0.5\\
1449	1\\
1450	1\\
1451	1\\
1452	1\\
1453	0\\
1454	0\\
1455	0\\
1456	0\\
1457	0\\
1458	0\\
1459	1\\
1460	1\\
1461	0.5\\
1462	0.5\\
1463	0.5\\
1464	0.5\\
1465	1\\
1466	1\\
1467	0\\
1468	0\\
1469	0\\
1470	0\\
1471	0\\
1472	0.5\\
1473	0.5\\
1474	0.5\\
1475	1\\
1476	1\\
1477	1\\
1478	1\\
1479	0\\
1480	0\\
1481	0\\
1482	0\\
1483	0\\
1484	1\\
1485	1\\
1486	1\\
1487	0.5\\
1488	0.5\\
1489	0.5\\
1490	0.5\\
1491	1\\
1492	1\\
1493	0\\
1494	0\\
1495	0\\
1496	0\\
1497	0\\
1498	0.5\\
1499	0.5\\
1500	0.5\\
1501	0.5\\
1502	1\\
1503	1\\
1504	1\\
1505	1\\
1506	0\\
1507	0\\
1508	0\\
1509	0\\
1510	0\\
1511	0\\
1512	1\\
1513	1\\
1514	1\\
1515	1\\
1516	1\\
1517	0.5\\
1518	0.5\\
1519	0.5\\
1520	1\\
1521	1\\
1522	0\\
1523	0\\
1524	0\\
1525	0\\
1526	0\\
1527	0\\
1528	0\\
1529	0.5\\
1530	0.5\\
1531	0.5\\
1532	0.5\\
1533	0.5\\
1534	0.5\\
1535	1\\
1536	1\\
1537	1\\
1538	0\\
1539	0\\
1540	0\\
1541	0\\
1542	1\\
1543	1\\
1544	1\\
1545	0\\
1546	0\\
1547	0\\
1548	0\\
1549	1\\
1550	1\\
1551	0.5\\
1552	0.5\\
1553	0.5\\
1554	0.5\\
1555	0.5\\
1556	0.5\\
1557	1\\
1558	1\\
1559	1\\
1560	0\\
1561	0\\
1562	0\\
1563	0\\
1564	0\\
1565	0\\
1566	0\\
1567	0\\
1568	0.5\\
1569	0.5\\
1570	1\\
1571	1\\
1572	1\\
1573	1\\
1574	1\\
1575	1\\
1576	1\\
1577	0\\
1578	0\\
1579	0\\
1580	0\\
1581	0\\
1582	0\\
1583	1\\
1584	1\\
1585	0.5\\
1586	0.5\\
1587	0.5\\
1588	0.5\\
1589	1\\
1590	1\\
1591	1\\
1592	0\\
1593	0\\
1594	0\\
1595	0\\
1596	0\\
1597	0\\
1598	0.5\\
1599	0.5\\
1600	1\\
1601	1\\
1602	1\\
1603	1\\
1604	0\\
1605	0\\
1606	0\\
1607	0\\
1608	0\\
1609	0\\
1610	1\\
1611	1\\
1612	0.5\\
1613	0.5\\
1614	0.5\\
1615	1\\
1616	1\\
1617	0\\
1618	0\\
1619	0\\
1620	0\\
1621	0.5\\
1622	0.5\\
1623	0.5\\
1624	1\\
1625	1\\
1626	1\\
1627	1\\
1628	0\\
1629	0\\
1630	0\\
1631	0\\
1632	0\\
1633	1\\
1634	1\\
1635	1\\
1636	1\\
1637	1\\
1638	1\\
1639	0.5\\
1640	0.5\\
1641	0.5\\
1642	0.5\\
1643	1\\
1644	1\\
1645	0\\
1646	0\\
1647	0\\
1648	0\\
1649	0\\
1650	0\\
1651	0.5\\
1652	0.5\\
1653	0.5\\
1654	0.5\\
1655	0.5\\
1656	1\\
1657	1\\
1658	1\\
1659	0\\
1660	0\\
1661	0\\
1662	0\\
1663	0\\
1664	1\\
1665	1\\
1666	1\\
1667	1\\
1668	1\\
1669	1\\
1670	0.5\\
1671	0.5\\
1672	0.5\\
1673	0.5\\
1674	0.5\\
1675	1\\
1676	1\\
1677	0\\
1678	0\\
1679	0\\
1680	0\\
1681	0\\
1682	0\\
1683	0.5\\
1684	0.5\\
1685	0.5\\
1686	1\\
1687	1\\
1688	1\\
1689	0\\
1690	0\\
1691	0\\
1692	0.5\\
1693	0.5\\
1694	1\\
1695	1\\
1696	1\\
1697	1\\
1698	1\\
1699	1\\
1700	0\\
1701	0\\
1702	0\\
1703	0\\
1704	0\\
1705	0\\
1706	0\\
1707	1\\
1708	1\\
1709	0.5\\
1710	0.5\\
1711	0.5\\
1712	0.5\\
1713	0.5\\
1714	1\\
1715	1\\
1716	1\\
1717	0\\
1718	0\\
1719	0\\
1720	0\\
1721	0\\
1722	0\\
1723	0.5\\
1724	0.5\\
1725	1\\
1726	1\\
1727	1\\
1728	1\\
1729	0\\
1730	0\\
1731	0\\
1732	0\\
1733	0\\
1734	1\\
1735	1\\
1736	0.5\\
1737	0.5\\
1738	0.5\\
1739	0.5\\
1740	0.5\\
1741	0.5\\
1742	1\\
1743	1\\
1744	1\\
1745	0\\
1746	0\\
1747	0\\
1748	0\\
1749	0\\
1750	0.5\\
1751	0.5\\
1752	0.5\\
1753	1\\
1754	1\\
1755	1\\
1756	1\\
1757	0\\
1758	0\\
1759	0\\
1760	0\\
1761	0\\
1762	1\\
1763	1\\
1764	1\\
1765	0.5\\
1766	0.5\\
1767	0.5\\
1768	0.5\\
1769	1\\
1770	1\\
1771	0\\
1772	0\\
1773	0\\
1774	0\\
1775	0\\
1776	0\\
1777	0\\
1778	0\\
1779	0\\
1780	0.5\\
1781	0.5\\
1782	0.5\\
1783	0.5\\
1784	1\\
1785	1\\
1786	1\\
1787	0\\
1788	0\\
1789	0\\
1790	0\\
1791	0\\
1792	1\\
1793	1\\
1794	1\\
1795	1\\
1796	0.5\\
1797	0.5\\
1798	0.5\\
1799	1\\
1800	1\\
1801	0\\
1802	0\\
1803	0\\
1804	0\\
1805	0\\
1806	0\\
1807	0.5\\
1808	0.5\\
1809	0.5\\
1810	0.5\\
1811	0.5\\
1812	0.5\\
1813	0.5\\
1814	0.5\\
1815	1\\
1816	1\\
1817	1\\
1818	1\\
1819	0\\
1820	0\\
1821	0\\
1822	1\\
1823	1\\
1824	1\\
1825	0\\
1826	0\\
1827	0\\
1828	0\\
1829	0\\
1830	0\\
1831	1\\
1832	1\\
1833	0.5\\
1834	0.5\\
1835	0.5\\
1836	0.5\\
1837	0.5\\
1838	0.5\\
1839	1\\
1840	1\\
1841	1\\
1842	1\\
1843	0\\
1844	0\\
1845	0\\
1846	0\\
1847	0\\
1848	0\\
1849	0\\
1850	0\\
1851	0.5\\
1852	0.5\\
1853	1\\
1854	1\\
1855	1\\
1856	1\\
1857	1\\
1858	0\\
1859	0\\
1860	0\\
1861	0\\
1862	0\\
1863	1\\
1864	1\\
1865	0.5\\
1866	0.5\\
1867	0.5\\
1868	0.5\\
1869	1\\
1870	1\\
1871	1\\
1872	0\\
1873	0\\
1874	0\\
1875	0\\
1876	0\\
1877	0\\
1878	0\\
1879	0.5\\
1880	0.5\\
1881	0.5\\
1882	0.5\\
1883	0.5\\
1884	1\\
1885	1\\
1886	1\\
1887	1\\
1888	0\\
1889	0\\
1890	0\\
1891	0\\
1892	0\\
1893	1\\
1894	1\\
1895	0.5\\
1896	0.5\\
1897	0.5\\
1898	1\\
1899	1\\
1900	0\\
1901	0\\
1902	0\\
1903	0\\
1904	0.5\\
1905	0.5\\
1906	0.5\\
1907	1\\
1908	1\\
1909	1\\
1910	1\\
1911	1\\
1912	1\\
1913	0\\
1914	0\\
1915	0\\
1916	0\\
1917	0\\
1918	0\\
1919	1\\
1920	1\\
1921	1\\
1922	1\\
1923	0.5\\
1924	0.5\\
1925	0.5\\
1926	1\\
1927	1\\
1928	0\\
1929	0\\
1930	0\\
1931	0\\
1932	0\\
1933	0\\
1934	0.5\\
1935	0.5\\
1936	0.5\\
1937	0.5\\
1938	0.5\\
1939	1\\
1940	1\\
1941	1\\
1942	0\\
1943	0\\
1944	0\\
1945	0\\
1946	0\\
1947	0\\
1948	1\\
1949	1\\
1950	1\\
1951	1\\
1952	1\\
1953	1\\
1954	0.5\\
1955	0.5\\
1956	0.5\\
1957	1\\
1958	1\\
1959	0\\
1960	0\\
1961	0\\
1962	0\\
1963	0\\
1964	0\\
1965	0.5\\
};
\end{axis}
\end{tikzpicture}%
}
      \caption{The calculated schedule for the three penduli.
        \texttt{Blue}: $P_1$, \texttt{Red}: $P_2$, \texttt{Orange}: $P_3$.
        $C_i = 10$ ms.}
      \label{fig:01.5.3}
    \end{figure}
  \end{minipage}
  \hfill
  \begin{minipage}{0.45\linewidth}
    \begin{figure}[H]\centering
    \scalebox{0.7}{% This file was created by matlab2tikz.
%
%The latest updates can be retrieved from
%  http://www.mathworks.com/matlabcentral/fileexchange/22022-matlab2tikz-matlab2tikz
%where you can also make suggestions and rate matlab2tikz.
%
\definecolor{mycolor1}{rgb}{0.00000,0.44700,0.74100}%
%
\begin{tikzpicture}

\begin{axis}[%
width=4.133in,
height=3.26in,
at={(0.693in,0.44in)},
scale only axis,
xmin=0,
xmax=1700,
xmajorgrids,
ymin=0,
ymax=1.25,
ymajorgrids,
axis background/.style={fill=white}
]
\addplot [color=mycolor1,solid,forget plot]
  table[row sep=crcr]{%
1	0\\
2	1\\
3	1\\
4	1\\
5	1\\
6	0.75\\
7	0.75\\
8	0.75\\
9	0.75\\
10	1\\
11	1\\
12	0.75\\
13	0.75\\
14	1\\
15	1\\
16	1\\
17	1\\
18	1\\
19	1\\
20	0.75\\
21	0.75\\
22	1\\
23	1\\
24	1\\
25	1\\
26	1\\
27	1\\
28	1\\
29	1\\
30	1\\
31	1\\
32	1\\
33	1\\
34	1\\
35	1\\
36	1\\
37	1\\
38	1\\
39	1\\
40	1\\
41	1\\
42	1\\
43	1\\
44	1\\
45	1\\
46	1\\
47	1\\
48	1\\
49	1\\
50	1\\
51	1\\
52	1\\
53	1\\
54	1\\
55	1\\
56	1\\
57	1\\
58	1\\
59	1\\
60	1\\
61	1\\
62	1\\
63	1\\
64	1\\
65	1\\
66	1\\
67	1\\
68	1\\
69	1\\
70	1\\
71	1\\
72	1\\
73	1\\
74	1\\
75	1\\
76	1\\
77	1\\
78	1\\
79	1\\
80	1\\
81	1\\
82	1\\
83	1\\
84	1\\
85	1\\
86	1\\
87	1\\
88	1\\
89	1\\
90	1\\
91	1\\
92	1\\
93	1\\
94	1\\
95	1\\
96	1\\
97	1\\
98	1\\
99	1\\
100	1\\
101	1\\
102	1\\
103	1\\
104	1\\
105	1\\
106	1\\
107	1\\
108	1\\
109	1\\
110	1\\
111	1\\
112	1\\
113	1\\
114	1\\
115	1\\
116	1\\
117	1\\
118	1\\
119	1\\
120	1\\
121	1\\
122	1\\
123	1\\
124	1\\
125	1\\
126	1\\
127	1\\
128	1\\
129	1\\
130	1\\
131	1\\
132	1\\
133	1\\
134	1\\
135	1\\
136	1\\
137	1\\
138	1\\
139	1\\
140	1\\
141	1\\
142	1\\
143	1\\
144	1\\
145	1\\
146	1\\
147	1\\
148	1\\
149	1\\
150	1\\
151	1\\
152	1\\
153	1\\
154	1\\
155	1\\
156	1\\
157	1\\
158	1\\
159	1\\
160	1\\
161	1\\
162	1\\
163	1\\
164	1\\
165	1\\
166	1\\
167	1\\
168	1\\
169	1\\
170	1\\
171	1\\
172	1\\
173	1\\
174	1\\
175	1\\
176	1\\
177	1\\
178	1\\
179	1\\
180	1\\
181	1\\
182	1\\
183	1\\
184	1\\
185	1\\
186	1\\
187	1\\
188	1\\
189	1\\
190	1\\
191	1\\
192	1\\
193	1\\
194	1\\
195	1\\
196	1\\
197	1\\
198	1\\
199	1\\
200	1\\
201	1\\
202	1\\
203	1\\
204	1\\
205	1\\
206	1\\
207	1\\
208	1\\
209	1\\
210	1\\
211	1\\
212	1\\
213	1\\
214	1\\
215	1\\
216	1\\
217	1\\
218	1\\
219	1\\
220	1\\
221	1\\
222	1\\
223	1\\
224	1\\
225	1\\
226	1\\
227	1\\
228	1\\
229	1\\
230	1\\
231	1\\
232	1\\
233	1\\
234	1\\
235	1\\
236	1\\
237	1\\
238	1\\
239	1\\
240	1\\
241	1\\
242	1\\
243	1\\
244	1\\
245	1\\
246	1\\
247	1\\
248	1\\
249	1\\
250	1\\
251	1\\
252	1\\
253	1\\
254	1\\
255	1\\
256	1\\
257	1\\
258	1\\
259	1\\
260	1\\
261	1\\
262	1\\
263	1\\
264	1\\
265	1\\
266	1\\
267	1\\
268	1\\
269	1\\
270	1\\
271	1\\
272	1\\
273	1\\
274	1\\
275	1\\
276	1\\
277	1\\
278	1\\
279	1\\
280	1\\
281	1\\
282	1\\
283	1\\
284	1\\
285	1\\
286	1\\
287	1\\
288	1\\
289	1\\
290	1\\
291	1\\
292	1\\
293	1\\
294	1\\
295	1\\
296	1\\
297	1\\
298	1\\
299	1\\
300	1\\
301	1\\
302	1\\
303	1\\
304	1\\
305	1\\
306	1\\
307	1\\
308	1\\
309	1\\
310	1\\
311	1\\
312	1\\
313	1\\
314	1\\
315	1\\
316	1\\
317	1\\
318	1\\
319	1\\
320	1\\
321	1\\
322	1\\
323	1\\
324	1\\
325	1\\
326	1\\
327	1\\
328	1\\
329	1\\
330	1\\
331	1\\
332	1\\
333	1\\
334	1\\
335	1\\
336	1\\
337	1\\
338	1\\
339	1\\
340	1\\
341	1\\
342	1\\
343	1\\
344	1\\
345	1\\
346	1\\
347	1\\
348	1\\
349	1\\
350	1\\
351	1\\
352	1\\
353	1\\
354	1\\
355	1\\
356	1\\
357	1\\
358	1\\
359	1\\
360	1\\
361	1\\
362	1\\
363	1\\
364	1\\
365	1\\
366	1\\
367	1\\
368	1\\
369	1\\
370	1\\
371	1\\
372	1\\
373	1\\
374	1\\
375	1\\
376	1\\
377	1\\
378	1\\
379	1\\
380	1\\
381	1\\
382	1\\
383	1\\
384	1\\
385	1\\
386	1\\
387	1\\
388	1\\
389	1\\
390	1\\
391	1\\
392	1\\
393	1\\
394	1\\
395	1\\
396	1\\
397	1\\
398	1\\
399	1\\
400	1\\
401	1\\
402	1\\
403	1\\
404	1\\
405	1\\
406	1\\
407	1\\
408	1\\
409	1\\
410	1\\
411	1\\
412	1\\
413	1\\
414	1\\
415	1\\
416	1\\
417	1\\
418	1\\
419	1\\
420	1\\
421	1\\
422	1\\
423	1\\
424	1\\
425	1\\
426	1\\
427	1\\
428	1\\
429	1\\
430	1\\
431	1\\
432	1\\
433	1\\
434	1\\
435	1\\
436	1\\
437	1\\
438	1\\
439	1\\
440	1\\
441	1\\
442	1\\
443	1\\
444	1\\
445	1\\
446	1\\
447	1\\
448	1\\
449	1\\
450	1\\
451	1\\
452	1\\
453	1\\
454	1\\
455	1\\
456	1\\
457	1\\
458	1\\
459	1\\
460	1\\
461	1\\
462	1\\
463	1\\
464	1\\
465	1\\
466	1\\
467	1\\
468	1\\
469	1\\
470	1\\
471	1\\
472	1\\
473	1\\
474	1\\
475	1\\
476	1\\
477	1\\
478	1\\
479	1\\
480	1\\
481	1\\
482	1\\
483	1\\
484	1\\
485	1\\
486	1\\
487	1\\
488	1\\
489	1\\
490	1\\
491	1\\
492	1\\
493	1\\
494	1\\
495	1\\
496	1\\
497	1\\
498	1\\
499	1\\
500	1\\
501	1\\
502	1\\
503	1\\
504	1\\
505	1\\
506	1\\
507	1\\
508	1\\
509	1\\
510	1\\
511	1\\
512	1\\
513	1\\
514	1\\
515	1\\
516	1\\
517	1\\
518	1\\
519	1\\
520	1\\
521	1\\
522	1\\
523	1\\
524	1\\
525	1\\
526	1\\
527	1\\
528	1\\
529	1\\
530	1\\
531	1\\
532	1\\
533	1\\
534	1\\
535	1\\
536	1\\
537	1\\
538	1\\
539	1\\
540	1\\
541	1\\
542	1\\
543	1\\
544	1\\
545	1\\
546	1\\
547	1\\
548	1\\
549	1\\
550	1\\
551	1\\
552	1\\
553	1\\
554	1\\
555	1\\
556	1\\
557	1\\
558	1\\
559	1\\
560	1\\
561	1\\
562	1\\
563	1\\
564	1\\
565	1\\
566	1\\
567	1\\
568	1\\
569	1\\
570	1\\
571	1\\
572	1\\
573	1\\
574	1\\
575	1\\
576	1\\
577	1\\
578	1\\
579	1\\
580	1\\
581	1\\
582	1\\
583	1\\
584	1\\
585	1\\
586	1\\
587	1\\
588	1\\
589	1\\
590	1\\
591	1\\
592	1\\
593	1\\
594	1\\
595	1\\
596	1\\
597	1\\
598	1\\
599	1\\
600	1\\
601	1\\
602	1\\
603	1\\
604	1\\
605	1\\
606	1\\
607	1\\
608	1\\
609	1\\
610	1\\
611	1\\
612	1\\
613	1\\
614	1\\
615	1\\
616	1\\
617	1\\
618	1\\
619	1\\
620	1\\
621	1\\
622	1\\
623	1\\
624	1\\
625	1\\
626	1\\
627	1\\
628	1\\
629	1\\
630	1\\
631	1\\
632	1\\
633	1\\
634	1\\
635	1\\
636	1\\
637	1\\
638	1\\
639	1\\
640	1\\
641	1\\
642	1\\
643	1\\
644	1\\
645	1\\
646	1\\
647	1\\
648	1\\
649	1\\
650	1\\
651	1\\
652	1\\
653	1\\
654	1\\
655	1\\
656	1\\
657	1\\
658	1\\
659	1\\
660	1\\
661	1\\
662	1\\
663	1\\
664	1\\
665	1\\
666	1\\
667	1\\
668	1\\
669	1\\
670	1\\
671	1\\
672	1\\
673	1\\
674	1\\
675	1\\
676	1\\
677	1\\
678	1\\
679	1\\
680	1\\
681	1\\
682	1\\
683	1\\
684	1\\
685	1\\
686	1\\
687	1\\
688	1\\
689	1\\
690	1\\
691	1\\
692	1\\
693	1\\
694	1\\
695	1\\
696	1\\
697	1\\
698	1\\
699	1\\
700	1\\
701	1\\
702	1\\
703	1\\
704	1\\
705	1\\
706	1\\
707	1\\
708	1\\
709	1\\
710	1\\
711	1\\
712	1\\
713	1\\
714	1\\
715	1\\
716	1\\
717	1\\
718	1\\
719	1\\
720	1\\
721	1\\
722	1\\
723	1\\
724	1\\
725	1\\
726	1\\
727	1\\
728	1\\
729	1\\
730	1\\
731	1\\
732	1\\
733	1\\
734	1\\
735	1\\
736	1\\
737	1\\
738	1\\
739	1\\
740	1\\
741	1\\
742	1\\
743	1\\
744	1\\
745	1\\
746	1\\
747	1\\
748	1\\
749	1\\
750	1\\
751	1\\
752	1\\
753	1\\
754	1\\
755	1\\
756	1\\
757	1\\
758	1\\
759	1\\
760	1\\
761	1\\
762	1\\
763	1\\
764	1\\
765	1\\
766	1\\
767	1\\
768	1\\
769	1\\
770	1\\
771	1\\
772	1\\
773	1\\
774	1\\
775	1\\
776	1\\
777	1\\
778	1\\
779	1\\
780	1\\
781	1\\
782	1\\
783	1\\
784	1\\
785	1\\
786	1\\
787	1\\
788	1\\
789	1\\
790	1\\
791	1\\
792	1\\
793	1\\
794	1\\
795	1\\
796	1\\
797	1\\
798	1\\
799	1\\
800	1\\
801	1\\
802	1\\
803	1\\
804	1\\
805	1\\
806	1\\
807	1\\
808	1\\
809	1\\
810	1\\
811	1\\
812	1\\
813	1\\
814	1\\
815	1\\
816	1\\
817	1\\
818	1\\
819	1\\
820	1\\
821	1\\
822	1\\
823	1\\
824	1\\
825	1\\
826	1\\
827	1\\
828	1\\
829	1\\
830	1\\
831	1\\
832	1\\
833	1\\
834	1\\
835	1\\
836	1\\
837	1\\
838	1\\
839	1\\
840	1\\
841	1\\
842	1\\
843	1\\
844	1\\
845	1\\
846	1\\
847	1\\
848	1\\
849	1\\
850	1\\
851	1\\
852	1\\
853	1\\
854	1\\
855	1\\
856	1\\
857	1\\
858	1\\
859	1\\
860	1\\
861	1\\
862	1\\
863	1\\
864	1\\
865	1\\
866	1\\
867	1\\
868	1\\
869	1\\
870	1\\
871	1\\
872	1\\
873	1\\
874	1\\
875	1\\
876	1\\
877	1\\
878	1\\
879	1\\
880	1\\
881	1\\
882	1\\
883	1\\
884	1\\
885	1\\
886	1\\
887	1\\
888	1\\
889	1\\
890	1\\
891	1\\
892	1\\
893	1\\
894	1\\
895	1\\
896	1\\
897	1\\
898	1\\
899	1\\
900	1\\
901	1\\
902	1\\
903	1\\
904	1\\
905	1\\
906	1\\
907	1\\
908	1\\
909	1\\
910	1\\
911	1\\
912	1\\
913	1\\
914	1\\
915	1\\
916	1\\
917	1\\
918	1\\
919	1\\
920	1\\
921	1\\
922	1\\
923	1\\
924	1\\
925	1\\
926	1\\
927	1\\
928	1\\
929	1\\
930	1\\
931	1\\
932	1\\
933	1\\
934	1\\
935	1\\
936	1\\
937	1\\
938	1\\
939	1\\
940	1\\
941	1\\
942	1\\
943	1\\
944	1\\
945	1\\
946	1\\
947	1\\
948	1\\
949	1\\
950	1\\
951	1\\
952	1\\
953	1\\
954	1\\
955	1\\
956	1\\
957	1\\
958	1\\
959	1\\
960	1\\
961	1\\
962	1\\
963	1\\
964	1\\
965	1\\
966	1\\
967	1\\
968	1\\
969	1\\
970	1\\
971	1\\
972	1\\
973	1\\
974	1\\
975	1\\
976	1\\
977	1\\
978	1\\
979	1\\
980	1\\
981	1\\
982	1\\
983	1\\
984	1\\
985	1\\
986	1\\
987	1\\
988	1\\
989	1\\
990	1\\
991	1\\
992	1\\
993	1\\
994	1\\
995	1\\
996	1\\
997	1\\
998	1\\
999	1\\
1000	1\\
1001	1\\
1002	1\\
1003	1\\
1004	1\\
1005	1\\
1006	1\\
1007	1\\
1008	1\\
1009	1\\
1010	1\\
1011	1\\
1012	1\\
1013	1\\
1014	1\\
1015	1\\
1016	1\\
1017	1\\
1018	1\\
1019	1\\
1020	1\\
1021	1\\
1022	1\\
1023	1\\
1024	1\\
1025	1\\
1026	1\\
1027	1\\
1028	1\\
1029	1\\
1030	1\\
1031	1\\
1032	1\\
1033	1\\
1034	1\\
1035	1\\
1036	1\\
1037	1\\
1038	1\\
1039	1\\
1040	1\\
1041	1\\
1042	1\\
1043	1\\
1044	1\\
1045	1\\
1046	1\\
1047	1\\
1048	1\\
1049	1\\
1050	1\\
1051	1\\
1052	1\\
1053	1\\
1054	1\\
1055	1\\
1056	1\\
1057	1\\
1058	1\\
1059	1\\
1060	1\\
1061	1\\
1062	1\\
1063	1\\
1064	1\\
1065	1\\
1066	1\\
1067	1\\
1068	1\\
1069	1\\
1070	1\\
1071	1\\
1072	1\\
1073	1\\
1074	1\\
1075	1\\
1076	1\\
1077	1\\
1078	1\\
1079	1\\
1080	1\\
1081	1\\
1082	1\\
1083	1\\
1084	1\\
1085	1\\
1086	1\\
1087	1\\
1088	1\\
1089	1\\
1090	1\\
1091	1\\
1092	1\\
1093	1\\
1094	1\\
1095	1\\
1096	1\\
1097	1\\
1098	1\\
1099	1\\
1100	1\\
1101	1\\
1102	1\\
1103	1\\
1104	1\\
1105	1\\
1106	1\\
1107	1\\
1108	1\\
1109	1\\
1110	1\\
1111	1\\
1112	1\\
1113	1\\
1114	1\\
1115	1\\
1116	1\\
1117	1\\
1118	1\\
1119	1\\
1120	1\\
1121	1\\
1122	1\\
1123	1\\
1124	1\\
1125	1\\
1126	1\\
1127	1\\
1128	1\\
1129	1\\
1130	1\\
1131	1\\
1132	1\\
1133	1\\
1134	1\\
1135	1\\
1136	1\\
1137	1\\
1138	1\\
1139	1\\
1140	1\\
1141	1\\
1142	1\\
1143	1\\
1144	1\\
1145	1\\
1146	1\\
1147	1\\
1148	1\\
1149	1\\
1150	1\\
1151	1\\
1152	1\\
1153	1\\
1154	1\\
1155	1\\
1156	1\\
1157	1\\
1158	1\\
1159	1\\
1160	1\\
1161	1\\
1162	1\\
1163	1\\
1164	1\\
1165	1\\
1166	1\\
1167	1\\
1168	1\\
1169	1\\
1170	1\\
1171	1\\
1172	1\\
1173	1\\
1174	1\\
1175	1\\
1176	1\\
1177	1\\
1178	1\\
1179	1\\
1180	1\\
1181	1\\
1182	1\\
1183	1\\
1184	1\\
1185	1\\
1186	1\\
1187	1\\
1188	1\\
1189	1\\
1190	1\\
1191	1\\
1192	1\\
1193	1\\
1194	1\\
1195	1\\
1196	1\\
1197	1\\
1198	1\\
1199	1\\
1200	1\\
1201	1\\
1202	1\\
1203	1\\
1204	1\\
1205	1\\
1206	1\\
1207	1\\
1208	1\\
1209	1\\
1210	1\\
1211	1\\
1212	1\\
1213	1\\
1214	1\\
1215	1\\
1216	1\\
1217	1\\
1218	1\\
1219	1\\
1220	1\\
1221	1\\
1222	1\\
1223	1\\
1224	1\\
1225	1\\
1226	1\\
1227	1\\
1228	1\\
1229	1\\
1230	1\\
1231	1\\
1232	1\\
1233	1\\
1234	1\\
1235	1\\
1236	1\\
1237	1\\
1238	1\\
1239	1\\
1240	1\\
1241	1\\
1242	1\\
1243	1\\
1244	1\\
1245	1\\
1246	1\\
1247	1\\
1248	1\\
1249	1\\
1250	1\\
1251	1\\
1252	1\\
1253	1\\
1254	1\\
1255	1\\
1256	1\\
1257	1\\
1258	1\\
1259	1\\
1260	1\\
1261	1\\
1262	1\\
1263	1\\
1264	1\\
1265	1\\
1266	1\\
1267	1\\
1268	1\\
1269	1\\
1270	1\\
1271	1\\
1272	1\\
1273	1\\
1274	1\\
1275	1\\
1276	1\\
1277	1\\
1278	1\\
1279	1\\
1280	1\\
1281	1\\
1282	1\\
1283	1\\
1284	1\\
1285	1\\
1286	1\\
1287	1\\
1288	1\\
1289	1\\
1290	1\\
1291	1\\
1292	1\\
1293	1\\
1294	1\\
1295	1\\
1296	1\\
1297	1\\
1298	1\\
1299	1\\
1300	1\\
1301	1\\
1302	1\\
1303	1\\
1304	1\\
1305	1\\
1306	1\\
1307	1\\
1308	1\\
1309	1\\
1310	1\\
1311	1\\
1312	1\\
1313	1\\
1314	1\\
1315	1\\
1316	1\\
1317	1\\
1318	1\\
1319	1\\
1320	1\\
1321	1\\
1322	1\\
1323	1\\
1324	1\\
1325	1\\
1326	1\\
1327	1\\
1328	1\\
1329	1\\
1330	1\\
1331	1\\
1332	1\\
1333	1\\
1334	1\\
1335	1\\
1336	1\\
1337	1\\
1338	1\\
1339	1\\
1340	1\\
1341	1\\
1342	1\\
1343	1\\
1344	1\\
1345	1\\
1346	1\\
1347	1\\
1348	1\\
1349	1\\
1350	1\\
1351	1\\
1352	1\\
1353	1\\
1354	1\\
1355	1\\
1356	1\\
1357	1\\
1358	1\\
1359	1\\
1360	1\\
1361	1\\
1362	1\\
1363	1\\
1364	1\\
1365	1\\
1366	1\\
1367	1\\
1368	1\\
1369	1\\
1370	1\\
1371	1\\
1372	1\\
1373	1\\
1374	1\\
1375	1\\
1376	1\\
1377	1\\
1378	1\\
1379	1\\
1380	1\\
1381	1\\
1382	1\\
1383	1\\
1384	1\\
1385	1\\
1386	1\\
1387	1\\
1388	1\\
1389	1\\
1390	1\\
1391	1\\
1392	1\\
1393	1\\
1394	1\\
1395	1\\
1396	1\\
1397	1\\
1398	1\\
1399	1\\
1400	1\\
1401	1\\
1402	1\\
1403	1\\
1404	1\\
1405	1\\
1406	1\\
1407	1\\
1408	1\\
1409	1\\
1410	1\\
1411	1\\
1412	1\\
1413	1\\
1414	1\\
1415	1\\
1416	1\\
1417	1\\
1418	1\\
1419	1\\
1420	1\\
1421	1\\
1422	1\\
1423	1\\
1424	1\\
1425	1\\
1426	1\\
1427	1\\
1428	1\\
1429	1\\
1430	1\\
1431	1\\
1432	1\\
1433	1\\
1434	1\\
1435	1\\
1436	1\\
1437	1\\
1438	1\\
1439	1\\
1440	1\\
1441	1\\
1442	1\\
1443	1\\
1444	1\\
1445	1\\
1446	1\\
1447	1\\
1448	1\\
1449	1\\
1450	1\\
1451	1\\
1452	1\\
1453	1\\
1454	1\\
1455	1\\
1456	1\\
1457	1\\
1458	1\\
1459	1\\
1460	1\\
1461	1\\
1462	1\\
1463	1\\
1464	1\\
1465	1\\
1466	1\\
1467	1\\
1468	1\\
1469	1\\
1470	1\\
1471	1\\
1472	1\\
1473	1\\
1474	1\\
1475	1\\
1476	1\\
1477	1\\
1478	1\\
1479	1\\
1480	1\\
1481	1\\
1482	1\\
1483	1\\
1484	1\\
1485	1\\
1486	1\\
1487	1\\
1488	1\\
1489	1\\
1490	1\\
1491	1\\
1492	1\\
1493	1\\
1494	1\\
1495	1\\
1496	1\\
1497	1\\
1498	1\\
1499	1\\
1500	1\\
1501	1\\
1502	1\\
1503	1\\
1504	1\\
1505	1\\
1506	1\\
1507	1\\
1508	1\\
1509	1\\
1510	1\\
1511	1\\
1512	1\\
1513	1\\
1514	1\\
1515	1\\
1516	1\\
1517	1\\
1518	1\\
1519	1\\
1520	1\\
1521	1\\
1522	1\\
1523	1\\
1524	1\\
1525	1\\
1526	1\\
1527	1\\
1528	1\\
1529	1\\
1530	1\\
1531	1\\
1532	1\\
1533	1\\
1534	1\\
1535	1\\
1536	1\\
1537	1\\
1538	1\\
1539	1\\
1540	1\\
1541	1\\
1542	1\\
1543	1\\
1544	1\\
1545	1\\
1546	1\\
1547	1\\
1548	1\\
1549	1\\
1550	1\\
1551	1\\
1552	1\\
1553	1\\
1554	1\\
1555	1\\
1556	1\\
1557	1\\
1558	1\\
1559	1\\
1560	1\\
1561	1\\
1562	1\\
1563	1\\
1564	1\\
1565	1\\
1566	1\\
1567	1\\
1568	1\\
1569	1\\
1570	1\\
1571	1\\
1572	1\\
1573	1\\
1574	1\\
1575	1\\
1576	1\\
1577	1\\
1578	1\\
1579	1\\
1580	1\\
1581	1\\
1582	1\\
1583	1\\
1584	1\\
1585	1\\
1586	1\\
1587	1\\
1588	1\\
1589	1\\
1590	1\\
1591	1\\
1592	1\\
1593	1\\
1594	1\\
1595	1\\
1596	1\\
1597	1\\
1598	1\\
1599	1\\
1600	1\\
1601	1\\
1602	1\\
1603	1\\
1604	1\\
1605	1\\
1606	1\\
1607	1\\
1608	1\\
1609	1\\
1610	1\\
1611	1\\
1612	1\\
1613	1\\
1614	1\\
1615	1\\
1616	1\\
1617	1\\
1618	1\\
1619	1\\
1620	1\\
1621	1\\
1622	1\\
1623	1\\
1624	1\\
1625	1\\
1626	1\\
1627	1\\
1628	1\\
1629	1\\
1630	1\\
1631	1\\
1632	1\\
1633	1\\
1634	1\\
1635	1\\
1636	1\\
1637	1\\
1638	1\\
1639	1\\
1640	1\\
1641	1\\
1642	1\\
1643	1\\
1644	1\\
1645	1\\
1646	1\\
1647	1\\
1648	1\\
1649	1\\
1650	1\\
1651	1\\
1652	1\\
1653	1\\
1654	1\\
1655	1\\
1656	1\\
1657	1\\
1658	1\\
1659	1\\
1660	1\\
1661	1\\
1662	1\\
1663	1\\
1664	1\\
1665	1\\
1666	1\\
1667	1\\
1668	1\\
1669	1\\
1670	1\\
1671	1\\
1672	1\\
1673	1\\
1674	1\\
1675	1\\
1676	1\\
1677	1\\
1678	1\\
1679	1\\
1680	1\\
1681	1\\
1682	1\\
1683	1\\
1684	1\\
1685	1\\
1686	1\\
1687	1\\
1688	1\\
1689	1\\
1690	1\\
1691	1\\
1692	1\\
1693	1\\
1694	1\\
1695	1\\
1696	1\\
1697	1\\
1698	1\\
1699	1\\
1700	1\\
1701	1\\
1702	1\\
1703	1\\
1704	1\\
1705	1\\
1706	1\\
1707	1\\
1708	1\\
1709	1\\
1710	1\\
1711	1\\
1712	1\\
1713	1\\
1714	1\\
1715	1\\
1716	1\\
1717	1\\
1718	1\\
1719	1\\
1720	1\\
1721	1\\
1722	1\\
1723	1\\
1724	1\\
1725	1\\
1726	1\\
1727	1\\
1728	1\\
1729	1\\
1730	1\\
1731	1\\
1732	1\\
1733	1\\
1734	1\\
1735	1\\
1736	1\\
1737	1\\
1738	1\\
};
\end{axis}
\end{tikzpicture}%
}
    \caption{The overall processing usage. Notice that it is always at $100\%$.
      $C_i = 10$ ms.}
      \label{fig:01.5.4}
  \end{figure}
\end{minipage}
\end{minipage}
}
