\subsubsection{Question 5}
In the case where $T_1 = 20, T_2 = 29, T_3 = 35$ ms and $C_i = 10$ ms,
$i=\{1,2,3\}$, $U=1.131 > 1$. Hence tasks $J_1, J_2, J_3$ are not schedulable
in any scheduling scheme.


\begin{figure}[H]\centering
  \scalebox{0.7}{\begin{ganttchart}[vgrid, hgrid]{0}{59}
%\gantttitle{2016}{12}\\
\gantttitlelist{10,20,...,60}{10}\\
\ganttset{progress label text={},
       bar incomplete/.append style={fill=black!40},
       group/.append style={draw=black, fill=black},}
\ganttbar{Task 1}{0}{9}
\ganttbar{}{20}{29}
\ganttbar{}{40}{49}\\

\ganttbar[progress=00]{Task 2}{0}{9}
\ganttbar{}{10}{19}
\ganttbar[progress=00]{}{29}{29}
\ganttbar{}{30}{39}
\ganttbar{}{58}{59}\\

\ganttbar[progress=00]{Task 3}{0}{49}
\ganttbar{}{50}{57}
\end{ganttchart}
}
  \caption{A portion of the RM schedule $\sigma$ for tasks $J_1, J_2, J_3$ for
    $C_i = 10$ ms.  Shaded areas denote the waiting time. Notice that $J_3$
    misses its deadlines consecutively, indicative of the inability of
    schedulability.}
\end{figure}

All penduli are still stable. However, due to the increased execution time,
delays are introduced, and the control input is not as swift as before,
hence the increased magnitude of the overshoot and the larger rise and settling
times.  Figure \ref{fig:01.5.2} shows the angular displacement of each pendulum
as a function of time.
Figures \ref{fig:01.5.6_10.1}, \ref{fig:01.5.6_10.2} and \ref{fig:01.5.6_10.3}
show the angular displacement of penduli $P_1$, $P_2$ and $P_3$ respectively, as
a function of time, for the two different cases of execution time.

\begin{figure}[H]\centering
  \scalebox{1}{% This file was created by matlab2tikz.
%
%The latest updates can be retrieved from
%  http://www.mathworks.com/matlabcentral/fileexchange/22022-matlab2tikz-matlab2tikz
%where you can also make suggestions and rate matlab2tikz.
%
\definecolor{mycolor1}{rgb}{0.92900,0.69400,0.12500}%
%
\begin{tikzpicture}

\begin{axis}[%
width=4.133in,
height=3.26in,
at={(0.693in,0.44in)},
scale only axis,
xmin=0,
xmax=5,
xmajorgrids,
ymin=-2000,
ymax=1500,
ymajorgrids,
axis background/.style={fill=white}
]
\addplot [color=mycolor1,solid,forget plot]
  table[row sep=crcr]{%
0	0.10153\\
3.15544362088405e-30	0.10153\\
0.000656101980281985	0.101530709989553\\
0.00393661188169191	0.101555560666546\\
0.00999999999999994	0.101694978093407\\
0.01	0.101694978093407\\
0.0199999999999999	0.102190448599525\\
0.02	0.102190448599525\\
0.0289999999999998	0.10292025163065\\
0.029	0.10292025163065\\
0.03	0.103018021716247\\
0.0300000000000002	0.103018021716247\\
0.0349999999999996	0.103557144832189\\
0.035	0.103557144832189\\
0.0399999999999993	0.104180386895362\\
0.04	0.104180386895362\\
0.0449999999999993	0.104888254163751\\
0.0499999999999987	0.105681321634387\\
0.05	0.105681321634387\\
0.0500000000000004	0.105681321634387\\
0.0579999999999996	0.107129080035817\\
0.058	0.107129080035817\\
0.0599999999999996	0.107525703737019\\
0.06	0.10752570373702\\
0.0619999999999995	0.10793630145586\\
0.0639999999999991	0.108360926555653\\
0.0679999999999982	0.109252481465054\\
0.0699999999999991	0.109719527143604\\
0.07	0.109719527143604\\
0.0779999999999982	0.111730935951441\\
0.0779999999999991	0.111730935951441\\
0.078	0.111730935951441\\
0.0799999999999991	0.112269921430949\\
0.08	0.11226992143095\\
0.0819999999999991	0.112763106359183\\
0.0839999999999982	0.113150163707717\\
0.0869999999999991	0.11353186678435\\
0.087	0.11353186678435\\
0.0899999999999991	0.113675004370016\\
0.09	0.113675004370016\\
0.0929999999999991	0.113579618320055\\
0.0959999999999982	0.113245680742359\\
0.0999999999999991	0.112429171660036\\
0.1	0.112429171660036\\
0.104999999999999	0.110811233393404\\
0.105	0.110811233393404\\
0.109999999999999	0.108528374543326\\
0.11	0.108528374543325\\
0.114999999999999	0.105578740754987\\
0.116	0.104908599711548\\
0.116000000000001	0.104908599711548\\
0.12	0.101959936054865\\
0.120000000000001	0.101959936054864\\
0.124	0.0985811321931027\\
0.127999999999999	0.0947704316341674\\
0.129999999999999	0.0927025098129282\\
0.130000000000001	0.0927025098129263\\
0.135999999999999	0.0858451907872454\\
0.136000000000001	0.0858451907872433\\
0.139999999999998	0.0807260106725555\\
0.14	0.0807260106725532\\
0.143999999999997	0.0752643057479906\\
0.144999999999998	0.0738604447744381\\
0.145	0.0738604447744356\\
0.148999999999997	0.0680898935446486\\
0.149999999999998	0.0666082445487197\\
0.15	0.066608244548717\\
0.153999999999997	0.060524159623913\\
0.157999999999995	0.054185701809464\\
0.159999999999998	0.0509200585468566\\
0.16	0.0509200585468537\\
0.167999999999995	0.0372047217761775\\
0.17	0.0336104685896896\\
0.170000000000002	0.0336104685896864\\
0.173999999999998	0.0262202198374446\\
0.174	0.0262202198374413\\
0.174999999999998	0.0243302786345852\\
0.175	0.0243302786345818\\
0.176	0.0224232638445833\\
0.177	0.0204991135084127\\
0.179000000000001	0.0165991555786198\\
0.179999999999998	0.0146232212758293\\
0.18	0.0146232212758258\\
0.184000000000001	0.00654494393444982\\
0.188000000000002	-0.00181576772216314\\
0.189999999999998	-0.00610338888733696\\
0.19	-0.0061033888873408\\
0.193999999999998	-0.0148959479911154\\
0.194	-0.0148959479911193\\
0.197999999999998	-0.0239820875537749\\
0.199999999999998	-0.0286367200824862\\
0.2	-0.0286367200824903\\
0.202999999999998	-0.0356216599767858\\
0.203	-0.0356216599767899\\
0.205999999999998	-0.0425014757150018\\
0.208999999999997	-0.0492781790488795\\
0.209999999999998	-0.0515145096526779\\
0.21	-0.0515145096526819\\
0.215999999999997	-0.0647005700563264\\
0.22	-0.0732756144567898\\
0.220000000000002	-0.0732756144567936\\
0.225999999999998	-0.0858258322153811\\
0.23	-0.0939907545960864\\
0.230000000000002	-0.09399075459609\\
0.231999999999998	-0.0980142484351137\\
0.232	-0.0980142484351173\\
0.233999999999996	-0.101999129580276\\
0.235999999999993	-0.105945915914193\\
0.239999999999986	-0.113727250958525\\
0.24	-0.113727250958552\\
0.240000000000002	-0.113727250958556\\
0.244999999999998	-0.123248670105518\\
0.245	-0.123248670105521\\
0.249999999999997	-0.132549244011439\\
0.249999999999999	-0.132549244011443\\
0.250000000000002	-0.132549244011448\\
0.251999999999996	-0.136209335081351\\
0.252	-0.136209335081358\\
0.253999999999995	-0.139835777288441\\
0.255999999999989	-0.143429041927733\\
0.259999999999979	-0.150517902171795\\
0.259999999999995	-0.150517902171823\\
0.26	-0.150517902171832\\
0.260999999999997	-0.15225563377892\\
0.261	-0.152255633778926\\
0.262	-0.153956530556074\\
0.263	-0.155620647774332\\
0.265	-0.158838758607568\\
0.269	-0.164835924650592\\
0.269999999999996	-0.166244032129369\\
0.27	-0.166244032129374\\
0.278	-0.176202794642548\\
0.279999999999996	-0.178331152662204\\
0.28	-0.178331152662208\\
0.288	-0.185407553163871\\
0.289999999999996	-0.186818544970816\\
0.29	-0.186818544970818\\
0.298	-0.191035580628601\\
0.299999999999996	-0.191733791720097\\
0.3	-0.191733791720098\\
0.308	-0.193105167224107\\
0.309999999999996	-0.193092866674311\\
0.31	-0.193092866674311\\
0.314999999999997	-0.192440670770938\\
0.315	-0.192440670770937\\
0.319	-0.191279396172207\\
0.319000000000004	-0.191279396172205\\
0.319999999999996	-0.190900186614032\\
0.32	-0.190900186614031\\
0.321	-0.190485397784907\\
0.322	-0.19003501620839\\
0.324	-0.189027415125274\\
0.328	-0.186584378608725\\
0.329999999999996	-0.185148625672876\\
0.33	-0.185148625672873\\
0.338	-0.177972811872225\\
0.339000000000004	-0.176914168873303\\
0.339000000000007	-0.1769141688733\\
0.339999999999996	-0.175819492201667\\
0.34	-0.175819492201663\\
0.340999999999996	-0.174685981624939\\
0.341999999999993	-0.173510835640693\\
0.343999999999986	-0.171035483375173\\
0.347999999999972	-0.165583383239669\\
0.348	-0.165583383239628\\
0.348000000000004	-0.165583383239623\\
0.349999999999996	-0.162605926800683\\
0.35	-0.162605926800678\\
0.351999999999993	-0.159460357471399\\
0.353999999999986	-0.156146266440133\\
0.357999999999972	-0.14901077452004\\
0.359999999999997	-0.145188446283621\\
0.36	-0.145188446283614\\
0.367999999999972	-0.128190068614893\\
0.369999999999997	-0.123510446626802\\
0.37	-0.123510446626793\\
0.376999999999997	-0.105763674610664\\
0.377	-0.105763674610655\\
0.379999999999997	-0.0975014778131437\\
0.38	-0.0975014778131337\\
0.382999999999997	-0.0888419844604141\\
0.384999999999997	-0.0828470229282403\\
0.385	-0.0828470229282295\\
0.387999999999997	-0.0735196839028996\\
0.39	-0.0670770148926004\\
0.390000000000004	-0.0670770148925888\\
0.393	-0.0570742536439874\\
0.395999999999997	-0.0466623743513427\\
0.397	-0.0431003100340245\\
0.397000000000004	-0.0431003100340117\\
0.399999999999997	-0.0321381831924003\\
0.4	-0.0321381831923871\\
0.402999999999993	-0.0209602098955915\\
0.405999999999986	-0.0097636847588793\\
0.405999999999997	-0.00976368475883907\\
0.406	-0.0097636847588256\\
0.409999999999997	0.00519953354445829\\
0.41	0.0051995335444716\\
0.413999999999997	0.0202093328138158\\
0.417999999999993	0.0352735159956755\\
0.419999999999997	0.0428284476381991\\
0.42	0.0428284476382125\\
0.427999999999993	0.0732233765165155\\
0.429999999999997	0.0808708469932227\\
0.43	0.0808708469932363\\
0.434999999999997	0.100085675137216\\
0.435	0.100085675137229\\
0.439999999999997	0.119450363314798\\
0.44	0.119450363314812\\
0.444999999999997	0.13898064141711\\
0.449999999999993	0.158692373790983\\
0.45	0.15869237379101\\
0.450000000000004	0.158692373791024\\
0.454999999999997	0.178601572391443\\
0.455	0.178601572391457\\
0.459999999999993	0.198724408934762\\
0.46	0.198724408934791\\
0.463999999999997	0.214510111453112\\
0.464	0.214510111453125\\
0.467999999999997	0.229496477696437\\
0.469999999999997	0.236692360448482\\
0.47	0.236692360448495\\
0.473999999999997	0.250494201203688\\
0.477999999999993	0.263515412235318\\
0.479999999999997	0.269735412764392\\
0.48	0.269735412764403\\
0.487999999999993	0.292696618189936\\
0.489999999999997	0.297960950451934\\
0.49	0.297960950451943\\
0.492999999999997	0.305503314066251\\
0.493	0.30550331406626\\
0.495999999999997	0.3126226671372\\
0.498999999999993	0.31932109151426\\
0.499999999999997	0.321460702011282\\
0.5	0.32146070201129\\
0.505999999999993	0.33332457677778\\
0.509999999999993	0.340311037804838\\
0.51	0.34031103780485\\
0.512999999999993	0.345069052576379\\
0.513	0.34506905257639\\
0.515999999999993	0.349415626407198\\
0.518999999999986	0.353352030303049\\
0.519999999999993	0.354573218459742\\
0.52	0.354573218459751\\
0.521999999999993	0.3568172160382\\
0.522	0.356817216038208\\
0.523999999999993	0.358755481653791\\
0.524999999999993	0.359610045246914\\
0.525	0.35961004524692\\
0.526999999999993	0.361090172827164\\
0.528999999999986	0.362265123763994\\
0.529999999999993	0.362738207257721\\
0.53	0.362738207257724\\
0.533999999999986	0.363868257811395\\
0.537999999999972	0.363779006335195\\
0.539999999999993	0.363277152598834\\
0.54	0.363277152598832\\
0.547999999999972	0.358219566734414\\
0.549999999999993	0.356191805948065\\
0.55	0.356191805948057\\
0.550999999999993	0.355063263613392\\
0.551	0.355063263613383\\
0.551999999999993	0.353858231928013\\
0.552999999999987	0.352576671737976\\
0.554999999999973	0.349783796806194\\
0.558999999999945	0.343277706075811\\
0.559999999999993	0.341459141078453\\
0.56	0.34145914107844\\
0.567999999999945	0.324135604123746\\
0.569999999999993	0.319031279175552\\
0.57	0.319031279175533\\
0.570999999999993	0.316362668760094\\
0.571	0.316362668760075\\
0.571999999999994	0.313616311630947\\
0.572999999999987	0.310792118549719\\
0.574999999999973	0.304909854970503\\
0.578999999999945	0.292207089264476\\
0.579999999999993	0.288835333301811\\
0.58	0.288835333301786\\
0.587999999999945	0.259976682388602\\
0.589999999999993	0.25226740296423\\
0.59	0.252267402964202\\
0.594999999999993	0.232117772367339\\
0.595	0.232117772367309\\
0.599999999999993	0.210702879780491\\
0.6	0.21070287978046\\
0.604999999999993	0.188005329948896\\
0.608999999999993	0.168911422516029\\
0.609	0.168911422515995\\
0.61	0.164006685760408\\
0.610000000000007	0.164006685760373\\
0.611000000000007	0.159049128908176\\
0.612000000000007	0.15403859088565\\
0.614000000000006	0.143857918439717\\
0.618000000000005	0.122853520722752\\
0.619999999999993	0.112027065660601\\
0.62	0.112027065660562\\
0.627999999999998	0.0665263756434592\\
0.629999999999993	0.0545950940860759\\
0.63	0.0545950940860332\\
0.637999999999998	0.00459900914228533\\
0.638000000000005	0.00459900914223929\\
0.639999999999993	-0.00847587366334768\\
0.64	-0.00847587366339454\\
0.641999999999989	-0.0217844543124832\\
0.643999999999977	-0.0353284624618757\\
0.647999999999954	-0.0631298327932696\\
0.649999999999993	-0.0773908081212815\\
0.65	-0.0773908081213326\\
0.657999999999954	-0.13688025805494\\
0.657999999999992	-0.136880258055235\\
0.658000000000005	-0.136880258055333\\
0.659999999999993	-0.152373671789546\\
0.66	-0.152373671789602\\
0.661999999999989	-0.167889695271649\\
0.663999999999977	-0.183200556040986\\
0.664999999999993	-0.190779673257465\\
0.665	-0.190779673257518\\
0.666999999999993	-0.205786512569255\\
0.667	-0.205786512569308\\
0.668999999999993	-0.220593114226993\\
0.669999999999993	-0.227921928279256\\
0.67	-0.227921928279308\\
0.671999999999993	-0.242431773392186\\
0.673999999999986	-0.256746143266472\\
0.677999999999971	-0.284795873308059\\
0.679999999999993	-0.298534878904419\\
0.68	-0.298534878904468\\
0.687999999999971	-0.351626549726854\\
0.689999999999993	-0.364442004338809\\
0.69	-0.364442004338854\\
0.695999999999993	-0.401817200563687\\
0.696	-0.40181720056373\\
0.699999999999993	-0.425857492173165\\
0.7	-0.425857492173207\\
0.703999999999993	-0.449211212388264\\
0.707999999999986	-0.471890501801733\\
0.709999999999993	-0.482980932923491\\
0.71	-0.482980932923531\\
0.716	-0.515272603730171\\
0.716000000000007	-0.515272603730208\\
0.719999999999993	-0.535997968593761\\
0.72	-0.535997968593797\\
0.723999999999986	-0.555382856129729\\
0.724999999999993	-0.55990971349352\\
0.725000000000001	-0.559909713493552\\
0.728999999999986	-0.576744097265804\\
0.729999999999993	-0.580635115469667\\
0.73	-0.580635115469694\\
0.733999999999986	-0.594932669963687\\
0.734999999999993	-0.598191009662175\\
0.735	-0.598191009662198\\
0.738999999999986	-0.609963348709664\\
0.739999999999993	-0.612591656635755\\
0.74	-0.612591656635773\\
0.743999999999986	-0.621848342848618\\
0.747999999999972	-0.629098018456068\\
0.749999999999993	-0.631971445757439\\
0.75	-0.631971445757449\\
0.753999999999993	-0.63621734647068\\
0.754	-0.636217346470686\\
0.757999999999993	-0.638463706343158\\
0.759999999999993	-0.638837463999925\\
0.76	-0.638837463999926\\
0.763999999999993	-0.638086377625853\\
0.767999999999986	-0.63533672208812\\
0.769999999999993	-0.633212024786585\\
0.77	-0.633212024786577\\
0.773999999999993	-0.627461510454347\\
0.774000000000001	-0.627461510454335\\
0.777999999999994	-0.619706903551507\\
0.779999999999993	-0.615076846362501\\
0.78	-0.615076846362483\\
0.782999999999993	-0.607200712453666\\
0.783	-0.607200712453646\\
0.785999999999993	-0.598214500247316\\
0.788999999999986	-0.588115582053105\\
0.79	-0.584501506969077\\
0.790000000000007	-0.584501506969051\\
0.795999999999993	-0.560207050492983\\
0.8	-0.541515156298028\\
0.800000000000007	-0.541515156297994\\
0.804999999999993	-0.515325782927863\\
0.805	-0.515325782927823\\
0.809999999999987	-0.485978095610198\\
0.809999999999997	-0.485978095610134\\
0.810000000000007	-0.485978095610069\\
0.811999999999993	-0.473349439465818\\
0.812	-0.473349439465772\\
0.813999999999987	-0.460210026591657\\
0.815999999999973	-0.44655814938274\\
0.819999999999945	-0.417709838348682\\
0.819999999999987	-0.417709838348369\\
0.82	-0.417709838348265\\
0.827999999999944	-0.353781088382634\\
0.829999999999993	-0.336488523526132\\
0.830000000000001	-0.33648852352607\\
0.831999999999994	-0.318667415505939\\
0.832000000000001	-0.318667415505875\\
0.833999999999994	-0.300315448398262\\
0.835999999999987	-0.281430237087253\\
0.839999999999973	-0.242050194961912\\
0.839999999999987	-0.242050194961771\\
0.84	-0.242050194961631\\
0.840999999999993	-0.231926142890954\\
0.841000000000001	-0.231926142890882\\
0.841999999999994	-0.221782869738117\\
0.842999999999987	-0.21162004590407\\
0.844999999999973	-0.191234424798309\\
0.848999999999945	-0.150214002305099\\
0.849999999999993	-0.139905041492604\\
0.85	-0.13990504149253\\
0.857999999999944	-0.0566054753643399\\
0.859999999999993	-0.0355382059464438\\
0.86	-0.0355382059463688\\
0.867999999999944	0.0497825287741899\\
0.869999999999993	0.0713894871348013\\
0.87	0.0713894871348783\\
0.874999999999993	0.125921892640849\\
0.875000000000001	0.125921892640927\\
0.879999999999994	0.1812255356272\\
0.880000000000001	0.181225535627279\\
0.884999999999994	0.237345339110402\\
0.889999999999987	0.294326888924984\\
0.890000000000001	0.294326888925146\\
0.898999999999994	0.399213493665838\\
0.899000000000001	0.399213493665922\\
0.899999999999993	0.411061108582496\\
0.9	0.41106110858258\\
0.900999999999993	0.42294883500248\\
0.901999999999986	0.434877059235569\\
0.903999999999972	0.458856552646799\\
0.907999999999944	0.507318654884404\\
0.909999999999993	0.531807561988279\\
0.91	0.531807561988366\\
0.917999999999944	0.631556759685358\\
0.918999999999994	0.644233864055871\\
0.919000000000001	0.644233864055961\\
0.919999999999993	0.656958655720917\\
0.92	0.656958655721007\\
0.920999999999993	0.669615448224512\\
0.921999999999986	0.682088553018725\\
0.923999999999972	0.706485314513614\\
0.927999999999944	0.75309201762227\\
0.927999999999987	0.753092017622749\\
0.928000000000001	0.753092017622908\\
0.929999999999993	0.775308016392032\\
0.93	0.775308016392109\\
0.931999999999993	0.796802993980331\\
0.933999999999986	0.817579744032382\\
0.937999999999972	0.856989269107697\\
0.939999999999993	0.875627165919682\\
0.940000000000001	0.875627165919747\\
0.945000000000001	0.919129551823751\\
0.945000000000008	0.91912955182381\\
0.949999999999993	0.95824212548519\\
0.950000000000001	0.958242125485242\\
0.954999999999986	0.992996658010532\\
0.956999999999994	1.00568466164755\\
0.957000000000001	1.00568466164759\\
0.96	1.02342138035473\\
0.960000000000008	1.02342138035477\\
0.963000000000007	1.0396083314994\\
0.966000000000007	1.05425024837574\\
0.97	1.07137675278674\\
0.970000000000008	1.07137675278677\\
0.976000000000007	1.09194965725398\\
0.976999999999994	1.09478299152472\\
0.977000000000001	1.09478299152474\\
0.979999999999993	1.10226409006557\\
0.980000000000001	1.10226409006559\\
0.982999999999993	1.10785632961364\\
0.985999999999986	1.11119919933584\\
0.985999999999993	1.11119919933585\\
0.986000000000001	1.11119919933585\\
0.989999999999993	1.1121589340505\\
0.990000000000001	1.1121589340505\\
0.993999999999993	1.10912195001271\\
0.997999999999986	1.10208666843699\\
0.999999999999993	1.0970686043837\\
1	1.09706860438368\\
1.00799999999999	1.06697679091394\\
1.00999999999999	1.05694405985335\\
1.01	1.05694405985327\\
1.01499999999999	1.02745639532305\\
1.015	1.02745639532296\\
1.01999999999999	0.991654902166451\\
1.02	0.99165490216634\\
1.02499999999999	0.949510498983256\\
1.02999999999997	0.9009889520865\\
1.03	0.900988952086209\\
1.03499999999999	0.846050847231949\\
1.035	0.846050847231784\\
1.03999999999999	0.784651559273489\\
1.04	0.784651559273305\\
1.04399999999999	0.731856116727561\\
1.044	0.73185611672737\\
1.04799999999999	0.676885008932872\\
1.04999999999999	0.648574694807303\\
1.05	0.6485746948071\\
1.05399999999999	0.590286149636473\\
1.05799999999997	0.529748343197551\\
1.05999999999999	0.498626178264817\\
1.06	0.498626178264594\\
1.06799999999997	0.36835385875602\\
1.06999999999999	0.334318701123809\\
1.07	0.334318701123565\\
1.07299999999999	0.28214788154676\\
1.073	0.282147881546509\\
1.07599999999999	0.228621772967359\\
1.07899999999997	0.173724723347677\\
1.07999999999999	0.155118290318627\\
1.08	0.155118290318362\\
1.08499999999999	0.0597531848015738\\
1.085	0.0597531848012972\\
1.08999999999999	-0.039557426333537\\
1.09	-0.0395574263338294\\
1.09299999999999	-0.101071058993752\\
1.093	-0.101071058994046\\
1.09599999999999	-0.164052039310691\\
1.09899999999997	-0.228518783801566\\
1.09999999999999	-0.250341114053899\\
1.1	-0.250341114054211\\
1.10199999999999	-0.293916186958983\\
1.102	-0.293916186959291\\
1.10399999999999	-0.33702055442283\\
1.10599999999997	-0.37965981837006\\
1.10999999999994	-0.463565141552973\\
1.10999999999999	-0.463565141553881\\
1.11	-0.463565141554176\\
1.11799999999994	-0.626034405030622\\
1.11999999999999	-0.665569808422754\\
1.12	-0.665569808423034\\
1.12799999999994	-0.819537928396479\\
1.12999999999999	-0.857011597265864\\
1.13	-0.857011597266129\\
1.13099999999999	-0.875599057219858\\
1.131	-0.875599057220122\\
1.13199999999999	-0.894087740395418\\
1.13299999999999	-0.91247824757512\\
1.13499999999997	-0.948967120784798\\
1.13899999999994	-1.02079283400656\\
1.13999999999999	-1.03851266316559\\
1.14	-1.03851266316584\\
1.14799999999994	-1.17695361005824\\
1.14999999999999	-1.21066285560826\\
1.15	-1.2106628556085\\
1.15099999999999	-1.22738536941702\\
1.151	-1.22738536941726\\
1.15199999999999	-1.24402053588407\\
1.15299999999999	-1.26056889556262\\
1.15499999999997	-1.29340734235625\\
1.155	-1.29340734235671\\
1.15899999999997	-1.3580660243097\\
1.15999999999999	-1.37402163495768\\
1.16	-1.37402163495791\\
1.16399999999997	-1.43491505462346\\
1.16799999999994	-1.49030207545394\\
1.16999999999999	-1.51593979579883\\
1.17	-1.51593979579901\\
1.17799999999994	-1.60486108433716\\
1.17999999999999	-1.62369845368145\\
1.18	-1.62369845368158\\
1.18799999999994	-1.68554957022626\\
1.18899999999999	-1.69176658406203\\
1.189	-1.69176658406212\\
1.18999999999999	-1.69764780696946\\
1.19	-1.69764780696954\\
1.19099999999999	-1.70319342999228\\
1.19199999999999	-1.7084036333448\\
1.19399999999997	-1.71781844725853\\
1.19799999999994	-1.73262975777187\\
1.19999999999999	-1.73802817931117\\
1.2	-1.7380281793112\\
1.20799999999994	-1.74625465004564\\
1.20999999999999	-1.74497080035168\\
1.21	-1.74497080035166\\
1.21799999999994	-1.72646852575547\\
1.21799999999997	-1.72646852575537\\
1.218	-1.72646852575527\\
1.21999999999999	-1.71849823252249\\
1.22	-1.71849823252243\\
1.22199999999999	-1.70918823873241\\
1.22399999999997	-1.69853733444802\\
1.22499999999999	-1.69270861512124\\
1.225	-1.69270861512116\\
1.22899999999997	-1.66603407723171\\
1.22999999999999	-1.65852444430374\\
1.23	-1.65852444430364\\
1.23399999999997	-1.62511459900633\\
1.23799999999995	-1.58629723127098\\
1.23799999999997	-1.58629723127071\\
1.238	-1.58629723127043\\
1.23999999999999	-1.56485453039292\\
1.24	-1.56485453039276\\
1.24199999999999	-1.54197529956948\\
1.24399999999997	-1.51757970335552\\
1.24699999999999	-1.47813613458486\\
1.247	-1.47813613458466\\
1.24999999999999	-1.43526203006235\\
1.25	-1.43526203006214\\
1.25299999999999	-1.38894485282299\\
1.25599999999997	-1.33917105909528\\
1.25999999999999	-1.26740373967825\\
1.26	-1.26740373967798\\
1.26599999999997	-1.1480994765668\\
1.26999999999999	-1.06073414670913\\
1.27	-1.0607341467088\\
1.27599999999997	-0.917828871722661\\
1.276	-0.917828871721983\\
1.28	-0.814581608102911\\
1.28000000000001	-0.814581608102533\\
1.28400000000002	-0.704890545851496\\
1.28800000000002	-0.588698661339529\\
1.29	-0.528146167290434\\
1.29000000000001	-0.528146167289998\\
1.29499999999999	-0.369536655342746\\
1.295	-0.369536655342281\\
1.29599999999999	-0.336567401600536\\
1.296	-0.336567401600065\\
1.29699999999999	-0.303179896291277\\
1.29799999999999	-0.26937305463783\\
1.29999999999997	-0.200496955142219\\
1.3	-0.200496955141236\\
1.30399999999997	-0.0610347988578159\\
1.30499999999999	-0.0261575792174206\\
1.305	-0.0261575792169249\\
1.30899999999997	0.113432017000308\\
1.30999999999999	0.148355264507196\\
1.31	0.148355264507693\\
1.31399999999997	0.288185688586926\\
1.31799999999994	0.428290542624484\\
1.31999999999999	0.498468635454011\\
1.32	0.49846863545451\\
1.32799999999994	0.780231738365098\\
1.32999999999999	0.850980969635345\\
1.33	0.850980969635848\\
1.33399999999999	0.992913882975373\\
1.334	0.992913882975878\\
1.33799999999999	1.13548758290661\\
1.33999999999999	1.20703787585181\\
1.34	1.20703787585232\\
1.34399999999999	1.35071184214119\\
1.34799999999997	1.49521259911122\\
1.35	1.56779648157791\\
1.35000000000001	1.56779648157843\\
1.354	1.7136784194628\\
1.35400000000001	1.71367841946332\\
1.358	1.86057583778167\\
1.35999999999999	1.9344291948861\\
1.36	1.93442919488663\\
1.36299999999999	2.04326556364987\\
1.363	2.04326556365038\\
1.36499999999999	2.11344048287062\\
1.365	2.11344048287111\\
1.36699999999999	2.18171995011054\\
1.36899999999997	2.24811283895894\\
1.36999999999999	2.28060452242438\\
1.37	2.28060452242484\\
1.37399999999997	2.40589730421103\\
1.37799999999995	2.52376006722307\\
1.37999999999999	2.57992448948302\\
1.38	2.57992448948341\\
1.38799999999994	2.7863033186293\\
1.38999999999999	2.83336183746595\\
1.39	2.83336183746628\\
1.39199999999999	2.87861846514156\\
1.392	2.87861846514188\\
1.39399999999998	2.92207908288439\\
1.39599999999997	2.96374933916621\\
1.39999999999994	3.0417401972099\\
1.39999999999999	3.04174019721079\\
1.4	3.04174019721106\\
1.40799999999994	3.17646035071985\\
1.41	3.20573676492491\\
1.41000000000001	3.20573676492511\\
1.41200000000001	3.2332596817438\\
1.41200000000003	3.23325968174399\\
1.41400000000003	3.2590326778479\\
1.41600000000003	3.28305910291541\\
1.41999999999999	3.32588450324034\\
1.42	3.32588450324048\\
1.42099999999999	3.33536704609381\\
1.421	3.33536704609394\\
1.42199999999999	3.34414181997995\\
1.42299999999999	3.35220910999324\\
1.42499999999997	3.36622226383921\\
1.42899999999994	3.38576875854953\\
1.42999999999999	3.38888971415989\\
1.43	3.38888971415993\\
1.43499999999999	3.39390526689427\\
1.435	3.39390526689426\\
1.43999999999998	3.38127299591869\\
1.44	3.38127299591861\\
1.44499999999999	3.35098264027019\\
1.44999999999997	3.30300959524515\\
1.44999999999998	3.30300959524498\\
1.45	3.30300959524481\\
1.45999999999997	3.15384516859618\\
1.45999999999999	3.15384516859586\\
1.46	3.15384516859559\\
1.46999999999997	2.93329495568954\\
1.46999999999998	2.93329495568914\\
1.47	2.93329495568874\\
1.47899999999998	2.67317734938418\\
1.479	2.67317734938372\\
1.47999999999999	2.64064220344756\\
1.48	2.64064220344709\\
1.48099999999999	2.60737671734577\\
1.48199999999999	2.57337981010344\\
1.48399999999997	2.50318729020061\\
1.48799999999994	2.35397499990694\\
1.49	2.27493583747894\\
1.49000000000001	2.27493583747836\\
1.49799999999996	1.92898155148901\\
1.49900000000001	1.88236194939949\\
1.49900000000003	1.88236194939883\\
1.49999999999999	1.83498737066207\\
1.5	1.83498737066139\\
1.50099999999999	1.78705667110597\\
1.50199999999999	1.73876868809254\\
1.50399999999997	1.64111458458662\\
1.50499999999999	1.5917452913228\\
1.505	1.5917452913221\\
1.50799999999998	1.44144916210034\\
1.508	1.44144916209962\\
1.50999999999999	1.33941189466744\\
1.51	1.33941189466671\\
1.51199999999999	1.23588730596979\\
1.51399999999998	1.13086194196824\\
1.51799999999995	0.91625409493214\\
1.51999999999999	0.806643720840894\\
1.52	0.80664372084011\\
1.52799999999995	0.352491134800439\\
1.52999999999999	0.234951438501977\\
1.53	0.234951438501136\\
1.53699999999998	-0.189382656021194\\
1.537	-0.189382656022077\\
1.53999999999999	-0.377522860544011\\
1.54	-0.377522860544911\\
1.54299999999999	-0.569511637615706\\
1.54599999999997	-0.765405127903688\\
1.54999999999999	-1.03276961911131\\
1.55	-1.03276961911228\\
1.55599999999997	-1.44732831832024\\
1.55699999999998	-1.51802649726095\\
1.557	-1.51802649726196\\
1.55999999999999	-1.73291828489246\\
1.56	-1.73291828489349\\
1.56299999999999	-1.9476455598842\\
1.56599999999998	-2.15786169027139\\
1.566	-2.15786169027298\\
1.56999999999999	-2.43123858504161\\
1.57	-2.43123858504257\\
1.57399999999999	-2.69684691377257\\
1.57499999999999	-2.76205142630351\\
1.575	-2.76205142630443\\
1.57899999999999	-3.01814275369467\\
1.57999999999999	-3.08099430528688\\
1.58	-3.08099430528777\\
1.58399999999999	-3.32777665303777\\
1.58799999999998	-3.56725650575947\\
1.58999999999999	-3.68429704848384\\
1.59	-3.68429704848467\\
1.59499999999998	-3.96914697255552\\
1.595	-3.96914697255631\\
1.59999999999998	-4.24310745111504\\
1.6	-4.24310745111601\\
1.60499999999998	-4.50640102177103\\
1.60999999999997	-4.7592415565679\\
1.60999999999998	-4.75924155656865\\
1.61	-4.75924155656941\\
1.61499999999998	-5.00183443954343\\
1.615	-5.0018344395441\\
1.61999999999998	-5.23437672168128\\
1.62	-5.23437672168211\\
1.624	-5.40683605511915\\
1.62400000000001	-5.40683605511972\\
1.62800000000001	-5.56014316864498\\
1.62999999999999	-5.62963993624\\
1.63	-5.62963993624047\\
1.634	-5.75436505310861\\
1.638	-5.8601186151684\\
1.63999999999999	-5.90589862817675\\
1.64	-5.90589862817706\\
1.64499999999998	-5.99969685851655\\
1.645	-5.99969685851678\\
1.64999999999998	-6.06405059342615\\
1.65	-6.0640505934263\\
1.65299999999998	-6.0885522164783\\
1.653	-6.08855221647839\\
1.65599999999998	-6.10248026301539\\
1.65899999999997	-6.10583880581431\\
1.65999999999999	-6.10460980024372\\
1.66	-6.1046098002437\\
1.66599999999997	-6.07257151007474\\
1.67	-6.02770805996257\\
1.67000000000002	-6.02770805996238\\
1.673	-5.98170528532847\\
1.67300000000001	-5.98170528532823\\
1.676	-5.92509769073349\\
1.67899999999998	-5.8578687233374\\
1.67999999999998	-5.83309545421878\\
1.68	-5.83309545421841\\
1.68199999999998	-5.78010865854205\\
1.682	-5.78010865854166\\
1.68399999999998	-5.72260234231344\\
1.68599999999997	-5.66056903195434\\
1.68999999999994	-5.52288859211892\\
1.68999999999998	-5.52288859211727\\
1.69	-5.52288859211675\\
1.69799999999994	-5.19280950872315\\
1.69999999999998	-5.09882841951038\\
1.7	-5.09882841950969\\
1.70799999999994	-4.67669233709057\\
1.70999999999998	-4.5595368090154\\
1.71	-4.55953680901455\\
1.711	-4.49920571563478\\
1.71100000000001	-4.49920571563391\\
1.71200000000001	-4.43770313538726\\
1.71300000000001	-4.37502706982605\\
1.715	-4.24614629926524\\
1.71500000000001	-4.2461462992643\\
1.71700000000001	-4.11254665462857\\
1.71900000000001	-3.97421077333477\\
1.71999999999999	-3.90326114986546\\
1.72	-3.90326114986444\\
1.724	-3.60753053228096\\
1.72799999999999	-3.29259181682321\\
1.72999999999999	-3.12786865042462\\
1.73	-3.12786865042343\\
1.731	-3.04368435297521\\
1.73100000000001	-3.043684352974\\
1.73200000000001	-2.95828127933565\\
1.73300000000001	-2.87165665442826\\
1.735	-2.69473145339897\\
1.73899999999999	-2.32609593176145\\
1.73999999999998	-2.23083940995488\\
1.74	-2.23083940995352\\
1.74799999999998	-1.45914512238955\\
1.74999999999998	-1.26465902554859\\
1.75	-1.26465902554721\\
1.75799999999998	-0.479706198140071\\
1.75999999999998	-0.281588365087759\\
1.76	-0.281588365086349\\
1.76799999999998	0.519173973063205\\
1.76899999999998	0.620255634249813\\
1.769	0.620255634251252\\
1.76999999999998	0.721567393221838\\
1.77	0.721567393223279\\
1.77099999999998	0.823112541092185\\
1.77199999999997	0.924894377502704\\
1.77399999999994	1.12918135118638\\
1.77799999999989	1.54074159136015\\
1.77999999999998	1.74806834544283\\
1.78	1.74806834544431\\
1.78499999999998	2.27111250033719\\
1.785	2.27111250033869\\
1.78999999999998	2.80125045498119\\
1.79	2.80125045498271\\
1.79499999999998	3.33891283916724\\
1.798	3.66530341518981\\
1.79800000000001	3.66530341519136\\
1.79999999999999	3.88453639369026\\
1.8	3.88453639369182\\
1.80199999999998	4.1051139959147\\
1.80399999999995	4.32706488827103\\
1.8079999999999	4.77520210438658\\
1.80999999999999	5.00144666938527\\
1.81	5.00144666938689\\
1.8179999999999	5.92161965161068\\
1.818	5.92161965162212\\
1.81800000000001	5.92161965162378\\
1.81999999999998	6.15561106051222\\
1.82	6.15561106051389\\
1.82199999999997	6.38692057670764\\
1.82399999999994	6.61125659898921\\
1.82699999999998	6.93474886032036\\
1.827	6.93474886032186\\
1.82999999999998	7.24271074033501\\
1.83	7.24271074033644\\
1.83299999999999	7.53523229123127\\
1.83599999999997	7.81239905029353\\
1.83999999999998	8.1582089356318\\
1.84	8.15820893563298\\
1.84599999999997	8.62639972969393\\
1.84999999999998	8.90508086848216\\
1.85	8.90508086848311\\
1.85499999999998	9.21606166969674\\
1.855	9.21606166969757\\
1.85599999999998	9.27329569185691\\
1.856	9.27329569185771\\
1.85699999999998	9.32888013206604\\
1.85799999999997	9.38281679630457\\
1.85999999999994	9.48575375293503\\
1.85999999999999	9.48575375293741\\
1.86	9.48575375293812\\
1.86399999999994	9.67192787947981\\
1.86799999999988	9.83191485524718\\
1.86999999999999	9.90211468342536\\
1.87	9.90211468342584\\
1.876	10.07364646578\\
1.87600000000001	10.0736464657803\\
1.87999999999999	10.1555167653624\\
1.88	10.1555167653626\\
1.88399999999998	10.2058407986276\\
1.88499999999998	10.2126195503095\\
1.885	10.2126195503095\\
1.88899999999997	10.2165321378206\\
1.89	10.2117098388018\\
1.89000000000001	10.2117098388017\\
1.89399999999999	10.169214158734\\
1.89799999999996	10.0895688210674\\
1.89999999999999	10.0358028466291\\
1.9	10.0358028466286\\
1.90799999999995	9.72760642251953\\
1.90999999999999	9.62722411870315\\
1.91	9.6272241187024\\
1.914	9.39837260066115\\
1.91400000000001	9.39837260066027\\
1.91800000000001	9.13197069877738\\
1.91999999999999	8.98464583979041\\
1.92	8.98464583978933\\
1.924	8.66165257180845\\
1.92499999999998	8.57498415686723\\
1.925	8.57498415686598\\
1.92899999999999	8.20454562818425\\
1.92999999999999	8.10597973448064\\
1.93	8.10597973447922\\
1.934	7.68779503973955\\
1.93400000000001	7.68779503973797\\
1.93800000000001	7.23117070648537\\
1.93999999999999	6.98837027832777\\
1.94	6.98837027832601\\
1.94299999999998	6.61150941044196\\
1.943	6.61150941044015\\
1.94599999999998	6.22378874890423\\
1.94899999999996	5.82509491892356\\
1.94999999999999	5.68973843634034\\
1.95	5.68973843633841\\
1.95599999999997	4.85147492382811\\
1.95999999999998	4.2674168800199\\
1.96	4.26741688001779\\
1.96599999999996	3.35274716232041\\
1.97	2.71678329234022\\
1.97000000000001	2.71678329233792\\
1.97199999999998	2.39082319005596\\
1.972	2.39082319005362\\
1.97399999999997	2.059488059174\\
1.97599999999994	1.72273483884592\\
1.97999999999988	1.03279836248664\\
1.98	1.03279836246583\\
1.98000000000002	1.03279836246334\\
1.9879999999999	-0.41406518094148\\
1.99	-0.79001059225319\\
1.99000000000002	-0.790010592255881\\
1.99199999999998	-1.17174441335813\\
1.992	-1.17174441336086\\
1.99399999999997	-1.55931625200781\\
1.995	-1.75530713919092\\
1.99500000000001	-1.75530713919371\\
1.99699999999998	-2.15173068784969\\
1.99899999999995	-2.55411961513155\\
1.99999999999999	-2.75756740737394\\
2	-2.75756740737684\\
2.00099999999997	-2.96117970113579\\
2.001	-2.96117970114156\\
2.00199999999997	-3.16361659718228\\
2.00299999999995	-3.36488467269794\\
2.00499999999989	-3.76394048093976\\
2.00899999999979	-4.54831145487792\\
2.00999999999997	-4.74157878357903\\
2.01	-4.7415787835845\\
2.01799999999979	-6.24800537459378\\
2.01999999999997	-6.61380490042227\\
2.02	-6.61380490042744\\
2.02799999999979	-8.03520083999894\\
2.02999999999997	-8.38033019122769\\
2.03	-8.38033019123257\\
2.03799999999979	-9.72131478838934\\
2.03999999999997	-10.0468955801035\\
2.04	-10.0468955801081\\
2.04799999999979	-11.3118268198963\\
2.05	-11.6189171374979\\
2.05000000000003	-11.6189171375023\\
2.05799999999981	-12.8119058437441\\
2.05899999999997	-12.957130078575\\
2.059	-12.9571300785791\\
2.05999999999997	-13.1015036810476\\
2.06	-13.1015036810517\\
2.06099999999998	-13.2450313411327\\
2.06199999999995	-13.3877177227079\\
2.0639999999999	-13.6705851665027\\
2.06499999999997	-13.8107754190668\\
2.065	-13.8107754190707\\
2.0689999999999	-14.3633525996959\\
2.06999999999997	-14.4994733788352\\
2.07	-14.4994733788391\\
2.0739999999999	-15.0359951103089\\
2.0779999999998	-15.5599868824959\\
2.07899999999997	-15.6890590830649\\
2.079	-15.6890590830686\\
2.07999999999997	-15.8173694107986\\
2.08	-15.8173694108023\\
2.08099999999998	-15.943252687561\\
2.08199999999995	-16.0650436575973\\
2.08399999999991	-16.2963643724755\\
2.08799999999981	-16.7100633346643\\
2.088	-16.7100633346822\\
2.08800000000003	-16.710063334685\\
2.08999999999997	-16.8924953477549\\
2.09	-16.8924953477574\\
2.09199999999995	-17.0586813627498\\
2.0939999999999	-17.2086429787905\\
2.09799999999979	-17.4599688639555\\
2.09999999999997	-17.5613657963628\\
2.1	-17.5613657963642\\
2.10799999999979	-17.8054631025855\\
2.10999999999997	-17.8261544780902\\
2.11	-17.8261544780904\\
2.11699999999997	-17.7715913439637\\
2.117	-17.771591343963\\
2.11999999999997	-17.6877219142401\\
2.12	-17.6877219142392\\
2.12299999999998	-17.5675310325111\\
2.12599999999995	-17.4109835534851\\
2.13	-17.1456182215248\\
2.13000000000003	-17.1456182215227\\
2.13499999999997	-16.7226905271546\\
2.135	-16.7226905271519\\
2.137	-16.5250707282863\\
2.13700000000002	-16.5250707282834\\
2.13900000000002	-16.3111571826497\\
2.14	-16.1980816500124\\
2.14000000000003	-16.1980816500092\\
2.14200000000003	-15.9609799872587\\
2.14400000000002	-15.7101218509572\\
2.14599999999997	-15.4454746395717\\
2.146	-15.4454746395678\\
2.14999999999999	-14.8746736213251\\
2.15000000000003	-14.8746736213196\\
2.15400000000002	-14.2482801980718\\
2.15800000000002	-13.5659687353654\\
2.15999999999997	-13.2037340327697\\
2.16	-13.2037340327644\\
2.16799999999999	-11.6131488693521\\
2.16999999999997	-11.1798325992498\\
2.17	-11.1798325992435\\
2.17499999999997	-10.0335111320972\\
2.175	-10.0335111320905\\
2.17999999999997	-8.79639196592697\\
2.18	-8.79639196591819\\
2.18499999999997	-7.46747019265003\\
2.18999999999994	-6.04566633579349\\
2.18999999999997	-6.04566633578462\\
2.19	-6.04566633577575\\
2.19499999999997	-4.52982545700674\\
2.195	-4.52982545699785\\
2.19999999999996	-2.91871627375565\\
2.2	-2.91871627374431\\
2.20399999999997	-1.5925336319887\\
2.204	-1.59253363197929\\
2.20499999999997	-1.26130428660711\\
2.205	-1.2613042865977\\
2.20599999999997	-0.930179938489748\\
2.20699999999995	-0.59914982938792\\
2.20899999999989	0.0626706893448039\\
2.20999999999997	0.393482601482822\\
2.21	0.393482601492224\\
2.21399999999989	1.71632537003608\\
2.21799999999979	3.0390360552247\\
2.21999999999997	3.70055674474857\\
2.22	3.70055674475797\\
2.22799999999979	6.34974820779595\\
2.22999999999997	7.01325361325161\\
2.23	7.01325361326105\\
2.23299999999997	8.0097675055256\\
2.233	8.00976750553505\\
2.23599999999997	9.00804738403039\\
2.23899999999994	10.0083851651852\\
2.24	10.3423389432533\\
2.24000000000003	10.3423389432628\\
2.24599999999997	12.3521540321955\\
2.25	13.6986317206947\\
2.25000000000003	13.6986317207043\\
2.25299999999997	14.7124527431287\\
2.253	14.7124527431383\\
2.25599999999994	15.7299997120731\\
2.25899999999988	16.7515701734437\\
2.26	17.093039365868\\
2.26000000000003	17.0930393658777\\
2.26199999999997	17.7673154822196\\
2.262	17.7673154822291\\
2.26399999999994	18.4233498246458\\
2.26599999999988	19.0612276527067\\
2.26999999999977	20.282843010831\\
2.27	20.2828430108993\\
2.27000000000003	20.2828430109077\\
2.27499999999997	21.7091587498306\\
2.275	21.7091587498384\\
2.27999999999994	23.0246588630537\\
2.28	23.024658863068\\
2.28499999999995	24.2304119296437\\
2.28999999999989	25.3273973767433\\
2.29	25.327397376766\\
2.29000000000003	25.3273973767719\\
2.29099999999997	25.5338223154706\\
2.291	25.5338223154764\\
2.29199999999997	25.7359391770859\\
2.29299999999994	25.9337545307211\\
2.29499999999989	26.316506282359\\
2.29899999999979	27.0306651378808\\
2.29999999999997	27.1985420909803\\
2.3	27.198542090985\\
2.30799999999979	28.3888433000323\\
2.31	28.6441739213928\\
2.31000000000003	28.6441739213963\\
2.31099999999999	28.7655266614992\\
2.31100000000002	28.7655266615026\\
2.31199999999999	28.8826763232704\\
2.31299999999996	28.9956267135719\\
2.31499999999989	29.208944222354\\
2.31899999999976	29.5853403234081\\
2.31999999999997	29.6689909404366\\
2.32	29.6689909404389\\
2.32799999999974	30.1118426407058\\
2.32999999999997	30.1570948447062\\
2.33	30.1570948447067\\
2.33799999999974	30.0764372174087\\
2.33999999999997	29.9908430984432\\
2.34	29.9908430984418\\
2.345	29.6622480848945\\
2.34500000000003	29.6622480848922\\
2.34899999999997	29.281339843105\\
2.349	29.2813398431019\\
2.34999999999997	29.1696954109062\\
2.35	29.169695410903\\
2.35099999999997	29.0514770955983\\
2.35199999999994	28.9266810562763\\
2.35399999999988	28.6573393734024\\
2.35799999999976	28.0395305824891\\
2.35999999999997	27.690983182481\\
2.36	27.6909831824758\\
2.36799999999976	26.0314097494131\\
2.36999999999997	25.5499008353274\\
2.37	25.5499008353204\\
2.37799999999976	23.3555254503956\\
2.378	23.3555254503258\\
2.37800000000002	23.3555254503172\\
2.37999999999997	22.739490195423\\
2.38	22.7394901954141\\
2.38199999999995	22.0963235292686\\
2.38399999999989	21.425941862927\\
2.38799999999979	20.00318149078\\
2.38999999999997	19.2506178790736\\
2.39	19.2506178790627\\
2.39799999999979	15.9634832889213\\
2.398	15.9634832888299\\
2.39800000000002	15.9634832888174\\
2.39999999999997	15.0719456093364\\
2.4	15.0719456093235\\
2.40199999999994	14.1601901011121\\
2.40399999999988	13.2360083211724\\
2.40699999999997	11.8261732132282\\
2.407	11.8261732132147\\
2.40999999999997	10.3876961173312\\
2.41	10.3876961173175\\
2.41299999999997	8.92015640339394\\
2.41499999999997	7.92543913329075\\
2.415	7.92543913327652\\
2.41799999999997	6.40850385021709\\
2.41999999999997	5.38044907820886\\
2.42	5.38044907819416\\
2.42299999999997	3.81288602708451\\
2.42599999999994	2.21433779443155\\
2.42999999999997	0.0339317400591248\\
2.42999999999999	0.0339317400434295\\
2.43599999999994	-3.34400900739964\\
2.43599999999997	-3.34400900741723\\
2.436	-3.34400900743456\\
2.43999999999997	-5.66923122050307\\
2.44	-5.66923122051981\\
2.44399999999998	-8.05446857992469\\
2.44799999999996	-10.5009610674146\\
2.44999999999997	-11.7475741749194\\
2.45	-11.7475741749373\\
2.45599999999999	-15.5828312405053\\
2.45600000000002	-15.5828312405238\\
2.45999999999997	-18.2208507570139\\
2.46	-18.2208507570329\\
2.46399999999995	-20.8523428251665\\
2.46499999999997	-21.4979403227666\\
2.465	-21.4979403227849\\
2.46899999999995	-24.0318575094261\\
2.46999999999997	-24.6533215130977\\
2.47	-24.6533215131153\\
2.47399999999995	-27.091722106473\\
2.4779999999999	-29.45513919926\\
2.47999999999997	-30.6091143948106\\
2.48	-30.6091143948269\\
2.48499999999997	-33.41436394752\\
2.485	-33.4143639475356\\
2.48999999999997	-36.1075847671242\\
2.49	-36.1075847671401\\
2.49399999999997	-38.1829738809439\\
2.494	-38.1829738809584\\
2.49799999999996	-40.1891453490189\\
2.49999999999997	-41.1666017696054\\
2.5	-41.1666017696192\\
2.50399999999997	-43.0708911494343\\
2.50799999999994	-44.9085038974864\\
2.50999999999997	-45.8026064104003\\
2.51	-45.8026064104129\\
2.51399999999997	-47.5419847001589\\
2.514	-47.541984700171\\
2.51799999999996	-49.2170106746517\\
2.51999999999997	-50.030664957943\\
2.52	-50.0306649579544\\
2.523	-51.1887363040527\\
2.52300000000002	-51.1887363040632\\
2.52600000000002	-52.2459068689005\\
2.52900000000001	-53.2024857835634\\
2.52999999999997	-53.4990409141401\\
2.53	-53.4990409141484\\
2.53599999999999	-55.0448298345565\\
2.53999999999997	-55.853528952522\\
2.54	-55.8535289525271\\
2.54599999999999	-56.7348884488664\\
2.54999999999997	-57.1017807815205\\
2.55	-57.1017807815225\\
2.55199999999997	-57.219102870771\\
2.552	-57.2191028707723\\
2.55399999999996	-57.2923644801764\\
2.55499999999997	-57.3124758041231\\
2.555	-57.3124758041235\\
2.55699999999997	-57.3196627575742\\
2.55899999999994	-57.2828023004996\\
2.55999999999997	-57.2478530196428\\
2.56	-57.2478530196416\\
2.56399999999994	-56.9978982106482\\
2.56799999999987	-56.5715807638921\\
2.56999999999997	-56.292220377338\\
2.57	-56.2922203773338\\
2.57199999999997	-55.9686790534992\\
2.572	-55.9686790534942\\
2.57399999999996	-55.6009147463357\\
2.57599999999993	-55.1888796604915\\
2.57999999999986	-54.2317772003503\\
2.57999999999997	-54.2317772003213\\
2.58	-54.2317772003139\\
2.58099999999997	-53.9651643994943\\
2.581	-53.9651643994866\\
2.58199999999996	-53.6882762457239\\
2.58299999999993	-53.4011037420785\\
2.58499999999987	-52.7958680301212\\
2.58899999999974	-51.4615514559572\\
2.58999999999997	-51.1021085064193\\
2.59	-51.1021085064089\\
2.59799999999975	-47.8522516587625\\
2.59999999999997	-46.9353244994045\\
2.6	-46.9353244993911\\
2.60799999999975	-42.8461887811262\\
2.60999999999997	-41.7178838001677\\
2.61	-41.7178838001514\\
2.61799999999974	-36.776180175475\\
2.61999999999997	-35.4328305625377\\
2.62	-35.4328305625183\\
2.62499999999997	-31.8837035627746\\
2.625	-31.8837035627536\\
2.62999999999998	-28.0597393713515\\
2.63000000000001	-28.0597393713289\\
2.63499999999998	-23.9578317938697\\
2.63899999999997	-20.4739550712781\\
2.639	-20.4739550712527\\
2.63999999999997	-19.5746488702665\\
2.64	-19.5746488702408\\
2.64099999999998	-18.6639499790327\\
2.64199999999996	-17.7418288085103\\
2.64399999999991	-15.8631994197262\\
2.64799999999982	-11.9675300906301\\
2.64999999999997	-9.94998385795414\\
2.65	-9.94998385792514\\
2.65799999999982	-1.40886595194012\\
2.65899999999997	-0.287675498115972\\
2.659	-0.287675498083936\\
2.66	0.845534274435645\\
2.66000000000003	0.845534274468024\\
2.66100000000003	1.98358221412241\\
2.66200000000004	3.11928733646549\\
2.66400000000005	5.38381664911932\\
2.66799999999997	9.88638005698527\\
2.668	9.88638005701715\\
2.67	12.1249993052483\\
2.67000000000003	12.12499930528\\
2.67200000000003	14.3555651841224\\
2.67400000000004	16.5783675781267\\
2.67800000000005	21.0018364401041\\
2.67999999999997	23.2030777952559\\
2.68	23.2030777952872\\
2.68800000000002	31.9447545801058\\
2.68999999999997	34.1157717101614\\
2.69	34.1157717101922\\
2.69499999999997	39.5211999518965\\
2.695	39.5211999519272\\
2.69699999999997	41.6752630304404\\
2.697	41.675263030471\\
2.69899999999996	43.8251130849265\\
2.69999999999997	44.898545559942\\
2.7	44.8985455599724\\
2.70199999999997	47.042599783943\\
2.70399999999993	49.1831385237979\\
2.70799999999987	53.4547818664504\\
2.71	55.5864416249413\\
2.71000000000003	55.5864416249716\\
2.71699999999999	63.0301256371064\\
2.71700000000002	63.0301256371366\\
2.72	66.2141938567808\\
2.72000000000003	66.2141938568109\\
2.72300000000001	69.3225040746599\\
2.72599999999999	72.2825111340645\\
2.72600000000002	72.2825111340918\\
2.73	75.9999832055231\\
2.73000000000003	75.9999832055486\\
2.73400000000002	79.4572715459484\\
2.738	82.6561734509978\\
2.74	84.1592544018743\\
2.74000000000003	84.1592544018952\\
2.748	89.5335725615926\\
2.75	90.7185237568342\\
2.75000000000003	90.7185237568506\\
2.75499999999997	93.4050202865792\\
2.755	93.4050202865934\\
2.75999999999993	95.6991078487298\\
2.75999999999997	95.6991078487443\\
2.76	95.6991078487589\\
2.76499999999994	97.6026499299943\\
2.76499999999997	97.602649930005\\
2.765	97.6026499300156\\
2.76999999999994	99.117192768026\\
2.76999999999997	99.1171927680343\\
2.77	99.1171927680426\\
2.77499999999994	100.243966618371\\
2.775	100.243966618382\\
2.77999999999993	100.98388675377\\
2.77999999999997	100.983886753774\\
2.78	100.983886753777\\
2.78399999999997	101.249975082412\\
2.784	101.249975082413\\
2.78799999999996	101.173550957512\\
2.78999999999997	101.006891264865\\
2.79	101.006891264862\\
2.79399999999997	100.416568320919\\
2.79799999999993	99.4832996719743\\
2.79999999999997	98.8879160624922\\
2.8	98.8879160624831\\
2.80799999999993	95.6461172409306\\
2.80999999999997	94.6200748119329\\
2.81	94.6200748119177\\
2.81299999999997	92.918803389726\\
2.813	92.918803389709\\
2.81599999999996	91.0224362222973\\
2.81899999999993	88.9304187805051\\
2.81999999999997	88.1894977128804\\
2.82	88.1894977128591\\
2.82599999999993	83.2846388029114\\
2.83	79.5752864316096\\
2.83000000000003	79.575286431582\\
2.83299999999999	76.5612162727972\\
2.83300000000002	76.5612162727677\\
2.835	74.4408491255859\\
2.83500000000003	74.4408491255552\\
2.83700000000001	72.2313725382208\\
2.83899999999999	69.9324993675019\\
2.84	68.7494461549354\\
2.84000000000003	68.7494461549014\\
2.84199999999997	66.3397406676356\\
2.842	66.339740667601\\
2.84399999999993	63.8875256771014\\
2.84599999999987	61.3924824933424\\
2.84999999999974	56.2726089166237\\
2.85	56.2726089162861\\
2.85000000000003	56.2726089162491\\
2.85799999999978	45.5042780453888\\
2.85999999999997	42.7000412475763\\
2.86	42.7000412475362\\
2.86799999999975	31.023543449082\\
2.86999999999997	27.9876344930183\\
2.87	27.9876344929748\\
2.87099999999997	26.4518947657135\\
2.871	26.4518947656697\\
2.87199999999996	24.904231811987\\
2.87299999999993	23.3445953419026\\
2.87499999999986	20.1891987734571\\
2.87899999999973	13.7324675996071\\
2.87999999999997	12.0875756940155\\
2.88	12.0875756939685\\
2.88799999999974	-1.52217091716199\\
2.88999999999997	-5.05180780338765\\
2.89	-5.05180780343818\\
2.89099999999997	-6.83601763821748\\
2.89099999999999	-6.83601763826837\\
2.89199999999996	-8.63323221221366\\
2.89299999999993	-10.4435099243019\\
2.89499999999986	-14.1034904506677\\
2.89899999999973	-21.5828257090451\\
2.89999999999997	-23.4862162372067\\
2.9	-23.486216237261\\
2.90499999999997	-32.8929489120788\\
2.905	-32.8929489121314\\
2.90999999999998	-42.0166716590273\\
2.91000000000001	-42.0166716590784\\
2.91499999999999	-50.8647956608635\\
2.91999999999997	-59.4445082042422\\
2.92	-59.4445082043\\
2.92899999999997	-74.2337040653034\\
2.92899999999999	-74.2337040653488\\
2.92999999999997	-75.8263635942877\\
2.93	-75.8263635943328\\
2.93099999999998	-77.4090983443669\\
2.93199999999996	-78.9819597497951\\
2.93399999999992	-82.0982666148179\\
2.93799999999983	-88.2146273986986\\
2.93999999999997	-91.2154762187781\\
2.94	-91.2154762188204\\
2.94799999999983	-102.845703872686\\
2.94999999999997	-105.661858222356\\
2.95	-105.661858222396\\
2.95799999999983	-116.571846102684\\
2.95799999999997	-116.571846102864\\
2.95799999999999	-116.571846102902\\
2.95999999999997	-119.212458032287\\
2.96	-119.212458032325\\
2.96199999999998	-121.819009001355\\
2.96399999999996	-124.391837774616\\
2.96799999999992	-129.437661852077\\
2.96999999999997	-131.911312927109\\
2.97	-131.911312927144\\
2.97499999999997	-137.954337079121\\
2.975	-137.954337079154\\
2.97799999999997	-141.484962978704\\
2.97799999999999	-141.484962978737\\
2.97999999999997	-143.799692153303\\
2.98	-143.799692153336\\
2.98199999999998	-146.022391487632\\
2.98399999999996	-148.092185593598\\
2.98699999999999	-150.910762440507\\
2.98700000000002	-150.910762440532\\
2.99	-153.38672729201\\
2.99000000000003	-153.386727292031\\
2.99300000000001	-155.520804153528\\
2.99599999999999	-157.313617058257\\
2.99999999999997	-159.174070967881\\
3	-159.174070967892\\
3.00599999999996	-160.830960953793\\
3.00999999999997	-161.180531116347\\
3.01	-161.180531116348\\
3.01599999999996	-160.572805949351\\
3.01599999999999	-160.572805949344\\
3.01600000000002	-160.572805949338\\
3.01999999999997	-159.412628411829\\
3.02	-159.412628411819\\
3.02399999999995	-157.647771085583\\
3.0279999999999	-155.27731649922\\
3.03	-153.864617449759\\
3.03000000000003	-153.864617449738\\
3.03600000000002	-148.714370888228\\
3.03600000000005	-148.7143708882\\
3.03999999999997	-144.518468083522\\
3.04	-144.51846808349\\
3.04399999999993	-139.765882048296\\
3.04499999999997	-138.499179270809\\
3.04499999999999	-138.499179270772\\
3.04899999999992	-133.116908410769\\
3.04999999999997	-131.69225717246\\
3.05	-131.692257172419\\
3.05399999999993	-125.675929496007\\
3.05799999999985	-119.148862554896\\
3.05999999999997	-115.692751904686\\
3.06	-115.692751904636\\
3.06799999999985	-100.573785228895\\
3.06999999999997	-96.4679564599445\\
3.07	-96.4679564598852\\
3.07399999999997	-87.8613186634316\\
3.07399999999999	-87.8613186633686\\
3.07799999999996	-78.7242834761204\\
3.08	-73.9553933278947\\
3.08000000000003	-73.9553933278259\\
3.08399999999999	-64.0137690268313\\
3.08799999999996	-53.5293500096781\\
3.09	-48.0819003031953\\
3.09000000000003	-48.0819003031169\\
3.09399999999997	-36.7729807345451\\
3.09399999999999	-36.7729807344628\\
3.09799999999993	-24.907105201809\\
3.09999999999997	-18.7633926565804\\
3.1	-18.7633926564921\\
3.10299999999997	-9.45308542492643\\
3.10299999999999	-9.45308542483832\\
3.10599999999996	-0.163446169990998\\
3.10899999999992	9.1082415250428\\
3.10999999999997	12.1952835928967\\
3.11	12.1952835929844\\
3.11499999999997	27.6086024552131\\
3.115	27.6086024553006\\
3.11999999999998	42.9946128063724\\
3.12	42.9946128064598\\
3.12499999999998	58.36581262806\\
3.12999999999995	73.7346878686298\\
3.13	73.7346878687895\\
3.13000000000003	73.7346878688769\\
3.13199999999997	79.8843834334704\\
3.13199999999999	79.8843834335578\\
3.13399999999993	86.0365035456544\\
3.13599999999986	92.1918478235945\\
3.13999999999973	104.515409189408\\
3.13999999999997	104.515409190169\\
3.14	104.515409190257\\
3.14799999999974	129.236460462985\\
3.14999999999997	135.436809041335\\
3.15	135.436809041423\\
3.15199999999997	141.646801839035\\
3.15199999999999	141.646801839124\\
3.15399999999996	147.86724586715\\
3.15599999999992	154.098949572386\\
3.15999999999985	166.5993770671\\
3.15999999999997	166.599377067498\\
3.16	166.599377067587\\
3.16099999999997	169.709337328346\\
3.16099999999999	169.709337328434\\
3.16199999999996	172.77591956039\\
3.16299999999993	175.799223412004\\
3.16499999999986	181.716387476084\\
3.16899999999973	193.035603172755\\
3.16999999999997	195.758630365084\\
3.17	195.75863036516\\
3.17799999999974	216.018849349845\\
3.17999999999997	220.663575598543\\
3.18	220.663575598608\\
3.18499999999997	231.546508123973\\
3.185	231.546508124032\\
3.18999999999998	241.395149848848\\
3.19	241.395149848901\\
3.19499999999998	250.217500814662\\
3.19999999999995	258.020727372771\\
3.2	258.020727372849\\
3.20999999999995	270.594338660612\\
3.21	270.594338660671\\
3.21899999999997	278.480200155015\\
3.21899999999999	278.480200155035\\
3.21999999999997	279.156845961272\\
3.22	279.156845961291\\
3.22099999999998	279.793669663426\\
3.22199999999996	280.390691958523\\
3.22399999999992	281.465408623374\\
3.22799999999984	283.137991973501\\
3.22999999999997	283.736076034815\\
3.23	283.736076034822\\
3.23799999999984	284.541934526299\\
3.23899999999997	284.46424740982\\
3.23899999999999	284.464247409817\\
3.23999999999997	284.346910631547\\
3.24	284.346910631543\\
3.24099999999998	284.175252125968\\
3.24199999999996	283.934598063788\\
3.24399999999992	283.246269752615\\
3.24799999999984	281.041172197297\\
3.24799999999997	281.041172197212\\
3.24799999999999	281.041172197192\\
3.24999999999997	279.524116375599\\
3.25	279.524116375575\\
3.25199999999998	277.730471626894\\
3.25399999999996	275.660004847712\\
3.25499999999998	274.520880515822\\
3.255	274.520880515789\\
3.25899999999996	269.270887065083\\
3.25999999999997	267.78480145817\\
3.26	267.784801458127\\
3.26399999999996	261.144661541337\\
3.26799999999992	253.388575291941\\
3.26999999999997	249.090814972776\\
3.27	249.090814972713\\
3.27699999999999	231.835375049183\\
3.27700000000002	231.835375049106\\
3.28	223.381404462921\\
3.28000000000003	223.381404462838\\
3.28300000000001	214.288683493514\\
3.28599999999999	204.554553286898\\
3.28999999999998	190.573018494462\\
3.29	190.573018494358\\
3.29599999999996	167.437093523152\\
3.29699999999999	163.327098209633\\
3.29700000000002	163.327098209515\\
3.29999999999997	150.559035040246\\
3.3	150.559035040122\\
3.30299999999995	137.342137608254\\
3.30599999999991	123.883751469995\\
3.30599999999995	123.883751469794\\
3.30599999999999	123.883751469593\\
3.31	105.556775242575\\
3.31000000000003	105.556775242443\\
3.31400000000004	86.7839538199624\\
3.31800000000005	67.5555280239548\\
3.31999999999997	57.7673490653658\\
3.32	57.7673490652259\\
3.32499999999998	32.7792822374356\\
3.325	32.7792822372914\\
3.32999999999998	7.03544856864592\\
3.33000000000001	7.03544856849739\\
3.33499999999998	-19.485063590791\\
3.33500000000001	-19.485063590944\\
3.33999999999998	-46.8037967197511\\
3.34000000000001	-46.8037967199087\\
3.34499999999998	-74.9429416879963\\
3.34999999999996	-103.925355781539\\
3.35000000000001	-103.925355781853\\
3.35499999999998	-133.774581575695\\
3.35500000000001	-133.774581575868\\
3.35999999999998	-164.514864844389\\
3.36000000000001	-164.514864844567\\
3.36399999999999	-189.080596986865\\
3.36400000000002	-189.080596987036\\
3.36800000000001	-212.87572098316\\
3.36999999999998	-224.488186040071\\
3.37000000000001	-224.488186040235\\
3.37399999999999	-247.150468223259\\
3.37799999999998	-269.072330241289\\
3.37999999999997	-279.759179949428\\
3.38	-279.759179949579\\
3.38799999999998	-320.711503532266\\
3.38999999999997	-330.507468364905\\
3.39	-330.507468365043\\
3.39299999999997	-344.875058556352\\
3.39299999999999	-344.875058556487\\
3.395	-354.237695932768\\
3.39500000000003	-354.2376959329\\
3.39700000000004	-363.429151316842\\
3.39900000000005	-372.450619228655\\
3.39999999999997	-376.897975126554\\
3.4	-376.89797512668\\
3.40400000000002	-394.268233925296\\
3.40800000000004	-410.974552629664\\
3.40999999999997	-419.081461884745\\
3.41	-419.081461884859\\
3.41299999999996	-430.936983760537\\
3.41299999999999	-430.936983760648\\
3.41599999999996	-442.429768770969\\
3.41899999999992	-453.56317764039\\
3.41999999999997	-457.195018270523\\
3.42	-457.195018270626\\
3.42199999999997	-464.208156465885\\
3.42199999999999	-464.208156465982\\
3.42399999999996	-470.799785421047\\
3.42599999999992	-476.970761799874\\
3.42999999999985	-488.053910180599\\
3.42999999999997	-488.053910180929\\
3.43	-488.053910181002\\
3.43799999999985	-505.204012564639\\
3.43999999999997	-508.449827185084\\
3.44	-508.449827185127\\
3.44799999999985	-517.278897740793\\
3.44999999999997	-518.449052713477\\
3.45	-518.449052713491\\
3.45099999999999	-518.878700262209\\
3.45100000000002	-518.87870026222\\
3.45200000000001	-519.204747584568\\
3.453	-519.427205279145\\
3.45499999999999	-519.561377329943\\
3.45899999999995	-518.586738436574\\
3.46	-518.084082651993\\
3.46000000000003	-518.084082651977\\
3.46499999999998	-514.015968686413\\
3.465	-514.015968686382\\
3.46999999999995	-507.353730912662\\
3.47	-507.353730912578\\
3.47000000000003	-507.353730912533\\
3.47099999999999	-505.709549401711\\
3.47100000000002	-505.709549401663\\
3.47199999999999	-503.961339809303\\
3.47299999999996	-502.109045335614\\
3.47499999999989	-498.091957638727\\
3.47899999999976	-488.805888905572\\
3.47999999999996	-486.223125564878\\
3.47999999999999	-486.223125564803\\
3.48799999999973	-462.138759969378\\
3.48999999999997	-455.175692203861\\
3.48999999999999	-455.17569220376\\
3.49799999999973	-423.528365201222\\
3.49999999999999	-414.662628351815\\
3.50000000000002	-414.662628351687\\
3.50799999999976	-375.349491200798\\
3.50899999999999	-369.999581270065\\
3.50900000000002	-369.999581269911\\
3.50999999999999	-364.552273057217\\
3.51000000000002	-364.552273057061\\
3.51099999999999	-359.00738960326\\
3.51199999999996	-353.364750734158\\
3.51399999999991	-341.785470257744\\
3.51799999999979	-317.445564371646\\
3.51999999999997	-304.681775679033\\
3.52	-304.681775678849\\
3.52799999999977	-249.628406187866\\
3.52999999999997	-234.85656669084\\
3.53	-234.856566690627\\
3.53499999999998	-196.141315726702\\
3.535	-196.141315726475\\
3.53799999999997	-171.677621671455\\
3.53799999999999	-171.677621671219\\
3.53999999999997	-154.849725354147\\
3.54	-154.849725353905\\
3.54199999999998	-137.604271909516\\
3.54399999999996	-119.939020121322\\
3.54799999999993	-83.3398835672436\\
3.54999999999997	-64.4012422736343\\
3.55	-64.4012422733622\\
3.55799999999993	15.6718919206187\\
3.55799999999996	15.6718919209713\\
3.55799999999999	15.6718919213284\\
3.56	36.7828256457151\\
3.56000000000003	36.7828256460182\\
3.56200000000004	58.0614859760141\\
3.56400000000005	79.2359034727973\\
3.56699999999996	110.808090404218\\
3.56699999999999	110.808090404516\\
3.56999999999998	142.161151607257\\
3.57	142.161151607553\\
3.57299999999999	173.30425513759\\
3.57599999999997	204.246507662144\\
3.57999999999997	245.206039767853\\
3.58	245.206039768143\\
3.58599999999997	306.052515021193\\
3.58999999999997	346.252369474148\\
3.59	346.252369474433\\
3.59599999999997	406.057705219749\\
3.596	406.057705220031\\
3.59999999999997	445.628525104573\\
3.6	445.628525104853\\
3.60399999999998	484.98383917773\\
3.60499999999998	494.791394781624\\
3.605	494.791394781903\\
3.60899999999998	533.906085120925\\
3.60999999999998	543.657463219285\\
3.61	543.657463219561\\
3.61399999999998	582.563302491314\\
3.61599999999997	601.96084132089\\
3.616	601.960841321165\\
3.61999999999997	640.657762175373\\
3.62	640.657762175647\\
3.62399999999998	678.038588966003\\
3.62499999999996	686.992653189769\\
3.62499999999999	686.992653190021\\
3.62899999999997	721.253067857523\\
3.62999999999997	729.430602358089\\
3.63	729.430602358319\\
3.63399999999998	760.598434465781\\
3.63799999999996	789.310548979301\\
3.63999999999997	802.750426316299\\
3.64	802.750426316485\\
3.64799999999996	850.441903016822\\
3.65	860.855511586764\\
3.65000000000003	860.855511586907\\
3.65399999999999	879.881363744856\\
3.65400000000002	879.881363744983\\
3.65799999999998	896.513508147078\\
3.65999999999997	903.934690445345\\
3.66	903.934690445446\\
3.66399999999997	916.992097628033\\
3.66799999999993	927.675089419509\\
3.66999999999997	932.12796322472\\
3.67	932.127963224779\\
3.67399999999999	939.259360085945\\
3.67400000000002	939.259360085988\\
3.67499999999998	940.672897780098\\
3.67500000000001	940.672897780136\\
3.67599999999997	941.938809920015\\
3.67699999999994	943.057137635565\\
3.67899999999988	944.851180336353\\
3.67999999999997	945.526953610036\\
3.68	945.526953610053\\
3.68299999999996	946.442514382834\\
3.68299999999999	946.442514382835\\
3.68599999999995	945.577071842283\\
3.68899999999992	942.930372921775\\
3.69	941.652186438489\\
3.69000000000003	941.652186438449\\
3.69599999999995	929.822310821403\\
3.69999999999997	917.968049718369\\
3.7	917.968049718274\\
3.70599999999993	894.219735053922\\
3.70999999999997	874.397573799608\\
3.71	874.397573799456\\
3.71199999999999	863.285799269289\\
3.71200000000002	863.285799269125\\
3.71400000000001	851.371671411439\\
3.716	838.653641852821\\
3.71999999999998	810.799161612157\\
3.72000000000001	810.799161611948\\
3.72799999999997	745.363628683301\\
3.73000000000001	726.966128427665\\
3.73000000000004	726.966128427397\\
3.73200000000002	707.748558740647\\
3.73200000000005	707.748558740368\\
3.73400000000003	687.708422218911\\
3.73600000000002	666.843114352176\\
3.73999999999999	622.626030385554\\
3.74000000000002	622.626030385228\\
3.74099999999999	611.109775220251\\
3.74100000000002	611.109775219922\\
3.742	599.500970881847\\
3.74299999999997	587.799240152618\\
3.74499999999993	564.115475746877\\
3.745	564.115475745955\\
3.74500000000003	564.115475745616\\
3.74899999999994	515.615898264014\\
3.75000000000001	503.252866953313\\
3.75000000000004	503.25286695296\\
3.75399999999995	452.836143513036\\
3.75799999999986	400.856292822889\\
3.76	374.271782331771\\
3.76000000000003	374.27178233139\\
3.76799999999985	263.888614152741\\
3.76999999999996	235.263608738052\\
3.76999999999999	235.263608737643\\
3.77799999999981	116.535118909682\\
3.77999999999996	85.7765918900891\\
3.77999999999999	85.7765918896489\\
3.78799999999981	-41.6830681531731\\
3.78999999999996	-74.6750770608391\\
3.78999999999999	-74.6750770613112\\
3.79799999999981	-211.280130788527\\
3.79899999999996	-228.886565310473\\
3.79899999999999	-228.886565310975\\
3.79999999999997	-246.612840274666\\
3.8	-246.612840275172\\
3.80099999999999	-264.459531523541\\
3.80199999999997	-282.4272189702\\
3.80399999999993	-318.727921466721\\
3.80799999999986	-392.807232209249\\
3.80999999999997	-430.595468019166\\
3.81	-430.595468019706\\
3.81499999999998	-527.288873056526\\
3.81500000000001	-527.288873057085\\
3.81899999999996	-606.971528405689\\
3.81899999999999	-606.971528406262\\
3.81999999999996	-627.220874342422\\
3.81999999999999	-627.220874342999\\
3.82099999999996	-647.440849603865\\
3.82199999999993	-667.469958269262\\
3.82399999999988	-706.958172591558\\
3.82799999999976	-783.672421731427\\
3.82799999999996	-783.672421735259\\
3.82799999999999	-783.672421735794\\
3.82999999999996	-820.908426575657\\
3.82999999999999	-820.908426576181\\
3.83199999999996	-857.403502388431\\
3.83399999999993	-893.162392323515\\
3.83799999999988	-962.490108365147\\
3.83999999999996	-996.067944519358\\
3.83999999999999	-996.06794451983\\
3.84799999999988	-1123.2398784706\\
3.84999999999996	-1153.26866852368\\
3.84999999999999	-1153.2686685241\\
3.85699999999996	-1252.89956103482\\
3.85699999999999	-1252.89956103521\\
3.85999999999997	-1293.02147575912\\
3.86	-1293.02147575949\\
3.86299999999999	-1331.61433027872\\
3.86599999999997	-1368.68940981061\\
3.86999999999997	-1415.7805403542\\
3.87	-1415.78054035453\\
3.87599999999997	-1481.44720213225\\
3.87699999999996	-1491.81642343046\\
3.87699999999999	-1491.81642343075\\
3.87999999999996	-1521.94480976647\\
3.87999999999999	-1521.94480976675\\
3.88299999999996	-1549.73668795286\\
3.88499999999998	-1566.48482716172\\
3.88500000000001	-1566.48482716195\\
3.88599999999996	-1574.32579644423\\
3.88599999999999	-1574.32579644445\\
3.88699999999996	-1581.81170902627\\
3.88799999999993	-1588.94280812366\\
3.88999999999986	-1602.14148109366\\
3.88999999999996	-1602.14148109428\\
3.88999999999999	-1602.14148109446\\
3.89399999999986	-1624.29048799013\\
3.89799999999973	-1640.78433142964\\
3.89999999999996	-1646.91336113906\\
3.89999999999999	-1646.91336113913\\
3.90799999999973	-1657.32664646573\\
3.90999999999996	-1656.40595115268\\
3.90999999999999	-1656.40595115265\\
3.91499999999996	-1647.93662619303\\
3.91499999999999	-1647.93662619296\\
3.91999999999997	-1630.65010060529\\
3.92	-1630.65010060513\\
3.92499999999998	-1604.53233268453\\
3.92999999999995	-1569.56210709931\\
3.93	-1569.56210709889\\
3.93499999999996	-1525.71101746251\\
3.93499999999999	-1525.71101746224\\
3.93999999999995	-1472.9434441536\\
3.93999999999999	-1472.94344415317\\
3.94399999999996	-1424.89121426004\\
3.94399999999999	-1424.89121425968\\
3.94799999999997	-1372.3006754184\\
3.94999999999996	-1344.29508907459\\
3.94999999999999	-1344.29508907418\\
3.95399999999996	-1284.8450840918\\
3.95499999999998	-1269.26335625306\\
3.95500000000001	-1269.26335625261\\
3.95899999999998	-1204.04435076416\\
3.95999999999999	-1187.01392717347\\
3.96000000000002	-1187.01392717298\\
3.96399999999999	-1115.97294410774\\
3.96799999999997	-1040.23305903673\\
3.97000000000002	-1000.58881966836\\
3.97000000000005	-1000.58881966779\\
3.97299999999996	-938.890153375273\\
3.97299999999999	-938.890153374676\\
3.97599999999991	-874.496623700241\\
3.97899999999983	-807.389400798111\\
3.97999999999997	-784.413914843656\\
3.98	-784.413914842999\\
3.98599999999984	-640.140903607721\\
3.98999999999997	-537.786679379315\\
3.99	-537.78667937857\\
3.99299999999996	-457.743112583451\\
3.99299999999999	-457.743112582679\\
3.99599999999995	-374.863988296966\\
3.99899999999991	-289.125071443265\\
3.99999999999997	-259.90561391325\\
4	-259.905613912415\\
4.00199999999993	-201.217088162074\\
4.00199999999999	-201.217088160408\\
4.00399999999992	-142.666588751257\\
4.00599999999986	-84.2465063548234\\
4.00999999999972	32.2327604882075\\
4.00999999999995	32.2327604947383\\
4.01	32.2327604963904\\
4.01799999999973	263.959335196482\\
4.01999999999995	321.678257163874\\
4.02	321.678257165513\\
4.02500000000001	465.685388171752\\
4.02500000000006	465.685388173387\\
4.02999999999995	609.371527354436\\
4.03	609.371527356068\\
4.03099999999999	638.08051073589\\
4.03100000000005	638.080510737522\\
4.03200000000004	666.782256332318\\
4.03300000000003	695.477696679156\\
4.035	752.853390678005\\
4.03899999999996	867.570127159875\\
4.03999999999994	896.247527816294\\
4.04	896.247527817925\\
4.04799999999991	1125.78641217534\\
4.04999999999994	1183.23856001756\\
4.05	1183.2385600192\\
4.05099999999999	1211.97987855018\\
4.05100000000005	1211.97987855181\\
4.05200000000004	1240.73260499945\\
4.05300000000002	1269.49767366686\\
4.055	1327.0685763648\\
4.05899999999996	1442.39963999727\\
4.05999999999994	1471.27729839337\\
4.06	1471.27729839502\\
};
\end{axis}
\end{tikzpicture}%}
  \caption{The angular displacement of each pendulum as a function of time.
    \texttt{Blue}: $P_1$, \texttt{Red}: $P_2$, \texttt{Orange}: $P_3$.
    $C_i = 10$ ms.}
  \label{fig:01.5.2}
\end{figure}

\begin{figure}[H]\centering
  \scalebox{1}{% This file was created by matlab2tikz.
%
%The latest updates can be retrieved from
%  http://www.mathworks.com/matlabcentral/fileexchange/22022-matlab2tikz-matlab2tikz
%where you can also make suggestions and rate matlab2tikz.
%
\definecolor{mycolor1}{rgb}{0.00000,0.44700,0.74100}%
\definecolor{mycolor2}{rgb}{0.85000,0.32500,0.09800}%
%
\begin{tikzpicture}

\begin{axis}[%
width=4.133in,
height=3.26in,
at={(0.693in,0.44in)},
scale only axis,
xmin=0,
xmax=1,
xmajorgrids,
ymin=-0.15,
ymax=0.2,
ymajorgrids,
axis background/.style={fill=white}
]
\addplot [color=mycolor1,solid,forget plot]
  table[row sep=crcr]{%
0	0.15314\\
3.15544362088405e-30	0.15314\\
0.000656101980281985	0.153143230512962\\
0.00393661188169191	0.153256312778436\\
0.00599999999999994	0.153410244700375\\
0.006	0.153410244700375\\
0.012	0.152025843789547\\
0.0120000000000001	0.152025843789547\\
0.018	0.146785790333147\\
0.0180000000000001	0.146785790333147\\
0.0199999999999998	0.144179493489919\\
0.02	0.144179493489918\\
0.026	0.133767457951154\\
0.0260000000000002	0.133767457951153\\
0.0289999999999998	0.127361824092399\\
0.029	0.127361824092398\\
0.0319999999999996	0.120505893144387\\
0.0349999999999991	0.113193617075476\\
0.035	0.113193617075474\\
0.0399999999999996	0.0999746854104061\\
0.04	0.0999746854104049\\
0.0449999999999996	0.0854376461080997\\
0.0459999999999996	0.0823689879834041\\
0.046	0.0823689879834027\\
0.047	0.0792808135350665\\
0.0470000000000004	0.0792808135350652\\
0.0490000000000003	0.0731494760963212\\
0.0510000000000002	0.0670761531370235\\
0.055	0.0550940482536636\\
0.0579999999999996	0.0462422041576017\\
0.058	0.0462422041576004\\
0.0599999999999996	0.0404007870498711\\
0.06	0.0404007870498698\\
0.0619999999999995	0.0346045456083864\\
0.0639999999999991	0.0288512070597887\\
0.0659999999999991	0.0231385157858626\\
0.066	0.02313851578586\\
0.0699999999999991	0.0121295381531389\\
0.07	0.0121295381531366\\
0.0700000000000009	0.0121295381531342\\
0.074	0.00186375329871775\\
0.076	-0.00299551806340728\\
0.0760000000000009	-0.0029955180634094\\
0.08	-0.0121763447641911\\
0.0800000000000009	-0.0121763447641931\\
0.0839999999999999	-0.0206520990256193\\
0.086	-0.0246297706860801\\
0.0860000000000009	-0.0246297706860818\\
0.0869999999999991	-0.0265502128214379\\
0.087	-0.0265502128214396\\
0.0880000000000004	-0.0284199307246156\\
0.0890000000000009	-0.0302391076551602\\
0.0910000000000017	-0.0337265468553195\\
0.0929999999999991	-0.0370138976863233\\
0.093	-0.0370138976863247\\
0.0970000000000017	-0.0429934114909054\\
0.0999999999999991	-0.0469618347664145\\
0.1	-0.0469618347664156\\
0.104000000000002	-0.0515714390916565\\
0.104999999999999	-0.0526028899776093\\
0.105	-0.0526028899776102\\
0.105999999999999	-0.053586170072105\\
0.106	-0.0535861700721058\\
0.106999999999999	-0.05452585761117\\
0.107999999999998	-0.0554265265666187\\
0.109999999999997	-0.0571111580107877\\
0.111999999999999	-0.058640724871287\\
0.112	-0.0586407248712876\\
0.115999999999997	-0.0612370029316692\\
0.115999999999998	-0.0612370029316702\\
0.116	-0.0612370029316711\\
0.119999999999997	-0.063219432648722\\
0.119999999999998	-0.0632194326487227\\
0.12	-0.0632194326487234\\
0.123999999999997	-0.0645911231830114\\
0.125999999999999	-0.0650486508548064\\
0.126	-0.0650486508548066\\
0.127999999999998	-0.0653835388327254\\
0.128	-0.0653835388327257\\
0.129999999999998	-0.0656252314014744\\
0.131999999999996	-0.0657738233185884\\
0.135999999999993	-0.0657919017428447\\
0.139999999999998	-0.0654378024812815\\
0.14	-0.0654378024812812\\
0.144999999999998	-0.0644709017753106\\
0.145	-0.0644709017753102\\
0.145999999999998	-0.0642074456912333\\
0.146	-0.0642074456912328\\
0.146999999999999	-0.0639273473319215\\
0.147999999999998	-0.0636373505175543\\
0.149999999999997	-0.0630275468598978\\
0.151999999999998	-0.0623777956425841\\
0.152	-0.0623777956425835\\
0.155999999999997	-0.0609574158314218\\
0.157999999999998	-0.0601862303730122\\
0.158	-0.0601862303730115\\
0.16	-0.0593739834145983\\
0.160000000000002	-0.0593739834145976\\
0.162000000000002	-0.0585203565157101\\
0.164000000000002	-0.0576250150091663\\
0.166	-0.0566876078731048\\
0.166000000000002	-0.056687607873104\\
0.170000000000002	-0.0547623883316001\\
0.174	-0.0528189567458071\\
0.174000000000001	-0.0528189567458063\\
0.175	-0.0523298967241606\\
0.175000000000002	-0.0523298967241597\\
0.176000000000001	-0.0518394601814495\\
0.177	-0.051347599048052\\
0.178999999999998	-0.0503594100348185\\
0.179999999999998	-0.0498629853004126\\
0.18	-0.0498629853004117\\
0.183999999999997	-0.0478606143562568\\
0.186	-0.0468487381919073\\
0.186000000000002	-0.0468487381919064\\
0.189999999999998	-0.0448460616191702\\
0.192	-0.0438655851705541\\
0.192000000000002	-0.0438655851705532\\
0.195999999999998	-0.0419444338237903\\
0.199999999999995	-0.0400738489287693\\
0.199999999999997	-0.0400738489287682\\
0.2	-0.040073848928767\\
0.202999999999998	-0.0387023267293148\\
0.203	-0.038702326729314\\
0.205999999999998	-0.0373563865838793\\
0.206	-0.0373563865838785\\
0.208999999999998	-0.0360456737975151\\
0.209999999999998	-0.0356188041094374\\
0.21	-0.0356188041094366\\
0.211999999999998	-0.0347798648300974\\
0.212	-0.0347798648300967\\
0.213999999999998	-0.0339603857274477\\
0.215999999999997	-0.0331600455233567\\
0.217999999999998	-0.0323785304413606\\
0.218	-0.0323785304413599\\
0.219999999999998	-0.031615534089747\\
0.22	-0.0316155340897463\\
0.221999999999998	-0.0308707573368921\\
0.223999999999996	-0.030143908190053\\
0.225999999999998	-0.0294347016852164\\
0.226	-0.0294347016852158\\
0.229999999999996	-0.0280719343105698\\
0.231999999999998	-0.0274187948368041\\
0.232	-0.0274187948368036\\
0.235999999999996	-0.0261677239027614\\
0.237999999999998	-0.0255693019550791\\
0.238	-0.0255693019550785\\
0.239999999999998	-0.0249886405204315\\
0.24	-0.024988640520431\\
0.241999999999998	-0.0244255119513503\\
0.243999999999996	-0.0238796954709565\\
0.245	-0.0236132121289339\\
0.245000000000002	-0.0236132121289334\\
0.245999999999998	-0.0233509770892397\\
0.246	-0.0233509770892393\\
0.246999999999999	-0.0230927460166266\\
0.247999999999998	-0.0228382749662833\\
0.249999999999997	-0.022340513535244\\
0.252	-0.021857497646821\\
0.252000000000003	-0.0218574976468202\\
0.256	-0.020934950743504\\
0.259999999999997	-0.020069187370574\\
0.26	-0.0200691873705733\\
0.260999999999996	-0.019861457897877\\
0.261	-0.0198614578978763\\
0.261999999999998	-0.0196571717576136\\
0.262999999999996	-0.0194563089176112\\
0.264999999999993	-0.0190647747253879\\
0.265999999999997	-0.0188740649980124\\
0.266	-0.0188740649980117\\
0.269999999999993	-0.0181390028091065\\
0.271999999999997	-0.0177870605763722\\
0.272	-0.0177870605763716\\
0.275999999999993	-0.0171136638873358\\
0.279999999999986	-0.0164800440753242\\
0.279999999999993	-0.0164800440753232\\
0.28	-0.0164800440753221\\
0.285999999999996	-0.015602038849948\\
0.286	-0.0156020388499475\\
0.289999999999996	-0.0150588753922395\\
0.29	-0.015058875392239\\
0.291999999999996	-0.0147974749263449\\
0.292	-0.0147974749263445\\
0.293999999999996	-0.0145427244433615\\
0.295999999999993	-0.0142945240673259\\
0.297999999999996	-0.0140527764903207\\
0.298	-0.0140527764903202\\
0.299999999999996	-0.0138173869356104\\
0.3	-0.01381738693561\\
0.301999999999996	-0.0135882631190604\\
0.303999999999993	-0.013365315211743\\
0.305999999999996	-0.0131484558060676\\
0.306	-0.0131484558060673\\
0.309999999999993	-0.0127295733955093\\
0.313999999999986	-0.0123278675480914\\
0.314999999999997	-0.0122300505117705\\
0.315	-0.0122300505117702\\
0.318999999999997	-0.011848932449539\\
0.319	-0.0118489324495386\\
0.319999999999996	-0.0117561437671958\\
0.32	-0.0117561437671955\\
0.320999999999998	-0.0116643330826606\\
0.321999999999996	-0.0115734913971812\\
0.323999999999993	-0.0113946795037688\\
0.325999999999996	-0.0112196379830671\\
0.326	-0.0112196379830668\\
0.329999999999993	-0.0108791202965964\\
0.331	-0.01079578148901\\
0.331000000000004	-0.0107957814890097\\
0.333	-0.0106311951896303\\
0.333000000000004	-0.01063119518963\\
0.335	-0.0104693436531452\\
0.336999999999996	-0.0103101634250184\\
0.339999999999996	-0.0100762655512865\\
0.34	-0.0100762655512863\\
0.343999999999993	-0.00977317734586651\\
0.345999999999997	-0.00962526027170982\\
0.346	-0.00962526027170956\\
0.347999999999997	-0.00947960356540665\\
0.348	-0.00947960356540639\\
0.349999999999997	-0.00933607010583894\\
0.35	-0.00933607010583868\\
0.351999999999997	-0.00919460362014799\\
0.353999999999993	-0.00905514864622743\\
0.354	-0.00905514864622694\\
0.357999999999993	-0.00878205530691853\\
0.359999999999996	-0.00864830987459709\\
0.36	-0.00864830987459685\\
0.363999999999993	-0.00838615929020967\\
0.365999999999996	-0.00825765136098592\\
0.366	-0.00825765136098569\\
0.369999999999993	-0.00800583848852589\\
0.373999999999986	-0.00776084841554911\\
0.376999999999997	-0.00758135178776813\\
0.377	-0.00758135178776792\\
0.379999999999997	-0.00740531830425179\\
0.38	-0.00740531830425158\\
0.382999999999996	-0.00723259268666171\\
0.384999999999997	-0.00711920461197321\\
0.385	-0.00711920461197301\\
0.385999999999997	-0.00706302255451955\\
0.386	-0.00706302255451935\\
0.386999999999998	-0.00700720978382274\\
0.387999999999996	-0.00695179614931686\\
0.388999999999997	-0.00689677621988011\\
0.389	-0.00689677621987991\\
0.390999999999997	-0.00678789594421677\\
0.392999999999993	-0.00668052627018024\\
0.394999999999997	-0.00657462510307921\\
0.395	-0.00657462510307902\\
0.398999999999993	-0.00636706277582973\\
0.399999999999997	-0.00631602581140577\\
0.4	-0.00631602581140559\\
0.403999999999993	-0.00611514234694707\\
0.405999999999997	-0.00601659044921387\\
0.406	-0.0060165904492137\\
0.409999999999993	-0.00582364296493607\\
0.411999999999997	-0.00572931417942692\\
0.412	-0.00572931417942676\\
0.415999999999993	-0.00554476159961642\\
0.419999999999986	-0.00536544202466481\\
0.419999999999996	-0.00536544202466433\\
0.42	-0.00536544202466417\\
0.426	-0.005105661471752\\
0.426000000000004	-0.00510566147175185\\
0.432000000000004	-0.00485715184030643\\
0.432000000000007	-0.00485715184030629\\
0.434999999999997	-0.0047372395980762\\
0.435	-0.00473723959807606\\
0.43799999999999	-0.00462008052598476\\
0.439999999999997	-0.00454345288070487\\
0.44	-0.00454345288070474\\
0.44299999999999	-0.00443065390957599\\
0.445999999999979	-0.0043203376519555\\
0.445999999999995	-0.00432033765195493\\
0.446	-0.00432033765195474\\
0.447	-0.00428412121687065\\
0.447000000000004	-0.00428412121687052\\
0.448000000000004	-0.00424820789709132\\
0.449000000000004	-0.00421259417212973\\
0.451000000000004	-0.00414225157361891\\
0.454999999999997	-0.00400500985723923\\
0.455	-0.00400500985723911\\
0.459	-0.00387218082350991\\
0.459999999999997	-0.00383963848634947\\
0.46	-0.00383963848634935\\
0.463999999999997	-0.0037120331307038\\
0.464	-0.00371203313070369\\
0.465999999999997	-0.00364972516138714\\
0.466	-0.00364972516138703\\
0.466999999999997	-0.00361894787712166\\
0.467	-0.00361894787712155\\
0.467999999999998	-0.00358843606096643\\
0.468999999999997	-0.00355818672244122\\
0.470999999999993	-0.0034984636447164\\
0.472999999999997	-0.00343975522928542\\
0.473	-0.00343975522928531\\
0.476999999999993	-0.00332529070805253\\
0.479999999999997	-0.00324193742300783\\
0.48	-0.00324193742300773\\
0.483999999999993	-0.00313398671261266\\
0.485999999999997	-0.00308132950356731\\
0.486	-0.00308132950356722\\
0.489999999999993	-0.00297868852415155\\
0.49	-0.00297868852415136\\
0.490000000000004	-0.00297868852415127\\
0.492999999999997	-0.00290402940479505\\
0.493	-0.00290402940479497\\
0.495999999999993	-0.00283128690175424\\
0.498999999999986	-0.00276039684508185\\
0.498999999999993	-0.00276039684508169\\
0.499	-0.00276039684508152\\
0.499999999999997	-0.00273716762589667\\
0.5	-0.00273716762589659\\
0.500999999999998	-0.00271413499636033\\
0.501999999999997	-0.00269129669864501\\
0.503999999999993	-0.00264619416394156\\
0.505999999999993	-0.00260184233916467\\
0.506	-0.00260184233916451\\
0.507999999999997	-0.00255824665041905\\
0.508000000000004	-0.0025582466504189\\
0.51	-0.00251541282024629\\
0.511999999999997	-0.0024733240554731\\
0.51599999999999	-0.00239131600395722\\
0.519999999999993	-0.00231209387431912\\
0.52	-0.00231209387431898\\
0.521999999999993	-0.00227348853796277\\
0.522	-0.00227348853796264\\
0.523999999999993	-0.00223553342270299\\
0.524999999999993	-0.00221679503717306\\
0.525	-0.00221679503717292\\
0.526	-0.00219821364578049\\
0.526000000000007	-0.00219821364578036\\
0.527000000000007	-0.00217979147688287\\
0.528000000000007	-0.00216153077443716\\
0.530000000000007	-0.00212548662559605\\
0.532	-0.00209006706801044\\
0.532000000000007	-0.00209006706801032\\
0.536000000000007	-0.00202104642138933\\
0.538	-0.00198741827253858\\
0.538000000000007	-0.00198741827253846\\
0.539999999999993	-0.00195436058517631\\
0.54	-0.00195436058517619\\
0.541999999999986	-0.00192186039908287\\
0.543999999999972	-0.00188990497243143\\
0.546	-0.0018584817769738\\
0.546000000000007	-0.00185848177697369\\
0.549999999999979	-0.00179723505058651\\
0.550999999999993	-0.00178225327986291\\
0.551	-0.00178225327986281\\
0.554999999999972	-0.00172360178594778\\
0.556999999999993	-0.0016950213156329\\
0.557	-0.0016950213156328\\
0.559999999999993	-0.00165305062302131\\
0.56	-0.00165305062302121\\
0.562999999999993	-0.00161212710529833\\
0.565999999999986	-0.00157221466169964\\
0.565999999999993	-0.00157221466169954\\
0.566	-0.00157221466169945\\
0.571999999999986	-0.00149538666290301\\
0.571999999999993	-0.00149538666290292\\
0.572	-0.00149538666290283\\
0.577999999999986	-0.00142239911156992\\
0.579999999999993	-0.00139887846372692\\
0.58	-0.00139887846372684\\
0.585999999999986	-0.00133061300722733\\
0.585999999999993	-0.00133061300722725\\
0.586	-0.00133061300722717\\
0.591999999999986	-0.00126570205795457\\
0.591999999999993	-0.0012657020579545\\
0.592	-0.00126570205795442\\
0.594999999999993	-0.00123446833087104\\
0.595	-0.00123446833087096\\
0.597999999999993	-0.00120401212899401\\
0.599999999999993	-0.00118412665745423\\
0.6	-0.00118412665745416\\
0.602999999999993	-0.00115490695376066\\
0.605999999999986	-0.00112639458966449\\
0.606	-0.00112639458966436\\
0.606999999999993	-0.00111704573809149\\
0.607	-0.00111704573809143\\
0.607999999999999	-0.00110777666046391\\
0.608999999999997	-0.00109858644815858\\
0.609000000000004	-0.00109858644815852\\
0.611	-0.00108043902421022\\
0.612999999999997	-0.00106259635152698\\
0.614999999999997	-0.0010450514348206\\
0.615000000000004	-0.00104505143482053\\
0.618999999999997	-0.00101082746959056\\
0.619999999999993	-0.00100244696554173\\
0.62	-0.00100244696554167\\
0.623999999999993	-0.00096960225516176\\
0.625999999999993	-0.000953575039258187\\
0.626	-0.000953575039258131\\
0.629999999999993	-0.000922314872374938\\
0.63	-0.000922314872374883\\
0.633999999999993	-0.000892096646841945\\
0.635999999999993	-0.00087736338994188\\
0.636	-0.000877363389941828\\
0.637999999999993	-0.00086287296956934\\
0.638	-0.000862872969569289\\
0.639999999999993	-0.000848619704775458\\
0.64	-0.000848619704775408\\
0.641999999999993	-0.000834598007511279\\
0.643999999999986	-0.000820802380553502\\
0.645999999999993	-0.000807227415273633\\
0.646	-0.000807227415273585\\
0.649999999999986	-0.0007807494801423\\
0.65	-0.000780749480142208\\
0.650000000000007	-0.000780749480142162\\
0.653999999999993	-0.00075515388929724\\
0.657999999999979	-0.000730400499675287\\
0.659999999999993	-0.000718327447732739\\
0.66	-0.000718327447732696\\
0.664999999999993	-0.000688992555410453\\
0.665	-0.000688992555410412\\
0.665999999999993	-0.000683266293778606\\
0.666	-0.000683266293778566\\
0.666999999999998	-0.000677587299556187\\
0.667000000000006	-0.000677587299556147\\
0.668000000000004	-0.000671956698056672\\
0.669000000000002	-0.000666373937422595\\
0.670999999999998	-0.00065534975471149\\
0.673000000000005	-0.000644510429349712\\
0.673000000000013	-0.000644510429349674\\
0.677000000000005	-0.000623369423345519\\
0.678	-0.000618193153072279\\
0.678000000000007	-0.000618193153072243\\
0.679999999999993	-0.000607967823817857\\
0.68	-0.00060796782381782\\
0.681999999999986	-0.000597908775826181\\
0.683999999999972	-0.000588012065407668\\
0.686	-0.000578273812518879\\
0.686000000000007	-0.000578273812518844\\
0.689999999999979	-0.000559280295899554\\
0.69399999999995	-0.000540921264682568\\
0.695999999999993	-0.000531970644944006\\
0.696	-0.000531970644943975\\
0.699999999999993	-0.000514509651760496\\
0.7	-0.000514509651760466\\
0.703999999999993	-0.000497612928358538\\
0.705999999999993	-0.000489367852791729\\
0.706	-0.0004893678527917\\
0.707999999999993	-0.000481258764669065\\
0.708	-0.000481258764669036\\
0.709999999999993	-0.000473287275730152\\
0.711999999999986	-0.00046545026072113\\
0.713999999999993	-0.000457744647110169\\
0.714	-0.000457744647110142\\
0.717999999999986	-0.000442715590537679\\
0.719999999999993	-0.000435386255375497\\
0.72	-0.000435386255375471\\
0.723999999999986	-0.000421083602878829\\
0.724999999999993	-0.000417580061616165\\
0.725	-0.00041758006161614\\
0.725999999999993	-0.000414104678126119\\
0.726	-0.000414104678126094\\
0.726999999999999	-0.000410658112971318\\
0.727999999999997	-0.000407241029513212\\
0.729999999999993	-0.000400493970904834\\
0.731999999999993	-0.000393860857090771\\
0.732	-0.000393860857090748\\
0.734999999999993	-0.000384119156237618\\
0.735	-0.000384119156237595\\
0.737999999999993	-0.000374619396625751\\
0.74	-0.000368416488403757\\
0.740000000000007	-0.000368416488403735\\
0.743	-0.000359301443504857\\
0.745999999999993	-0.000350406446863639\\
0.746000000000007	-0.000350406446863598\\
0.746999999999993	-0.000347489822771658\\
0.747	-0.000347489822771637\\
0.747999999999999	-0.000344598163091137\\
0.748999999999997	-0.000341731184360338\\
0.750999999999993	-0.000336070148164364\\
0.753999999999993	-0.000327756753293898\\
0.754	-0.000327756753293879\\
0.757999999999993	-0.000316993431217072\\
0.759999999999993	-0.000311744671911115\\
0.76	-0.000311744671911096\\
0.763999999999993	-0.000301502667102156\\
0.766	-0.000296505406181009\\
0.766000000000007	-0.000296505406180992\\
0.77	-0.000286759795598509\\
0.770000000000007	-0.000286759795598492\\
0.774	-0.000277340867131346\\
0.776	-0.000272749272973379\\
0.776000000000007	-0.000272749272973363\\
0.779999999999993	-0.000263792823452599\\
0.78	-0.000263792823452584\\
0.782999999999993	-0.000257266983875963\\
0.783	-0.000257266983875948\\
0.785999999999993	-0.00025089887536838\\
0.786000000000001	-0.000250898875368365\\
0.788999999999994	-0.000244688191913039\\
0.791999999999987	-0.000238634766370875\\
0.792	-0.000238634766370847\\
0.792000000000008	-0.000238634766370833\\
0.797999999999994	-0.000226978462895243\\
0.799999999999993	-0.00022322092671551\\
0.8	-0.000223220926715497\\
0.804999999999993	-0.000214092868713142\\
0.805000000000001	-0.00021409286871313\\
0.805999999999993	-0.000212311385418157\\
0.806	-0.000212311385418145\\
0.806999999999994	-0.000210544699061964\\
0.807999999999987	-0.00020879313470276\\
0.809999999999973	-0.000205334686754476\\
0.811999999999993	-0.000201934685675467\\
0.812	-0.000201934685675455\\
0.815999999999973	-0.000195304714664832\\
0.817999999999993	-0.000192072145426957\\
0.818000000000001	-0.000192072145426945\\
0.819999999999993	-0.000188892823472461\\
0.82	-0.00018889282347245\\
0.821999999999993	-0.000185765502352623\\
0.823999999999986	-0.000182688955988815\\
0.825999999999993	-0.000179661978209226\\
0.826	-0.000179661978209215\\
0.829999999999986	-0.000173758748889766\\
0.833999999999972	-0.00016805330440493\\
0.839999999999993	-0.000159846491347907\\
0.84	-0.000159846491347897\\
0.840999999999993	-0.000158517949109501\\
0.841000000000001	-0.000158517949109491\\
0.841999999999994	-0.000157200273212734\\
0.842999999999987	-0.000155893334450983\\
0.844999999999973	-0.000153311156997933\\
0.845999999999993	-0.000152035665221624\\
0.846	-0.000152035665221615\\
0.849999999999973	-0.000147040493095658\\
0.851999999999993	-0.000144606132116096\\
0.852	-0.000144606132116087\\
0.855999999999973	-0.000139859088005421\\
0.857999999999993	-0.000137544543776967\\
0.858	-0.000137544543776958\\
0.86	-0.00013526810311292\\
0.860000000000007	-0.000135268103112912\\
0.862000000000007	-0.000133028873537915\\
0.864000000000007	-0.000130825977153128\\
0.866	-0.000128658550304601\\
0.866000000000007	-0.000128658550304593\\
0.87	-0.000124431571562097\\
0.870000000000007	-0.000124431571562089\\
0.874	-0.000120346159599326\\
0.874999999999994	-0.000119346170814537\\
0.875000000000001	-0.00011934617081453\\
0.876	-0.000118354530447473\\
0.876000000000007	-0.000118354530447466\\
0.877000000000007	-0.000117371141310725\\
0.878000000000006	-0.000116395907018226\\
0.879999999999998	-0.000114469521423438\\
0.880000000000006	-0.000114469521423431\\
0.882000000000005	-0.000112574618421282\\
0.884000000000004	-0.000110710455112767\\
0.886000000000005	-0.000108876300645523\\
0.886000000000013	-0.000108876300645517\\
0.888000000000007	-0.000107072464331694\\
0.888000000000014	-0.000107072464331687\\
0.890000000000009	-0.000105299267369401\\
0.892000000000004	-0.000103556014569742\\
0.895999999999993	-0.000100156619152621\\
0.898999999999993	-9.76806754823634e-05\\
0.899000000000001	-9.76806754823576e-05\\
0.899999999999993	-9.68689464304756e-05\\
0.9	-9.68689464304698e-05\\
0.900999999999994	-9.6063877013777e-05\\
0.901999999999987	-9.52653882900104e-05\\
0.903999999999973	-9.36878405177649e-05\\
0.905999999999993	-9.21356846300636e-05\\
0.906	-9.21356846300581e-05\\
0.909999999999973	-8.91086105839449e-05\\
0.909999999999987	-8.91086105839346e-05\\
0.910000000000001	-8.91086105839244e-05\\
0.910999999999993	-8.83678628003285e-05\\
0.911000000000001	-8.83678628003233e-05\\
0.911999999999994	-8.76333771741875e-05\\
0.912999999999987	-8.69050817155083e-05\\
0.914999999999973	-8.54667763781069e-05\\
0.916999999999993	-8.40523828949788e-05\\
0.917000000000001	-8.40523828949738e-05\\
0.919999999999993	-8.19744169173503e-05\\
0.92	-8.19744169173455e-05\\
0.922999999999993	-7.99471740298812e-05\\
0.925999999999986	-7.79688658961427e-05\\
0.925999999999993	-7.7968865896138e-05\\
0.926	-7.79688658961333e-05\\
0.927999999999993	-7.66770543219352e-05\\
0.928000000000001	-7.66770543219306e-05\\
0.929999999999994	-7.5407184820723e-05\\
0.931999999999987	-7.41587595138373e-05\\
0.933999999999994	-7.29312889513422e-05\\
0.934000000000001	-7.29312889513379e-05\\
0.937999999999987	-7.0537295177499e-05\\
0.939999999999993	-6.93698333914894e-05\\
0.940000000000001	-6.93698333914852e-05\\
0.943999999999987	-6.70916913305193e-05\\
0.944999999999994	-6.65336590054775e-05\\
0.945000000000001	-6.65336590054735e-05\\
0.945999999999993	-6.59801178998789e-05\\
0.946000000000001	-6.59801178998749e-05\\
0.946999999999994	-6.54311699499357e-05\\
0.947999999999987	-6.48869175379424e-05\\
0.949999999999973	-6.3812286413191e-05\\
0.952	-6.27558032123494e-05\\
0.952000000000008	-6.27558032123457e-05\\
0.95599999999998	-6.06956307672097e-05\\
0.956999999999994	-6.01912913074925e-05\\
0.957000000000001	-6.01912913074889e-05\\
0.96	-5.87031691074909e-05\\
0.960000000000008	-5.87031691074874e-05\\
0.963000000000007	-5.72513743971698e-05\\
0.966000000000007	-5.58346264728782e-05\\
0.966000000000014	-5.58346264728749e-05\\
0.969000000000007	-5.44528649793914e-05\\
0.969000000000014	-5.44528649793881e-05\\
0.972000000000007	-5.31060604824633e-05\\
0.975	-5.17930248772815e-05\\
0.979999999999994	-4.9676541243674e-05\\
0.980000000000001	-4.9676541243671e-05\\
0.985999999999987	-4.72490603455223e-05\\
0.986000000000001	-4.72490603455167e-05\\
0.991999999999987	-4.49400389592495e-05\\
0.992000000000001	-4.49400389592443e-05\\
0.997999999999987	-4.27453534169505e-05\\
0.998000000000001	-4.27453534169455e-05\\
0.999999999999993	-4.20378496340661e-05\\
1	-4.20378496340636e-05\\
1.00199999999999	-4.13419087208149e-05\\
1.00399999999999	-4.06572578191729e-05\\
1.00599999999999	-3.99836285091284e-05\\
1.006	-3.99836285091237e-05\\
1.00999999999999	-3.86698957854157e-05\\
1.01399999999997	-3.74001634276287e-05\\
1.01499999999999	-3.708937054333e-05\\
1.015	-3.70893705433256e-05\\
1.01999999999999	-3.55737274428047e-05\\
1.02	-3.55737274428004e-05\\
1.02499999999999	-3.41192521065527e-05\\
1.02599999999999	-3.3835380944933e-05\\
1.026	-3.3835380944929e-05\\
1.02699999999999	-3.35538656382162e-05\\
1.027	-3.35538656382122e-05\\
1.02799999999999	-3.32747585990048e-05\\
1.02899999999999	-3.29980324694628e-05\\
1.03099999999997	-3.24516146819163e-05\\
1.03299999999999	-3.19143980503788e-05\\
1.033	-3.1914398050375e-05\\
1.03699999999997	-3.08667293216363e-05\\
1.04	-3.01035897706801e-05\\
1.04000000000001	-3.01035897706765e-05\\
1.04399999999999	-2.91149397160916e-05\\
1.044	-2.91149397160882e-05\\
1.046	-2.86325503103997e-05\\
1.04600000000001	-2.86325503103963e-05\\
1.04800000000001	-2.81581355429319e-05\\
1.05	-2.76917801657761e-05\\
1.05000000000001	-2.76917801657728e-05\\
1.05200000000001	-2.723330134373e-05\\
1.05200000000002	-2.72333013437268e-05\\
1.05400000000002	-2.67825193296244e-05\\
1.05600000000002	-2.63392573938291e-05\\
1.05800000000001	-2.59033417538038e-05\\
1.05800000000002	-2.59033417538007e-05\\
1.05999999999999	-2.5474601509497e-05\\
1.06	-2.54746015094939e-05\\
1.06199999999996	-2.50528685738976e-05\\
1.06399999999992	-2.46379776049399e-05\\
1.06599999999999	-2.42297659430081e-05\\
1.066	-2.42297659430052e-05\\
1.06999999999992	-2.34336593983795e-05\\
1.07299999999999	-2.28541444051524e-05\\
1.073	-2.28541444051497e-05\\
1.07699999999992	-2.2103905639253e-05\\
1.07899999999999	-2.17380751798321e-05\\
1.079	-2.17380751798295e-05\\
1.07999999999999	-2.15574190393219e-05\\
1.08	-2.15574190393193e-05\\
1.08099999999999	-2.13782453018156e-05\\
1.08199999999999	-2.12005364057517e-05\\
1.08399999999997	-2.08494436105259e-05\\
1.08499999999999	-2.0676025299972e-05\\
1.085	-2.06760252999696e-05\\
1.08599999999999	-2.05040030060072e-05\\
1.086	-2.05040030060048e-05\\
1.08699999999999	-2.03334083333607e-05\\
1.08799999999999	-2.01642730254155e-05\\
1.08999999999997	-1.98303143369535e-05\\
1.09199999999999	-1.9501996010831e-05\\
1.092	-1.95019960108287e-05\\
1.09599999999997	-1.88617677405113e-05\\
1.09999999999995	-1.82425841060603e-05\\
1.09999999999997	-1.82425841060561e-05\\
1.1	-1.8242584106052e-05\\
1.10199999999999	-1.79405792921785e-05\\
1.102	-1.79405792921763e-05\\
1.10399999999999	-1.76434739873749e-05\\
1.10599999999997	-1.73511516923925e-05\\
1.10599999999999	-1.73511516923903e-05\\
1.106	-1.73511516923883e-05\\
1.10999999999997	-1.6781055835308e-05\\
1.11199999999999	-1.65032228232008e-05\\
1.112	-1.65032228231988e-05\\
1.11599999999997	-1.59614419683402e-05\\
1.11999999999994	-1.54374695874575e-05\\
1.12	-1.54374695874501e-05\\
1.12000000000001	-1.54374695874483e-05\\
1.126	-1.46831115783722e-05\\
1.12600000000001	-1.46831115783705e-05\\
1.13099999999999	-1.40826238211678e-05\\
1.131	-1.40826238211662e-05\\
1.132	-1.39655673521018e-05\\
1.13200000000001	-1.39655673521001e-05\\
1.13300000000001	-1.38494975788315e-05\\
1.13400000000001	-1.37344031242957e-05\\
1.13600000000001	-1.35070951440944e-05\\
1.138	-1.32835542945336e-05\\
1.13800000000001	-1.3283554294532e-05\\
1.13999999999999	-1.30636929367676e-05\\
1.14	-1.30636929367661e-05\\
1.14199999999997	-1.28474248744612e-05\\
1.14399999999994	-1.26346653188649e-05\\
1.14599999999999	-1.24253308567719e-05\\
1.146	-1.24253308567705e-05\\
1.14999999999994	-1.2017080284122e-05\\
1.15399999999989	-1.16225029095039e-05\\
1.15499999999999	-1.15259220212841e-05\\
1.155	-1.15259220212827e-05\\
1.15999999999999	-1.10549259910857e-05\\
1.16	-1.10549259910844e-05\\
1.16499999999999	-1.06029379264291e-05\\
1.16599999999999	-1.05147229999796e-05\\
1.166	-1.05147229999784e-05\\
1.17099999999999	-1.00847077908383e-05\\
1.17199999999999	-1.00008824871519e-05\\
1.172	-1.00008824871507e-05\\
1.173	-9.91776376173988e-06\\
1.17300000000001	-9.9177637617387e-06\\
1.17400000000001	-9.83534346881401e-06\\
1.17500000000001	-9.75361352960694e-06\\
1.17700000000001	-9.59219273768623e-06\\
1.17999999999999	-9.35504110010023e-06\\
1.18	-9.35504110009912e-06\\
1.184	-9.04780999310616e-06\\
1.18599999999999	-8.89790339915877e-06\\
1.186	-8.89790339915772e-06\\
1.18899999999999	-8.6777034118826e-06\\
1.189	-8.67770341188157e-06\\
1.18999999999999	-8.60555087815935e-06\\
1.19	-8.60555087815833e-06\\
1.19099999999999	-8.53401023871499e-06\\
1.19199999999999	-8.46307448174067e-06\\
1.19399999999997	-8.32298986403477e-06\\
1.19499999999999	-8.2538272732956e-06\\
1.195	-8.25382727329462e-06\\
1.19899999999997	-7.98288418604953e-06\\
1.19999999999999	-7.91654203892987e-06\\
1.2	-7.91654203892893e-06\\
1.20399999999997	-7.65655286739852e-06\\
1.20599999999999	-7.52969692657781e-06\\
1.206	-7.52969692657691e-06\\
1.20699999999999	-7.46704952780753e-06\\
1.207	-7.46704952780665e-06\\
1.20799999999999	-7.40493804245904e-06\\
1.20899999999999	-7.34335638272751e-06\\
1.21099999999997	-7.22175844887707e-06\\
1.21499999999995	-6.98465832714022e-06\\
1.21799999999999	-6.81198437922706e-06\\
1.218	-6.81198437922625e-06\\
1.21999999999999	-6.69923636513134e-06\\
1.22	-6.69923636513054e-06\\
1.22199999999999	-6.5883310412403e-06\\
1.22399999999997	-6.47922491998147e-06\\
1.22499999999999	-6.42533316579872e-06\\
1.225	-6.42533316579796e-06\\
1.22599999999999	-6.37187522576968e-06\\
1.226	-6.37187522576892e-06\\
1.22699999999999	-6.3188609252805e-06\\
1.22799999999999	-6.26630013280449e-06\\
1.22999999999997	-6.16251850981608e-06\\
1.23	-6.1625185098147e-06\\
1.23399999999997	-5.96017361010833e-06\\
1.236	-5.86153100355807e-06\\
1.23600000000001	-5.86153100355737e-06\\
1.23999999999999	-5.66911209821449e-06\\
1.24	-5.66911209821382e-06\\
1.24399999999997	-5.48293117095466e-06\\
1.24599999999999	-5.39208832980723e-06\\
1.246	-5.39208832980659e-06\\
1.247	-5.34722588613589e-06\\
1.24700000000001	-5.34722588613526e-06\\
1.24800000000001	-5.30274721643946e-06\\
1.24900000000001	-5.25864796117829e-06\\
1.25100000000001	-5.17157044198016e-06\\
1.253	-5.08595918946622e-06\\
1.25300000000001	-5.08595918946562e-06\\
1.25700000000001	-4.91900179135527e-06\\
1.25999999999999	-4.79738711157346e-06\\
1.26	-4.79738711157289e-06\\
1.264	-4.63983471791204e-06\\
1.266	-4.56296053566249e-06\\
1.26600000000001	-4.56296053566195e-06\\
1.27000000000001	-4.41303812081527e-06\\
1.272	-4.33997426245643e-06\\
1.27200000000001	-4.33997426245592e-06\\
1.276	-4.19749800956031e-06\\
1.27600000000001	-4.19749800955981e-06\\
1.27999999999999	-4.05970495345541e-06\\
1.28000000000001	-4.05970495345493e-06\\
1.28399999999999	-3.92637898522597e-06\\
1.28599999999999	-3.86132554959552e-06\\
1.28600000000001	-3.86132554959506e-06\\
1.288	-3.79734751779369e-06\\
1.28800000000001	-3.79734751779324e-06\\
1.29	-3.73445632351069e-06\\
1.29199999999999	-3.67262731000453e-06\\
1.29499999999999	-3.58182246485316e-06\\
1.295	-3.58182246485274e-06\\
1.29899999999998	-3.46424411588173e-06\\
1.29999999999999	-3.4354543087659e-06\\
1.3	-3.4354543087655e-06\\
1.30399999999998	-3.32262954319313e-06\\
1.30499999999999	-3.29499314494559e-06\\
1.305	-3.2949931449452e-06\\
1.30599999999999	-3.26757921342542e-06\\
1.306	-3.26757921342503e-06\\
1.30699999999999	-3.24039278693323e-06\\
1.30700000000001	-3.24039278693284e-06\\
1.308	-3.21343892586474e-06\\
1.30899999999999	-3.18671498846163e-06\\
1.31099999999998	-3.13394643003336e-06\\
1.31300000000001	-3.08206642353151e-06\\
1.31300000000002	-3.08206642353115e-06\\
1.31699999999999	-2.98089104891791e-06\\
1.31999999999999	-2.9071931457201e-06\\
1.32	-2.90719314571975e-06\\
1.32399999999997	-2.81171716818259e-06\\
1.32599999999999	-2.76513179317114e-06\\
1.326	-2.76513179317081e-06\\
1.32999999999997	-2.67427955214637e-06\\
1.33	-2.67427955214579e-06\\
1.33399999999997	-2.5864701600707e-06\\
1.334	-2.58647016007012e-06\\
1.33799999999997	-2.50156590033478e-06\\
1.34	-2.46016141016012e-06\\
1.34000000000001	-2.46016141015982e-06\\
1.34399999999999	-2.37936653988299e-06\\
1.346	-2.33994448383389e-06\\
1.34600000000001	-2.33994448383361e-06\\
1.348	-2.30117411735104e-06\\
1.34800000000001	-2.30117411735077e-06\\
1.35	-2.26306236891223e-06\\
1.35199999999999	-2.22559429665541e-06\\
1.35599999999996	-2.15253066967173e-06\\
1.35999999999999	-2.08186864330043e-06\\
1.36	-2.08186864330018e-06\\
1.36299999999999	-2.0303813715584e-06\\
1.363	-2.03038137155816e-06\\
1.36499999999999	-1.99674987871469e-06\\
1.365	-1.99674987871446e-06\\
1.36599999999999	-1.98013717357585e-06\\
1.366	-1.98013717357561e-06\\
1.36699999999999	-1.96366233525072e-06\\
1.36799999999999	-1.9473284303516e-06\\
1.36999999999997	-1.91507703091651e-06\\
1.37199999999999	-1.8833703309852e-06\\
1.372	-1.88337033098498e-06\\
1.37599999999997	-1.82154151613983e-06\\
1.378	-1.79139516099549e-06\\
1.37800000000001	-1.79139516099528e-06\\
1.37999999999999	-1.76174501581677e-06\\
1.38	-1.76174501581656e-06\\
1.38199999999997	-1.73257945630545e-06\\
1.38399999999994	-1.70388704798853e-06\\
1.38599999999999	-1.67565654189632e-06\\
1.386	-1.67565654189612e-06\\
1.38999999999994	-1.62060053392155e-06\\
1.39199999999999	-1.59376929346458e-06\\
1.392	-1.59376929346439e-06\\
1.39599999999994	-1.54144773968102e-06\\
1.39799999999999	-1.51593691348235e-06\\
1.398	-1.51593691348217e-06\\
1.39999999999999	-1.49084599621229e-06\\
1.4	-1.49084599621211e-06\\
1.40199999999999	-1.46616515101157e-06\\
1.40399999999997	-1.4418847016572e-06\\
1.40599999999999	-1.41799512890556e-06\\
1.406	-1.41799512890539e-06\\
1.40999999999997	-1.3714049434806e-06\\
1.412	-1.34869947465724e-06\\
1.41200000000001	-1.34869947465708e-06\\
1.41599999999999	-1.30442327301461e-06\\
1.41999999999996	-1.26160243071545e-06\\
1.41999999999998	-1.26160243071522e-06\\
1.42	-1.261602430715e-06\\
1.42099999999999	-1.25111670079919e-06\\
1.421	-1.25111670079905e-06\\
1.42199999999999	-1.24071669557977e-06\\
1.42299999999999	-1.23040139526944e-06\\
1.42499999999997	-1.21002087355164e-06\\
1.42599999999999	-1.19995365460354e-06\\
1.426	-1.1999536546034e-06\\
1.42999999999997	-1.16052752266519e-06\\
1.43199999999999	-1.14131341558744e-06\\
1.432	-1.1413134155873e-06\\
1.43499999999999	-1.11309472116344e-06\\
1.435	-1.11309472116331e-06\\
1.43799999999998	-1.08557690585826e-06\\
1.43999999999999	-1.06760906053696e-06\\
1.44	-1.06760906053683e-06\\
1.44299999999999	-1.0412057285083e-06\\
1.44599999999997	-1.01543985861267e-06\\
1.44599999999999	-1.01543985861253e-06\\
1.446	-1.0154398586124e-06\\
1.44699999999999	-1.00699134973081e-06\\
1.447	-1.00699134973069e-06\\
1.44799999999999	-9.98615113433553e-07\\
1.44899999999999	-9.90310328620577e-07\\
1.44999999999999	-9.82076181335089e-07\\
1.45	-9.82076181334972e-07\\
1.45199999999999	-9.65816578069283e-07\\
1.45399999999997	-9.49829929037836e-07\\
1.45599999999999	-9.34109966613451e-07\\
1.456	-9.3410996661334e-07\\
1.45999999999997	-9.03445551816085e-07\\
1.46	-9.03445551815871e-07\\
1.46000000000001	-9.03445551815764e-07\\
1.46399999999999	-8.73775240836027e-07\\
1.466	-8.59298273570955e-07\\
1.46600000000001	-8.59298273570853e-07\\
1.46999999999999	-8.3106484305882e-07\\
1.47	-8.31064843058722e-07\\
1.47399999999997	-8.03777013858898e-07\\
1.47599999999999	-7.90474269540854e-07\\
1.476	-7.90474269540761e-07\\
1.47899999999998	-7.70931916199948e-07\\
1.479	-7.70931916199856e-07\\
1.47999999999999	-7.64525042761406e-07\\
1.48	-7.64525042761315e-07\\
1.48099999999999	-7.58170740579864e-07\\
1.48199999999999	-7.51868386860326e-07\\
1.48399999999997	-7.39417059053101e-07\\
1.48599999999999	-7.27166177748748e-07\\
1.486	-7.27166177748662e-07\\
1.48999999999997	-7.0327412912771e-07\\
1.491	-6.97427583694441e-07\\
1.49100000000001	-6.97427583694358e-07\\
1.49499999999999	-6.74530076842024e-07\\
1.49899999999996	-6.52387687670176e-07\\
1.49999999999999	-6.46965982437778e-07\\
1.5	-6.46965982437701e-07\\
1.50499999999999	-6.20514293231628e-07\\
1.505	-6.20514293231554e-07\\
1.50599999999999	-6.15351693299492e-07\\
1.506	-6.15351693299419e-07\\
1.50699999999999	-6.10231937175109e-07\\
1.50799999999999	-6.05155977810956e-07\\
1.508	-6.05155977810884e-07\\
1.50999999999999	-5.951334636196e-07\\
1.51199999999997	-5.85280221386688e-07\\
1.51399999999999	-5.75592388110665e-07\\
1.514	-5.75592388110597e-07\\
1.51799999999997	-5.56697819413171e-07\\
1.518	-5.56697819413046e-07\\
1.51999999999999	-5.47483676301743e-07\\
1.52	-5.47483676301678e-07\\
1.52199999999999	-5.38420124017042e-07\\
1.52399999999997	-5.29503609157478e-07\\
1.52599999999999	-5.20730635969083e-07\\
1.526	-5.20730635969022e-07\\
1.52999999999997	-5.03621311646199e-07\\
1.53399999999994	-4.8708501728248e-07\\
1.537	-4.75043212661656e-07\\
1.53700000000001	-4.750432126616e-07\\
1.53999999999999	-4.63298508796156e-07\\
1.54	-4.63298508796101e-07\\
1.54299999999997	-4.51840545741969e-07\\
1.54599999999994	-4.40659215145519e-07\\
1.54599999999997	-4.40659215145417e-07\\
1.546	-4.40659215145314e-07\\
1.549	-4.2975402991908e-07\\
1.54900000000001	-4.29754029919029e-07\\
1.55200000000001	-4.1912474703272e-07\\
1.55500000000001	-4.08761989696897e-07\\
1.55500000000003	-4.08761989696848e-07\\
1.55999999999999	-3.9205828064636e-07\\
1.56	-3.92058280646314e-07\\
1.56499999999996	-3.76028684247008e-07\\
1.56599999999998	-3.72900173450336e-07\\
1.566	-3.72900173450292e-07\\
1.57099999999996	-3.57649839507866e-07\\
1.57199999999998	-3.54677005739556e-07\\
1.572	-3.54677005739514e-07\\
1.57499999999999	-3.45907703154635e-07\\
1.575	-3.45907703154594e-07\\
1.57799999999999	-3.37356207353412e-07\\
1.57999999999999	-3.31772480900963e-07\\
1.58	-3.31772480900924e-07\\
1.58299999999999	-3.23567324397437e-07\\
1.58599999999997	-3.15560267043704e-07\\
1.586	-3.15560267043631e-07\\
1.58699999999999	-3.12934790192868e-07\\
1.587	-3.1293479019283e-07\\
1.58799999999999	-3.10331772929613e-07\\
1.58899999999999	-3.07750960087046e-07\\
1.59099999999997	-3.02654938029343e-07\\
1.59499999999995	-2.92718360093961e-07\\
1.59499999999997	-2.92718360093896e-07\\
1.595	-2.9271836009383e-07\\
1.59999999999999	-2.80756674576221e-07\\
1.6	-2.80756674576188e-07\\
1.60499999999999	-2.69277727852186e-07\\
1.60599999999999	-2.67037371457235e-07\\
1.606	-2.67037371457203e-07\\
1.60699999999998	-2.64815607498324e-07\\
1.607	-2.64815607498293e-07\\
1.60799999999999	-2.62612849555249e-07\\
1.60899999999999	-2.60428881715541e-07\\
1.60999999999998	-2.58263489927241e-07\\
1.61	-2.5826348992721e-07\\
1.61199999999999	-2.53987587382857e-07\\
1.61399999999997	-2.49783465549024e-07\\
1.61599999999999	-2.45649476183591e-07\\
1.616	-2.45649476183562e-07\\
1.61999999999997	-2.37585438831863e-07\\
1.61999999999999	-2.37585438831834e-07\\
1.62	-2.37585438831805e-07\\
1.62399999999997	-2.29782828014192e-07\\
1.624	-2.29782828014144e-07\\
1.62599999999999	-2.25975717904888e-07\\
1.626	-2.25975717904861e-07\\
1.62799999999999	-2.22231543444522e-07\\
1.62999999999998	-2.18550973730036e-07\\
1.63199999999999	-2.14932565779908e-07\\
1.632	-2.14932565779882e-07\\
1.63599999999998	-2.07876584631267e-07\\
1.63999999999995	-2.01052533680171e-07\\
1.63999999999998	-2.0105253368013e-07\\
1.64	-2.0105253368009e-07\\
1.645	-1.92832350197346e-07\\
1.64500000000001	-1.92832350197323e-07\\
1.64599999999999	-1.9122800961036e-07\\
1.646	-1.91228009610337e-07\\
1.64699999999999	-1.89636983257357e-07\\
1.64799999999999	-1.88059567279348e-07\\
1.64999999999997	-1.84944949354369e-07\\
1.65199999999999	-1.81882934737465e-07\\
1.652	-1.81882934737443e-07\\
1.65299999999998	-1.80371277732718e-07\\
1.653	-1.80371277732697e-07\\
1.65399999999999	-1.78872322973086e-07\\
1.65499999999997	-1.77385923539781e-07\\
1.65699999999995	-1.7445020913795e-07\\
1.65899999999998	-1.71562983569048e-07\\
1.659	-1.71562983569027e-07\\
1.65999999999999	-1.70137199636775e-07\\
1.66	-1.70137199636754e-07\\
1.66099999999999	-1.68723114903204e-07\\
1.66199999999998	-1.67320590768011e-07\\
1.66399999999995	-1.64549675564482e-07\\
1.66599999999999	-1.61823367678176e-07\\
1.666	-1.61823367678156e-07\\
1.66999999999995	-1.56506437506861e-07\\
1.67399999999991	-1.51367583346046e-07\\
1.67999999999998	-1.43975637480808e-07\\
1.68	-1.43975637480791e-07\\
1.68199999999998	-1.41592131203646e-07\\
1.682	-1.41592131203629e-07\\
1.68399999999998	-1.39247292542355e-07\\
1.68599999999997	-1.36940202052067e-07\\
1.68599999999999	-1.36940202052049e-07\\
1.686	-1.36940202052033e-07\\
1.68999999999997	-1.32440842633399e-07\\
1.69199999999999	-1.30248104741518e-07\\
1.692	-1.30248104741502e-07\\
1.69599999999997	-1.25972213951827e-07\\
1.69799999999999	-1.23887384674211e-07\\
1.698	-1.23887384674196e-07\\
1.69999999999999	-1.21836871740851e-07\\
1.7	-1.21836871740837e-07\\
1.70199999999999	-1.19819871251032e-07\\
1.70399999999998	-1.17835592431731e-07\\
1.70599999999999	-1.15883257338806e-07\\
1.706	-1.15883257338792e-07\\
1.70999999999998	-1.12075752908577e-07\\
1.71099999999998	-1.1114403089721e-07\\
1.711	-1.11144030897197e-07\\
1.71499999999997	-1.0749501946948e-07\\
1.715	-1.07495019469458e-07\\
1.71699999999998	-1.05715990597716e-07\\
1.717	-1.05715990597704e-07\\
1.71899999999998	-1.03966345737004e-07\\
1.71999999999999	-1.03102327506724e-07\\
1.72	-1.03102327506712e-07\\
1.72199999999999	-1.01395476024732e-07\\
1.72399999999997	-9.97163146860125e-08\\
1.72599999999999	-9.80641851688996e-08\\
1.726	-9.8064185168888e-08\\
1.72899999999998	-9.56373482049413e-08\\
1.729	-9.56373482049299e-08\\
1.73199999999998	-9.32719104375856e-08\\
1.73499999999997	-9.09657851911036e-08\\
1.73999999999998	-8.72485462369585e-08\\
1.74	-8.72485462369481e-08\\
1.74599999999997	-8.29851061664371e-08\\
1.74599999999998	-8.29851061664257e-08\\
1.746	-8.29851061664141e-08\\
1.74999999999998	-8.02585158985349e-08\\
1.75	-8.02585158985254e-08\\
1.75199999999998	-7.89297272427249e-08\\
1.752	-7.89297272427156e-08\\
1.75399999999998	-7.76232453177377e-08\\
1.75599999999997	-7.6338557912159e-08\\
1.75799999999998	-7.50751613593442e-08\\
1.758	-7.50751613593353e-08\\
1.75999999999999	-7.38325603466569e-08\\
1.76	-7.38325603466481e-08\\
1.76199999999999	-7.26102677141145e-08\\
1.76399999999998	-7.1407804257055e-08\\
1.76599999999999	-7.02246985450567e-08\\
1.766	-7.02246985450484e-08\\
1.76899999999998	-6.8486817437662e-08\\
1.769	-6.84868174376539e-08\\
1.77199999999998	-6.67929048806059e-08\\
1.77499999999996	-6.5141466585487e-08\\
1.77499999999998	-6.51414665854777e-08\\
1.775	-6.51414665854685e-08\\
1.78	-6.24795162982505e-08\\
1.78000000000002	-6.24795162982431e-08\\
1.78499999999998	-5.99249944998296e-08\\
1.785	-5.99249944998225e-08\\
1.78600000000001	-5.94264261737771e-08\\
1.78600000000003	-5.94264261737701e-08\\
1.78700000000005	-5.89319954077242e-08\\
1.78800000000006	-5.84417942311079e-08\\
1.79000000000009	-5.7473888876745e-08\\
1.79200000000003	-5.65223306397248e-08\\
1.79200000000004	-5.65223306397181e-08\\
1.79600000000011	-5.46667695547242e-08\\
1.79799999999998	-5.37620392268632e-08\\
1.798	-5.37620392268568e-08\\
1.8	-5.28722007837155e-08\\
1.80000000000002	-5.28722007837092e-08\\
1.80200000000002	-5.19969053654383e-08\\
1.80400000000002	-5.11358098090003e-08\\
1.806	-5.02885765185956e-08\\
1.80600000000002	-5.02885765185896e-08\\
1.80999999999998	-4.86362758870654e-08\\
1.81	-4.86362758870596e-08\\
1.81399999999997	-4.70393145393403e-08\\
1.81799999999994	-4.54951878674641e-08\\
1.82	-4.47421775616881e-08\\
1.82000000000001	-4.47421775616828e-08\\
1.826	-4.25558305689674e-08\\
1.82600000000001	-4.25558305689624e-08\\
1.827	-4.22017639742739e-08\\
1.82700000000001	-4.22017639742689e-08\\
1.828	-4.18507262349338e-08\\
1.82899999999999	-4.15026829396494e-08\\
1.83099999999996	-4.0815443523925e-08\\
1.83200000000001	-4.04761800455945e-08\\
1.83200000000003	-4.04761800455897e-08\\
1.83599999999998	-3.91473950511301e-08\\
1.83800000000001	-3.84995090327869e-08\\
1.83800000000003	-3.84995090327824e-08\\
1.83999999999999	-3.78622872354582e-08\\
1.84	-3.78622872354538e-08\\
1.84199999999996	-3.72354798372176e-08\\
1.84399999999992	-3.66188410957449e-08\\
1.84599999999999	-3.60121292554251e-08\\
1.846	-3.60121292554209e-08\\
1.84999999999992	-3.48289010140466e-08\\
1.85399999999984	-3.36853017625717e-08\\
1.85499999999998	-3.34053824571342e-08\\
1.855	-3.34053824571302e-08\\
1.85599999999998	-3.31278001407826e-08\\
1.856	-3.31278001407786e-08\\
1.85699999999999	-3.2852527619997e-08\\
1.85799999999997	-3.25795379025201e-08\\
1.85999999999995	-3.20403000752706e-08\\
1.85999999999998	-3.20403000752638e-08\\
1.86	-3.20403000752569e-08\\
1.86399999999995	-3.09880554735901e-08\\
1.86599999999999	-3.04746361639989e-08\\
1.866	-3.04746361639953e-08\\
1.86799999999998	-2.99697042299132e-08\\
1.868	-2.99697042299096e-08\\
1.86999999999998	-2.94733499071916e-08\\
1.87199999999996	-2.89853785981544e-08\\
1.87299999999998	-2.87444766622487e-08\\
1.873	-2.87444766622453e-08\\
1.87699999999996	-2.78008784246803e-08\\
1.87999999999999	-2.71135450349128e-08\\
1.88	-2.71135450349096e-08\\
1.88399999999997	-2.62231013986987e-08\\
1.88499999999998	-2.60049874485378e-08\\
1.885	-2.60049874485347e-08\\
1.886	-2.57886292648668e-08\\
1.88600000000002	-2.57886292648638e-08\\
1.88700000000002	-2.55740666113488e-08\\
1.88800000000002	-2.53613394260269e-08\\
1.88999999999998	-2.4941308239916e-08\\
1.89	-2.4941308239913e-08\\
1.892	-2.45283710327764e-08\\
1.89200000000002	-2.45283710327735e-08\\
1.89400000000002	-2.41223659120891e-08\\
1.89600000000003	-2.37231337030588e-08\\
1.898	-2.33305178851724e-08\\
1.89800000000002	-2.33305178851696e-08\\
1.9	-2.29443645339956e-08\\
1.90000000000002	-2.29443645339929e-08\\
1.902	-2.2564522258656e-08\\
1.90399999999998	-2.2190842140487e-08\\
1.906	-2.18231776767582e-08\\
1.90600000000002	-2.18231776767556e-08\\
1.90999999999998	-2.11061470402425e-08\\
1.91399999999995	-2.04131313808562e-08\\
1.91399999999997	-2.04131313808518e-08\\
1.914	-2.04131313808474e-08\\
1.91999999999998	-1.94162682523969e-08\\
1.92	-1.94162682523946e-08\\
1.92499999999998	-1.86224195898359e-08\\
1.925	-1.86224195898337e-08\\
1.92599999999998	-1.84674834298463e-08\\
1.926	-1.84674834298441e-08\\
1.92699999999999	-1.8313833066864e-08\\
1.92799999999997	-1.81614971001522e-08\\
1.92999999999995	-1.78607087608587e-08\\
1.93199999999998	-1.75650004867345e-08\\
1.932	-1.75650004867324e-08\\
1.93599999999995	-1.69883623530699e-08\\
1.9399999999999	-1.64306783229831e-08\\
1.93999999999999	-1.64306783229713e-08\\
1.94	-1.64306783229694e-08\\
1.94299999999998	-1.60243266729119e-08\\
1.943	-1.602432667291e-08\\
1.94599999999998	-1.56277856954226e-08\\
1.946	-1.56277856954198e-08\\
1.94899999999998	-1.52410380971515e-08\\
1.95199999999997	-1.48640752480954e-08\\
1.95199999999998	-1.48640752480932e-08\\
1.952	-1.4864075248091e-08\\
1.95799999999997	-1.41381819830471e-08\\
1.95999999999998	-1.39041748992468e-08\\
1.96	-1.39041748992451e-08\\
1.96599999999996	-1.32247410380267e-08\\
1.96599999999998	-1.32247410380248e-08\\
1.966	-1.32247410380228e-08\\
1.97199999999996	-1.2578464393245e-08\\
1.97199999999998	-1.25784643932432e-08\\
1.972	-1.25784643932414e-08\\
1.97799999999996	-1.19641898798447e-08\\
1.97799999999998	-1.1964189879843e-08\\
1.978	-1.19641898798412e-08\\
1.98	-1.17661654672371e-08\\
1.98000000000002	-1.17661654672357e-08\\
1.98200000000002	-1.15713774588398e-08\\
1.98400000000002	-1.13797494839859e-08\\
1.986	-1.11912064141954e-08\\
1.98600000000002	-1.11912064141941e-08\\
1.99000000000002	-1.08235038738479e-08\\
1.99400000000003	-1.04681165230686e-08\\
1.995	-1.03811282006563e-08\\
1.99500000000001	-1.03811282006551e-08\\
1.99999999999999	-9.95691227019017e-09\\
2	-9.95691227018899e-09\\
2.00099999999997	-9.87415601302283e-09\\
2.001	-9.87415601302048e-09\\
2.00199999999999	-9.79207631566493e-09\\
2.00299999999997	-9.71066513273897e-09\\
2.00499999999995	-9.54981645834466e-09\\
2.00599999999997	-9.47036320173591e-09\\
2.006	-9.47036320173366e-09\\
2.00999999999995	-9.15920134191744e-09\\
2.01199999999997	-9.00755833307511e-09\\
2.012	-9.00755833307297e-09\\
2.01299999999998	-8.93269513185397e-09\\
2.01300000000001	-8.93269513185185e-09\\
2.014	-8.85846099580636e-09\\
2.01499999999999	-8.78484864893714e-09\\
2.01699999999996	-8.63946052385991e-09\\
2.01999999999997	-8.42586332845568e-09\\
2.02	-8.42586332845369e-09\\
2.02399999999995	-8.14914715807546e-09\\
2.02599999999997	-8.0141296679016e-09\\
2.026	-8.01412966789969e-09\\
2.02999999999995	-7.7508143724346e-09\\
2.03	-7.75081437243172e-09\\
2.03399999999995	-7.49631810159424e-09\\
2.03599999999997	-7.37225184535954e-09\\
2.036	-7.37225184535779e-09\\
2.03999999999995	-7.13023987191646e-09\\
2.04	-7.13023987191334e-09\\
2.04399999999995	-6.89607363978448e-09\\
2.04599999999997	-6.78181744392865e-09\\
2.046	-6.78181744392704e-09\\
2.04799999999997	-6.66945002629583e-09\\
2.048	-6.66945002629424e-09\\
2.04999999999996	-6.55899146794323e-09\\
2.05199999999993	-6.450398463156e-09\\
2.05599999999987	-6.23863953290441e-09\\
2.05899999999997	-6.08440591409081e-09\\
2.059	-6.08440591408937e-09\\
2.05999999999997	-6.03384111045324e-09\\
2.06	-6.03384111045181e-09\\
2.06099999999999	-5.98369121589198e-09\\
2.06199999999998	-5.93395131288756e-09\\
2.06399999999995	-5.83568202113935e-09\\
2.06499999999997	-5.78714300080149e-09\\
2.065	-5.78714300080012e-09\\
2.06599999999997	-5.73899470825349e-09\\
2.066	-5.73899470825212e-09\\
2.06699999999999	-5.69124599249692e-09\\
2.06799999999998	-5.64390574134267e-09\\
2.06999999999995	-5.55043211306818e-09\\
2.07099999999997	-5.50428957438758e-09\\
2.071	-5.50428957438628e-09\\
2.07499999999995	-5.32357617608778e-09\\
2.07699999999997	-5.23547166903123e-09\\
2.077	-5.23547166902999e-09\\
2.07999999999997	-5.10603279254529e-09\\
2.08	-5.10603279254408e-09\\
2.08299999999998	-4.9797540912234e-09\\
2.08599999999995	-4.8565241680537e-09\\
2.086	-4.85652416805177e-09\\
2.08799999999997	-4.77605678697469e-09\\
2.088	-4.77605678697356e-09\\
2.08999999999997	-4.69695635957358e-09\\
2.09199999999993	-4.61919187288798e-09\\
2.09399999999997	-4.54273283904935e-09\\
2.094	-4.54273283904828e-09\\
2.09799999999993	-4.3936117275811e-09\\
2.09999999999997	-4.32089118643073e-09\\
2.1	-4.3208911864297e-09\\
2.10399999999993	-4.17898757203687e-09\\
2.10599999999997	-4.10974886492132e-09\\
2.106	-4.10974886492035e-09\\
2.10999999999993	-3.9747173915453e-09\\
2.112	-3.90891055089514e-09\\
2.11200000000003	-3.90891055089421e-09\\
2.11599999999996	-3.78058565402447e-09\\
2.11699999999997	-3.74917121257253e-09\\
2.117	-3.74917121257164e-09\\
2.11999999999997	-3.65647879810409e-09\\
2.12	-3.65647879810322e-09\\
2.12299999999998	-3.56604941518713e-09\\
2.12599999999995	-3.47780329135854e-09\\
2.126	-3.47780329135713e-09\\
2.12899999999997	-3.39173658233563e-09\\
2.129	-3.39173658233483e-09\\
2.13199999999996	-3.30784737000392e-09\\
2.13499999999993	-3.22606165017266e-09\\
2.13499999999997	-3.22606165017175e-09\\
2.135	-3.22606165017085e-09\\
2.13999999999997	-3.09423140190775e-09\\
2.14	-3.09423140190701e-09\\
2.14499999999998	-2.9677214349161e-09\\
2.14599999999997	-2.94303037042066e-09\\
2.146	-2.94303037041996e-09\\
2.14699999999997	-2.91854421395811e-09\\
2.14699999999999	-2.91854421395741e-09\\
2.14799999999998	-2.89426752361201e-09\\
2.14899999999997	-2.87019791980234e-09\\
2.15099999999995	-2.82267055556964e-09\\
2.15299999999999	-2.77594348797241e-09\\
2.15300000000002	-2.77594348797175e-09\\
2.15699999999997	-2.68481727228376e-09\\
2.15999999999997	-2.61843934966851e-09\\
2.16	-2.61843934966789e-09\\
2.16399999999995	-2.53244643891227e-09\\
2.16599999999997	-2.49048811397047e-09\\
2.166	-2.49048811396988e-09\\
2.16999999999995	-2.40865968773891e-09\\
2.17	-2.40865968773797e-09\\
2.17399999999995	-2.32957188049521e-09\\
2.17499999999997	-2.31021352298225e-09\\
2.175	-2.3102135229817e-09\\
2.17899999999995	-2.23437754607426e-09\\
2.17999999999997	-2.21580862418798e-09\\
2.18	-2.21580862418745e-09\\
2.18399999999995	-2.14303862332668e-09\\
2.18599999999997	-2.10753212267536e-09\\
2.186	-2.10753212267486e-09\\
2.187	-2.08999735245014e-09\\
2.18700000000002	-2.08999735244965e-09\\
2.18800000000002	-2.07261258282432e-09\\
2.18800000000005	-2.07261258282383e-09\\
2.18900000000004	-2.05537610984398e-09\\
2.19000000000004	-2.03828624415599e-09\\
2.19200000000003	-2.00453964892216e-09\\
2.19600000000001	-1.93873298928637e-09\\
2.2	-1.87508939265602e-09\\
2.20000000000003	-1.87508939265558e-09\\
2.20399999999997	-1.81350904606906e-09\\
2.204	-1.81350904606863e-09\\
2.20499999999997	-1.79842495596648e-09\\
2.205	-1.79842495596605e-09\\
2.20599999999999	-1.78346228932641e-09\\
2.20600000000003	-1.78346228932579e-09\\
2.20700000000002	-1.76862379597292e-09\\
2.20800000000001	-1.7539122379575e-09\\
2.20999999999998	-1.7248641726959e-09\\
2.21200000000003	-1.69630670513471e-09\\
2.21200000000006	-1.69630670513431e-09\\
2.21600000000001	-1.64061896749459e-09\\
2.21800000000003	-1.6134668648184e-09\\
2.21800000000006	-1.61346686481801e-09\\
2.21999999999997	-1.58676168647414e-09\\
2.22	-1.58676168647377e-09\\
2.22199999999992	-1.56049296273329e-09\\
2.22399999999983	-1.53465039483897e-09\\
2.226	-1.50922385110997e-09\\
2.22600000000003	-1.50922385110961e-09\\
2.22999999999986	-1.45963621722653e-09\\
2.23299999999997	-1.42353954717499e-09\\
2.233	-1.42353954717465e-09\\
2.23699999999983	-1.37680885088518e-09\\
2.23899999999997	-1.35402207535656e-09\\
2.239	-1.35402207535624e-09\\
2.24	-1.34276939779072e-09\\
2.24000000000003	-1.3427693977904e-09\\
2.24100000000003	-1.3316090535144e-09\\
2.24200000000003	-1.32053994865601e-09\\
2.24400000000004	-1.29867112647859e-09\\
2.246	-1.27715435750312e-09\\
2.24600000000003	-1.27715435750281e-09\\
2.25000000000004	-1.23519168755036e-09\\
2.252	-1.21474141278389e-09\\
2.25200000000003	-1.2147414127836e-09\\
2.25600000000004	-1.17486289297273e-09\\
2.25999999999997	-1.13629518070236e-09\\
2.26	-1.13629518070209e-09\\
2.26199999999997	-1.11748389715364e-09\\
2.262	-1.11748389715338e-09\\
2.26399999999996	-1.0989777879204e-09\\
2.26599999999993	-1.08076959760946e-09\\
2.26599999999997	-1.08076959760914e-09\\
2.266	-1.08076959760882e-09\\
2.26999999999994	-1.04525942023253e-09\\
2.272	-1.02795373197208e-09\\
2.27200000000003	-1.02795373197183e-09\\
2.27499999999997	-1.00253782658167e-09\\
2.275	-1.00253782658143e-09\\
2.27799999999994	-9.77753186379527e-10\\
2.27999999999997	-9.61569976505633e-10\\
2.28	-9.61569976505405e-10\\
2.28299999999994	-9.37789124946285e-10\\
2.28599999999989	-9.14582420408248e-10\\
2.28599999999997	-9.1458242040758e-10\\
2.286	-9.14582420407362e-10\\
2.28699999999997	-9.06973049958816e-10\\
2.287	-9.069730499586e-10\\
2.28799999999999	-8.99428773710269e-10\\
2.28899999999997	-8.91948852116791e-10\\
2.29099999999995	-8.77179146689534e-10\\
2.291	-8.77179146689196e-10\\
2.29499999999995	-8.48380148659471e-10\\
2.29699999999997	-8.34339565336599e-10\\
2.297	-8.34339565336401e-10\\
2.29999999999997	-8.13711820037174e-10\\
2.3	-8.13711820036981e-10\\
2.30299999999998	-7.93587689175409e-10\\
2.30599999999995	-7.73949420208547e-10\\
2.306	-7.73949420208246e-10\\
2.30999999999997	-7.48520243150941e-10\\
2.31	-7.48520243150763e-10\\
2.31399999999997	-7.23942746888686e-10\\
2.31599999999997	-7.11961282718487e-10\\
2.316	-7.11961282718318e-10\\
2.31999999999997	-6.88589433850948e-10\\
2.32	-6.88589433850785e-10\\
2.32399999999997	-6.65975272601678e-10\\
2.32599999999997	-6.54941196445864e-10\\
2.326	-6.54941196445709e-10\\
2.32999999999997	-6.33422199008811e-10\\
2.33199999999997	-6.22935034795106e-10\\
2.332	-6.22935034794958e-10\\
2.33599999999997	-6.02484816352391e-10\\
2.33999999999994	-5.82706796871669e-10\\
2.34	-5.82706796871356e-10\\
2.34000000000003	-5.82706796871218e-10\\
2.34499999999997	-5.58882392537186e-10\\
2.345	-5.58882392537054e-10\\
2.34600000000003	-5.54232562163309e-10\\
2.34600000000006	-5.54232562163177e-10\\
2.34700000000009	-5.49621320185701e-10\\
2.34800000000012	-5.45049524904668e-10\\
2.34899999999997	-5.40516728236649e-10\\
2.349	-5.40516728236521e-10\\
2.35100000000006	-5.31566357473108e-10\\
2.35300000000011	-5.22766698891713e-10\\
2.35499999999997	-5.14114302553359e-10\\
2.355	-5.14114302553237e-10\\
2.35800000000003	-5.01404419877563e-10\\
2.35800000000006	-5.01404419877444e-10\\
2.35999999999997	-4.93105461450458e-10\\
2.36	-4.93105461450341e-10\\
2.36199999999992	-4.849421365681e-10\\
2.36399999999983	-4.76911244785728e-10\\
2.36599999999997	-4.69009637560487e-10\\
2.366	-4.69009637560376e-10\\
2.36999999999983	-4.5359967816564e-10\\
2.37399999999966	-4.3870583316e-10\\
2.37799999999997	-4.24304743013925e-10\\
2.378	-4.24304743013824e-10\\
2.37999999999997	-4.17281893796132e-10\\
2.38	-4.17281893796033e-10\\
2.38199999999997	-4.10373822535638e-10\\
2.38399999999995	-4.03577820478402e-10\\
2.38599999999997	-3.9689122322555e-10\\
2.386	-3.96891223225456e-10\\
2.38999999999995	-3.83850813957594e-10\\
2.39	-3.83850813957444e-10\\
2.39399999999995	-3.71247157499278e-10\\
2.396	-3.65102908450872e-10\\
2.39600000000002	-3.65102908450786e-10\\
2.39999999999997	-3.53117523537237e-10\\
2.4	-3.53117523537153e-10\\
2.40399999999995	-3.41520690614927e-10\\
2.40599999999997	-3.35862274369834e-10\\
2.406	-3.35862274369754e-10\\
2.40699999999997	-3.33067883855288e-10\\
2.407	-3.33067883855209e-10\\
2.40799999999998	-3.30297397860602e-10\\
2.40899999999997	-3.27550544839006e-10\\
2.41099999999995	-3.22126663124085e-10\\
2.41299999999997	-3.16794112256723e-10\\
2.413	-3.16794112256648e-10\\
2.41499999999997	-3.11550801618787e-10\\
2.415	-3.11550801618713e-10\\
2.41699999999997	-3.06394675577696e-10\\
2.41899999999995	-3.01323712653857e-10\\
2.41999999999997	-2.98819544794563e-10\\
2.42	-2.98819544794493e-10\\
2.42399999999995	-2.8900592727624e-10\\
2.426	-2.84217591254418e-10\\
2.42600000000003	-2.8421759125435e-10\\
2.427	-2.81852887036128e-10\\
2.42700000000003	-2.81852887036061e-10\\
2.42800000000002	-2.79508411600243e-10\\
2.429	-2.77183935155039e-10\\
2.43099999999998	-2.72594069882784e-10\\
2.43499999999993	-2.63644431616398e-10\\
2.43599999999997	-2.61453675942175e-10\\
2.436	-2.61453675942113e-10\\
2.43999999999997	-2.52870827160071e-10\\
2.44	-2.52870827160011e-10\\
2.44399999999998	-2.44566224542467e-10\\
2.446	-2.40514178717604e-10\\
2.44600000000003	-2.40514178717547e-10\\
2.448	-2.36529117583715e-10\\
2.44800000000002	-2.36529117583659e-10\\
2.44999999999999	-2.32611753307324e-10\\
2.45000000000002	-2.32611753307269e-10\\
2.45199999999998	-2.28760550070214e-10\\
2.45399999999995	-2.24973998003557e-10\\
2.45600000000002	-2.21250612574871e-10\\
2.45600000000005	-2.21250612574819e-10\\
2.45999999999998	-2.13987526843694e-10\\
2.46000000000001	-2.13987526843643e-10\\
2.46399999999994	-2.06959901743582e-10\\
2.46499999999997	-2.05238486699303e-10\\
2.465	-2.05238486699255e-10\\
2.46599999999998	-2.03530928644891e-10\\
2.46600000000001	-2.03530928644842e-10\\
2.46699999999999	-2.01837541406044e-10\\
2.46799999999998	-2.00158640184797e-10\\
2.46999999999996	-1.96843638999486e-10\\
2.47199999999998	-1.93584625429924e-10\\
2.47200000000001	-1.93584625429878e-10\\
2.47599999999996	-1.87229471830543e-10\\
2.47999999999991	-1.81083212218112e-10\\
2.47999999999997	-1.81083212218014e-10\\
2.48	-1.81083212217971e-10\\
2.48499999999997	-1.73679489399505e-10\\
2.485	-1.73679489399464e-10\\
2.48599999999997	-1.72234498173133e-10\\
2.486	-1.72234498173092e-10\\
2.48699999999999	-1.70801498759175e-10\\
2.48799999999998	-1.69380757884777e-10\\
2.48999999999995	-1.66575495952824e-10\\
2.49199999999997	-1.63817612563486e-10\\
2.492	-1.63817612563447e-10\\
2.494	-1.61106026492917e-10\\
2.49400000000002	-1.61106026492879e-10\\
2.49600000000002	-1.58439674667629e-10\\
2.49800000000001	-1.55817511733848e-10\\
2.49999999999997	-1.53238509662561e-10\\
2.5	-1.53238509662525e-10\\
2.50399999999999	-1.48205960263291e-10\\
2.50599999999997	-1.45750439904683e-10\\
2.506	-1.45750439904648e-10\\
2.50999999999999	-1.40961607966064e-10\\
2.51399999999998	-1.36333164834736e-10\\
2.51999999999997	-1.29675415851786e-10\\
2.52	-1.29675415851756e-10\\
2.52299999999997	-1.2646837729115e-10\\
2.523	-1.2646837729112e-10\\
2.52599999999996	-1.23338767245638e-10\\
2.526	-1.233387672456e-10\\
2.52899999999997	-1.20286449231785e-10\\
2.53199999999993	-1.17311355139667e-10\\
2.53199999999997	-1.17311355139635e-10\\
2.532	-1.17311355139603e-10\\
2.53799999999993	-1.11582406597712e-10\\
2.53799999999997	-1.11582406597681e-10\\
2.538	-1.11582406597651e-10\\
2.53999999999997	-1.09735558574493e-10\\
2.54	-1.09735558574467e-10\\
2.54199999999998	-1.0791889443866e-10\\
2.54399999999995	-1.06131701929468e-10\\
2.54599999999997	-1.04373280371184e-10\\
2.546	-1.04373280371159e-10\\
2.54999999999995	-1.00943952128948e-10\\
2.55199999999997	-9.92726880049976e-11\\
2.552	-9.9272688004974e-11\\
2.55499999999997	-9.68181949962042e-11\\
2.555	-9.68181949961812e-11\\
2.55799999999998	-9.44246652301689e-11\\
2.55800000000001	-9.44246652301465e-11\\
2.55999999999997	-9.28618023354243e-11\\
2.56	-9.28618023354023e-11\\
2.56199999999997	-9.13244820000885e-11\\
2.56399999999994	-8.98121015152303e-11\\
2.56599999999997	-8.83240679462208e-11\\
2.566	-8.83240679461998e-11\\
2.56999999999994	-8.54220587440507e-11\\
2.56999999999997	-8.5422058744027e-11\\
2.57	-8.54220587440032e-11\\
2.57399999999994	-8.26172443914214e-11\\
2.57799999999988	-7.99052259338734e-11\\
2.57999999999997	-7.85826803907801e-11\\
2.58	-7.85826803907615e-11\\
2.58099999999997	-7.79295453081817e-11\\
2.581	-7.79295453081632e-11\\
2.58199999999998	-7.72817498450949e-11\\
2.58299999999997	-7.66392304810246e-11\\
2.58499999999995	-7.53697686664986e-11\\
2.58599999999997	-7.47427017932331e-11\\
2.586	-7.47427017932153e-11\\
2.58999999999995	-7.22869271288568e-11\\
2.59	-7.22869271288283e-11\\
2.59199999999997	-7.10901189482887e-11\\
2.592	-7.10901189482719e-11\\
2.59399999999997	-6.99134019071945e-11\\
2.59599999999995	-6.87563146689936e-11\\
2.59799999999997	-6.76184035931566e-11\\
2.598	-6.76184035931406e-11\\
2.59999999999997	-6.64992225634527e-11\\
2.6	-6.64992225634369e-11\\
2.60199999999997	-6.53983328065227e-11\\
2.60399999999995	-6.43153027141899e-11\\
2.60599999999997	-6.32497076803159e-11\\
2.606	-6.32497076803009e-11\\
2.60999999999995	-6.11715512064784e-11\\
2.61	-6.11715512064525e-11\\
2.61399999999994	-5.91629968912518e-11\\
2.61599999999997	-5.81838319959368e-11\\
2.616	-5.8183831995923e-11\\
2.61999999999994	-5.62738071762015e-11\\
2.61999999999997	-5.62738071761869e-11\\
2.62	-5.62738071761723e-11\\
2.62399999999995	-5.44257030964152e-11\\
2.62499999999997	-5.3973010455304e-11\\
2.625	-5.39730104552912e-11\\
2.62599999999997	-5.35239618843806e-11\\
2.626	-5.35239618843679e-11\\
2.62699999999999	-5.30786399126141e-11\\
2.62799999999998	-5.26371274309284e-11\\
2.62999999999995	-5.17653582156297e-11\\
2.63199999999997	-5.09083124579923e-11\\
2.632	-5.09083124579802e-11\\
2.63599999999995	-4.92370529446546e-11\\
2.63899999999997	-4.80197990243561e-11\\
2.639	-4.80197990243447e-11\\
2.63999999999997	-4.76207277519393e-11\\
2.64	-4.7620727751928e-11\\
2.64099999999999	-4.72249310389222e-11\\
2.64199999999998	-4.68323700916334e-11\\
2.64399999999995	-4.60568019073524e-11\\
2.64599999999997	-4.52937191345926e-11\\
2.646	-4.52937191345819e-11\\
2.64999999999995	-4.38055314512521e-11\\
2.65099999999997	-4.34413618888486e-11\\
2.651	-4.34413618888383e-11\\
2.65499999999995	-4.20151222173587e-11\\
2.6589999999999	-4.06359173117102e-11\\
2.65999999999997	-4.02982101933277e-11\\
2.66	-4.02982101933181e-11\\
2.66599999999997	-3.83290197128048e-11\\
2.666	-3.83290197127957e-11\\
2.66799999999997	-3.76939490847073e-11\\
2.668	-3.76939490846984e-11\\
2.66999999999996	-3.70696668296457e-11\\
2.67199999999993	-3.64559281848261e-11\\
2.67199999999996	-3.64559281848151e-11\\
2.672	-3.64559281848042e-11\\
2.67599999999993	-3.52591232972855e-11\\
2.67799999999997	-3.46755878441791e-11\\
2.678	-3.46755878441709e-11\\
2.67999999999997	-3.41016573998113e-11\\
2.68	-3.41016573998032e-11\\
2.68199999999998	-3.35371069555499e-11\\
2.68399999999995	-3.29817151771637e-11\\
2.68599999999997	-3.2435264321158e-11\\
2.686	-3.24352643211503e-11\\
2.68999999999995	-3.13695589191206e-11\\
2.69399999999989	-3.03395464007265e-11\\
2.69499999999997	-3.00874297711861e-11\\
2.695	-3.0087429771179e-11\\
2.69699999999997	-2.95894866108812e-11\\
2.697	-2.95894866108742e-11\\
2.69899999999996	-2.90997679484516e-11\\
2.69999999999997	-2.88579326700239e-11\\
2.7	-2.88579326700171e-11\\
2.70199999999997	-2.83801917047306e-11\\
2.70399999999993	-2.79102011008219e-11\\
2.70599999999997	-2.74477765966589e-11\\
2.706	-2.74477765966524e-11\\
2.70899999999997	-2.67685145486583e-11\\
2.709	-2.67685145486519e-11\\
2.71199999999996	-2.61064378928702e-11\\
2.71299999999997	-2.58894632759895e-11\\
2.713	-2.58894632759833e-11\\
2.71599999999997	-2.52493944966632e-11\\
2.71899999999993	-2.46251710153901e-11\\
2.71999999999997	-2.44205221265747e-11\\
2.72	-2.44205221265689e-11\\
2.72599999999993	-2.32272021525107e-11\\
2.726	-2.32272021524986e-11\\
2.72600000000002	-2.32272021524931e-11\\
2.72999999999997	-2.24640403465451e-11\\
2.73	-2.24640403465398e-11\\
2.73199999999999	-2.20921176702104e-11\\
2.73200000000002	-2.20921176702051e-11\\
2.73400000000002	-2.17264385595445e-11\\
2.73600000000001	-2.13668596486061e-11\\
2.73799999999999	-2.10132399631027e-11\\
2.73800000000002	-2.10132399630977e-11\\
2.73999999999997	-2.06654408669829e-11\\
2.74	-2.0665440866978e-11\\
2.74199999999995	-2.03233260061008e-11\\
2.7439999999999	-1.99867612529741e-11\\
2.74599999999997	-1.96556146560785e-11\\
2.746	-1.96556146560738e-11\\
2.7499999999999	-1.90098022958202e-11\\
2.7539999999998	-1.83856196517471e-11\\
2.75499999999997	-1.82328381838975e-11\\
2.755	-1.82328381838931e-11\\
2.75999999999997	-1.74877688372908e-11\\
2.76	-1.74877688372867e-11\\
2.76499999999998	-1.67727683190952e-11\\
2.76500000000001	-1.67727683190912e-11\\
2.76599999999997	-1.66332210223947e-11\\
2.766	-1.66332210223908e-11\\
2.76699999999999	-1.64948318123411e-11\\
2.76700000000002	-1.64948318123371e-11\\
2.76800000000001	-1.63576264476108e-11\\
2.76899999999999	-1.6221591480655e-11\\
2.77099999999997	-1.59529795220935e-11\\
2.77299999999999	-1.56888906262995e-11\\
2.77300000000002	-1.56888906262958e-11\\
2.77699999999997	-1.51738696137965e-11\\
2.77999999999997	-1.47987193386536e-11\\
2.78	-1.47987193386501e-11\\
2.78399999999995	-1.43127103941947e-11\\
2.784	-1.43127103941897e-11\\
2.786	-1.40755731578462e-11\\
2.78600000000003	-1.40755731578429e-11\\
2.78800000000004	-1.3842356056594e-11\\
2.79000000000004	-1.36131007671595e-11\\
2.792	-1.33877174091302e-11\\
2.79200000000003	-1.3387717409127e-11\\
2.79600000000004	-1.29482145263775e-11\\
2.79999999999997	-1.25231581096245e-11\\
2.8	-1.25231581096216e-11\\
2.80400000000001	-1.21118815179885e-11\\
2.80599999999997	-1.19112082681299e-11\\
2.806	-1.19112082681271e-11\\
2.81000000000001	-1.15198490743189e-11\\
2.81199999999997	-1.13291223388397e-11\\
2.812	-1.1329122338837e-11\\
2.81299999999997	-1.12349642626375e-11\\
2.813	-1.12349642626349e-11\\
2.81399999999998	-1.1141597383761e-11\\
2.81499999999997	-1.10490125509372e-11\\
2.81699999999995	-1.08661528020932e-11\\
2.81899999999997	-1.0686313325234e-11\\
2.819	-1.06863133252314e-11\\
2.81999999999997	-1.05975041101858e-11\\
2.82	-1.05975041101833e-11\\
2.82099999999999	-1.05094236147193e-11\\
2.82199999999998	-1.04220632057012e-11\\
2.82399999999995	-1.02494684678246e-11\\
2.82599999999997	-1.00796522301221e-11\\
2.826	-1.00796522301197e-11\\
2.82999999999995	-9.74847133810219e-12\\
2.83399999999991	-9.4283824427818e-12\\
2.83499999999997	-9.35003413920577e-12\\
2.835	-9.35003413920355e-12\\
2.83999999999997	-8.96795298622807e-12\\
2.84	-8.96795298622594e-12\\
2.84199999999997	-8.81948917961099e-12\\
2.842	-8.8194891796089e-12\\
2.84399999999996	-8.67343389307826e-12\\
2.84599999999992	-8.52972986423031e-12\\
2.846	-8.52972986422483e-12\\
2.84600000000003	-8.5297298642228e-12\\
2.847	-8.45876209530731e-12\\
2.84700000000003	-8.4587620953053e-12\\
2.84800000000002	-8.38840141777417e-12\\
2.84900000000001	-8.31864093528458e-12\\
2.85099999999998	-8.18089326457407e-12\\
2.853	-8.04546507866881e-12\\
2.85300000000003	-8.0454650786669e-12\\
2.85699999999998	-7.78135567251753e-12\\
2.85999999999997	-7.58897378128907e-12\\
2.86	-7.58897378128727e-12\\
2.86399999999995	-7.33974213704654e-12\\
2.86599999999997	-7.21813510900272e-12\\
2.866	-7.218135109001e-12\\
2.86999999999995	-6.98097331253226e-12\\
2.87	-6.98097331252944e-12\\
2.87099999999997	-6.9229382215505e-12\\
2.871	-6.92293822154886e-12\\
2.87199999999998	-6.86539382472867e-12\\
2.87299999999997	-6.80833448190192e-12\\
2.87499999999995	-6.69564863558482e-12\\
2.87699999999997	-6.5848365036775e-12\\
2.87699999999999	-6.58483650367594e-12\\
2.87999999999997	-6.42203668407538e-12\\
2.88	-6.42203668407385e-12\\
2.88299999999998	-6.26321152773433e-12\\
2.88599999999996	-6.10822092672135e-12\\
2.886	-6.1082209267192e-12\\
2.888	-6.00701427630391e-12\\
2.88800000000003	-6.00701427630248e-12\\
2.89000000000003	-5.90752689152468e-12\\
2.89200000000003	-5.80971976629227e-12\\
2.89600000000002	-5.61899355657615e-12\\
2.89999999999997	-5.43453651189328e-12\\
2.9	-5.43453651189199e-12\\
2.90499999999997	-5.21234141214886e-12\\
2.905	-5.21234141214762e-12\\
2.90599999999997	-5.16897539515079e-12\\
2.90599999999999	-5.16897539514956e-12\\
2.90699999999998	-5.12596926764053e-12\\
2.90799999999997	-5.08333103444278e-12\\
2.90999999999995	-4.99914157066787e-12\\
2.91199999999997	-4.9163739970212e-12\\
2.91199999999999	-4.91637399702003e-12\\
2.91599999999995	-4.75497526945917e-12\\
2.91799999999997	-4.67628083853704e-12\\
2.91799999999999	-4.67628083853594e-12\\
2.91999999999997	-4.5988817200333e-12\\
2.92	-4.59888172003221e-12\\
2.92199999999998	-4.52274756973222e-12\\
2.92399999999996	-4.44784853893951e-12\\
2.926	-4.37415526320013e-12\\
2.92600000000003	-4.37415526319909e-12\\
2.92899999999997	-4.26590614373211e-12\\
2.92899999999999	-4.2659061437311e-12\\
2.93199999999993	-4.16039573693212e-12\\
2.93499999999987	-4.05753096658108e-12\\
2.93499999999997	-4.05753096657774e-12\\
2.935	-4.05753096657678e-12\\
2.93999999999997	-3.89172343657807e-12\\
2.94	-3.89172343657715e-12\\
2.94499999999998	-3.73260741135184e-12\\
2.94599999999997	-3.70155259306461e-12\\
2.946	-3.70155259306373e-12\\
2.95099999999998	-3.55017183054663e-12\\
2.95199999999997	-3.52066232309052e-12\\
2.952	-3.52066232308968e-12\\
2.95699999999998	-3.37678891866988e-12\\
2.95799999999999	-3.34872932271585e-12\\
2.95800000000002	-3.34872932271505e-12\\
2.95999999999997	-3.29330307555662e-12\\
2.96	-3.29330307555584e-12\\
2.96199999999995	-3.2387826844961e-12\\
2.9639999999999	-3.18514677428297e-12\\
2.96599999999997	-3.13237431674945e-12\\
2.966	-3.13237431674871e-12\\
2.9699999999999	-3.02945583386099e-12\\
2.96999999999999	-3.02945583385871e-12\\
2.97000000000002	-3.02945583385799e-12\\
2.97399999999992	-2.92998432362861e-12\\
2.97499999999997	-2.90563663812967e-12\\
2.975	-2.90563663812898e-12\\
2.9789999999999	-2.81025506900498e-12\\
2.97999999999997	-2.78690028439128e-12\\
2.98	-2.78690028439062e-12\\
2.9839999999999	-2.69537490091183e-12\\
2.986	-2.65071712916476e-12\\
2.98600000000003	-2.65071712916413e-12\\
2.98699999999997	-2.62866303338006e-12\\
2.98699999999999	-2.62866303337944e-12\\
2.98799999999998	-2.60679759860873e-12\\
2.98899999999997	-2.58511868172988e-12\\
2.99099999999995	-2.54231192068294e-12\\
2.99299999999999	-2.50022596768247e-12\\
2.99300000000002	-2.50022596768187e-12\\
2.99699999999997	-2.41815076312507e-12\\
2.99999999999997	-2.35836575463967e-12\\
3	-2.35836575463911e-12\\
3.00399999999995	-2.28091399510675e-12\\
3.00599999999997	-2.24312313460389e-12\\
3.006	-2.24312313460336e-12\\
3.00999999999995	-2.16942222671251e-12\\
3.01	-2.16942222671163e-12\\
3.01399999999995	-2.0981897288136e-12\\
3.01599999999997	-2.06346407521942e-12\\
3.01599999999999	-2.06346407521893e-12\\
3.01999999999995	-1.99572588261004e-12\\
3.02	-1.99572588260911e-12\\
3.02399999999995	-1.93018368184909e-12\\
3.02599999999997	-1.89820382521276e-12\\
3.026	-1.89820382521231e-12\\
3.02799999999997	-1.86675263011009e-12\\
3.02799999999999	-1.86675263010965e-12\\
3.02999999999996	-1.83583571715665e-12\\
3.03199999999992	-1.8054409652544e-12\\
3.03599999999985	-1.74617047944877e-12\\
3.03999999999997	-1.68884821277428e-12\\
3.04	-1.68884821277388e-12\\
3.04499999999997	-1.61979838715869e-12\\
3.04499999999999	-1.6197983871583e-12\\
3.04599999999997	-1.60632187072683e-12\\
3.046	-1.60632187072644e-12\\
3.04699999999998	-1.59295719432291e-12\\
3.04799999999996	-1.57970684554389e-12\\
3.04999999999992	-1.5535439472495e-12\\
3.05199999999997	-1.52782291857753e-12\\
3.052	-1.52782291857716e-12\\
3.05599999999992	-1.47766630389402e-12\\
3.05799999999997	-1.45321105378284e-12\\
3.058	-1.45321105378249e-12\\
3.05999999999997	-1.42915833773473e-12\\
3.06	-1.42915833773439e-12\\
3.06199999999997	-1.40549872591561e-12\\
3.06399999999995	-1.38222294248124e-12\\
3.066	-1.35932186207036e-12\\
3.06600000000003	-1.35932186207004e-12\\
3.06999999999997	-1.31465946503438e-12\\
3.07399999999992	-1.2714929126966e-12\\
3.07399999999996	-1.2714929126962e-12\\
3.07399999999999	-1.27149291269579e-12\\
3.07999999999997	-1.20940031238581e-12\\
3.07999999999999	-1.20940031238553e-12\\
3.08599999999997	-1.15030241338282e-12\\
3.08599999999999	-1.15030241338254e-12\\
3.09199999999997	-1.09408856533709e-12\\
3.09199999999999	-1.09408856533683e-12\\
3.09799999999997	-1.04065829763865e-12\\
3.09999999999997	-1.02343391807908e-12\\
3.1	-1.02343391807884e-12\\
3.10299999999997	-9.98123091956453e-13\\
3.10299999999999	-9.98123091956216e-13\\
3.10599999999996	-9.73423351779104e-13\\
3.106	-9.73423351778743e-13\\
3.10600000000003	-9.73423351778511e-13\\
3.10899999999999	-9.49333621515988e-13\\
3.11199999999996	-9.25853363615271e-13\\
3.11199999999999	-9.25853363614986e-13\\
3.11200000000003	-9.25853363614704e-13\\
3.11499999999997	-9.0296186462205e-13\\
3.115	-9.02961864621836e-13\\
3.11799999999995	-8.80638931411589e-13\\
3.11999999999997	-8.66063100972871e-13\\
3.12	-8.66063100972666e-13\\
3.12299999999995	-8.44644256216736e-13\\
3.12599999999989	-8.23742531969026e-13\\
3.12599999999995	-8.23742531968644e-13\\
3.126	-8.23742531968259e-13\\
3.127	-8.16888953834061e-13\\
3.12700000000003	-8.16888953833867e-13\\
3.12800000000003	-8.10094004491197e-13\\
3.12900000000003	-8.03357017849089e-13\\
3.13100000000003	-7.90054297103768e-13\\
3.13199999999997	-7.83487259171601e-13\\
3.13199999999999	-7.83487259171415e-13\\
3.13599999999999	-7.57766301793502e-13\\
3.13799999999997	-7.45225334866115e-13\\
3.13799999999999	-7.45225334865939e-13\\
3.13999999999997	-7.32890792582541e-13\\
3.14	-7.32890792582367e-13\\
3.14199999999998	-7.20757839203102e-13\\
3.14399999999996	-7.08821717955511e-13\\
3.14599999999997	-6.97077749237384e-13\\
3.146	-6.97077749237219e-13\\
3.14999999999996	-6.74174297319376e-13\\
3.15	-6.74174297319146e-13\\
3.15399999999996	-6.52037933782806e-13\\
3.15599999999997	-6.41246515336628e-13\\
3.156	-6.41246515336476e-13\\
3.15999999999996	-6.20196049604061e-13\\
3.16	-6.20196049603852e-13\\
3.16099999999997	-6.15041328744142e-13\\
3.16099999999999	-6.15041328743996e-13\\
3.16199999999996	-6.0992874940831e-13\\
3.16299999999992	-6.04857810491435e-13\\
3.16499999999985	-5.94838869927586e-13\\
3.16599999999997	-5.89889886301117e-13\\
3.166	-5.89889886300977e-13\\
3.16999999999986	-5.70508239582549e-13\\
3.17199999999997	-5.61062701401517e-13\\
3.172	-5.61062701401383e-13\\
3.17599999999986	-5.42643678376649e-13\\
3.17999999999972	-5.24830088889422e-13\\
3.17999999999997	-5.24830088888312e-13\\
3.18	-5.24830088888187e-13\\
3.18499999999997	-5.03372017259894e-13\\
3.185	-5.03372017259775e-13\\
3.18599999999997	-4.99184026163413e-13\\
3.186	-4.99184026163295e-13\\
3.18699999999997	-4.95030790713011e-13\\
3.18799999999994	-4.90913083959207e-13\\
3.18999999999989	-4.82782645671317e-13\\
3.18999999999994	-4.827826456711e-13\\
3.18999999999999	-4.82782645670883e-13\\
3.19399999999988	-4.66930584486508e-13\\
3.19599999999999	-4.59202746748628e-13\\
3.19600000000002	-4.59202746748519e-13\\
3.198	-4.51602980852058e-13\\
3.19800000000003	-4.51602980851951e-13\\
3.19999999999998	-4.44128307318988e-13\\
3.20000000000001	-4.44128307318882e-13\\
3.20199999999996	-4.36775795673916e-13\\
3.20399999999991	-4.29542563355874e-13\\
3.20599999999998	-4.22425774547926e-13\\
3.20600000000001	-4.22425774547826e-13\\
3.20999999999991	-4.08546392358818e-13\\
3.21399999999982	-3.9513186206234e-13\\
3.21899999999997	-3.78985412170138e-13\\
3.21899999999999	-3.78985412170048e-13\\
3.21999999999997	-3.75835832180895e-13\\
3.22	-3.75835832180806e-13\\
3.22099999999998	-3.72712096044634e-13\\
3.22199999999996	-3.69613897458912e-13\\
3.22399999999992	-3.63492900789264e-13\\
3.22599999999997	-3.57470442411411e-13\\
3.226	-3.57470442411326e-13\\
3.22999999999992	-3.45725257362968e-13\\
3.23099999999999	-3.42851131374777e-13\\
3.23100000000002	-3.42851131374696e-13\\
3.23499999999994	-3.31594857033304e-13\\
3.23699999999999	-3.26107004384073e-13\\
3.23700000000002	-3.26107004383996e-13\\
3.23999999999997	-3.18044516951811e-13\\
3.24	-3.18044516951736e-13\\
3.24299999999996	-3.10178870483153e-13\\
3.24599999999991	-3.02503126282153e-13\\
3.24599999999997	-3.02503126281985e-13\\
3.246	-3.02503126281913e-13\\
3.24799999999997	-2.97490974870232e-13\\
3.24799999999999	-2.97490974870162e-13\\
3.24999999999996	-2.92563968256655e-13\\
3.25199999999992	-2.87720174706215e-13\\
3.25399999999997	-2.82957695190835e-13\\
3.25399999999999	-2.82957695190767e-13\\
3.25499999999998	-2.8060636349376e-13\\
3.255	-2.80606363493693e-13\\
3.25599999999998	-2.78274662591899e-13\\
3.25699999999997	-2.75962363945579e-13\\
3.25899999999993	-2.71395068781511e-13\\
3.25999999999997	-2.69139624612817e-13\\
3.26	-2.69139624612753e-13\\
3.26399999999993	-2.60300733749432e-13\\
3.26599999999997	-2.55987993880336e-13\\
3.266	-2.55987993880275e-13\\
3.267	-2.53858161279608e-13\\
3.26700000000003	-2.53858161279548e-13\\
3.26800000000003	-2.5174654826002e-13\\
3.26900000000003	-2.49652947853616e-13\\
3.27100000000003	-2.4551896584704e-13\\
3.27500000000003	-2.37458240452527e-13\\
3.27699999999997	-2.33528336607141e-13\\
3.27699999999999	-2.33528336607086e-13\\
3.27999999999997	-2.27754712143853e-13\\
3.28	-2.27754712143798e-13\\
3.28299999999998	-2.2212204770616e-13\\
3.28599999999996	-2.16625374089402e-13\\
3.286	-2.16625374089331e-13\\
3.28899999999999	-2.1126445184219e-13\\
3.28900000000002	-2.11264451842139e-13\\
3.29	-2.09507848239217e-13\\
3.29000000000003	-2.09507848239167e-13\\
3.29100000000001	-2.07766141690374e-13\\
3.29199999999999	-2.06039161488317e-13\\
3.29399999999996	-2.02628704503076e-13\\
3.296	-1.99275140200586e-13\\
3.29600000000003	-1.99275140200539e-13\\
3.29999999999996	-1.92733452403307e-13\\
3.3	-1.92733452403236e-13\\
3.30000000000003	-1.9273345240319e-13\\
3.30399999999996	-1.86403838375003e-13\\
3.30599999999997	-1.83315444232568e-13\\
3.30599999999999	-1.83315444232524e-13\\
3.30999999999992	-1.77292362223209e-13\\
3.31199999999999	-1.74357046567384e-13\\
3.31200000000002	-1.74357046567343e-13\\
3.31599999999995	-1.68633111564474e-13\\
3.31999999999988	-1.63097322388287e-13\\
3.32	-1.63097322388114e-13\\
3.32000000000003	-1.63097322388076e-13\\
3.32499999999998	-1.56428966153141e-13\\
3.325	-1.56428966153104e-13\\
3.326	-1.55127497083498e-13\\
3.32600000000003	-1.55127497083461e-13\\
3.32700000000003	-1.53836828752955e-13\\
3.32800000000003	-1.52557201396182e-13\\
3.33000000000002	-1.50030569006517e-13\\
3.332	-1.47546609338581e-13\\
3.33200000000003	-1.47546609338546e-13\\
3.33499999999999	-1.43898555416479e-13\\
3.33500000000002	-1.43898555416445e-13\\
3.33799999999998	-1.40341109693515e-13\\
3.33999999999997	-1.38018264161172e-13\\
3.34	-1.38018264161139e-13\\
3.34299999999997	-1.34604896525945e-13\\
3.34599999999993	-1.3127393867434e-13\\
3.34599999999997	-1.31273938674299e-13\\
3.346	-1.31273938674259e-13\\
3.34699999999999	-1.30181733091829e-13\\
3.34700000000002	-1.30181733091799e-13\\
3.34800000000001	-1.29098870759209e-13\\
3.349	-1.28025245526321e-13\\
3.35099999999999	-1.25905286348911e-13\\
3.35499999999995	-1.21771642623318e-13\\
3.35999999999997	-1.16795548571875e-13\\
3.36	-1.16795548571848e-13\\
3.36399999999997	-1.1295983263934e-13\\
3.36399999999999	-1.12959832639314e-13\\
3.36599999999997	-1.11088280592854e-13\\
3.366	-1.11088280592828e-13\\
3.36799999999998	-1.09247667326159e-13\\
3.36999999999996	-1.07438321763781e-13\\
3.37199999999997	-1.0565953454481e-13\\
3.372	-1.05659534544785e-13\\
3.37599999999996	-1.02190857365482e-13\\
3.37799999999997	-1.00499607496602e-13\\
3.378	-1.00499607496579e-13\\
3.37999999999997	-9.88361956447507e-14\\
3.38	-9.88361956447273e-14\\
3.38199999999997	-9.71999696716479e-14\\
3.38399999999995	-9.55902880884394e-14\\
3.38599999999997	-9.40065198132321e-14\\
3.386	-9.40065198132098e-14\\
3.38999999999995	-9.09178058185998e-14\\
3.39299999999997	-8.86694168291371e-14\\
3.39299999999999	-8.8669416829116e-14\\
3.39499999999998	-8.72018349647445e-14\\
3.395	-8.72018349647238e-14\\
3.39699999999999	-8.57586557228615e-14\\
3.39899999999997	-8.43393132993619e-14\\
3.399	-8.43393132993419e-14\\
3.39999999999997	-8.36384066406688e-14\\
3.4	-8.3638406640649e-14\\
3.40099999999998	-8.29432512367137e-14\\
3.40199999999996	-8.22537789524355e-14\\
3.40399999999991	-8.08916139914134e-14\\
3.40599999999997	-7.95513777155253e-14\\
3.406	-7.95513777155064e-14\\
3.40999999999991	-7.69376073763618e-14\\
3.41099999999997	-7.62980001472681e-14\\
3.411	-7.629800014725e-14\\
3.41499999999991	-7.3793031862531e-14\\
3.41899999999982	-7.13706728335635e-14\\
3.41999999999997	-7.07775427744152e-14\\
3.42	-7.07775427743984e-14\\
3.42199999999997	-6.96058256731416e-14\\
3.42199999999999	-6.96058256731251e-14\\
3.42399999999996	-6.84531173353533e-14\\
3.42599999999992	-6.73189657675871e-14\\
3.426	-6.73189657675401e-14\\
3.42600000000003	-6.73189657675241e-14\\
3.42999999999996	-6.51071082089807e-14\\
3.43	-6.51071082089541e-14\\
3.432	-6.40291716784043e-14\\
3.43200000000003	-6.40291716783891e-14\\
3.43400000000003	-6.29693307567402e-14\\
3.43600000000003	-6.1927169929153e-14\\
3.438	-6.09022806124953e-14\\
3.43800000000003	-6.09022806124808e-14\\
3.43999999999997	-5.98942610004869e-14\\
3.44	-5.98942610004726e-14\\
3.44199999999995	-5.89027159004588e-14\\
3.44399999999989	-5.7927256573239e-14\\
3.44599999999997	-5.69675005861849e-14\\
3.446	-5.69675005861714e-14\\
3.44999999999989	-5.50957547025057e-14\\
3.45099999999996	-5.46377258617942e-14\\
3.45099999999999	-5.46377258617812e-14\\
3.45499999999988	-5.28438941673753e-14\\
3.45699999999996	-5.19693344571355e-14\\
3.45699999999999	-5.19693344571232e-14\\
3.45999999999997	-5.0684473669263e-14\\
3.46	-5.0684473669251e-14\\
3.46299999999998	-4.9430982012412e-14\\
3.46499999999998	-4.86122010339132e-14\\
3.465	-4.86122010339017e-14\\
3.46599999999997	-4.82077537155713e-14\\
3.466	-4.82077537155599e-14\\
3.46699999999997	-4.78066628550027e-14\\
3.46799999999994	-4.74090031111321e-14\\
3.46999999999988	-4.6623821406753e-14\\
3.47199999999997	-4.58519007688688e-14\\
3.472	-4.58519007688579e-14\\
3.47599999999988	-4.43466372539689e-14\\
3.47999999999976	-4.289085177134e-14\\
3.47999999999996	-4.28908517712666e-14\\
3.47999999999999	-4.28908517712564e-14\\
3.48599999999996	-4.07949706807939e-14\\
3.48599999999999	-4.07949706807842e-14\\
3.49199999999996	-3.88013712151697e-14\\
3.49199999999999	-3.88013712151603e-14\\
3.49799999999996	-3.69064902015406e-14\\
3.5	-3.62956351327353e-14\\
3.50000000000003	-3.62956351327267e-14\\
3.506	-3.45220323361206e-14\\
3.50600000000003	-3.45220323361124e-14\\
3.50899999999999	-3.3667700600039e-14\\
3.50900000000002	-3.3667700600031e-14\\
3.51199999999998	-3.28349835459348e-14\\
3.51200000000003	-3.2834983545922e-14\\
3.51499999999999	-3.20231465793283e-14\\
3.51799999999995	-3.12314735372487e-14\\
3.51799999999999	-3.12314735372389e-14\\
3.51800000000003	-3.12314735372292e-14\\
3.51999999999997	-3.07145480897088e-14\\
3.52	-3.07145480897015e-14\\
3.52199999999995	-3.02060709927274e-14\\
3.52399999999989	-2.97058428873405e-14\\
3.52599999999997	-2.92136676571637e-14\\
3.526	-2.92136676571568e-14\\
3.52999999999989	-2.82538122635131e-14\\
3.53399999999978	-2.73261045973481e-14\\
3.53499999999998	-2.70990291726791e-14\\
3.535	-2.70990291726727e-14\\
3.53799999999997	-2.64290896315196e-14\\
3.53799999999999	-2.64290896315134e-14\\
3.53999999999997	-2.59916504973033e-14\\
3.54	-2.59916504972971e-14\\
3.54199999999998	-2.55613606242436e-14\\
3.54399999999996	-2.51380513164979e-14\\
3.54599999999997	-2.47215566140195e-14\\
3.546	-2.47215566140137e-14\\
3.54999999999996	-2.39092957322407e-14\\
3.54999999999999	-2.39092957322349e-14\\
3.553	-2.33180210381976e-14\\
3.55300000000003	-2.33180210381921e-14\\
3.55600000000004	-2.2741526381495e-14\\
3.55900000000005	-2.2179303206204e-14\\
3.55999999999997	-2.19949808416657e-14\\
3.56	-2.19949808416605e-14\\
3.56599999999999	-2.09201860526347e-14\\
3.56600000000002	-2.09201860526297e-14\\
3.56699999999996	-2.07461290814083e-14\\
3.56699999999999	-2.07461290814034e-14\\
3.56799999999996	-2.05735610769371e-14\\
3.56899999999992	-2.04024651228437e-14\\
3.56999999999998	-2.02328244498974e-14\\
3.57	-2.02328244498926e-14\\
3.57199999999993	-1.9897842582458e-14\\
3.57299999999996	-1.97324685561817e-14\\
3.57299999999999	-1.9732468556177e-14\\
3.57499999999992	-1.9405873275197e-14\\
3.57699999999985	-1.90847085465714e-14\\
3.57899999999996	-1.87688484564101e-14\\
3.57899999999999	-1.87688484564056e-14\\
3.57999999999997	-1.86128688731163e-14\\
3.58	-1.86128688731119e-14\\
3.58099999999998	-1.84581691725508e-14\\
3.58199999999997	-1.83047341919548e-14\\
3.58399999999993	-1.80015983602447e-14\\
3.58599999999997	-1.77033425324064e-14\\
3.586	-1.77033425324022e-14\\
3.58999999999993	-1.71216747696748e-14\\
3.59399999999986	-1.65594883754431e-14\\
3.59599999999996	-1.62854239850297e-14\\
3.59599999999999	-1.62854239850258e-14\\
3.59999999999997	-1.57508155831232e-14\\
3.6	-1.57508155831194e-14\\
3.60399999999998	-1.52335385755425e-14\\
3.60499999999998	-1.51068316994099e-14\\
3.605	-1.51068316994063e-14\\
3.60599999999997	-1.49811447840816e-14\\
3.606	-1.4981144784078e-14\\
3.60699999999997	-1.48565009291083e-14\\
3.60799999999994	-1.47329233353427e-14\\
3.60799999999999	-1.47329233353364e-14\\
3.60999999999993	-1.44889185871781e-14\\
3.61199999999987	-1.42490348772403e-14\\
3.61399999999996	-1.40131781581774e-14\\
3.61399999999999	-1.40131781581741e-14\\
3.61799999999987	-1.35531773668312e-14\\
3.61999999999997	-1.33288529493994e-14\\
3.62	-1.33288529493962e-14\\
3.62399999999988	-1.28911163057463e-14\\
3.62499999999996	-1.2783892822292e-14\\
3.62499999999999	-1.27838928222889e-14\\
3.62599999999997	-1.2677532462906e-14\\
3.626	-1.2677532462903e-14\\
3.62699999999998	-1.25720547751537e-14\\
3.62799999999997	-1.24674793923341e-14\\
3.62999999999993	-1.22609946310141e-14\\
3.63199999999997	-1.20579972257929e-14\\
3.632	-1.20579972257901e-14\\
3.63599999999993	-1.16621474809316e-14\\
3.63999999999985	-1.12793093228497e-14\\
3.63999999999997	-1.12793093228382e-14\\
3.64	-1.12793093228355e-14\\
3.64599999999997	-1.07281407165758e-14\\
3.646	-1.07281407165732e-14\\
3.65199999999997	-1.02038698296567e-14\\
3.652	-1.02038698296543e-14\\
3.65399999999996	-1.00349705806596e-14\\
3.65399999999999	-1.00349705806573e-14\\
3.65599999999995	-9.86888888468687e-15\\
3.65799999999992	-9.7055596259728e-15\\
3.65999999999996	-9.54491877064701e-15\\
3.65999999999999	-9.54491877064474e-15\\
3.66399999999992	-9.23145138388975e-15\\
3.66599999999996	-9.07850195632958e-15\\
3.66599999999999	-9.07850195632742e-15\\
3.66999999999992	-8.78021524041745e-15\\
3.67399999999984	-8.49191882090594e-15\\
3.67499999999998	-8.42135237599986e-15\\
3.67500000000001	-8.42135237599786e-15\\
3.67999999999997	-8.0772210113429e-15\\
3.68	-8.07722101134098e-15\\
3.68299999999996	-7.8774610363059e-15\\
3.68299999999999	-7.87746103630403e-15\\
3.68599999999995	-7.68252391691928e-15\\
3.686	-7.68252391691614e-15\\
3.68699999999998	-7.61860494339898e-15\\
3.68700000000001	-7.61860494339717e-15\\
3.68799999999998	-7.55523276241712e-15\\
3.68899999999995	-7.49240116276722e-15\\
3.6909999999999	-7.36833512723599e-15\\
3.69299999999998	-7.24635819623115e-15\\
3.693	-7.24635819622943e-15\\
3.6969999999999	-7.00848116495709e-15\\
3.69999999999997	-6.83520739242763e-15\\
3.7	-6.83520739242601e-15\\
3.7039999999999	-6.61073040574145e-15\\
3.70599999999997	-6.50120186131134e-15\\
3.706	-6.5012018613098e-15\\
3.7099999999999	-6.28759589803682e-15\\
3.70999999999998	-6.28759589803265e-15\\
3.71000000000001	-6.28759589803115e-15\\
3.71199999999996	-6.18349621494276e-15\\
3.71199999999999	-6.18349621494129e-15\\
3.71399999999995	-6.08114408110302e-15\\
3.71599999999991	-5.98049936895461e-15\\
3.71799999999996	-5.88152262034658e-15\\
3.71799999999999	-5.88152262034519e-15\\
3.71999999999997	-5.7841750316048e-15\\
3.72	-5.78417503160343e-15\\
3.72199999999998	-5.6884184377422e-15\\
3.72399999999997	-5.59421529700798e-15\\
3.72599999999997	-5.50152867669773e-15\\
3.726	-5.50152867669643e-15\\
3.728	-5.41037426623489e-15\\
3.72800000000003	-5.41037426623361e-15\\
3.73000000000004	-5.32076835573408e-15\\
3.73200000000004	-5.23267581484353e-15\\
3.73600000000004	-5.06089327452333e-15\\
3.74	-4.89475722128985e-15\\
3.74000000000003	-4.89475722128869e-15\\
3.74099999999996	-4.8540747496314e-15\\
3.74099999999999	-4.85407474963024e-15\\
3.74199999999996	-4.8137248721349e-15\\
3.74299999999992	-4.77370363223674e-15\\
3.74499999999985	-4.69463140706647e-15\\
3.745	-4.69463140706046e-15\\
3.74500000000003	-4.69463140705934e-15\\
3.746	-4.65557267174186e-15\\
3.74600000000003	-4.65557267174076e-15\\
3.747	-4.61683808009048e-15\\
3.74799999999997	-4.57843484205317e-15\\
3.74999999999991	-4.50260740325773e-15\\
3.752	-4.42806062691068e-15\\
3.75200000000003	-4.42806062690963e-15\\
3.75599999999991	-4.28269264860725e-15\\
3.758	-4.21181445455413e-15\\
3.75800000000003	-4.21181445455313e-15\\
3.75999999999997	-4.14210291743815e-15\\
3.76	-4.14210291743717e-15\\
3.76199999999995	-4.07353070694509e-15\\
3.76399999999989	-4.006070939069e-15\\
3.76599999999997	-3.93969716594415e-15\\
3.766	-3.93969716594322e-15\\
3.76999999999989	-3.81025297569902e-15\\
3.76999999999996	-3.8102529756967e-15\\
3.76999999999999	-3.8102529756958e-15\\
3.77399999999988	-3.68514416406106e-15\\
3.77599999999999	-3.62415395060065e-15\\
3.77600000000002	-3.6241539505998e-15\\
3.77999999999991	-3.50518234326653e-15\\
3.77999999999996	-3.50518234326517e-15\\
3.78	-3.50518234326379e-15\\
3.78399999999989	-3.3900676544095e-15\\
3.78599999999997	-3.33390000667705e-15\\
3.786	-3.33390000667626e-15\\
3.78999999999989	-3.22436011859564e-15\\
3.79199999999997	-3.17097646085078e-15\\
3.792	-3.17097646085003e-15\\
3.79599999999989	-3.0668770647031e-15\\
3.79899999999996	-2.99105676449011e-15\\
3.79899999999999	-2.9910567644894e-15\\
3.79999999999997	-2.96619941700361e-15\\
3.8	-2.96619941700291e-15\\
3.80099999999999	-2.94154603524286e-15\\
3.80199999999997	-2.91709420282777e-15\\
3.80399999999993	-2.86878561941597e-15\\
3.80599999999997	-2.8212547272006e-15\\
3.806	-2.82125472719993e-15\\
3.80999999999993	-2.7285585074914e-15\\
3.81099999999996	-2.70587511739019e-15\\
3.81099999999999	-2.70587511738955e-15\\
3.81499999999992	-2.61703751485092e-15\\
3.81499999999998	-2.61703751484968e-15\\
3.81500000000001	-2.61703751484906e-15\\
3.81899999999993	-2.53112961335576e-15\\
3.81999999999997	-2.51009451659137e-15\\
3.82	-2.51009451659077e-15\\
3.82399999999993	-2.42765978940598e-15\\
3.826	-2.38743760086111e-15\\
3.82600000000003	-2.38743760086055e-15\\
3.82700000000001	-2.36757400345833e-15\\
3.82700000000003	-2.36757400345776e-15\\
3.82799999999998	-2.34788032852752e-15\\
3.82800000000001	-2.34788032852696e-15\\
3.82899999999997	-2.32835464580846e-15\\
3.82999999999993	-2.30899504158203e-15\\
3.83199999999986	-2.27076649492656e-15\\
3.83399999999998	-2.23317970090865e-15\\
3.83400000000001	-2.23317970090812e-15\\
3.83799999999985	-2.15987267334099e-15\\
3.84	-2.12412369947627e-15\\
3.84000000000003	-2.12412369947576e-15\\
3.84399999999988	-2.0543647493511e-15\\
3.846	-2.02032742381392e-15\\
3.84600000000003	-2.02032742381344e-15\\
3.84999999999988	-1.95394677621798e-15\\
3.84999999999998	-1.95394677621635e-15\\
3.85000000000001	-1.95394677621589e-15\\
3.85399999999985	-1.88978936657767e-15\\
3.85599999999998	-1.85851279973314e-15\\
3.85600000000001	-1.8585127997327e-15\\
3.85699999999996	-1.84306965218393e-15\\
3.85699999999999	-1.8430696521835e-15\\
3.85799999999995	-1.82775457293141e-15\\
3.85899999999992	-1.81256606088122e-15\\
3.85999999999997	-1.79750262739686e-15\\
3.86	-1.79750262739644e-15\\
3.86199999999993	-1.76774510276659e-15\\
3.86399999999985	-1.73847033256396e-15\\
3.86599999999997	-1.70966683949997e-15\\
3.866	-1.70966683949956e-15\\
3.86899999999996	-1.66735697157706e-15\\
3.86899999999999	-1.66735697157666e-15\\
3.87199999999995	-1.62611754717516e-15\\
3.87499999999991	-1.58591218685251e-15\\
3.87999999999997	-1.52110524205442e-15\\
3.88	-1.52110524205405e-15\\
3.88499999999998	-1.45891371496958e-15\\
3.88500000000001	-1.45891371496924e-15\\
3.88599999999999	-1.44677573849589e-15\\
3.88600000000002	-1.44677573849555e-15\\
3.88700000000001	-1.4347384936649e-15\\
3.88799999999999	-1.42280422099449e-15\\
3.88999999999996	-1.39923992338075e-15\\
3.89199999999999	-1.37607360718049e-15\\
3.89200000000002	-1.37607360718016e-15\\
3.89599999999996	-1.33089874227508e-15\\
3.89799999999999	-1.30887248258588e-15\\
3.89800000000002	-1.30887248258557e-15\\
3.89999999999997	-1.28720877571083e-15\\
3.9	-1.28720877571052e-15\\
3.90199999999996	-1.26589912842366e-15\\
3.90399999999991	-1.24493518619302e-15\\
3.90599999999997	-1.22430873002368e-15\\
3.906	-1.22430873002339e-15\\
3.90999999999991	-1.18408237569279e-15\\
3.91399999999982	-1.14520329293432e-15\\
3.91499999999996	-1.13568684228049e-15\\
3.91499999999999	-1.13568684228022e-15\\
3.91999999999997	-1.08927796973181e-15\\
3.92	-1.08927796973155e-15\\
3.92499999999999	-1.04474202410593e-15\\
3.92599999999997	-1.03604990338948e-15\\
3.926	-1.03604990338924e-15\\
3.92699999999996	-1.02742991749022e-15\\
3.92699999999999	-1.02742991748998e-15\\
3.92799999999995	-1.01888367101379e-15\\
3.92899999999992	-1.01041032626442e-15\\
3.93099999999984	-9.93679027106171e-16\\
3.93299999999996	-9.77229460443666e-16\\
3.93299999999999	-9.77229460443434e-16\\
3.93699999999984	-9.45149836470864e-16\\
3.93999999999997	-9.21782479682593e-16\\
3.94	-9.21782479682374e-16\\
3.94399999999985	-8.91509959172686e-16\\
3.94399999999996	-8.9150995917184e-16\\
3.94399999999999	-8.91509959171628e-16\\
3.94599999999997	-8.76739157259053e-16\\
3.946	-8.76739157258845e-16\\
3.94799999999999	-8.62212532689804e-16\\
3.94999999999997	-8.47932681423118e-16\\
3.95199999999997	-8.3389400498823e-16\\
3.952	-8.33894004988032e-16\\
3.95499999999998	-8.13276178842771e-16\\
3.95500000000001	-8.13276178842578e-16\\
3.95799999999998	-7.93170446052971e-16\\
3.95999999999998	-7.80042343706762e-16\\
3.96	-7.80042343706577e-16\\
3.96299999999998	-7.60750901812056e-16\\
3.96599999999995	-7.41925218018914e-16\\
3.966	-7.41925218018606e-16\\
3.96600000000003	-7.4192521801843e-16\\
3.96700000000001	-7.35752363892729e-16\\
3.96700000000003	-7.35752363892554e-16\\
3.96800000000001	-7.29632315319631e-16\\
3.96899999999998	-7.23564472367265e-16\\
3.97099999999993	-7.11583029535743e-16\\
3.97299999999996	-6.99803338026933e-16\\
3.97299999999999	-6.99803338026767e-16\\
3.97699999999989	-6.7683081345564e-16\\
3.97899999999996	-6.65628973923737e-16\\
3.97899999999999	-6.65628973923579e-16\\
3.97999999999997	-6.6009722642908e-16\\
3.98	-6.60097226428923e-16\\
3.98099999999999	-6.5461086941376e-16\\
3.98199999999997	-6.49169365137196e-16\\
3.98399999999994	-6.3841878584758e-16\\
3.98599999999997	-6.27841273753923e-16\\
3.986	-6.27841273753774e-16\\
3.98999999999994	-6.07212681718417e-16\\
3.98999999999997	-6.07212681718237e-16\\
3.99000000000001	-6.07212681718057e-16\\
3.99399999999994	-5.87274987872209e-16\\
3.99599999999998	-5.77555415065065e-16\\
3.99600000000001	-5.77555415064928e-16\\
3.99999999999994	-5.58595763443891e-16\\
3.99999999999997	-5.5859576344374e-16\\
4	-5.58595763443591e-16\\
4.00199999999993	-5.49348251423728e-16\\
4.00199999999999	-5.49348251423468e-16\\
4.00399999999992	-5.40250761228686e-16\\
4.00599999999986	-5.3129972615191e-16\\
4.00599999999993	-5.31299726151587e-16\\
4.006	-5.3129972615126e-16\\
4.00999999999987	-5.13843139970736e-16\\
4.01199999999995	-5.05335769383313e-16\\
4.012	-5.05335769383073e-16\\
4.01599999999987	-4.88746195376633e-16\\
4.01999999999973	-4.72701920900456e-16\\
4.01999999999995	-4.72701920899619e-16\\
4.02	-4.72701920899395e-16\\
4.02500000000001	-4.53375148437054e-16\\
4.02500000000006	-4.53375148436839e-16\\
4.02599999999995	-4.49603124913084e-16\\
4.026	-4.49603124912871e-16\\
4.02699999999996	-4.45862404969258e-16\\
4.02799999999992	-4.42153684874015e-16\\
4.02999999999985	-4.34830793356763e-16\\
4.03099999999993	-4.31215904201346e-16\\
4.03099999999999	-4.31215904201141e-16\\
4.03499999999984	-4.17058492786135e-16\\
4.03699999999994	-4.10156227840072e-16\\
4.03699999999999	-4.10156227839877e-16\\
4.03799999999995	-4.06748014188217e-16\\
4.038	-4.06748014188024e-16\\
4.03899999999996	-4.03367966780376e-16\\
4.03999999999991	-4.00015754324191e-16\\
4.04	-4.00015754323889e-16\\
4.04199999999991	-3.93393522755086e-16\\
4.04399999999982	-3.86878723227418e-16\\
4.046	-3.80468801587922e-16\\
4.04600000000006	-3.80468801587741e-16\\
4.04999999999988	-3.6796797367067e-16\\
4.05399999999969	-3.55885826749259e-16\\
4.05999999999994	-3.38506354211919e-16\\
4.06	-3.38506354211758e-16\\
};
\addplot [color=mycolor2,solid,forget plot]
  table[row sep=crcr]{%
0	0.15314\\
3.15544362088405e-30	0.15314\\
0.000656101980281985	0.153143230512962\\
0.00393661188169191	0.153256312778436\\
0.00999999999999994	0.153891071773171\\
0.01	0.153891071773171\\
0.0199999999999999	0.150048203824684\\
0.02	0.150048203824684\\
0.0289999999999998	0.137414337712804\\
0.029	0.137414337712803\\
0.03	0.135470213386942\\
0.0300000000000002	0.135470213386942\\
0.0349999999999996	0.124115011067004\\
0.035	0.124115011067003\\
0.0399999999999993	0.110014119663841\\
0.04	0.110014119663839\\
0.0449999999999993	0.0939630779639858\\
0.0499999999999987	0.0767526455719492\\
0.05	0.0767526455719445\\
0.0500000000000004	0.0767526455719429\\
0.0579999999999996	0.0466980443355424\\
0.058	0.0466980443355407\\
0.0599999999999996	0.0386819498575326\\
0.06	0.0386819498575308\\
0.0619999999999995	0.0306155753047838\\
0.0639999999999991	0.0226526210813104\\
0.0679999999999982	0.00702452540129683\\
0.0699999999999991	-0.000646743531092999\\
0.07	-0.000646743531096385\\
0.0779999999999982	-0.0304497245417899\\
0.0799999999999991	-0.0376945518218964\\
0.08	-0.0376945518218996\\
0.087	-0.0613629561565828\\
0.0870000000000009	-0.0613629561565856\\
0.09	-0.0705722498255595\\
0.0900000000000009	-0.0705722498255621\\
0.0929999999999999	-0.0792329294508874\\
0.095999999999999	-0.0873526353855747\\
0.0999999999999991	-0.0973497076333374\\
0.1	-0.0973497076333395\\
0.104999999999999	-0.108387871323025\\
0.105	-0.108387871323027\\
0.109999999999999	-0.117699946526397\\
0.11	-0.117699946526398\\
0.114999999999999	-0.125308754745586\\
0.115999999999999	-0.126627838237935\\
0.116	-0.126627838237936\\
0.119999999999999	-0.131232943213937\\
0.12	-0.131232943213938\\
0.123999999999999	-0.13476926202664\\
0.127999999999998	-0.137242340899987\\
0.129999999999998	-0.138081442501912\\
0.13	-0.138081442501912\\
0.137999999999998	-0.138771927810782\\
0.139999999999998	-0.138276983003393\\
0.14	-0.138276983003392\\
0.144999999999998	-0.135869686909084\\
0.145	-0.135869686909083\\
0.149999999999998	-0.131785845077204\\
0.15	-0.131785845077202\\
0.154999999999998	-0.126461814037968\\
0.159999999999996	-0.120330910865585\\
0.16	-0.12033091086558\\
0.169999999999996	-0.105586372774747\\
0.17	-0.105586372774741\\
0.173999999999998	-0.0990022340863097\\
0.174	-0.0990022340863068\\
0.174999999999998	-0.0973523335757812\\
0.175	-0.0973523335757782\\
0.176	-0.0957005634817941\\
0.177	-0.0940467619077872\\
0.179000000000001	-0.0907324157456936\\
0.179999999999998	-0.0890715463112649\\
0.18	-0.0890715463112619\\
0.184000000000001	-0.0823996248083219\\
0.188000000000002	-0.0756743350020158\\
0.189999999999998	-0.0722883843683618\\
0.19	-0.0722883843683588\\
0.198000000000002	-0.058558154107107\\
0.199999999999998	-0.0550722607951608\\
0.2	-0.0550722607951577\\
0.202999999999998	-0.0499542634358041\\
0.203	-0.0499542634358012\\
0.205999999999998	-0.045090375635058\\
0.208999999999996	-0.0404763063780602\\
0.209999999999998	-0.0389930887103154\\
0.21	-0.0389930887103128\\
0.215999999999996	-0.0305025648988234\\
0.219999999999998	-0.0251966015169791\\
0.22	-0.0251966015169768\\
0.225999999999996	-0.0177484283157274\\
0.229999999999998	-0.0131121370239185\\
0.23	-0.0131121370239165\\
0.231999999999998	-0.0108899737028716\\
0.232	-0.0108899737028697\\
0.233999999999998	-0.00873062873450069\\
0.235999999999997	-0.00663325550056961\\
0.239999999999993	-0.00262115909906458\\
0.239999999999996	-0.0026211590990612\\
0.24	-0.00262115909905783\\
0.244999999999998	0.00188687770292411\\
0.245	0.00188687770292559\\
0.249999999999998	0.00568827569695383\\
0.25	0.00568827569695508\\
0.254999999999999	0.00879235129348301\\
0.259999999999997	0.0112067117758103\\
0.26	0.0112067117758117\\
0.260999999999996	0.0116103534305937\\
0.261	0.0116103534305951\\
0.262	0.0119926581405718\\
0.263	0.0123536634279823\\
0.265	0.0130119151800953\\
0.269	0.0140742429758504\\
0.269999999999997	0.0142870265310786\\
0.27	0.0142870265310794\\
0.278	0.015231929494095\\
0.279999999999996	0.0152581815612772\\
0.28	0.0152581815612772\\
0.288	0.0148381646770985\\
0.289999999999996	0.0146213487868583\\
0.29	0.0146213487868579\\
0.298	0.0133042875626112\\
0.299999999999996	0.0128619274084787\\
0.3	0.0128619274084778\\
0.308	0.01063490283314\\
0.309999999999996	0.00996265931864272\\
0.31	0.00996265931864148\\
0.314999999999997	0.00820076427699913\\
0.315	0.00820076427699786\\
0.319	0.00675620801608072\\
0.319000000000004	0.00675620801607942\\
0.319999999999996	0.00638990890522511\\
0.32	0.0063899089052238\\
0.321	0.00602147428209788\\
0.321999999999999	0.00565086803499177\\
0.323999999999998	0.00490299515937\\
0.327999999999996	0.00337957699652842\\
0.329999999999996	0.00260343440859324\\
0.33	0.00260343440859186\\
0.337999999999996	-0.000386934584574294\\
0.339999999999996	-0.00109385045098048\\
0.34	-0.00109385045098172\\
0.347999999999996	-0.00376712537690133\\
0.348	-0.0037671253769025\\
0.349999999999996	-0.00439815708383284\\
0.35	-0.00439815708383395\\
0.351999999999996	-0.00501477739830985\\
0.353999999999993	-0.00561722810735759\\
0.357999999999985	-0.00678056001263201\\
0.359999999999996	-0.00734189732970837\\
0.36	-0.00734189732970936\\
0.367999999999985	-0.00927534065693867\\
0.369999999999996	-0.00967033707065848\\
0.37	-0.00967033707065915\\
0.377	-0.0108624199510937\\
0.377000000000004	-0.0108624199510943\\
0.379999999999997	-0.011295123023204\\
0.38	-0.0112951230232045\\
0.382999999999993	-0.0116814835302473\\
0.384999999999997	-0.0119134813252346\\
0.385	-0.011913481325235\\
0.387999999999993	-0.0122233415981324\\
0.389999999999997	-0.0124046089407074\\
0.39	-0.0124046089407077\\
0.392999999999993	-0.0126387283665326\\
0.395999999999986	-0.0128276905077501\\
0.399999999999997	-0.0130096777594896\\
0.4	-0.0130096777594898\\
0.405999999999986	-0.0131344066698936\\
0.406	-0.0131344066698937\\
0.406000000000004	-0.0131344066698937\\
0.41	-0.0131194192205885\\
0.410000000000004	-0.0131194192205884\\
0.414	-0.013025910252926\\
0.417999999999996	-0.0128537331127516\\
0.419999999999997	-0.0127380644190282\\
0.42	-0.012738064419028\\
0.427999999999993	-0.0121511608627946\\
0.429999999999997	-0.0119777204409489\\
0.43	-0.0119777204409485\\
0.435	-0.0114966423981403\\
0.435000000000004	-0.01149664239814\\
0.439999999999997	-0.0109467768969577\\
0.44	-0.0109467768969573\\
0.444999999999993	-0.010355451915085\\
0.449999999999986	-0.00974989383042669\\
0.449999999999993	-0.00974989383042581\\
0.45	-0.00974989383042494\\
0.454999999999997	-0.00912861851892244\\
0.455	-0.00912861851892199\\
0.459999999999997	-0.00849010351173955\\
0.46	-0.00849010351173909\\
0.463999999999997	-0.00797875498394609\\
0.464	-0.00797875498394564\\
0.467999999999997	-0.00748042273240351\\
0.469999999999997	-0.00723589270700806\\
0.47	-0.00723589270700762\\
0.473999999999997	-0.00675562548054035\\
0.477999999999993	-0.00628645622411489\\
0.479999999999997	-0.00605580272715413\\
0.48	-0.00605580272715372\\
0.487999999999993	-0.00515729754952254\\
0.489999999999997	-0.00493825812254068\\
0.49	-0.0049382581225403\\
0.492999999999997	-0.00462164187662262\\
0.493	-0.00462164187662226\\
0.495999999999997	-0.00432559297658477\\
0.498999999999993	-0.00404985024110665\\
0.499999999999997	-0.00396240592092304\\
0.5	-0.00396240592092273\\
0.505999999999993	-0.00348408566948026\\
0.509999999999993	-0.00320878304145653\\
0.51	-0.00320878304145607\\
0.515999999999993	-0.00285520465084399\\
0.519999999999993	-0.00265634789090958\\
0.52	-0.00265634789090925\\
0.521999999999993	-0.00256785689554054\\
0.522	-0.00256785689554024\\
0.523999999999993	-0.00248661028412352\\
0.524999999999993	-0.00244869355699522\\
0.525	-0.00244869355699495\\
0.526999999999993	-0.00237825467839496\\
0.528999999999986	-0.00231498584786251\\
0.529999999999993	-0.00228603233951768\\
0.53	-0.00228603233951748\\
0.533999999999986	-0.00218245323015712\\
0.537999999999972	-0.00209610281135348\\
0.539999999999993	-0.00205934498713197\\
0.54	-0.00205934498713185\\
0.547999999999972	-0.00195477946619444\\
0.549999999999993	-0.00193920328278263\\
0.55	-0.00193920328278258\\
0.550999999999993	-0.00193299477772194\\
0.551	-0.0019329947777219\\
0.551999999999997	-0.00192783848496902\\
0.552999999999993	-0.00192373389857262\\
0.554999999999986	-0.00191867833871959\\
0.558999999999972	-0.00192117689668293\\
0.559999999999993	-0.00192442875915677\\
0.56	-0.0019244287591568\\
0.567999999999972	-0.00197140268441475\\
0.57	-0.001988401838639\\
0.570000000000007	-0.00198840183863906\\
0.577999999999979	-0.0020592072895557\\
0.579999999999993	-0.00207649525762915\\
0.58	-0.00207649525762921\\
0.587999999999972	-0.00214419694775721\\
0.589999999999993	-0.00216079300364885\\
0.59	-0.00216079300364891\\
0.594999999999993	-0.00220177812842115\\
0.595	-0.00220177812842121\\
0.599999999999993	-0.00224212195069527\\
0.6	-0.00224212195069533\\
0.604999999999993	-0.00227749864227557\\
0.608999999999993	-0.00229909690584826\\
0.609	-0.0022990969058483\\
0.609999999999993	-0.00230357020084521\\
0.61	-0.00230357020084524\\
0.610999999999997	-0.00230767391869955\\
0.611999999999993	-0.00231140846167129\\
0.613999999999986	-0.00231777145091609\\
0.617999999999972	-0.00232608101572473\\
0.619999999999993	-0.00232803084947865\\
0.62	-0.00232803084947866\\
0.627999999999972	-0.00232051978163345\\
0.629999999999993	-0.00231477188658906\\
0.63	-0.00231477188658903\\
0.637999999999972	-0.00227623216538157\\
0.637999999999993	-0.00227623216538144\\
0.638	-0.00227623216538139\\
0.639999999999993	-0.00226269131337631\\
0.64	-0.00226269131337626\\
0.641999999999993	-0.00224783108334466\\
0.643999999999986	-0.00223189833618672\\
0.647999999999971	-0.00219678988174738\\
0.649999999999993	-0.0021776004092714\\
0.65	-0.00217760040927134\\
0.657999999999971	-0.00208976029042206\\
0.659999999999993	-0.00206498665594992\\
0.66	-0.00206498665594983\\
0.664999999999993	-0.00199802433862666\\
0.665	-0.00199802433862656\\
0.666999999999993	-0.00196920055372759\\
0.667	-0.00196920055372749\\
0.668999999999993	-0.00193919501446106\\
0.669999999999993	-0.0019237454289288\\
0.67	-0.00192374542892869\\
0.671999999999993	-0.00189246372396009\\
0.673999999999986	-0.00186100751792911\\
0.677999999999972	-0.00179752219965694\\
0.679999999999993	-0.00176546819611356\\
0.68	-0.00176546819611344\\
0.687999999999972	-0.00163488227406933\\
0.689999999999993	-0.00160157931998955\\
0.69	-0.00160157931998944\\
0.695999999999993	-0.00150276358338633\\
0.696	-0.00150276358338622\\
0.699999999999993	-0.00143846754788163\\
0.7	-0.00143846754788152\\
0.703999999999993	-0.00137531848145053\\
0.707999999999986	-0.00131321734327865\\
0.709999999999993	-0.00128252924894784\\
0.71	-0.00128252924894773\\
0.717999999999986	-0.00116191364707716\\
0.719999999999993	-0.00113223482981509\\
0.72	-0.00113223482981499\\
0.724999999999993	-0.00106143079390376\\
0.725	-0.00106143079390366\\
0.729999999999993	-0.000996873375159433\\
0.730000000000001	-0.000996873375159346\\
0.734999999999994	-0.000936717601452401\\
0.735000000000001	-0.000936717601452317\\
0.739999999999994	-0.000879129290255161\\
0.740000000000001	-0.000879129290255081\\
0.744999999999994	-0.000823967307569958\\
0.749999999999987	-0.000771096465700289\\
0.750000000000001	-0.000771096465700144\\
0.753999999999993	-0.00073036215460301\\
0.754	-0.000730362154602938\\
0.757999999999993	-0.000690947265248271\\
0.759999999999993	-0.000671715205610819\\
0.76	-0.000671715205610751\\
0.763999999999993	-0.000635005393658975\\
0.767999999999986	-0.00060114792352203\\
0.77	-0.000585272049199614\\
0.770000000000007	-0.000585272049199559\\
0.777999999999993	-0.000528645554914799\\
0.779999999999993	-0.000516180430556586\\
0.78	-0.000516180430556543\\
0.782999999999993	-0.000498598482522057\\
0.783	-0.000498598482522017\\
0.785999999999993	-0.000482235829801109\\
0.788999999999986	-0.000467078036854374\\
0.79	-0.000462290844231463\\
0.790000000000007	-0.000462290844231429\\
0.795999999999993	-0.000436322274410771\\
0.8	-0.000421602903029317\\
0.800000000000007	-0.000421602903029293\\
0.804999999999993	-0.000405445605427315\\
0.805	-0.000405445605427293\\
0.809999999999987	-0.00039117882537358\\
0.809999999999997	-0.000391178825373552\\
0.810000000000007	-0.000391178825373524\\
0.811999999999993	-0.00038599347258466\\
0.812	-0.000385993472584642\\
0.813999999999987	-0.00038110292454074\\
0.815999999999973	-0.000376505263846724\\
0.819999999999945	-0.000368181508555571\\
0.819999999999987	-0.00036818150855549\\
0.82	-0.000368181508555463\\
0.827999999999944	-0.00035497694563609\\
0.829999999999993	-0.000352385372545459\\
0.830000000000001	-0.00035238537254545\\
0.837999999999945	-0.000344115707971929\\
0.839999999999993	-0.000342523970614159\\
0.84	-0.000342523970614154\\
0.840999999999993	-0.000341798980015613\\
0.841000000000001	-0.000341798980015608\\
0.841999999999997	-0.000341121145212625\\
0.842999999999993	-0.000340490399724983\\
0.844999999999986	-0.000339369934026441\\
0.848999999999972	-0.000337691662864691\\
0.849999999999993	-0.000337389069357229\\
0.85	-0.000337389069357227\\
0.857999999999972	-0.000335259240190865\\
0.859999999999993	-0.000334758414659414\\
0.86	-0.000334758414659412\\
0.867999999999972	-0.000332875747277424\\
0.869999999999993	-0.000332434316465353\\
0.87	-0.000332434316465352\\
0.874999999999994	-0.000331068242922347\\
0.875000000000001	-0.000331068242922344\\
0.879999999999994	-0.000329146222069652\\
0.880000000000001	-0.000329146222069649\\
0.884999999999994	-0.000326663543542365\\
0.889999999999987	-0.000323614122940048\\
0.890000000000001	-0.000323614122940039\\
0.890000000000008	-0.000323614122940034\\
0.899	-0.000316486733966596\\
0.899000000000008	-0.000316486733966589\\
0.9	-0.000315554720433605\\
0.900000000000007	-0.000315554720433599\\
0.901000000000004	-0.000314594375941841\\
0.902	-0.000313605606321269\\
0.903999999999993	-0.0003115424012594\\
0.907999999999979	-0.00030707043771199\\
0.909999999999993	-0.00030465992587264\\
0.910000000000001	-0.000304659925872631\\
0.917999999999972	-0.000293831753429642\\
0.919999999999993	-0.000290822872510042\\
0.920000000000001	-0.000290822872510032\\
0.927999999999972	-0.000278102617137219\\
0.927999999999994	-0.000278102617137184\\
0.928000000000001	-0.000278102617137172\\
0.929999999999994	-0.000274779995718834\\
0.930000000000001	-0.000274779995718822\\
0.931999999999994	-0.000271402704015578\\
0.933999999999987	-0.000267974341572245\\
0.937999999999972	-0.000260959007541334\\
0.939999999999993	-0.000257369285387744\\
0.940000000000001	-0.000257369285387731\\
0.944999999999994	-0.000248144342019191\\
0.945000000000001	-0.000248144342019178\\
0.949999999999994	-0.000238543151774327\\
0.950000000000001	-0.000238543151774313\\
0.954999999999994	-0.000228542184423157\\
0.956999999999994	-0.000224424403949383\\
0.957000000000001	-0.000224424403949368\\
0.959999999999993	-0.000218116930134359\\
0.960000000000001	-0.000218116930134344\\
0.962999999999993	-0.000211787464606918\\
0.965999999999986	-0.000205570352027769\\
0.969999999999993	-0.000197446174189792\\
0.970000000000001	-0.000197446174189778\\
0.975999999999986	-0.000185588360519031\\
0.979999999999993	-0.000177884039190569\\
0.980000000000001	-0.000177884039190555\\
0.985999999999986	-0.000166598677044058\\
0.985999999999993	-0.000166598677044044\\
0.986000000000001	-0.000166598677044031\\
0.989999999999993	-0.000159238640787061\\
0.990000000000001	-0.000159238640787048\\
0.993999999999993	-0.000152203250136343\\
0.997999999999986	-0.000145688781985193\\
0.999999999999993	-0.000142623669357667\\
1	-0.000142623669357657\\
1.00799999999999	-0.000131616565662225\\
1.00999999999999	-0.00012917273208024\\
1.01	-0.000129172732080223\\
1.01499999999999	-0.000123518085292801\\
1.015	-0.000123518085292785\\
1.01999999999999	-0.000118460570614788\\
1.02	-0.000118460570614774\\
1.02499999999999	-0.000113987793199313\\
1.02999999999997	-0.000110088791430314\\
1.03	-0.000110088791430294\\
1.03999999999997	-0.000103479707608394\\
1.04	-0.000103479707608377\\
1.04399999999999	-0.000101175945377768\\
1.044	-0.000101175945377761\\
1.04799999999999	-9.90608062925625e-05\\
1.04999999999999	-9.80729222172685e-05\\
1.05	-9.80729222172616e-05\\
1.05399999999999	-9.6177602974403e-05\\
1.05799999999997	-9.43490965384546e-05\\
1.05999999999999	-9.34589979262636e-05\\
1.06	-9.34589979262574e-05\\
1.06799999999997	-9.00515625303967e-05\\
1.06999999999999	-8.92362735415734e-05\\
1.07	-8.92362735415677e-05\\
1.07299999999999	-8.80394545716419e-05\\
1.073	-8.80394545716363e-05\\
1.07599999999999	-8.68730366506243e-05\\
1.07899999999997	-8.57359907502116e-05\\
1.07999999999999	-8.53633284642103e-05\\
1.08	-8.5363328464205e-05\\
1.08499999999999	-8.35461100915477e-05\\
1.085	-8.35461100915426e-05\\
1.08999999999999	-8.18025023960743e-05\\
1.09	-8.18025023960694e-05\\
1.09499999999999	-8.00618123239189e-05\\
1.09999999999997	-7.82533539751721e-05\\
1.09999999999999	-7.82533539751668e-05\\
1.1	-7.82533539751615e-05\\
1.10199999999999	-7.7510008469497e-05\\
1.102	-7.75100084694916e-05\\
1.10399999999999	-7.675482155282e-05\\
1.10599999999997	-7.59874970617088e-05\\
1.10999999999994	-7.44152271576691e-05\\
1.10999999999999	-7.44152271576516e-05\\
1.11	-7.44152271576459e-05\\
1.11799999999994	-7.11114664205761e-05\\
1.11999999999999	-7.0250474023457e-05\\
1.12	-7.02504740234508e-05\\
1.12799999999994	-6.67858740899437e-05\\
1.12999999999999	-6.59209714122933e-05\\
1.13	-6.59209714122871e-05\\
1.13099999999999	-6.54884739975316e-05\\
1.131	-6.54884739975254e-05\\
1.132	-6.50558893623983e-05\\
1.13299999999999	-6.46231750904298e-05\\
1.13499999999999	-6.37571879771102e-05\\
1.13899999999997	-6.20217906837041e-05\\
1.13999999999999	-6.15869802056827e-05\\
1.14	-6.15869802056765e-05\\
1.14799999999997	-5.80879995338669e-05\\
1.14999999999999	-5.72059883990769e-05\\
1.15	-5.72059883990706e-05\\
1.15499999999999	-5.50513794777781e-05\\
1.155	-5.50513794777721e-05\\
1.15999999999999	-5.30028342394061e-05\\
1.16	-5.30028342394004e-05\\
1.16499999999999	-5.10129318808779e-05\\
1.16999999999997	-4.90343953812653e-05\\
1.16999999999999	-4.90343953812597e-05\\
1.17	-4.9034395381254e-05\\
1.17999999999997	-4.50920404146876e-05\\
1.17999999999999	-4.50920404146819e-05\\
1.18	-4.50920404146763e-05\\
1.18899999999999	-4.15342623684143e-05\\
1.189	-4.15342623684087e-05\\
1.18999999999999	-4.11370988638729e-05\\
1.19	-4.11370988638673e-05\\
1.191	-4.07417576403654e-05\\
1.19199999999999	-4.03505355699611e-05\\
1.19399999999999	-3.9580295913081e-05\\
1.19799999999997	-3.80875858982017e-05\\
1.19999999999999	-3.73645302765287e-05\\
1.2	-3.73645302765236e-05\\
1.20799999999997	-3.4621091388731e-05\\
1.20999999999999	-3.39710830709112e-05\\
1.21	-3.39710830709066e-05\\
1.21799999999997	-3.15228970327698e-05\\
1.218	-3.15228970327618e-05\\
1.21999999999999	-3.09486176065481e-05\\
1.22	-3.0948617606544e-05\\
1.22199999999999	-3.03889859706495e-05\\
1.22399999999997	-2.98437826837512e-05\\
1.22499999999999	-2.9576524655378e-05\\
1.225	-2.95765246553742e-05\\
1.22899999999997	-2.85425094680746e-05\\
1.22999999999999	-2.82926332217404e-05\\
1.23	-2.82926332217369e-05\\
1.23399999999997	-2.7323487331463e-05\\
1.23799999999994	-2.6400420021507e-05\\
1.23999999999999	-2.59557106396736e-05\\
1.24	-2.59557106396705e-05\\
1.24699999999999	-2.44846078433585e-05\\
1.247	-2.44846078433556e-05\\
1.24999999999999	-2.38935603471624e-05\\
1.25	-2.38935603471597e-05\\
1.25299999999999	-2.33254048757642e-05\\
1.25599999999997	-2.27796402106095e-05\\
1.25999999999999	-2.20859547896303e-05\\
1.26	-2.20859547896279e-05\\
1.26599999999997	-2.11368468557044e-05\\
1.26999999999999	-2.05728278320262e-05\\
1.27	-2.05728278320243e-05\\
1.27599999999997	-1.98276669456435e-05\\
1.276	-1.98276669456403e-05\\
1.27999999999999	-1.93970018886283e-05\\
1.28	-1.93970018886269e-05\\
1.28399999999999	-1.90065384414176e-05\\
1.28799999999997	-1.86437787838281e-05\\
1.28999999999999	-1.8472608399611e-05\\
1.29	-1.84726083996098e-05\\
1.295	-1.8073924372266e-05\\
1.29500000000001	-1.80739243722649e-05\\
1.3	-1.77162451166257e-05\\
1.30000000000001	-1.77162451166248e-05\\
1.305	-1.7398694040283e-05\\
1.30500000000001	-1.73986940402822e-05\\
1.31	-1.71204929080031e-05\\
1.31000000000001	-1.71204929080023e-05\\
1.315	-1.68726622586677e-05\\
1.31999999999998	-1.66462970613828e-05\\
1.32	-1.66462970613821e-05\\
1.32999999999997	-1.6182407913925e-05\\
1.33	-1.61824079139237e-05\\
1.33399999999999	-1.59781921743013e-05\\
1.334	-1.59781921743006e-05\\
1.33799999999999	-1.5762842652407e-05\\
1.33999999999999	-1.56508878659748e-05\\
1.34	-1.5650887865974e-05\\
1.34399999999999	-1.54181987659343e-05\\
1.34799999999997	-1.51734975921986e-05\\
1.34999999999999	-1.50465232921688e-05\\
1.35	-1.50465232921679e-05\\
1.35799999999997	-1.45157703747024e-05\\
1.35999999999999	-1.43776793979001e-05\\
1.36	-1.43776793978991e-05\\
1.36299999999999	-1.41662731147417e-05\\
1.363	-1.41662731147407e-05\\
1.36499999999999	-1.40224084739941e-05\\
1.365	-1.40224084739931e-05\\
1.36699999999999	-1.3876136464917e-05\\
1.36899999999997	-1.37273997413389e-05\\
1.36999999999999	-1.36520889270863e-05\\
1.37	-1.36520889270852e-05\\
1.37399999999997	-1.33472884960282e-05\\
1.37799999999995	-1.30375964274968e-05\\
1.37999999999999	-1.28807645570106e-05\\
1.38	-1.28807645570095e-05\\
1.38799999999995	-1.22387569278706e-05\\
1.38999999999999	-1.20742703535102e-05\\
1.39	-1.2074270353509e-05\\
1.39199999999999	-1.19089782731835e-05\\
1.392	-1.19089782731824e-05\\
1.39399999999998	-1.17437322663343e-05\\
1.39599999999997	-1.15784675384878e-05\\
1.39999999999994	-1.12476227181129e-05\\
1.39999999999999	-1.12476227181091e-05\\
1.4	-1.12476227181079e-05\\
1.40799999999994	-1.05828539493579e-05\\
1.41	-1.04156403682309e-05\\
1.41000000000001	-1.04156403682297e-05\\
1.41799999999995	-9.74073318180252e-06\\
1.41999999999999	-9.57016241575112e-06\\
1.42	-9.5701624157499e-06\\
1.42099999999999	-9.48510940411088e-06\\
1.421	-9.48510940410967e-06\\
1.422	-9.40093985731432e-06\\
1.42299999999999	-9.31764552193155e-06\\
1.42499999999998	-9.15364991553894e-06\\
1.42899999999997	-8.83580768595152e-06\\
1.42999999999999	-8.75841594688794e-06\\
1.43	-8.75841594688685e-06\\
1.43499999999999	-8.38349188679117e-06\\
1.435	-8.38349188679013e-06\\
1.43999999999998	-8.02795891160401e-06\\
1.44	-8.02795891160285e-06\\
1.44499999999999	-7.6918743754733e-06\\
1.44999999999997	-7.37534330672069e-06\\
1.44999999999998	-7.37534330671979e-06\\
1.45	-7.37534330671889e-06\\
1.45999999999997	-6.79788465090023e-06\\
1.45999999999998	-6.79788465089938e-06\\
1.46	-6.79788465089855e-06\\
1.46999999999997	-6.28420796862364e-06\\
1.46999999999998	-6.2842079686229e-06\\
1.47	-6.28420796862214e-06\\
1.47899999999998	-5.86737449984043e-06\\
1.479	-5.8673744998398e-06\\
1.47999999999999	-5.82356391729689e-06\\
1.48	-5.82356391729627e-06\\
1.481	-5.78018912457834e-06\\
1.48199999999999	-5.73719690438202e-06\\
1.48399999999999	-5.65234336389853e-06\\
1.48799999999997	-5.48704421281186e-06\\
1.48999999999999	-5.40653379153875e-06\\
1.49	-5.40653379153819e-06\\
1.49799999999997	-5.09802290556219e-06\\
1.49999999999999	-5.02412669001387e-06\\
1.5	-5.02412669001335e-06\\
1.50499999999999	-4.84472233789819e-06\\
1.505	-4.84472233789769e-06\\
1.50799999999998	-4.74059844340479e-06\\
1.508	-4.7405984434043e-06\\
1.50999999999999	-4.67259159220506e-06\\
1.51	-4.67259159220458e-06\\
1.51199999999999	-4.60620380003686e-06\\
1.51399999999997	-4.54193200582875e-06\\
1.51799999999994	-4.41963643965947e-06\\
1.51999999999999	-4.36156471780063e-06\\
1.52	-4.36156471780023e-06\\
1.52799999999994	-4.1354138083054e-06\\
1.52999999999999	-4.07943410167076e-06\\
1.53	-4.07943410167036e-06\\
1.53699999999998	-3.88477286306704e-06\\
1.537	-3.88477286306665e-06\\
1.53999999999999	-3.80178680749949e-06\\
1.54	-3.8017868074991e-06\\
1.54299999999999	-3.71895902663506e-06\\
1.54599999999997	-3.63621645142227e-06\\
1.54999999999999	-3.52589940392205e-06\\
1.55	-3.52589940392166e-06\\
1.55599999999997	-3.36465081405609e-06\\
1.56	-3.26175233456832e-06\\
1.56000000000001	-3.26175233456796e-06\\
1.56599999999999	-3.1139019785524e-06\\
1.566	-3.11390197855206e-06\\
1.57	-3.01944120594047e-06\\
1.57000000000001	-3.01944120594014e-06\\
1.57400000000001	-2.92809175786984e-06\\
1.57499999999999	-2.90572364684587e-06\\
1.575	-2.90572364684556e-06\\
1.579	-2.81806243236287e-06\\
1.57999999999999	-2.79658919644889e-06\\
1.58	-2.79658919644859e-06\\
1.584	-2.71408272039318e-06\\
1.588	-2.63757133604752e-06\\
1.59	-2.60152593969683e-06\\
1.59000000000001	-2.60152593969658e-06\\
1.59499999999998	-2.51412913041771e-06\\
1.595	-2.51412913041746e-06\\
1.59999999999997	-2.42837582822453e-06\\
1.6	-2.42837582822399e-06\\
1.60000000000001	-2.42837582822375e-06\\
1.60499999999998	-2.34405587167826e-06\\
1.60999999999995	-2.26096261446063e-06\\
1.60999999999998	-2.26096261446026e-06\\
1.61	-2.26096261445989e-06\\
1.61999999999994	-2.09764414540558e-06\\
1.62	-2.0976441454046e-06\\
1.62000000000001	-2.09764414540437e-06\\
1.62399999999998	-2.03418756527293e-06\\
1.624	-2.03418756527271e-06\\
1.62799999999997	-1.9731982826703e-06\\
1.63000000000001	-1.94359883578264e-06\\
1.63000000000003	-1.94359883578244e-06\\
1.634	-1.886132304118e-06\\
1.63799999999997	-1.83090085696096e-06\\
1.63999999999999	-1.80409608563959e-06\\
1.64	-1.8040960856394e-06\\
1.64499999999999	-1.7392881786609e-06\\
1.645	-1.73928817866072e-06\\
1.64999999999998	-1.67745862571263e-06\\
1.65	-1.67745862571242e-06\\
1.653	-1.64172717467939e-06\\
1.65300000000001	-1.64172717467923e-06\\
1.65600000000001	-1.60698172692253e-06\\
1.65900000000001	-1.57319163153351e-06\\
1.66	-1.56213539098119e-06\\
1.66000000000002	-1.56213539098104e-06\\
1.66600000000001	-1.49789742039325e-06\\
1.67	-1.45699517814464e-06\\
1.67000000000002	-1.4569951781445e-06\\
1.67600000000001	-1.39968997688897e-06\\
1.67999999999998	-1.36468623029255e-06\\
1.68	-1.36468623029243e-06\\
1.68199999999998	-1.34784840090347e-06\\
1.682	-1.34784840090336e-06\\
1.68399999999998	-1.33107786948138e-06\\
1.68599999999997	-1.31436806081161e-06\\
1.68999999999994	-1.28110442856421e-06\\
1.68999999999998	-1.28110442856383e-06\\
1.69	-1.28110442856372e-06\\
1.69799999999994	-1.21501886846055e-06\\
1.69999999999998	-1.19855169779309e-06\\
1.7	-1.19855169779298e-06\\
1.70799999999994	-1.13270615264987e-06\\
1.70999999999998	-1.11621828157094e-06\\
1.71	-1.11621828157083e-06\\
1.711	-1.10800176251657e-06\\
1.71100000000001	-1.10800176251645e-06\\
1.71200000000001	-1.09985021937416e-06\\
1.71300000000001	-1.09176285283314e-06\\
1.715	-1.0757774851832e-06\\
1.71500000000001	-1.07577748518309e-06\\
1.71700000000001	-1.06003939248359e-06\\
1.71900000000001	-1.04454240459638e-06\\
1.71999999999998	-1.03688242267318e-06\\
1.72	-1.03688242267307e-06\\
1.724	-1.00681512337756e-06\\
1.72799999999999	-9.77628863119831e-07\\
1.72999999999998	-9.63351765227398e-07\\
1.73	-9.63351765227298e-07\\
1.73799999999999	-9.08220568807584e-07\\
1.74	-8.94905050029216e-07\\
1.74000000000001	-8.94905050029122e-07\\
1.74800000000001	-8.45423432646866e-07\\
1.74999999999998	-8.34103079362706e-07\\
1.75	-8.34103079362627e-07\\
1.75799999999999	-7.90179089452211e-07\\
1.75999999999998	-7.79346654904122e-07\\
1.76	-7.79346654904045e-07\\
1.76799999999999	-7.36483722537377e-07\\
1.76899999999998	-7.31169702581359e-07\\
1.769	-7.31169702581284e-07\\
1.76999999999998	-7.25863685036019e-07\\
1.77	-7.25863685035943e-07\\
1.771	-7.20565150087193e-07\\
1.77199999999999	-7.15273578191753e-07\\
1.77399999999998	-7.04709249676107e-07\\
1.77799999999997	-6.83641369082132e-07\\
1.77999999999998	-6.73129556706323e-07\\
1.78	-6.73129556706249e-07\\
1.78499999999998	-6.47467912446302e-07\\
1.785	-6.47467912446231e-07\\
1.78999999999998	-6.22974034316735e-07\\
1.79	-6.22974034316664e-07\\
1.79499999999998	-5.99587893434757e-07\\
1.798	-5.86063971681208e-07\\
1.79800000000001	-5.86063971681145e-07\\
1.8	-5.77252176305366e-07\\
1.80000000000001	-5.77252176305304e-07\\
1.802	-5.68601568826153e-07\\
1.80399999999999	-5.60110663255775e-07\\
1.80799999999996	-5.43594703668865e-07\\
1.81	-5.35563174055897e-07\\
1.81000000000001	-5.35563174055841e-07\\
1.81799999999996	-5.04873917348146e-07\\
1.81999999999998	-4.97545778528657e-07\\
1.82	-4.97545778528606e-07\\
1.82699999999998	-4.72812389832366e-07\\
1.827	-4.72812389832317e-07\\
1.82999999999998	-4.6261313542562e-07\\
1.83	-4.62613135425572e-07\\
1.83299999999999	-4.52641246370177e-07\\
1.83599999999997	-4.42887925632578e-07\\
1.84	-4.3020864893363e-07\\
1.84000000000001	-4.30208648933585e-07\\
1.84599999999999	-4.11822465360447e-07\\
1.85	-3.9994820017622e-07\\
1.85000000000001	-3.99948200176179e-07\\
1.85499999999998	-3.85503668253401e-07\\
1.855	-3.85503668253361e-07\\
1.85599999999998	-3.8266502424376e-07\\
1.856	-3.8266502424372e-07\\
1.85699999999999	-3.79842481433457e-07\\
1.85799999999999	-3.77035763159663e-07\\
1.85999999999998	-3.71468701395167e-07\\
1.86	-3.71468701395117e-07\\
1.86399999999999	-3.60512469061441e-07\\
1.86799999999997	-3.49779143941944e-07\\
1.86999999999999	-3.44490797809161e-07\\
1.87	-3.44490797809124e-07\\
1.87799999999997	-3.24598744015289e-07\\
1.87999999999999	-3.19980501534169e-07\\
1.88	-3.19980501534137e-07\\
1.88499999999998	-3.09035713196465e-07\\
1.885	-3.09035713196435e-07\\
1.88999999999998	-2.98928028982588e-07\\
1.89	-2.98928028982553e-07\\
1.89499999999998	-2.89519334852739e-07\\
1.89999999999996	-2.80673230156768e-07\\
1.89999999999998	-2.80673230156738e-07\\
1.9	-2.80673230156709e-07\\
1.90999999999996	-2.64583396732426e-07\\
1.91	-2.64583396732374e-07\\
1.91000000000001	-2.64583396732352e-07\\
1.91399999999998	-2.58640594949897e-07\\
1.914	-2.58640594949876e-07\\
1.91799999999997	-2.5285533944778e-07\\
1.91999999999999	-2.50018946547335e-07\\
1.92	-2.50018946547315e-07\\
1.92399999999997	-2.44453069985938e-07\\
1.92499999999998	-2.43083076799473e-07\\
1.925	-2.43083076799454e-07\\
1.92899999999997	-2.37684979347253e-07\\
1.92999999999999	-2.36355262466908e-07\\
1.93	-2.36355262466889e-07\\
1.93399999999997	-2.31111714788622e-07\\
1.93799999999994	-2.25982537400944e-07\\
1.93999999999999	-2.23458316883346e-07\\
1.94	-2.23458316883328e-07\\
1.94299999999998	-2.19750345120494e-07\\
1.943	-2.19750345120476e-07\\
1.94599999999998	-2.16158052084892e-07\\
1.94899999999996	-2.12678268605521e-07\\
1.95	-2.11542809512277e-07\\
1.95000000000002	-2.11542809512261e-07\\
1.95599999999998	-2.04711314393199e-07\\
1.95999999999998	-2.0008841789635e-07\\
1.96	-2.00088417896333e-07\\
1.96599999999996	-1.93033095225397e-07\\
1.97	-1.88238136792647e-07\\
1.97000000000001	-1.8823813679263e-07\\
1.97199999999998	-1.85810425800462e-07\\
1.972	-1.85810425800445e-07\\
1.97399999999997	-1.83361293977331e-07\\
1.97599999999994	-1.8088978109603e-07\\
1.97999999999988	-1.75875727165959e-07\\
1.98	-1.75875727165809e-07\\
1.98000000000002	-1.75875727165791e-07\\
1.9879999999999	-1.65941906997073e-07\\
1.99	-1.63514690440074e-07\\
1.99000000000002	-1.63514690440057e-07\\
1.99499999999998	-1.57534622188661e-07\\
1.995	-1.57534622188644e-07\\
1.99999999999997	-1.51668541679637e-07\\
1.99999999999998	-1.51668541679617e-07\\
2	-1.51668541679597e-07\\
2.00099999999997	-1.50509532132948e-07\\
2.001	-1.50509532132915e-07\\
2.00199999999999	-1.49357988038231e-07\\
2.00299999999999	-1.48213796522265e-07\\
2.00499999999998	-1.45947023365456e-07\\
2.00899999999997	-1.4149682065517e-07\\
2.00999999999997	-1.4040099612552e-07\\
2.01	-1.40400996125489e-07\\
2.01799999999997	-1.31858598355349e-07\\
2.01999999999997	-1.29781444248783e-07\\
2.02	-1.29781444248754e-07\\
2.02799999999997	-1.2168217103391e-07\\
2.02999999999997	-1.19705719280349e-07\\
2.03	-1.19705719280321e-07\\
2.03799999999997	-1.12298491979534e-07\\
2.03999999999997	-1.10588212849529e-07\\
2.04	-1.10588212849505e-07\\
2.04799999999997	-1.04293085417701e-07\\
2.04999999999997	-1.02852715724958e-07\\
2.05	-1.02852715724938e-07\\
2.05799999999997	-9.75006450775584e-08\\
2.05899999999997	-9.68735927366414e-08\\
2.059	-9.68735927366237e-08\\
2.05999999999997	-9.62556539200199e-08\\
2.06	-9.62556539200025e-08\\
2.06099999999999	-9.56467680918408e-08\\
2.06199999999999	-9.5046875545577e-08\\
2.06399999999998	-9.38738359915828e-08\\
2.06499999999997	-9.33005740116015e-08\\
2.065	-9.33005740115854e-08\\
2.06899999999998	-9.1094610399996e-08\\
2.06999999999997	-9.05646203133432e-08\\
2.07	-9.05646203133283e-08\\
2.07399999999998	-8.8497237322697e-08\\
2.07799999999997	-8.64988165408296e-08\\
2.07999999999997	-8.55244833791364e-08\\
2.08	-8.55244833791227e-08\\
2.08799999999997	-8.17841586434837e-08\\
2.088	-8.178415864347e-08\\
2.08999999999997	-8.08864963937365e-08\\
2.09	-8.08864963937238e-08\\
2.09199999999997	-8.00030892620572e-08\\
2.09399999999994	-7.91335908519449e-08\\
2.09799999999988	-7.74349619640981e-08\\
2.09999999999997	-7.66051654862678e-08\\
2.1	-7.66051654862561e-08\\
2.10799999999988	-7.34621867475569e-08\\
2.10999999999997	-7.27223032497461e-08\\
2.11	-7.27223032497357e-08\\
2.11699999999997	-7.01359118043542e-08\\
2.117	-7.01359118043436e-08\\
2.11999999999997	-6.90084786995951e-08\\
2.12	-6.90084786995844e-08\\
2.12299999999997	-6.78682262649285e-08\\
2.12599999999994	-6.67141485939064e-08\\
2.12999999999997	-6.5152110935404e-08\\
2.13	-6.51521109353928e-08\\
2.13499999999997	-6.31587170592184e-08\\
2.135	-6.3158717059207e-08\\
2.13999999999997	-6.11153729487431e-08\\
2.14	-6.11153729487314e-08\\
2.14499999999998	-5.90655764122158e-08\\
2.14599999999997	-5.8660219728273e-08\\
2.146	-5.86602197282615e-08\\
2.14999999999997	-5.70528094421023e-08\\
2.15	-5.7052809442091e-08\\
2.15399999999998	-5.54659396014975e-08\\
2.15799999999995	-5.38971214237435e-08\\
2.15999999999997	-5.31187117659023e-08\\
2.16	-5.31187117658912e-08\\
2.16799999999995	-5.00767085310798e-08\\
2.16999999999997	-4.93350460865037e-08\\
2.17	-4.93350460864932e-08\\
2.17499999999997	-4.75106647076705e-08\\
2.175	-4.75106647076602e-08\\
2.17999999999997	-4.57252541013196e-08\\
2.18	-4.57252541013083e-08\\
2.18499999999997	-4.39838928097029e-08\\
2.18999999999994	-4.22917673781084e-08\\
2.18999999999997	-4.22917673780976e-08\\
2.19	-4.22917673780881e-08\\
2.19999999999994	-3.90387467611347e-08\\
2.19999999999997	-3.90387467611244e-08\\
2.2	-3.90387467611154e-08\\
2.20399999999997	-3.77808690538949e-08\\
2.204	-3.77808690538861e-08\\
2.20499999999997	-3.74698790930857e-08\\
2.205	-3.74698790930769e-08\\
2.206	-3.71602197407196e-08\\
2.20699999999999	-3.68518606458321e-08\\
2.20899999999998	-3.62389224633498e-08\\
2.20999999999997	-3.59342833003281e-08\\
2.21	-3.59342833003195e-08\\
2.21399999999998	-3.47532303962502e-08\\
2.21799999999997	-3.36412082875508e-08\\
2.21999999999997	-3.31105345788058e-08\\
2.22	-3.31105345787984e-08\\
2.22799999999997	-3.11519764607153e-08\\
2.22999999999997	-3.07024112634375e-08\\
2.23	-3.07024112634312e-08\\
2.23299999999997	-3.0053460407616e-08\\
2.233	-3.005346040761e-08\\
2.23599999999997	-2.94313374040768e-08\\
2.23899999999994	-2.88354934054765e-08\\
2.23999999999997	-2.86426270073689e-08\\
2.24	-2.86426270073634e-08\\
2.24599999999994	-2.75444767703541e-08\\
2.24999999999997	-2.68673123794119e-08\\
2.25	-2.68673123794073e-08\\
2.25599999999994	-2.59107138025959e-08\\
2.25999999999997	-2.53018006409581e-08\\
2.26	-2.53018006409538e-08\\
2.26199999999997	-2.50056283860713e-08\\
2.262	-2.50056283860671e-08\\
2.26399999999997	-2.47148234869504e-08\\
2.26599999999994	-2.44292719274784e-08\\
2.26999999999988	-2.38734830398353e-08\\
2.26999999999997	-2.38734830398229e-08\\
2.27	-2.3873483039819e-08\\
2.27499999999997	-2.31986194585733e-08\\
2.275	-2.31986194585695e-08\\
2.27999999999997	-2.25374278449599e-08\\
2.28	-2.25374278449562e-08\\
2.28499999999997	-2.18882877717095e-08\\
2.28999999999995	-2.12496083650575e-08\\
2.29	-2.12496083650507e-08\\
2.29099999999997	-2.11229896032781e-08\\
2.291	-2.11229896032745e-08\\
2.29199999999999	-2.0996714190757e-08\\
2.29299999999999	-2.087076973941e-08\\
2.29499999999998	-2.06198243759904e-08\\
2.29899999999997	-2.01213640778561e-08\\
2.29999999999997	-1.99973923854767e-08\\
2.3	-1.99973923854732e-08\\
2.30799999999997	-1.9043034047624e-08\\
2.30999999999997	-1.8815452475335e-08\\
2.31	-1.88154524753318e-08\\
2.31799999999997	-1.794648069634e-08\\
2.31999999999997	-1.77391505165955e-08\\
2.32	-1.77391505165925e-08\\
2.32799999999997	-1.69383337235426e-08\\
2.32999999999997	-1.6744319071297e-08\\
2.33	-1.67443190712943e-08\\
2.33799999999997	-1.5990736025936e-08\\
2.33999999999997	-1.58075898429967e-08\\
2.34	-1.58075898429941e-08\\
2.34499999999997	-1.53557189594839e-08\\
2.345	-1.53557189594813e-08\\
2.34899999999997	-1.49988156417685e-08\\
2.349	-1.4998815641766e-08\\
2.34999999999997	-1.49101547050569e-08\\
2.35	-1.49101547050544e-08\\
2.35099999999999	-1.4821702314004e-08\\
2.35199999999999	-1.47334497990372e-08\\
2.35399999999998	-1.4557509817439e-08\\
2.35799999999996	-1.42076489904189e-08\\
2.35999999999997	-1.40335909714773e-08\\
2.36	-1.40335909714748e-08\\
2.36799999999996	-1.33434189214e-08\\
2.36999999999997	-1.31721672622281e-08\\
2.37	-1.31721672622257e-08\\
2.37799999999996	-1.2490311225873e-08\\
2.378	-1.24903112258702e-08\\
2.37999999999997	-1.23203007069544e-08\\
2.38	-1.23203007069519e-08\\
2.38199999999997	-1.21503382091269e-08\\
2.38399999999994	-1.19803570885372e-08\\
2.38799999999988	-1.16400723804213e-08\\
2.38999999999997	-1.14696353740644e-08\\
2.39	-1.14696353740619e-08\\
2.39799999999988	-1.08132943012622e-08\\
2.39999999999997	-1.06570464716145e-08\\
2.4	-1.06570464716123e-08\\
2.40699999999997	-1.01334965078332e-08\\
2.407	-1.01334965078311e-08\\
2.40999999999997	-9.91978269868879e-09\\
2.41	-9.9197826986868e-09\\
2.41299999999997	-9.71205926303422e-09\\
2.41499999999997	-9.57674009994162e-09\\
2.415	-9.57674009993971e-09\\
2.41799999999997	-9.37837341183913e-09\\
2.41999999999997	-9.24912804760087e-09\\
2.42	-9.24912804759905e-09\\
2.42299999999997	-9.05963196180421e-09\\
2.42599999999994	-8.87523397927092e-09\\
2.42999999999997	-8.63701986088292e-09\\
2.42999999999999	-8.63701986088126e-09\\
2.43599999999994	-8.29333449110274e-09\\
2.43599999999997	-8.29333449110102e-09\\
2.436	-8.29333449109933e-09\\
2.43999999999997	-8.07188734077937e-09\\
2.43999999999999	-8.07188734077782e-09\\
2.44399999999996	-7.85617476063091e-09\\
2.44799999999993	-7.64585843445041e-09\\
2.44999999999999	-7.54262046545015e-09\\
2.45000000000002	-7.5426204654487e-09\\
2.45799999999996	-7.14153100452718e-09\\
2.45999999999997	-7.04402767899343e-09\\
2.46	-7.04402767899205e-09\\
2.46499999999997	-6.80958804106677e-09\\
2.465	-6.80958804106549e-09\\
2.46999999999996	-6.59080123893297e-09\\
2.47	-6.59080123893132e-09\\
2.47499999999997	-6.38017963605352e-09\\
2.47999999999993	-6.17025561394541e-09\\
2.48	-6.1702556139426e-09\\
2.48000000000003	-6.17025561394141e-09\\
2.48499999999997	-5.96051468229027e-09\\
2.485	-5.96051468228908e-09\\
2.48999999999994	-5.75044286275796e-09\\
2.48999999999999	-5.75044286275615e-09\\
2.49000000000003	-5.75044286275433e-09\\
2.49399999999997	-5.58180124510572e-09\\
2.494	-5.58180124510452e-09\\
2.49799999999993	-5.41235390295718e-09\\
2.49999999999997	-5.32724509746643e-09\\
2.5	-5.32724509746521e-09\\
2.50399999999994	-5.1595007673463e-09\\
2.50799999999988	-4.99710918787815e-09\\
2.50999999999997	-4.91784076780849e-09\\
2.51	-4.91784076780737e-09\\
2.51799999999988	-4.61289670873426e-09\\
2.51999999999997	-4.53954365351169e-09\\
2.52	-4.53954365351066e-09\\
2.52299999999997	-4.43181696154967e-09\\
2.523	-4.43181696154866e-09\\
2.52599999999996	-4.32697636218877e-09\\
2.52899999999993	-4.224929363717e-09\\
2.52999999999997	-4.19151886857787e-09\\
2.53	-4.19151886857693e-09\\
2.53599999999993	-3.99718248669507e-09\\
2.53999999999997	-3.8732284782851e-09\\
2.54	-3.87322847828423e-09\\
2.54599999999993	-3.69510654296947e-09\\
2.54999999999997	-3.58124405766897e-09\\
2.55	-3.58124405766818e-09\\
2.55199999999997	-3.52571092877424e-09\\
2.552	-3.52571092877346e-09\\
2.55399999999996	-3.47108072725826e-09\\
2.55499999999997	-3.4440975171927e-09\\
2.555	-3.44409751719194e-09\\
2.55699999999997	-3.39078165503003e-09\\
2.55899999999994	-3.33831581999899e-09\\
2.55999999999997	-3.31239522185945e-09\\
2.56	-3.31239522185872e-09\\
2.56399999999994	-3.21073591335066e-09\\
2.56799999999987	-3.11219467034416e-09\\
2.56999999999997	-3.06404483987443e-09\\
2.57	-3.06404483987375e-09\\
2.57799999999987	-2.88524407180878e-09\\
2.57999999999997	-2.84432842347087e-09\\
2.58	-2.8443284234703e-09\\
2.58099999999997	-2.82442689914827e-09\\
2.581	-2.82442689914771e-09\\
2.58199999999999	-2.80489362923989e-09\\
2.58299999999999	-2.78572669831305e-09\\
2.58499999999998	-2.74848437488904e-09\\
2.58899999999996	-2.67831552252334e-09\\
2.58999999999997	-2.66166235193637e-09\\
2.59	-2.6616623519359e-09\\
2.59799999999997	-2.53710187026083e-09\\
2.59999999999997	-2.50817797495435e-09\\
2.6	-2.50817797495394e-09\\
2.60799999999997	-2.40100700001944e-09\\
2.60999999999997	-2.37629288570651e-09\\
2.61	-2.37629288570616e-09\\
2.61799999999996	-2.28546018087545e-09\\
2.62	-2.26471342760547e-09\\
2.62000000000002	-2.26471342760518e-09\\
2.62499999999997	-2.21602389595769e-09\\
2.625	-2.21602389595742e-09\\
2.62999999999995	-2.17168505826279e-09\\
2.62999999999999	-2.17168505826248e-09\\
2.63000000000002	-2.17168505826217e-09\\
2.63499999999997	-2.12898815044444e-09\\
2.63899999999997	-2.09407051001531e-09\\
2.639	-2.09407051001506e-09\\
2.63999999999997	-2.08522843386654e-09\\
2.64	-2.08522843386628e-09\\
2.641	-2.0763395648682e-09\\
2.64199999999999	-2.06740303174738e-09\\
2.64399999999999	-2.04938346488736e-09\\
2.64799999999997	-2.01273349967945e-09\\
2.64999999999997	-1.99408873163685e-09\\
2.65	-1.99408873163658e-09\\
2.65799999999997	-1.91720616116652e-09\\
2.65999999999997	-1.89737196597236e-09\\
2.66	-1.89737196597208e-09\\
2.66799999999997	-1.8173013964392e-09\\
2.668	-1.81730139643891e-09\\
2.66999999999997	-1.79718311763324e-09\\
2.67	-1.79718311763295e-09\\
2.67199999999998	-1.77700883752361e-09\\
2.67399999999995	-1.75677064560987e-09\\
2.6779999999999	-1.7160707605713e-09\\
2.67999999999997	-1.69559310984288e-09\\
2.68	-1.69559310984259e-09\\
2.6879999999999	-1.61365437524818e-09\\
2.68999999999997	-1.59318566162748e-09\\
2.69	-1.59318566162719e-09\\
2.69499999999997	-1.54195404365513e-09\\
2.695	-1.54195404365484e-09\\
2.69699999999997	-1.52141637299945e-09\\
2.697	-1.52141637299916e-09\\
2.69899999999996	-1.50084090326274e-09\\
2.69999999999997	-1.49053647351744e-09\\
2.7	-1.49053647351715e-09\\
2.70199999999997	-1.469962942423e-09\\
2.70399999999993	-1.44947897560284e-09\\
2.70799999999986	-1.40874764369428e-09\\
2.70999999999997	-1.3884843086538e-09\\
2.71	-1.38848430865351e-09\\
2.71799999999986	-1.30792805561576e-09\\
2.71999999999997	-1.28787378560283e-09\\
2.72	-1.28787378560255e-09\\
2.72599999999997	-1.22778861772034e-09\\
2.726	-1.22778861772005e-09\\
2.72999999999997	-1.18771801916427e-09\\
2.73	-1.18771801916399e-09\\
2.73399999999998	-1.14851586634797e-09\\
2.73799999999995	-1.11107349048925e-09\\
2.73999999999997	-1.09299376156537e-09\\
2.74	-1.09299376156512e-09\\
2.74799999999995	-1.02478774765229e-09\\
2.74999999999997	-1.00873104341549e-09\\
2.75	-1.00873104341526e-09\\
2.75499999999997	-9.70215503026923e-10\\
2.755	-9.7021550302671e-10\\
2.75999999999996	-9.33922475639151e-10\\
2.76	-9.33922475638894e-10\\
2.76499999999997	-8.99763015306507e-10\\
2.765	-8.99763015306281e-10\\
2.76500000000003	-8.99763015306092e-10\\
2.77	-8.67653406061231e-10\\
2.77000000000003	-8.67653406061054e-10\\
2.77499999999999	-8.37269809356801e-10\\
2.77999999999996	-8.08292616656451e-10\\
2.78	-8.08292616656215e-10\\
2.78399999999997	-7.86075636475899e-10\\
2.784	-7.86075636475744e-10\\
2.78799999999996	-7.64678401112527e-10\\
2.78999999999997	-7.54276650686385e-10\\
2.79	-7.54276650686239e-10\\
2.79399999999997	-7.33950032010138e-10\\
2.79799999999993	-7.14168481770912e-10\\
2.79999999999997	-7.0447237949933e-10\\
2.8	-7.04472379499193e-10\\
2.80799999999993	-6.66897653688194e-10\\
2.80999999999997	-6.57787978600437e-10\\
2.81	-6.57787978600308e-10\\
2.81299999999997	-6.44321850768275e-10\\
2.813	-6.44321850768148e-10\\
2.81599999999996	-6.31083241571144e-10\\
2.81899999999993	-6.18060471774781e-10\\
2.81999999999997	-6.13765521871101e-10\\
2.82	-6.13765521870979e-10\\
2.82599999999993	-5.89079510097158e-10\\
2.82999999999997	-5.73725002916078e-10\\
2.83	-5.73725002915972e-10\\
2.83499999999997	-5.55730073334501e-10\\
2.835	-5.55730073334402e-10\\
2.83999999999997	-5.39025472383856e-10\\
2.84	-5.39025472383765e-10\\
2.84199999999997	-5.32662449737619e-10\\
2.842	-5.3266244973753e-10\\
2.84399999999996	-5.26430400524361e-10\\
2.84599999999993	-5.20326881407357e-10\\
2.84999999999985	-5.08495911346259e-10\\
2.85	-5.08495911345832e-10\\
2.85000000000003	-5.0849591134575e-10\\
2.85799999999989	-4.86274307529679e-10\\
2.85999999999997	-4.81006203849977e-10\\
2.86	-4.81006203849903e-10\\
2.86799999999986	-4.60544455576838e-10\\
2.86999999999997	-4.55541887995795e-10\\
2.87	-4.55541887995724e-10\\
2.87099999999997	-4.53056176012854e-10\\
2.871	-4.53056176012784e-10\\
2.87199999999999	-4.50580520147854e-10\\
2.87299999999999	-4.4811467765042e-10\\
2.87499999999998	-4.43211466983318e-10\\
2.87899999999996	-4.33512233753437e-10\\
2.87999999999997	-4.31108368620924e-10\\
2.88	-4.31108368620856e-10\\
2.88799999999997	-4.12142390804831e-10\\
2.88999999999997	-4.07465977989865e-10\\
2.89	-4.07465977989799e-10\\
2.89799999999997	-3.89446722652551e-10\\
2.89999999999997	-3.85134730299006e-10\\
2.9	-3.85134730298945e-10\\
2.90499999999997	-3.74372072640642e-10\\
2.905	-3.7437207264058e-10\\
2.90999999999998	-3.63440760263061e-10\\
2.91000000000001	-3.63440760262998e-10\\
2.91499999999999	-3.52314003096606e-10\\
2.91999999999997	-3.4096453234862e-10\\
2.92	-3.40964532348542e-10\\
2.92899999999997	-3.19885072259964e-10\\
2.92899999999999	-3.19885072259896e-10\\
2.92999999999997	-3.17485577454242e-10\\
2.93	-3.17485577454173e-10\\
2.93099999999999	-3.15086616052151e-10\\
2.93199999999999	-3.1270080638304e-10\\
2.93399999999997	-3.07967708172994e-10\\
2.93799999999994	-2.98649127399725e-10\\
2.94	-2.94059991201173e-10\\
2.94000000000003	-2.94059991201108e-10\\
2.94799999999997	-2.76129359545762e-10\\
2.95	-2.71744392561114e-10\\
2.95000000000003	-2.71744392561052e-10\\
2.95799999999997	-2.54771485580687e-10\\
2.958	-2.54771485580628e-10\\
2.96	-2.50675910802522e-10\\
2.96000000000003	-2.50675910802464e-10\\
2.96200000000003	-2.46636151661437e-10\\
2.96400000000003	-2.42650624113409e-10\\
2.96800000000003	-2.34836034321906e-10\\
2.97	-2.31003908127785e-10\\
2.97000000000003	-2.31003908127731e-10\\
2.97499999999997	-2.2163080318579e-10\\
2.975	-2.21630803185738e-10\\
2.97999999999995	-2.12535422483586e-10\\
2.98	-2.12535422483485e-10\\
2.98499999999995	-2.03927993474582e-10\\
2.98699999999997	-2.00681970088336e-10\\
2.98699999999999	-2.00681970088291e-10\\
2.98999999999997	-1.96019939664457e-10\\
2.99	-1.96019939664414e-10\\
2.99299999999998	-1.91602675355719e-10\\
2.99599999999996	-1.87426280443344e-10\\
2.99999999999997	-1.8222610200507e-10\\
3	-1.82226102005035e-10\\
3.00599999999996	-1.75068142506096e-10\\
3.00999999999997	-1.70655384825394e-10\\
3.01	-1.70655384825364e-10\\
3.01599999999996	-1.64557767045681e-10\\
3.01599999999999	-1.64557767045647e-10\\
3.01600000000002	-1.6455776704562e-10\\
3.01999999999997	-1.60831069957029e-10\\
3.02	-1.60831069957003e-10\\
3.02399999999995	-1.57301711999836e-10\\
3.0279999999999	-1.5389767224842e-10\\
3.02999999999997	-1.52240970790692e-10\\
3.03	-1.52240970790669e-10\\
3.0379999999999	-1.45901265434991e-10\\
3.03999999999997	-1.4438500727574e-10\\
3.04	-1.44385007275719e-10\\
3.04499999999997	-1.40707985246066e-10\\
3.04499999999999	-1.40707985246045e-10\\
3.04999999999996	-1.3718612049681e-10\\
3.05	-1.37186120496782e-10\\
3.05499999999997	-1.33834606884127e-10\\
3.05999999999993	-1.30669055856601e-10\\
3.05999999999997	-1.3066905585658e-10\\
3.06	-1.30669055856558e-10\\
3.06999999999993	-1.24381496211569e-10\\
3.06999999999997	-1.24381496211547e-10\\
3.07	-1.24381496211524e-10\\
3.07399999999997	-1.21781565723582e-10\\
3.07399999999999	-1.21781565723563e-10\\
3.07799999999996	-1.19127045421511e-10\\
3.07999999999997	-1.17778016481e-10\\
3.08	-1.17778016480981e-10\\
3.08399999999997	-1.15033776384914e-10\\
3.08799999999993	-1.12224363213874e-10\\
3.08999999999997	-1.10793843899565e-10\\
3.09	-1.10793843899544e-10\\
3.09799999999993	-1.05097859121997e-10\\
3.09999999999997	-1.03690797307052e-10\\
3.1	-1.03690797307032e-10\\
3.10299999999997	-1.01590657914021e-10\\
3.10299999999999	-1.01590657914002e-10\\
3.10599999999996	-9.95014166311225e-11\\
3.10899999999993	-9.74212303151314e-11\\
3.10999999999997	-9.67295297139426e-11\\
3.11	-9.6729529713923e-11\\
3.11499999999997	-9.33106721281124e-11\\
3.115	-9.33106721280931e-11\\
3.11999999999998	-8.99617653087373e-11\\
3.12000000000001	-8.99617653087184e-11\\
3.12499999999998	-8.66746019490183e-11\\
3.12999999999996	-8.34411260651008e-11\\
3.13000000000001	-8.34411260650683e-11\\
3.13199999999999	-8.21643803651155e-11\\
3.13200000000002	-8.21643803650975e-11\\
3.13400000000001	-8.09012378939449e-11\\
3.136	-7.96512032548095e-11\\
3.13999999999997	-7.718850210907e-11\\
3.14	-7.71885021090527e-11\\
3.14000000000003	-7.71885021090353e-11\\
3.14799999999998	-7.23991512266233e-11\\
3.14999999999998	-7.12274129791639e-11\\
3.15	-7.12274129791473e-11\\
3.15799999999995	-6.66761896726878e-11\\
3.15999999999997	-6.55730292283979e-11\\
3.16	-6.55730292283823e-11\\
3.16099999999997	-6.50263452937866e-11\\
3.16099999999999	-6.50263452937711e-11\\
3.16199999999999	-6.4482854022313e-11\\
3.16299999999998	-6.39425021216175e-11\\
3.16499999999997	-6.28710048945904e-11\\
3.16899999999994	-6.07633726243679e-11\\
3.16999999999997	-6.02435299932207e-11\\
3.17	-6.0243529993206e-11\\
3.17799999999994	-5.61786235759564e-11\\
3.17999999999997	-5.51866383593599e-11\\
3.18	-5.51866383593459e-11\\
3.185	-5.28127408562041e-11\\
3.18500000000003	-5.28127408561911e-11\\
3.18999999999997	-5.06245674515098e-11\\
3.18999999999999	-5.06245674514979e-11\\
3.19499999999993	-4.86167554250972e-11\\
3.19999999999986	-4.67843841510351e-11\\
3.19999999999999	-4.678438415099e-11\\
3.20000000000002	-4.67843841509801e-11\\
3.20999999999989	-4.35511899275112e-11\\
3.21000000000002	-4.35511899274727e-11\\
3.21000000000005	-4.35511899274642e-11\\
3.21899999999997	-4.10679448270629e-11\\
3.21899999999999	-4.10679448270556e-11\\
3.22	-4.08160401971188e-11\\
3.22000000000003	-4.08160401971117e-11\\
3.22100000000004	-4.05678091728686e-11\\
3.22200000000005	-4.03221926468718e-11\\
3.22400000000006	-3.98387070516099e-11\\
3.22800000000009	-3.89020675097027e-11\\
3.23000000000003	-3.84485463242212e-11\\
3.23000000000006	-3.84485463242149e-11\\
3.23800000000012	-3.67290137691329e-11\\
3.23999999999997	-3.63219259741442e-11\\
3.24	-3.63219259741384e-11\\
3.24799999999997	-3.47799667099646e-11\\
3.24799999999999	-3.47799667099594e-11\\
3.24999999999997	-3.44153191943706e-11\\
3.25	-3.44153191943655e-11\\
3.25199999999998	-3.40596910884091e-11\\
3.25399999999996	-3.37139161727704e-11\\
3.25499999999998	-3.35446811482818e-11\\
3.255	-3.3544681148277e-11\\
3.25899999999996	-3.28917063315147e-11\\
3.25999999999997	-3.27343739492348e-11\\
3.26	-3.27343739492304e-11\\
3.26399999999996	-3.21282270655415e-11\\
3.26799999999992	-3.15584675779899e-11\\
3.26999999999997	-3.12869516539733e-11\\
3.27	-3.12869516539695e-11\\
3.27699999999999	-3.03700649485645e-11\\
3.27700000000002	-3.03700649485608e-11\\
3.27999999999997	-2.99875051523858e-11\\
3.28	-2.99875051523822e-11\\
3.28299999999995	-2.96106903056755e-11\\
3.2859999999999	-2.92392879890677e-11\\
3.28999999999997	-2.8751938760565e-11\\
3.29	-2.87519387605616e-11\\
3.2959999999999	-2.80215938385712e-11\\
3.29999999999997	-2.75275469030488e-11\\
3.3	-2.75275469030452e-11\\
3.3059999999999	-2.67738136550483e-11\\
3.30599999999995	-2.67738136550426e-11\\
3.30599999999999	-2.67738136550369e-11\\
3.30999999999997	-2.62617336256493e-11\\
3.31	-2.62617336256456e-11\\
3.31399999999998	-2.57412339406962e-11\\
3.31799999999996	-2.52116867395691e-11\\
3.31999999999997	-2.49432613577354e-11\\
3.32	-2.49432613577315e-11\\
3.32499999999998	-2.42607017722203e-11\\
3.325	-2.42607017722164e-11\\
3.32999999999998	-2.35603760196132e-11\\
3.33000000000001	-2.35603760196091e-11\\
3.33499999999998	-2.284056776732e-11\\
3.33500000000001	-2.28405677673159e-11\\
3.33999999999998	-2.20995129522146e-11\\
3.34000000000001	-2.20995129522104e-11\\
3.34499999999998	-2.13561550583863e-11\\
3.34999999999996	-2.06294319255422e-11\\
3.35000000000001	-2.06294319255347e-11\\
3.35999999999996	-1.92188023084681e-11\\
3.36	-1.92188023084619e-11\\
3.36399999999997	-1.86711101522306e-11\\
3.36399999999999	-1.86711101522267e-11\\
3.36799999999996	-1.81359109882585e-11\\
3.36999999999997	-1.78727332124254e-11\\
3.37	-1.78727332124217e-11\\
3.37399999999997	-1.7354705339603e-11\\
3.37799999999993	-1.68471057882617e-11\\
3.37999999999997	-1.65969672329849e-11\\
3.38	-1.65969672329814e-11\\
3.38799999999993	-1.5624490524753e-11\\
3.38999999999997	-1.53882856596439e-11\\
3.39	-1.53882856596406e-11\\
3.39299999999997	-1.50387851933971e-11\\
3.39299999999999	-1.50387851933938e-11\\
3.39499999999998	-1.48088562328646e-11\\
3.395	-1.48088562328613e-11\\
3.39699999999999	-1.45812790349981e-11\\
3.39899999999997	-1.43559643777905e-11\\
3.39999999999997	-1.42441278631695e-11\\
3.4	-1.42441278631663e-11\\
3.40399999999997	-1.3801998741322e-11\\
3.40799999999993	-1.33676984782305e-11\\
3.40999999999997	-1.31532708498908e-11\\
3.41	-1.31532708498877e-11\\
3.41799999999993	-1.23468274070594e-11\\
3.41999999999997	-1.21598289578157e-11\\
3.42	-1.21598289578131e-11\\
3.42199999999997	-1.19763027025746e-11\\
3.42199999999999	-1.1976302702572e-11\\
3.42399999999996	-1.17939539581156e-11\\
3.42599999999992	-1.16127112234657e-11\\
3.42999999999985	-1.12532599636107e-11\\
3.42999999999997	-1.12532599635996e-11\\
3.43	-1.12532599635971e-11\\
3.43799999999986	-1.05445287171985e-11\\
3.43999999999997	-1.03690592521473e-11\\
3.44	-1.03690592521449e-11\\
3.44799999999986	-9.67197034212084e-12\\
3.44999999999997	-9.49855372689938e-12\\
3.45	-9.49855372689692e-12\\
3.45099999999996	-9.41237932611694e-12\\
3.45099999999999	-9.41237932611451e-12\\
3.45199999999999	-9.32715275430196e-12\\
3.45299999999998	-9.24286565436004e-12\\
3.45499999999996	-9.07707691730687e-12\\
3.45899999999993	-8.75641569149809e-12\\
3.45999999999997	-8.67847863817919e-12\\
3.46	-8.67847863817699e-12\\
3.465	-8.30178202135442e-12\\
3.46500000000003	-8.30178202135234e-12\\
3.46999999999997	-7.94606389936375e-12\\
3.47	-7.94606389936179e-12\\
3.47499999999994	-7.61045248817232e-12\\
3.47999999999988	-7.29412529135568e-12\\
3.47999999999994	-7.29412529135216e-12\\
3.47999999999999	-7.29412529134867e-12\\
3.48999999999987	-6.74676202698775e-12\\
3.48999999999996	-6.7467620269833e-12\\
3.48999999999999	-6.74676202698193e-12\\
3.49999999999987	-6.28108238255635e-12\\
3.49999999999996	-6.28108238255219e-12\\
3.49999999999999	-6.28108238255091e-12\\
3.50899999999999	-5.88697449801625e-12\\
3.50900000000002	-5.88697449801504e-12\\
3.50999999999999	-5.84450180558707e-12\\
3.51000000000002	-5.84450180558586e-12\\
3.51100000000001	-5.80227708211787e-12\\
3.512	-5.76029618630208e-12\\
3.51399999999999	-5.67704944286389e-12\\
3.51799999999996	-5.51330239871449e-12\\
3.51999999999997	-5.43273789594903e-12\\
3.52	-5.43273789594789e-12\\
3.52799999999994	-5.12675371196895e-12\\
3.52999999999997	-5.05469343172973e-12\\
3.53	-5.05469343172872e-12\\
3.53499999999998	-4.88199347124512e-12\\
3.535	-4.88199347124417e-12\\
3.53799999999997	-4.78334756684714e-12\\
3.53799999999999	-4.78334756684623e-12\\
3.53999999999997	-4.71960194109289e-12\\
3.54	-4.71960194109199e-12\\
3.54199999999998	-4.65726093066475e-12\\
3.54399999999996	-4.59611931052526e-12\\
3.54799999999992	-4.47733881935329e-12\\
3.54999999999997	-4.41965337672261e-12\\
3.55	-4.4196533767218e-12\\
3.55799999999992	-4.19975427755904e-12\\
3.56	-4.14738240323028e-12\\
3.56000000000003	-4.14738240322954e-12\\
3.56699999999996	-3.96970437885065e-12\\
3.56699999999999	-3.96970437884994e-12\\
3.56999999999998	-3.89579765989381e-12\\
3.57000000000001	-3.89579765989312e-12\\
3.57299999999999	-3.82314081415196e-12\\
3.57599999999997	-3.75166974502726e-12\\
3.57999999999997	-3.65811069832464e-12\\
3.58	-3.65811069832398e-12\\
3.58599999999997	-3.52077931836244e-12\\
3.58999999999997	-3.43083591917435e-12\\
3.59	-3.43083591917371e-12\\
3.59599999999997	-3.29798439624951e-12\\
3.596	-3.29798439624889e-12\\
3.6	-3.21058985795257e-12\\
3.60000000000003	-3.21058985795195e-12\\
3.60400000000003	-3.12397371909621e-12\\
3.60499999999998	-3.1024253659982e-12\\
3.605	-3.10242536599759e-12\\
3.60900000000001	-3.0165913162159e-12\\
3.61	-2.99521212858456e-12\\
3.61000000000003	-2.99521212858395e-12\\
3.61400000000003	-2.91157677322764e-12\\
3.61800000000004	-2.83150478164585e-12\\
3.61999999999997	-2.79276558775134e-12\\
3.62	-2.7927655877508e-12\\
3.62499999999999	-2.69957845574993e-12\\
3.62500000000002	-2.69957845574941e-12\\
3.62999999999998	-2.61143993886586e-12\\
3.63000000000001	-2.61143993886537e-12\\
3.63499999999997	-2.52770267319305e-12\\
3.63999999999993	-2.44773008393152e-12\\
3.63999999999997	-2.44773008393098e-12\\
3.64	-2.44773008393043e-12\\
3.64999999999993	-2.29830371275204e-12\\
3.64999999999997	-2.29830371275133e-12\\
3.65	-2.29830371275093e-12\\
3.65399999999996	-2.24186112647559e-12\\
3.65399999999999	-2.24186112647519e-12\\
3.65799999999995	-2.18670203552576e-12\\
3.65999999999997	-2.15957668875149e-12\\
3.66	-2.15957668875111e-12\\
3.66399999999996	-2.10618121610029e-12\\
3.66799999999993	-2.05385643619316e-12\\
3.66999999999997	-2.02806984741587e-12\\
3.67	-2.0280698474155e-12\\
3.67499999999998	-1.96461846743619e-12\\
3.67500000000001	-1.96461846743583e-12\\
3.67999999999998	-1.90249323602781e-12\\
3.68000000000001	-1.90249323602746e-12\\
3.68299999999996	-1.86621791613807e-12\\
3.68299999999999	-1.86621791613773e-12\\
3.68599999999995	-1.83118660230184e-12\\
3.68899999999991	-1.79736839142776e-12\\
3.68999999999998	-1.7863600461193e-12\\
3.69000000000001	-1.78636004611898e-12\\
3.69599999999992	-1.72301122731574e-12\\
3.69999999999997	-1.68327609679402e-12\\
3.7	-1.68327609679375e-12\\
3.70599999999992	-1.62653687151272e-12\\
3.71	-1.59021020007978e-12\\
3.71000000000003	-1.59021020007953e-12\\
3.71199999999996	-1.57247517091489e-12\\
3.71199999999999	-1.57247517091464e-12\\
3.71399999999993	-1.5550163721993e-12\\
3.71599999999986	-1.53782695885147e-12\\
3.71999999999973	-1.50422943470876e-12\\
3.71999999999997	-1.50422943470671e-12\\
3.72	-1.50422943470648e-12\\
3.72799999999974	-1.43894586721611e-12\\
3.72999999999997	-1.42287841822669e-12\\
3.73	-1.42287841822646e-12\\
3.73799999999974	-1.35943335690188e-12\\
3.73999999999997	-1.34374718242011e-12\\
3.74	-1.34374718241989e-12\\
3.74099999999996	-1.33592450997468e-12\\
3.74099999999999	-1.33592450997446e-12\\
3.74199999999998	-1.32811228419084e-12\\
3.74299999999998	-1.32030973904822e-12\\
3.74499999999996	-1.3047306326375e-12\\
3.74500000000001	-1.30473063263715e-12\\
3.74899999999997	-1.27365499372016e-12\\
3.74999999999997	-1.26589884427959e-12\\
3.75	-1.26589884427937e-12\\
3.75399999999997	-1.23490248038562e-12\\
3.75799999999994	-1.20391483933303e-12\\
3.75999999999997	-1.18840910311248e-12\\
3.76	-1.18840910311226e-12\\
3.76799999999994	-1.12616478981275e-12\\
3.76999999999996	-1.11051786509164e-12\\
3.76999999999999	-1.11051786509141e-12\\
3.77799999999993	-1.04995351630452e-12\\
3.77999999999996	-1.03544793405671e-12\\
3.77999999999999	-1.03544793405651e-12\\
3.78799999999993	-9.79796963995601e-13\\
3.78999999999996	-9.66449788002072e-13\\
3.78999999999999	-9.66449788001884e-13\\
3.79799999999993	-9.15150941002443e-13\\
3.79899999999996	-9.08962871059808e-13\\
3.79899999999999	-9.08962871059633e-13\\
3.79999999999996	-9.02822592760261e-13\\
3.79999999999999	-9.02822592760087e-13\\
3.80099999999998	-8.9672950456746e-13\\
3.80199999999997	-8.90683009016079e-13\\
3.80399999999995	-8.78727429867746e-13\\
3.80799999999991	-8.55349606863182e-13\\
3.80999999999996	-8.43918197008127e-13\\
3.80999999999999	-8.43918197007966e-13\\
3.81499999999998	-8.15975383097839e-13\\
3.81500000000001	-8.15975383097682e-13\\
3.81999999999999	-7.88840799030302e-13\\
3.82000000000002	-7.8884079903015e-13\\
3.825	-7.62447944208307e-13\\
3.82799999999996	-7.46941282714382e-13\\
3.82799999999999	-7.46941282714236e-13\\
3.82999999999999	-7.36732136444989e-13\\
3.83000000000002	-7.36732136444845e-13\\
3.83200000000002	-7.26621217321988e-13\\
3.83400000000002	-7.16604561263466e-13\\
3.83800000000001	-6.96838365587595e-13\\
3.83999999999997	-6.87081076029482e-13\\
3.84	-6.87081076029344e-13\\
3.84799999999999	-6.50160464939608e-13\\
3.84999999999997	-6.41524234131149e-13\\
3.85	-6.41524234131028e-13\\
3.85699999999996	-6.11815381982768e-13\\
3.85699999999999	-6.11815381982647e-13\\
3.85999999999997	-5.99131020047213e-13\\
3.86	-5.99131020047093e-13\\
3.86299999999998	-5.86459287561072e-13\\
3.86599999999997	-5.73789005770894e-13\\
3.86999999999997	-5.56878233748017e-13\\
3.87	-5.56878233747896e-13\\
3.87599999999997	-5.3142161779762e-13\\
3.87999999999997	-5.14351418679477e-13\\
3.88	-5.14351418679355e-13\\
3.88500000000001	-4.93409839818484e-13\\
3.88500000000003	-4.93409839818367e-13\\
3.88599999999996	-4.89329381498576e-13\\
3.88599999999999	-4.89329381498461e-13\\
3.88699999999998	-4.85283934389823e-13\\
3.88799999999997	-4.8127310196119e-13\\
3.88999999999995	-4.7335371208484e-13\\
3.89	-4.73353712084647e-13\\
3.89399999999996	-4.57913234540109e-13\\
3.89799999999992	-4.42983732659483e-13\\
3.9	-4.35703249727157e-13\\
3.90000000000003	-4.35703249727055e-13\\
3.90799999999995	-4.07923415634119e-13\\
3.91	-4.01311593417117e-13\\
3.91000000000003	-4.01311593417024e-13\\
3.91499999999996	-3.85336460384293e-13\\
3.91499999999999	-3.85336460384205e-13\\
3.91999999999993	-3.70122269912516e-13\\
3.92	-3.7012226991229e-13\\
3.92000000000003	-3.70122269912205e-13\\
3.92499999999997	-3.5563173550312e-13\\
3.9299999999999	-3.41829344619301e-13\\
3.93000000000003	-3.41829344618948e-13\\
3.93000000000006	-3.41829344618872e-13\\
3.93999999999993	-3.17320749675053e-13\\
3.93999999999998	-3.17320749674947e-13\\
3.94	-3.17320749674885e-13\\
3.94399999999996	-3.08551938701094e-13\\
3.94399999999999	-3.08551938701032e-13\\
3.94799999999995	-2.99930586884422e-13\\
3.94999999999998	-2.95670973821325e-13\\
3.95	-2.95670973821265e-13\\
3.95399999999996	-2.87245522843779e-13\\
3.95499999999998	-2.85157491221809e-13\\
3.95500000000001	-2.8515749122175e-13\\
3.95899999999997	-2.76872549279073e-13\\
3.96	-2.7481709491124e-13\\
3.96000000000003	-2.74817094911182e-13\\
3.96399999999999	-2.66652357761793e-13\\
3.96799999999995	-2.58569365010945e-13\\
3.97	-2.54554556514455e-13\\
3.97000000000003	-2.54554556514398e-13\\
3.97299999999996	-2.48636804435287e-13\\
3.97299999999999	-2.48636804435232e-13\\
3.97599999999992	-2.42900198771178e-13\\
3.97899999999986	-2.37339678609551e-13\\
3.97999999999997	-2.35524449763293e-13\\
3.98	-2.35524449763242e-13\\
3.98599999999987	-2.2502263670499e-13\\
3.98999999999997	-2.18379955024267e-13\\
3.99	-2.1837995502422e-13\\
3.99599999999987	-2.08927674346956e-13\\
3.99999999999997	-2.0295290216578e-13\\
4	-2.02952902165738e-13\\
4.00199999999993	-2.00085179746725e-13\\
4.00199999999999	-2.00085179746645e-13\\
4.00399999999992	-1.9732931051357e-13\\
4.00599999999986	-1.94684213965039e-13\\
4.00999999999972	-1.89722233890722e-13\\
4.00999999999995	-1.89722233890456e-13\\
4.01	-1.89722233890388e-13\\
4.01799999999973	-1.80373970631728e-13\\
4.01999999999995	-1.78077473266697e-13\\
4.02	-1.78077473266632e-13\\
4.02500000000001	-1.72397351036583e-13\\
4.02500000000006	-1.72397351036519e-13\\
4.02999999999995	-1.6679343343451e-13\\
4.03	-1.66793433434447e-13\\
4.03099999999999	-1.65680582965713e-13\\
4.03100000000005	-1.6568058296565e-13\\
4.03200000000004	-1.64570121774813e-13\\
4.03300000000003	-1.63461941015748e-13\\
4.035	-1.61251986552836e-13\\
4.03899999999996	-1.56854677492997e-13\\
4.03999999999994	-1.55759429715498e-13\\
4.04	-1.55759429715436e-13\\
4.04799999999991	-1.47297263977629e-13\\
4.04999999999994	-1.45270158350113e-13\\
4.05	-1.45270158350056e-13\\
4.05799999999991	-1.37491629793765e-13\\
4.05999999999994	-1.35625659186116e-13\\
4.06	-1.35625659186064e-13\\
};
\end{axis}
\end{tikzpicture}%}
  \caption{The angular displacement of pendulum $P_1$ as a function of time.
    \texttt{Blue}: $C_1 = 6$ ms, \texttt{Red}: $C_1 = 10$ ms}
  \label{fig:01.5.6_10.1}
\end{figure}

\begin{figure}[H]\centering
  \scalebox{1}{% This file was created by matlab2tikz.
%
%The latest updates can be retrieved from
%  http://www.mathworks.com/matlabcentral/fileexchange/22022-matlab2tikz-matlab2tikz
%where you can also make suggestions and rate matlab2tikz.
%
\definecolor{mycolor1}{rgb}{0.00000,0.44700,0.74100}%
\definecolor{mycolor2}{rgb}{0.85000,0.32500,0.09800}%
%
\begin{tikzpicture}

\begin{axis}[%
width=4.133in,
height=3.26in,
at={(0.693in,0.44in)},
scale only axis,
xmin=0,
xmax=800,
ymin=-0.15,
ymax=0.2,
axis background/.style={fill=white}
]
\addplot [color=mycolor1,solid,forget plot]
  table[row sep=crcr]{%
1	0.15313\\
2	0.15313\\
3	0.153131614989962\\
4	0.153188143215565\\
5	0.153265080494076\\
6	0.153265080494076\\
7	0.153670560289007\\
8	0.153670560289007\\
9	0.153071664853654\\
10	0.153071664853654\\
11	0.152365329512728\\
12	0.152365329512728\\
13	0.148724274488254\\
14	0.148724274488254\\
15	0.146046083995443\\
16	0.146046083995443\\
17	0.142794627452511\\
18	0.138968470912041\\
19	0.13896847091204\\
20	0.131789348388764\\
21	0.131789348388763\\
22	0.123959478342115\\
23	0.122314584640995\\
24	0.122314584640994\\
25	0.120643199324198\\
26	0.120643199324197\\
27	0.117220624955326\\
28	0.113691084393681\\
29	0.106308316385691\\
30	0.100485154924159\\
31	0.100485154924158\\
32	0.0964653994777478\\
33	0.0964653994777469\\
34	0.092334609798713\\
35	0.0880919761882657\\
36	0.0837366670740184\\
37	0.0837366670740165\\
38	0.0746845854767947\\
39	0.0746845854767927\\
40	0.0746845854767906\\
41	0.0654820973483735\\
42	0.0609387524318376\\
43	0.0609387524318356\\
44	0.0519634081808075\\
45	0.0519634081808055\\
46	0.0431306724709095\\
47	0.0387656144357706\\
48	0.0387656144357686\\
49	0.0365955374773486\\
50	0.0365955374773467\\
51	0.0344336199800889\\
52	0.03227975600762\\
53	0.0279957668674395\\
54	0.0237427303784554\\
55	0.0237427303784535\\
56	0.0154879858559104\\
57	0.00958444869935146\\
58	0.00958444869934975\\
59	0.00209147738315415\\
60	0.000285191714757269\\
61	0.000285191714755677\\
62	-0.00149449006990914\\
63	-0.00149449006991071\\
64	-0.00324765517394124\\
65	-0.00497438950448962\\
66	-0.00834890299548232\\
67	-0.0116186919605986\\
68	-0.0116186919606\\
69	-0.0178466394573827\\
70	-0.0178466394573853\\
71	-0.0178466394573879\\
72	-0.0236631150271634\\
73	-0.0236631150271658\\
74	-0.0236631150271682\\
75	-0.0290726790847185\\
76	-0.031626207097874\\
77	-0.0316262070978751\\
78	-0.03407957300888\\
79	-0.0340795730088821\\
80	-0.0364271843737179\\
81	-0.0386634280235806\\
82	-0.0428035436794524\\
83	-0.046503166088059\\
84	-0.0465031660880606\\
85	-0.0505126366966201\\
86	-0.0505126366966214\\
87	-0.0512329261213502\\
88	-0.0512329261213514\\
89	-0.0519260994409184\\
90	-0.0525921906227423\\
91	-0.0538432558021337\\
92	-0.0549863668732067\\
93	-0.0549863668732077\\
94	-0.0570007720155776\\
95	-0.0578852528495309\\
96	-0.0578852528495317\\
97	-0.0586881556806868\\
98	-0.0586881556806875\\
99	-0.0594096378799837\\
100	-0.0600498408609338\\
101	-0.0606088901058539\\
102	-0.0606088901058543\\
103	-0.0614839498018867\\
104	-0.0620355030603464\\
105	-0.0620355030603465\\
106	-0.0621228997391115\\
107	-0.0621228997391117\\
108	-0.0621901098124187\\
109	-0.0622371365737359\\
110	-0.0622706483888091\\
111	-0.0622571350847035\\
112	-0.0622571350847035\\
113	-0.0620804705828559\\
114	-0.0619304203383477\\
115	-0.0619304203383475\\
116	-0.0615067368047821\\
117	-0.061233020472304\\
118	-0.0612330204723037\\
119	-0.0605615704193962\\
120	-0.0597243116296848\\
121	-0.0597243116296842\\
122	-0.0597243116296836\\
123	-0.0589871655511613\\
124	-0.0589871655511608\\
125	-0.0581560603911522\\
126	-0.0581560603911517\\
127	-0.0572306296136106\\
128	-0.0569011221000455\\
129	-0.0569011221000449\\
130	-0.0562104650884198\\
131	-0.0562104650884192\\
132	-0.0554939446732612\\
133	-0.0547678599943136\\
134	-0.0540320687361973\\
135	-0.0540320687361967\\
136	-0.0532864266819164\\
137	-0.0532864266819157\\
138	-0.0525307876837086\\
139	-0.0517650036334558\\
140	-0.0509889244345494\\
141	-0.0509889244345487\\
142	-0.0494052700894414\\
143	-0.0485973845409472\\
144	-0.0485973845409465\\
145	-0.0469487049222538\\
146	-0.0461075877045604\\
147	-0.0461075877045596\\
148	-0.0452638608140846\\
149	-0.0452638608140839\\
150	-0.0444261532252999\\
151	-0.0435943007442838\\
152	-0.0431805191730594\\
153	-0.0431805191730586\\
154	-0.042768140324533\\
155	-0.0427681403245322\\
156	-0.0423571439924296\\
157	-0.0419475100374713\\
158	-0.0411322490361785\\
159	-0.0403221975263589\\
160	-0.0403221975263574\\
161	-0.0387170888829923\\
162	-0.0371309256194401\\
163	-0.0371309256194387\\
164	-0.0367371989641445\\
165	-0.0367371989641431\\
166	-0.0363445593118938\\
167	-0.0359529874189366\\
168	-0.0351729702138744\\
169	-0.0347844866804151\\
170	-0.0347844866804137\\
171	-0.0332402760675451\\
172	-0.032473740201349\\
173	-0.0324737402013477\\
174	-0.0309768515277924\\
175	-0.0295444303379096\\
176	-0.0295444303379072\\
177	-0.0295444303379047\\
178	-0.0275142319192832\\
179	-0.0275142319192821\\
180	-0.0262381743224773\\
181	-0.0262381743224762\\
182	-0.0256229882303971\\
183	-0.0256229882303961\\
184	-0.0250228688456157\\
185	-0.0244376985423862\\
186	-0.0238673626250648\\
187	-0.0238673626250638\\
188	-0.0233108682285088\\
189	-0.0233108682285079\\
190	-0.0227672252005078\\
191	-0.022236326984904\\
192	-0.0217180695235936\\
193	-0.0217180695235927\\
194	-0.0207190730028746\\
195	-0.0197694523735308\\
196	-0.0195396741022794\\
197	-0.0195396741022786\\
198	-0.0186507307825328\\
199	-0.018650730782532\\
200	-0.0184359829485392\\
201	-0.0184359829485385\\
202	-0.0182242090470824\\
203	-0.0180153987009941\\
204	-0.01760662789264\\
205	-0.0172095904028153\\
206	-0.0172095904028146\\
207	-0.0164504063821836\\
208	-0.0162678248691355\\
209	-0.0162678248691348\\
210	-0.0159097955544508\\
211	-0.0159097955544502\\
212	-0.0155602458559603\\
213	-0.0152191072605565\\
214	-0.0147230243676384\\
215	-0.0147230243676379\\
216	-0.0140904089839253\\
217	-0.0137863164050161\\
218	-0.0137863164050155\\
219	-0.0134902872333528\\
220	-0.0134902872333523\\
221	-0.0132022634464662\\
222	-0.0132022634464657\\
223	-0.0129221885905648\\
224	-0.0126500077700806\\
225	-0.0126500077700797\\
226	-0.012125862983203\\
227	-0.0118729829735022\\
228	-0.0118729829735018\\
229	-0.011385359886108\\
230	-0.0111505212323883\\
231	-0.0111505212323879\\
232	-0.0106985595508986\\
233	-0.0102699253278371\\
234	-0.00996355545253901\\
235	-0.00996355545253866\\
236	-0.0096699827491936\\
237	-0.00966998274919326\\
238	-0.00938907775174519\\
239	-0.00920878395989238\\
240	-0.00920878395989206\\
241	-0.00912071657186842\\
242	-0.00912071657186811\\
243	-0.00903402974610353\\
244	-0.00894871923490015\\
245	-0.00886478085802219\\
246	-0.0088647808580219\\
247	-0.00870031158854401\\
248	-0.00853989716740149\\
249	-0.00838350615274735\\
250	-0.00838350615274707\\
251	-0.00808267251348065\\
252	-0.00800993178341249\\
253	-0.00800993178341223\\
254	-0.00772873264866916\\
255	-0.00759394329312424\\
256	-0.00759394329312401\\
257	-0.00733585290516443\\
258	-0.00721250128602441\\
259	-0.00721250128602419\\
260	-0.00697476061535019\\
261	-0.00674727983018301\\
262	-0.00674727983018241\\
263	-0.00674727983018221\\
264	-0.0064249079467603\\
265	-0.00642490794676012\\
266	-0.00612465260943283\\
267	-0.00612465260943266\\
268	-0.0059826516063656\\
269	-0.00598265160636544\\
270	-0.00584598408830854\\
271	-0.00575780559630316\\
272	-0.00575780559630301\\
273	-0.00562989442946651\\
274	-0.00550716117548773\\
275	-0.0055071611754871\\
276	-0.00550716117548689\\
277	-0.00546739131729642\\
278	-0.00546739131729628\\
279	-0.00542809205493715\\
280	-0.00538916471605226\\
281	-0.00531241819704646\\
282	-0.00516330552264952\\
283	-0.00516330552264939\\
284	-0.00501993638212905\\
285	-0.00498497825355961\\
286	-0.00498497825355948\\
287	-0.00484863097486232\\
288	-0.0048486309748622\\
289	-0.00478252515245442\\
290	-0.0047825251524543\\
291	-0.00474998348273337\\
292	-0.00474998348273326\\
293	-0.00471778050903213\\
294	-0.00468591465339736\\
295	-0.00462318806703604\\
296	-0.00456179142899598\\
297	-0.00456179142899587\\
298	-0.00444244961269803\\
299	-0.00435570274098324\\
300	-0.00435570274098314\\
301	-0.00424364682919994\\
302	-0.00418913978671694\\
303	-0.00418913978671685\\
304	-0.00408311410884927\\
305	-0.00408311410884908\\
306	-0.00408311410884899\\
307	-0.00400616888817767\\
308	-0.00400616888817757\\
309	-0.00393139219129555\\
310	-0.00385875104045262\\
311	-0.00385875104045245\\
312	-0.00385875104045228\\
313	-0.00383494896778564\\
314	-0.00383494896778555\\
315	-0.00381126472966637\\
316	-0.00378769716546426\\
317	-0.00374090744521033\\
318	-0.00369457063605833\\
319	-0.00369457063605816\\
320	-0.0036486776558763\\
321	-0.00364867765587614\\
322	-0.00360321950952306\\
323	-0.00355818728702618\\
324	-0.00346936538951466\\
325	-0.00338214232149069\\
326	-0.00338214232149054\\
327	-0.00333910893017963\\
328	-0.00333910893017948\\
329	-0.00329644969685139\\
330	-0.00327525777344733\\
331	-0.00327525777344718\\
332	-0.00325415625951213\\
333	-0.00325415625951198\\
334	-0.00323314412106835\\
335	-0.00321222032851632\\
336	-0.00317063368428566\\
337	-0.00312938817567781\\
338	-0.00312938817567766\\
339	-0.00304820674104637\\
340	-0.00300833451107474\\
341	-0.0030083345110746\\
342	-0.00296893082144862\\
343	-0.00296893082144848\\
344	-0.0029299879489443\\
345	-0.00289149826060799\\
346	-0.00285345421231157\\
347	-0.00285345421231143\\
348	-0.00277867329483269\\
349	-0.00276024504295249\\
350	-0.00276024504295236\\
351	-0.00268757285469358\\
352	-0.00265184880636015\\
353	-0.00265184880636002\\
354	-0.00259888476765015\\
355	-0.00259888476765003\\
356	-0.00254654778824355\\
357	-0.00249481478667443\\
358	-0.00249481478667431\\
359	-0.00249481478667418\\
360	-0.0023930697129838\\
361	-0.00239306971298368\\
362	-0.00239306971298356\\
363	-0.00229347004188118\\
364	-0.00226071643314977\\
365	-0.00226071643314965\\
366	-0.00216370456659175\\
367	-0.00216370456659164\\
368	-0.00216370456659152\\
369	-0.00206843344302369\\
370	-0.00206843344302358\\
371	-0.00206843344302346\\
372	-0.00202174583165955\\
373	-0.00202174583165944\\
374	-0.00197612637154641\\
375	-0.00194629686101901\\
376	-0.0019462968610189\\
377	-0.00190241317260978\\
378	-0.00185954500765136\\
379	-0.00185954500765116\\
380	-0.00184547801078418\\
381	-0.00184547801078408\\
382	-0.00183152105639463\\
383	-0.00181767346053036\\
384	-0.00181767346053026\\
385	-0.00179030363556277\\
386	-0.0017633631713082\\
387	-0.00173684678733163\\
388	-0.00173684678733154\\
389	-0.00168477807008363\\
390	-0.00167192806435549\\
391	-0.0016719280643554\\
392	-0.00162117784545424\\
393	-0.00159618392566223\\
394	-0.00159618392566214\\
395	-0.00154693397006005\\
396	-0.00154693397005996\\
397	-0.00149863576091037\\
398	-0.0014748316990718\\
399	-0.00147483169907172\\
400	-0.00145125142991222\\
401	-0.00145125142991213\\
402	-0.00142789033163865\\
403	-0.00142789033163857\\
404	-0.00140474382538547\\
405	-0.00138180737436343\\
406	-0.00135907648293958\\
407	-0.0013590764829395\\
408	-0.0013142135970691\\
409	-0.00131421359706894\\
410	-0.00131421359706887\\
411	-0.00127027370634741\\
412	-0.00122737607303918\\
413	-0.00120630756079553\\
414	-0.00120630756079546\\
415	-0.00115471242812662\\
416	-0.00115471242812654\\
417	-0.00114457383303655\\
418	-0.00114457383303648\\
419	-0.00113449421574781\\
420	-0.00113449421574773\\
421	-0.00112447308233544\\
422	-0.00111450994176199\\
423	-0.00109475568916581\\
424	-0.00107522758604948\\
425	-0.00107522758604941\\
426	-0.00103698988691533\\
427	-0.00102761442563734\\
428	-0.00102761442563727\\
429	-0.00100908103090358\\
430	-0.00100908103090352\\
431	-0.000990834652012911\\
432	-0.000972871712604208\\
433	-0.000955188691871652\\
434	-0.00095518869187159\\
435	-0.000920648596967051\\
436	-0.000887187286082886\\
437	-0.000870852943369295\\
438	-0.000870852943369238\\
439	-0.00083896087550719\\
440	-0.000838960875507134\\
441	-0.000808083544671208\\
442	-0.000793017809496112\\
443	-0.000793017809496059\\
444	-0.000778196741068832\\
445	-0.00077819674106878\\
446	-0.000763631499413554\\
447	-0.000749333294677653\\
448	-0.000735299324357685\\
449	-0.000735299324357636\\
450	-0.000708013135452678\\
451	-0.000694755568668496\\
452	-0.000694755568668449\\
453	-0.000668998497336283\\
454	-0.00066271531406811\\
455	-0.000662715314068066\\
456	-0.00065649394429886\\
457	-0.000656493944298816\\
458	-0.000650334083185832\\
459	-0.000644235428888822\\
460	-0.000632220548386102\\
461	-0.000620446947827684\\
462	-0.000620446947827642\\
463	-0.000603234887845982\\
464	-0.000603234887845942\\
465	-0.000586554887958984\\
466	-0.00057572683474829\\
467	-0.000575726834748252\\
468	-0.000559917369083065\\
469	-0.000544620859781388\\
470	-0.000544620859781318\\
471	-0.00053963484119366\\
472	-0.000539634841193625\\
473	-0.000534704822068924\\
474	-0.000529830560814073\\
475	-0.000520248359275258\\
476	-0.000506287358166193\\
477	-0.00050628735816616\\
478	-0.000488432797560279\\
479	-0.000479827328689932\\
480	-0.000479827328689902\\
481	-0.000463158262339389\\
482	-0.000455068069382597\\
483	-0.000455068069382568\\
484	-0.000439368435098841\\
485	-0.000439368435098813\\
486	-0.00042429959742365\\
487	-0.000416998016139368\\
488	-0.000416998016139343\\
489	-0.000402853372750483\\
490	-0.000402853372750459\\
491	-0.000392640650914832\\
492	-0.000392640650914808\\
493	-0.000382762134321589\\
494	-0.000382762134321566\\
495	-0.000373213466382827\\
496	-0.000363990435981253\\
497	-0.000363990435981211\\
498	-0.000363990435981189\\
499	-0.000346452790223192\\
500	-0.000340864097740224\\
501	-0.000340864097740204\\
502	-0.000327444554621578\\
503	-0.00032744455462156\\
504	-0.000324854251627215\\
505	-0.000324854251627197\\
506	-0.000322294850800829\\
507	-0.000319766226725563\\
508	-0.000314800814717995\\
509	-0.000309957042364469\\
510	-0.000309957042364452\\
511	-0.000300630642712753\\
512	-0.000296146187403729\\
513	-0.000296146187403713\\
514	-0.000291755346293803\\
515	-0.000291755346293788\\
516	-0.000287432889674865\\
517	-0.000283177970328631\\
518	-0.000278989754274457\\
519	-0.000278989754274443\\
520	-0.000270810161354624\\
521	-0.000262887697692615\\
522	-0.000251472585533723\\
523	-0.00025147258553371\\
524	-0.00024962353120844\\
525	-0.000249623531208427\\
526	-0.000247789508349448\\
527	-0.000245970427068194\\
528	-0.000242376733425243\\
529	-0.000240601944970145\\
530	-0.000240601944970133\\
531	-0.000233647826061438\\
532	-0.000230256597378268\\
533	-0.000230256597378256\\
534	-0.000223652582586451\\
535	-0.000220441028094253\\
536	-0.000220441028094242\\
537	-0.000217288932313668\\
538	-0.000217288932313657\\
539	-0.00021419567742588\\
540	-0.000211160657140891\\
541	-0.000208183276583013\\
542	-0.000208183276583002\\
543	-0.000202399111537331\\
544	-0.000202399111537321\\
545	-0.000196838647205059\\
546	-0.000195482969651698\\
547	-0.000195482969651688\\
548	-0.000194140933644273\\
549	-0.000194140933644264\\
550	-0.000192809225187038\\
551	-0.000191484530787782\\
552	-0.00018885592486254\\
553	-0.000188855924862531\\
554	-0.000186254600653333\\
555	-0.000183680048292316\\
556	-0.000181131763157299\\
557	-0.00018113176315729\\
558	-0.000178609245777949\\
559	-0.00017860924577794\\
560	-0.00017611200173442\\
561	-0.000173639541557222\\
562	-0.000168767039124104\\
563	-0.000165174149997176\\
564	-0.000165174149997168\\
565	-0.00016398791812363\\
566	-0.000163987918123621\\
567	-0.000162807288071465\\
568	-0.000161632201976358\\
569	-0.000159298431608309\\
570	-0.000156986149591459\\
571	-0.000156986149591451\\
572	-0.000152424241883398\\
573	-0.000152424241883383\\
574	-0.000152424241883367\\
575	-0.000151296491998817\\
576	-0.000151296491998809\\
577	-0.000150174064988998\\
578	-0.000149057248801709\\
579	-0.000146840230270077\\
580	-0.000144645001859707\\
581	-0.000144645001859699\\
582	-0.000141392075653412\\
583	-0.000141392075653405\\
584	-0.000138185775500394\\
585	-0.000135024687369718\\
586	-0.00013502468736971\\
587	-0.000135024687369702\\
588	-0.000132941723821195\\
589	-0.000132941723821187\\
590	-0.000130877826253349\\
591	-0.000128832590125961\\
592	-0.000126805614564855\\
593	-0.000126805614564848\\
594	-0.000122820316751409\\
595	-0.000120865077495109\\
596	-0.000120865077495103\\
597	-0.000117027502047741\\
598	-0.000116083021664212\\
599	-0.000116083021664206\\
600	-0.000115144413676528\\
601	-0.000115144413676522\\
602	-0.000114211632093562\\
603	-0.000113284631207966\\
604	-0.000111447790116981\\
605	-0.000109633530375599\\
606	-0.000109633530375592\\
607	-0.000106071336901298\\
608	-0.000105194351508317\\
609	-0.00010519435150831\\
610	-0.000102595257840317\\
611	-0.000102595257840311\\
612	-0.000100043069139321\\
613	-9.7536659848261e-05\\
614	-9.75366598482552e-05\\
615	-9.50749245916848e-05\\
616	-9.5074924591679e-05\\
617	-9.26625234900481e-05\\
618	-9.03041383956829e-05\\
619	-8.6490717806386e-05\\
620	-8.64907178063807e-05\\
621	-8.21023051300174e-05\\
622	-8.21023051300074e-05\\
623	-7.79113493409311e-05\\
624	-7.79113493409216e-05\\
625	-7.3910456506309e-05\\
626	-7.39104565062999e-05\\
627	-7.26226099703555e-05\\
628	-7.26226099703509e-05\\
629	-7.13643974323865e-05\\
630	-7.01355722724094e-05\\
631	-6.8935893635937e-05\\
632	-6.89358936359286e-05\\
633	-6.66230410409537e-05\\
634	-6.44240262516722e-05\\
635	-6.38918557160394e-05\\
636	-6.38918557160319e-05\\
637	-6.13352024439096e-05\\
638	-6.13352024439025e-05\\
639	-5.89499490360049e-05\\
640	-5.84932055236174e-05\\
641	-5.8493205523611e-05\\
642	-5.80431780747783e-05\\
643	-5.8043178074772e-05\\
644	-5.75982749071178e-05\\
645	-5.71569044873816e-05\\
646	-5.62846755547533e-05\\
647	-5.54263203166417e-05\\
648	-5.54263203166356e-05\\
649	-5.37505606517162e-05\\
650	-5.25289275156282e-05\\
651	-5.25289275156224e-05\\
652	-5.09459744840712e-05\\
653	-5.09459744840656e-05\\
654	-5.01738158365423e-05\\
655	-5.01738158365368e-05\\
656	-4.94143331365673e-05\\
657	-4.86673775225938e-05\\
658	-4.86673775225885e-05\\
659	-4.79328025889087e-05\\
660	-4.79328025889035e-05\\
661	-4.72100886209416e-05\\
662	-4.64987182289034e-05\\
663	-4.57985519814528e-05\\
664	-4.57985519814479e-05\\
665	-4.51094526442447e-05\\
666	-4.51094526442399e-05\\
667	-4.44312851520494e-05\\
668	-4.37639165814064e-05\\
669	-4.31072161254649e-05\\
670	-4.31072161254602e-05\\
671	-4.18253067642421e-05\\
672	-4.08909361827503e-05\\
673	-4.0890936182746e-05\\
674	-3.96804071947624e-05\\
675	-3.90900031293295e-05\\
676	-3.90900031293254e-05\\
677	-3.87981956482676e-05\\
678	-3.87981956482634e-05\\
679	-3.85082726205468e-05\\
680	-3.82202198396166e-05\\
681	-3.76496686504906e-05\\
682	-3.73671422849065e-05\\
683	-3.73671422849025e-05\\
684	-3.70864302505011e-05\\
685	-3.70864302504971e-05\\
686	-3.68075187925797e-05\\
687	-3.65303942441707e-05\\
688	-3.59814516461387e-05\\
689	-3.54394948615293e-05\\
690	-3.54394948615255e-05\\
691	-3.43761151817767e-05\\
692	-3.33394214604027e-05\\
693	-3.33394214603956e-05\\
694	-3.33394214603886e-05\\
695	-3.28308270200997e-05\\
696	-3.28308270200961e-05\\
697	-3.23286008688398e-05\\
698	-3.18326445609249e-05\\
699	-3.18326445609213e-05\\
700	-3.18326445609177e-05\\
701	-3.08591538488759e-05\\
702	-3.03814286371267e-05\\
703	-3.03814286371234e-05\\
704	-2.94480294466614e-05\\
705	-2.85464106227169e-05\\
706	-2.85464106227042e-05\\
707	-2.85464106227011e-05\\
708	-2.7252030869233e-05\\
709	-2.725203086923e-05\\
710	-2.62251417991066e-05\\
711	-2.62251417991037e-05\\
712	-2.6025288472609e-05\\
713	-2.60252884726062e-05\\
714	-2.58272535828071e-05\\
715	-2.5631027425519e-05\\
716	-2.52439629361489e-05\\
717	-2.48640191383652e-05\\
718	-2.48640191383625e-05\\
719	-2.44901822689232e-05\\
720	-2.44901822689206e-05\\
721	-2.41214397615318e-05\\
722	-2.3757719341231e-05\\
723	-2.33989497174015e-05\\
724	-2.3398949717399e-05\\
725	-2.26959825370229e-05\\
726	-2.20119870578637e-05\\
727	-2.18438893475066e-05\\
728	-2.18438893475042e-05\\
729	-2.10203966062128e-05\\
730	-2.10203966062105e-05\\
731	-2.02244954409214e-05\\
732	-2.00685398077086e-05\\
733	-2.00685398077064e-05\\
734	-1.93044643132627e-05\\
735	-1.9154737332757e-05\\
736	-1.91547373327549e-05\\
737	-1.90062037346705e-05\\
738	-1.90062037346684e-05\\
739	-1.88590374865917e-05\\
740	-1.87132313770234e-05\\
741	-1.84256710615945e-05\\
742	-1.80043551613652e-05\\
743	-1.80043551613633e-05\\
744	-1.7461025495754e-05\\
745	-1.71971369393908e-05\\
746	-1.71971369393889e-05\\
747	-1.68108786213554e-05\\
748	-1.68108786213536e-05\\
749	-1.66846537550029e-05\\
750	-1.66846537550012e-05\\
751	-1.65596824819567e-05\\
752	-1.64359586785654e-05\\
753	-1.61922292916064e-05\\
754	-1.60722117651356e-05\\
755	-1.60722117651339e-05\\
756	-1.55998671664565e-05\\
757	-1.54834052032631e-05\\
758	-1.54834052032615e-05\\
759	-1.50238828899385e-05\\
760	-1.4797838355262e-05\\
761	-1.47978383552604e-05\\
762	-1.46857257808375e-05\\
763	-1.46857257808359e-05\\
764	-1.45742123338824e-05\\
765	-1.44632925501146e-05\\
766	-1.42432122605927e-05\\
767	-1.38099384121128e-05\\
768	-1.34908419348825e-05\\
769	-1.3490841934881e-05\\
770	-1.32808225252269e-05\\
771	-1.32808225252255e-05\\
772	-1.30729242920309e-05\\
773	-1.28671064834061e-05\\
774	-1.27649651159534e-05\\
775	-1.27649651159519e-05\\
776	-1.26633287582128e-05\\
777	-1.26633287582114e-05\\
778	-1.25621924300732e-05\\
779	-1.24615511757429e-05\\
780	-1.22617341867563e-05\\
781	-1.22617341867536e-05\\
782	-1.18685858088014e-05\\
783	-1.1675367377563e-05\\
784	-1.16753673775617e-05\\
785	-1.12954526665125e-05\\
786	-1.12954526665112e-05\\
787	-1.09239866368018e-05\\
788	-1.07413306097238e-05\\
789	-1.07413306097225e-05\\
790	-1.06507561147432e-05\\
791	-1.06507561147419e-05\\
792	-1.05606780398285e-05\\
793	-1.04710919710537e-05\\
794	-1.02933783168721e-05\\
795	-1.01175803196372e-05\\
796	-1.0117580319636e-05\\
797	-9.77290823288604e-06\\
798	-9.5209048441266e-06\\
799	-9.52090484412542e-06\\
800	-9.19335659052538e-06\\
801	-9.03313421889334e-06\\
802	-9.0331342188922e-06\\
803	-8.71963595975306e-06\\
804	-8.56629862537267e-06\\
805	-8.56629862537159e-06\\
806	-8.26629726846761e-06\\
807	-8.26629726846656e-06\\
808	-7.97499837083127e-06\\
809	-7.97499837083025e-06\\
810	-7.69217353920681e-06\\
811	-7.55386934669907e-06\\
812	-7.5538693466981e-06\\
813	-7.4176010242562e-06\\
814	-7.41760102425524e-06\\
815	-7.28349907203066e-06\\
816	-7.15169441461058e-06\\
817	-6.95823846590688e-06\\
818	-6.95823846590598e-06\\
819	-6.70810033377469e-06\\
820	-6.6469386986188e-06\\
821	-6.64693869861793e-06\\
822	-6.40769383803153e-06\\
823	-6.34921838904478e-06\\
824	-6.34921838904395e-06\\
825	-6.29127147678454e-06\\
826	-6.29127147678373e-06\\
827	-6.23385026189332e-06\\
828	-6.23385026189251e-06\\
829	-6.17695193066612e-06\\
830	-6.12057369507411e-06\\
831	-6.00936648596339e-06\\
832	-5.90020683751614e-06\\
833	-5.90020683751537e-06\\
834	-5.68835595601555e-06\\
835	-5.53522588918257e-06\\
836	-5.53522588918185e-06\\
837	-5.33860152695866e-06\\
838	-5.24348118236777e-06\\
839	-5.2434811823671e-06\\
840	-5.05953094337961e-06\\
841	-5.05953094337843e-06\\
842	-4.88384846585816e-06\\
843	-4.88384846585701e-06\\
844	-4.71629600574267e-06\\
845	-4.63552723297305e-06\\
846	-4.63552723297248e-06\\
847	-4.47887982618637e-06\\
848	-4.40270909690946e-06\\
849	-4.40270909690892e-06\\
850	-4.32795368402734e-06\\
851	-4.32795368402681e-06\\
852	-4.25459893528839e-06\\
853	-4.18263047287656e-06\\
854	-4.04279625178921e-06\\
855	-3.90834138206002e-06\\
856	-3.90834138205955e-06\\
857	-3.81096685718704e-06\\
858	-3.81096685718659e-06\\
859	-3.74767730526322e-06\\
860	-3.74767730526278e-06\\
861	-3.71651589275579e-06\\
862	-3.71651589275535e-06\\
863	-3.6856746814089e-06\\
864	-3.65515215999635e-06\\
865	-3.59505722007696e-06\\
866	-3.53621929415617e-06\\
867	-3.53621929415575e-06\\
868	-3.42215863432302e-06\\
869	-3.36688605426768e-06\\
870	-3.36688605426729e-06\\
871	-3.31277078635789e-06\\
872	-3.31277078635751e-06\\
873	-3.25980222386241e-06\\
874	-3.20796998473659e-06\\
875	-3.1572639096593e-06\\
876	-3.15726390965894e-06\\
877	-3.05919071640367e-06\\
878	-3.0118043754947e-06\\
879	-3.01180437549437e-06\\
880	-2.92028576405832e-06\\
881	-2.87613555552382e-06\\
882	-2.87613555552351e-06\\
883	-2.83284513793707e-06\\
884	-2.83284513793677e-06\\
885	-2.79020469373484e-06\\
886	-2.74820586522469e-06\\
887	-2.70684042047345e-06\\
888	-2.70684042047316e-06\\
889	-2.62597737390653e-06\\
890	-2.5864639226192e-06\\
891	-2.58646392261892e-06\\
892	-2.50923443881342e-06\\
893	-2.43435124846182e-06\\
894	-2.43435124846142e-06\\
895	-2.43435124846102e-06\\
896	-2.41599013891771e-06\\
897	-2.41599013891745e-06\\
898	-2.39777109495273e-06\\
899	-2.37969322362116e-06\\
900	-2.34395746245432e-06\\
901	-2.32629782154287e-06\\
902	-2.32629782154262e-06\\
903	-2.25702740992588e-06\\
904	-2.22320123860696e-06\\
905	-2.22320123860672e-06\\
906	-2.17355606186817e-06\\
907	-2.17355606186794e-06\\
908	-2.12528554065162e-06\\
909	-2.0938583533811e-06\\
910	-2.09385835338088e-06\\
911	-2.04783191295891e-06\\
912	-2.00312468166224e-06\\
913	-2.00312468166201e-06\\
914	-2.00312468166179e-06\\
915	-1.98851200435102e-06\\
916	-1.98851200435081e-06\\
917	-1.97404299441789e-06\\
918	-1.95971694281675e-06\\
919	-1.94553314756733e-06\\
920	-1.94553314756713e-06\\
921	-1.91758955302813e-06\\
922	-1.89020673376602e-06\\
923	-1.86337932264217e-06\\
924	-1.86337932264198e-06\\
925	-1.8108746123559e-06\\
926	-1.81087461235553e-06\\
927	-1.81087461235534e-06\\
928	-1.75953906842417e-06\\
929	-1.7342971208543e-06\\
930	-1.73429712085412e-06\\
931	-1.68464013609193e-06\\
932	-1.68464013609176e-06\\
933	-1.63605334338901e-06\\
934	-1.6121493325462e-06\\
935	-1.61214933254603e-06\\
936	-1.57676685289909e-06\\
937	-1.57676685289892e-06\\
938	-1.56509665470064e-06\\
939	-1.56509665470047e-06\\
940	-1.553487481758e-06\\
941	-1.54193876521626e-06\\
942	-1.51902044071195e-06\\
943	-1.4963371891074e-06\\
944	-1.49633718910724e-06\\
945	-1.45165816591769e-06\\
946	-1.44062843834867e-06\\
947	-1.44062843834851e-06\\
948	-1.39709761722863e-06\\
949	-1.35449992138319e-06\\
950	-1.34399236851035e-06\\
951	-1.3439923685102e-06\\
952	-1.29228012082052e-06\\
953	-1.29228012082037e-06\\
954	-1.28209922717746e-06\\
955	-1.28209922717731e-06\\
956	-1.27197101536964e-06\\
957	-1.2618949890911e-06\\
958	-1.26189498909096e-06\\
959	-1.24189752074501e-06\\
960	-1.2221029025715e-06\\
961	-1.20250725474878e-06\\
962	-1.20250725474864e-06\\
963	-1.1640287229794e-06\\
964	-1.16402872297914e-06\\
965	-1.14517108992208e-06\\
966	-1.14517108992195e-06\\
967	-1.1265629340214e-06\\
968	-1.10820060800662e-06\\
969	-1.09008051279067e-06\\
970	-1.09008051279054e-06\\
971	-1.05455285521203e-06\\
972	-1.01995210576965e-06\\
973	-9.94593480424472e-07\\
974	-9.94593480424353e-07\\
975	-9.69729782962207e-07\\
976	-9.69729782962091e-07\\
977	-9.45350048435924e-07\\
978	-9.21443524626113e-07\\
979	-9.21443524625894e-07\\
980	-9.21443524625673e-07\\
981	-8.97999668287964e-07\\
982	-8.97999668287854e-07\\
983	-8.75053552116595e-07\\
984	-8.52640468075433e-07\\
985	-8.52640468075328e-07\\
986	-8.16443101525268e-07\\
987	-8.16443101525167e-07\\
988	-7.81654671402262e-07\\
989	-7.7486227737189e-07\\
990	-7.74862277371794e-07\\
991	-7.41708460508551e-07\\
992	-7.35237053425477e-07\\
993	-7.35237053425385e-07\\
994	-7.16147748981336e-07\\
995	-7.16147748981247e-07\\
996	-6.97545267539713e-07\\
997	-6.85410010373826e-07\\
998	-6.85410010373741e-07\\
999	-6.67600773846542e-07\\
1000	-6.50256950296673e-07\\
1001	-6.50256950296515e-07\\
1002	-6.44577772981549e-07\\
1003	-6.44577772981469e-07\\
1004	-6.38949178422497e-07\\
1005	-6.3337089079341e-07\\
1006	-6.22364145427712e-07\\
1007	-6.00942474538823e-07\\
1008	-6.00942474538682e-07\\
1009	-6.00942474538541e-07\\
1010	-5.75253545681064e-07\\
1011	-5.75253545680993e-07\\
1012	-5.50744378397886e-07\\
1013	-5.45981448921479e-07\\
1014	-5.45981448921412e-07\\
1015	-5.41264270937806e-07\\
1016	-5.41264270937739e-07\\
1017	-5.36591460388822e-07\\
1018	-5.31961635370232e-07\\
1019	-5.273745690194e-07\\
1020	-5.27374569019336e-07\\
1021	-5.18327815337533e-07\\
1022	-5.09449426150442e-07\\
1023	-5.0073766125941e-07\\
1024	-5.00737661259349e-07\\
1025	-4.83807206584525e-07\\
1026	-4.83807206584465e-07\\
1027	-4.83807206584405e-07\\
1028	-4.67523176962907e-07\\
1029	-4.67523176962808e-07\\
1030	-4.59619562172117e-07\\
1031	-4.59619562172062e-07\\
1032	-4.51872804875926e-07\\
1033	-4.44281386679705e-07\\
1034	-4.36843819636052e-07\\
1035	-4.36843819635999e-07\\
1036	-4.22417270529958e-07\\
1037	-4.08574679174971e-07\\
1038	-4.0857467917489e-07\\
1039	-4.08574679174809e-07\\
1040	-3.92076139029517e-07\\
1041	-3.92076139029472e-07\\
1042	-3.88882094301436e-07\\
1043	-3.88882094301391e-07\\
1044	-3.85722901474031e-07\\
1045	-3.82598405739308e-07\\
1046	-3.7645289483677e-07\\
1047	-3.70444357049822e-07\\
1048	-3.70444357049779e-07\\
1049	-3.67491084004262e-07\\
1050	-3.6749108400422e-07\\
1051	-3.64571614691419e-07\\
1052	-3.61685806053969e-07\\
1053	-3.56014606826774e-07\\
1054	-3.50476374745519e-07\\
1055	-3.5047637474548e-07\\
1056	-3.47751586552093e-07\\
1057	-3.47751586552055e-07\\
1058	-3.45049245653866e-07\\
1059	-3.42369219633085e-07\\
1060	-3.37075588023497e-07\\
1061	-3.31869654150935e-07\\
1062	-3.31869654150898e-07\\
1063	-3.21716815105324e-07\\
1064	-3.11902742146659e-07\\
1065	-2.97800057981531e-07\\
1066	-2.97800057981499e-07\\
1067	-2.93260375010377e-07\\
1068	-2.93260375010345e-07\\
1069	-2.88799801928544e-07\\
1070	-2.84417464378762e-07\\
1071	-2.84417464378729e-07\\
1072	-2.84417464378698e-07\\
1073	-2.75884075232543e-07\\
1074	-2.7173135106169e-07\\
1075	-2.71731351061661e-07\\
1076	-2.6365531746971e-07\\
1077	-2.59731811004002e-07\\
1078	-2.59731811003975e-07\\
1079	-2.55883614430066e-07\\
1080	-2.55883614430039e-07\\
1081	-2.52109973491559e-07\\
1082	-2.48410148540172e-07\\
1083	-2.44783414395868e-07\\
1084	-2.44783414395842e-07\\
1085	-2.37746389321702e-07\\
1086	-2.36031721247582e-07\\
1087	-2.36031721247558e-07\\
1088	-2.29348881058384e-07\\
1089	-2.29348881058343e-07\\
1090	-2.26111817106046e-07\\
1091	-2.26111817106023e-07\\
1092	-2.22928827510215e-07\\
1093	-2.21351923118652e-07\\
1094	-2.21351923118629e-07\\
1095	-2.18226908801564e-07\\
1096	-2.15139777460543e-07\\
1097	-2.12089924005886e-07\\
1098	-2.12089924005865e-07\\
1099	-2.07583734189946e-07\\
1100	-2.07583734189925e-07\\
1101	-2.03158088998718e-07\\
1102	-1.98811036650114e-07\\
1103	-1.91735408499294e-07\\
1104	-1.91735408499275e-07\\
1105	-1.83513629856202e-07\\
1106	-1.8351362985618e-07\\
1107	-1.83513629856157e-07\\
1108	-1.78188734508626e-07\\
1109	-1.78188734508607e-07\\
1110	-1.75571599313201e-07\\
1111	-1.75571599313182e-07\\
1112	-1.72991476358216e-07\\
1113	-1.70455349769457e-07\\
1114	-1.67962722456277e-07\\
1115	-1.67962722456259e-07\\
1116	-1.65513105857513e-07\\
1117	-1.65513105857496e-07\\
1118	-1.63106019842063e-07\\
1119	-1.60740992611713e-07\\
1120	-1.58417560611964e-07\\
1121	-1.58417560611947e-07\\
1122	-1.5500940969396e-07\\
1123	-1.55009409693944e-07\\
1124	-1.51692322238845e-07\\
1125	-1.48464835356644e-07\\
1126	-1.48464835356626e-07\\
1127	-1.48464835356608e-07\\
1128	-1.43225225325629e-07\\
1129	-1.43225225325614e-07\\
1130	-1.38112690231254e-07\\
1131	-1.3811269023124e-07\\
1132	-1.37104878361819e-07\\
1133	-1.37104878361805e-07\\
1134	-1.36101849095579e-07\\
1135	-1.35103553281763e-07\\
1136	-1.3312096657413e-07\\
1137	-1.31156729644371e-07\\
1138	-1.31156729644357e-07\\
1139	-1.27281768658706e-07\\
1140	-1.25370285095737e-07\\
1141	-1.25370285095723e-07\\
1142	-1.23475632153339e-07\\
1143	-1.23475632153325e-07\\
1144	-1.21597438474833e-07\\
1145	-1.19735335927007e-07\\
1146	-1.17888959530602e-07\\
1147	-1.17888959530589e-07\\
1148	-1.1424194062674e-07\\
1149	-1.14241940626727e-07\\
1150	-1.10662852743448e-07\\
1151	-1.07158220123571e-07\\
1152	-1.0543296359934e-07\\
1153	-1.05432963599328e-07\\
1154	-1.00361385610016e-07\\
1155	-1.00361385610004e-07\\
1156	-9.95308817704226e-08\\
1157	-9.95308817704108e-08\\
1158	-9.87044856343912e-08\\
1159	-9.788215670494e-08\\
1160	-9.62495394859337e-08\\
1161	-9.54391711970167e-08\\
1162	-9.54391711970052e-08\\
1163	-9.22363758777221e-08\\
1164	-9.06576362891712e-08\\
1165	-9.06576362891601e-08\\
1166	-8.91008341433525e-08\\
1167	-8.91008341433416e-08\\
1168	-8.75729099284202e-08\\
1169	-8.60735641652866e-08\\
1170	-8.46025029763092e-08\\
1171	-8.46025029762989e-08\\
1172	-8.17440864827255e-08\\
1173	-7.89954193016209e-08\\
1174	-7.83251464120328e-08\\
1175	-7.83251464120233e-08\\
1176	-7.76615649058536e-08\\
1177	-7.76615649058442e-08\\
1178	-7.70046422748691e-08\\
1179	-7.63543463220908e-08\\
1180	-7.50735073157444e-08\\
1181	-7.50735073157281e-08\\
1182	-7.50735073157118e-08\\
1183	-7.25899689623508e-08\\
1184	-7.13867828330052e-08\\
1185	-7.13867828329967e-08\\
1186	-7.02090026227763e-08\\
1187	-7.02090026227681e-08\\
1188	-6.90524149336459e-08\\
1189	-6.79128105198424e-08\\
1190	-6.73493071099085e-08\\
1191	-6.73493071099005e-08\\
1192	-6.51366435477561e-08\\
1193	-6.35198073114671e-08\\
1194	-6.35198073114595e-08\\
1195	-6.14195518673429e-08\\
1196	-6.090423123367e-08\\
1197	-6.09042312336627e-08\\
1198	-6.03927571041919e-08\\
1199	-6.03927571041846e-08\\
1200	-5.98851044170427e-08\\
1201	-5.93812482966702e-08\\
1202	-5.8384827184874e-08\\
1203	-5.8384827184867e-08\\
1204	-5.74032984670194e-08\\
1205	-5.74032984670125e-08\\
1206	-5.64369952288924e-08\\
1207	-5.54862535405005e-08\\
1208	-5.45508870528139e-08\\
1209	-5.45508870528073e-08\\
1210	-5.36307124316243e-08\\
1211	-5.36307124316178e-08\\
1212	-5.27255493203731e-08\\
1213	-5.18352203035764e-08\\
1214	-5.09595508732905e-08\\
1215	-5.09595508732844e-08\\
1216	-4.92515070789121e-08\\
1217	-4.76000787427945e-08\\
1218	-4.7600078742784e-08\\
1219	-4.76000787427735e-08\\
1220	-4.52262697180054e-08\\
1221	-4.5226269718e-08\\
1222	-4.33371521238859e-08\\
1223	-4.33371521238806e-08\\
1224	-4.29684259535666e-08\\
1225	-4.29684259535613e-08\\
1226	-4.26026897307241e-08\\
1227	-4.22399255334169e-08\\
1228	-4.1523242258079e-08\\
1229	-4.08182356548415e-08\\
1230	-4.08182356548365e-08\\
1231	-3.94427019945113e-08\\
1232	-3.81122460630739e-08\\
1233	-3.81122460630458e-08\\
1234	-3.81122460630411e-08\\
1235	-3.71433575371429e-08\\
1236	-3.71433575371384e-08\\
1237	-3.61988104751906e-08\\
1238	-3.61988104751841e-08\\
1239	-3.5278188302255e-08\\
1240	-3.43810850090458e-08\\
1241	-3.43810850090406e-08\\
1242	-3.43810850090353e-08\\
1243	-3.26625036054325e-08\\
1244	-3.21126215778703e-08\\
1245	-3.21126215778664e-08\\
1246	-3.05304017833844e-08\\
1247	-3.05304017833799e-08\\
1248	-3.05304017833754e-08\\
1249	-2.90471733417579e-08\\
1250	-2.90471733417538e-08\\
1251	-2.90471733417496e-08\\
1252	-2.76603194533904e-08\\
1253	-2.76603194533865e-08\\
1254	-2.76603194533826e-08\\
1255	-2.7217379652368e-08\\
1256	-2.72173796523649e-08\\
1257	-2.67815000429602e-08\\
1258	-2.63525951892528e-08\\
1259	-2.59305810242426e-08\\
1260	-2.59305810242396e-08\\
1261	-2.51068952315779e-08\\
1262	-2.43097968528297e-08\\
1263	-2.411460285019e-08\\
1264	-2.41146028501873e-08\\
1265	-2.31626391432532e-08\\
1266	-2.31626391432506e-08\\
1267	-2.29769823588253e-08\\
1268	-2.29769823588201e-08\\
1269	-2.27928828202309e-08\\
1270	-2.26103315061639e-08\\
1271	-2.22498378468183e-08\\
1272	-2.20718778371068e-08\\
1273	-2.20718778371018e-08\\
1274	-2.13750806589733e-08\\
1275	-2.10355885404165e-08\\
1276	-2.10355885404117e-08\\
1277	-2.08681600833547e-08\\
1278	-2.086816008335e-08\\
1279	-2.07024261118696e-08\\
1280	-2.05383785048111e-08\\
1281	-2.02153103127361e-08\\
1282	-1.97431574631391e-08\\
1283	-1.97431574631347e-08\\
1284	-1.91365370702398e-08\\
1285	-1.88429142116733e-08\\
1286	-1.88429142116692e-08\\
1287	-1.82747554047517e-08\\
1288	-1.82747554047456e-08\\
1289	-1.77316766630693e-08\\
1290	-1.74694084946324e-08\\
1291	-1.74694084946287e-08\\
1292	-1.6957670134545e-08\\
1293	-1.69576701345384e-08\\
1294	-1.64590043940177e-08\\
1295	-1.62144511828978e-08\\
1296	-1.62144511828944e-08\\
1297	-1.59730202932478e-08\\
1298	-1.59730202932443e-08\\
1299	-1.57346644039945e-08\\
1300	-1.54993367964626e-08\\
1301	-1.50375825112089e-08\\
1302	-1.46988772047929e-08\\
1303	-1.46988772047897e-08\\
1304	-1.45873953924117e-08\\
1305	-1.45873953924086e-08\\
1306	-1.44766144260293e-08\\
1307	-1.43665288761049e-08\\
1308	-1.41484224825758e-08\\
1309	-1.40403909516116e-08\\
1310	-1.40403909516085e-08\\
1311	-1.39330334620676e-08\\
1312	-1.39330334620645e-08\\
1313	-1.38263447535004e-08\\
1314	-1.37203195980469e-08\\
1315	-1.35102391977167e-08\\
1316	-1.34061736587596e-08\\
1317	-1.34061736587566e-08\\
1318	-1.29961670106311e-08\\
1319	-1.27948282383822e-08\\
1320	-1.27948282383794e-08\\
1321	-1.24972896109556e-08\\
1322	-1.24972896109528e-08\\
1323	-1.22049979556526e-08\\
1324	-1.19178243670625e-08\\
1325	-1.1917824367058e-08\\
1326	-1.17291560439996e-08\\
1327	-1.17291560439969e-08\\
1328	-1.15426691051607e-08\\
1329	-1.13583269976241e-08\\
1330	-1.11760935896177e-08\\
1331	-1.11760935896151e-08\\
1332	-1.08186796305074e-08\\
1333	-1.06436463204004e-08\\
1334	-1.0643646320398e-08\\
1335	-1.03007554974024e-08\\
1336	-1.01328307765728e-08\\
1337	-1.01328307765704e-08\\
1338	-9.80385814135552e-09\\
1339	-9.64274574704631e-09\\
1340	-9.64274574704404e-09\\
1341	-9.3271109065708e-09\\
1342	-9.24955250837899e-09\\
1343	-9.24955250837679e-09\\
1344	-9.02005714984252e-09\\
1345	-9.02005714984037e-09\\
1346	-8.79525309737646e-09\\
1347	-8.57504120870862e-09\\
1348	-8.5750412087051e-09\\
1349	-8.35932436608786e-09\\
1350	-8.35932436608584e-09\\
1351	-8.14831987337215e-09\\
1352	-7.94224711040477e-09\\
1353	-7.94224711040248e-09\\
1354	-7.94224711040019e-09\\
1355	-7.60950603390769e-09\\
1356	-7.60950603390584e-09\\
1357	-7.28980498755184e-09\\
1358	-7.22739483875215e-09\\
1359	-7.22739483875038e-09\\
1360	-7.16548751651524e-09\\
1361	-7.16548751651348e-09\\
1362	-7.10407998750108e-09\\
1363	-7.04316924259715e-09\\
1364	-6.9228261907563e-09\\
1365	-6.80443477328446e-09\\
1366	-6.80443477328279e-09\\
1367	-6.57376022468586e-09\\
1368	-6.40616154117094e-09\\
1369	-6.40616154116937e-09\\
1370	-6.18976573636064e-09\\
1371	-6.08454962395508e-09\\
1372	-6.0845496239536e-09\\
1373	-5.87997786191998e-09\\
1374	-5.87997786191764e-09\\
1375	-5.68308702417037e-09\\
1376	-5.63504623800271e-09\\
1377	-5.63504623800135e-09\\
1378	-5.44754016342283e-09\\
1379	-5.40181642747196e-09\\
1380	-5.40181642747067e-09\\
1381	-5.22346539767049e-09\\
1382	-5.13698572108021e-09\\
1383	-5.136985721079e-09\\
1384	-5.09441236909743e-09\\
1385	-5.09441236909622e-09\\
1386	-5.05224649303867e-09\\
1387	-5.05224649303748e-09\\
1388	-5.01045196711214e-09\\
1389	-4.96902674337888e-09\\
1390	-4.88727610110798e-09\\
1391	-4.72811832998898e-09\\
1392	-4.5746483901216e-09\\
1393	-4.57464839012053e-09\\
1394	-4.42674595504692e-09\\
1395	-4.42674595504589e-09\\
1396	-4.3906265852502e-09\\
1397	-4.39062658524918e-09\\
1398	-4.35484614611123e-09\\
1399	-4.35484614610972e-09\\
1400	-4.31940288434744e-09\\
1401	-4.28429506323391e-09\\
1402	-4.2150788781444e-09\\
1403	-4.14718402419941e-09\\
1404	-4.14718402419846e-09\\
1405	-4.01523781093887e-09\\
1406	-3.9511437092303e-09\\
1407	-3.9511437092294e-09\\
1408	-3.88828544574074e-09\\
1409	-3.88828544573985e-09\\
1410	-3.82665070008822e-09\\
1411	-3.76622739162316e-09\\
1412	-3.70700367713569e-09\\
1413	-3.70700367713485e-09\\
1414	-3.59210883102941e-09\\
1415	-3.5090019835504e-09\\
1416	-3.50900198354963e-09\\
1417	-3.40220888502965e-09\\
1418	-3.35050988073183e-09\\
1419	-3.3505098807311e-09\\
1420	-3.32503858109399e-09\\
1421	-3.32503858109327e-09\\
1422	-3.2997621294787e-09\\
1423	-3.27467928731174e-09\\
1424	-3.22508952449019e-09\\
1425	-3.17625957284491e-09\\
1426	-3.17625957284423e-09\\
1427	-3.0808409669833e-09\\
1428	-3.03423361035298e-09\\
1429	-3.03423361035232e-09\\
1430	-2.94317711206793e-09\\
1431	-2.85493897376965e-09\\
1432	-2.85493897376903e-09\\
1433	-2.81185508500588e-09\\
1434	-2.81185508500528e-09\\
1435	-2.76945001220587e-09\\
1436	-2.72771544381106e-09\\
1437	-2.72771544381033e-09\\
1438	-2.7277154438096e-09\\
1439	-2.64622522961595e-09\\
1440	-2.60645361144537e-09\\
1441	-2.6064536114448e-09\\
1442	-2.5480754855516e-09\\
1443	-2.54807548555105e-09\\
1444	-2.49127721465512e-09\\
1445	-2.45427699974086e-09\\
1446	-2.45427699974033e-09\\
1447	-2.4000564951968e-09\\
1448	-2.34735056686552e-09\\
1449	-2.34735056686401e-09\\
1450	-2.34735056686351e-09\\
1451	-2.33011445995248e-09\\
1452	-2.33011445995199e-09\\
1453	-2.31304320668827e-09\\
1454	-2.29613597050458e-09\\
1455	-2.26281024354173e-09\\
1456	-2.26281024354096e-09\\
1457	-2.19809107593592e-09\\
1458	-2.16668495010748e-09\\
1459	-2.16668495010704e-09\\
1460	-2.12047063806669e-09\\
1461	-2.12047063806626e-09\\
1462	-2.07508842244407e-09\\
1463	-2.0305182889387e-09\\
1464	-2.03051828893801e-09\\
1465	-1.97232075992021e-09\\
1466	-1.9723207599198e-09\\
1467	-1.91548639869764e-09\\
1468	-1.8875664398366e-09\\
1469	-1.88756643983621e-09\\
1470	-1.83269360261039e-09\\
1471	-1.83269360261001e-09\\
1472	-1.7790744582363e-09\\
1473	-1.75272184581226e-09\\
1474	-1.75272184581188e-09\\
1475	-1.70090471356611e-09\\
1476	-1.67543003738564e-09\\
1477	-1.67543003738528e-09\\
1478	-1.62564104710781e-09\\
1479	-1.57757850727103e-09\\
1480	-1.57757850727027e-09\\
1481	-1.57757850726994e-09\\
1482	-1.51987087104506e-09\\
1483	-1.51987087104473e-09\\
1484	-1.5086397218698e-09\\
1485	-1.50863972186949e-09\\
1486	-1.49751073617442e-09\\
1487	-1.48648336861221e-09\\
1488	-1.47555707884112e-09\\
1489	-1.47555707884081e-09\\
1490	-1.45400559602965e-09\\
1491	-1.4328520635825e-09\\
1492	-1.41209233532746e-09\\
1493	-1.41209233532717e-09\\
1494	-1.38148800108238e-09\\
1495	-1.38148800108209e-09\\
1496	-1.36134978915803e-09\\
1497	-1.36134978915775e-09\\
1498	-1.34141870825641e-09\\
1499	-1.32169085182004e-09\\
1500	-1.30216235311065e-09\\
1501	-1.30216235311038e-09\\
1502	-1.26368815667936e-09\\
1503	-1.22596595320688e-09\\
1504	-1.188966166152e-09\\
1505	-1.18896616615174e-09\\
1506	-1.17072809182309e-09\\
1507	-1.17072809182283e-09\\
1508	-1.15265978632628e-09\\
1509	-1.13475770794173e-09\\
1510	-1.1170183477925e-09\\
1511	-1.11701834779225e-09\\
1512	-1.08201390558644e-09\\
1513	-1.08201390558604e-09\\
1514	-1.0477011583532e-09\\
1515	-1.03082654402201e-09\\
1516	-1.03082654402177e-09\\
1517	-9.97624295936824e-10\\
1518	-9.9762429593659e-10\\
1519	-9.65129720981008e-10\\
1520	-9.49139828099996e-10\\
1521	-9.4913982809977e-10\\
1522	-9.41207853686677e-10\\
1523	-9.41207853686453e-10\\
1524	-9.33317341725171e-10\\
1525	-9.25467905571892e-10\\
1526	-9.09890724187453e-10\\
1527	-8.94473256345259e-10\\
1528	-8.94473256345041e-10\\
1529	-8.79243402562226e-10\\
1530	-8.79243402562011e-10\\
1531	-8.64229100127463e-10\\
1532	-8.49427406179605e-10\\
1533	-8.42105379150392e-10\\
1534	-8.42105379150185e-10\\
1535	-8.13334395457072e-10\\
1536	-7.99254252777932e-10\\
1537	-7.99254252777733e-10\\
1538	-7.92289304838051e-10\\
1539	-7.92289304837854e-10\\
1540	-7.85373983310724e-10\\
1541	-7.78507949337657e-10\\
1542	-7.64922400704173e-10\\
1543	-7.38328110650226e-10\\
1544	-7.31797802427922e-10\\
1545	-7.31797802427737e-10\\
1546	-7.06140026998827e-10\\
1547	-7.06140026998647e-10\\
1548	-6.81208777546781e-10\\
1549	-6.69009469560939e-10\\
1550	-6.69009469560767e-10\\
1551	-6.56984506389637e-10\\
1552	-6.56984506389467e-10\\
1553	-6.45146786155684e-10\\
1554	-6.45146786155517e-10\\
1555	-6.33509243666346e-10\\
1556	-6.220695979262e-10\\
1557	-6.10825606719827e-10\\
1558	-6.10825606719669e-10\\
1559	-5.88915810432843e-10\\
1560	-5.8891581043269e-10\\
1561	-5.67762676394512e-10\\
1562	-5.62590664937169e-10\\
1563	-5.62590664937023e-10\\
1564	-5.57464652036668e-10\\
1565	-5.57464652036523e-10\\
1566	-5.5238438652194e-10\\
1567	-5.47349619454327e-10\\
1568	-5.37415596058988e-10\\
1569	-5.27660634743381e-10\\
1570	-5.27660634743244e-10\\
1571	-5.08701837993906e-10\\
1572	-4.90479917432525e-10\\
1573	-4.90479917432237e-10\\
1574	-4.9047991743211e-10\\
1575	-4.68717037588441e-10\\
1576	-4.6871703758832e-10\\
1577	-4.64497581859411e-10\\
1578	-4.64497581859292e-10\\
1579	-4.60322012063178e-10\\
1580	-4.56190123586914e-10\\
1581	-4.48056582870958e-10\\
1582	-4.40095365505868e-10\\
1583	-4.40095365505757e-10\\
1584	-4.32304911073312e-10\\
1585	-4.32304911073202e-10\\
1586	-4.24683692624427e-10\\
1587	-4.17230216370744e-10\\
1588	-4.09943021402142e-10\\
1589	-4.09943021402039e-10\\
1590	-3.95801568461802e-10\\
1591	-3.88929482963257e-10\\
1592	-3.8892948296316e-10\\
1593	-3.75575858910049e-10\\
1594	-3.62734281577958e-10\\
1595	-3.44410086208213e-10\\
1596	-3.44410086208129e-10\\
1597	-3.3566138821935e-10\\
1598	-3.35661388219268e-10\\
1599	-3.27183105482373e-10\\
1600	-3.27183105482271e-10\\
1601	-3.18971498804839e-10\\
1602	-3.11022946735238e-10\\
1603	-3.11022946735153e-10\\
1604	-3.11022946735068e-10\\
1605	-2.9588286600721e-10\\
1606	-2.95882866007129e-10\\
1607	-2.9588286600705e-10\\
1608	-2.91054137892289e-10\\
1609	-2.91054137892221e-10\\
1610	-2.86332797406583e-10\\
1611	-2.81717919128634e-10\\
1612	-2.77208598524297e-10\\
1613	-2.77208598524234e-10\\
1614	-2.68503115503731e-10\\
1615	-2.64305246778778e-10\\
1616	-2.64305246778719e-10\\
1617	-2.58199713432034e-10\\
1618	-2.58199713431977e-10\\
1619	-2.52321317798459e-10\\
1620	-2.52321317798405e-10\\
1621	-2.48507762119094e-10\\
1622	-2.4850776211904e-10\\
1623	-2.44754197708056e-10\\
1624	-2.41059888854356e-10\\
1625	-2.37424111459042e-10\\
1626	-2.3742411145899e-10\\
1627	-2.30325311892969e-10\\
1628	-2.30325311892911e-10\\
1629	-2.30325311892853e-10\\
1630	-2.23452233184687e-10\\
1631	-2.16799486488609e-10\\
1632	-2.13554100930566e-10\\
1633	-2.13554100930521e-10\\
1634	-2.11951375096382e-10\\
1635	-2.11951375096337e-10\\
1636	-2.10361855643429e-10\\
1637	-2.08785464666732e-10\\
1638	-2.05671759807589e-10\\
1639	-2.04134293351497e-10\\
1640	-2.04134293351454e-10\\
1641	-1.98111916818838e-10\\
1642	-1.98111916818768e-10\\
1643	-1.95176191448851e-10\\
1644	-1.95176191448809e-10\\
1645	-1.92292564224302e-10\\
1646	-1.89463029977122e-10\\
1647	-1.86687034107568e-10\\
1648	-1.86687034107528e-10\\
1649	-1.83964032513601e-10\\
1650	-1.83964032513562e-10\\
1651	-1.81293491479945e-10\\
1652	-1.78674887570187e-10\\
1653	-1.76107707527744e-10\\
1654	-1.76107707527708e-10\\
1655	-1.71125616332435e-10\\
1656	-1.71125616332374e-10\\
1657	-1.66343311677438e-10\\
1658	-1.64025901522118e-10\\
1659	-1.64025901522085e-10\\
1660	-1.59490406701906e-10\\
1661	-1.59490406701871e-10\\
1662	-1.59490406701836e-10\\
1663	-1.55053800476911e-10\\
1664	-1.53959690958384e-10\\
1665	-1.53959690958353e-10\\
1666	-1.52871490592379e-10\\
1667	-1.52871490592348e-10\\
1668	-1.51789146057808e-10\\
1669	-1.50712604318615e-10\\
1670	-1.48576718504981e-10\\
1671	-1.46463414509556e-10\\
1672	-1.46463414509526e-10\\
1673	-1.42302899468046e-10\\
1674	-1.39238741128887e-10\\
1675	-1.39238741128858e-10\\
1676	-1.38227797134185e-10\\
1677	-1.38227797134156e-10\\
1678	-1.37221991426941e-10\\
1679	-1.36221274721581e-10\\
1680	-1.34234912422112e-10\\
1681	-1.32268320901048e-10\\
1682	-1.3226832090102e-10\\
1683	-1.2839291216746e-10\\
1684	-1.27435819024646e-10\\
1685	-1.27435819024618e-10\\
1686	-1.23656305061751e-10\\
1687	-1.19954042790006e-10\\
1688	-1.1904020632283e-10\\
1689	-1.19040206322804e-10\\
1690	-1.13652272640948e-10\\
1691	-1.13652272640923e-10\\
1692	-1.11891462909331e-10\\
1693	-1.11891462909306e-10\\
1694	-1.10147659215674e-10\\
1695	-1.08420519760689e-10\\
1696	-1.08420519760658e-10\\
1697	-1.08420519760627e-10\\
1698	-1.0501488266705e-10\\
1699	-1.03335717511515e-10\\
1700	-1.03335717511491e-10\\
1701	-1.01678081316763e-10\\
1702	-1.0167808131674e-10\\
1703	-1.00047849066362e-10\\
1704	-9.8444701228471e-11\\
1705	-9.68683235798675e-11\\
1706	-9.68683235798453e-11\\
1707	-9.37946481422319e-11\\
1708	-9.08244128333673e-11\\
1709	-9.00977418540617e-11\\
1710	-9.00977418540411e-11\\
1711	-8.86632149799054e-11\\
1712	-8.86632149798852e-11\\
1713	-8.72535411158407e-11\\
1714	-8.65579375214488e-11\\
1715	-8.65579375214291e-11\\
1716	-8.51850265828549e-11\\
1717	-8.38362868959798e-11\\
1718	-8.25114541025741e-11\\
1719	-8.25114541025554e-11\\
1720	-8.05684635779014e-11\\
1721	-8.05684635778833e-11\\
1722	-7.86711452059434e-11\\
1723	-7.80472259386162e-11\\
1724	-7.80472259385985e-11\\
1725	-7.62006006868876e-11\\
1726	-7.43910429939378e-11\\
1727	-7.37959559313848e-11\\
1728	-7.37959559313679e-11\\
1729	-7.03084403914061e-11\\
1730	-7.03084403913706e-11\\
1731	-7.03084403913545e-11\\
1732	-6.80604059761885e-11\\
1733	-6.80604059761728e-11\\
1734	-6.69588161304778e-11\\
1735	-6.69588161304623e-11\\
1736	-6.58723788514433e-11\\
1737	-6.48013710442619e-11\\
1738	-6.37455827872648e-11\\
1739	-6.37455827872499e-11\\
1740	-6.27048071432907e-11\\
1741	-6.2704807143276e-11\\
1742	-6.16788401176581e-11\\
1743	-6.06674806168605e-11\\
1744	-5.96705304104633e-11\\
1745	-5.96705304104492e-11\\
1746	-5.77190790487262e-11\\
1747	-5.58229559938976e-11\\
1748	-5.53573951187365e-11\\
1749	-5.53573951187233e-11\\
1750	-5.30792714771791e-11\\
1751	-5.30792714771664e-11\\
1752	-5.08819171637348e-11\\
1753	-5.08819171637226e-11\\
1754	-5.04519000145789e-11\\
1755	-5.04519000145667e-11\\
1756	-5.00249845805003e-11\\
1757	-5.00249845804882e-11\\
1758	-4.96012923640465e-11\\
1759	-4.91809450266339e-11\\
1760	-4.8350202763372e-11\\
1761	-4.75325949619976e-11\\
1762	-4.75325949619861e-11\\
1763	-4.5936144275735e-11\\
1764	-4.4772110750182e-11\\
1765	-4.47721107501711e-11\\
1766	-4.32634959983698e-11\\
1767	-4.32634959983543e-11\\
1768	-4.25274681208067e-11\\
1769	-4.25274681207963e-11\\
1770	-4.18034334315656e-11\\
1771	-4.10912500170369e-11\\
1772	-4.03907782865305e-11\\
1773	-4.03907782865206e-11\\
1774	-3.90275324263901e-11\\
1775	-3.77157364528605e-11\\
1776	-3.77157364528514e-11\\
1777	-3.64543618506334e-11\\
1778	-3.5842271983563e-11\\
1779	-3.58422719835544e-11\\
1780	-3.46546872377297e-11\\
1781	-3.4078959587776e-11\\
1782	-3.4078959587768e-11\\
1783	-3.37955621886837e-11\\
1784	-3.37955621886757e-11\\
1785	-3.35151238432256e-11\\
1786	-3.32376308096137e-11\\
1787	-3.26914264327023e-11\\
1788	-3.21568419997632e-11\\
1789	-3.21568419997557e-11\\
1790	-3.18935331083627e-11\\
1791	-3.18935331083553e-11\\
1792	-3.16324074272388e-11\\
1793	-3.13734521609446e-11\\
1794	-3.08620022231127e-11\\
1795	-3.03590830486882e-11\\
1796	-3.03590830486811e-11\\
1797	-2.9378444349466e-11\\
1798	-2.84307671919571e-11\\
1799	-2.81989105078471e-11\\
1800	-2.81989105078405e-11\\
1801	-2.70694358863701e-11\\
1802	-2.70694358863639e-11\\
1803	-2.66313506422189e-11\\
1804	-2.66313506422128e-11\\
1805	-2.62009669414929e-11\\
1806	-2.57782004266896e-11\\
1807	-2.57782004266735e-11\\
1808	-2.57782004266676e-11\\
1809	-2.55696476610475e-11\\
1810	-2.55696476610416e-11\\
1811	-2.53629682339943e-11\\
1812	-2.51581520179943e-11\\
1813	-2.47540691658004e-11\\
1814	-2.43573199026532e-11\\
1815	-2.43573199026476e-11\\
1816	-2.35876581370586e-11\\
1817	-2.30319114378984e-11\\
1818	-2.30319114378933e-11\\
1819	-2.2319116014759e-11\\
1820	-2.19746464756795e-11\\
1821	-2.19746464756747e-11\\
1822	-2.1309226144578e-11\\
1823	-2.13092261445701e-11\\
1824	-2.11477231109902e-11\\
1825	-2.11477231109856e-11\\
1826	-2.09881449276362e-11\\
1827	-2.08304837750047e-11\\
1828	-2.05208817537397e-11\\
1829	-2.02188563639917e-11\\
1830	-2.02188563639874e-11\\
1831	-1.97759646024369e-11\\
1832	-1.97759646024328e-11\\
1833	-1.93419305511428e-11\\
1834	-1.89165627939782e-11\\
1835	-1.89165627939722e-11\\
1836	-1.86377061019167e-11\\
1837	-1.86377061019127e-11\\
1838	-1.83625629885689e-11\\
1839	-1.80910795236882e-11\\
1840	-1.75588793996783e-11\\
1841	-1.70406884520758e-11\\
1842	-1.70406884520722e-11\\
1843	-1.6412032248061e-11\\
1844	-1.64120322480575e-11\\
1845	-1.6288783420559e-11\\
1846	-1.62887834205555e-11\\
1847	-1.61663478786139e-11\\
1848	-1.60447196226005e-11\\
1849	-1.58038611683885e-11\\
1850	-1.55661608487291e-11\\
1851	-1.55661608487257e-11\\
1852	-1.51022962082032e-11\\
1853	-1.48766027766688e-11\\
1854	-1.48766027766656e-11\\
1855	-1.46550093514556e-11\\
1856	-1.46550093514525e-11\\
1857	-1.4437472499687e-11\\
1858	-1.42239495832958e-11\\
1859	-1.40143987509569e-11\\
1860	-1.40143987509539e-11\\
1861	-1.37074305249999e-11\\
1862	-1.3707430524997e-11\\
1863	-1.3409171877743e-11\\
1864	-1.3119491272232e-11\\
1865	-1.31194912722226e-11\\
1866	-1.31194912722199e-11\\
1867	-1.26499582301345e-11\\
1868	-1.26499582301319e-11\\
1869	-1.21924264059038e-11\\
1870	-1.21023106090409e-11\\
1871	-1.21023106090383e-11\\
1872	-1.165844158684e-11\\
1873	-1.15709793252031e-11\\
1874	-1.15709793252006e-11\\
1875	-1.11399900049709e-11\\
1876	-1.10550269724475e-11\\
1877	-1.1055026972445e-11\\
1878	-1.08863023130504e-11\\
1879	-1.0886302313048e-11\\
1880	-1.07191520019713e-11\\
1881	-1.05535432768704e-11\\
1882	-1.03894436777879e-11\\
1883	-1.03894436777856e-11\\
1884	-1.00656434915668e-11\\
1885	-1.00656434915595e-11\\
1886	-1.00656434915573e-11\\
1887	-9.74819873334464e-12\\
1888	-9.66991085063486e-12\\
1889	-9.66991085063264e-12\\
1890	-9.36093745905984e-12\\
1891	-9.2847197212227e-12\\
1892	-9.28471972122053e-12\\
1893	-8.98383916524345e-12\\
1894	-8.8357415905065e-12\\
1895	-8.83574159050441e-12\\
1896	-8.76226572963944e-12\\
1897	-8.76226572963736e-12\\
1898	-8.68916701318394e-12\\
1899	-8.6164418592264e-12\\
1900	-8.4720980027406e-12\\
1901	-8.32920586822312e-12\\
1902	-8.3292058682211e-12\\
1903	-8.04882457266744e-12\\
1904	-7.84366547942586e-12\\
1905	-7.84366547942394e-12\\
1906	-7.57678383474065e-12\\
1907	-7.44613997929631e-12\\
1908	-7.44613997929447e-12\\
1909	-7.19031816447464e-12\\
1910	-7.19031816447159e-12\\
1911	-6.94161765599174e-12\\
1912	-6.81987674204591e-12\\
1913	-6.8198767420442e-12\\
1914	-6.58149428235554e-12\\
1915	-6.58149428235225e-12\\
1916	-6.34975577989442e-12\\
1917	-6.23632103334458e-12\\
1918	-6.23632103334298e-12\\
1919	-6.12447953965855e-12\\
1920	-6.12447953965698e-12\\
1921	-6.01435762815523e-12\\
1922	-5.90608196505215e-12\\
1923	-5.69498485224775e-12\\
1924	-5.49102268801804e-12\\
1925	-5.49102268801662e-12\\
1926	-5.24586040784268e-12\\
1927	-5.24586040784131e-12\\
1928	-5.19810915700904e-12\\
1929	-5.19810915700769e-12\\
1930	-5.15077946638651e-12\\
1931	-5.1038690167082e-12\\
1932	-5.01129666612474e-12\\
1933	-4.9203739607151e-12\\
1934	-4.92037396071382e-12\\
1935	-4.74363105250425e-12\\
1936	-4.65783233751927e-12\\
1937	-4.65783233751806e-12\\
1938	-4.57372624787792e-12\\
1939	-4.57372624787674e-12\\
1940	-4.49129629857033e-12\\
1941	-4.41052633300613e-12\\
1942	-4.33140051995588e-12\\
1943	-4.33140051995477e-12\\
1944	-4.17801963525039e-12\\
1945	-4.03103338592601e-12\\
1946	-4.03103338592465e-12\\
1947	-4.0310333859233e-12\\
1948	-3.82229320334724e-12\\
1949	-3.82229320334628e-12\\
1950	-3.62608712734713e-12\\
1951	-3.62608712734623e-12\\
1952	-3.44084159370827e-12\\
1953	-3.44084159370742e-12\\
1954	-3.26622978127088e-12\\
1955	-3.21033511783159e-12\\
1956	-3.21033511783081e-12\\
1957	-3.12862011676307e-12\\
1958	-3.1286201167623e-12\\
1959	-3.0494255584865e-12\\
1960	-3.04942555848534e-12\\
1961	-3.0494255584846e-12\\
1962	-2.9727165166237e-12\\
1963	-2.89845916124697e-12\\
1964	-2.89845916124607e-12\\
1965	-2.89845916124518e-12\\
1966	-2.82658241870818e-12\\
1967	-2.82658241870751e-12\\
1968	-2.75701626601513e-12\\
1969	-2.71190729971945e-12\\
1970	-2.71190729971882e-12\\
1971	-2.64612444940945e-12\\
1972	-2.58257260379649e-12\\
1973	-2.58257260379534e-12\\
1974	-2.58257260379417e-12\\
1975	-2.5618795616458e-12\\
1976	-2.56187956164522e-12\\
1977	-2.5414302728821e-12\\
1978	-2.52122373539655e-12\\
1979	-2.48153496560698e-12\\
1980	-2.46205078852823e-12\\
1981	-2.46205078852768e-12\\
1982	-2.38649332495359e-12\\
1983	-2.35012922236835e-12\\
1984	-2.35012922236784e-12\\
1985	-2.314516599372e-12\\
1986	-2.3145165993715e-12\\
1987	-2.27946643400058e-12\\
1988	-2.2449718562859e-12\\
1989	-2.21102610515666e-12\\
1990	-2.21102610515619e-12\\
1991	-2.14475457513491e-12\\
1992	-2.14475457513425e-12\\
1993	-2.08059988363723e-12\\
1994	-2.04930056964218e-12\\
1995	-2.04930056964174e-12\\
1996	-1.98822733021419e-12\\
1997	-1.98822733021358e-12\\
1998	-1.97327243637491e-12\\
1999	-1.97327243637449e-12\\
2000	-1.9584414341952e-12\\
2001	-1.94373359693927e-12\\
2002	-1.91468454044941e-12\\
2003	-1.90034189779036e-12\\
2004	-1.90034189778996e-12\\
2005	-1.84416756786716e-12\\
2006	-1.81678853368838e-12\\
2007	-1.816788533688e-12\\
2008	-1.76351374942234e-12\\
2009	-1.71224460211712e-12\\
2010	-1.71224460211394e-12\\
2011	-1.71224460211358e-12\\
2012	-1.65091750310505e-12\\
2013	-1.65091750310471e-12\\
2014	-1.63901393987906e-12\\
2015	-1.63901393987873e-12\\
2016	-1.62722962079605e-12\\
2017	-1.6155639683963e-12\\
2018	-1.59258638295502e-12\\
2019	-1.59258638295441e-12\\
2020	-1.5925863829538e-12\\
2021	-1.54803044718028e-12\\
2022	-1.52644336372526e-12\\
2023	-1.52644336372496e-12\\
2024	-1.50520358369057e-12\\
2025	-1.50520358369027e-12\\
2026	-1.48419932932466e-12\\
2027	-1.48419932932436e-12\\
2028	-1.463426483712e-12\\
2029	-1.44288097530863e-12\\
2030	-1.4225587771143e-12\\
2031	-1.42255877711401e-12\\
2032	-1.38256842155084e-12\\
2033	-1.34342406251998e-12\\
2034	-1.29563660113993e-12\\
2035	-1.29563660113966e-12\\
2036	-1.2862268055264e-12\\
2037	-1.28622680552614e-12\\
2038	-1.27686516080441e-12\\
2039	-1.26755120814657e-12\\
2040	-1.24906455579493e-12\\
2041	-1.23076322501534e-12\\
2042	-1.23076322501508e-12\\
2043	-1.19470221537947e-12\\
2044	-1.18579722704082e-12\\
2045	-1.18579722704057e-12\\
2046	-1.15063342051144e-12\\
2047	-1.13332340948752e-12\\
2048	-1.13332340948728e-12\\
2049	-1.10768870340634e-12\\
2050	-1.1076887034061e-12\\
2051	-1.0824403335762e-12\\
2052	-1.05756716505222e-12\\
2053	-1.05756716505167e-12\\
2054	-1.05756716505144e-12\\
2055	-1.04118807411113e-12\\
2056	-1.0411880741109e-12\\
2057	-1.02496765040343e-12\\
2058	-1.00890271459748e-12\\
2059	-9.92990117903129e-13\\
2060	-9.92990117902904e-13\\
2061	-9.85096867423981e-13\\
2062	-9.85096867423758e-13\\
2063	-9.77254327797006e-13\\
2064	-9.69462114729095e-13\\
2065	-9.54027143437104e-13\\
2066	-9.46383628888793e-13\\
2067	-9.46383628888577e-13\\
2068	-9.16294000041679e-13\\
2069	-9.01534691024276e-13\\
2070	-9.01534691024067e-13\\
2071	-8.94225141162788e-13\\
2072	-8.94225141162581e-13\\
2073	-8.86961849420162e-13\\
2074	-8.7974445988708e-13\\
2075	-8.65445975066691e-13\\
2076	-8.37384419757946e-13\\
2077	-8.23615848907968e-13\\
2078	-8.23615848907774e-13\\
2079	-8.03283149260644e-13\\
2080	-8.03283149260453e-13\\
2081	-7.83326679624501e-13\\
2082	-7.63737638578494e-13\\
2083	-7.63737638578238e-13\\
2084	-7.44507386972403e-13\\
2085	-7.44507386972222e-13\\
2086	-7.381801444034e-13\\
2087	-7.38180144403221e-13\\
2088	-7.31900674893263e-13\\
2089	-7.25668670745078e-13\\
2090	-7.13345839367081e-13\\
2091	-7.01209234946281e-13\\
2092	-7.0120923494611e-13\\
2093	-6.77485227800774e-13\\
2094	-6.77485227800517e-13\\
2095	-6.77485227800351e-13\\
2096	-6.54478048462734e-13\\
2097	-6.43237610522232e-13\\
2098	-6.43237610522073e-13\\
2099	-6.2127202188811e-13\\
2100	-6.1054256585329e-13\\
2101	-6.10542565853138e-13\\
2102	-5.89604706977556e-13\\
2103	-5.69364557178409e-13\\
2104	-5.6936455717778e-13\\
2105	-5.69364557177638e-13\\
2106	-5.45021398316581e-13\\
2107	-5.45021398316446e-13\\
2108	-5.40277969206347e-13\\
2109	-5.40277969206212e-13\\
2110	-5.35575727232871e-13\\
2111	-5.30914441975173e-13\\
2112	-5.21713829986766e-13\\
2113	-5.1267432988564e-13\\
2114	-5.12674329885512e-13\\
2115	-4.99413297340533e-13\\
2116	-4.99413297340409e-13\\
2117	-4.86504939062543e-13\\
2118	-4.78092479442034e-13\\
2119	-4.78092479441915e-13\\
2120	-4.65759336617874e-13\\
2121	-4.53764026116269e-13\\
2122	-4.53764026116122e-13\\
2123	-4.53764026115977e-13\\
2124	-4.49839744525567e-13\\
2125	-4.49839744525456e-13\\
2126	-4.45951611499564e-13\\
2127	-4.42098829414964e-13\\
2128	-4.34498564644221e-13\\
2129	-4.19714054693102e-13\\
2130	-4.01998615635722e-13\\
2131	-4.01998615635623e-13\\
2132	-3.88424880544722e-13\\
2133	-3.88424880544627e-13\\
2134	-3.81833866867395e-13\\
2135	-3.81833866867303e-13\\
2136	-3.75371691408737e-13\\
2137	-3.69037087557253e-13\\
2138	-3.62828813705956e-13\\
2139	-3.62828813705869e-13\\
2140	-3.5078400380992e-13\\
2141	-3.44944504622079e-13\\
2142	-3.44944504621997e-13\\
2143	-3.3922540858859e-13\\
2144	-3.3922540858851e-13\\
2145	-3.33625594752062e-13\\
2146	-3.28143965527206e-13\\
2147	-3.22779446493302e-13\\
2148	-3.22779446493227e-13\\
2149	-3.123975559136e-13\\
2150	-3.04910896893513e-13\\
2151	-3.04910896893443e-13\\
2152	-3.00060688498007e-13\\
2153	-3.00060688497939e-13\\
2154	-2.95322085888371e-13\\
2155	-2.90694160280035e-13\\
2156	-2.90694160279971e-13\\
2157	-2.88417186972103e-13\\
2158	-2.88417186972038e-13\\
2159	-2.86159084097203e-13\\
2160	-2.8391974100578e-13\\
2161	-2.79496896173988e-13\\
2162	-2.75147785581789e-13\\
2163	-2.75147785581728e-13\\
2164	-2.66667371658563e-13\\
2165	-2.64592037398338e-13\\
2166	-2.6459203739828e-13\\
2167	-2.56466726368137e-13\\
2168	-2.4861830992822e-13\\
2169	-2.46698744754804e-13\\
2170	-2.4669874475475e-13\\
2171	-2.42909999816595e-13\\
2172	-2.42909999816542e-13\\
2173	-2.39187818031333e-13\\
2174	-2.35531469780483e-13\\
2175	-2.35531469780332e-13\\
2176	-2.35531469780281e-13\\
2177	-2.2841342002098e-13\\
2178	-2.28413420020894e-13\\
2179	-2.2495032334948e-13\\
2180	-2.24950323349432e-13\\
2181	-2.21551472542229e-13\\
2182	-2.18217404333269e-13\\
2183	-2.1494746523235e-13\\
2184	-2.14947465232304e-13\\
2185	-2.11741014323186e-13\\
2186	-2.1174101432314e-13\\
2187	-2.08597423133236e-13\\
2188	-2.05516075506416e-13\\
2189	-2.02496367486448e-13\\
2190	-2.02496367486405e-13\\
2191	-1.96639514754116e-13\\
2192	-1.95212915986067e-13\\
2193	-1.95212915986027e-13\\
2194	-1.89654881384628e-13\\
2195	-1.86963929681985e-13\\
2196	-1.86963929681948e-13\\
2197	-1.83008302930096e-13\\
2198	-1.83008302930059e-13\\
2199	-1.79125854378142e-13\\
2200	-1.76577363051057e-13\\
2201	-1.76577363051021e-13\\
2202	-1.75314871802041e-13\\
2203	-1.75314871802005e-13\\
2204	-1.74060134219138e-13\\
2205	-1.7281308881963e-13\\
2206	-1.70341830522718e-13\\
2207	-1.67900612534936e-13\\
2208	-1.67900612534901e-13\\
2209	-1.63106389343816e-13\\
2210	-1.58426660327924e-13\\
2211	-1.58426660327687e-13\\
2212	-1.58426660327654e-13\\
2213	-1.51613740938768e-13\\
2214	-1.51613740938736e-13\\
2215	-1.45038178492298e-13\\
2216	-1.45038178492266e-13\\
2217	-1.38742720757769e-13\\
2218	-1.36716621523399e-13\\
2219	-1.3671662152337e-13\\
2220	-1.30849923658407e-13\\
2221	-1.3084992365838e-13\\
2222	-1.28033233235286e-13\\
2223	-1.28033233235259e-13\\
2224	-1.2529264752307e-13\\
2225	-1.25292647523028e-13\\
2226	-1.22626957869034e-13\\
2227	-1.20034988660896e-13\\
2228	-1.20034988660864e-13\\
2229	-1.20034988660832e-13\\
2230	-1.18342420479173e-13\\
2231	-1.18342420479149e-13\\
2232	-1.16671819119517e-13\\
2233	-1.15022857130709e-13\\
2234	-1.13395211309709e-13\\
2235	-1.13395211309686e-13\\
2236	-1.10202596197434e-13\\
2237	-1.07091470607722e-13\\
2238	-1.0632613470975e-13\\
2239	-1.06326134709728e-13\\
2240	-1.04059395217808e-13\\
2241	-1.04059395217787e-13\\
2242	-1.02572256107124e-13\\
2243	-1.02572256107103e-13\\
2244	-1.01103992807932e-13\\
2245	-9.96543175357902e-14\\
2246	-9.82229461487186e-14\\
2247	-9.82229461486984e-14\\
2248	-9.54139963537645e-14\\
2249	-9.54139963537441e-14\\
2250	-9.33479021665389e-14\\
2251	-9.33479021665194e-14\\
2252	-9.13094743862282e-14\\
2253	-8.92978140327765e-14\\
2254	-8.86330552246655e-14\\
2255	-8.86330552246467e-14\\
2256	-8.47030732310526e-14\\
2257	-8.47030732310342e-14\\
2258	-8.40575384876842e-14\\
2259	-8.40575384876659e-14\\
2260	-8.34146224620194e-14\\
2261	-8.27742936484327e-14\\
2262	-8.21365206705835e-14\\
2263	-8.21365206705654e-14\\
2264	-8.08685173421828e-14\\
2265	-8.02382248590833e-14\\
2266	-8.02382248590654e-14\\
2267	-7.8987313043547e-14\\
2268	-7.77507008602865e-14\\
2269	-7.65281459284218e-14\\
2270	-7.65281459284045e-14\\
2271	-7.59220649429326e-14\\
2272	-7.59220649429154e-14\\
2273	-7.53194086245056e-14\\
2274	-7.47201474423177e-14\\
2275	-7.353169319586e-14\\
2276	-7.23564692618849e-14\\
2277	-7.23564692618683e-14\\
2278	-7.00447934929168e-14\\
2279	-6.77833076623847e-14\\
2280	-6.66708303581155e-14\\
2281	-6.66708303580998e-14\\
2282	-6.44813167269379e-14\\
2283	-6.44813167269225e-14\\
2284	-6.23376310075543e-14\\
2285	-6.18086724501077e-14\\
2286	-6.18086724500927e-14\\
2287	-6.12824470001096e-14\\
2288	-6.12824470000947e-14\\
2289	-6.07589288728904e-14\\
2290	-6.02380924154869e-14\\
2291	-6.02380924154601e-14\\
2292	-5.92064880598895e-14\\
2293	-5.81895572396438e-14\\
2294	-5.71871006322851e-14\\
2295	-5.7187100632271e-14\\
2296	-5.52248269196181e-14\\
2297	-5.42646252009165e-14\\
2298	-5.4264625200903e-14\\
2299	-5.23851509917096e-14\\
2300	-5.19236748143419e-14\\
2301	-5.19236748143288e-14\\
2302	-5.14655101166031e-14\\
2303	-5.14655101165901e-14\\
2304	-5.10106344487166e-14\\
2305	-5.05590255212702e-14\\
2306	-4.96655195309347e-14\\
2307	-4.87848170149286e-14\\
2308	-4.87848170149162e-14\\
2309	-4.70667888188776e-14\\
2310	-4.5409248326602e-14\\
2311	-4.54092483265523e-14\\
2312	-4.54092483265408e-14\\
2313	-4.30335218739657e-14\\
2314	-4.30335218739548e-14\\
2315	-4.07867881487847e-14\\
2316	-4.07867881487744e-14\\
2317	-4.0065850428228e-14\\
2318	-4.00658504282179e-14\\
2319	-3.93586616898057e-14\\
2320	-3.86650833187905e-14\\
2321	-3.79849793711063e-14\\
2322	-3.79849793710968e-14\\
2323	-3.66620174405385e-14\\
2324	-3.60182385062685e-14\\
2325	-3.60182385062594e-14\\
2326	-3.47654537632079e-14\\
2327	-3.35582184224623e-14\\
2328	-3.32634149819908e-14\\
2329	-3.32634149819824e-14\\
2330	-3.18307095656636e-14\\
2331	-3.18307095656557e-14\\
2332	-3.1003632689695e-14\\
2333	-3.10036326896873e-14\\
2334	-3.02005175971104e-14\\
2335	-3.02005175970975e-14\\
2336	-2.99380758878437e-14\\
2337	-2.99380758878363e-14\\
2338	-2.96782442994591e-14\\
2339	-2.94210101000822e-14\\
2340	-2.89142835768752e-14\\
2341	-2.84177969978292e-14\\
2342	-2.84177969978222e-14\\
2343	-2.745692688457e-14\\
2344	-2.67647692759721e-14\\
2345	-2.67647692759657e-14\\
2346	-2.58793053955078e-14\\
2347	-2.54524117325379e-14\\
2348	-2.54524117325319e-14\\
2349	-2.46298825833841e-14\\
2350	-2.46298825833682e-14\\
2351	-2.46298825833625e-14\\
2352	-2.42340858782358e-14\\
2353	-2.42340858782302e-14\\
2354	-2.38484969779765e-14\\
2355	-2.34730403056608e-14\\
2356	-2.31076422703066e-14\\
2357	-2.31076422703015e-14\\
2358	-2.27501313329533e-14\\
2359	-2.27501313329483e-14\\
2360	-2.23983375004584e-14\\
2361	-2.20521918198678e-14\\
2362	-2.17116264452839e-14\\
2363	-2.17116264452791e-14\\
2364	-2.13765746250376e-14\\
2365	-2.13765746250329e-14\\
2366	-2.10469706881228e-14\\
2367	-2.07227500308906e-14\\
2368	-2.00902054050009e-14\\
2369	-1.94784447931998e-14\\
2370	-1.94784447931955e-14\\
2371	-1.93286955817515e-14\\
2372	-1.93286955817473e-14\\
2373	-1.91802079796183e-14\\
2374	-1.90329747091763e-14\\
2375	-1.87422423666236e-14\\
2376	-1.87422423666015e-14\\
2377	-1.87422423665974e-14\\
2378	-1.85987290484101e-14\\
2379	-1.8598729048406e-14\\
2380	-1.84564415694561e-14\\
2381	-1.83153729574943e-14\\
2382	-1.80368647443602e-14\\
2383	-1.77631498202981e-14\\
2384	-1.77631498202943e-14\\
2385	-1.72307583624399e-14\\
2386	-1.69721955143894e-14\\
2387	-1.69721955143858e-14\\
2388	-1.67186533502142e-14\\
2389	-1.67186533502106e-14\\
2390	-1.6470082174989e-14\\
2391	-1.62264332677977e-14\\
2392	-1.59876588724943e-14\\
2393	-1.59876588724909e-14\\
2394	-1.55245473625053e-14\\
2395	-1.5524547362497e-14\\
2396	-1.55245473624938e-14\\
2397	-1.50803845416847e-14\\
2398	-1.48652994897593e-14\\
2399	-1.48652994897563e-14\\
2400	-1.44446501590058e-14\\
2401	-1.4444650159001e-14\\
2402	-1.44446501589961e-14\\
2403	-1.4033580523772e-14\\
2404	-1.38315370867457e-14\\
2405	-1.38315370867428e-14\\
2406	-1.34342349608915e-14\\
2407	-1.3238898399286e-14\\
2408	-1.32388983992832e-14\\
2409	-1.28546628388249e-14\\
2410	-1.2571970969019e-14\\
2411	-1.25719709690163e-14\\
2412	-1.24787603422333e-14\\
2413	-1.24787603422307e-14\\
2414	-1.23860520808388e-14\\
2415	-1.22938416420288e-14\\
2416	-1.21108961830063e-14\\
2417	-1.19298881071415e-14\\
2418	-1.19298881071389e-14\\
2419	-1.15735425656467e-14\\
2420	-1.14856120598298e-14\\
2421	-1.14856120598273e-14\\
2422	-1.11386352763876e-14\\
2423	-1.11386352763827e-14\\
2424	-1.11386352763803e-14\\
2425	-1.07991480734361e-14\\
2426	-1.07154152011171e-14\\
2427	-1.07154152011148e-14\\
2428	-1.03849163329116e-14\\
2429	-1.02222698653995e-14\\
2430	-1.02222698653972e-14\\
2431	-1.01415833708547e-14\\
2432	-1.01415833708524e-14\\
2433	-1.00613160907547e-14\\
2434	-1.00613160907524e-14\\
2435	-9.98146409190876e-15\\
2436	-9.90202346155654e-15\\
2437	-9.7443607558656e-15\\
2438	-9.58829707116651e-15\\
2439	-9.58829707116431e-15\\
2440	-9.28185894385006e-15\\
2441	-9.13167798247637e-15\\
2442	-9.13167798247425e-15\\
2443	-8.83724507924813e-15\\
2444	-8.69293542737535e-15\\
2445	-8.69293542737331e-15\\
2446	-8.40998829224045e-15\\
2447	-8.40998829223347e-15\\
2448	-8.40998829223149e-15\\
2449	-8.13441979263783e-15\\
2450	-7.99933479104834e-15\\
2451	-7.99933479104644e-15\\
2452	-7.93245546115579e-15\\
2453	-7.93245546115389e-15\\
2454	-7.86601386859672e-15\\
2455	-7.80000675765987e-15\\
2456	-7.7344308939727e-15\\
2457	-7.73443089397084e-15\\
2458	-7.60456007644265e-15\\
2459	-7.47637596090515e-15\\
2460	-7.34985342276511e-15\\
2461	-7.34985342276333e-15\\
2462	-7.1631309138908e-15\\
2463	-7.16313091388904e-15\\
2464	-6.98036688197598e-15\\
2465	-6.80183872246308e-15\\
2466	-6.5134918256232e-15\\
2467	-6.5134918256216e-15\\
2468	-6.23634002943552e-15\\
2469	-6.23634002943398e-15\\
2470	-6.18222296551753e-15\\
2471	-6.182222965516e-15\\
2472	-6.1285374296198e-15\\
2473	-6.07528079102458e-15\\
2474	-5.97004378828145e-15\\
2475	-5.8664913304252e-15\\
2476	-5.86649133042374e-15\\
2477	-5.66453058494059e-15\\
2478	-5.56612555898365e-15\\
2479	-5.56612555898226e-15\\
2480	-5.46941160225526e-15\\
2481	-5.4694116022539e-15\\
2482	-5.37436975857132e-15\\
2483	-5.28098139936188e-15\\
2484	-5.18922822014204e-15\\
2485	-5.18922822014074e-15\\
2486	-5.01055578331232e-15\\
2487	-4.83821235967253e-15\\
2488	-4.7960994689954e-15\\
2489	-4.79609946899421e-15\\
2490	-4.59126986449174e-15\\
2491	-4.59126986449061e-15\\
2492	-4.39581603031912e-15\\
2493	-4.35782908477845e-15\\
2494	-4.35782908477737e-15\\
2495	-4.32020569459746e-15\\
2496	-4.32020569459639e-15\\
2497	-4.2829351122064e-15\\
2498	-4.24600660716146e-15\\
2499	-4.17316860776341e-15\\
2500	-4.1016774198662e-15\\
2501	-4.1016774198652e-15\\
2502	-3.96267969007723e-15\\
2503	-3.86186444783726e-15\\
2504	-3.86186444783632e-15\\
2505	-3.73193699856929e-15\\
2506	-3.73193699856567e-15\\
2507	-3.73193699856476e-15\\
2508	-3.66887007118261e-15\\
2509	-3.66887007118172e-15\\
2510	-3.60705111127102e-15\\
2511	-3.54646800207454e-15\\
2512	-3.48710886906835e-15\\
2513	-3.48710886906751e-15\\
2514	-3.40031014019776e-15\\
2515	-3.40031014019695e-15\\
2516	-3.31614175764132e-15\\
2517	-3.26147242817937e-15\\
2518	-3.2614724281786e-15\\
2519	-3.18160603349842e-15\\
2520	-3.10427353305979e-15\\
2521	-3.10427353305853e-15\\
2522	-3.10427353305781e-15\\
2523	-3.07905319384385e-15\\
2524	-3.07905319384313e-15\\
2525	-3.05410936423452e-15\\
2526	-3.0294408218713e-15\\
2527	-2.98092477725691e-15\\
2528	-2.93349555066585e-15\\
2529	-2.93349555066519e-15\\
2530	-2.84186057780222e-15\\
2531	-2.79763687071404e-15\\
2532	-2.79763687071342e-15\\
2533	-2.7758783808727e-15\\
2534	-2.77587838087208e-15\\
2535	-2.75429903518935e-15\\
2536	-2.73289777625201e-15\\
2537	-2.69062533262455e-15\\
2538	-2.6490527644262e-15\\
2539	-2.64905276442562e-15\\
2540	-2.56797479661868e-15\\
2541	-2.56797479661797e-15\\
2542	-2.56797479661726e-15\\
2543	-2.48960030351875e-15\\
2544	-2.45140757548199e-15\\
2545	-2.45140757548145e-15\\
2546	-2.3769737255901e-15\\
2547	-2.37697372558951e-15\\
2548	-2.37697372558892e-15\\
2549	-2.34071801441014e-15\\
2550	-2.34071801440912e-15\\
2551	-2.30509359575872e-15\\
2552	-2.27009348711197e-15\\
2553	-2.27009348711071e-15\\
2554	-2.27009348710943e-15\\
2555	-2.20193888031619e-15\\
2556	-2.16877102362353e-15\\
2557	-2.16877102362259e-15\\
2558	-2.10426758795762e-15\\
2559	-2.04219583783107e-15\\
2560	-2.04219583782784e-15\\
2561	-2.04219583782697e-15\\
2562	-1.96795174407805e-15\\
2563	-1.96795174407723e-15\\
2564	-1.95354167437605e-15\\
2565	-1.95354167437523e-15\\
2566	-1.93927619014224e-15\\
2567	-1.92515459233546e-15\\
2568	-1.8973402951726e-15\\
2569	-1.88364623289588e-15\\
2570	-1.8836462328951e-15\\
2571	-1.8302749827595e-15\\
2572	-1.80442322775104e-15\\
2573	-1.80442322775031e-15\\
2574	-1.79167436130337e-15\\
2575	-1.79167436130265e-15\\
2576	-1.77900370508835e-15\\
2577	-1.76641063823019e-15\\
2578	-1.76641063822905e-15\\
2579	-1.74145480810164e-15\\
2580	-1.71680197949668e-15\\
2581	-1.69244732036146e-15\\
2582	-1.69244732036077e-15\\
2583	-1.64461347373902e-15\\
2584	-1.59791576402969e-15\\
2585	-1.5299196008647e-15\\
2586	-1.52991960086407e-15\\
};
\addplot [color=mycolor2,solid,forget plot]
  table[row sep=crcr]{%
1	0.15313\\
2	0.15313\\
3	0.153131614989962\\
4	0.153188143215565\\
5	0.153505321610126\\
6	0.153505321610126\\
7	0.154633126647887\\
8	0.154633126647887\\
9	0.153421137821655\\
10	0.153421137821655\\
11	0.152969434437754\\
12	0.152969434437754\\
13	0.14975837658894\\
14	0.14975837658894\\
15	0.144956580293631\\
16	0.14495658029363\\
17	0.138558162777922\\
18	0.130555285180189\\
19	0.130555285180187\\
20	0.130555285180186\\
21	0.115637633000987\\
22	0.115637633000986\\
23	0.111647492594861\\
24	0.11164749259486\\
25	0.107551527943661\\
26	0.103348936174503\\
27	0.0946205553357658\\
28	0.090093055415954\\
29	0.0900930554159519\\
30	0.0708733669527515\\
31	0.0657863137092437\\
32	0.0657863137092414\\
33	0.0470707264257497\\
34	0.0470707264257473\\
35	0.038608115811774\\
36	0.0386081158117714\\
37	0.0301187078522066\\
38	0.0218422822023367\\
39	0.0111320584192047\\
40	0.0111320584192024\\
41	-0.0017451412647055\\
42	-0.00174514126470775\\
43	-0.0140697219629554\\
44	-0.0140697219629576\\
45	-0.0258567828338202\\
46	-0.028150974796736\\
47	-0.0281509747967381\\
48	-0.0371207645085133\\
49	-0.0371207645085153\\
50	-0.045606556121459\\
51	-0.0534569316870435\\
52	-0.0571457813072444\\
53	-0.0571457813072476\\
54	-0.0703421052010178\\
55	-0.0732546507332732\\
56	-0.0732546507332757\\
57	-0.0798659456623785\\
58	-0.0798659456623807\\
59	-0.0855263385064872\\
60	-0.0855263385064891\\
61	-0.0902427640154369\\
62	-0.0940210004002968\\
63	-0.0940210004002991\\
64	-0.0986317226348858\\
65	-0.0986317226348867\\
66	-0.0993526370731427\\
67	-0.0993526370731429\\
68	-0.0994327300022726\\
69	-0.0994327300022727\\
70	-0.0994727794614947\\
71	-0.0994727874134758\\
72	-0.0993526768352302\\
73	-0.0992325524195822\\
74	-0.099232552419582\\
75	-0.0985283580974968\\
76	-0.0975365687988356\\
77	-0.0969325918476496\\
78	-0.096932591847649\\
79	-0.093793207822887\\
80	-0.0928267236024457\\
81	-0.0928267236024448\\
82	-0.0912400352354355\\
83	-0.0912400352354345\\
84	-0.0894883650625657\\
85	-0.0875709405558083\\
86	-0.0868948207118116\\
87	-0.0868948207118103\\
88	-0.0824474621646144\\
89	-0.0791078050053071\\
90	-0.0791078050053055\\
91	-0.0738479984213533\\
92	-0.0703120493504639\\
93	-0.0703120493504624\\
94	-0.0685342136045761\\
95	-0.0685342136045745\\
96	-0.0667493395513993\\
97	-0.0649570773420509\\
98	-0.0613489817813503\\
99	-0.0613489817813471\\
100	-0.061348981781344\\
101	-0.0567910135981784\\
102	-0.0567910135981768\\
103	-0.0521746653123061\\
104	-0.0521746653123044\\
105	-0.0476242423318699\\
106	-0.0432641308592881\\
107	-0.0432641308592851\\
108	-0.0424144718283425\\
109	-0.0424144718283395\\
110	-0.0415721692525926\\
111	-0.0407371818383915\\
112	-0.0390889891928653\\
113	-0.0358786093708109\\
114	-0.0350937022935772\\
115	-0.0350937022935744\\
116	-0.0290632043510933\\
117	-0.0276233282023782\\
118	-0.0276233282023757\\
119	-0.0221263490325744\\
120	-0.0208163887975932\\
121	-0.0208163887975909\\
122	-0.0160570438801743\\
123	-0.0150006033982667\\
124	-0.0150006033982649\\
125	-0.0113022115444085\\
126	-0.010508549962371\\
127	-0.0105085499623696\\
128	-0.00875179114672262\\
129	-0.00875179114672148\\
130	-0.00757916087479443\\
131	-0.00757916087479348\\
132	-0.00731820843277085\\
133	-0.00731820843276995\\
134	-0.00706827602449595\\
135	-0.0068275159056266\\
136	-0.00637346579579224\\
137	-0.00557494401156089\\
138	-0.00523031581749665\\
139	-0.00523031581749607\\
140	-0.00421452794011216\\
141	-0.00405101331592819\\
142	-0.00405101331592793\\
143	-0.00375772258736748\\
144	-0.00375772258736748\\
145	-0.00377451998418073\\
146	-0.00377451998418079\\
147	-0.00381647679652969\\
148	-0.0038727166390681\\
149	-0.00402809230915985\\
150	-0.00412725859153486\\
151	-0.00412725859153505\\
152	-0.00466759973152374\\
153	-0.00483873640420929\\
154	-0.00483873640420961\\
155	-0.0055520172634314\\
156	-0.00555201726343181\\
157	-0.0059124410860251\\
158	-0.00591244108602555\\
159	-0.00628759771240643\\
160	-0.00653578599841538\\
161	-0.00653578599841582\\
162	-0.00690531584631486\\
163	-0.00714990444139671\\
164	-0.00714990444139714\\
165	-0.0075142602310904\\
166	-0.00787572623178794\\
167	-0.00835346102560518\\
168	-0.0083534610256056\\
169	-0.00906178009991046\\
170	-0.00906178009991213\\
171	-0.00906178009991255\\
172	-0.00952901067364299\\
173	-0.00952901067364341\\
174	-0.00997687415267218\\
175	-0.0103899124737084\\
176	-0.0105834744637126\\
177	-0.010583474463713\\
178	-0.0112722069208244\\
179	-0.0114231799044292\\
180	-0.0114231799044295\\
181	-0.011763825106213\\
182	-0.0117638251062133\\
183	-0.0120522432341001\\
184	-0.0120522432341002\\
185	-0.0122887876395593\\
186	-0.0124737481190688\\
187	-0.012473748119069\\
188	-0.0124737481190693\\
189	-0.0126006355076542\\
190	-0.0126006355076543\\
191	-0.012662889470727\\
192	-0.0126628894707271\\
193	-0.0126662104915079\\
194	-0.0126662104915079\\
195	-0.012628219059769\\
196	-0.0125937231314455\\
197	-0.0125937231314454\\
198	-0.0124936970420011\\
199	-0.0123522232406856\\
200	-0.0122659100556399\\
201	-0.0122659100556397\\
202	-0.0118547784292342\\
203	-0.0117378233332886\\
204	-0.0117378233332884\\
205	-0.0115516712471934\\
206	-0.0115516712471932\\
207	-0.0113525824375043\\
208	-0.0111404691009665\\
209	-0.0110668543137872\\
210	-0.0110668543137869\\
211	-0.0105943507351475\\
212	-0.0102497139092464\\
213	-0.0102497139092458\\
214	-0.00968763143281645\\
215	-0.00928239649471308\\
216	-0.00928239649471234\\
217	-0.00907396621485642\\
218	-0.00907396621485568\\
219	-0.00886621779288999\\
220	-0.0087625865484552\\
221	-0.00876258654845447\\
222	-0.00855578460269177\\
223	-0.00834956294861666\\
224	-0.00824665709535903\\
225	-0.0082466570953583\\
226	-0.00783628259099601\\
227	-0.007427665240691\\
228	-0.00722391533061898\\
229	-0.00722391533061826\\
230	-0.00641170964803699\\
231	-0.00620915772116902\\
232	-0.0062091577211683\\
233	-0.00610839228405739\\
234	-0.00610839228405667\\
235	-0.00600857782898421\\
236	-0.00590970946386365\\
237	-0.00571479167132077\\
238	-0.00533609908161176\\
239	-0.00524372026360131\\
240	-0.00524372026360066\\
241	-0.00453702881423189\\
242	-0.00436918072026969\\
243	-0.0043691807202691\\
244	-0.00373210030225669\\
245	-0.00358125209794317\\
246	-0.00358125209794264\\
247	-0.00301944684051335\\
248	-0.002889803828792\\
249	-0.00288980382879155\\
250	-0.002584329428027\\
251	-0.00258432942802658\\
252	-0.00230517829132263\\
253	-0.00230517829132225\\
254	-0.00205200842116886\\
255	-0.00186796954658788\\
256	-0.00186796954658757\\
257	-0.0018245096526498\\
258	-0.00182450965264949\\
259	-0.00178199627320724\\
260	-0.00174035824342754\\
261	-0.00165970011456107\\
262	-0.00150880114248635\\
263	-0.00143853072140514\\
264	-0.00143853072140489\\
265	-0.00119166260307253\\
266	-0.00113843848024014\\
267	-0.00113843848023996\\
268	-0.000959200605772057\\
269	-0.000959200605771646\\
270	-0.000959200605771511\\
271	-0.000922761876475018\\
272	-0.000922761876474895\\
273	-0.000889656176771994\\
274	-0.000859877017096072\\
275	-0.000810275621453308\\
276	-0.000790443662997569\\
277	-0.000790443662997505\\
278	-0.000737416471409183\\
279	-0.000730300623093721\\
280	-0.0007303006230937\\
281	-0.000723240435327344\\
282	-0.000723240435327345\\
283	-0.000724706680384332\\
284	-0.000724706680384341\\
285	-0.0007286246977068\\
286	-0.000731503342169344\\
287	-0.000731503342169367\\
288	-0.000739100607974611\\
289	-0.000749152467286245\\
290	-0.000776628328711077\\
291	-0.000794057716305854\\
292	-0.00079405771630592\\
293	-0.000873105436707512\\
294	-0.000894259205442502\\
295	-0.000894259205442578\\
296	-0.000961126830509811\\
297	-0.000961126830509893\\
298	-0.00100858781550804\\
299	-0.00100858781550812\\
300	-0.00105839800415164\\
301	-0.00111059645024464\\
302	-0.00113760398563549\\
303	-0.00113760398563559\\
304	-0.0012413546296336\\
305	-0.00126559425618062\\
306	-0.00126559425618071\\
307	-0.00132327427264495\\
308	-0.00132327427264502\\
309	-0.00137684013859725\\
310	-0.00137684013859733\\
311	-0.00142635747936927\\
312	-0.00142635747936934\\
313	-0.00147188695989593\\
314	-0.001471886959896\\
315	-0.00151348435948468\\
316	-0.00155120064015199\\
317	-0.00155120064015209\\
318	-0.00157735379664861\\
319	-0.00157735379664865\\
320	-0.00159855985748476\\
321	-0.00160731307452895\\
322	-0.00160731307452897\\
323	-0.00162112846046538\\
324	-0.00163003107373084\\
325	-0.00163264242228832\\
326	-0.00163264242228833\\
327	-0.00163402803412833\\
328	-0.00163230962377641\\
329	-0.0016323096237764\\
330	-0.00162818244094053\\
331	-0.00162818244094052\\
332	-0.00162219409866477\\
333	-0.00161434195603207\\
334	-0.00161130982246483\\
335	-0.0016113098224648\\
336	-0.0015887511829903\\
337	-0.00156954007733136\\
338	-0.00156954007733132\\
339	-0.00154080649094398\\
340	-0.00154080649094393\\
341	-0.00150679563310058\\
342	-0.0015067956331005\\
343	-0.00150679563310043\\
344	-0.0014919412420881\\
345	-0.00149194124208805\\
346	-0.00147670638900483\\
347	-0.00146108808774405\\
348	-0.0014286888199843\\
349	-0.00142868881998395\\
350	-0.00142868881998383\\
351	-0.0013591491013104\\
352	-0.00134075666164557\\
353	-0.00134075666164551\\
354	-0.00126304869651438\\
355	-0.00124256811471934\\
356	-0.00124256811471927\\
357	-0.00123222570545369\\
358	-0.00123222570545361\\
359	-0.00122189271573142\\
360	-0.0012115686391217\\
361	-0.00119094520212441\\
362	-0.00114978305289414\\
363	-0.00113950722460618\\
364	-0.00113950722460611\\
365	-0.00105743389112843\\
366	-0.0010369341522691\\
367	-0.00103693415226903\\
368	-0.000954889015545967\\
369	-0.000934346084458745\\
370	-0.000934346084458672\\
371	-0.000882889345164364\\
372	-0.000882889345164291\\
373	-0.000831240134318731\\
374	-0.000831240134318657\\
375	-0.000780901096501069\\
376	-0.000733376481736367\\
377	-0.000733376481736241\\
378	-0.000733376481736175\\
379	-0.000654740828904449\\
380	-0.000654740828904391\\
381	-0.000646541003483311\\
382	-0.000646541003483253\\
383	-0.000638447160425065\\
384	-0.000630458902985742\\
385	-0.000614797584413385\\
386	-0.000584725096243026\\
387	-0.000570308032111906\\
388	-0.000570308032111856\\
389	-0.000516911344203481\\
390	-0.000504630022639575\\
391	-0.000504630022639532\\
392	-0.000459703911601104\\
393	-0.000459703911600993\\
394	-0.000459703911600956\\
395	-0.000449511170401967\\
396	-0.000449511170401931\\
397	-0.000439729826664271\\
398	-0.000430357963039267\\
399	-0.000412835408365774\\
400	-0.000404681282702018\\
401	-0.00040468128270199\\
402	-0.000386054087615957\\
403	-0.000386054087615933\\
404	-0.000369920603371119\\
405	-0.000369920603371098\\
406	-0.000356151362573767\\
407	-0.000351271066955804\\
408	-0.000351271066955787\\
409	-0.000344619794471207\\
410	-0.000344619794471192\\
411	-0.000338768945938035\\
412	-0.00033371594102731\\
413	-0.000328215890113132\\
414	-0.000328215890113124\\
415	-0.000321412951872142\\
416	-0.000317305783940397\\
417	-0.00031730578394039\\
418	-0.000311779167283901\\
419	-0.000311779167283894\\
420	-0.000311779167283888\\
421	-0.000308513300192605\\
422	-0.0003085133001926\\
423	-0.000305579242748409\\
424	-0.000302974694493122\\
425	-0.000301795338087393\\
426	-0.000301795338087389\\
427	-0.000297066384821916\\
428	-0.000295828507863003\\
429	-0.000295828507862994\\
430	-0.000292633811980978\\
431	-0.000292633811980969\\
432	-0.000289293101770914\\
433	-0.000289293101770904\\
434	-0.000285802284427652\\
435	-0.000282157083263405\\
436	-0.000282157083263384\\
437	-0.000274385471548152\\
438	-0.00027438547154813\\
439	-0.000271077044776881\\
440	-0.000271077044776869\\
441	-0.000267631488441315\\
442	-0.000265856446693054\\
443	-0.000265856446693042\\
444	-0.000262200096402301\\
445	-0.000258399656475589\\
446	-0.000256444476180106\\
447	-0.000256444476180092\\
448	-0.000248248325564716\\
449	-0.000246103422448143\\
450	-0.000246103422448128\\
451	-0.000242832830394358\\
452	-0.000242832830394343\\
453	-0.000239513349855294\\
454	-0.000236143516844424\\
455	-0.000235008793518116\\
456	-0.0002350087935181\\
457	-0.00022924682593477\\
458	-0.000229246825934754\\
459	-0.000223332403462394\\
460	-0.000223332403462377\\
461	-0.000217258280191639\\
462	-0.000211017014561905\\
463	-0.000211017014561886\\
464	-0.000211017014561868\\
465	-0.000208472002683127\\
466	-0.000208472002683109\\
467	-0.000205898528261598\\
468	-0.000203296086807971\\
469	-0.000198002256679318\\
470	-0.00019800225667926\\
471	-0.000198002256679241\\
472	-0.000187459979950326\\
473	-0.000184874152920385\\
474	-0.000184874152920366\\
475	-0.000174714606126115\\
476	-0.000172218170602076\\
477	-0.000172218170602059\\
478	-0.000170976132116411\\
479	-0.000170976132116394\\
480	-0.000169738131944935\\
481	-0.000168504109411711\\
482	-0.000166047755582507\\
483	-0.000161180131714471\\
484	-0.00015997227009231\\
485	-0.000159972270092293\\
486	-0.000150574007329293\\
487	-0.000148300827564533\\
488	-0.000148300827564517\\
489	-0.00014274665575784\\
490	-0.000142746655757825\\
491	-0.000137371035206542\\
492	-0.000137371035206527\\
493	-0.000132167380064451\\
494	-0.000127129315203278\\
495	-0.000127129315203264\\
496	-0.000127129315203249\\
497	-0.000117598202612298\\
498	-0.000117598202612284\\
499	-0.000117598202612271\\
500	-0.000109651272375484\\
501	-0.000109651272375472\\
502	-0.000108803715891997\\
503	-0.000108803715891985\\
504	-0.000107963091773186\\
505	-0.000107129358813907\\
506	-0.000105482403297245\\
507	-0.000102269411021494\\
508	-0.000100702744481673\\
509	-0.000100702744481662\\
510	-9.46969480513911e-05\\
511	-9.32593976637507e-05\\
512	-9.32593976637405e-05\\
513	-8.7756336747459e-05\\
514	-8.77563367474409e-05\\
515	-8.64410083882754e-05\\
516	-8.64410083882661e-05\\
517	-8.5149332128061e-05\\
518	-8.38810547736021e-05\\
519	-8.32556128243637e-05\\
520	-8.32556128243549e-05\\
521	-8.08111114059406e-05\\
522	-8.02141529192064e-05\\
523	-8.02141529191979e-05\\
524	-7.78821088770761e-05\\
525	-7.56379671689237e-05\\
526	-7.45483071685095e-05\\
527	-7.45483071685019e-05\\
528	-7.09589188106284e-05\\
529	-7.09589188106215e-05\\
530	-6.95338486163823e-05\\
531	-6.95338486163757e-05\\
532	-6.81757997245152e-05\\
533	-6.68841732014343e-05\\
534	-6.52643478308392e-05\\
535	-6.52643478308337e-05\\
536	-6.30513718604445e-05\\
537	-6.17188757027156e-05\\
538	-6.17188757027111e-05\\
539	-5.98765565749754e-05\\
540	-5.98765565749673e-05\\
541	-5.8726693945246e-05\\
542	-5.8726693945242e-05\\
543	-5.76384524691438e-05\\
544	-5.66109789074614e-05\\
545	-5.611977649559e-05\\
546	-5.61197764955865e-05\\
547	-5.4956726414285e-05\\
548	-5.49567264142818e-05\\
549	-5.38853442480733e-05\\
550	-5.38853442480704e-05\\
551	-5.28715102331097e-05\\
552	-5.28715102331069e-05\\
553	-5.18811751184775e-05\\
554	-5.18811751184746e-05\\
555	-5.09131256198127e-05\\
556	-4.99661757554206e-05\\
557	-4.99661757554175e-05\\
558	-4.81309588303592e-05\\
559	-4.81309588303542e-05\\
560	-4.74087739080074e-05\\
561	-4.74087739080048e-05\\
562	-4.66805248203141e-05\\
563	-4.63139479711243e-05\\
564	-4.63139479711217e-05\\
565	-4.55755303977213e-05\\
566	-4.48296112798524e-05\\
567	-4.44536561165536e-05\\
568	-4.44536561165509e-05\\
569	-4.292814050137e-05\\
570	-4.25409640962527e-05\\
571	-4.25409640962499e-05\\
572	-4.19565299070295e-05\\
573	-4.19565299070267e-05\\
574	-4.15647874257736e-05\\
575	-4.15647874257708e-05\\
576	-4.11712596588597e-05\\
577	-4.07758694735498e-05\\
578	-4.05774517704439e-05\\
579	-4.05774517704411e-05\\
580	-3.97787364111842e-05\\
581	-3.8971479203737e-05\\
582	-3.85644498853865e-05\\
583	-3.85644498853836e-05\\
584	-3.69117941574778e-05\\
585	-3.64920907009311e-05\\
586	-3.64920907009281e-05\\
587	-3.60696076938282e-05\\
588	-3.60696076938252e-05\\
589	-3.56442623341218e-05\\
590	-3.52159712468501e-05\\
591	-3.43502155100772e-05\\
592	-3.4350215510067e-05\\
593	-3.4350215510064e-05\\
594	-3.26374817401185e-05\\
595	-3.22187456197105e-05\\
596	-3.22187456197076e-05\\
597	-3.05791278339553e-05\\
598	-3.0177653344093e-05\\
599	-3.01776533440902e-05\\
600	-2.99781261035091e-05\\
601	-2.99781261035063e-05\\
602	-2.9779392485211e-05\\
603	-2.95814427490035e-05\\
604	-2.91878561624801e-05\\
605	-2.84096653083054e-05\\
606	-2.8216933247764e-05\\
607	-2.82169332477612e-05\\
608	-2.72697096653642e-05\\
609	-2.72697096653615e-05\\
610	-2.63510395303839e-05\\
611	-2.63510395303809e-05\\
612	-2.54597973504498e-05\\
613	-2.45948912424385e-05\\
614	-2.4594891242436e-05\\
615	-2.45948912424335e-05\\
616	-2.29398797371002e-05\\
617	-2.29398797370977e-05\\
618	-2.29398797370952e-05\\
619	-2.13778921452464e-05\\
620	-2.13778921452441e-05\\
621	-2.13778921452418e-05\\
622	-2.00825007778209e-05\\
623	-2.0082500777819e-05\\
624	-1.99472001369451e-05\\
625	-1.99472001369432e-05\\
626	-1.98136009192515e-05\\
627	-1.96816965759824e-05\\
628	-1.94229467420879e-05\\
629	-1.89255067061309e-05\\
630	-1.86867190004445e-05\\
631	-1.86867190004428e-05\\
632	-1.77967064743597e-05\\
633	-1.75902698573198e-05\\
634	-1.75902698573184e-05\\
635	-1.70954363856924e-05\\
636	-1.7095436385691e-05\\
637	-1.68111616161718e-05\\
638	-1.68111616161705e-05\\
639	-1.66268276207325e-05\\
640	-1.66268276207312e-05\\
641	-1.64465973692497e-05\\
642	-1.62704355359988e-05\\
643	-1.59301798017135e-05\\
644	-1.5766019206891e-05\\
645	-1.57660192068898e-05\\
646	-1.51484168379769e-05\\
647	-1.50036249312513e-05\\
648	-1.50036249312503e-05\\
649	-1.45138793544253e-05\\
650	-1.45138793544243e-05\\
651	-1.43101758937224e-05\\
652	-1.43101758937215e-05\\
653	-1.41100543910602e-05\\
654	-1.39134265880492e-05\\
655	-1.36565411650538e-05\\
656	-1.36565411650529e-05\\
657	-1.32821306401946e-05\\
658	-1.30395166255789e-05\\
659	-1.3039516625578e-05\\
660	-1.2681617645863e-05\\
661	-1.26816176458622e-05\\
662	-1.24449861810741e-05\\
663	-1.24449861810732e-05\\
664	-1.22097123736273e-05\\
665	-1.21510844170999e-05\\
666	-1.2151084417099e-05\\
667	-1.19172483854888e-05\\
668	-1.18589440036623e-05\\
669	-1.18589440036614e-05\\
670	-1.16262592695052e-05\\
671	-1.13942902684683e-05\\
672	-1.12785173120565e-05\\
673	-1.12785173120557e-05\\
674	-1.09895199384073e-05\\
675	-1.09895199384064e-05\\
676	-1.07008608553855e-05\\
677	-1.07008608553836e-05\\
678	-1.07008608553828e-05\\
679	-1.04159270973358e-05\\
680	-1.01381102644311e-05\\
681	-1.01381102644299e-05\\
682	-1.01381102644286e-05\\
683	-9.60247423492051e-06\\
684	-9.60247423491735e-06\\
685	-9.60247423491661e-06\\
686	-9.39521957090541e-06\\
687	-9.39521957090468e-06\\
688	-9.19171929165986e-06\\
689	-9.09132707123955e-06\\
690	-9.09132707123884e-06\\
691	-8.8931948952185e-06\\
692	-8.69849311581866e-06\\
693	-8.60238081659412e-06\\
694	-8.60238081659344e-06\\
695	-8.36556168232076e-06\\
696	-8.3655616823201e-06\\
697	-8.13345632467182e-06\\
698	-8.13345632467102e-06\\
699	-7.99633719922108e-06\\
700	-7.99633719922044e-06\\
701	-7.86075208809108e-06\\
702	-7.72664119589708e-06\\
703	-7.68225492836099e-06\\
704	-7.68225492836036e-06\\
705	-7.41942386233204e-06\\
706	-7.24741527412089e-06\\
707	-7.24741527412028e-06\\
708	-6.99388339439223e-06\\
709	-6.82765621484327e-06\\
710	-6.82765621484269e-06\\
711	-6.74533181369082e-06\\
712	-6.74533181369023e-06\\
713	-6.66351199034159e-06\\
714	-6.58218070747033e-06\\
715	-6.42092009102402e-06\\
716	-6.42092009102222e-06\\
717	-6.42092009102165e-06\\
718	-6.10356596424353e-06\\
719	-6.02521308104601e-06\\
720	-6.02521308104545e-06\\
721	-5.72315586650732e-06\\
722	-5.65089804825134e-06\\
723	-5.65089804825083e-06\\
724	-5.61524772292431e-06\\
725	-5.61524772292381e-06\\
726	-5.57991411940159e-06\\
727	-5.54489550592157e-06\\
728	-5.47579640076286e-06\\
729	-5.47579640076237e-06\\
730	-5.40793686380038e-06\\
731	-5.34130359434911e-06\\
732	-5.30844271923725e-06\\
733	-5.30844271923678e-06\\
734	-5.17831949697285e-06\\
735	-5.04956011290942e-06\\
736	-4.98566027155525e-06\\
737	-4.9856602715548e-06\\
738	-4.73296835109412e-06\\
739	-4.67046032103547e-06\\
740	-4.67046032103502e-06\\
741	-4.42272039032727e-06\\
742	-4.36129775712774e-06\\
743	-4.3612977571273e-06\\
744	-4.11729539473771e-06\\
745	-4.05665706457511e-06\\
746	-4.05665706457468e-06\\
747	-3.82198440890924e-06\\
748	-3.79370400941905e-06\\
749	-3.79370400941865e-06\\
750	-3.76565307604698e-06\\
751	-3.76565307604658e-06\\
752	-3.73783023460633e-06\\
753	-3.71023412145316e-06\\
754	-3.6557166821913e-06\\
755	-3.54934378189782e-06\\
756	-3.49746747064406e-06\\
757	-3.49746747064369e-06\\
758	-3.37151654476955e-06\\
759	-3.3715165447692e-06\\
760	-3.25078560239942e-06\\
761	-3.25078560239907e-06\\
762	-3.13522513971395e-06\\
763	-3.06835876900969e-06\\
764	-3.06835876900938e-06\\
765	-3.02479198811123e-06\\
766	-3.02479198811092e-06\\
767	-2.98202386151106e-06\\
768	-2.94004600642752e-06\\
769	-2.85842835291708e-06\\
770	-2.81877255655673e-06\\
771	-2.81877255655645e-06\\
772	-2.66765660481527e-06\\
773	-2.63171740082162e-06\\
774	-2.63171740082136e-06\\
775	-2.51156791513601e-06\\
776	-2.51156791513578e-06\\
777	-2.46270957629172e-06\\
778	-2.46270957629149e-06\\
779	-2.41564305361668e-06\\
780	-2.37059028419625e-06\\
781	-2.31361821526057e-06\\
782	-2.31361821526038e-06\\
783	-2.23470962066111e-06\\
784	-2.18641007991448e-06\\
785	-2.18641007991431e-06\\
786	-2.13081325168858e-06\\
787	-2.13081325168843e-06\\
788	-2.12032537272879e-06\\
789	-2.12032537272864e-06\\
790	-2.11004676622006e-06\\
791	-2.09997692848483e-06\\
792	-2.08046159584367e-06\\
793	-2.0804615958435e-06\\
794	-2.04309517452561e-06\\
795	-2.00737685847022e-06\\
796	-1.99012693821797e-06\\
797	-1.99012693821785e-06\\
798	-1.92511086669034e-06\\
799	-1.90983682176868e-06\\
800	-1.90983682176857e-06\\
801	-1.8733336028579e-06\\
802	-1.8733336028578e-06\\
803	-1.83919766418033e-06\\
804	-1.83919766418021e-06\\
805	-1.80738718466383e-06\\
806	-1.77786319249299e-06\\
807	-1.77786319249289e-06\\
808	-1.77786319249279e-06\\
809	-1.72327311875675e-06\\
810	-1.72327311875657e-06\\
811	-1.72327311875649e-06\\
812	-1.70263217405444e-06\\
813	-1.70263217405436e-06\\
814	-1.68264956254415e-06\\
815	-1.67290021556997e-06\\
816	-1.6729002155699e-06\\
817	-1.65324589747104e-06\\
818	-1.64823287417424e-06\\
819	-1.64823287417417e-06\\
820	-1.62777563643481e-06\\
821	-1.62255878784806e-06\\
822	-1.62255878784798e-06\\
823	-1.60127356779355e-06\\
824	-1.5793072361601e-06\\
825	-1.56806329247164e-06\\
826	-1.56806329247155e-06\\
827	-1.55086521839835e-06\\
828	-1.55086521839827e-06\\
829	-1.53326179932593e-06\\
830	-1.51524527164897e-06\\
831	-1.50914659249219e-06\\
832	-1.5091465924921e-06\\
833	-1.47155519285674e-06\\
834	-1.44551987848354e-06\\
835	-1.44551987848344e-06\\
836	-1.4056015981619e-06\\
837	-1.37866977500037e-06\\
838	-1.37866977500027e-06\\
839	-1.3651000736299e-06\\
840	-1.3651000736298e-06\\
841	-1.35145763745685e-06\\
842	-1.33773979246052e-06\\
843	-1.31006710572417e-06\\
844	-1.31006710572334e-06\\
845	-1.31006710572324e-06\\
846	-1.25369746637742e-06\\
847	-1.239375580162e-06\\
848	-1.2393755801619e-06\\
849	-1.20354420374886e-06\\
850	-1.20354420374876e-06\\
851	-1.16787137375492e-06\\
852	-1.1678713737548e-06\\
853	-1.16787137375467e-06\\
854	-1.16075198454539e-06\\
855	-1.16075198454519e-06\\
856	-1.15363683974937e-06\\
857	-1.14652559070881e-06\\
858	-1.13231338625683e-06\\
859	-1.10392040519096e-06\\
860	-1.09682667822671e-06\\
861	-1.09682667822651e-06\\
862	-1.04008770603536e-06\\
863	-1.02589323214731e-06\\
864	-1.02589323214711e-06\\
865	-9.69952660623121e-07\\
866	-9.56223111084106e-07\\
867	-9.56223111083911e-07\\
868	-9.02246542625002e-07\\
869	-8.88974583315773e-07\\
870	-8.88974583315585e-07\\
871	-8.36697424719018e-07\\
872	-8.23817996479287e-07\\
873	-8.23817996479104e-07\\
874	-7.72983985051307e-07\\
875	-7.66701509228406e-07\\
876	-7.66701509228228e-07\\
877	-7.60433952955367e-07\\
878	-7.60433952955189e-07\\
879	-7.54206980640011e-07\\
880	-7.4804625854213e-07\\
881	-7.35922360723288e-07\\
882	-7.29958590923375e-07\\
883	-7.29958590923207e-07\\
884	-7.06751257636437e-07\\
885	-7.01109939334066e-07\\
886	-7.01109939333907e-07\\
887	-6.79178433106552e-07\\
888	-6.58248169058283e-07\\
889	-6.48153344134381e-07\\
890	-6.48153344134239e-07\\
891	-6.10197147297826e-07\\
892	-6.1019714729769e-07\\
893	-6.01304574258161e-07\\
894	-6.01304574258037e-07\\
895	-5.9263059507994e-07\\
896	-5.84157068455276e-07\\
897	-5.67804768476809e-07\\
898	-5.59922789895716e-07\\
899	-5.59922789895605e-07\\
900	-5.30320413073835e-07\\
901	-5.23393950444267e-07\\
902	-5.2339395044417e-07\\
903	-5.00614447489844e-07\\
904	-5.00614447489756e-07\\
905	-4.91538991482364e-07\\
906	-4.91538991482281e-07\\
907	-4.82793960871108e-07\\
908	-4.74299479782027e-07\\
909	-4.63356789466559e-07\\
910	-4.63356789466483e-07\\
911	-4.50282414202632e-07\\
912	-4.5028241420256e-07\\
913	-4.378642245997e-07\\
914	-4.37864224599631e-07\\
915	-4.26087006816703e-07\\
916	-4.23807196541371e-07\\
917	-4.23807196541307e-07\\
918	-4.14936332281455e-07\\
919	-4.14936332281393e-07\\
920	-4.06364934857181e-07\\
921	-3.97993520436526e-07\\
922	-3.93880749676839e-07\\
923	-3.93880749676781e-07\\
924	-3.7789720819064e-07\\
925	-3.74014292130432e-07\\
926	-3.74014292130377e-07\\
927	-3.64496337819235e-07\\
928	-3.64496337819181e-07\\
929	-3.55239574232692e-07\\
930	-3.55239574232633e-07\\
931	-3.46232660612257e-07\\
932	-3.37464562362428e-07\\
933	-3.37464562362372e-07\\
934	-3.37464562362323e-07\\
935	-3.20825427913457e-07\\
936	-3.20825427913405e-07\\
937	-3.2082542791336e-07\\
938	-3.14531859757312e-07\\
939	-3.14531859757268e-07\\
940	-3.12989815518968e-07\\
941	-3.12989815518924e-07\\
942	-3.11460158341291e-07\\
943	-3.09942813269412e-07\\
944	-3.06944762641721e-07\\
945	-3.05463910179471e-07\\
946	-3.0546391017943e-07\\
947	-2.99659966411562e-07\\
948	-2.94043776447001e-07\\
949	-2.9130470738753e-07\\
950	-2.91304707387491e-07\\
951	-2.80630480944969e-07\\
952	-2.78019467860774e-07\\
953	-2.78019467860737e-07\\
954	-2.74144015524375e-07\\
955	-2.74144015524339e-07\\
956	-2.70316326353418e-07\\
957	-2.66534712220028e-07\\
958	-2.65284124242842e-07\\
959	-2.65284124242806e-07\\
960	-2.57880794941947e-07\\
961	-2.53036247935297e-07\\
962	-2.53036247935263e-07\\
963	-2.45896439656918e-07\\
964	-2.41215799782419e-07\\
965	-2.41215799782385e-07\\
966	-2.3890470124077e-07\\
967	-2.38904701240738e-07\\
968	-2.36621588154551e-07\\
969	-2.34366013015447e-07\\
970	-2.29935713489584e-07\\
971	-2.29935713489485e-07\\
972	-2.29935713489453e-07\\
973	-2.24544976740287e-07\\
974	-2.24544976740257e-07\\
975	-2.19311571597477e-07\\
976	-2.19311571597448e-07\\
977	-2.1413808806575e-07\\
978	-2.08927189598742e-07\\
979	-2.08927189598687e-07\\
980	-2.07879959192013e-07\\
981	-2.07879959191983e-07\\
982	-2.06830925708429e-07\\
983	-2.05780037721703e-07\\
984	-2.03672492170454e-07\\
985	-1.99432874907036e-07\\
986	-1.98367558155249e-07\\
987	-1.98367558155219e-07\\
988	-1.89758923816801e-07\\
989	-1.87580913980111e-07\\
990	-1.8758091398008e-07\\
991	-1.78856030527641e-07\\
992	-1.76675920017718e-07\\
993	-1.76675920017687e-07\\
994	-1.67947098121961e-07\\
995	-1.65760659066734e-07\\
996	-1.65760659066703e-07\\
997	-1.56985110032383e-07\\
998	-1.54781624505008e-07\\
999	-1.54781624504976e-07\\
1000	-1.49251392998297e-07\\
1001	-1.49251392998265e-07\\
1002	-1.44801496057214e-07\\
1003	-1.44801496057182e-07\\
1004	-1.43684997087239e-07\\
1005	-1.43684997087207e-07\\
1006	-1.42571877725055e-07\\
1007	-1.41467182223686e-07\\
1008	-1.39282846692213e-07\\
1009	-1.35012907556548e-07\\
1010	-1.32926466998869e-07\\
1011	-1.32926466998839e-07\\
1012	-1.24894761690194e-07\\
1013	-1.22963381769841e-07\\
1014	-1.22963381769813e-07\\
1015	-1.15532689252618e-07\\
1016	-1.15532689252588e-07\\
1017	-1.13746902342923e-07\\
1018	-1.13746902342898e-07\\
1019	-1.11991598093262e-07\\
1020	-1.10268868994763e-07\\
1021	-1.06919791928173e-07\\
1022	-1.05292787505084e-07\\
1023	-1.05292787505061e-07\\
1024	-9.90942941096801e-08\\
1025	-9.76205331599606e-08\\
1026	-9.76205331599398e-08\\
1027	-9.26948739541156e-08\\
1028	-9.26948739540963e-08\\
1029	-9.06925299068886e-08\\
1030	-9.069252990687e-08\\
1031	-8.87540882280021e-08\\
1032	-8.74968717687368e-08\\
1033	-8.74968717687192e-08\\
1034	-8.56630480135529e-08\\
1035	-8.44748166810101e-08\\
1036	-8.44748166809934e-08\\
1037	-8.27531529850905e-08\\
1038	-8.11114678455013e-08\\
1039	-7.90457190752179e-08\\
1040	-7.90457190752037e-08\\
1041	-7.62077722019112e-08\\
1042	-7.62077722018974e-08\\
1043	-7.62077722018839e-08\\
1044	-7.44874285056517e-08\\
1045	-7.448742850564e-08\\
1046	-7.29027922256744e-08\\
1047	-7.14526209225375e-08\\
1048	-7.07776002173176e-08\\
1049	-7.07776002173083e-08\\
1050	-6.8278345551895e-08\\
1051	-6.76950081507494e-08\\
1052	-6.76950081507412e-08\\
1053	-6.63079842293241e-08\\
1054	-6.63079842293165e-08\\
1055	-6.50215009408386e-08\\
1056	-6.50215009408289e-08\\
1057	-6.38339821715942e-08\\
1058	-6.27439730626019e-08\\
1059	-6.2743973062588e-08\\
1060	-6.27439730625821e-08\\
1061	-6.17096541067625e-08\\
1062	-6.17096541067566e-08\\
1063	-6.06892741053204e-08\\
1064	-6.06892741053117e-08\\
1065	-6.06892741053029e-08\\
1066	-5.98821655343195e-08\\
1067	-5.98821655343138e-08\\
1068	-5.90825447044485e-08\\
1069	-5.86853460563177e-08\\
1070	-5.86853460563121e-08\\
1071	-5.78957830184499e-08\\
1072	-5.71121502958338e-08\\
1073	-5.67223656260395e-08\\
1074	-5.6722365626034e-08\\
1075	-5.5137723611356e-08\\
1076	-5.47324703415959e-08\\
1077	-5.47324703415901e-08\\
1078	-5.41174500067724e-08\\
1079	-5.41174500067665e-08\\
1080	-5.34936197849682e-08\\
1081	-5.28607045490664e-08\\
1082	-5.26476657846919e-08\\
1083	-5.26476657846859e-08\\
1084	-5.13469960431428e-08\\
1085	-5.0457732251795e-08\\
1086	-5.04577322517886e-08\\
1087	-4.90888674328573e-08\\
1088	-4.81519346787634e-08\\
1089	-4.81519346787566e-08\\
1090	-4.76787722066012e-08\\
1091	-4.76787722065945e-08\\
1092	-4.7206201875636e-08\\
1093	-4.69701098128472e-08\\
1094	-4.69701098128404e-08\\
1095	-4.64982540493465e-08\\
1096	-4.60267590395767e-08\\
1097	-4.57911179353064e-08\\
1098	-4.57911179352997e-08\\
1099	-4.48489934670821e-08\\
1100	-4.3907018979694e-08\\
1101	-4.34358571844073e-08\\
1102	-4.34358571844006e-08\\
1103	-4.1547890866913e-08\\
1104	-4.1074606939574e-08\\
1105	-4.10746069395673e-08\\
1106	-4.08380854487833e-08\\
1107	-4.08380854487765e-08\\
1108	-4.06021328605197e-08\\
1109	-4.03667376104763e-08\\
1110	-3.98975730141167e-08\\
1111	-3.89654261260681e-08\\
1112	-3.87336107356735e-08\\
1113	-3.8733610735667e-08\\
1114	-3.68949062192327e-08\\
1115	-3.64392113881465e-08\\
1116	-3.643921138814e-08\\
1117	-3.46296773072977e-08\\
1118	-3.41801617503112e-08\\
1119	-3.41801617503049e-08\\
1120	-3.24124266543962e-08\\
1121	-3.19789849079878e-08\\
1122	-3.19789849079817e-08\\
1123	-3.09093096539883e-08\\
1124	-3.09093096539823e-08\\
1125	-2.98584876387825e-08\\
1126	-2.98584876387748e-08\\
1127	-2.98584876387671e-08\\
1128	-2.88252314668236e-08\\
1129	-2.80104225895021e-08\\
1130	-2.80104225894964e-08\\
1131	-2.7808275269436e-08\\
1132	-2.78082752694303e-08\\
1133	-2.76067300538536e-08\\
1134	-2.7405777066581e-08\\
1135	-2.70056084185385e-08\\
1136	-2.62119464907582e-08\\
1137	-2.58182976449571e-08\\
1138	-2.58182976449515e-08\\
1139	-2.43108409691476e-08\\
1140	-2.3953373932121e-08\\
1141	-2.3953373932116e-08\\
1142	-2.25989920459409e-08\\
1143	-2.25989920459363e-08\\
1144	-2.2278936306579e-08\\
1145	-2.22789363065745e-08\\
1146	-2.19640480982096e-08\\
1147	-2.16521449603378e-08\\
1148	-2.10370499331222e-08\\
1149	-2.07337374786536e-08\\
1150	-2.07337374786493e-08\\
1151	-1.95473404319023e-08\\
1152	-1.92571636877083e-08\\
1153	-1.92571636877042e-08\\
1154	-1.85423370914376e-08\\
1155	-1.85423370914336e-08\\
1156	-1.82605054441357e-08\\
1157	-1.82605054441317e-08\\
1158	-1.79809329616859e-08\\
1159	-1.78419767659603e-08\\
1160	-1.78419767659563e-08\\
1161	-1.75656904207969e-08\\
1162	-1.72915270539451e-08\\
1163	-1.67493547131533e-08\\
1164	-1.64812394677143e-08\\
1165	-1.64812394677105e-08\\
1166	-1.54551789832776e-08\\
1167	-1.52117506821754e-08\\
1168	-1.5211750682172e-08\\
1169	-1.45121066059906e-08\\
1170	-1.45121066059874e-08\\
1171	-1.40707566001509e-08\\
1172	-1.40707566001478e-08\\
1173	-1.36490640994622e-08\\
1174	-1.3246698474602e-08\\
1175	-1.30526640664642e-08\\
1176	-1.30526640664615e-08\\
1177	-1.23185961219594e-08\\
1178	-1.21451224225111e-08\\
1179	-1.21451224225086e-08\\
1180	-1.17286388067552e-08\\
1181	-1.17286388067529e-08\\
1182	-1.13363229287377e-08\\
1183	-1.1336322928735e-08\\
1184	-1.09676941493265e-08\\
1185	-1.09676941493241e-08\\
1186	-1.09676941493221e-08\\
1187	-1.0622300852166e-08\\
1188	-1.06223008521641e-08\\
1189	-1.02997198872775e-08\\
1190	-9.99955605263701e-09\\
1191	-9.9995560526346e-09\\
1192	-9.77557047723486e-09\\
1193	-9.77557047723332e-09\\
1194	-9.56602678297024e-09\\
1195	-9.46661885759622e-09\\
1196	-9.46661885759484e-09\\
1197	-9.27843342547682e-09\\
1198	-9.10430008475841e-09\\
1199	-9.02245985447064e-09\\
1200	-9.0224598544695e-09\\
1201	-8.72957548293167e-09\\
1202	-8.66490176841955e-09\\
1203	-8.66490176841866e-09\\
1204	-8.57188365644877e-09\\
1205	-8.57188365644791e-09\\
1206	-8.48172826214965e-09\\
1207	-8.39439582412667e-09\\
1208	-8.36590564249617e-09\\
1209	-8.36590564249536e-09\\
1210	-8.20138424603791e-09\\
1211	-8.09771959295828e-09\\
1212	-8.09771959295756e-09\\
1213	-7.97476380024738e-09\\
1214	-7.9747638002467e-09\\
1215	-7.85902897025972e-09\\
1216	-7.85902897025909e-09\\
1217	-7.81472485362635e-09\\
1218	-7.81472485362573e-09\\
1219	-7.77154462390382e-09\\
1220	-7.7294798175334e-09\\
1221	-7.64866371251142e-09\\
1222	-7.64866371250852e-09\\
1223	-7.64866371250796e-09\\
1224	-7.49264994306895e-09\\
1225	-7.4539997802996e-09\\
1226	-7.45399978029905e-09\\
1227	-7.30058494537825e-09\\
1228	-7.26249010304648e-09\\
1229	-7.26249010304594e-09\\
1230	-7.24334630052128e-09\\
1231	-7.24334630052073e-09\\
1232	-7.22396357290065e-09\\
1233	-7.20434097023926e-09\\
1234	-7.16437228194562e-09\\
1235	-7.08151330500107e-09\\
1236	-7.06018399028553e-09\\
1237	-7.06018399028492e-09\\
1238	-6.8805347084627e-09\\
1239	-6.83307782688353e-09\\
1240	-6.83307782688284e-09\\
1241	-6.63279296073897e-09\\
1242	-6.58005833831243e-09\\
1243	-6.58005833831168e-09\\
1244	-6.44344862625123e-09\\
1245	-6.44344862625044e-09\\
1246	-6.29988522275047e-09\\
1247	-6.29988522274963e-09\\
1248	-6.15165805633759e-09\\
1249	-6.00105134257282e-09\\
1250	-6.00105134257178e-09\\
1251	-5.72337183609377e-09\\
1252	-5.72337183609287e-09\\
1253	-5.69195808263331e-09\\
1254	-5.69195808263242e-09\\
1255	-5.66042663170437e-09\\
1256	-5.62877593777139e-09\\
1257	-5.56511061141038e-09\\
1258	-5.43628040260602e-09\\
1259	-5.37109026811409e-09\\
1260	-5.37109026811316e-09\\
1261	-5.11049977650083e-09\\
1262	-5.04568300063036e-09\\
1263	-5.04568300062944e-09\\
1264	-4.78735809673918e-09\\
1265	-4.78735809673826e-09\\
1266	-4.72294912938923e-09\\
1267	-4.72294912938831e-09\\
1268	-4.65858380004019e-09\\
1269	-4.59424949194967e-09\\
1270	-4.46562350422881e-09\\
1271	-4.40130661256745e-09\\
1272	-4.40130661256654e-09\\
1273	-4.24224625713634e-09\\
1274	-4.24224625713544e-09\\
1275	-4.08655845577084e-09\\
1276	-4.08655845576912e-09\\
1277	-3.93405247036823e-09\\
1278	-3.87389900692069e-09\\
1279	-3.87389900691984e-09\\
1280	-3.78454146180396e-09\\
1281	-3.78454146180312e-09\\
1282	-3.69619610131254e-09\\
1283	-3.6088239636112e-09\\
1284	-3.49377514219923e-09\\
1285	-3.49377514219842e-09\\
1286	-3.3241206508405e-09\\
1287	-3.21283416009907e-09\\
1288	-3.21283416009828e-09\\
1289	-3.0525685527903e-09\\
1290	-3.05256855278939e-09\\
1291	-3.05256855278866e-09\\
1292	-2.9518930377862e-09\\
1293	-2.9518930377855e-09\\
1294	-2.85605944940288e-09\\
1295	-2.76499264901262e-09\\
1296	-2.72122433635601e-09\\
1297	-2.7212243363554e-09\\
1298	-2.55772082225784e-09\\
1299	-2.51969731829169e-09\\
1300	-2.51969731829116e-09\\
1301	-2.4289490115089e-09\\
1302	-2.4289490115084e-09\\
1303	-2.34392827919007e-09\\
1304	-2.34392827918938e-09\\
1305	-2.2645309596167e-09\\
1306	-2.19065978114395e-09\\
1307	-2.19065978114345e-09\\
1308	-2.19065978114295e-09\\
1309	-2.05914050125253e-09\\
1310	-2.0591405012521e-09\\
1311	-2.05914050125168e-09\\
1312	-2.01178120845177e-09\\
1313	-2.01178120845144e-09\\
1314	-1.96637347154133e-09\\
1315	-1.94439025125697e-09\\
1316	-1.94439025125666e-09\\
1317	-1.90184356233891e-09\\
1318	-1.86116223082802e-09\\
1319	-1.84151104977674e-09\\
1320	-1.84151104977646e-09\\
1321	-1.7674140530845e-09\\
1322	-1.74999858148673e-09\\
1323	-1.74999858148649e-09\\
1324	-1.72442638272539e-09\\
1325	-1.72442638272515e-09\\
1326	-1.69929219878846e-09\\
1327	-1.67458494472488e-09\\
1328	-1.66644217799946e-09\\
1329	-1.66644217799923e-09\\
1330	-1.62640782444718e-09\\
1331	-1.62640782444695e-09\\
1332	-1.58747017223412e-09\\
1333	-1.5874701722339e-09\\
1334	-1.54958151784988e-09\\
1335	-1.51269544295481e-09\\
1336	-1.51269544295445e-09\\
1337	-1.49821157218079e-09\\
1338	-1.49821157218059e-09\\
1339	-1.48387803220585e-09\\
1340	-1.46969201310117e-09\\
1341	-1.44175144385109e-09\\
1342	-1.44175144385089e-09\\
1343	-1.44175144385069e-09\\
1344	-1.38792455800962e-09\\
1345	-1.37492249127017e-09\\
1346	-1.37492249126999e-09\\
1347	-1.32465637790813e-09\\
1348	-1.31251307242197e-09\\
1349	-1.3125130724218e-09\\
1350	-1.30650324705547e-09\\
1351	-1.3065032470553e-09\\
1352	-1.30053424667552e-09\\
1353	-1.29460577873232e-09\\
1354	-1.28286928014306e-09\\
1355	-1.25987002170035e-09\\
1356	-1.25421725629834e-09\\
1357	-1.25421725629818e-09\\
1358	-1.2097187053096e-09\\
1359	-1.19876316926243e-09\\
1360	-1.19876316926228e-09\\
1361	-1.17164671643283e-09\\
1362	-1.17164671643268e-09\\
1363	-1.14489286793191e-09\\
1364	-1.14489286793176e-09\\
1365	-1.11846884677001e-09\\
1366	-1.09234228021684e-09\\
1367	-1.09234228021617e-09\\
1368	-1.09234228021602e-09\\
1369	-1.04085380290306e-09\\
1370	-1.0408538029024e-09\\
1371	-1.04085380290225e-09\\
1372	-9.95984366512801e-10\\
1373	-9.95984366512663e-10\\
1374	-9.91126696063267e-10\\
1375	-9.91126696063129e-10\\
1376	-9.8629370836548e-10\\
1377	-9.81485166521391e-10\\
1378	-9.71940479102149e-10\\
1379	-9.53134166088737e-10\\
1380	-9.43868854259104e-10\\
1381	-9.43868854258973e-10\\
1382	-9.07684400174781e-10\\
1383	-8.98848619527772e-10\\
1384	-8.98848619527647e-10\\
1385	-8.64117606171835e-10\\
1386	-8.64117606171713e-10\\
1387	-8.55568294842235e-10\\
1388	-8.55568294842114e-10\\
1389	-8.47068993714665e-10\\
1390	-8.38618036772326e-10\\
1391	-8.3441016937583e-10\\
1392	-8.34410169375711e-10\\
1393	-8.17691302441889e-10\\
1394	-8.13538712333899e-10\\
1395	-8.13538712333781e-10\\
1396	-7.96880155238405e-10\\
1397	-7.80074436549674e-10\\
1398	-7.71612278606435e-10\\
1399	-7.71612278606315e-10\\
1400	-7.4165505506951e-10\\
1401	-7.41655055069387e-10\\
1402	-7.28641906247479e-10\\
1403	-7.28641906247355e-10\\
1404	-7.15515887312569e-10\\
1405	-7.02271209386724e-10\\
1406	-6.84416954520296e-10\\
1407	-6.84416954520168e-10\\
1408	-6.57189887839347e-10\\
1409	-6.38720632575379e-10\\
1410	-6.38720632575246e-10\\
1411	-6.11236846836976e-10\\
1412	-6.11236846836773e-10\\
1413	-6.11236846836573e-10\\
1414	-5.93365907758516e-10\\
1415	-5.9336590775839e-10\\
1416	-5.75839848639448e-10\\
1417	-5.5864492812115e-10\\
1418	-5.50167421828728e-10\\
1419	-5.50167421828609e-10\\
1420	-5.29310209543835e-10\\
1421	-5.29310209543718e-10\\
1422	-5.08913415851679e-10\\
1423	-5.08913415851564e-10\\
1424	-4.89114790823196e-10\\
1425	-4.89114790823085e-10\\
1426	-4.70052817501838e-10\\
1427	-4.70052817501732e-10\\
1428	-4.51704142586451e-10\\
1429	-4.34046286655729e-10\\
1430	-4.34046286655547e-10\\
1431	-4.00717319275285e-10\\
1432	-4.0071731927514e-10\\
1433	-3.88098430492688e-10\\
1434	-3.880984304926e-10\\
1435	-3.75871779924386e-10\\
1436	-3.69902536267505e-10\\
1437	-3.69902536267421e-10\\
1438	-3.58391056630175e-10\\
1439	-3.47537911834624e-10\\
1440	-3.42355535747464e-10\\
1441	-3.42355535747391e-10\\
1442	-3.23230673920218e-10\\
1443	-3.18845933357438e-10\\
1444	-3.18845933357377e-10\\
1445	-3.12562608631592e-10\\
1446	-3.12562608631535e-10\\
1447	-3.08568385610196e-10\\
1448	-3.08568385610141e-10\\
1449	-3.04728972832318e-10\\
1450	-3.01043617760366e-10\\
1451	-2.99258484955039e-10\\
1452	-2.99258484954989e-10\\
1453	-2.92387722754243e-10\\
1454	-2.85899858699581e-10\\
1455	-2.8279791218048e-10\\
1456	-2.82797912180436e-10\\
1457	-2.71322723728758e-10\\
1458	-2.68684264514149e-10\\
1459	-2.68684264514112e-10\\
1460	-2.66136879194771e-10\\
1461	-2.66136879194736e-10\\
1462	-2.63680068514875e-10\\
1463	-2.61313350889051e-10\\
1464	-2.56848356830251e-10\\
1465	-2.56848356830118e-10\\
1466	-2.56848356830088e-10\\
1467	-2.48980246685014e-10\\
1468	-2.47232169502878e-10\\
1469	-2.47232169502854e-10\\
1470	-2.40830725835927e-10\\
1471	-2.39359350989314e-10\\
1472	-2.39359350989294e-10\\
1473	-2.38642807153703e-10\\
1474	-2.38642807153683e-10\\
1475	-2.37938978693375e-10\\
1476	-2.37247831111954e-10\\
1477	-2.35903443739996e-10\\
1478	-2.3336539085241e-10\\
1479	-2.32762095748708e-10\\
1480	-2.32762095748691e-10\\
1481	-2.29728728860019e-10\\
1482	-2.29728728860002e-10\\
1483	-2.26596897247448e-10\\
1484	-2.2659689724743e-10\\
1485	-2.23362764008284e-10\\
1486	-2.20022366924878e-10\\
1487	-2.20022366924839e-10\\
1488	-2.200223669248e-10\\
1489	-2.13006276405633e-10\\
1490	-2.13006276405566e-10\\
1491	-2.13006276405545e-10\\
1492	-2.05514232808178e-10\\
1493	-2.05514232808106e-10\\
1494	-2.05514232808084e-10\\
1495	-1.98395310594579e-10\\
1496	-1.98395310594556e-10\\
1497	-1.97585021993062e-10\\
1498	-1.97585021993039e-10\\
1499	-1.96770729779622e-10\\
1500	-1.95952394041307e-10\\
1501	-1.94303431395428e-10\\
1502	-1.90955205948383e-10\\
1503	-1.89255286860669e-10\\
1504	-1.89255286860645e-10\\
1505	-1.82275752940353e-10\\
1506	-1.80484195039721e-10\\
1507	-1.80484195039695e-10\\
1508	-1.75963464353764e-10\\
1509	-1.75963464353739e-10\\
1510	-1.73232542580327e-10\\
1511	-1.73232542580302e-10\\
1512	-1.71403484196574e-10\\
1513	-1.71403484196548e-10\\
1514	-1.69567252661401e-10\\
1515	-1.67723488067394e-10\\
1516	-1.6401191262041e-10\\
1517	-1.62143374260207e-10\\
1518	-1.6214337426018e-10\\
1519	-1.54575650043194e-10\\
1520	-1.52658472155032e-10\\
1521	-1.52658472155005e-10\\
1522	-1.45860442273196e-10\\
1523	-1.45860442273168e-10\\
1524	-1.42902282882489e-10\\
1525	-1.42902282882461e-10\\
1526	-1.39955659560897e-10\\
1527	-1.37059518239973e-10\\
1528	-1.3327431958075e-10\\
1529	-1.33274319580723e-10\\
1530	-1.27754068281914e-10\\
1531	-1.24174723916463e-10\\
1532	-1.24174723916438e-10\\
1533	-1.18949925682811e-10\\
1534	-1.18949925682787e-10\\
1535	-1.15558889660856e-10\\
1536	-1.15558889660832e-10\\
1537	-1.12253420963083e-10\\
1538	-1.11442453783596e-10\\
1539	-1.11442453783573e-10\\
1540	-1.08258992871902e-10\\
1541	-1.074780346073e-10\\
1542	-1.07478034607278e-10\\
1543	-1.04412681343952e-10\\
1544	-1.0143910349225e-10\\
1545	-9.99859989891013e-11\\
1546	-9.99859989890808e-11\\
1547	-9.64064378297693e-11\\
1548	-9.64064378297491e-11\\
1549	-9.28749403236646e-11\\
1550	-9.28749403236446e-11\\
1551	-8.93871799210329e-11\\
1552	-8.5938883679231e-11\\
1553	-8.59388836792073e-11\\
1554	-8.59388836791832e-11\\
1555	-7.91438284754363e-11\\
1556	-7.91438284754031e-11\\
1557	-7.91438284753839e-11\\
1558	-7.645780593555e-11\\
1559	-7.6457805935531e-11\\
1560	-7.37869023263404e-11\\
1561	-7.24564652548595e-11\\
1562	-7.24564652548406e-11\\
1563	-6.98486215724684e-11\\
1564	-6.73393245773694e-11\\
1565	-6.61210132751789e-11\\
1566	-6.61210132751618e-11\\
1567	-6.31793078884666e-11\\
1568	-6.31793078884503e-11\\
1569	-6.03834166375941e-11\\
1570	-6.03834166375786e-11\\
1571	-5.87744404241679e-11\\
1572	-5.87744404241529e-11\\
1573	-5.72160112464776e-11\\
1574	-5.5707441810096e-11\\
1575	-5.52155496532158e-11\\
1576	-5.52155496532019e-11\\
1577	-5.23814032688593e-11\\
1578	-5.06025111122407e-11\\
1579	-5.06025111122284e-11\\
1580	-4.80964939411661e-11\\
1581	-4.65321195903187e-11\\
1582	-4.65321195903079e-11\\
1583	-4.57813573832577e-11\\
1584	-4.57813573832472e-11\\
1585	-4.5051347592427e-11\\
1586	-4.43419471298421e-11\\
1587	-4.2984422022586e-11\\
1588	-4.29844220225046e-11\\
1589	-4.29844220224952e-11\\
1590	-4.05108331380188e-11\\
1591	-3.99420275868978e-11\\
1592	-3.99420275868898e-11\\
1593	-3.78754168441444e-11\\
1594	-3.74112598503283e-11\\
1595	-3.74112598503218e-11\\
1596	-3.71869906953979e-11\\
1597	-3.71869906953916e-11\\
1598	-3.69679130330945e-11\\
1599	-3.67540161257749e-11\\
1600	-3.63417229054475e-11\\
1601	-3.63417229054384e-11\\
1602	-3.55788613072727e-11\\
1603	-3.54009503697639e-11\\
1604	-3.54009503697589e-11\\
1605	-3.47402629276464e-11\\
1606	-3.41607260506438e-11\\
1607	-3.39012445981515e-11\\
1608	-3.3901244598148e-11\\
1609	-3.29710878000729e-11\\
1610	-3.27594516627741e-11\\
1611	-3.27594516627711e-11\\
1612	-3.1995274731182e-11\\
1613	-3.18246351732874e-11\\
1614	-3.1824635173285e-11\\
1615	-3.12226921119543e-11\\
1616	-3.10922126581493e-11\\
1617	-3.10922126581475e-11\\
1618	-3.06495527414723e-11\\
1619	-3.06030916286865e-11\\
1620	-3.06030916286852e-11\\
1621	-3.05585937818203e-11\\
1622	-3.0558593781819e-11\\
1623	-3.05151471777003e-11\\
1624	-3.04718398434596e-11\\
1625	-3.03856345030603e-11\\
1626	-3.0214802134267e-11\\
1627	-3.01301416209328e-11\\
1628	-3.01301416209316e-11\\
1629	-2.99205613771692e-11\\
1630	-2.9920561377168e-11\\
1631	-2.97137347810007e-11\\
1632	-2.97137347809996e-11\\
1633	-2.94918688888438e-11\\
1634	-2.93430001011814e-11\\
1635	-2.93430001011799e-11\\
1636	-2.92371523332003e-11\\
1637	-2.92371523331988e-11\\
1638	-2.91259989847081e-11\\
1639	-2.90095182691229e-11\\
1640	-2.87604823656372e-11\\
1641	-2.86278783642711e-11\\
1642	-2.86278783642691e-11\\
1643	-2.80429364716247e-11\\
1644	-2.7882926212907e-11\\
1645	-2.78829262129047e-11\\
1646	-2.72788028968254e-11\\
1647	-2.72788028968228e-11\\
1648	-2.69986441233585e-11\\
1649	-2.69986441233558e-11\\
1650	-2.67075811939677e-11\\
1651	-2.64075066045443e-11\\
1652	-2.59931597019167e-11\\
1653	-2.59931597019137e-11\\
1654	-2.53404442859146e-11\\
1655	-2.48840064176546e-11\\
1656	-2.48840064176513e-11\\
1657	-2.42888754296832e-11\\
1658	-2.42888754296798e-11\\
1659	-2.41665134923125e-11\\
1660	-2.41665134923091e-11\\
1661	-2.40430257795003e-11\\
1662	-2.3918406240092e-11\\
1663	-2.36657472001881e-11\\
1664	-2.36657472001818e-11\\
1665	-2.31533885244295e-11\\
1666	-2.26358495065109e-11\\
1667	-2.23750107275294e-11\\
1668	-2.23750107275256e-11\\
1669	-2.13166641911943e-11\\
1670	-2.10480704005008e-11\\
1671	-2.1048070400497e-11\\
1672	-2.03689988799489e-11\\
1673	-2.0368998879945e-11\\
1674	-1.96784215558526e-11\\
1675	-1.9678421555842e-11\\
1676	-1.9678421555838e-11\\
1677	-1.89863152201799e-11\\
1678	-1.83026547957653e-11\\
1679	-1.83026547957475e-11\\
1680	-1.83026547957436e-11\\
1681	-1.69573307239781e-11\\
1682	-1.69573307239717e-11\\
1683	-1.69573307239679e-11\\
1684	-1.6426243528226e-11\\
1685	-1.64262435282223e-11\\
1686	-1.58985543070182e-11\\
1687	-1.56358545427046e-11\\
1688	-1.56358545427009e-11\\
1689	-1.51124872472867e-11\\
1690	-1.49820313943769e-11\\
1691	-1.49820313943732e-11\\
1692	-1.44615600881362e-11\\
1693	-1.43317484088412e-11\\
1694	-1.43317484088375e-11\\
1695	-1.38234373999194e-11\\
1696	-1.3336287516044e-11\\
1697	-1.31005279929804e-11\\
1698	-1.31005279929771e-11\\
1699	-1.27565276623758e-11\\
1700	-1.27565276623726e-11\\
1701	-1.24239596051431e-11\\
1702	-1.21026771481382e-11\\
1703	-1.19980662745821e-11\\
1704	-1.19980662745791e-11\\
1705	-1.13961198041881e-11\\
1706	-1.10189589912312e-11\\
1707	-1.10189589912286e-11\\
1708	-1.04876507654436e-11\\
1709	-1.01555365661971e-11\\
1710	-1.01555365661948e-11\\
1711	-9.99600791483577e-12\\
1712	-9.99600791483129e-12\\
1713	-9.84078958892054e-12\\
1714	-9.68985116437656e-12\\
1715	-9.40069651480981e-12\\
1716	-9.40069651479404e-12\\
1717	-9.40069651479005e-12\\
1718	-8.87249982303307e-12\\
1719	-8.75073860929074e-12\\
1720	-8.75073860928734e-12\\
1721	-8.45998891038452e-12\\
1722	-8.45998891038131e-12\\
1723	-8.18612672482141e-12\\
1724	-8.18612672481839e-12\\
1725	-8.1333509041531e-12\\
1726	-8.13335090415012e-12\\
1727	-8.08123451252534e-12\\
1728	-8.02977499611054e-12\\
1729	-7.9288165348679e-12\\
1730	-7.7346733091981e-12\\
1731	-7.68774310306373e-12\\
1732	-7.68774310306108e-12\\
1733	-7.33506821207475e-12\\
1734	-7.25314466495094e-12\\
1735	-7.25314466494865e-12\\
1736	-6.94104301547849e-12\\
1737	-6.86628276000688e-12\\
1738	-6.86628276000477e-12\\
};
\end{axis}
\end{tikzpicture}%}
  \caption{The angular displacement of pendulum $P_2$ as a function of time.
    \texttt{Blue}: $C_2 = 6$ ms, \texttt{Red}: $C_2 = 10$ ms}
  \label{fig:01.5.6_10.2}
\end{figure}

\begin{figure}[H]\centering
  \scalebox{1}{% This file was created by matlab2tikz.
%
%The latest updates can be retrieved from
%  http://www.mathworks.com/matlabcentral/fileexchange/22022-matlab2tikz-matlab2tikz
%where you can also make suggestions and rate matlab2tikz.
%
\definecolor{mycolor1}{rgb}{0.00000,0.44700,0.74100}%
\definecolor{mycolor2}{rgb}{0.85000,0.32500,0.09800}%
%
\begin{tikzpicture}

\begin{axis}[%
width=4.133in,
height=3.26in,
at={(0.693in,0.44in)},
scale only axis,
xmin=0,
xmax=1.5,
xmajorgrids,
ymin=-0.1,
ymax=0.15,
ymajorgrids,
axis background/.style={fill=white}
]
\addplot [color=mycolor1,solid,forget plot]
  table[row sep=crcr]{%
0	0.10153\\
3.15544362088405e-30	0.10153\\
0.000656101980281985	0.101530709989553\\
0.00393661188169191	0.101555560666546\\
0.00599999999999994	0.10158938182706\\
0.006	0.10158938182706\\
0.012	0.101767596768679\\
0.0120000000000001	0.101767596768679\\
0.018	0.102064853328018\\
0.0180000000000001	0.102064853328018\\
0.0199999999999998	0.10213229397037\\
0.02	0.10213229397037\\
0.026	0.101716366820929\\
0.0260000000000002	0.101716366820929\\
0.0289999999999998	0.10116051629751\\
0.029	0.10116051629751\\
0.0319999999999996	0.100372547032607\\
0.0349999999999991	0.0993522286136822\\
0.035	0.0993522286136819\\
0.0399999999999996	0.0971345253230206\\
0.04	0.0971345253230203\\
0.0449999999999996	0.0942687495276435\\
0.0459999999999996	0.0936176301011948\\
0.046	0.0936176301011944\\
0.047	0.0929404751960583\\
0.0470000000000004	0.092940475196058\\
0.0490000000000003	0.0915346663918733\\
0.0510000000000002	0.0900778367053162\\
0.055	0.087010350712579\\
0.0579999999999996	0.0845743497767112\\
0.058	0.0845743497767108\\
0.0599999999999996	0.082885486841699\\
0.06	0.0828854868416986\\
0.0619999999999995	0.0811444789878217\\
0.0639999999999991	0.0793510999448183\\
0.0659999999999991	0.0775051166450086\\
0.066	0.0775051166450078\\
0.0699999999999991	0.0736543707951944\\
0.07	0.0736543707951936\\
0.0700000000000009	0.0736543707951927\\
0.074	0.0695902396041448\\
0.076	0.067477498629349\\
0.0760000000000009	0.0674774986293481\\
0.08	0.063214520109214\\
0.0800000000000009	0.0632145201092131\\
0.0839999999999999	0.0589831831846656\\
0.086	0.0568786923069218\\
0.0860000000000009	0.0568786923069209\\
0.0869999999999991	0.0558291214523434\\
0.087	0.0558291214523424\\
0.0880000000000004	0.0547812881682954\\
0.0890000000000009	0.053735158410586\\
0.0910000000000017	0.0516478735747052\\
0.0929999999999991	0.0495669956809782\\
0.093	0.0495669956809773\\
0.0970000000000017	0.0454233797984614\\
0.0999999999999991	0.0423304804899824\\
0.1	0.0423304804899815\\
0.104000000000002	0.0382246453687673\\
0.104999999999999	0.037201183877005\\
0.105	0.037201183877004\\
0.105999999999999	0.0361788547394758\\
0.106	0.0361788547394749\\
0.106999999999999	0.0351576247414335\\
0.107999999999998	0.0341374607030283\\
0.109999999999997	0.032100197959437\\
0.111999999999999	0.0300668017458478\\
0.112	0.0300668017458469\\
0.115999999999997	0.0260810521578851\\
0.115999999999998	0.0260810521578834\\
0.116	0.0260810521578817\\
0.119999999999997	0.0222486397395016\\
0.119999999999998	0.0222486397395\\
0.12	0.0222486397394984\\
0.123999999999997	0.0185675721876527\\
0.125999999999999	0.0167831916823141\\
0.126	0.0167831916823133\\
0.127999999999998	0.0150359358757995\\
0.128	0.015035935875798\\
0.129999999999998	0.0133255776955832\\
0.131999999999996	0.0116518948633202\\
0.135999999999993	0.00841368993471907\\
0.139999999999998	0.00531963767383442\\
0.14	0.00531963767383307\\
0.144999999999998	0.00165234374314154\\
0.145	0.00165234374314027\\
0.145999999999998	0.000945347349630184\\
0.146	0.000945347349628936\\
0.146999999999999	0.000247117489500776\\
0.147999999999998	-0.000442368523203164\\
0.149999999999997	-0.00179519832772279\\
0.151999999999998	-0.00311331787782769\\
0.152	-0.00311331787782884\\
0.155999999999997	-0.00564610693435692\\
0.157999999999998	-0.00686110560234216\\
0.158	-0.00686110560234323\\
0.16	-0.00803453236788899\\
0.160000000000002	-0.00803453236789001\\
0.162000000000002	-0.00915901971699857\\
0.164000000000002	-0.0102347137882067\\
0.166	-0.01126175437883\\
0.166000000000002	-0.0112617543788309\\
0.170000000000002	-0.013170402707014\\
0.174	-0.0148859569244241\\
0.174000000000001	-0.0148859569244248\\
0.175	-0.0152847812356085\\
0.175000000000002	-0.0152847812356092\\
0.176000000000001	-0.0156716063502979\\
0.177	-0.0160464448364982\\
0.178999999999998	-0.0167602102479289\\
0.179999999999998	-0.0170991603633261\\
0.18	-0.0170991603633267\\
0.183999999999997	-0.0183356626541074\\
0.186	-0.0188824741501191\\
0.186000000000002	-0.0188824741501196\\
0.189999999999998	-0.0198335731693998\\
0.192	-0.0202379842976751\\
0.192000000000002	-0.0202379842976754\\
0.195999999999998	-0.0209395293867196\\
0.199999999999995	-0.0215214927991707\\
0.199999999999997	-0.021521492799171\\
0.2	-0.0215214927991714\\
0.202999999999998	-0.0218796742141849\\
0.203	-0.0218796742141851\\
0.205999999999998	-0.022170867118344\\
0.206	-0.0221708671183442\\
0.208999999999998	-0.0223951566640238\\
0.209999999999998	-0.0224550643318199\\
0.21	-0.02245506433182\\
0.211999999999998	-0.0225526084365377\\
0.212	-0.0225526084365378\\
0.213999999999998	-0.0226204677288109\\
0.215999999999997	-0.0226586510276281\\
0.217999999999998	-0.022667163295306\\
0.218	-0.022667163295306\\
0.219999999999998	-0.0226554236175242\\
0.22	-0.0226554236175242\\
0.221999999999998	-0.0226328484480922\\
0.223999999999996	-0.0225994348531525\\
0.225999999999998	-0.0225551784902755\\
0.226	-0.0225551784902754\\
0.229999999999996	-0.022434113044661\\
0.231999999999998	-0.0223572882282681\\
0.232	-0.022357288228268\\
0.235999999999996	-0.0221710044863672\\
0.237999999999998	-0.0220615213514281\\
0.238	-0.022061521351428\\
0.239999999999998	-0.0219411255414157\\
0.24	-0.0219411255414156\\
0.241999999999998	-0.0218098014097432\\
0.243999999999996	-0.0216675318895284\\
0.245	-0.0215922868864084\\
0.245000000000002	-0.0215922868864082\\
0.245999999999998	-0.0215142984913975\\
0.246	-0.0215142984913974\\
0.246999999999999	-0.0214335641707048\\
0.247999999999998	-0.0213500813012717\\
0.249999999999997	-0.0211748589774669\\
0.252	-0.0209886087480779\\
0.252000000000003	-0.0209886087480776\\
0.256	-0.0206014286709987\\
0.259999999999997	-0.0202068421468002\\
0.26	-0.0202068421467999\\
0.260999999999996	-0.0201070143055804\\
0.261	-0.02010701430558\\
0.261999999999998	-0.0200067075173039\\
0.262999999999996	-0.0199059185225771\\
0.264999999999993	-0.0197028807995861\\
0.265999999999997	-0.0196006254746493\\
0.266	-0.019600625474649\\
0.269999999999993	-0.019186616711248\\
0.271999999999997	-0.0189765727153971\\
0.272	-0.0189765727153967\\
0.275999999999993	-0.018550269005692\\
0.279999999999986	-0.018115492540964\\
0.279999999999993	-0.0181154925409633\\
0.28	-0.0181154925409625\\
0.285999999999996	-0.0174469457727508\\
0.286	-0.0174469457727504\\
0.289999999999996	-0.0169899886525383\\
0.29	-0.0169899886525379\\
0.291999999999996	-0.0167580436729622\\
0.292	-0.0167580436729618\\
0.293999999999996	-0.0165260501331934\\
0.295999999999993	-0.0162962803602752\\
0.297999999999996	-0.0160687044933052\\
0.298	-0.0160687044933048\\
0.299999999999996	-0.0158432929566358\\
0.3	-0.0158432929566354\\
0.301999999999996	-0.0156200164558907\\
0.303999999999993	-0.015398845974034\\
0.305999999999996	-0.015179752767728\\
0.306	-0.0151797527677276\\
0.309999999999993	-0.0147476845551074\\
0.313999999999986	-0.0143235872046597\\
0.314999999999997	-0.0142187823880662\\
0.315	-0.0142187823880658\\
0.318999999999997	-0.013804339052661\\
0.319	-0.0138043390526606\\
0.319999999999996	-0.0137019053706168\\
0.32	-0.0137019053706164\\
0.320999999999998	-0.0135999358702287\\
0.321999999999996	-0.0134984272384985\\
0.323999999999993	-0.0132967794039424\\
0.325999999999996	-0.013096935660776\\
0.326	-0.0130969356607756\\
0.329999999999993	-0.0127025567933712\\
0.331	-0.0126050493319295\\
0.331000000000004	-0.0126050493319291\\
0.333	-0.01241131689032\\
0.333000000000004	-0.0124113168903197\\
0.335	-0.0122194776563815\\
0.336999999999996	-0.0120297108893724\\
0.339999999999996	-0.0117488927389454\\
0.34	-0.0117488927389451\\
0.343999999999993	-0.0113814942163562\\
0.345999999999997	-0.0112007521713454\\
0.346	-0.0112007521713451\\
0.347999999999997	-0.0110219501998974\\
0.348	-0.0110219501998971\\
0.349999999999997	-0.0108450650649932\\
0.35	-0.0108450650649929\\
0.351999999999997	-0.0106700737786224\\
0.353999999999993	-0.0104969535989516\\
0.354	-0.010496953598951\\
0.357999999999993	-0.0101562368052926\\
0.359999999999996	-0.0099885959117665\\
0.36	-0.00998859591176621\\
0.363999999999993	-0.00966025817714279\\
0.365999999999996	-0.00949992314743398\\
0.366	-0.00949992314743369\\
0.369999999999993	-0.00918681657515862\\
0.373999999999986	-0.00888365955520996\\
0.376999999999997	-0.00866272618535034\\
0.377	-0.00866272618535008\\
0.379999999999997	-0.00844723608155697\\
0.38	-0.00844723608155672\\
0.382999999999996	-0.00823712623294379\\
0.384999999999997	-0.00810001162597915\\
0.385	-0.0081000116259789\\
0.385999999999997	-0.00803233519913159\\
0.386	-0.00803233519913136\\
0.386999999999998	-0.00796524308531946\\
0.387999999999996	-0.00789873310473264\\
0.388999999999997	-0.00783280309648106\\
0.389	-0.00783280309648082\\
0.390999999999997	-0.00770267444758192\\
0.392999999999993	-0.00757484022712702\\
0.394999999999997	-0.00744928382176812\\
0.395	-0.0074492838217679\\
0.398999999999993	-0.00720465294344195\\
0.399999999999997	-0.00714480416393434\\
0.4	-0.00714480416393413\\
0.403999999999993	-0.00691058647064839\\
0.405999999999997	-0.00679655750351996\\
0.406	-0.00679655750351975\\
0.409999999999993	-0.00657458523059078\\
0.411999999999997	-0.00646661307724917\\
0.412	-0.00646661307724898\\
0.415999999999993	-0.00625662657491753\\
0.419999999999986	-0.00605449327202308\\
0.419999999999996	-0.00605449327202254\\
0.42	-0.00605449327202236\\
0.426	-0.00576578930604481\\
0.426000000000004	-0.00576578930604465\\
0.432000000000004	-0.00549418184424848\\
0.432000000000007	-0.00549418184424832\\
0.434999999999997	-0.00536439922956594\\
0.435	-0.00536439922956578\\
0.43799999999999	-0.00523819302221601\\
0.439999999999997	-0.00515602417639017\\
0.44	-0.00515602417639002\\
0.44299999999999	-0.00503569714935346\\
0.445999999999979	-0.00491885041251289\\
0.445999999999995	-0.00491885041251229\\
0.446	-0.00491885041251209\\
0.447	-0.00488066896335491\\
0.447000000000004	-0.00488066896335477\\
0.448000000000004	-0.00484286916798622\\
0.449000000000004	-0.00480544979823171\\
0.451000000000004	-0.00473174748491064\\
0.454999999999997	-0.00458885529620634\\
0.455	-0.00458885529620622\\
0.459	-0.00445191811386207\\
0.459999999999997	-0.00441860587821894\\
0.46	-0.00441860587821882\\
0.463999999999997	-0.00428901270545446\\
0.464	-0.00428901270545435\\
0.465999999999997	-0.00422639488576227\\
0.466	-0.00422639488576216\\
0.466999999999997	-0.00419562707643343\\
0.467	-0.00419562707643332\\
0.467999999999998	-0.00416513880253127\\
0.468999999999997	-0.00413484921200659\\
0.470999999999993	-0.00407486215111517\\
0.472999999999997	-0.00401565809783476\\
0.473	-0.00401565809783466\\
0.476999999999993	-0.00389956833836478\\
0.479999999999997	-0.00381449991875875\\
0.48	-0.00381449991875865\\
0.483999999999993	-0.00370369326502329\\
0.485999999999997	-0.00364939567002454\\
0.486	-0.00364939567002445\\
0.489999999999993	-0.00354297666038756\\
0.49	-0.00354297666038737\\
0.490000000000004	-0.00354297666038727\\
0.492999999999997	-0.00346503937862841\\
0.493	-0.00346503937862832\\
0.495999999999993	-0.00338868556746301\\
0.498999999999986	-0.00331389289999764\\
0.498999999999993	-0.00331389289999746\\
0.499	-0.00331389289999729\\
0.499999999999997	-0.00328930513852606\\
0.5	-0.00328930513852597\\
0.500999999999998	-0.00326488760491461\\
0.501999999999997	-0.00324063950580012\\
0.503999999999993	-0.00319264846528835\\
0.505999999999993	-0.00314532578007042\\
0.506	-0.00314532578007025\\
0.507999999999997	-0.00309866530011343\\
0.508000000000004	-0.00309866530011327\\
0.51	-0.00305261406543181\\
0.511999999999997	-0.0030071191951903\\
0.51599999999999	-0.00291777496954556\\
0.519999999999993	-0.00283058617662023\\
0.52	-0.00283058617662008\\
0.521999999999993	-0.00278778586013303\\
0.522	-0.00278778586013288\\
0.523999999999993	-0.00274550749134963\\
0.524999999999993	-0.00272456231758125\\
0.525	-0.0027245623175811\\
0.526	-0.00270374557549404\\
0.526000000000007	-0.0027037455754939\\
0.527000000000007	-0.00268305658875907\\
0.528000000000007	-0.00266249468519739\\
0.530000000000007	-0.00262174945950128\\
0.532	-0.00258150460315236\\
0.532000000000007	-0.00258150460315222\\
0.536000000000007	-0.00250249514204711\\
0.538	-0.00246372026921363\\
0.538000000000007	-0.00246372026921349\\
0.539999999999993	-0.00242551753421773\\
0.54	-0.00242551753421759\\
0.541999999999986	-0.00238797427817236\\
0.543999999999972	-0.00235108562195193\\
0.546	-0.00231484677150225\\
0.546000000000007	-0.00231484677150212\\
0.549999999999979	-0.00224429973343307\\
0.550999999999993	-0.0022270618451542\\
0.551	-0.00222706184515408\\
0.554999999999972	-0.0021596889749526\\
0.556999999999993	-0.00212694210620408\\
0.557	-0.00212694210620396\\
0.559999999999993	-0.00207898426298481\\
0.56	-0.0020789842629847\\
0.562999999999993	-0.00203240893645441\\
0.565999999999986	-0.00198720250735346\\
0.565999999999993	-0.00198720250735335\\
0.566	-0.00198720250735324\\
0.571999999999986	-0.00190084386197493\\
0.571999999999993	-0.00190084386197483\\
0.572	-0.00190084386197473\\
0.577999999999986	-0.00181923712544194\\
0.579999999999993	-0.00179294751424575\\
0.58	-0.00179294751424566\\
0.585999999999986	-0.00171676874852746\\
0.585999999999993	-0.00171676874852737\\
0.586	-0.00171676874852728\\
0.591999999999986	-0.00164455638742728\\
0.591999999999993	-0.0016445563874272\\
0.592	-0.00164455638742711\\
0.594999999999993	-0.00160991103328873\\
0.595	-0.00160991103328864\\
0.597999999999993	-0.00157622596123363\\
0.599999999999993	-0.00155429786089437\\
0.6	-0.00155429786089429\\
0.602999999999993	-0.00152219150742449\\
0.605999999999986	-0.00149101978560816\\
0.606	-0.00149101978560801\\
0.606999999999993	-0.00148083532431449\\
0.607	-0.00148083532431442\\
0.607999999999999	-0.00147072972894193\\
0.608999999999997	-0.00146067903498334\\
0.609000000000004	-0.00146067903498327\\
0.611	-0.00144074104690527\\
0.612999999999997	-0.00142101876894327\\
0.614999999999997	-0.00140150963798814\\
0.615000000000004	-0.00140150963798808\\
0.618999999999997	-0.00136312070288643\\
0.619999999999993	-0.00135365275748734\\
0.62	-0.00135365275748728\\
0.623999999999993	-0.00131628890083929\\
0.625999999999993	-0.00129790748440702\\
0.626	-0.00129790748440696\\
0.629999999999993	-0.00126173373081034\\
0.63	-0.00126173373081028\\
0.633999999999993	-0.0012263297876626\\
0.635999999999993	-0.0012089107280837\\
0.636	-0.00120891072808364\\
0.637999999999993	-0.00119167546743098\\
0.638	-0.00119167546743092\\
0.639999999999993	-0.00117461998325188\\
0.64	-0.00117461998325182\\
0.641999999999993	-0.00115774205901393\\
0.643999999999986	-0.00114103950126541\\
0.645999999999993	-0.00112451013934017\\
0.646	-0.00112451013934012\\
0.649999999999986	-0.00109196243260266\\
0.65	-0.00109196243260255\\
0.650000000000007	-0.00109196243260249\\
0.653999999999993	-0.00106008201867179\\
0.657999999999979	-0.00102885232431956\\
0.659999999999993	-0.00101347641048094\\
0.66	-0.00101347641048088\\
0.664999999999993	-0.000975717507553919\\
0.665	-0.000975717507553866\\
0.665999999999993	-0.000968280484185491\\
0.666	-0.000968280484185439\\
0.666999999999998	-0.000960881146542803\\
0.667000000000006	-0.000960881146542751\\
0.668000000000004	-0.000953519254216742\\
0.669000000000002	-0.000946194568021774\\
0.670999999999998	-0.000931655863318971\\
0.673000000000005	-0.000917263142982852\\
0.673000000000013	-0.000917263142982801\\
0.677000000000005	-0.000888908192260236\\
0.678	-0.000881907794811557\\
0.678000000000007	-0.000881907794811507\\
0.679999999999993	-0.000868030488427701\\
0.68	-0.000868030488427652\\
0.681999999999986	-0.000854329050333715\\
0.683999999999972	-0.000840801699889314\\
0.686	-0.000827446679078431\\
0.686000000000007	-0.000827446679078384\\
0.689999999999979	-0.000801246706084755\\
0.69399999999995	-0.000775715511132179\\
0.695999999999993	-0.000763196544275791\\
0.696	-0.000763196544275747\\
0.699999999999993	-0.000738643736996597\\
0.7	-0.000738643736996554\\
0.703999999999993	-0.000714727163342571\\
0.705999999999993	-0.000703003565966427\\
0.706	-0.000703003565966385\\
0.707999999999993	-0.000691434390009101\\
0.708	-0.000691434390009061\\
0.709999999999993	-0.00068001813194603\\
0.711999999999986	-0.000668753308119149\\
0.713999999999993	-0.000657638454550774\\
0.714	-0.000657638454550734\\
0.717999999999986	-0.000635973497491851\\
0.719999999999993	-0.000625450726921634\\
0.72	-0.000625450726921597\\
0.723999999999986	-0.000605017763848642\\
0.724999999999993	-0.000600036172966322\\
0.725	-0.000600036172966287\\
0.725999999999993	-0.000595104915874549\\
0.726	-0.000595104915874515\\
0.726999999999999	-0.00059022383236026\\
0.727999999999997	-0.000585392763836592\\
0.729999999999993	-0.000575880045541254\\
0.731999999999993	-0.000566565524714689\\
0.732	-0.000566565524714657\\
0.734999999999993	-0.000552962722743504\\
0.735	-0.000552962722743472\\
0.737999999999993	-0.00053979916978624\\
0.74	-0.000531265588952171\\
0.740000000000007	-0.000531265588952141\\
0.743	-0.000518825626820687\\
0.745999999999993	-0.00050681493155342\\
0.746000000000007	-0.000506814931553365\\
0.746999999999993	-0.000502906148469695\\
0.747	-0.000502906148469668\\
0.747999999999999	-0.00049903455798491\\
0.748999999999997	-0.000495190048830314\\
0.750999999999993	-0.000487581775759804\\
0.753999999999993	-0.000476369381836014\\
0.754	-0.000476369381835987\\
0.757999999999993	-0.000461787803823608\\
0.759999999999993	-0.000454652713606544\\
0.76	-0.000454652713606519\\
0.763999999999993	-0.000440689294263871\\
0.766	-0.000433859150450819\\
0.766000000000007	-0.000433859150450795\\
0.77	-0.000420497556213229\\
0.770000000000007	-0.000420497556213206\\
0.774	-0.000407528452914783\\
0.776	-0.000401188970596003\\
0.776000000000007	-0.000401188970595981\\
0.779999999999993	-0.000388767641396413\\
0.78	-0.000388767641396391\\
0.782999999999993	-0.000379661485049944\\
0.783	-0.000379661485049923\\
0.785999999999993	-0.000370732228558406\\
0.786000000000001	-0.000370732228558385\\
0.788999999999994	-0.00036197726088567\\
0.791999999999987	-0.000353394021960795\\
0.792	-0.000353394021960756\\
0.792000000000008	-0.000353394021960736\\
0.797999999999994	-0.000336732740416052\\
0.799999999999993	-0.000331326016780796\\
0.8	-0.000331326016780777\\
0.804999999999993	-0.000318124178754444\\
0.805000000000001	-0.000318124178754425\\
0.805999999999993	-0.000315537118742536\\
0.806	-0.000315537118742517\\
0.806999999999994	-0.000312967630466049\\
0.807999999999987	-0.00031041563044068\\
0.809999999999973	-0.000305363764053996\\
0.811999999999993	-0.000300380863041349\\
0.812	-0.000300380863041332\\
0.815999999999973	-0.00029062570780146\\
0.817999999999993	-0.000285853768762768\\
0.818000000000001	-0.000285853768762752\\
0.819999999999993	-0.000281151425523863\\
0.82	-0.000281151425523846\\
0.821999999999993	-0.000276518066970423\\
0.823999999999986	-0.000271953090950688\\
0.825999999999993	-0.000267455904199894\\
0.826	-0.000267455904199878\\
0.829999999999986	-0.000258662569431718\\
0.833999999999972	-0.000250133491395636\\
0.839999999999993	-0.000237825693550472\\
0.84	-0.000237825693550457\\
0.840999999999993	-0.000235830212485902\\
0.841000000000001	-0.000235830212485888\\
0.841999999999994	-0.000233850504990185\\
0.842999999999987	-0.000231886506733272\\
0.844999999999973	-0.00022800538320537\\
0.845999999999993	-0.000226088131837057\\
0.846	-0.000226088131837043\\
0.849999999999973	-0.000218573081396153\\
0.851999999999993	-0.000214907076367157\\
0.852	-0.000214907076367144\\
0.855999999999973	-0.00020775572446821\\
0.857999999999993	-0.000204269448207704\\
0.858	-0.000204269448207691\\
0.86	-0.000200852785067654\\
0.860000000000007	-0.000200852785067642\\
0.862000000000007	-0.000197515911023471\\
0.864000000000007	-0.00019425839241462\\
0.866	-0.00019107980589366\\
0.866000000000007	-0.000191079805893649\\
0.87	-0.000184957786970514\\
0.870000000000007	-0.000184957786970504\\
0.874	-0.000179146671684028\\
0.874999999999994	-0.000177742114009642\\
0.875000000000001	-0.000177742114009632\\
0.876	-0.00017635675258845\\
0.876000000000007	-0.00017635675258844\\
0.877000000000007	-0.000174990542411077\\
0.878000000000006	-0.000173643439089208\\
0.879999999999998	-0.00017100637856345\\
0.880000000000006	-0.000171006378563441\\
0.882000000000005	-0.00016844522829926\\
0.884000000000004	-0.000165959655450069\\
0.886000000000005	-0.000163549336990568\\
0.886000000000013	-0.000163549336990559\\
0.888000000000007	-0.000161213959677009\\
0.888000000000014	-0.000161213959677\\
0.890000000000009	-0.000158932560524497\\
0.892000000000004	-0.000156684183561721\\
0.895999999999993	-0.000152285331664397\\
0.898999999999993	-0.000149070733578774\\
0.899000000000001	-0.000149070733578767\\
0.899999999999993	-0.000148015117189559\\
0.9	-0.000148015117189551\\
0.900999999999994	-0.00014696740158166\\
0.901999999999987	-0.000145927552709866\\
0.903999999999973	-0.000143871320291598\\
0.905999999999993	-0.000141846152706551\\
0.906	-0.000141846152706543\\
0.909999999999973	-0.000137887963280688\\
0.909999999999987	-0.000137887963280675\\
0.910000000000001	-0.000137887963280661\\
0.910999999999993	-0.000136917425049025\\
0.911000000000001	-0.000136917425049018\\
0.911999999999994	-0.000135954427033275\\
0.912999999999987	-0.000134998937945522\\
0.914999999999973	-0.000133110362622773\\
0.916999999999993	-0.000131251453641309\\
0.917000000000001	-0.000131251453641302\\
0.919999999999993	-0.00012850837738974\\
0.92	-0.000128508377389734\\
0.922999999999993	-0.000125811086121122\\
0.925999999999986	-0.000123158791110783\\
0.925999999999993	-0.000123158791110777\\
0.926	-0.00012315879111077\\
0.927999999999993	-0.000121415208824316\\
0.928000000000001	-0.00012141520882431\\
0.929999999999994	-0.000119691053320369\\
0.931999999999987	-0.000117986100524433\\
0.933999999999994	-0.000116300128860666\\
0.934000000000001	-0.00011630012886066\\
0.937999999999987	-0.000112984254936241\\
0.939999999999993	-0.000111353921744261\\
0.940000000000001	-0.000111353921744255\\
0.943999999999987	-0.000108147403486229\\
0.944999999999994	-0.000107356902640734\\
0.945000000000001	-0.000107356902640728\\
0.945999999999993	-0.000106570801700549\\
0.946000000000001	-0.000106570801700544\\
0.946999999999994	-0.000105789075125811\\
0.947999999999987	-0.000105011697518153\\
0.949999999999973	-0.000103469888317514\\
0.952	-0.000101945173724927\\
0.952000000000008	-0.000101945173724922\\
0.95599999999998	-9.89474188185791e-05\\
0.956999999999994	-9.82087263275e-05\\
0.957000000000001	-9.82087263274948e-05\\
0.96	-9.60180552597849e-05\\
0.960000000000008	-9.60180552597798e-05\\
0.963000000000007	-9.38649970109269e-05\\
0.966000000000007	-9.17489219973963e-05\\
0.966000000000014	-9.17489219973914e-05\\
0.969000000000007	-8.96692114471455e-05\\
0.969000000000014	-8.96692114471407e-05\\
0.972000000000007	-8.76252572309117e-05\\
0.975	-8.56164616656996e-05\\
0.979999999999994	-8.23450085686041e-05\\
0.980000000000001	-8.23450085685995e-05\\
0.985999999999987	-7.85422953672995e-05\\
0.986000000000001	-7.85422953672908e-05\\
0.991999999999987	-7.48696023506846e-05\\
0.992000000000001	-7.48696023506762e-05\\
0.997999999999987	-7.1347030512513e-05\\
0.998000000000001	-7.1347030512505e-05\\
0.999999999999993	-7.02109037638929e-05\\
1	-7.02109037638889e-05\\
1.00199999999999	-6.90935608126913e-05\\
1.00399999999999	-6.79948564468458e-05\\
1.00599999999999	-6.69146478786238e-05\\
1.006	-6.69146478786162e-05\\
1.00999999999999	-6.48091589873392e-05\\
1.01399999999997	-6.27759995872311e-05\\
1.01499999999999	-6.22788865380725e-05\\
1.015	-6.22788865380655e-05\\
1.01999999999999	-5.98595763402566e-05\\
1.02	-5.98595763402498e-05\\
1.02499999999999	-5.75492632127948e-05\\
1.02599999999999	-5.71001135478326e-05\\
1.026	-5.71001135478262e-05\\
1.02699999999999	-5.66552338280099e-05\\
1.027	-5.66552338280036e-05\\
1.02799999999999	-5.62146095997392e-05\\
1.02899999999999	-5.57782265468593e-05\\
1.03099999999997	-5.491812739279e-05\\
1.03299999999999	-5.40748245872153e-05\\
1.033	-5.40748245872094e-05\\
1.03699999999997	-5.24403503118484e-05\\
1.04	-5.12606455202036e-05\\
1.04000000000001	-5.12606455201981e-05\\
1.04399999999999	-4.97485844878559e-05\\
1.044	-4.97485844878507e-05\\
1.046	-4.90184231333625e-05\\
1.04600000000001	-4.90184231333574e-05\\
1.04800000000001	-4.83053806191598e-05\\
1.05	-4.76093642781677e-05\\
1.05000000000001	-4.76093642781628e-05\\
1.05200000000001	-4.69302836562168e-05\\
1.05200000000002	-4.6930283656212e-05\\
1.05400000000002	-4.62680505001301e-05\\
1.05600000000002	-4.56225787462189e-05\\
1.05800000000001	-4.49937845089083e-05\\
1.05800000000002	-4.49937845089039e-05\\
1.05999999999999	-4.43768842540106e-05\\
1.06	-4.43768842540062e-05\\
1.06199999999996	-4.37670959930363e-05\\
1.06399999999992	-4.31643404778443e-05\\
1.06599999999999	-4.2568539374237e-05\\
1.066	-4.25685393742328e-05\\
1.06999999999992	-4.13974915757357e-05\\
1.07299999999999	-4.05368916442088e-05\\
1.073	-4.05368916442048e-05\\
1.07699999999992	-3.94125278367813e-05\\
1.07899999999999	-3.88600824913482e-05\\
1.079	-3.88600824913443e-05\\
1.07999999999999	-3.8586262409986e-05\\
1.08	-3.85862624099822e-05\\
1.08099999999999	-3.83140321772306e-05\\
1.08199999999999	-3.8043382948301e-05\\
1.08399999999997	-3.75067923797528e-05\\
1.08499999999999	-3.72408336063612e-05\\
1.085	-3.72408336063575e-05\\
1.08599999999999	-3.69764209689098e-05\\
1.086	-3.6976420968906e-05\\
1.08699999999999	-3.67135458768152e-05\\
1.08799999999999	-3.64521997892459e-05\\
1.08999999999997	-3.59340607128617e-05\\
1.09199999999999	-3.54219364023437e-05\\
1.092	-3.54219364023401e-05\\
1.09599999999997	-3.44129821696437e-05\\
1.09999999999995	-3.34223281191999e-05\\
1.09999999999997	-3.34223281191932e-05\\
1.1	-3.34223281191864e-05\\
1.10199999999999	-3.29337023581072e-05\\
1.102	-3.29337023581037e-05\\
1.10399999999999	-3.24494592502548e-05\\
1.10599999999997	-3.19695358602703e-05\\
1.10599999999999	-3.19695358602668e-05\\
1.106	-3.19695358602634e-05\\
1.10999999999997	-3.10223993041679e-05\\
1.11199999999999	-3.055506304823e-05\\
1.112	-3.05550630482267e-05\\
1.11599999999997	-2.96325508985978e-05\\
1.11999999999994	-2.87258537947723e-05\\
1.12	-2.87258537947595e-05\\
1.12000000000001	-2.87258537947563e-05\\
1.126	-2.73944324532531e-05\\
1.12600000000001	-2.739443245325e-05\\
1.13099999999999	-2.63101633009274e-05\\
1.131	-2.63101633009243e-05\\
1.132	-2.60959788772883e-05\\
1.13200000000001	-2.60959788772853e-05\\
1.13300000000001	-2.58829886877625e-05\\
1.13400000000001	-2.56715065479812e-05\\
1.13600000000001	-2.52530389824609e-05\\
1.138	-2.4840521796643e-05\\
1.13800000000001	-2.48405217966401e-05\\
1.13999999999999	-2.4433901379994e-05\\
1.14	-2.44339013799912e-05\\
1.14199999999997	-2.40331248883197e-05\\
1.14399999999994	-2.36381402366669e-05\\
1.14599999999999	-2.32488960927767e-05\\
1.146	-2.3248896092774e-05\\
1.14999999999994	-2.24874277240065e-05\\
1.15399999999989	-2.17483239278908e-05\\
1.15499999999999	-2.156699716316e-05\\
1.155	-2.15669971631574e-05\\
1.15999999999999	-2.06807644357507e-05\\
1.16	-2.06807644357482e-05\\
1.16499999999999	-1.98280096638531e-05\\
1.16599999999999	-1.96614146348624e-05\\
1.166	-1.966141463486e-05\\
1.17099999999999	-1.88479214107754e-05\\
1.17199999999999	-1.86890821478633e-05\\
1.172	-1.8689082147861e-05\\
1.173	-1.85315172069141e-05\\
1.17300000000001	-1.85315172069118e-05\\
1.17400000000001	-1.83750889396583e-05\\
1.17500000000001	-1.82196597344588e-05\\
1.17700000000001	-1.79117783431312e-05\\
1.17999999999999	-1.7457324021528e-05\\
1.18	-1.74573240215259e-05\\
1.184	-1.68649320379695e-05\\
1.18599999999999	-1.65744551996525e-05\\
1.186	-1.65744551996505e-05\\
1.18899999999999	-1.6145782536612e-05\\
1.189	-1.614578253661e-05\\
1.18999999999999	-1.60047510737459e-05\\
1.19	-1.60047510737439e-05\\
1.19099999999999	-1.5864641661413e-05\\
1.19199999999999	-1.57254497474873e-05\\
1.19399999999997	-1.54498003553065e-05\\
1.19499999999999	-1.53133339212306e-05\\
1.195	-1.53133339212287e-05\\
1.19899999999997	-1.47764201510806e-05\\
1.19999999999999	-1.4644407894837e-05\\
1.2	-1.46444078948351e-05\\
1.20399999999997	-1.41250949406154e-05\\
1.20599999999999	-1.38706211513074e-05\\
1.206	-1.38706211513056e-05\\
1.20699999999999	-1.37446653865764e-05\\
1.207	-1.37446653865746e-05\\
1.20799999999999	-1.36196871553865e-05\\
1.20899999999999	-1.34958113096386e-05\\
1.21099999999997	-1.32513507113985e-05\\
1.21499999999995	-1.2775483437645e-05\\
1.21799999999999	-1.24298836838874e-05\\
1.218	-1.24298836838858e-05\\
1.21999999999999	-1.22048087313558e-05\\
1.22	-1.22048087313542e-05\\
1.22199999999999	-1.19839594806224e-05\\
1.22399999999997	-1.1767307228614e-05\\
1.22499999999999	-1.16605461571029e-05\\
1.225	-1.16605461571014e-05\\
1.22599999999999	-1.15548238191545e-05\\
1.226	-1.1554823819153e-05\\
1.22699999999999	-1.14501367799238e-05\\
1.22799999999999	-1.13464816381201e-05\\
1.22999999999997	-1.11422536093129e-05\\
1.23	-1.11422536093102e-05\\
1.23399999999997	-1.07460343738581e-05\\
1.236	-1.05539916745757e-05\\
1.23600000000001	-1.05539916745743e-05\\
1.23999999999999	-1.0180960373095e-05\\
1.24	-1.01809603730937e-05\\
1.24399999999997	-9.82187041654044e-06\\
1.24599999999999	-9.64749482615837e-06\\
1.246	-9.64749482615714e-06\\
1.247	-9.56158939016521e-06\\
1.24700000000001	-9.561589390164e-06\\
1.24800000000001	-9.4765351297098e-06\\
1.24900000000001	-9.3923292813714e-06\\
1.25100000000001	-9.22645190523081e-06\\
1.253	-9.06393570430299e-06\\
1.25300000000001	-9.06393570430184e-06\\
1.25700000000001	-8.74890278010037e-06\\
1.25999999999999	-8.52129723765548e-06\\
1.26	-8.52129723765442e-06\\
1.264	-8.22925551011215e-06\\
1.266	-8.08809139882927e-06\\
1.26600000000001	-8.08809139882828e-06\\
1.27000000000001	-7.81538495072435e-06\\
1.272	-7.68380717294294e-06\\
1.27200000000001	-7.68380717294202e-06\\
1.276	-7.42876844391719e-06\\
1.27600000000001	-7.4287684439163e-06\\
1.27999999999999	-7.18354318634862e-06\\
1.28000000000001	-7.18354318634777e-06\\
1.28399999999999	-6.94800391838649e-06\\
1.28599999999999	-6.8338280709989e-06\\
1.28600000000001	-6.8338280709981e-06\\
1.288	-6.72202819343346e-06\\
1.28800000000001	-6.72202819343268e-06\\
1.29	-6.61258975623202e-06\\
1.29199999999999	-6.50549853676335e-06\\
1.29499999999999	-6.34923237910887e-06\\
1.295	-6.34923237910814e-06\\
1.29899999999998	-6.14896540500326e-06\\
1.29999999999999	-6.10033202304176e-06\\
1.3	-6.10033202304107e-06\\
1.30399999999998	-5.91148453773299e-06\\
1.30499999999999	-5.86568650752996e-06\\
1.305	-5.86568650752932e-06\\
1.30599999999999	-5.82045101034084e-06\\
1.306	-5.8204510103402e-06\\
1.30699999999999	-5.77577657649645e-06\\
1.30700000000001	-5.77577657649581e-06\\
1.308	-5.7315345471217e-06\\
1.30899999999999	-5.68759627740947e-06\\
1.31099999999998	-5.60062531663063e-06\\
1.31300000000001	-5.5148523914033e-06\\
1.31300000000002	-5.5148523914027e-06\\
1.31699999999999	-5.34685621385352e-06\\
1.31999999999999	-5.22392213606576e-06\\
1.32	-5.22392213606519e-06\\
1.32399999999997	-5.06402576030405e-06\\
1.32599999999999	-4.98577517923626e-06\\
1.326	-4.98577517923571e-06\\
1.32999999999997	-4.83261838234848e-06\\
1.33	-4.8326183823475e-06\\
1.33399999999997	-4.68385463036508e-06\\
1.334	-4.68385463036409e-06\\
1.33799999999997	-4.53940658753969e-06\\
1.34	-4.4687774041641e-06\\
1.34000000000001	-4.4687774041636e-06\\
1.34399999999999	-4.33066281760675e-06\\
1.346	-4.26315946503717e-06\\
1.34600000000001	-4.26315946503669e-06\\
1.348	-4.19668033124229e-06\\
1.34800000000001	-4.19668033124182e-06\\
1.35	-4.13113710381037e-06\\
1.35199999999999	-4.06644159192198e-06\\
1.35599999999996	-3.93956019258578e-06\\
1.35999999999999	-3.81597017256711e-06\\
1.36	-3.81597017256667e-06\\
1.36299999999999	-3.72539891231839e-06\\
1.363	-3.72539891231796e-06\\
1.36499999999999	-3.66601343156786e-06\\
1.365	-3.66601343156744e-06\\
1.36599999999999	-3.63661639510304e-06\\
1.366	-3.63661639510263e-06\\
1.36699999999999	-3.60741521897946e-06\\
1.36799999999999	-3.57840895446008e-06\\
1.36999999999997	-3.52097739691673e-06\\
1.37199999999999	-3.46431425866118e-06\\
1.372	-3.46431425866078e-06\\
1.37599999999997	-3.35326388328442e-06\\
1.378	-3.29886221404358e-06\\
1.37800000000001	-3.2988622140432e-06\\
1.37999999999999	-3.24532091911658e-06\\
1.38	-3.2453209191162e-06\\
1.38199999999997	-3.19275386135538e-06\\
1.38399999999994	-3.14115420913963e-06\\
1.38599999999999	-3.09051525657142e-06\\
1.386	-3.09051525657107e-06\\
1.38999999999994	-2.9920932503445e-06\\
1.39199999999999	-2.94429740574817e-06\\
1.392	-2.94429740574783e-06\\
1.39599999999994	-2.85150497515212e-06\\
1.39799999999999	-2.80649632983515e-06\\
1.398	-2.80649632983484e-06\\
1.39999999999999	-2.76240489211035e-06\\
1.4	-2.76240489211004e-06\\
1.40199999999999	-2.71922493187476e-06\\
1.40399999999997	-2.67695083745608e-06\\
1.40599999999999	-2.63557711490801e-06\\
1.406	-2.63557711490772e-06\\
1.40999999999997	-2.55550939411761e-06\\
1.412	-2.51680499026436e-06\\
1.41200000000001	-2.51680499026408e-06\\
1.41599999999999	-2.4416216329719e-06\\
1.41999999999996	-2.36910091893579e-06\\
1.41999999999998	-2.36910091893541e-06\\
1.42	-2.36910091893502e-06\\
1.42099999999999	-2.35138233020653e-06\\
1.421	-2.35138233020628e-06\\
1.42199999999999	-2.33382721816548e-06\\
1.42299999999999	-2.316435012363e-06\\
1.42499999999997	-2.28213706447805e-06\\
1.42599999999999	-2.26523020805867e-06\\
1.426	-2.26523020805843e-06\\
1.42999999999997	-2.19920414305382e-06\\
1.43199999999999	-2.16714443696985e-06\\
1.432	-2.16714443696963e-06\\
1.43499999999999	-2.12023478857781e-06\\
1.435	-2.12023478857759e-06\\
1.43799999999998	-2.07472887077619e-06\\
1.43999999999999	-2.04516485465885e-06\\
1.44	-2.04516485465864e-06\\
1.44299999999999	-2.00196911866891e-06\\
1.44599999999997	-1.96014253078412e-06\\
1.44599999999999	-1.96014253078391e-06\\
1.446	-1.9601425307837e-06\\
1.44699999999999	-1.94650246452656e-06\\
1.447	-1.94650246452636e-06\\
1.44799999999999	-1.93297720768573e-06\\
1.44899999999999	-1.91953080801798e-06\\
1.44999999999999	-1.9061628286535e-06\\
1.45	-1.90616282865331e-06\\
1.45199999999999	-1.87966039608351e-06\\
1.45399999999997	-1.85346646572479e-06\\
1.45599999999999	-1.82757763341117e-06\\
1.456	-1.82757763341098e-06\\
1.45999999999997	-1.77670184414105e-06\\
1.46	-1.77670184414069e-06\\
1.46000000000001	-1.77670184414052e-06\\
1.46399999999999	-1.72700658017452e-06\\
1.466	-1.70259354832369e-06\\
1.46600000000001	-1.70259354832352e-06\\
1.46999999999999	-1.65462082093921e-06\\
1.47	-1.65462082093904e-06\\
1.47399999999997	-1.6077651542917e-06\\
1.47599999999999	-1.58474858446257e-06\\
1.476	-1.58474858446241e-06\\
1.47899999999998	-1.55070498138273e-06\\
1.479	-1.55070498138257e-06\\
1.47999999999999	-1.53947963257197e-06\\
1.48	-1.53947963257181e-06\\
1.48099999999999	-1.52831493370332e-06\\
1.48199999999999	-1.517210522038e-06\\
1.48399999999997	-1.49518111914843e-06\\
1.48599999999999	-1.47338856095956e-06\\
1.486	-1.4733885609594e-06\\
1.48999999999997	-1.43050268050243e-06\\
1.491	-1.41992484981844e-06\\
1.49100000000001	-1.41992484981829e-06\\
1.49499999999999	-1.37817777442642e-06\\
1.49899999999996	-1.33731728373921e-06\\
1.49999999999999	-1.32723819066221e-06\\
1.5	-1.32723819066207e-06\\
1.50499999999999	-1.27764255119209e-06\\
1.505	-1.27764255119195e-06\\
1.50599999999999	-1.26788113260993e-06\\
1.506	-1.26788113260979e-06\\
1.50699999999999	-1.25817153982713e-06\\
1.50799999999999	-1.24851345737308e-06\\
1.508	-1.24851345737295e-06\\
1.50999999999999	-1.22935056996169e-06\\
1.51199999999997	-1.21038997997074e-06\\
1.51399999999999	-1.19162922327978e-06\\
1.514	-1.19162922327965e-06\\
1.51799999999997	-1.1546974829024e-06\\
1.518	-1.15469748290216e-06\\
1.51999999999999	-1.13654290464229e-06\\
1.52	-1.13654290464216e-06\\
1.52199999999999	-1.1186209727003e-06\\
1.52399999999997	-1.10092935794029e-06\\
1.52599999999999	-1.08346576115795e-06\\
1.526	-1.08346576115783e-06\\
1.52999999999997	-1.04921357262852e-06\\
1.53399999999994	-1.01584660089692e-06\\
1.537	-9.91391843703725e-07\\
1.53700000000001	-9.9139184370361e-07\\
1.53999999999999	-9.6741810367338e-07\\
1.54	-9.67418103673268e-07\\
1.54299999999997	-9.4391837070677e-07\\
1.54599999999994	-9.20885773010966e-07\\
1.54599999999997	-9.20885773010755e-07\\
1.546	-9.20885773010543e-07\\
1.549	-8.98313575508408e-07\\
1.54900000000001	-8.98313575508302e-07\\
1.55200000000001	-8.76195177852366e-07\\
1.55500000000001	-8.54524112289572e-07\\
1.55500000000003	-8.54524112289471e-07\\
1.55999999999999	-8.1964574175582e-07\\
1.56	-8.19645741755723e-07\\
1.56499999999996	-7.86490386777966e-07\\
1.56599999999998	-7.80063683249524e-07\\
1.566	-7.80063683249433e-07\\
1.57099999999996	-7.48940518397185e-07\\
1.57199999999998	-7.42916540326751e-07\\
1.572	-7.42916540326666e-07\\
1.57499999999999	-7.25242373455314e-07\\
1.575	-7.25242373455232e-07\\
1.57799999999999	-7.08160860499542e-07\\
1.57999999999999	-6.97099963677889e-07\\
1.58	-6.97099963677812e-07\\
1.58299999999999	-6.80995192449376e-07\\
1.58599999999997	-6.65470136653115e-07\\
1.586	-6.65470136652974e-07\\
1.58699999999999	-6.60423154193241e-07\\
1.587	-6.6042315419317e-07\\
1.58799999999999	-6.5542521801101e-07\\
1.58899999999999	-6.50461468932103e-07\\
1.59099999999997	-6.40635888106903e-07\\
1.59499999999995	-6.21387949563937e-07\\
1.59499999999997	-6.21387949563809e-07\\
1.595	-6.21387949563682e-07\\
1.59999999999999	-5.98071238189109e-07\\
1.6	-5.98071238189044e-07\\
1.60499999999999	-5.75562671655537e-07\\
1.60599999999999	-5.71156315376054e-07\\
1.606	-5.71156315375992e-07\\
1.60699999999998	-5.66781408329532e-07\\
1.607	-5.6678140832947e-07\\
1.60799999999999	-5.62437808380658e-07\\
1.60899999999999	-5.58125374403174e-07\\
1.60999999999998	-5.53843966287283e-07\\
1.61	-5.53843966287222e-07\\
1.61199999999999	-5.45373672237784e-07\\
1.61399999999997	-5.37025825434391e-07\\
1.61599999999999	-5.28799340989976e-07\\
1.616	-5.28799340989918e-07\\
1.61999999999997	-5.12668493258539e-07\\
1.61999999999999	-5.12668493258481e-07\\
1.62	-5.12668493258424e-07\\
1.62399999999997	-4.969350382049e-07\\
1.624	-4.96935038204804e-07\\
1.62599999999999	-4.89214768998135e-07\\
1.626	-4.8921476899808e-07\\
1.62799999999999	-4.81590796693942e-07\\
1.62999999999998	-4.74062130480093e-07\\
1.63199999999999	-4.66627791930287e-07\\
1.632	-4.66627791930234e-07\\
1.63599999999998	-4.52038245306433e-07\\
1.63999999999995	-4.37814572355073e-07\\
1.63999999999998	-4.37814572354989e-07\\
1.64	-4.37814572354905e-07\\
1.645	-4.20538244066133e-07\\
1.64500000000001	-4.20538244066084e-07\\
1.64599999999999	-4.17148952118714e-07\\
1.646	-4.17148952118666e-07\\
1.64699999999999	-4.13781392808056e-07\\
1.64799999999999	-4.10435456720119e-07\\
1.64999999999997	-4.03808020077581e-07\\
1.65199999999999	-3.97265780888672e-07\\
1.652	-3.97265780888626e-07\\
1.65299999999998	-3.94027198333299e-07\\
1.653	-3.94027198333253e-07\\
1.65399999999999	-3.90811305432929e-07\\
1.65499999999997	-3.87617997702943e-07\\
1.65699999999995	-3.81298723486112e-07\\
1.65899999999998	-3.75068554429104e-07\\
1.659	-3.7506855442906e-07\\
1.65999999999999	-3.71986630864172e-07\\
1.66	-3.71986630864128e-07\\
1.66099999999999	-3.68926680863035e-07\\
1.66199999999998	-3.65888605007657e-07\\
1.66399999999995	-3.59877681616973e-07\\
1.66599999999999	-3.53953079511738e-07\\
1.666	-3.53953079511696e-07\\
1.66999999999995	-3.42359770444758e-07\\
1.67399999999991	-3.31102650952164e-07\\
1.67999999999998	-3.14834572750618e-07\\
1.68	-3.1483457275058e-07\\
1.68199999999998	-3.09573744030886e-07\\
1.682	-3.09573744030849e-07\\
1.68399999999998	-3.0439269914411e-07\\
1.68599999999997	-2.99290764727075e-07\\
1.68599999999999	-2.99290764727037e-07\\
1.686	-2.99290764727001e-07\\
1.68999999999997	-2.89321585311426e-07\\
1.69199999999999	-2.84453044718731e-07\\
1.692	-2.84453044718697e-07\\
1.69599999999997	-2.7494489811434e-07\\
1.69799999999999	-2.70304056422813e-07\\
1.698	-2.70304056422781e-07\\
1.69999999999999	-2.65753192396858e-07\\
1.7	-2.65753192396826e-07\\
1.70199999999999	-2.6130701195764e-07\\
1.70399999999998	-2.56964937279055e-07\\
1.70599999999999	-2.52726404064569e-07\\
1.706	-2.52726404064539e-07\\
1.70999999999998	-2.44557772063317e-07\\
1.71099999999998	-2.42579482982904e-07\\
1.711	-2.42579482982877e-07\\
1.71499999999997	-2.34919874071443e-07\\
1.715	-2.34919874071395e-07\\
1.71699999999998	-2.3124132992782e-07\\
1.717	-2.31241329927795e-07\\
1.71899999999998	-2.27662984215705e-07\\
1.71999999999999	-2.25911239938646e-07\\
1.72	-2.25911239938621e-07\\
1.72199999999999	-2.22482323972695e-07\\
1.72399999999997	-2.19152468118394e-07\\
1.72599999999999	-2.15921239627272e-07\\
1.726	-2.15921239627249e-07\\
1.72899999999998	-2.11258408094454e-07\\
1.729	-2.11258408094432e-07\\
1.73199999999998	-2.06757624049903e-07\\
1.73499999999997	-2.02360012647313e-07\\
1.73999999999998	-1.95256453718612e-07\\
1.74	-1.95256453718592e-07\\
1.74599999999997	-1.87097675318713e-07\\
1.74599999999998	-1.87097675318691e-07\\
1.746	-1.87097675318669e-07\\
1.74999999999998	-1.81875573125122e-07\\
1.75	-1.81875573125104e-07\\
1.75199999999998	-1.79328620630159e-07\\
1.752	-1.79328620630141e-07\\
1.75399999999998	-1.76823957354738e-07\\
1.75599999999997	-1.74361257792563e-07\\
1.75799999999998	-1.71940201890909e-07\\
1.758	-1.71940201890892e-07\\
1.75999999999999	-1.69554019960508e-07\\
1.76	-1.69554019960491e-07\\
1.76199999999999	-1.67195946844166e-07\\
1.76399999999998	-1.64865676086473e-07\\
1.76599999999999	-1.62562904845182e-07\\
1.766	-1.62562904845166e-07\\
1.76899999999998	-1.5915965589808e-07\\
1.769	-1.59159655898064e-07\\
1.77199999999998	-1.55816613184853e-07\\
1.77499999999996	-1.52532799154329e-07\\
1.77499999999998	-1.5253279915431e-07\\
1.775	-1.52532799154292e-07\\
1.78	-1.47188794748675e-07\\
1.78000000000002	-1.4718879474866e-07\\
1.78499999999998	-1.42002313256241e-07\\
1.785	-1.42002313256226e-07\\
1.78600000000001	-1.40983546753248e-07\\
1.78600000000003	-1.40983546753234e-07\\
1.78700000000005	-1.39970879151721e-07\\
1.78800000000006	-1.38964277549456e-07\\
1.79000000000009	-1.36969141722007e-07\\
1.79200000000003	-1.3499787998294e-07\\
1.79200000000004	-1.34997879982927e-07\\
1.79600000000011	-1.31121924558128e-07\\
1.79799999999998	-1.29215719049733e-07\\
1.798	-1.29215719049719e-07\\
1.8	-1.27330363790059e-07\\
1.80000000000002	-1.27330363790045e-07\\
1.80200000000002	-1.25465613759372e-07\\
1.80400000000002	-1.2362122661458e-07\\
1.806	-1.21796962658964e-07\\
1.80600000000002	-1.21796962658951e-07\\
1.80999999999998	-1.18207858578257e-07\\
1.81	-1.18207858578244e-07\\
1.81399999999997	-1.14696435698988e-07\\
1.81799999999994	-1.11260868586497e-07\\
1.82	-1.09570971291549e-07\\
1.82000000000001	-1.09570971291537e-07\\
1.826	-1.04610196147631e-07\\
1.82600000000001	-1.04610196147619e-07\\
1.827	-1.03799001540376e-07\\
1.82700000000001	-1.03799001540364e-07\\
1.828	-1.02992193673242e-07\\
1.82899999999999	-1.02189746331835e-07\\
1.83099999999996	-1.00597829081938e-07\\
1.83200000000001	-9.98083074521864e-08\\
1.83200000000003	-9.98083074521752e-08\\
1.83599999999998	-9.67068784143788e-08\\
1.83800000000001	-9.51919574659811e-08\\
1.83800000000003	-9.51919574659704e-08\\
1.83999999999999	-9.37006353623608e-08\\
1.84	-9.37006353623503e-08\\
1.84199999999996	-9.22327182919678e-08\\
1.84399999999992	-9.078801548415e-08\\
1.84599999999999	-8.93663391851083e-08\\
1.846	-8.93663391850983e-08\\
1.84999999999992	-8.6591330041212e-08\\
1.85399999999984	-8.39062482543018e-08\\
1.85499999999998	-8.32488643600815e-08\\
1.855	-8.32488643600722e-08\\
1.85599999999998	-8.25969921019009e-08\\
1.856	-8.25969921018916e-08\\
1.85699999999999	-8.19506103037564e-08\\
1.85799999999997	-8.13096979616436e-08\\
1.85999999999995	-8.00441985303418e-08\\
1.85999999999998	-8.00441985303257e-08\\
1.86	-8.00441985303095e-08\\
1.86399999999995	-7.75779287493652e-08\\
1.86599999999999	-7.63768378833824e-08\\
1.866	-7.63768378833739e-08\\
1.86799999999998	-7.51969006528353e-08\\
1.868	-7.5196900652827e-08\\
1.86999999999998	-7.40379637136263e-08\\
1.87199999999996	-7.28998764501895e-08\\
1.87299999999998	-7.23386051208194e-08\\
1.873	-7.23386051208115e-08\\
1.87699999999996	-7.01324887615064e-08\\
1.87999999999999	-6.8515057916358e-08\\
1.88	-6.85150579163504e-08\\
1.88399999999997	-6.64071261357416e-08\\
1.88499999999998	-6.58887148650397e-08\\
1.885	-6.58887148650323e-08\\
1.886	-6.53736984115734e-08\\
1.88600000000002	-6.53736984115661e-08\\
1.88700000000002	-6.4862060042826e-08\\
1.88800000000002	-6.43537831356236e-08\\
1.88999999999998	-6.33472477594973e-08\\
1.89	-6.33472477594902e-08\\
1.892	-6.2353961473994e-08\\
1.89200000000002	-6.2353961473987e-08\\
1.89400000000002	-6.1373795191848e-08\\
1.89600000000003	-6.04066215308447e-08\\
1.898	-5.94523147970117e-08\\
1.89800000000002	-5.9452314797005e-08\\
1.9	-5.85111993133228e-08\\
1.90000000000002	-5.85111993133162e-08\\
1.902	-5.75836011171071e-08\\
1.90399999999998	-5.66693996576102e-08\\
1.906	-5.57684761251e-08\\
1.90600000000002	-5.57684761250937e-08\\
1.90999999999998	-5.40059962174294e-08\\
1.91399999999995	-5.2295245157851e-08\\
1.91399999999997	-5.22952451578401e-08\\
1.914	-5.22952451578292e-08\\
1.91999999999998	-4.98241719534748e-08\\
1.92	-4.98241719534691e-08\\
1.92499999999998	-4.7850240108676e-08\\
1.925	-4.78502401086705e-08\\
1.92599999999998	-4.74646033171379e-08\\
1.926	-4.74646033171325e-08\\
1.92699999999999	-4.70819869198595e-08\\
1.92799999999997	-4.6702378485393e-08\\
1.92999999999995	-4.59521362687676e-08\\
1.93199999999998	-4.5213779165685e-08\\
1.932	-4.52137791656798e-08\\
1.93599999999995	-4.37701446388508e-08\\
1.9399999999999	-4.23685310547912e-08\\
1.93999999999999	-4.23685310547615e-08\\
1.94	-4.23685310547566e-08\\
1.94299999999998	-4.13444579724297e-08\\
1.943	-4.13444579724249e-08\\
1.94599999999998	-4.03433119254613e-08\\
1.946	-4.03433119254543e-08\\
1.94899999999998	-3.93648001586018e-08\\
1.95199999999997	-3.8408636541785e-08\\
1.95199999999998	-3.84086365417795e-08\\
1.952	-3.84086365417739e-08\\
1.95799999999997	-3.65622418307297e-08\\
1.95999999999998	-3.59660186285265e-08\\
1.96	-3.59660186285223e-08\\
1.96599999999996	-3.42339892726863e-08\\
1.96599999999998	-3.42339892726814e-08\\
1.966	-3.42339892726764e-08\\
1.97199999999996	-3.25853573391773e-08\\
1.97199999999998	-3.25853573391726e-08\\
1.972	-3.2585357339168e-08\\
1.97799999999996	-3.10220369080688e-08\\
1.97799999999998	-3.10220369080643e-08\\
1.978	-3.10220369080599e-08\\
1.98	-3.05204230781025e-08\\
1.98000000000002	-3.0520423078099e-08\\
1.98200000000002	-3.00284460383846e-08\\
1.98400000000002	-2.95460418506071e-08\\
1.986	-2.90731478214773e-08\\
1.98600000000002	-2.90731478214739e-08\\
1.99000000000002	-2.81556456389185e-08\\
1.99400000000003	-2.72754625214488e-08\\
1.995	-2.70611938007662e-08\\
1.99500000000001	-2.70611938007632e-08\\
1.99999999999999	-2.60241654028837e-08\\
2	-2.60241654028809e-08\\
2.00099999999997	-2.58235755949382e-08\\
2.001	-2.58235755949325e-08\\
2.00199999999999	-2.5625242365797e-08\\
2.00299999999997	-2.54291592714937e-08\\
2.00499999999995	-2.50437180775691e-08\\
2.00599999999997	-2.48543474549988e-08\\
2.006	-2.48543474549935e-08\\
2.00999999999995	-2.41190554093091e-08\\
2.01199999999997	-2.37646402913546e-08\\
2.012	-2.37646402913497e-08\\
2.01299999999998	-2.35907201199138e-08\\
2.01300000000001	-2.35907201199089e-08\\
2.014	-2.34185407505901e-08\\
2.01499999999999	-2.32476533574679e-08\\
2.01699999999996	-2.29097323332971e-08\\
2.01999999999997	-2.24124031711204e-08\\
2.02	-2.24124031711157e-08\\
2.02399999999995	-2.17669033657222e-08\\
2.02599999999997	-2.14516042233281e-08\\
2.026	-2.14516042233236e-08\\
2.02999999999995	-2.08357025731386e-08\\
2.03	-2.08357025731319e-08\\
2.03399999999995	-2.02391306773921e-08\\
2.03599999999997	-1.99479960831541e-08\\
2.036	-1.994799608315e-08\\
2.03999999999995	-1.93798404184137e-08\\
2.04	-1.93798404184064e-08\\
2.04399999999995	-1.88302576692654e-08\\
2.04599999999997	-1.85623414824093e-08\\
2.046	-1.85623414824055e-08\\
2.04799999999997	-1.82989621316002e-08\\
2.048	-1.82989621315965e-08\\
2.04999999999996	-1.80401239766215e-08\\
2.05199999999993	-1.77858319672819e-08\\
2.05599999999987	-1.72907547810689e-08\\
2.05899999999997	-1.69311389834356e-08\\
2.059	-1.69311389834322e-08\\
2.05999999999997	-1.68134732011613e-08\\
2.06	-1.68134732011579e-08\\
2.06099999999999	-1.66969040931563e-08\\
2.06199999999998	-1.65814278715372e-08\\
2.06399999999995	-1.63537391156869e-08\\
2.06499999999997	-1.62415191838698e-08\\
2.065	-1.62415191838666e-08\\
2.06599999999997	-1.61303773431243e-08\\
2.066	-1.61303773431212e-08\\
2.06699999999999	-1.60203099825285e-08\\
2.06799999999998	-1.5911313525987e-08\\
2.06999999999995	-1.56965191946919e-08\\
2.07099999999997	-1.55907143412913e-08\\
2.071	-1.55907143412883e-08\\
2.07499999999995	-1.51780305328708e-08\\
2.07699999999997	-1.49779631829772e-08\\
2.077	-1.49779631829744e-08\\
2.07999999999997	-1.46844178551145e-08\\
2.08	-1.46844178551118e-08\\
2.08299999999998	-1.43976918845526e-08\\
2.08599999999995	-1.41177014287175e-08\\
2.086	-1.41177014287131e-08\\
2.08799999999997	-1.39347425656365e-08\\
2.088	-1.39347425656339e-08\\
2.08999999999997	-1.3754717059712e-08\\
2.09199999999993	-1.35776015144807e-08\\
2.09399999999997	-1.34033729119811e-08\\
2.094	-1.34033729119786e-08\\
2.09799999999993	-1.30634863368428e-08\\
2.09999999999997	-1.28977841924998e-08\\
2.1	-1.28977841924975e-08\\
2.10399999999993	-1.25747545145896e-08\\
2.10599999999997	-1.24173850000424e-08\\
2.106	-1.24173850000402e-08\\
2.10999999999993	-1.21108343587518e-08\\
2.112	-1.19616133926442e-08\\
2.11200000000003	-1.19616133926421e-08\\
2.11599999999996	-1.16695716309546e-08\\
2.11699999999997	-1.15977128334947e-08\\
2.117	-1.15977128334927e-08\\
2.11999999999997	-1.13848629693614e-08\\
2.12	-1.13848629693594e-08\\
2.12299999999998	-1.11760546432405e-08\\
2.12599999999995	-1.09712267967374e-08\\
2.126	-1.09712267967341e-08\\
2.12899999999997	-1.0770319535115e-08\\
2.129	-1.07703195351131e-08\\
2.13199999999996	-1.05732741109821e-08\\
2.13499999999993	-1.03800329052864e-08\\
2.13499999999997	-1.03800329052843e-08\\
2.135	-1.03800329052821e-08\\
2.13999999999997	-1.00662650703644e-08\\
2.14	-1.00662650703627e-08\\
2.14499999999998	-9.76265312581216e-09\\
2.14599999999997	-9.7031275979927e-09\\
2.146	-9.70312759799101e-09\\
2.14699999999997	-9.64399648116707e-09\\
2.14699999999999	-9.64399648116539e-09\\
2.14799999999998	-9.5851635222707e-09\\
2.14899999999997	-9.5265324778186e-09\\
2.15099999999995	-9.40986851912303e-09\\
2.15299999999999	-9.29398944341576e-09\\
2.15300000000002	-9.29398944341412e-09\\
2.15699999999997	-9.06452580270846e-09\\
2.15999999999997	-8.89437710658337e-09\\
2.16	-8.89437710658177e-09\\
2.16399999999995	-8.67001629585317e-09\\
2.16599999999997	-8.55887625503718e-09\\
2.166	-8.5588762550356e-09\\
2.16999999999995	-8.33860468206925e-09\\
2.17	-8.3386046820667e-09\\
2.17399999999995	-8.12091584014499e-09\\
2.17499999999997	-8.06688389096518e-09\\
2.175	-8.06688389096364e-09\\
2.17899999999995	-7.85226447365822e-09\\
2.17999999999997	-7.79897799801965e-09\\
2.18	-7.79897799801814e-09\\
2.18399999999995	-7.58725367105386e-09\\
2.18599999999997	-7.48222039272814e-09\\
2.186	-7.48222039272665e-09\\
2.187	-7.42990499129917e-09\\
2.18700000000002	-7.42990499129768e-09\\
2.18800000000002	-7.37772148065659e-09\\
2.18800000000005	-7.37772148065511e-09\\
2.18900000000004	-7.32570323052513e-09\\
2.19000000000004	-7.27388361602122e-09\\
2.19200000000003	-7.17083356587807e-09\\
2.19600000000001	-6.96704344024927e-09\\
2.2	-6.76624516128715e-09\\
2.20000000000003	-6.76624516128573e-09\\
2.20399999999997	-6.56833434374509e-09\\
2.204	-6.56833434374369e-09\\
2.20499999999997	-6.51929571634699e-09\\
2.205	-6.5192957163456e-09\\
2.20599999999999	-6.47042952466547e-09\\
2.20600000000003	-6.47042952466341e-09\\
2.20700000000002	-6.42173418103352e-09\\
2.20800000000001	-6.37320810335734e-09\\
2.20999999999998	-6.2766574449258e-09\\
2.21200000000003	-6.18076500163295e-09\\
2.21200000000006	-6.18076500163159e-09\\
2.21600000000001	-5.99090499588505e-09\\
2.21800000000003	-5.89691275920133e-09\\
2.21800000000006	-5.8969127592e-09\\
2.21999999999997	-5.80392343288967e-09\\
2.22	-5.80392343288836e-09\\
2.22199999999992	-5.71231897884445e-09\\
2.22399999999983	-5.62208749214222e-09\\
2.226	-5.53321724627805e-09\\
2.22600000000003	-5.53321724627679e-09\\
2.22999999999986	-5.35951445441455e-09\\
2.23299999999997	-5.23272598974714e-09\\
2.233	-5.23272598974595e-09\\
2.23699999999983	-5.06825572044294e-09\\
2.23899999999997	-4.98795978466435e-09\\
2.239	-4.98795978466321e-09\\
2.24	-4.94829202362755e-09\\
2.24000000000003	-4.94829202362643e-09\\
2.24100000000003	-4.90894267865719e-09\\
2.24200000000003	-4.86991047129017e-09\\
2.24400000000004	-4.79279240704828e-09\\
2.246	-4.71692780863133e-09\\
2.24600000000003	-4.71692780863027e-09\\
2.25000000000004	-4.56891973359997e-09\\
2.252	-4.49675702184465e-09\\
2.25200000000003	-4.49675702184363e-09\\
2.25600000000004	-4.35594226052024e-09\\
2.25999999999997	-4.21966423239372e-09\\
2.26	-4.21966423239277e-09\\
2.26199999999997	-4.15320426309415e-09\\
2.262	-4.15320426309322e-09\\
2.26399999999996	-4.08785209247352e-09\\
2.26599999999993	-4.02359922735743e-09\\
2.26599999999997	-4.0235992273563e-09\\
2.266	-4.02359922735518e-09\\
2.26999999999994	-3.89835815425466e-09\\
2.272	-3.83735366994323e-09\\
2.27200000000003	-3.83735366994237e-09\\
2.27499999999997	-3.74784466160139e-09\\
2.275	-3.74784466160055e-09\\
2.27799999999994	-3.66070970136271e-09\\
2.27999999999997	-3.60392603269303e-09\\
2.28	-3.60392603269223e-09\\
2.28299999999994	-3.52069153773857e-09\\
2.28599999999989	-3.43976466823471e-09\\
2.28599999999997	-3.43976466823239e-09\\
2.286	-3.43976466823163e-09\\
2.28699999999997	-3.41329774127334e-09\\
2.287	-3.4132977412726e-09\\
2.28799999999999	-3.3870371982741e-09\\
2.28899999999997	-3.36093566181777e-09\\
2.29099999999995	-3.30920622154707e-09\\
2.291	-3.30920622154588e-09\\
2.29499999999995	-3.20761844890866e-09\\
2.29699999999997	-3.15774691419902e-09\\
2.297	-3.15774691419832e-09\\
2.29999999999997	-3.08407429483783e-09\\
2.3	-3.08407429483713e-09\\
2.30299999999998	-3.01174417490839e-09\\
2.30599999999995	-2.94073540409989e-09\\
2.306	-2.9407354040988e-09\\
2.30999999999997	-2.84807664154388e-09\\
2.31	-2.84807664154323e-09\\
2.31399999999997	-2.75768190511268e-09\\
2.31599999999997	-2.71331881627093e-09\\
2.316	-2.7133188162703e-09\\
2.31999999999997	-2.62624045260451e-09\\
2.32	-2.6262404526039e-09\\
2.32399999999997	-2.54132695820948e-09\\
2.32599999999997	-2.49966819902131e-09\\
2.326	-2.49966819902072e-09\\
2.32999999999997	-2.41791958618761e-09\\
2.33199999999997	-2.37781910848398e-09\\
2.332	-2.37781910848341e-09\\
2.33599999999997	-2.2991397526214e-09\\
2.33999999999994	-2.22245522073022e-09\\
2.34	-2.22245522072901e-09\\
2.34000000000003	-2.22245522072847e-09\\
2.34499999999997	-2.12934414213662e-09\\
2.345	-2.1293441421361e-09\\
2.34600000000003	-2.11108179147541e-09\\
2.34600000000006	-2.1110817914749e-09\\
2.34700000000009	-2.09293800232195e-09\\
2.34800000000012	-2.07491218517344e-09\\
2.34899999999997	-2.057003754379e-09\\
2.349	-2.0570037543785e-09\\
2.35100000000006	-2.02153672827532e-09\\
2.35300000000011	-1.98653231472689e-09\\
2.35499999999997	-1.95198596455982e-09\\
2.355	-1.95198596455933e-09\\
2.35800000000003	-1.90101550322076e-09\\
2.35800000000006	-1.90101550322028e-09\\
2.35999999999997	-1.8676265459832e-09\\
2.36	-1.86762654598273e-09\\
2.36199999999992	-1.83474369481848e-09\\
2.36399999999983	-1.80236267627651e-09\\
2.36599999999997	-1.77047928211305e-09\\
2.366	-1.7704792821126e-09\\
2.36999999999983	-1.70818885687958e-09\\
2.37399999999966	-1.64784003704684e-09\\
2.37799999999997	-1.58940144962044e-09\\
2.378	-1.58940144962004e-09\\
2.37999999999997	-1.56088896112764e-09\\
2.38	-1.56088896112723e-09\\
2.38199999999997	-1.53284271534291e-09\\
2.38399999999995	-1.50525906718773e-09\\
2.38599999999997	-1.47813443188754e-09\\
2.386	-1.47813443188716e-09\\
2.38999999999995	-1.42524815864059e-09\\
2.39	-1.42524815863998e-09\\
2.39399999999995	-1.37415640234045e-09\\
2.396	-1.34927513186415e-09\\
2.39600000000002	-1.3492751318638e-09\\
2.39999999999997	-1.3010793662642e-09\\
2.4	-1.30107936626387e-09\\
2.40399999999995	-1.25512102414162e-09\\
2.40599999999997	-1.23297337562175e-09\\
2.406	-1.23297337562144e-09\\
2.40699999999997	-1.22210616095164e-09\\
2.407	-1.22210616095133e-09\\
2.40799999999998	-1.21137621375939e-09\\
2.40899999999997	-1.20078318542834e-09\\
2.41099999999995	-1.18000651312949e-09\\
2.41299999999997	-1.15977344391593e-09\\
2.413	-1.15977344391564e-09\\
2.41499999999997	-1.14008134830888e-09\\
2.415	-1.14008134830861e-09\\
2.41699999999997	-1.12092766713299e-09\\
2.41899999999995	-1.10230991117372e-09\\
2.41999999999997	-1.09320124632161e-09\\
2.42	-1.09320124632136e-09\\
2.42399999999995	-1.05809448628276e-09\\
2.426	-1.04133386879716e-09\\
2.42600000000003	-1.04133386879693e-09\\
2.427	-1.03315078647841e-09\\
2.42700000000003	-1.03315078647818e-09\\
2.42800000000002	-1.02506530123807e-09\\
2.429	-1.01704361908082e-09\\
2.43099999999998	-1.00119062359158e-09\\
2.43499999999993	-9.70238940092997e-10\\
2.43599999999997	-9.6265669708941e-10\\
2.436	-9.62656697089195e-10\\
2.43999999999997	-9.32943046162917e-10\\
2.44	-9.3294304616271e-10\\
2.44399999999998	-9.04202395852785e-10\\
2.446	-8.90192257260786e-10\\
2.44600000000003	-8.90192257260589e-10\\
2.448	-8.76419804998703e-10\\
2.44800000000002	-8.76419804998509e-10\\
2.44999999999999	-8.62883249205061e-10\\
2.45000000000002	-8.6288324920487e-10\\
2.45199999999998	-8.49580830667571e-10\\
2.45399999999995	-8.36510820606665e-10\\
2.45600000000002	-8.23671520441748e-10\\
2.45600000000005	-8.23671520441567e-10\\
2.45999999999998	-7.98552096371417e-10\\
2.46000000000001	-7.9855209637124e-10\\
2.46399999999994	-7.74083191121005e-10\\
2.46499999999997	-7.68066110374526e-10\\
2.465	-7.68066110374355e-10\\
2.46599999999998	-7.6208869492313e-10\\
2.46600000000001	-7.62088694922961e-10\\
2.46699999999999	-7.56150750564709e-10\\
2.46799999999998	-7.50252084375068e-10\\
2.46999999999996	-7.38571821188239e-10\\
2.47199999999998	-7.27046387394115e-10\\
2.47200000000001	-7.27046387393953e-10\\
2.47599999999996	-7.04454036556288e-10\\
2.47999999999991	-6.82463287086331e-10\\
2.47999999999997	-6.82463287085982e-10\\
2.48	-6.82463287085828e-10\\
2.48499999999997	-6.55803465411233e-10\\
2.485	-6.55803465411084e-10\\
2.48599999999997	-6.50580241856073e-10\\
2.486	-6.50580241855926e-10\\
2.48699999999999	-6.45392866521513e-10\\
2.48799999999998	-6.40241170866621e-10\\
2.48999999999995	-6.30044150241243e-10\\
2.49199999999997	-6.19987854773809e-10\\
2.492	-6.19987854773667e-10\\
2.494	-6.10070654587919e-10\\
2.49400000000002	-6.1007065458778e-10\\
2.49600000000002	-6.00290937883608e-10\\
2.49800000000001	-5.90647433687987e-10\\
2.49999999999997	-5.81138888730354e-10\\
2.5	-5.8113888873022e-10\\
2.50399999999999	-5.62521751006095e-10\\
2.50599999999997	-5.53410738754673e-10\\
2.506	-5.53410738754545e-10\\
2.50999999999999	-5.35577907002691e-10\\
2.51399999999998	-5.18256301522083e-10\\
2.51999999999997	-4.93212816166493e-10\\
2.52	-4.93212816166377e-10\\
2.52299999999997	-4.81105640615888e-10\\
2.523	-4.81105640615775e-10\\
2.52599999999996	-4.69270096043635e-10\\
2.526	-4.69270096043491e-10\\
2.52899999999997	-4.57702721498261e-10\\
2.53199999999993	-4.46400134523169e-10\\
2.53199999999997	-4.46400134523049e-10\\
2.532	-4.46400134522926e-10\\
2.53799999999993	-4.24576179627388e-10\\
2.53799999999997	-4.24576179627271e-10\\
2.538	-4.24576179627156e-10\\
2.53999999999997	-4.17552371054245e-10\\
2.54	-4.17552371054146e-10\\
2.54199999999998	-4.10686714594308e-10\\
2.54399999999995	-4.0397831797306e-10\\
2.54599999999997	-3.97426309366355e-10\\
2.546	-3.97426309366263e-10\\
2.54999999999995	-3.84788070429591e-10\\
2.55199999999997	-3.7870019763236e-10\\
2.552	-3.78700197632274e-10\\
2.55499999999997	-3.69855214533069e-10\\
2.555	-3.69855214532987e-10\\
2.55799999999998	-3.61352131178514e-10\\
2.55800000000001	-3.61352131178435e-10\\
2.55999999999997	-3.55872121291957e-10\\
2.56	-3.5587212129188e-10\\
2.56199999999997	-3.50542247550265e-10\\
2.56399999999994	-3.45361817283884e-10\\
2.56599999999997	-3.40330157243336e-10\\
2.566	-3.40330157243266e-10\\
2.56999999999994	-3.30710551443107e-10\\
2.56999999999997	-3.3071055144303e-10\\
2.57	-3.30710551442952e-10\\
2.57399999999994	-3.21530185498483e-10\\
2.57799999999988	-3.12636043101998e-10\\
2.57999999999997	-3.08294855060593e-10\\
2.58	-3.08294855060532e-10\\
2.58099999999997	-3.06150483512964e-10\\
2.581	-3.06150483512903e-10\\
2.58199999999998	-3.04023500539571e-10\\
2.58299999999997	-3.01913837024768e-10\\
2.58499999999995	-2.97746194761989e-10\\
2.58599999999997	-2.95688080607698e-10\\
2.586	-2.9568808060764e-10\\
2.58999999999995	-2.87625450048361e-10\\
2.59	-2.87625450048267e-10\\
2.59199999999997	-2.83695115487463e-10\\
2.592	-2.83695115487407e-10\\
2.59399999999997	-2.79831417451586e-10\\
2.59599999999995	-2.76033853814125e-10\\
2.59799999999997	-2.7230193104319e-10\\
2.598	-2.72301931043137e-10\\
2.59999999999997	-2.686234491143e-10\\
2.6	-2.68623449114248e-10\\
2.60199999999997	-2.6498621494782e-10\\
2.60399999999995	-2.61389755848473e-10\\
2.60599999999997	-2.57833604419946e-10\\
2.606	-2.57833604419896e-10\\
2.60999999999995	-2.50840381132278e-10\\
2.61	-2.50840381132191e-10\\
2.61399999999994	-2.44002909613744e-10\\
2.61599999999997	-2.40641466873175e-10\\
2.616	-2.40641466873127e-10\\
2.61999999999994	-2.3403098311458e-10\\
2.61999999999997	-2.3403098311453e-10\\
2.62	-2.34030983114479e-10\\
2.62399999999995	-2.27567512656683e-10\\
2.62499999999997	-2.25974220456848e-10\\
2.625	-2.25974220456803e-10\\
2.62599999999997	-2.24389854453506e-10\\
2.626	-2.24389854453461e-10\\
2.62699999999999	-2.22814363171672e-10\\
2.62799999999998	-2.21247695423551e-10\\
2.62999999999995	-2.18140627211147e-10\\
2.63199999999997	-2.15068246021307e-10\\
2.632	-2.15068246021263e-10\\
2.63599999999995	-2.09020651778157e-10\\
2.63899999999997	-2.04566114088466e-10\\
2.639	-2.04566114088424e-10\\
2.63999999999997	-2.03096468565584e-10\\
2.64	-2.03096468565542e-10\\
2.64099999999999	-2.01634343444653e-10\\
2.64199999999998	-2.00179691221145e-10\\
2.64399999999995	-1.97292616662743e-10\\
2.64599999999997	-1.94434869685796e-10\\
2.646	-1.94434869685756e-10\\
2.64999999999995	-1.8880587667484e-10\\
2.65099999999997	-1.87416382405251e-10\\
2.651	-1.87416382405212e-10\\
2.65499999999995	-1.81928066897559e-10\\
2.6589999999999	-1.76549078175463e-10\\
2.65999999999997	-1.75221084379237e-10\\
2.66	-1.75221084379199e-10\\
2.66599999999997	-1.67390535998993e-10\\
2.666	-1.67390535998956e-10\\
2.66799999999997	-1.64831685153533e-10\\
2.668	-1.64831685153497e-10\\
2.66999999999996	-1.6229794327925e-10\\
2.67199999999993	-1.5978898108633e-10\\
2.67199999999996	-1.59788981086285e-10\\
2.672	-1.5978898108624e-10\\
2.67599999999993	-1.54869252357506e-10\\
2.67799999999997	-1.52464135672581e-10\\
2.678	-1.52464135672547e-10\\
2.67999999999997	-1.50095099106683e-10\\
2.68	-1.5009509910665e-10\\
2.68199999999998	-1.47761834780912e-10\\
2.68399999999995	-1.45464039464016e-10\\
2.68599999999997	-1.43201414534206e-10\\
2.686	-1.43201414534174e-10\\
2.68999999999995	-1.38780504170762e-10\\
2.69399999999989	-1.34496805452586e-10\\
2.69499999999997	-1.33447057571004e-10\\
2.695	-1.33447057570974e-10\\
2.69699999999997	-1.31372734264259e-10\\
2.697	-1.3137273426423e-10\\
2.69899999999996	-1.29331750056121e-10\\
2.69999999999997	-1.28323677051056e-10\\
2.7	-1.28323677051027e-10\\
2.70199999999997	-1.26332205478112e-10\\
2.70399999999993	-1.24373417923541e-10\\
2.70599999999997	-1.22447059823027e-10\\
2.706	-1.22447059822999e-10\\
2.70899999999997	-1.19617781406167e-10\\
2.709	-1.1961778140614e-10\\
2.71199999999996	-1.16860079637936e-10\\
2.71299999999997	-1.15956611662643e-10\\
2.713	-1.15956611662617e-10\\
2.71599999999997	-1.13283232221075e-10\\
2.71899999999993	-1.10659852310813e-10\\
2.71999999999997	-1.0979637034201e-10\\
2.72	-1.09796370341985e-10\\
2.72599999999993	-1.04728792984089e-10\\
2.726	-1.04728792984037e-10\\
2.72600000000002	-1.04728792984013e-10\\
2.72999999999997	-1.01456323809824e-10\\
2.73	-1.01456323809801e-10\\
2.73199999999999	-9.98512219024294e-11\\
2.73200000000002	-9.98512219024067e-11\\
2.73400000000002	-9.8266596087112e-11\\
2.73600000000001	-9.67022404257558e-11\\
2.73799999999999	-9.51579516145365e-11\\
2.73800000000002	-9.51579516145147e-11\\
2.73999999999997	-9.36345920164655e-11\\
2.74	-9.3634592016444e-11\\
2.74199999999995	-9.21330267144909e-11\\
2.7439999999999	-9.0653060565046e-11\\
2.74599999999997	-8.91945012314817e-11\\
2.746	-8.91945012314611e-11\\
2.7499999999999	-8.63408475555862e-11\\
2.7539999999998	-8.35705821969859e-11\\
2.75499999999997	-8.2890875570351e-11\\
2.755	-8.28908755703318e-11\\
2.75999999999997	-7.95683986044432e-11\\
2.76	-7.95683986044247e-11\\
2.76499999999998	-7.63707178432406e-11\\
2.76500000000001	-7.63707178432228e-11\\
2.76599999999997	-7.57459268971584e-11\\
2.766	-7.57459268971407e-11\\
2.76699999999999	-7.51260032841652e-11\\
2.76700000000002	-7.51260032841477e-11\\
2.76800000000001	-7.45109268625562e-11\\
2.76899999999999	-7.39006776487074e-11\\
2.77099999999997	-7.26945816932022e-11\\
2.77299999999999	-7.15075586732612e-11\\
2.77300000000002	-7.15075586732445e-11\\
2.77699999999997	-6.91885673252657e-11\\
2.77999999999997	-6.74962265911433e-11\\
2.78	-6.74962265911275e-11\\
2.78399999999995	-6.53013684486913e-11\\
2.784	-6.53013684486687e-11\\
2.786	-6.42300161952202e-11\\
2.78600000000003	-6.42300161952051e-11\\
2.78800000000004	-6.31758620927788e-11\\
2.79000000000004	-6.21387691433918e-11\\
2.792	-6.11186025663693e-11\\
2.79200000000003	-6.1118602566355e-11\\
2.79600000000004	-5.91285203907031e-11\\
2.79999999999997	-5.72045810091602e-11\\
2.8	-5.72045810091468e-11\\
2.80400000000001	-5.53457842499754e-11\\
2.80599999999997	-5.44405110736612e-11\\
2.806	-5.44405110736484e-11\\
2.81000000000001	-5.2677626867633e-11\\
2.81199999999997	-5.18197867333533e-11\\
2.812	-5.18197867333412e-11\\
2.81299999999997	-5.13967496507354e-11\\
2.813	-5.13967496507234e-11\\
2.81399999999998	-5.09776583245847e-11\\
2.81499999999997	-5.05624991386017e-11\\
2.81699999999995	-4.97439233608487e-11\\
2.81899999999997	-4.89409159352597e-11\\
2.819	-4.89409159352484e-11\\
2.81999999999997	-4.85452176638304e-11\\
2.82	-4.85452176638192e-11\\
2.82099999999999	-4.81533725034787e-11\\
2.82199999999998	-4.77653677231287e-11\\
2.82399999999995	-4.70008290021468e-11\\
2.82599999999997	-4.62515021413308e-11\\
2.826	-4.62515021413202e-11\\
2.82999999999995	-4.47980964361652e-11\\
2.83399999999991	-4.34043950455514e-11\\
2.83499999999997	-4.30652129135926e-11\\
2.835	-4.3065212913583e-11\\
2.83999999999997	-4.14242120398079e-11\\
2.84	-4.14242120397988e-11\\
2.84199999999997	-4.0793237764968e-11\\
2.842	-4.07932377649591e-11\\
2.84399999999996	-4.01766686137662e-11\\
2.84599999999992	-3.95744244564068e-11\\
2.846	-3.9574424456384e-11\\
2.84600000000003	-3.95744244563755e-11\\
2.847	-3.92786497192185e-11\\
2.84700000000003	-3.92786497192102e-11\\
2.84800000000002	-3.89856909568888e-11\\
2.84900000000001	-3.86948025828463e-11\\
2.85099999999998	-3.81191992630671e-11\\
2.853	-3.75517649543981e-11\\
2.85300000000003	-3.75517649543901e-11\\
2.85699999999998	-3.64411094487245e-11\\
2.85999999999997	-3.5629020632153e-11\\
2.86	-3.56290206321454e-11\\
2.86399999999995	-3.45736544750587e-11\\
2.86599999999997	-3.40575677819514e-11\\
2.866	-3.40575677819441e-11\\
2.86999999999995	-3.30482518059052e-11\\
2.87	-3.30482518058932e-11\\
2.87099999999997	-3.28006368600938e-11\\
2.871	-3.28006368600867e-11\\
2.87199999999998	-3.25548913521842e-11\\
2.87299999999997	-3.2311007297888e-11\\
2.87499999999995	-3.18287919154679e-11\\
2.87699999999997	-3.13539280440591e-11\\
2.87699999999999	-3.13539280440524e-11\\
2.87999999999997	-3.06552815821347e-11\\
2.88	-3.06552815821281e-11\\
2.88299999999998	-2.99728330940074e-11\\
2.88599999999996	-2.93063830224564e-11\\
2.886	-2.93063830224472e-11\\
2.888	-2.88708745557116e-11\\
2.88800000000003	-2.88708745557054e-11\\
2.89000000000003	-2.84416967299812e-11\\
2.89200000000003	-2.8018157307795e-11\\
2.89600000000002	-2.7187774227244e-11\\
2.89999999999997	-2.63792936304578e-11\\
2.9	-2.63792936304521e-11\\
2.90499999999997	-2.53988540351364e-11\\
2.905	-2.53988540351309e-11\\
2.90599999999997	-2.52067235697621e-11\\
2.90599999999999	-2.52067235697567e-11\\
2.90699999999998	-2.5015897590732e-11\\
2.90799999999997	-2.48263698979974e-11\\
2.90999999999995	-2.44511847825973e-11\\
2.91199999999997	-2.40811194644551e-11\\
2.91199999999999	-2.40811194644499e-11\\
2.91599999999995	-2.33561565051707e-11\\
2.91799999999997	-2.30011646477571e-11\\
2.91799999999999	-2.30011646477521e-11\\
2.91999999999997	-2.26517650986021e-11\\
2.92	-2.26517650985972e-11\\
2.92199999999998	-2.23085734051255e-11\\
2.92399999999996	-2.19715449661014e-11\\
2.926	-2.16406359812663e-11\\
2.92600000000003	-2.16406359812616e-11\\
2.92899999999997	-2.11556526238781e-11\\
2.92899999999999	-2.11556526238736e-11\\
2.93199999999993	-2.06841996471354e-11\\
2.93499999999987	-2.02261391917741e-11\\
2.93499999999997	-2.02261391917593e-11\\
2.935	-2.0226139191755e-11\\
2.93999999999997	-1.94921042181311e-11\\
2.94	-1.9492104218127e-11\\
2.94499999999998	-1.87943039713519e-11\\
2.94599999999997	-1.86590416961837e-11\\
2.946	-1.86590416961799e-11\\
2.95099999999998	-1.80039774769861e-11\\
2.95199999999997	-1.78771842684517e-11\\
2.952	-1.78771842684481e-11\\
2.95699999999998	-1.72596680839597e-11\\
2.95799999999999	-1.71392498725293e-11\\
2.95800000000002	-1.71392498725259e-11\\
2.95999999999997	-1.69014669763732e-11\\
2.96	-1.69014669763699e-11\\
2.96199999999995	-1.66677297525617e-11\\
2.9639999999999	-1.64380078244331e-11\\
2.96599999999997	-1.62122713372871e-11\\
2.966	-1.6212271337284e-11\\
2.9699999999999	-1.57726378537941e-11\\
2.96999999999999	-1.57726378537844e-11\\
2.97000000000002	-1.57726378537813e-11\\
2.97399999999992	-1.53486007558365e-11\\
2.97499999999997	-1.52450024074328e-11\\
2.975	-1.52450024074299e-11\\
2.9789999999999	-1.48401517392571e-11\\
2.97999999999997	-1.47413083116871e-11\\
2.98	-1.47413083116843e-11\\
2.9839999999999	-1.43553152151184e-11\\
2.986	-1.41679032482607e-11\\
2.98600000000003	-1.41679032482581e-11\\
2.98699999999997	-1.40755826753259e-11\\
2.98699999999999	-1.40755826753233e-11\\
2.98799999999998	-1.39839560690341e-11\\
2.98899999999997	-1.38927948213271e-11\\
2.99099999999995	-1.37118565694808e-11\\
2.99299999999999	-1.35327444050262e-11\\
2.99300000000002	-1.35327444050237e-11\\
2.99699999999997	-1.31799054633445e-11\\
2.99999999999997	-1.29198983070989e-11\\
3	-1.29198983070965e-11\\
3.00399999999995	-1.25792402797173e-11\\
3.00599999999997	-1.24114406179577e-11\\
3.006	-1.24114406179553e-11\\
3.00999999999995	-1.20807909596136e-11\\
3.01	-1.20807909596096e-11\\
3.01399999999995	-1.17565979689463e-11\\
3.01599999999997	-1.15968699284531e-11\\
3.01599999999999	-1.15968699284508e-11\\
3.01999999999995	-1.12820156488957e-11\\
3.02	-1.12820156488914e-11\\
3.02399999999995	-1.09731401595441e-11\\
3.02599999999997	-1.08208941636547e-11\\
3.026	-1.08208941636525e-11\\
3.02799999999997	-1.06700828894996e-11\\
3.02799999999999	-1.06700828894975e-11\\
3.02999999999996	-1.05206867377158e-11\\
3.03199999999992	-1.03726862927663e-11\\
3.03599999999985	-1.0080795765813e-11\\
3.03999999999997	-9.79425956619819e-12\\
3.04	-9.79425956619617e-12\\
3.04499999999997	-9.44339217328454e-12\\
3.04499999999999	-9.44339217328256e-12\\
3.04599999999997	-9.37416953136604e-12\\
3.046	-9.37416953136408e-12\\
3.04699999999998	-9.30525856340003e-12\\
3.04799999999996	-9.23665703042779e-12\\
3.04999999999992	-9.10037336407018e-12\\
3.05199999999997	-8.96530082084684e-12\\
3.052	-8.96530082084492e-12\\
3.05599999999992	-8.69871904313831e-12\\
3.05799999999997	-8.56717516364819e-12\\
3.058	-8.56717516364633e-12\\
3.05999999999997	-8.43721175756052e-12\\
3.06	-8.43721175755869e-12\\
3.06199999999997	-8.30925057944518e-12\\
3.06399999999995	-8.18327499945535e-12\\
3.066	-8.05926864578576e-12\\
3.06600000000003	-8.05926864578401e-12\\
3.06999999999997	-7.81709940807259e-12\\
3.07399999999992	-7.58261697310577e-12\\
3.07399999999996	-7.58261697310359e-12\\
3.07399999999999	-7.58261697310142e-12\\
3.07999999999997	-7.24504050239408e-12\\
3.07999999999999	-7.24504050239252e-12\\
3.08599999999997	-6.92409111493065e-12\\
3.08599999999999	-6.92409111492917e-12\\
3.09199999999997	-6.61939338301488e-12\\
3.09199999999999	-6.61939338301347e-12\\
3.09799999999997	-6.32963046213013e-12\\
3.09999999999997	-6.23608919357034e-12\\
3.1	-6.23608919356902e-12\\
3.10299999999997	-6.09859122149346e-12\\
3.10299999999999	-6.09859122149217e-12\\
3.10599999999996	-5.96443430053832e-12\\
3.106	-5.96443430053636e-12\\
3.10600000000003	-5.96443430053511e-12\\
3.10899999999999	-5.83357920119237e-12\\
3.11199999999996	-5.70598765965977e-12\\
3.11199999999999	-5.70598765965822e-12\\
3.11200000000003	-5.7059876596567e-12\\
3.11499999999997	-5.58162236625439e-12\\
3.115	-5.58162236625322e-12\\
3.11799999999995	-5.46044695526774e-12\\
3.11999999999997	-5.38141796871775e-12\\
3.12	-5.38141796871664e-12\\
3.12299999999995	-5.26548074329917e-12\\
3.12599999999989	-5.15264095601907e-12\\
3.12599999999995	-5.15264095601701e-12\\
3.126	-5.15264095601493e-12\\
3.127	-5.11571028420988e-12\\
3.12700000000003	-5.11571028420883e-12\\
3.12800000000003	-5.07905060955641e-12\\
3.12900000000003	-5.04259245166776e-12\\
3.13100000000003	-4.97027595463383e-12\\
3.13199999999997	-4.93441526593593e-12\\
3.13199999999999	-4.93441526593492e-12\\
3.13599999999999	-4.79292913192514e-12\\
3.13799999999997	-4.72334397046571e-12\\
3.13799999999999	-4.72334397046473e-12\\
3.13999999999997	-4.65451865504145e-12\\
3.14	-4.65451865504048e-12\\
3.14199999999998	-4.58644424114856e-12\\
3.14399999999996	-4.51911188183016e-12\\
3.14599999999997	-4.45251282656451e-12\\
3.146	-4.45251282656357e-12\\
3.14999999999996	-4.32148010166643e-12\\
3.15	-4.32148010166511e-12\\
3.15399999999996	-4.19327794827004e-12\\
3.15599999999997	-4.13021745225328e-12\\
3.156	-4.13021745225239e-12\\
3.15999999999996	-4.00619030931576e-12\\
3.16	-4.00619030931452e-12\\
3.16099999999997	-3.97561944725066e-12\\
3.16099999999999	-3.9756194472498e-12\\
3.16199999999996	-3.94522095972702e-12\\
3.16299999999992	-3.91499385909558e-12\\
3.16499999999985	-3.85504989576685e-12\\
3.16599999999997	-3.82533108549228e-12\\
3.166	-3.82533108549143e-12\\
3.16999999999986	-3.70812118741818e-12\\
3.17199999999997	-3.65050213452436e-12\\
3.172	-3.65050213452355e-12\\
3.17599999999986	-3.53719837933177e-12\\
3.17999999999972	-3.42642481273379e-12\\
3.17999999999997	-3.42642481272685e-12\\
3.18	-3.42642481272607e-12\\
3.18499999999997	-3.29142832544425e-12\\
3.185	-3.29142832544349e-12\\
3.18599999999997	-3.2648829482323e-12\\
3.186	-3.26488294823155e-12\\
3.18699999999997	-3.23848685396699e-12\\
3.18799999999994	-3.21223918502321e-12\\
3.18999999999989	-3.16018571677825e-12\\
3.18999999999994	-3.16018571677686e-12\\
3.18999999999999	-3.16018571677547e-12\\
3.19399999999988	-3.05782268157658e-12\\
3.19599999999999	-3.00749981152321e-12\\
3.19600000000002	-3.0074998115225e-12\\
3.198	-2.95774062854675e-12\\
3.19800000000003	-2.95774062854604e-12\\
3.19999999999998	-2.90860270417661e-12\\
3.20000000000001	-2.90860270417592e-12\\
3.20199999999996	-2.86014369065341e-12\\
3.20399999999991	-2.81235729025376e-12\\
3.20599999999998	-2.76523729264911e-12\\
3.20600000000001	-2.76523729264844e-12\\
3.20999999999991	-2.67297209689359e-12\\
3.21399999999982	-2.58330013874443e-12\\
3.21899999999997	-2.4747860401545e-12\\
3.21899999999999	-2.4747860401539e-12\\
3.21999999999997	-2.4535529106153e-12\\
3.22	-2.4535529106147e-12\\
3.22099999999998	-2.43247472308926e-12\\
3.22199999999996	-2.41155079264742e-12\\
3.22399999999992	-2.37016298875636e-12\\
3.22599999999997	-2.32938412017674e-12\\
3.226	-2.32938412017617e-12\\
3.22999999999992	-2.24963206901337e-12\\
3.23099999999999	-2.23006645497329e-12\\
3.23100000000002	-2.23006645497274e-12\\
3.23499999999994	-2.15327452291503e-12\\
3.23699999999999	-2.11575215562313e-12\\
3.23700000000002	-2.1157521556226e-12\\
3.23999999999997	-2.06083010559489e-12\\
3.24	-2.06083010559438e-12\\
3.24299999999996	-2.00775421352683e-12\\
3.24599999999991	-1.95650895930596e-12\\
3.24599999999997	-1.95650895930485e-12\\
3.246	-1.95650895930437e-12\\
3.24799999999997	-1.92335505564688e-12\\
3.24799999999999	-1.92335505564641e-12\\
3.24999999999996	-1.89100378936893e-12\\
3.25199999999992	-1.85945095603547e-12\\
3.25399999999997	-1.82869245503542e-12\\
3.25399999999999	-1.82869245503499e-12\\
3.25499999999998	-1.81360982596709e-12\\
3.255	-1.81360982596667e-12\\
3.25599999999998	-1.7987242890141e-12\\
3.25699999999997	-1.78403536054407e-12\\
3.25899999999993	-1.75524542646907e-12\\
3.25999999999997	-1.74114348548134e-12\\
3.26	-1.74114348548094e-12\\
3.26399999999993	-1.68667861169982e-12\\
3.26599999999997	-1.66060573749454e-12\\
3.266	-1.66060573749418e-12\\
3.267	-1.64785769218243e-12\\
3.26700000000003	-1.64785769218207e-12\\
3.26800000000003	-1.63525105221471e-12\\
3.26900000000003	-1.62273510646564e-12\\
3.27100000000003	-1.59797367399911e-12\\
3.27500000000003	-1.54952144336905e-12\\
3.27699999999997	-1.52582434820566e-12\\
3.27699999999999	-1.52582434820533e-12\\
3.27999999999997	-1.49093130312942e-12\\
3.28	-1.49093130312909e-12\\
3.28299999999998	-1.45681232287604e-12\\
3.28599999999996	-1.42345743036958e-12\\
3.286	-1.42345743036915e-12\\
3.28899999999999	-1.39085687213843e-12\\
3.28900000000002	-1.39085687213813e-12\\
3.29	-1.38015599530777e-12\\
3.29000000000003	-1.38015599530747e-12\\
3.29100000000001	-1.36953752473796e-12\\
3.29199999999999	-1.35900111543714e-12\\
3.29399999999996	-1.33817311399801e-12\\
3.296	-1.3176692841802e-12\\
3.29600000000003	-1.31766928417991e-12\\
3.29999999999996	-1.27747873705213e-12\\
3.3	-1.27747873705169e-12\\
3.30000000000003	-1.27747873705141e-12\\
3.30399999999996	-1.23826379526547e-12\\
3.30599999999997	-1.21901578618486e-12\\
3.30599999999999	-1.21901578618459e-12\\
3.30999999999992	-1.1812261840096e-12\\
3.31199999999999	-1.16267967977326e-12\\
3.31200000000002	-1.162679679773e-12\\
3.31599999999995	-1.1262712132146e-12\\
3.31999999999988	-1.09075974584147e-12\\
3.32	-1.09075974584036e-12\\
3.32000000000003	-1.09075974584011e-12\\
3.32499999999998	-1.04760373626409e-12\\
3.325	-1.04760373626384e-12\\
3.326	-1.03913416014032e-12\\
3.32600000000003	-1.03913416014008e-12\\
3.32700000000003	-1.03071781314122e-12\\
3.32800000000003	-1.02235442181453e-12\\
3.33000000000002	-1.00578542099349e-12\\
3.332	-9.89425004368199e-13\\
3.33200000000003	-9.89425004367968e-13\\
3.33499999999999	-9.65289783590867e-13\\
3.33500000000002	-9.6528978359064e-13\\
3.33799999999998	-9.41649947129182e-13\\
3.33999999999997	-9.2616185012817e-13\\
3.34	-9.26161850127951e-13\\
3.34299999999997	-9.0333236345637e-13\\
3.34599999999993	-8.80980143910773e-13\\
3.34599999999997	-8.80980143910498e-13\\
3.346	-8.80980143910226e-13\\
3.34699999999999	-8.73634329731536e-13\\
3.34700000000002	-8.73634329731328e-13\\
3.34800000000001	-8.66340579218316e-13\\
3.349	-8.59098655385848e-13\\
3.35099999999999	-8.44769348287355e-13\\
3.35499999999995	-8.16722413472853e-13\\
3.35999999999997	-7.82791950518157e-13\\
3.36	-7.82791950517967e-13\\
3.36399999999997	-7.56532946631132e-13\\
3.36399999999999	-7.56532946630948e-13\\
3.36599999999997	-7.43693848316785e-13\\
3.366	-7.43693848316604e-13\\
3.36799999999998	-7.31046119266725e-13\\
3.36999999999996	-7.18588115780895e-13\\
3.37199999999997	-7.06318218815622e-13\\
3.372	-7.06318218815449e-13\\
3.37599999999996	-6.82336390330809e-13\\
3.37799999999997	-6.70621342130165e-13\\
3.378	-6.7062134213e-13\\
3.37999999999997	-6.59129005982098e-13\\
3.38	-6.59129005981936e-13\\
3.38199999999997	-6.47898727628992e-13\\
3.38399999999995	-6.36929047582701e-13\\
3.38599999999997	-6.26218540222072e-13\\
3.386	-6.26218540221921e-13\\
3.38999999999995	-6.05569509334561e-13\\
3.39299999999997	-5.9075295514787e-13\\
3.39299999999999	-5.90752955147732e-13\\
3.39499999999998	-5.81191981778849e-13\\
3.395	-5.81191981778715e-13\\
3.39699999999999	-5.7188293752358e-13\\
3.39899999999997	-5.62824612577657e-13\\
3.399	-5.6282461257753e-13\\
3.39999999999997	-5.58389100694485e-13\\
3.4	-5.5838910069436e-13\\
3.40099999999998	-5.54015829720831e-13\\
3.40199999999996	-5.49704657568737e-13\\
3.40399999999991	-5.41268051467328e-13\\
3.40599999999997	-5.33078185967531e-13\\
3.406	-5.33078185967417e-13\\
3.40999999999991	-5.17434451302054e-13\\
3.41099999999997	-5.1367610608931e-13\\
3.411	-5.13676106089204e-13\\
3.41499999999991	-4.99019454729664e-13\\
3.41899999999982	-4.84867912648881e-13\\
3.41999999999997	-4.81408082600115e-13\\
3.42	-4.81408082600017e-13\\
3.42199999999997	-4.74581298670417e-13\\
3.42199999999999	-4.74581298670321e-13\\
3.42399999999996	-4.67877612026575e-13\\
3.42599999999992	-4.61296151412028e-13\\
3.426	-4.61296151411756e-13\\
3.42600000000003	-4.61296151411663e-13\\
3.42999999999996	-4.48496502742112e-13\\
3.43	-4.48496502741959e-13\\
3.432	-4.42276651245099e-13\\
3.43200000000003	-4.42276651245011e-13\\
3.43400000000003	-4.36175698683289e-13\\
3.43600000000003	-4.30192852176216e-13\\
3.438	-4.24327334192544e-13\\
3.43800000000003	-4.24327334192462e-13\\
3.43999999999997	-4.18554197259641e-13\\
3.44	-4.18554197259559e-13\\
3.44199999999995	-4.12848505910518e-13\\
3.44399999999989	-4.07209518633014e-13\\
3.44599999999997	-4.01636502583354e-13\\
3.446	-4.01636502583275e-13\\
3.44999999999989	-3.90685495584634e-13\\
3.45099999999996	-3.87987854428571e-13\\
3.45099999999999	-3.87987854428495e-13\\
3.45499999999988	-3.77355102747119e-13\\
3.45699999999996	-3.7213220735603e-13\\
3.45699999999999	-3.72132207355956e-13\\
3.45999999999997	-3.64412802946371e-13\\
3.46	-3.64412802946298e-13\\
3.46299999999998	-3.56829321475419e-13\\
3.46499999999998	-3.51848083220725e-13\\
3.465	-3.51848083220654e-13\\
3.46599999999997	-3.49379545429496e-13\\
3.466	-3.49379545429426e-13\\
3.46699999999997	-3.46925621434294e-13\\
3.46799999999994	-3.44486231507651e-13\\
3.46999999999988	-3.39650737309398e-13\\
3.47199999999997	-3.34872434413023e-13\\
3.472	-3.34872434412955e-13\\
3.47599999999988	-3.25469660448457e-13\\
3.47999999999976	-3.16257756033951e-13\\
3.47999999999996	-3.16257756033484e-13\\
3.47999999999999	-3.16257756033419e-13\\
3.48599999999996	-3.0278733067587e-13\\
3.48599999999999	-3.02787330675808e-13\\
3.49199999999996	-2.89719852659886e-13\\
3.49199999999999	-2.89719852659824e-13\\
3.49799999999996	-2.77040036420389e-13\\
3.5	-2.72897001763409e-13\\
3.50000000000003	-2.72897001763351e-13\\
3.506	-2.60711071630858e-13\\
3.50600000000003	-2.60711071630801e-13\\
3.50899999999999	-2.54751629494733e-13\\
3.50900000000002	-2.54751629494677e-13\\
3.51199999999998	-2.48878870545927e-13\\
3.51200000000003	-2.48878870545837e-13\\
3.51499999999999	-2.43112755320934e-13\\
3.51799999999995	-2.37473275549401e-13\\
3.51799999999999	-2.37473275549331e-13\\
3.51800000000003	-2.37473275549262e-13\\
3.51999999999997	-2.33783159530413e-13\\
3.52	-2.33783159530361e-13\\
3.52199999999995	-2.3014811270856e-13\\
3.52399999999989	-2.26567662666115e-13\\
3.52599999999997	-2.23041344087154e-13\\
3.526	-2.23041344087105e-13\\
3.52999999999989	-2.16149275182676e-13\\
3.53399999999978	-2.09468323109701e-13\\
3.53499999999998	-2.07830662694518e-13\\
3.535	-2.07830662694472e-13\\
3.53799999999997	-2.02995014695301e-13\\
3.53799999999999	-2.02995014695256e-13\\
3.53999999999997	-1.99835171104662e-13\\
3.54	-1.99835171104618e-13\\
3.54199999999998	-1.96725984829474e-13\\
3.54399999999996	-1.93667051800383e-13\\
3.54599999999997	-1.90657974478074e-13\\
3.546	-1.90657974478032e-13\\
3.54999999999996	-1.84787829150406e-13\\
3.54999999999999	-1.84787829150363e-13\\
3.553	-1.80513229506385e-13\\
3.55300000000003	-1.80513229506345e-13\\
3.55600000000004	-1.76332123334914e-13\\
3.55900000000005	-1.72228450990501e-13\\
3.55999999999997	-1.70877559676222e-13\\
3.56	-1.70877559676183e-13\\
3.56599999999999	-1.62947648196046e-13\\
3.56600000000002	-1.6294764819601e-13\\
3.56699999999996	-1.61654834909373e-13\\
3.56699999999999	-1.61654834909337e-13\\
3.56799999999996	-1.60370148336711e-13\\
3.56899999999992	-1.59093546736702e-13\\
3.56999999999998	-1.57824988632788e-13\\
3.57	-1.57824988632752e-13\\
3.57199999999993	-1.5531183831365e-13\\
3.57299999999996	-1.54067164446752e-13\\
3.57299999999999	-1.54067164446717e-13\\
3.57499999999992	-1.51601417102002e-13\\
3.57699999999985	-1.49166870327812e-13\\
3.57899999999996	-1.46763207730002e-13\\
3.57899999999999	-1.46763207729968e-13\\
3.57999999999997	-1.45573505076217e-13\\
3.58	-1.45573505076184e-13\\
3.58099999999998	-1.44392696363181e-13\\
3.58199999999997	-1.43220743226331e-13\\
3.58399999999993	-1.40903251662004e-13\\
3.58599999999997	-1.38620729201687e-13\\
3.586	-1.38620729201655e-13\\
3.58999999999993	-1.34159409558185e-13\\
3.59399999999986	-1.2983446505078e-13\\
3.59599999999996	-1.27722428112649e-13\\
3.59599999999999	-1.27722428112619e-13\\
3.59999999999997	-1.23597852469784e-13\\
3.6	-1.23597852469755e-13\\
3.60399999999998	-1.19604161493547e-13\\
3.60499999999998	-1.18625944697342e-13\\
3.605	-1.18625944697314e-13\\
3.60599999999997	-1.1765574632357e-13\\
3.606	-1.17655746323542e-13\\
3.60699999999997	-1.16693534851186e-13\\
3.60799999999994	-1.1573927901787e-13\\
3.60799999999999	-1.15739279017821e-13\\
3.60999999999993	-1.13854510511877e-13\\
3.61199999999987	-1.12001195861509e-13\\
3.61399999999996	-1.10179094209628e-13\\
3.61399999999999	-1.10179094209602e-13\\
3.61799999999987	-1.06623354111207e-13\\
3.61999999999997	-1.04888195439617e-13\\
3.62	-1.04888195439593e-13\\
3.62399999999988	-1.0150217332053e-13\\
3.62499999999996	-1.0067306853105e-13\\
3.62499999999999	-1.00673068531026e-13\\
3.62599999999997	-9.98508698248068e-14\\
3.626	-9.98508698247835e-14\\
3.62699999999998	-9.90355504890917e-14\\
3.62799999999997	-9.82270840341185e-14\\
3.62999999999993	-9.66306049208317e-14\\
3.63199999999997	-9.50612250062793e-14\\
3.632	-9.50612250062572e-14\\
3.63599999999993	-9.20029504460641e-14\\
3.63999999999985	-8.90506704902203e-14\\
3.63999999999997	-8.90506704901318e-14\\
3.64	-8.90506704901112e-14\\
3.64599999999997	-8.48176480144035e-14\\
3.646	-8.4817648014384e-14\\
3.65199999999997	-8.08147213859459e-14\\
3.652	-8.08147213859274e-14\\
3.65399999999996	-7.9529974896962e-14\\
3.65399999999999	-7.9529974896944e-14\\
3.65599999999995	-7.82685990709824e-14\\
3.65799999999992	-7.70304299771485e-14\\
3.65999999999996	-7.58153067028163e-14\\
3.65999999999999	-7.58153067027992e-14\\
3.66399999999992	-7.34535689204675e-14\\
3.66599999999996	-7.23066474806705e-14\\
3.66599999999999	-7.23066474806544e-14\\
3.66999999999992	-7.00799542175549e-14\\
3.67399999999984	-6.79418339817725e-14\\
3.67499999999998	-6.74210121966049e-14\\
3.67500000000001	-6.74210121965902e-14\\
3.67999999999997	-6.48983090529137e-14\\
3.68	-6.48983090528997e-14\\
3.68299999999996	-6.34492248225999e-14\\
3.68299999999999	-6.34492248225864e-14\\
3.68599999999995	-6.20480230617279e-14\\
3.686	-6.20480230617055e-14\\
3.68699999999998	-6.15915251380261e-14\\
3.68700000000001	-6.15915251380132e-14\\
3.68799999999998	-6.11391712039542e-14\\
3.68899999999995	-6.06898307491993e-14\\
3.6909999999999	-5.98001319793103e-14\\
3.69299999999998	-5.89223132029539e-14\\
3.693	-5.89223132029415e-14\\
3.6969999999999	-5.72018608368551e-14\\
3.69999999999997	-5.5941869149571e-14\\
3.7	-5.59418691495592e-14\\
3.7039999999999	-5.43016431837418e-14\\
3.70599999999997	-5.34983312421547e-14\\
3.706	-5.34983312421434e-14\\
3.7099999999999	-5.19247874378742e-14\\
3.70999999999998	-5.19247874378434e-14\\
3.71000000000001	-5.19247874378324e-14\\
3.71199999999996	-5.11543510771801e-14\\
3.71199999999999	-5.11543510771692e-14\\
3.71399999999995	-5.03946711065675e-14\\
3.71599999999991	-4.96456487979261e-14\\
3.71799999999996	-4.89071868081999e-14\\
3.71799999999999	-4.89071868081895e-14\\
3.71999999999997	-4.8179189167284e-14\\
3.72	-4.81791891672737e-14\\
3.72199999999998	-4.74615612649377e-14\\
3.72399999999997	-4.67542098381721e-14\\
3.72599999999997	-4.60570429595098e-14\\
3.726	-4.60570429595e-14\\
3.728	-4.53699700254876e-14\\
3.72800000000003	-4.53699700254779e-14\\
3.73000000000004	-4.46921711810974e-14\\
3.73200000000004	-4.4022827776191e-14\\
3.73600000000004	-4.27091604214052e-14\\
3.74	-4.14282850369943e-14\\
3.74000000000003	-4.14282850369853e-14\\
3.74099999999996	-4.11131115359076e-14\\
3.74099999999999	-4.11131115358987e-14\\
3.74199999999996	-4.07999355965395e-14\\
3.74299999999992	-4.04887470423596e-14\\
3.74499999999985	-3.98722917121025e-14\\
3.745	-3.98722917120556e-14\\
3.74500000000003	-3.98722917120468e-14\\
3.746	-3.95670049074038e-14\\
3.74600000000003	-3.95670049073951e-14\\
3.747	-3.92636654305224e-14\\
3.74799999999997	-3.89622634258905e-14\\
3.74999999999991	-3.83652327261511e-14\\
3.752	-3.77758352179644e-14\\
3.75200000000003	-3.77758352179561e-14\\
3.75599999999991	-3.66196343674435e-14\\
3.758	-3.60526807650458e-14\\
3.75800000000003	-3.60526807650378e-14\\
3.75999999999997	-3.54943227294292e-14\\
3.76	-3.54943227294214e-14\\
3.76199999999995	-3.49457506103997e-14\\
3.76399999999989	-3.44068931154712e-14\\
3.76599999999997	-3.38776802146275e-14\\
3.766	-3.38776802146201e-14\\
3.76999999999989	-3.28479143350018e-14\\
3.76999999999996	-3.28479143349834e-14\\
3.76999999999999	-3.28479143349762e-14\\
3.77399999999988	-3.18559176412169e-14\\
3.77599999999999	-3.13739208244165e-14\\
3.77600000000002	-3.13739208244097e-14\\
3.77999999999991	-3.04376170424985e-14\\
3.77999999999996	-3.04376170424878e-14\\
3.78	-3.0437617042477e-14\\
3.78399999999989	-2.95378294386278e-14\\
3.78599999999997	-2.91014822934737e-14\\
3.786	-2.91014822934676e-14\\
3.78999999999989	-2.82555977720232e-14\\
3.79199999999997	-2.78459504644292e-14\\
3.792	-2.78459504644235e-14\\
3.79599999999989	-2.70489670685353e-14\\
3.79899999999996	-2.64687929440349e-14\\
3.79899999999999	-2.64687929440295e-14\\
3.79999999999997	-2.62787132556788e-14\\
3.8	-2.62787132556734e-14\\
3.80099999999999	-2.60902790732e-14\\
3.80199999999997	-2.59034842743579e-14\\
3.80399999999993	-2.55347886050143e-14\\
3.80599999999997	-2.51725783319041e-14\\
3.806	-2.5172578331899e-14\\
3.80999999999993	-2.4467426520601e-14\\
3.81099999999996	-2.42951194009421e-14\\
3.81099999999999	-2.42951194009372e-14\\
3.81499999999992	-2.36216463045562e-14\\
3.81499999999998	-2.36216463045468e-14\\
3.81500000000001	-2.36216463045421e-14\\
3.81899999999993	-2.29731210776856e-14\\
3.81999999999997	-2.28148480844999e-14\\
3.82	-2.28148480844955e-14\\
3.82399999999993	-2.21970350947492e-14\\
3.826	-2.18972259084023e-14\\
3.82600000000003	-2.1897225908398e-14\\
3.82700000000001	-2.17495784671099e-14\\
3.82700000000003	-2.17495784671057e-14\\
3.82799999999998	-2.16030760573187e-14\\
3.82800000000001	-2.16030760573145e-14\\
3.82899999999997	-2.14573605949561e-14\\
3.82999999999993	-2.13124273457786e-14\\
3.83199999999986	-2.10248886768832e-14\\
3.83399999999998	-2.07404226820477e-14\\
3.83400000000001	-2.07404226820437e-14\\
3.83799999999985	-2.01805612330449e-14\\
3.84	-1.99050930191724e-14\\
3.84000000000003	-1.99050930191685e-14\\
3.84399999999988	-1.93629026094901e-14\\
3.846	-1.90961099505069e-14\\
3.84600000000003	-1.90961099505031e-14\\
3.84999999999988	-1.85709563585486e-14\\
3.84999999999998	-1.85709563585356e-14\\
3.85000000000001	-1.85709563585319e-14\\
3.85399999999985	-1.8056818170852e-14\\
3.85599999999998	-1.78037961093084e-14\\
3.85600000000001	-1.78037961093049e-14\\
3.85699999999996	-1.76782740755903e-14\\
3.85699999999999	-1.76782740755868e-14\\
3.85799999999995	-1.75533948383861e-14\\
3.85899999999992	-1.74291543403613e-14\\
3.85999999999997	-1.73055485449632e-14\\
3.86	-1.73055485449597e-14\\
3.86199999999993	-1.70602250189506e-14\\
3.86399999999985	-1.68173923781427e-14\\
3.86599999999997	-1.65770190639593e-14\\
3.866	-1.65770190639559e-14\\
3.86899999999996	-1.62210020817481e-14\\
3.86899999999999	-1.62210020817447e-14\\
3.87199999999995	-1.5870344267002e-14\\
3.87499999999991	-1.55249430826071e-14\\
3.87999999999997	-1.49606822050036e-14\\
3.88	-1.49606822050004e-14\\
3.88499999999998	-1.44102847970004e-14\\
3.88500000000001	-1.44102847969973e-14\\
3.88599999999999	-1.43018293941066e-14\\
3.88600000000002	-1.43018293941035e-14\\
3.88700000000001	-1.41939070881438e-14\\
3.88799999999999	-1.40865143726512e-14\\
3.88999999999996	-1.38733037735374e-14\\
3.89199999999999	-1.36621698878863e-14\\
3.89200000000002	-1.36621698878833e-14\\
3.89599999999996	-1.32460227679542e-14\\
3.89799999999999	-1.3040955451143e-14\\
3.89800000000002	-1.30409554511401e-14\\
3.89999999999997	-1.28385087942447e-14\\
3.9	-1.28385087942419e-14\\
3.90199999999996	-1.2639308605591e-14\\
3.90399999999991	-1.24433289971063e-14\\
3.90599999999997	-1.22505444992488e-14\\
3.906	-1.22505444992461e-14\\
3.90999999999991	-1.18744610307849e-14\\
3.91399999999982	-1.15108626908422e-14\\
3.91499999999996	-1.14218916148904e-14\\
3.91499999999999	-1.14218916148879e-14\\
3.91999999999997	-1.09884630669536e-14\\
3.92	-1.09884630669512e-14\\
3.92499999999999	-1.05738233173555e-14\\
3.92599999999997	-1.04931201182458e-14\\
3.926	-1.04931201182435e-14\\
3.92699999999996	-1.04131523286452e-14\\
3.92699999999999	-1.04131523286429e-14\\
3.92799999999995	-1.03339173505001e-14\\
3.92899999999992	-1.02554126094167e-14\\
3.93099999999984	-1.0100583659692e-14\\
3.93299999999996	-9.9486453578684e-15\\
3.93299999999999	-9.94864535786626e-15\\
3.93699999999984	-9.65271788783832e-15\\
3.93999999999997	-9.43737039441022e-15\\
3.94	-9.4373703944082e-15\\
3.94399999999985	-9.15891815838489e-15\\
3.94399999999996	-9.15891815837712e-15\\
3.94399999999999	-9.15891815837517e-15\\
3.94599999999997	-9.02337027679908e-15\\
3.946	-9.02337027679718e-15\\
3.94799999999999	-8.89025095924434e-15\\
3.94999999999997	-8.75954290551334e-15\\
3.95199999999997	-8.63122912877139e-15\\
3.952	-8.63122912876959e-15\\
3.95499999999998	-8.44321133972912e-15\\
3.95500000000001	-8.44321133972736e-15\\
3.95799999999998	-8.26048824731601e-15\\
3.95999999999998	-8.14158790171556e-15\\
3.96	-8.14158790171389e-15\\
3.96299999999998	-7.96757132623824e-15\\
3.96599999999995	-7.79871036380253e-15\\
3.966	-7.79871036379977e-15\\
3.96600000000003	-7.7987103637982e-15\\
3.96700000000001	-7.74356049052123e-15\\
3.96700000000003	-7.74356049051967e-15\\
3.96800000000001	-7.68884599908964e-15\\
3.96899999999998	-7.63443493804758e-15\\
3.97099999999993	-7.52651604573469e-15\\
3.97299999999996	-7.41978978842718e-15\\
3.97299999999999	-7.41978978842567e-15\\
3.97699999999989	-7.20985985170138e-15\\
3.97899999999996	-7.10662888975801e-15\\
3.97899999999999	-7.10662888975655e-15\\
3.97999999999997	-7.05544101644666e-15\\
3.98	-7.05544101644521e-15\\
3.98099999999999	-7.00453599441383e-15\\
3.98199999999997	-6.9539121697519e-15\\
3.98399999999994	-6.85350154260989e-15\\
3.98599999999997	-6.75419608564106e-15\\
3.986	-6.75419608563966e-15\\
3.98999999999994	-6.55884920146257e-15\\
3.98999999999997	-6.55884920146085e-15\\
3.99000000000001	-6.55884920145913e-15\\
3.99399999999994	-6.3677699649093e-15\\
3.99599999999998	-6.27379958743651e-15\\
3.99600000000001	-6.27379958743518e-15\\
3.99999999999994	-6.08891953184325e-15\\
3.99999999999997	-6.08891953184177e-15\\
4	-6.0889195318403e-15\\
4.00199999999993	-5.99798164709298e-15\\
4.00199999999999	-5.99798164709041e-15\\
4.00399999999992	-5.9080293901827e-15\\
4.00599999999986	-5.81905107091003e-15\\
4.00599999999993	-5.81905107090681e-15\\
4.006	-5.81905107090355e-15\\
4.00999999999987	-5.64397011591273e-15\\
4.01199999999995	-5.55784472667584e-15\\
4.012	-5.55784472667341e-15\\
4.01599999999987	-5.38836815919313e-15\\
4.01999999999973	-5.22251731975485e-15\\
4.01999999999995	-5.22251731974614e-15\\
4.02	-5.22251731974381e-15\\
4.02500000000001	-5.02017127712649e-15\\
4.02500000000006	-5.02017127712422e-15\\
4.02599999999995	-4.9803512012435e-15\\
4.026	-4.98035120124124e-15\\
4.02699999999996	-4.94074446785103e-15\\
4.02799999999992	-4.90134979010663e-15\\
4.02999999999985	-4.82319148872011e-15\\
4.03099999999993	-4.78442532571864e-15\\
4.03099999999999	-4.78442532571644e-15\\
4.03499999999984	-4.63141794826962e-15\\
4.03699999999994	-4.55613124833273e-15\\
4.03699999999999	-4.5561312483306e-15\\
4.03799999999995	-4.51878784776469e-15\\
4.038	-4.51878784776257e-15\\
4.03899999999996	-4.48166803011022e-15\\
4.03999999999991	-4.44479582583654e-15\\
4.04	-4.44479582583321e-15\\
4.04199999999991	-4.37178947356219e-15\\
4.04399999999982	-4.29975930305465e-15\\
4.046	-4.22869595325316e-15\\
4.04600000000006	-4.22869595325116e-15\\
4.04999999999988	-4.08943289884979e-15\\
4.05399999999969	-3.95392791357355e-15\\
4.05999999999994	-3.75756324615279e-15\\
4.06	-3.75756324615097e-15\\
};
\addplot [color=mycolor2,solid,forget plot]
  table[row sep=crcr]{%
0	0.10153\\
3.15544362088405e-30	0.10153\\
0.000656101980281985	0.101530709989553\\
0.00393661188169191	0.101555560666546\\
0.00999999999999994	0.101694978093407\\
0.01	0.101694978093407\\
0.0199999999999999	0.102190448599525\\
0.02	0.102190448599525\\
0.0289999999999998	0.10292025163065\\
0.029	0.10292025163065\\
0.03	0.103018021716247\\
0.0300000000000002	0.103018021716247\\
0.0349999999999996	0.103192147327812\\
0.035	0.103192147327812\\
0.0399999999999993	0.102720100364795\\
0.04	0.102720100364795\\
0.0449999999999993	0.101601497399549\\
0.0499999999999987	0.0998354297964996\\
0.05	0.099835429796499\\
0.0500000000000004	0.0998354297964988\\
0.0579999999999996	0.0956592171978746\\
0.058	0.0956592171978743\\
0.0599999999999996	0.0943546352961654\\
0.06	0.0943546352961651\\
0.0619999999999995	0.0929455231467101\\
0.0639999999999991	0.0914316976162889\\
0.0679999999999982	0.0880891058325206\\
0.0699999999999991	0.0862599051673818\\
0.07	0.086259905167381\\
0.0779999999999982	0.0783631580954561\\
0.0799999999999991	0.0762726013656067\\
0.08	0.0762726013656058\\
0.087	0.0685830418499841\\
0.0870000000000009	0.0685830418499831\\
0.09	0.065107935293602\\
0.0900000000000009	0.065107935293601\\
0.0929999999999999	0.0615236275464586\\
0.095999999999999	0.0578290704987055\\
0.0999999999999991	0.0527296235853229\\
0.1	0.0527296235853218\\
0.104999999999999	0.0460729370657436\\
0.105	0.0460729370657424\\
0.109999999999999	0.0390974387049891\\
0.11	0.0390974387049878\\
0.114999999999999	0.0320420412716109\\
0.115999999999999	0.0306502167315683\\
0.116	0.0306502167315671\\
0.119999999999999	0.0251455926314253\\
0.12	0.0251455926314241\\
0.123999999999999	0.0197391081211328\\
0.127999999999998	0.0144279526059265\\
0.129999999999998	0.011807258483974\\
0.13	0.0118072584839717\\
0.137999999999998	0.00154912692971668\\
0.139999999999998	-0.000960911168698507\\
0.14	-0.000960911168700728\\
0.144999999999998	-0.00705216971721133\\
0.145	-0.00705216971721343\\
0.149999999999998	-0.0128321349015648\\
0.15	-0.0128321349015668\\
0.154999999999998	-0.0183055017781117\\
0.159999999999996	-0.0234767163352839\\
0.16	-0.0234767163352874\\
0.169999999999996	-0.0329292486698219\\
0.17	-0.032929248669825\\
0.173999999999998	-0.0363834968265353\\
0.174	-0.0363834968265368\\
0.174999999999998	-0.0372182447500583\\
0.175	-0.0372182447500598\\
0.176	-0.0380415218348881\\
0.177	-0.0388533548293884\\
0.179000000000001	-0.0404427936812492\\
0.179999999999998	-0.0412204511793037\\
0.18	-0.0412204511793051\\
0.184000000000001	-0.0442179202275956\\
0.188000000000002	-0.0470354872327416\\
0.189999999999998	-0.0483772688386875\\
0.19	-0.0483772688386887\\
0.198000000000002	-0.0531152697228068\\
0.199999999999998	-0.0541316071613497\\
0.2	-0.0541316071613506\\
0.202999999999998	-0.0555305499846143\\
0.203	-0.0555305499846151\\
0.205999999999998	-0.0567791771522262\\
0.208999999999996	-0.057877853791348\\
0.209999999999998	-0.0582108137877957\\
0.21	-0.0582108137877963\\
0.215999999999996	-0.0598600144100443\\
0.219999999999998	-0.0606281454731618\\
0.22	-0.0606281454731621\\
0.225999999999996	-0.0612843363763174\\
0.229999999999998	-0.0613914581691176\\
0.23	-0.0613914581691176\\
0.231999999999998	-0.0613569673420705\\
0.232	-0.0613569673420704\\
0.233999999999998	-0.0612784599866984\\
0.235999999999997	-0.061155925900476\\
0.239999999999993	-0.0607787081136675\\
0.239999999999996	-0.0607787081136671\\
0.24	-0.0607787081136667\\
0.244999999999998	-0.0600591304216388\\
0.245	-0.0600591304216385\\
0.249999999999998	-0.0590633795606718\\
0.25	-0.0590633795606714\\
0.254999999999999	-0.0577906466877398\\
0.259999999999997	-0.0562398979683205\\
0.26	-0.0562398979683193\\
0.260999999999996	-0.0558962794551987\\
0.261	-0.0558962794551974\\
0.262	-0.0555414794859283\\
0.263	-0.0551754865306404\\
0.265	-0.0544098737339776\\
0.269	-0.0527437846910428\\
0.269999999999997	-0.0522990874689763\\
0.27	-0.0522990874689747\\
0.278	-0.0486075012896723\\
0.279999999999996	-0.0476563426766285\\
0.28	-0.0476563426766268\\
0.288	-0.0437349511915884\\
0.289999999999996	-0.042724777029565\\
0.29	-0.0427247770295632\\
0.298	-0.0385608363448714\\
0.299999999999996	-0.037488363729724\\
0.3	-0.0374883637297221\\
0.308	-0.0332378303916068\\
0.309999999999996	-0.0321949371574688\\
0.31	-0.032194937157467\\
0.314999999999997	-0.029620762633946\\
0.315	-0.0296207626339442\\
0.319	-0.0275943241006294\\
0.319000000000004	-0.0275943241006276\\
0.319999999999996	-0.0270921464080909\\
0.32	-0.0270921464080891\\
0.321	-0.0265917087750128\\
0.321999999999999	-0.0260929949421357\\
0.323999999999998	-0.0251006739202311\\
0.327999999999996	-0.0231360082320983\\
0.329999999999996	-0.0221634082319868\\
0.33	-0.0221634082319851\\
0.337999999999996	-0.0183349540470543\\
0.339999999999996	-0.0173927050188045\\
0.34	-0.0173927050188028\\
0.347999999999996	-0.0137940267762798\\
0.348	-0.0137940267762782\\
0.349999999999996	-0.0129435092749043\\
0.35	-0.0129435092749028\\
0.351999999999996	-0.0121124277201381\\
0.353999999999993	-0.0113006740988322\\
0.357999999999985	-0.00973473117336946\\
0.359999999999996	-0.00898033835435596\\
0.36	-0.00898033835435464\\
0.367999999999985	-0.00615103237994316\\
0.369999999999996	-0.00549031258752466\\
0.37	-0.0054903125875235\\
0.377	-0.00332268163643367\\
0.377000000000004	-0.00332268163643262\\
0.379999999999997	-0.00246208994964999\\
0.38	-0.002462089949649\\
0.382999999999993	-0.00164216481576596\\
0.384999999999997	-0.00111802170159374\\
0.385	-0.00111802170159282\\
0.387999999999993	-0.000365347197260713\\
0.389999999999997	0.000114170794730868\\
0.39	0.000114170794731703\\
0.392999999999993	0.000793425379057254\\
0.395999999999986	0.00141940400203702\\
0.399999999999997	0.00217148258534534\\
0.4	0.00217148258534597\\
0.405999999999986	0.00312346260189153\\
0.406	0.00312346260189354\\
0.406000000000004	0.00312346260189404\\
0.41	0.00364117185530289\\
0.410000000000004	0.00364117185530331\\
0.414	0.00406564691138126\\
0.417999999999996	0.00439710843664848\\
0.419999999999997	0.0045280148603674\\
0.42	0.00452801486036761\\
0.427999999999993	0.00481984009384359\\
0.429999999999997	0.00483489369551712\\
0.43	0.00483489369551712\\
0.435	0.00480207239530616\\
0.435000000000004	0.00480207239530611\\
0.439999999999997	0.00468619349116265\\
0.44	0.00468619349116254\\
0.444999999999993	0.00448716285586529\\
0.449999999999986	0.0042048188178285\\
0.449999999999993	0.00420481881782804\\
0.45	0.00420481881782758\\
0.454999999999997	0.00383893202858688\\
0.455	0.00383893202858659\\
0.459999999999997	0.00338920528286627\\
0.46	0.00338920528286592\\
0.463999999999997	0.00296881592364295\\
0.464	0.00296881592364255\\
0.467999999999997	0.00249431796400048\\
0.469999999999997	0.00223670227654669\\
0.47	0.00223670227654622\\
0.473999999999997	0.00170503456494174\\
0.477999999999993	0.00116753023061576\\
0.479999999999997	0.000896502130446468\\
0.48	0.000896502130445985\\
0.487999999999993	-0.000203607417958384\\
0.489999999999997	-0.000482812805593269\\
0.49	-0.000482812805593767\\
0.492999999999997	-0.000904901238874969\\
0.493	-0.000904901238875472\\
0.495999999999997	-0.00133103588832417\\
0.498999999999993	-0.00176134136389649\\
0.499999999999997	-0.00190572508498463\\
0.5	-0.00190572508498514\\
0.505999999999993	-0.00275260833343492\\
0.509999999999993	-0.00329432190879493\\
0.51	-0.00329432190879587\\
0.515999999999993	-0.00407328344997044\\
0.519999999999993	-0.0045705789638816\\
0.52	-0.00457057896388246\\
0.521999999999993	-0.0048127208643742\\
0.522	-0.00481272086437505\\
0.523999999999993	-0.00505056767222024\\
0.524999999999993	-0.0051678900978277\\
0.525	-0.00516789009782853\\
0.526999999999993	-0.00539935205137182\\
0.528999999999986	-0.00562659515125581\\
0.529999999999993	-0.00573864388133106\\
0.53	-0.00573864388133186\\
0.533999999999986	-0.00617643788416967\\
0.537999999999972	-0.00659775989691432\\
0.539999999999993	-0.00680231268562629\\
0.54	-0.00680231268562701\\
0.547999999999972	-0.00755442208911304\\
0.549999999999993	-0.00772442179205999\\
0.55	-0.00772442179206058\\
0.550999999999993	-0.00780673324749164\\
0.551	-0.00780673324749222\\
0.551999999999997	-0.00788725602367983\\
0.552999999999993	-0.00796599273723251\\
0.554999999999986	-0.00811811815101259\\
0.558999999999972	-0.00840104490687427\\
0.559999999999993	-0.00846734757065569\\
0.56	-0.00846734757065616\\
0.567999999999972	-0.00893431802287783\\
0.57	-0.00903350441148642\\
0.570000000000007	-0.00903350441148676\\
0.577999999999979	-0.0093604118139374\\
0.579999999999993	-0.00942473223335562\\
0.58	-0.00942473223335584\\
0.587999999999972	-0.00961263898270645\\
0.589999999999993	-0.00964230246399086\\
0.59	-0.00964230246399095\\
0.594999999999993	-0.00968719012414359\\
0.595	-0.00968719012414363\\
0.599999999999993	-0.00969084781869305\\
0.6	-0.00969084781869302\\
0.604999999999993	-0.00965327851924945\\
0.608999999999993	-0.00959352026206663\\
0.609	-0.0095935202620665\\
0.609999999999993	-0.00957445170845045\\
0.61	-0.00957445170845031\\
0.610999999999997	-0.00955373040056255\\
0.611999999999993	-0.0095313556651646\\
0.613999999999986	-0.00948164295030244\\
0.617999999999972	-0.00936234080248655\\
0.619999999999993	-0.00929273586475684\\
0.62	-0.00929273586475658\\
0.627999999999972	-0.00894773586386479\\
0.629999999999993	-0.00884478475601719\\
0.63	-0.00884478475601681\\
0.637999999999972	-0.00839861878260049\\
0.637999999999993	-0.00839861878259923\\
0.638	-0.00839861878259882\\
0.639999999999993	-0.00828046686037032\\
0.64	-0.0082804668603699\\
0.641999999999993	-0.00815964057471369\\
0.643999999999986	-0.00803612422239629\\
0.647999999999971	-0.00778095675744499\\
0.649999999999993	-0.00764927248252216\\
0.65	-0.00764927248252169\\
0.657999999999971	-0.00709479492220257\\
0.659999999999993	-0.00694915034179599\\
0.66	-0.00694915034179547\\
0.664999999999993	-0.00658266984273853\\
0.665	-0.00658266984273801\\
0.666999999999993	-0.00643670081249966\\
0.667	-0.00643670081249914\\
0.668999999999993	-0.00629105939838003\\
0.669999999999993	-0.00621835563086651\\
0.67	-0.00621835563086599\\
0.671999999999993	-0.00607317016424685\\
0.673999999999986	-0.00592826506924562\\
0.677999999999972	-0.00563922070073368\\
0.679999999999993	-0.00549504386215189\\
0.68	-0.00549504386215137\\
0.687999999999972	-0.00492020190943284\\
0.689999999999993	-0.00477686438809373\\
0.69	-0.00477686438809322\\
0.695999999999993	-0.00434744880791587\\
0.696	-0.00434744880791536\\
0.699999999999993	-0.00406148324064083\\
0.7	-0.00406148324064033\\
0.703999999999993	-0.00378043682998459\\
0.707999999999986	-0.00350900689331351\\
0.709999999999993	-0.00337685381341326\\
0.71	-0.00337685381341279\\
0.717999999999986	-0.00287159023084518\\
0.719999999999993	-0.00275102943883562\\
0.72	-0.00275102943883519\\
0.724999999999993	-0.00245952629397335\\
0.725	-0.00245952629397294\\
0.729999999999993	-0.00218197628778351\\
0.730000000000001	-0.00218197628778313\\
0.734999999999994	-0.00191815396685727\\
0.735000000000001	-0.0019181539668569\\
0.739999999999994	-0.0016678450296479\\
0.740000000000001	-0.00166784502964756\\
0.744999999999994	-0.00143328760783166\\
0.749999999999987	-0.00121673262796016\\
0.750000000000001	-0.00121673262795958\\
0.753999999999993	-0.00105633170278198\\
0.754	-0.00105633170278171\\
0.757999999999993	-0.000907256099697013\\
0.759999999999993	-0.000836940845916116\\
0.76	-0.000836940845915871\\
0.763999999999993	-0.000704709741995797\\
0.767999999999986	-0.000583621164516536\\
0.77	-0.00052723542596818\\
0.770000000000007	-0.000527235425967985\\
0.777999999999993	-0.000329249340358925\\
0.779999999999993	-0.000286609872933135\\
0.78	-0.000286609872932989\\
0.782999999999993	-0.000227770046491974\\
0.783	-0.000227770046491842\\
0.785999999999993	-0.000175058309258881\\
0.788999999999986	-0.000128459247121661\\
0.79	-0.000114282192066528\\
0.790000000000007	-0.000114282192066429\\
0.795999999999993	-4.22451738027334e-05\\
0.8	-6.41055769183859e-06\\
0.800000000000007	-6.41055769178359e-06\\
0.804999999999993	2.47021951101581e-05\\
0.805	2.47021951101915e-05\\
0.809999999999987	4.06373850288279e-05\\
0.809999999999997	4.06373850288453e-05\\
0.810000000000007	4.06373850288627e-05\\
0.811999999999993	4.27650568151917e-05\\
0.812	4.27650568151949e-05\\
0.813999999999987	4.24668040450288e-05\\
0.815999999999973	3.97425879671313e-05\\
0.819999999999945	2.70145344093136e-05\\
0.819999999999987	2.70145344091291e-05\\
0.82	2.70145344090681e-05\\
0.827999999999944	-1.49686270216265e-05\\
0.829999999999993	-2.7600809865971e-05\\
0.830000000000001	-2.76008098660174e-05\\
0.837999999999945	-8.67243667958422e-05\\
0.839999999999993	-0.000103663567035473\\
0.84	-0.000103663567035535\\
0.840999999999993	-0.00011245835966093\\
0.841000000000001	-0.000112458359660993\\
0.841999999999997	-0.000121470325897288\\
0.842999999999993	-0.000130699758575071\\
0.844999999999986	-0.000149812229775815\\
0.848999999999972	-0.000190660437977041\\
0.849999999999993	-0.000201420928829159\\
0.85	-0.000201420928829236\\
0.857999999999972	-0.000295458030977637\\
0.859999999999993	-0.000321190591083767\\
0.86	-0.00032119059108386\\
0.867999999999972	-0.000424070723010823\\
0.869999999999993	-0.000449229809659529\\
0.87	-0.000449229809659618\\
0.874999999999994	-0.000511181751801168\\
0.875000000000001	-0.000511181751801255\\
0.879999999999994	-0.000571822715192518\\
0.880000000000001	-0.000571822715192603\\
0.884999999999994	-0.000631201958367439\\
0.889999999999987	-0.000689367714786955\\
0.890000000000001	-0.000689367714787112\\
0.890000000000008	-0.000689367714787194\\
0.899	-0.000791158298843613\\
0.899000000000008	-0.000791158298843692\\
0.9	-0.000802246811158136\\
0.900000000000007	-0.000802246811158215\\
0.901000000000004	-0.000813224446872723\\
0.902	-0.000824023665330358\\
0.903999999999993	-0.000845088248142637\\
0.907999999999979	-0.000885091457673038\\
0.909999999999993	-0.000904035283336731\\
0.910000000000001	-0.000904035283336797\\
0.917999999999972	-0.000972815107728738\\
0.919999999999993	-0.0009882723676427\\
0.920000000000001	-0.000988272367642754\\
0.927999999999972	-0.00104321089333297\\
0.927999999999994	-0.0010432108933331\\
0.928000000000001	-0.00104321089333314\\
0.929999999999994	-0.00105523182111481\\
0.930000000000001	-0.00105523182111485\\
0.931999999999994	-0.00106657052520826\\
0.933999999999987	-0.00107722847928501\\
0.937999999999972	-0.00109650758952085\\
0.939999999999993	-0.00110513125125947\\
0.940000000000001	-0.0011051312512595\\
0.944999999999994	-0.00112367898475253\\
0.945000000000001	-0.00112367898475255\\
0.949999999999994	-0.00113790357842256\\
0.950000000000001	-0.00113790357842258\\
0.954999999999994	-0.00114781658692478\\
0.956999999999994	-0.00115057646770512\\
0.957000000000001	-0.00115057646770513\\
0.959999999999993	-0.00115342606255304\\
0.960000000000001	-0.00115342606255305\\
0.962999999999993	-0.00115472809295829\\
0.965999999999986	-0.00115448293965585\\
0.969999999999993	-0.00115174914931009\\
0.970000000000001	-0.00115174914931008\\
0.975999999999986	-0.00114248734805354\\
0.979999999999993	-0.00113286738899141\\
0.980000000000001	-0.00113286738899139\\
0.985999999999986	-0.00111457699106654\\
0.985999999999993	-0.00111457699106651\\
0.986000000000001	-0.00111457699106649\\
0.989999999999993	-0.00110038722398059\\
0.990000000000001	-0.00110038722398057\\
0.993999999999993	-0.00108459203725712\\
0.997999999999986	-0.00106718321966508\\
0.999999999999993	-0.00105787090398675\\
1	-0.00105787090398672\\
1.00799999999999	-0.00101654007220815\\
1.00999999999999	-0.00100518025779439\\
1.01	-0.0010051802577943\\
1.01499999999999	-0.000974967357120618\\
1.015	-0.000974967357120529\\
1.01999999999999	-0.000942144049261689\\
1.02	-0.000942144049261592\\
1.02499999999999	-0.000907834278353474\\
1.02999999999997	-0.000873160781146008\\
1.03	-0.000873160781145812\\
1.03999999999997	-0.00080260962906665\\
1.04	-0.00080260962906645\\
1.04399999999999	-0.000773898989750929\\
1.044	-0.000773898989750827\\
1.04799999999999	-0.000744885944582557\\
1.04999999999999	-0.000730261312694193\\
1.05	-0.000730261312694089\\
1.05399999999999	-0.000700766327006555\\
1.05799999999997	-0.000670930916618006\\
1.05999999999999	-0.000655880712287418\\
1.06	-0.000655880712287311\\
1.06799999999997	-0.000596963764472701\\
1.06999999999999	-0.00058268421222812\\
1.07	-0.000582684212228019\\
1.07299999999999	-0.000561594549354705\\
1.073	-0.000561594549354606\\
1.07599999999999	-0.000540895012050007\\
1.07899999999997	-0.000520579547355215\\
1.07999999999999	-0.000513892045289247\\
1.08	-0.000513892045289152\\
1.08499999999999	-0.000481077184281581\\
1.085	-0.000481077184281489\\
1.08999999999999	-0.000449280648136233\\
1.09	-0.000449280648136142\\
1.09499999999999	-0.00041847660868036\\
1.09999999999997	-0.000388640043947459\\
1.09999999999999	-0.000388640043947374\\
1.1	-0.00038864004394729\\
1.10199999999999	-0.000377031332482515\\
1.102	-0.000377031332482434\\
1.10399999999999	-0.000365692975200122\\
1.10599999999997	-0.000354623498379099\\
1.10999999999994	-0.000333285466529387\\
1.10999999999999	-0.000333285466529159\\
1.11	-0.000333285466529085\\
1.11799999999994	-0.000293774989146232\\
1.11999999999999	-0.000284549328906529\\
1.12	-0.000284549328906464\\
1.12799999999994	-0.000250218551746881\\
1.12999999999999	-0.000242273246327815\\
1.13	-0.000242273246327759\\
1.13099999999999	-0.00023838758341131\\
1.131	-0.000238387583411255\\
1.132	-0.000234549226395464\\
1.13299999999999	-0.000230758050553369\\
1.13499999999999	-0.00022331675122269\\
1.13899999999997	-0.000208995009200804\\
1.13999999999999	-0.000205530738025934\\
1.14	-0.000205530738025885\\
1.14799999999997	-0.000179472010287227\\
1.14999999999999	-0.0001734133064334\\
1.15	-0.000173413306433358\\
1.15499999999999	-0.000159055799185892\\
1.155	-0.000159055799185852\\
1.15999999999999	-0.000145816575045398\\
1.16	-0.000145816575045362\\
1.16499999999999	-0.000133684879806955\\
1.16999999999997	-0.000122650858955459\\
1.16999999999999	-0.000122650858955429\\
1.17	-0.000122650858955399\\
1.17999999999997	-0.000103840873261597\\
1.17999999999999	-0.000103840873261573\\
1.18	-0.00010384087326155\\
1.18899999999999	-9.05175638747658e-05\\
1.189	-9.05175638747474e-05\\
1.18999999999999	-8.92419544422263e-05\\
1.19	-8.92419544422085e-05\\
1.191	-8.80071349733773e-05\\
1.19199999999999	-8.68130653382185e-05\\
1.19399999999999	-8.45470217118094e-05\\
1.19799999999997	-8.05023140385002e-05\\
1.19999999999999	-7.87231243272134e-05\\
1.2	-7.87231243272013e-05\\
1.20799999999997	-7.32223096612853e-05\\
1.20999999999999	-7.22501984179339e-05\\
1.21	-7.22501984179275e-05\\
1.21799999999997	-6.99703327205124e-05\\
1.218	-6.9970332720509e-05\\
1.21999999999999	-6.98021407393362e-05\\
1.22	-6.98021407393356e-05\\
1.22199999999999	-6.97516807254694e-05\\
1.22399999999997	-6.97760433360014e-05\\
1.22499999999999	-6.98162837349887e-05\\
1.225	-6.98162837349894e-05\\
1.22899999999997	-7.01643435534376e-05\\
1.22999999999999	-7.02981471990735e-05\\
1.23	-7.02981471990756e-05\\
1.23399999999997	-7.10206469665274e-05\\
1.23799999999994	-7.20431066353716e-05\\
1.23999999999999	-7.26669814405712e-05\\
1.24	-7.26669814405759e-05\\
1.24699999999999	-7.54433753357816e-05\\
1.247	-7.54433753357881e-05\\
1.24999999999999	-7.69163417978979e-05\\
1.25	-7.69163417979053e-05\\
1.25299999999999	-7.84719933250123e-05\\
1.25599999999997	-8.00230216534142e-05\\
1.25999999999999	-8.20846515046753e-05\\
1.26	-8.20846515046826e-05\\
1.26599999999997	-8.51655406976183e-05\\
1.26999999999999	-8.72133202396144e-05\\
1.27	-8.72133202396216e-05\\
1.27599999999997	-9.02784311661345e-05\\
1.276	-9.02784311661485e-05\\
1.27999999999999	-9.23190153542042e-05\\
1.28	-9.23190153542115e-05\\
1.28399999999999	-9.43585788242722e-05\\
1.28799999999997	-9.63981818573301e-05\\
1.28999999999999	-9.74183295443636e-05\\
1.29	-9.74183295443708e-05\\
1.295	-9.98470566919873e-05\\
1.29500000000001	-9.98470566919939e-05\\
1.3	-0.000102032873558179\\
1.30000000000001	-0.000102032873558185\\
1.305	-0.000103977555679882\\
1.30500000000001	-0.000103977555679887\\
1.31	-0.000105682682712619\\
1.31000000000001	-0.000105682682712624\\
1.315	-0.000107149639722519\\
1.31999999999998	-0.00010837961831139\\
1.32	-0.000108379618311394\\
1.32999999999997	-0.000110132444960289\\
1.33	-0.000110132444960292\\
1.33399999999999	-0.000110570603730501\\
1.334	-0.000110570603730502\\
1.33799999999999	-0.000110858878402109\\
1.33999999999999	-0.000110946859079117\\
1.34	-0.000110946859079117\\
1.34399999999999	-0.000110997591199642\\
1.34799999999997	-0.000110872715033632\\
1.34999999999999	-0.000110744407123242\\
1.35	-0.000110744407123241\\
1.35799999999997	-0.000109791587577639\\
1.35999999999999	-0.000109443330931573\\
1.36	-0.000109443330931571\\
1.36299999999999	-0.000108838237807575\\
1.363	-0.000108838237807572\\
1.36499999999999	-0.000108379627169461\\
1.365	-0.000108379627169458\\
1.36699999999999	-0.000107876775655957\\
1.36899999999997	-0.000107329617916565\\
1.36999999999999	-0.000107039402215815\\
1.37	-0.000107039402215811\\
1.37399999999997	-0.000105767402984714\\
1.37799999999995	-0.000104317076521735\\
1.37999999999999	-0.000103524808593072\\
1.38	-0.000103524808593066\\
1.38799999999995	-0.000100088894688057\\
1.38999999999999	-9.91740827071158e-05\\
1.39	-9.91740827071092e-05\\
1.39199999999999	-9.82367039155331e-05\\
1.392	-9.82367039155263e-05\\
1.39399999999998	-9.72766364972978e-05\\
1.39599999999997	-9.62937556769454e-05\\
1.39999999999994	-9.42590399084285e-05\\
1.39999999999999	-9.42590399084044e-05\\
1.4	-9.4259039908397e-05\\
1.40799999999994	-8.99100264875795e-05\\
1.41	-8.87637070946423e-05\\
1.41000000000001	-8.87637070946341e-05\\
1.41799999999995	-8.41067770143407e-05\\
1.41999999999999	-8.29344407152559e-05\\
1.42	-8.29344407152475e-05\\
1.42099999999999	-8.23469525171e-05\\
1.421	-8.23469525170917e-05\\
1.422	-8.17585588485311e-05\\
1.42299999999999	-8.11692405900797e-05\\
1.42499999999998	-7.99877536857773e-05\\
1.42899999999997	-7.76128402515078e-05\\
1.42999999999999	-7.70165119908241e-05\\
1.43	-7.70165119908156e-05\\
1.43499999999999	-7.40183021003585e-05\\
1.435	-7.40183021003499e-05\\
1.43999999999998	-7.09906885969952e-05\\
1.44	-7.0990688596985e-05\\
1.44499999999999	-6.79312121573987e-05\\
1.44999999999997	-6.48373875842296e-05\\
1.44999999999998	-6.48373875842205e-05\\
1.45	-6.48373875842114e-05\\
1.45999999999997	-5.87815967384615e-05\\
1.45999999999998	-5.87815967384523e-05\\
1.46	-5.87815967384433e-05\\
1.46999999999997	-5.3048620743912e-05\\
1.46999999999998	-5.30486207439034e-05\\
1.47	-5.30486207438947e-05\\
1.47899999999998	-4.814952962174e-05\\
1.479	-4.81495296217325e-05\\
1.47999999999999	-4.761982834079e-05\\
1.48	-4.76198283407825e-05\\
1.481	-4.70929917047267e-05\\
1.48199999999999	-4.65690025929232e-05\\
1.48399999999999	-4.55294989370645e-05\\
1.48799999999997	-4.34839189804327e-05\\
1.48999999999999	-4.24775768298489e-05\\
1.49	-4.24775768298418e-05\\
1.49799999999997	-3.86363915780705e-05\\
1.49999999999999	-3.77263642180953e-05\\
1.5	-3.77263642180889e-05\\
1.50499999999999	-3.5537962225018e-05\\
1.505	-3.5537962225012e-05\\
1.50799999999998	-3.42837822091437e-05\\
1.508	-3.42837822091378e-05\\
1.50999999999999	-3.34719593226635e-05\\
1.51	-3.34719593226578e-05\\
1.51199999999999	-3.26794506789285e-05\\
1.51399999999997	-3.19061532836352e-05\\
1.51799999999994	-3.04167927316692e-05\\
1.51999999999999	-2.97005360130005e-05\\
1.52	-2.97005360129954e-05\\
1.52799999999994	-2.70228560311519e-05\\
1.52999999999999	-2.63998377664187e-05\\
1.53	-2.63998377664143e-05\\
1.53699999999998	-2.43636258663969e-05\\
1.537	-2.4363625866393e-05\\
1.53999999999999	-2.35591378422435e-05\\
1.54	-2.35591378422398e-05\\
1.54299999999999	-2.27993316555148e-05\\
1.54599999999997	-2.20881079686948e-05\\
1.54999999999999	-2.1215025078216e-05\\
1.55	-2.1215025078213e-05\\
1.55599999999997	-2.00656715373837e-05\\
1.56	-1.94057021360547e-05\\
1.56000000000001	-1.94057021360525e-05\\
1.56599999999999	-1.85742897130511e-05\\
1.566	-1.85742897130493e-05\\
1.57	-1.81252890060872e-05\\
1.57000000000001	-1.81252890060858e-05\\
1.57400000000001	-1.77371797349349e-05\\
1.57499999999999	-1.76460411823251e-05\\
1.575	-1.76460411823238e-05\\
1.579	-1.73049531984089e-05\\
1.57999999999999	-1.72255339067306e-05\\
1.58	-1.72255339067294e-05\\
1.584	-1.69311901459937e-05\\
1.588	-1.66740678003567e-05\\
1.59	-1.65594221325629e-05\\
1.59000000000001	-1.65594221325621e-05\\
1.59499999999998	-1.63132765547669e-05\\
1.595	-1.63132765547663e-05\\
1.59999999999997	-1.61247889283009e-05\\
1.6	-1.61247889282999e-05\\
1.60000000000001	-1.61247889282995e-05\\
1.60499999999998	-1.59938061444255e-05\\
1.60999999999995	-1.59202218064847e-05\\
1.60999999999998	-1.59202218064845e-05\\
1.61	-1.59202218064843e-05\\
1.61999999999994	-1.58514896986698e-05\\
1.62	-1.58514896986695e-05\\
1.62000000000001	-1.58514896986694e-05\\
1.62399999999998	-1.58357774762215e-05\\
1.624	-1.58357774762215e-05\\
1.62799999999997	-1.5826782818672e-05\\
1.63000000000001	-1.58248029789593e-05\\
1.63000000000003	-1.58248029789593e-05\\
1.634	-1.58258769911354e-05\\
1.63799999999997	-1.58336634208179e-05\\
1.63999999999999	-1.58400749208184e-05\\
1.64	-1.58400749208185e-05\\
1.64499999999999	-1.58634555715512e-05\\
1.645	-1.58634555715513e-05\\
1.64999999999998	-1.58973551551237e-05\\
1.65	-1.58973551551238e-05\\
1.653	-1.59183225698975e-05\\
1.65300000000001	-1.59183225698976e-05\\
1.65600000000001	-1.59342276358198e-05\\
1.65900000000001	-1.59450750037646e-05\\
1.66	-1.59475674510204e-05\\
1.66000000000002	-1.59475674510204e-05\\
1.66600000000001	-1.59507317536535e-05\\
1.67	-1.5941612607951e-05\\
1.67000000000002	-1.59416126079509e-05\\
1.67600000000001	-1.59110790160194e-05\\
1.67999999999998	-1.58794712736365e-05\\
1.68	-1.58794712736364e-05\\
1.68199999999998	-1.58602860753517e-05\\
1.682	-1.58602860753515e-05\\
1.68399999999998	-1.58388434323182e-05\\
1.68599999999997	-1.58151405578342e-05\\
1.68999999999994	-1.57609414986901e-05\\
1.68999999999998	-1.57609414986894e-05\\
1.69	-1.57609414986892e-05\\
1.69799999999994	-1.56191080152225e-05\\
1.69999999999998	-1.55760163358303e-05\\
1.7	-1.557601633583e-05\\
1.70799999999994	-1.53729483603715e-05\\
1.70999999999998	-1.53144730636587e-05\\
1.71	-1.53144730636583e-05\\
1.711	-1.52840741867392e-05\\
1.71100000000001	-1.52840741867388e-05\\
1.71200000000001	-1.52528998490715e-05\\
1.71300000000001	-1.52209490376927e-05\\
1.715	-1.51547138162521e-05\\
1.71500000000001	-1.51547138162516e-05\\
1.71700000000001	-1.50853599144988e-05\\
1.71900000000001	-1.50128783192147e-05\\
1.71999999999998	-1.4975461708742e-05\\
1.72	-1.49754617087415e-05\\
1.724	-1.4817927851412e-05\\
1.72799999999999	-1.4647743787288e-05\\
1.72999999999998	-1.45578805380425e-05\\
1.73	-1.45578805380418e-05\\
1.73799999999999	-1.41768387335477e-05\\
1.74	-1.4076775115053e-05\\
1.74000000000001	-1.40767751150523e-05\\
1.74800000000001	-1.36569178031095e-05\\
1.74999999999998	-1.35469845595881e-05\\
1.75	-1.35469845595873e-05\\
1.75799999999999	-1.30869472712148e-05\\
1.75999999999998	-1.29667871372401e-05\\
1.76	-1.29667871372393e-05\\
1.76799999999999	-1.24650748249715e-05\\
1.76899999999998	-1.23999566623436e-05\\
1.769	-1.23999566623427e-05\\
1.76999999999998	-1.23342972995524e-05\\
1.77	-1.23342972995515e-05\\
1.771	-1.22683512096646e-05\\
1.77199999999999	-1.22023728558398e-05\\
1.77399999999998	-1.20703107807947e-05\\
1.77799999999997	-1.18057050655396e-05\\
1.77999999999998	-1.16731270363845e-05\\
1.78	-1.16731270363836e-05\\
1.78499999999998	-1.1340741155306e-05\\
1.785	-1.13407411553051e-05\\
1.78999999999998	-1.10067951312909e-05\\
1.79	-1.10067951312899e-05\\
1.79499999999998	-1.06710177007494e-05\\
1.798	-1.04685587715748e-05\\
1.79800000000001	-1.04685587715738e-05\\
1.8	-1.03331361131471e-05\\
1.80000000000001	-1.03331361131462e-05\\
1.802	-1.01973328978557e-05\\
1.80399999999999	-1.00611314766248e-05\\
1.80799999999996	-9.78746315944903e-06\\
1.81	-9.64996069675765e-06\\
1.81000000000001	-9.64996069675667e-06\\
1.81799999999996	-9.10903401394586e-06\\
1.81999999999998	-8.97685738295142e-06\\
1.82	-8.97685738295049e-06\\
1.82699999999998	-8.52347880551274e-06\\
1.827	-8.52347880551183e-06\\
1.82999999999998	-8.33344740472986e-06\\
1.83	-8.33344740472897e-06\\
1.83299999999999	-8.14589960466557e-06\\
1.83599999999997	-7.96078056317618e-06\\
1.84	-7.71763978325926e-06\\
1.84000000000001	-7.7176397832584e-06\\
1.84599999999999	-7.36056827567835e-06\\
1.85	-7.1274332418698e-06\\
1.85000000000001	-7.12743324186898e-06\\
1.85499999999998	-6.84424231197226e-06\\
1.855	-6.84424231197148e-06\\
1.85599999999998	-6.78899352192175e-06\\
1.856	-6.78899352192097e-06\\
1.85699999999999	-6.73420364053495e-06\\
1.85799999999999	-6.6798708876544e-06\\
1.85999999999998	-6.5725697211876e-06\\
1.86	-6.57256972118665e-06\\
1.86399999999999	-6.36337624720241e-06\\
1.86799999999997	-6.16130434945915e-06\\
1.86999999999999	-6.06290602030551e-06\\
1.87	-6.06290602030481e-06\\
1.87799999999997	-5.68660108448496e-06\\
1.87999999999999	-5.59678581457653e-06\\
1.88	-5.5967858145759e-06\\
1.88499999999998	-5.37829657021964e-06\\
1.885	-5.37829657021903e-06\\
1.88999999999998	-5.16757652175505e-06\\
1.89	-5.1675765217543e-06\\
1.89499999999998	-4.96445450166978e-06\\
1.89999999999996	-4.7687655149676e-06\\
1.89999999999998	-4.76876551496693e-06\\
1.9	-4.76876551496626e-06\\
1.90999999999996	-4.39905672176424e-06\\
1.91	-4.39905672176301e-06\\
1.91000000000001	-4.39905672176251e-06\\
1.91399999999998	-4.25904960060962e-06\\
1.914	-4.25904960060913e-06\\
1.91799999999997	-4.12343279803423e-06\\
1.91999999999999	-4.05724864559808e-06\\
1.92	-4.05724864559761e-06\\
1.92399999999997	-3.92808583099458e-06\\
1.92499999999998	-3.89645680534762e-06\\
1.925	-3.89645680534717e-06\\
1.92899999999997	-3.77255658546691e-06\\
1.92999999999999	-3.74223046931385e-06\\
1.93	-3.74223046931342e-06\\
1.93399999999997	-3.62446812902034e-06\\
1.93799999999994	-3.51271807084249e-06\\
1.93999999999999	-3.45907937155821e-06\\
1.94	-3.45907937155783e-06\\
1.94299999999998	-3.38139706159848e-06\\
1.943	-3.38139706159812e-06\\
1.94599999999998	-3.30702551950355e-06\\
1.94899999999996	-3.23594299749251e-06\\
1.95	-3.21297609665566e-06\\
1.95000000000002	-3.21297609665534e-06\\
1.95599999999998	-3.08275944379171e-06\\
1.95999999999998	-3.00312084902371e-06\\
1.96	-3.00312084902343e-06\\
1.96599999999996	-2.89431821155743e-06\\
1.97	-2.8288316322305e-06\\
1.97000000000001	-2.82883163223028e-06\\
1.97199999999998	-2.79798511300451e-06\\
1.972	-2.7979851130043e-06\\
1.97399999999997	-2.76812472003519e-06\\
1.97599999999994	-2.73924657260328e-06\\
1.97999999999988	-2.68442212952212e-06\\
1.98	-2.68442212952055e-06\\
1.98000000000002	-2.68442212952036e-06\\
1.9879999999999	-2.58640355199744e-06\\
1.99	-2.56430312836444e-06\\
1.99000000000002	-2.56430312836429e-06\\
1.99499999999998	-2.51322752878369e-06\\
1.995	-2.51322752878356e-06\\
1.99999999999997	-2.46808426134855e-06\\
1.99999999999998	-2.46808426134841e-06\\
2	-2.46808426134826e-06\\
2.00099999999997	-2.45976417629559e-06\\
2.001	-2.45976417629535e-06\\
2.00199999999999	-2.45167963210308e-06\\
2.00299999999999	-2.44383036609841e-06\\
2.00499999999998	-2.4288366562069e-06\\
2.00899999999997	-2.40166186494384e-06\\
2.00999999999997	-2.39545283270838e-06\\
2.01	-2.39545283270821e-06\\
2.01799999999997	-2.34953658590569e-06\\
2.01999999999997	-2.33893321265291e-06\\
2.02	-2.33893321265276e-06\\
2.02799999999997	-2.29998112885018e-06\\
2.02999999999997	-2.29110213238267e-06\\
2.03	-2.29110213238255e-06\\
2.03799999999997	-2.25898762354966e-06\\
2.03999999999997	-2.25180414855175e-06\\
2.04	-2.25180414855165e-06\\
2.04799999999997	-2.22280698404066e-06\\
2.04999999999997	-2.21526121139377e-06\\
2.05	-2.21526121139367e-06\\
2.05799999999997	-2.18386277222777e-06\\
2.05899999999997	-2.17979914070355e-06\\
2.059	-2.17979914070344e-06\\
2.05999999999997	-2.17570422460866e-06\\
2.06	-2.17570422460854e-06\\
2.06099999999999	-2.17157789091912e-06\\
2.06199999999999	-2.16742000555347e-06\\
2.06399999999998	-2.15900903841038e-06\\
2.06499999999997	-2.15475568336164e-06\\
2.065	-2.15475568336151e-06\\
2.06899999999998	-2.13741988335591e-06\\
2.06999999999997	-2.1330046343575e-06\\
2.07	-2.13300463435737e-06\\
2.07399999999998	-2.11501413876106e-06\\
2.07799999999997	-2.09648939082995e-06\\
2.07999999999997	-2.08702367372818e-06\\
2.08	-2.08702367372804e-06\\
2.08799999999997	-2.04719165981228e-06\\
2.088	-2.04719165981213e-06\\
2.08999999999997	-2.03669836691592e-06\\
2.09	-2.03669836691577e-06\\
2.09199999999997	-2.02598827281814e-06\\
2.09399999999994	-2.01505998557698e-06\\
2.09799999999988	-1.99254312217067e-06\\
2.09999999999997	-1.98095161964919e-06\\
2.1	-1.98095161964903e-06\\
2.10799999999988	-1.9323296465967e-06\\
2.10999999999997	-1.9196022639249e-06\\
2.11	-1.91960226392471e-06\\
2.11699999999997	-1.87321814744028e-06\\
2.117	-1.87321814744009e-06\\
2.11999999999997	-1.8524509241818e-06\\
2.12	-1.8524509241816e-06\\
2.12299999999997	-1.83114201509627e-06\\
2.12599999999994	-1.80928518911093e-06\\
2.12999999999997	-1.77927936938337e-06\\
2.13	-1.77927936938315e-06\\
2.13499999999997	-1.74101317886056e-06\\
2.135	-1.74101317886034e-06\\
2.13999999999997	-1.70245310742473e-06\\
2.14	-1.70245310742451e-06\\
2.14499999999998	-1.66356783291681e-06\\
2.14599999999997	-1.65574897898146e-06\\
2.146	-1.65574897898124e-06\\
2.14999999999997	-1.62432576901268e-06\\
2.15	-1.62432576901246e-06\\
2.15399999999998	-1.59265381917804e-06\\
2.15799999999995	-1.56071666461704e-06\\
2.15999999999997	-1.5446434527329e-06\\
2.16	-1.54464345273267e-06\\
2.16799999999995	-1.47960418938081e-06\\
2.16999999999997	-1.46314720391144e-06\\
2.17	-1.4631472039112e-06\\
2.17499999999997	-1.42219943870778e-06\\
2.175	-1.42219943870754e-06\\
2.17999999999997	-1.381825013614e-06\\
2.18	-1.38182501361375e-06\\
2.18499999999997	-1.34199113256839e-06\\
2.18999999999994	-1.30266543870963e-06\\
2.18999999999997	-1.30266543870938e-06\\
2.19	-1.30266543870916e-06\\
2.19999999999994	-1.22541122315258e-06\\
2.19999999999997	-1.22541122315234e-06\\
2.2	-1.22541122315212e-06\\
2.20399999999997	-1.19504986794073e-06\\
2.204	-1.19504986794052e-06\\
2.20499999999997	-1.18751879150601e-06\\
2.205	-1.18751879150579e-06\\
2.206	-1.18001092946554e-06\\
2.20699999999999	-1.17252603788858e-06\\
2.20899999999998	-1.15762419413582e-06\\
2.20999999999997	-1.15020675780087e-06\\
2.21	-1.15020675780066e-06\\
2.21399999999998	-1.12075463459705e-06\\
2.21799999999997	-1.09163924898507e-06\\
2.21999999999997	-1.07720309611922e-06\\
2.22	-1.07720309611902e-06\\
2.22799999999997	-1.02022509258566e-06\\
2.22999999999997	-1.0061629875567e-06\\
2.23	-1.0061629875565e-06\\
2.23299999999997	-9.85199178934939e-07\\
2.233	-9.85199178934741e-07\\
2.23599999999997	-9.64385143498137e-07\\
2.23899999999994	-9.43714794812915e-07\\
2.23999999999997	-9.36855563092864e-07\\
2.24	-9.36855563092669e-07\\
2.24599999999994	-8.96473231639888e-07\\
2.24999999999997	-8.70345658314884e-07\\
2.25	-8.703456583147e-07\\
2.25599999999994	-8.32311311861706e-07\\
2.25999999999997	-8.07707200013516e-07\\
2.26	-8.07707200013344e-07\\
2.26199999999997	-7.95625954343191e-07\\
2.262	-7.9562595434302e-07\\
2.26399999999997	-7.83689815754667e-07\\
2.26599999999994	-7.71897233004876e-07\\
2.26999999999988	-7.48736623231207e-07\\
2.26999999999997	-7.48736623230689e-07\\
2.27	-7.48736623230527e-07\\
2.27499999999997	-7.20566452961005e-07\\
2.275	-7.20566452960847e-07\\
2.27999999999997	-6.93242282991744e-07\\
2.28	-6.93242282991591e-07\\
2.28499999999997	-6.66741917951098e-07\\
2.28999999999995	-6.41043831725739e-07\\
2.29	-6.41043831725469e-07\\
2.29099999999997	-6.3601341251552e-07\\
2.291	-6.36013412515378e-07\\
2.29199999999999	-6.31043629784599e-07\\
2.29299999999999	-6.26134322011326e-07\\
2.29499999999998	-6.16496495289422e-07\\
2.29899999999997	-5.9793965426352e-07\\
2.29999999999997	-5.93449314501783e-07\\
2.3	-5.93449314501656e-07\\
2.30799999999997	-5.59647872049745e-07\\
2.30999999999997	-5.51781738698148e-07\\
2.31	-5.51781738698038e-07\\
2.31799999999997	-5.22623442770512e-07\\
2.31999999999997	-5.15905691355242e-07\\
2.32	-5.15905691355148e-07\\
2.32799999999997	-4.91295782162891e-07\\
2.32999999999997	-4.85704581068361e-07\\
2.33	-4.85704581068284e-07\\
2.33799999999997	-4.65115475172862e-07\\
2.33999999999997	-4.60380807811207e-07\\
2.34	-4.60380807811141e-07\\
2.34499999999997	-4.49259387376462e-07\\
2.345	-4.49259387376402e-07\\
2.34899999999997	-4.41093208630454e-07\\
2.349	-4.41093208630399e-07\\
2.34999999999997	-4.39152622110099e-07\\
2.35	-4.39152622110044e-07\\
2.35099999999999	-4.3725229079296e-07\\
2.35199999999999	-4.35392152937125e-07\\
2.35399999999998	-4.31792217171923e-07\\
2.35799999999996	-4.25072095278971e-07\\
2.35999999999997	-4.21951035782077e-07\\
2.36	-4.21951035782034e-07\\
2.36799999999996	-4.11050917039958e-07\\
2.36999999999997	-4.08720146422948e-07\\
2.37	-4.08720146422916e-07\\
2.37799999999996	-4.00221484343344e-07\\
2.378	-4.00221484343311e-07\\
2.37999999999997	-3.98255080360972e-07\\
2.38	-3.98255080360945e-07\\
2.38199999999997	-3.9635145759756e-07\\
2.38399999999994	-3.9451036864441e-07\\
2.38799999999988	-3.91014843194353e-07\\
2.38999999999997	-3.89359952407652e-07\\
2.39	-3.89359952407629e-07\\
2.39799999999988	-3.83354610545822e-07\\
2.39999999999997	-3.82005854825011e-07\\
2.4	-3.82005854824992e-07\\
2.40699999999997	-3.77293191791246e-07\\
2.407	-3.77293191791227e-07\\
2.40999999999997	-3.75212203199024e-07\\
2.41	-3.75212203199004e-07\\
2.41299999999997	-3.73093575348541e-07\\
2.41499999999997	-3.7165994060385e-07\\
2.415	-3.7165994060383e-07\\
2.41799999999997	-3.69477198427921e-07\\
2.41999999999997	-3.68000234482032e-07\\
2.42	-3.68000234482011e-07\\
2.42299999999997	-3.65751604950144e-07\\
2.42599999999994	-3.6346256974696e-07\\
2.42999999999997	-3.60346510922243e-07\\
2.42999999999999	-3.60346510922221e-07\\
2.43599999999994	-3.55531949249599e-07\\
2.43599999999997	-3.55531949249574e-07\\
2.436	-3.5553194924955e-07\\
2.43999999999997	-3.52226159146334e-07\\
2.43999999999999	-3.5222615914631e-07\\
2.44399999999996	-3.488156770853e-07\\
2.44799999999993	-3.45272903096135e-07\\
2.44999999999999	-3.43451333788528e-07\\
2.45000000000002	-3.43451333788502e-07\\
2.45799999999996	-3.35824932708649e-07\\
2.45999999999997	-3.33832062563194e-07\\
2.46	-3.33832062563166e-07\\
2.46499999999997	-3.2869610056866e-07\\
2.465	-3.2869610056863e-07\\
2.46999999999996	-3.23337084380885e-07\\
2.47	-3.23337084380842e-07\\
2.47499999999997	-3.17750660885565e-07\\
2.47999999999993	-3.11932292258999e-07\\
2.48	-3.1193229225892e-07\\
2.48000000000003	-3.11932292258886e-07\\
2.48499999999997	-3.05877252209088e-07\\
2.485	-3.05877252209053e-07\\
2.48999999999994	-2.99580622375579e-07\\
2.48999999999999	-2.99580622375524e-07\\
2.49000000000003	-2.99580622375469e-07\\
2.49399999999997	-2.9443399023238e-07\\
2.494	-2.94433990232344e-07\\
2.49799999999993	-2.89262889116143e-07\\
2.49999999999997	-2.86667323155703e-07\\
2.5	-2.86667323155666e-07\\
2.50399999999994	-2.81454473793265e-07\\
2.50799999999988	-2.76210407976078e-07\\
2.50999999999997	-2.73575817627966e-07\\
2.51	-2.73575817627928e-07\\
2.51799999999988	-2.62945737228555e-07\\
2.51999999999997	-2.60263560511936e-07\\
2.52	-2.60263560511898e-07\\
2.52299999999997	-2.56220390209037e-07\\
2.523	-2.56220390208999e-07\\
2.52599999999996	-2.52152282219269e-07\\
2.52899999999993	-2.4805804694937e-07\\
2.52999999999997	-2.46687289096333e-07\\
2.53	-2.46687289096294e-07\\
2.53599999999993	-2.38525955167872e-07\\
2.53999999999997	-2.33162583950013e-07\\
2.54	-2.33162583949975e-07\\
2.54599999999993	-2.25226833466476e-07\\
2.54999999999997	-2.20005193170384e-07\\
2.55	-2.20005193170348e-07\\
2.55199999999997	-2.1741401047391e-07\\
2.552	-2.17414010473874e-07\\
2.55399999999996	-2.14835469857485e-07\\
2.55499999999997	-2.1355083554349e-07\\
2.555	-2.13550835543453e-07\\
2.55699999999997	-2.10990630215881e-07\\
2.55899999999994	-2.08442232181887e-07\\
2.55999999999997	-2.07172357343986e-07\\
2.56	-2.0717235734395e-07\\
2.56399999999994	-2.0212966975525e-07\\
2.56799999999987	-1.97147497490154e-07\\
2.56999999999997	-1.94678294016891e-07\\
2.57	-1.94678294016856e-07\\
2.57799999999987	-1.84939898172047e-07\\
2.57999999999997	-1.82538321711359e-07\\
2.58	-1.82538321711325e-07\\
2.58099999999997	-1.81342271269878e-07\\
2.581	-1.81342271269844e-07\\
2.58199999999999	-1.80149327512177e-07\\
2.58299999999999	-1.78959451674175e-07\\
2.58499999999998	-1.76588749220809e-07\\
2.58899999999996	-1.7188246489155e-07\\
2.58999999999997	-1.707129874661e-07\\
2.59	-1.70712987466067e-07\\
2.59799999999997	-1.6145340307114e-07\\
2.59999999999997	-1.59163860832581e-07\\
2.6	-1.59163860832548e-07\\
2.60799999999997	-1.50234135365755e-07\\
2.60999999999997	-1.48065860062525e-07\\
2.61	-1.48065860062494e-07\\
2.61799999999996	-1.39640928571683e-07\\
2.62	-1.37595369490926e-07\\
2.62000000000002	-1.37595369490897e-07\\
2.62499999999997	-1.32584730360557e-07\\
2.625	-1.32584730360529e-07\\
2.62999999999995	-1.27718361699426e-07\\
2.62999999999999	-1.27718361699391e-07\\
2.63000000000002	-1.27718361699355e-07\\
2.63499999999997	-1.22992310565466e-07\\
2.63899999999997	-1.19309914755031e-07\\
2.639	-1.19309914755006e-07\\
2.63999999999997	-1.18402738009798e-07\\
2.64	-1.18402738009772e-07\\
2.641	-1.1750204292381e-07\\
2.64199999999999	-1.16608972216669e-07\\
2.64399999999999	-1.14845588123274e-07\\
2.64799999999997	-1.11409052277999e-07\\
2.64999999999997	-1.09735453903298e-07\\
2.65	-1.09735453903274e-07\\
2.65799999999997	-1.03333826950149e-07\\
2.65999999999997	-1.01805571896947e-07\\
2.66	-1.01805571896926e-07\\
2.66799999999997	-9.59751994601071e-08\\
2.668	-9.59751994600872e-08\\
2.66999999999997	-9.45873211443869e-08\\
2.67	-9.45873211443673e-08\\
2.67199999999998	-9.32269626084326e-08\\
2.67399999999995	-9.18939470507382e-08\\
2.6779999999999	-8.93092554472713e-08\\
2.67999999999997	-8.80572434868788e-08\\
2.68	-8.80572434868612e-08\\
2.6879999999999	-8.33143316628194e-08\\
2.68999999999997	-8.21941172195605e-08\\
2.69	-8.21941172195448e-08\\
2.69499999999997	-7.95280746204579e-08\\
2.695	-7.95280746204433e-08\\
2.69699999999997	-7.85185231681594e-08\\
2.697	-7.85185231681452e-08\\
2.69899999999996	-7.75412671635159e-08\\
2.69999999999997	-7.70647101014357e-08\\
2.7	-7.70647101014222e-08\\
2.70199999999997	-7.61356604413649e-08\\
2.70399999999993	-7.52385965515971e-08\\
2.70799999999986	-7.35399638956518e-08\\
2.70999999999997	-7.27381743696496e-08\\
2.71	-7.27381743696385e-08\\
2.71799999999986	-6.98454780375708e-08\\
2.71999999999997	-6.92004494746398e-08\\
2.72	-6.92004494746308e-08\\
2.72599999999997	-6.73714794887525e-08\\
2.726	-6.73714794887441e-08\\
2.72999999999997	-6.62175664530617e-08\\
2.73	-6.62175664530537e-08\\
2.73399999999998	-6.51152698473385e-08\\
2.73799999999995	-6.40640166350299e-08\\
2.73999999999997	-6.35573594878516e-08\\
2.74	-6.35573594878445e-08\\
2.74799999999995	-6.16556765740164e-08\\
2.74999999999997	-6.12111833335601e-08\\
2.75	-6.12111833335539e-08\\
2.75499999999997	-6.01534349253385e-08\\
2.755	-6.01534349253327e-08\\
2.75999999999996	-5.91714132927367e-08\\
2.76	-5.91714132927298e-08\\
2.76499999999997	-5.82377494446059e-08\\
2.765	-5.82377494445996e-08\\
2.76500000000003	-5.82377494445944e-08\\
2.77	-5.73251136742009e-08\\
2.77000000000003	-5.73251136741958e-08\\
2.77499999999999	-5.64327646520784e-08\\
2.77999999999996	-5.55599775276065e-08\\
2.78	-5.55599775275993e-08\\
2.78399999999997	-5.4875355443675e-08\\
2.784	-5.48753554436702e-08\\
2.78799999999996	-5.42024430506464e-08\\
2.78999999999997	-5.38702684335045e-08\\
2.79	-5.38702684334998e-08\\
2.79399999999997	-5.32142665060041e-08\\
2.79799999999993	-5.25691107319846e-08\\
2.79999999999997	-5.22504951261717e-08\\
2.8	-5.22504951261672e-08\\
2.80799999999993	-5.09773019154831e-08\\
2.80999999999997	-5.06576028475708e-08\\
2.81	-5.06576028475663e-08\\
2.81299999999997	-5.0176835075781e-08\\
2.813	-5.01768350757765e-08\\
2.81599999999996	-4.96944793078516e-08\\
2.81899999999993	-4.92103944936115e-08\\
2.81999999999997	-4.90486242054751e-08\\
2.82	-4.90486242054705e-08\\
2.82599999999993	-4.80733430186001e-08\\
2.82999999999997	-4.74183302809284e-08\\
2.83	-4.74183302809238e-08\\
2.83499999999997	-4.65896409051091e-08\\
2.835	-4.65896409051044e-08\\
2.83999999999997	-4.57458330282257e-08\\
2.84	-4.57458330282209e-08\\
2.84199999999997	-4.54039235878529e-08\\
2.842	-4.5403923587848e-08\\
2.84399999999996	-4.50594412295124e-08\\
2.84599999999993	-4.47123411840584e-08\\
2.84999999999985	-4.40101072495176e-08\\
2.85	-4.40101072494915e-08\\
2.85000000000003	-4.40101072494864e-08\\
2.85799999999989	-4.25722188186294e-08\\
2.85999999999997	-4.22055121122642e-08\\
2.86	-4.22055121122589e-08\\
2.86799999999986	-4.07083170290449e-08\\
2.86999999999997	-4.0326182973369e-08\\
2.87	-4.03261829733635e-08\\
2.87099999999997	-4.01339067583652e-08\\
2.871	-4.01339067583597e-08\\
2.87199999999999	-3.9940816098699e-08\\
2.87299999999999	-3.97469047200408e-08\\
2.87499999999998	-3.93565945783815e-08\\
2.87899999999996	-3.85658466500053e-08\\
2.87999999999997	-3.83660123149367e-08\\
2.88	-3.8366012314931e-08\\
2.88799999999997	-3.67790960863404e-08\\
2.88999999999997	-3.63869141146096e-08\\
2.89	-3.63869141146041e-08\\
2.89799999999997	-3.48348453740432e-08\\
2.89999999999997	-3.44507408405502e-08\\
2.9	-3.44507408405447e-08\\
2.90499999999997	-3.34967798331249e-08\\
2.905	-3.34967798331195e-08\\
2.90999999999998	-3.25512002396096e-08\\
2.91000000000001	-3.25512002396042e-08\\
2.91499999999999	-3.16132339677211e-08\\
2.91999999999997	-3.06821191121712e-08\\
2.92	-3.06821191121648e-08\\
2.92899999999997	-2.90517188436672e-08\\
2.92899999999999	-2.90517188436622e-08\\
2.92999999999997	-2.88753905796682e-08\\
2.93	-2.88753905796632e-08\\
2.93099999999999	-2.87000065474288e-08\\
2.93199999999999	-2.85255610474922e-08\\
2.93399999999997	-2.81794630041017e-08\\
2.93799999999994	-2.74982819474212e-08\\
2.94	-2.71631104057552e-08\\
2.94000000000003	-2.71631104057504e-08\\
2.94799999999997	-2.58575398048434e-08\\
2.95	-2.55397139548099e-08\\
2.95000000000003	-2.55397139548054e-08\\
2.95799999999997	-2.4301438489922e-08\\
2.958	-2.43014384899177e-08\\
2.96	-2.39999254484358e-08\\
2.96000000000003	-2.39999254484316e-08\\
2.96200000000003	-2.37021737876158e-08\\
2.96400000000003	-2.3408762850329e-08\\
2.96800000000003	-2.28348111789922e-08\\
2.97	-2.25541958524246e-08\\
2.97000000000003	-2.25541958524206e-08\\
2.97499999999997	-2.18709170576436e-08\\
2.975	-2.18709170576398e-08\\
2.97999999999995	-2.12132818089244e-08\\
2.98	-2.12132818089172e-08\\
2.98499999999995	-2.0580755910329e-08\\
2.98699999999997	-2.03346604023567e-08\\
2.98699999999999	-2.03346604023532e-08\\
2.98999999999997	-1.99728255638423e-08\\
2.99	-1.99728255638389e-08\\
2.99299999999998	-1.96180101815447e-08\\
2.99599999999996	-1.92684537102025e-08\\
2.99999999999997	-1.88103824926706e-08\\
3	-1.88103824926674e-08\\
3.00599999999996	-1.81399453398524e-08\\
3.00999999999997	-1.77037615027729e-08\\
3.01	-1.77037615027698e-08\\
3.01599999999996	-1.70650802374638e-08\\
3.01599999999999	-1.70650802374601e-08\\
3.01600000000002	-1.70650802374572e-08\\
3.01999999999997	-1.66493662532682e-08\\
3.02	-1.66493662532652e-08\\
3.02399999999995	-1.62414583538886e-08\\
3.0279999999999	-1.58411444855966e-08\\
3.02999999999997	-1.56437701275624e-08\\
3.03	-1.56437701275597e-08\\
3.0379999999999	-1.48763090864314e-08\\
3.03999999999997	-1.4690083311209e-08\\
3.04	-1.46900833112064e-08\\
3.04499999999997	-1.4234122061906e-08\\
3.04499999999999	-1.42341220619035e-08\\
3.04999999999996	-1.37915846853361e-08\\
3.05	-1.37915846853325e-08\\
3.05499999999997	-1.33621117092238e-08\\
3.05999999999993	-1.29453542748313e-08\\
3.05999999999997	-1.29453542748284e-08\\
3.06	-1.29453542748256e-08\\
3.06999999999993	-1.21486419652344e-08\\
3.06999999999997	-1.21486419652317e-08\\
3.07	-1.2148641965229e-08\\
3.07399999999997	-1.18432369152687e-08\\
3.07399999999999	-1.18432369152666e-08\\
3.07799999999996	-1.15451800220024e-08\\
3.07999999999997	-1.13988585644977e-08\\
3.08	-1.13988585644956e-08\\
3.08399999999997	-1.11141744691514e-08\\
3.08799999999993	-1.08417393647674e-08\\
3.08999999999997	-1.07100706781213e-08\\
3.09	-1.07100706781195e-08\\
3.09799999999993	-1.02133269829048e-08\\
3.09999999999997	-1.00965431253345e-08\\
3.1	-1.00965431253328e-08\\
3.10299999999997	-9.92685625741634e-09\\
3.10299999999999	-9.92685625741476e-09\\
3.10599999999996	-9.76371212958586e-09\\
3.10899999999993	-9.60706303513866e-09\\
3.10999999999997	-9.55628203938517e-09\\
3.11	-9.55628203938374e-09\\
3.11499999999997	-9.31306860683698e-09\\
3.115	-9.31306860683565e-09\\
3.11999999999998	-9.08753165983052e-09\\
3.12000000000001	-9.08753165982928e-09\\
3.12499999999998	-8.87708916757306e-09\\
3.12999999999996	-8.6791713604741e-09\\
3.13000000000001	-8.67917136047216e-09\\
3.13199999999999	-8.60347479749268e-09\\
3.13200000000002	-8.60347479749161e-09\\
3.13400000000001	-8.52974651258964e-09\\
3.136	-8.45797692279424e-09\\
3.13999999999997	-8.32027677321498e-09\\
3.14	-8.32027677321402e-09\\
3.14000000000003	-8.32027677321307e-09\\
3.14799999999998	-8.06798732886381e-09\\
3.14999999999998	-8.00968155016023e-09\\
3.15	-8.00968155015942e-09\\
3.15799999999995	-7.79529741947815e-09\\
3.15999999999997	-7.74637630584818e-09\\
3.16	-7.7463763058475e-09\\
3.16099999999997	-7.72247657684281e-09\\
3.16099999999999	-7.72247657684213e-09\\
3.16199999999999	-7.69876850310049e-09\\
3.16299999999998	-7.675251314235e-09\\
3.16499999999997	-7.62878654104499e-09\\
3.16899999999994	-7.53811440675289e-09\\
3.16999999999997	-7.51591236225511e-09\\
3.17	-7.51591236225448e-09\\
3.17799999999994	-7.34489465260164e-09\\
3.17999999999997	-7.30394777011225e-09\\
3.18	-7.30394777011168e-09\\
3.185	-7.20468517884171e-09\\
3.18500000000003	-7.20468517884115e-09\\
3.18999999999997	-7.10979367848363e-09\\
3.18999999999999	-7.10979367848311e-09\\
3.19499999999993	-7.01631471007992e-09\\
3.19999999999986	-6.92129086238315e-09\\
3.19999999999999	-6.92129086238067e-09\\
3.20000000000002	-6.92129086238013e-09\\
3.20999999999989	-6.72629846191441e-09\\
3.21000000000002	-6.72629846191185e-09\\
3.21000000000005	-6.72629846191128e-09\\
3.21899999999997	-6.54473342740628e-09\\
3.21899999999999	-6.5447334274057e-09\\
3.22	-6.52418278282108e-09\\
3.22000000000003	-6.52418278282049e-09\\
3.22100000000004	-6.50355435800728e-09\\
3.22200000000005	-6.48284748261372e-09\\
3.22400000000006	-6.44119568647521e-09\\
3.22800000000009	-6.35692091233e-09\\
3.23000000000003	-6.3142869817289e-09\\
3.23000000000006	-6.31428698172829e-09\\
3.23800000000012	-6.14065368213734e-09\\
3.23999999999997	-6.09646415968225e-09\\
3.24	-6.09646415968162e-09\\
3.24799999999997	-5.91640897576431e-09\\
3.24799999999999	-5.91640897576366e-09\\
3.24999999999997	-5.87054165617114e-09\\
3.25	-5.87054165617048e-09\\
3.25199999999998	-5.82432107295238e-09\\
3.25399999999996	-5.77774121900983e-09\\
3.25499999999998	-5.75431467563101e-09\\
3.255	-5.75431467563034e-09\\
3.25899999999996	-5.65967993153748e-09\\
3.25999999999997	-5.6357852595463e-09\\
3.26	-5.63578525954562e-09\\
3.26399999999996	-5.53923933719313e-09\\
3.26799999999992	-5.44110808376726e-09\\
3.26999999999997	-5.39143204859332e-09\\
3.27	-5.39143204859261e-09\\
3.27699999999999	-5.21425078007477e-09\\
3.27700000000002	-5.21425078007404e-09\\
3.27999999999997	-5.13668791482295e-09\\
3.28	-5.13668791482221e-09\\
3.28299999999995	-5.05891968641114e-09\\
3.2859999999999	-4.9817274340378e-09\\
3.28999999999997	-4.87966140352138e-09\\
3.29	-4.87966140352066e-09\\
3.2959999999999	-4.72829113782755e-09\\
3.29999999999997	-4.62845364219638e-09\\
3.3	-4.62845364219567e-09\\
3.3059999999999	-4.48018166248763e-09\\
3.30599999999995	-4.48018166248653e-09\\
3.30599999999999	-4.48018166248544e-09\\
3.30999999999997	-4.38224824588065e-09\\
3.31	-4.38224824587995e-09\\
3.31399999999998	-4.28498703436338e-09\\
3.31799999999996	-4.18834746602853e-09\\
3.31999999999997	-4.14024508625264e-09\\
3.32	-4.14024508625195e-09\\
3.32499999999998	-4.02172418856901e-09\\
3.325	-4.02172418856835e-09\\
3.32999999999998	-3.90626294353069e-09\\
3.33000000000001	-3.90626294353005e-09\\
3.33499999999998	-3.7937675621917e-09\\
3.33500000000001	-3.79376756219107e-09\\
3.33999999999998	-3.68414666511856e-09\\
3.34000000000001	-3.68414666511795e-09\\
3.34499999999998	-3.57731120780697e-09\\
3.34999999999996	-3.47317440835193e-09\\
3.35000000000001	-3.47317440835085e-09\\
3.35999999999996	-3.27150746621745e-09\\
3.36	-3.27150746621655e-09\\
3.36399999999997	-3.19297566630732e-09\\
3.36399999999999	-3.19297566630677e-09\\
3.36799999999996	-3.11560226529661e-09\\
3.36999999999997	-3.07733737128451e-09\\
3.37	-3.07733737128397e-09\\
3.37399999999997	-3.00162633115513e-09\\
3.37799999999993	-2.92697421464485e-09\\
3.37999999999997	-2.89003310394345e-09\\
3.38	-2.89003310394293e-09\\
3.38799999999993	-2.74472333427562e-09\\
3.38999999999997	-2.70898595443498e-09\\
3.39	-2.70898595443447e-09\\
3.39299999999997	-2.65600563868671e-09\\
3.39299999999999	-2.65600563868621e-09\\
3.39499999999998	-2.6211857478064e-09\\
3.395	-2.62118574780591e-09\\
3.39699999999999	-2.58676082549833e-09\\
3.39899999999997	-2.55272639790512e-09\\
3.39999999999997	-2.5358542354316e-09\\
3.4	-2.53585423543112e-09\\
3.40399999999997	-2.46931983998081e-09\\
3.40799999999993	-2.40428642172163e-09\\
3.40999999999997	-2.37232198382372e-09\\
3.41	-2.37232198382327e-09\\
3.41799999999993	-2.24804965871791e-09\\
3.41999999999997	-2.21785774589844e-09\\
3.42	-2.21785774589801e-09\\
3.42199999999997	-2.18800838522547e-09\\
3.42199999999999	-2.18800838522505e-09\\
3.42399999999996	-2.15849769767556e-09\\
3.42599999999992	-2.12932184784681e-09\\
3.42999999999985	-2.07195953762301e-09\\
3.42999999999997	-2.07195953762125e-09\\
3.43	-2.07195953762085e-09\\
3.43799999999986	-1.96200665985137e-09\\
3.43999999999997	-1.93558721061468e-09\\
3.44	-1.93558721061431e-09\\
3.44799999999986	-1.83408147990564e-09\\
3.44999999999997	-1.80973157406968e-09\\
3.45	-1.80973157406934e-09\\
3.45099999999996	-1.79770839269644e-09\\
3.45099999999999	-1.7977083926961e-09\\
3.45199999999999	-1.78578587046348e-09\\
3.45299999999998	-1.77396361995209e-09\\
3.45499999999996	-1.75061840093346e-09\\
3.45899999999993	-1.70511453984337e-09\\
3.45999999999997	-1.69398361736764e-09\\
3.46	-1.69398361736732e-09\\
3.465	-1.63978121911068e-09\\
3.46500000000003	-1.63978121911038e-09\\
3.46999999999997	-1.58796717810536e-09\\
3.47	-1.58796717810507e-09\\
3.47499999999994	-1.53849018382174e-09\\
3.47999999999988	-1.49130082427377e-09\\
3.47999999999994	-1.49130082427324e-09\\
3.47999999999999	-1.49130082427272e-09\\
3.48999999999987	-1.40363350978731e-09\\
3.48999999999996	-1.40363350978655e-09\\
3.48999999999999	-1.40363350978631e-09\\
3.49999999999987	-1.32468032989627e-09\\
3.49999999999996	-1.32468032989558e-09\\
3.49999999999999	-1.32468032989537e-09\\
3.50899999999999	-1.26086041393415e-09\\
3.50900000000002	-1.26086041393396e-09\\
3.50999999999999	-1.25418469932673e-09\\
3.51000000000002	-1.25418469932654e-09\\
3.51100000000001	-1.24758152074919e-09\\
3.512	-1.24104093714675e-09\\
3.51399999999999	-1.2281467068873e-09\\
3.51799999999996	-1.2031001714141e-09\\
3.51999999999997	-1.19094461107466e-09\\
3.52	-1.19094461107449e-09\\
3.52799999999994	-1.14473725937249e-09\\
3.52999999999997	-1.13378163742364e-09\\
3.53	-1.13378163742349e-09\\
3.53499999999998	-1.10742026070816e-09\\
3.535	-1.10742026070801e-09\\
3.53799999999997	-1.09230129534727e-09\\
3.53799999999999	-1.09230129534713e-09\\
3.53999999999997	-1.08251000784214e-09\\
3.54	-1.08251000784201e-09\\
3.54199999999998	-1.07294764960877e-09\\
3.54399999999996	-1.06361297792459e-09\\
3.54799999999992	-1.04562187110737e-09\\
3.54999999999997	-1.03696309779331e-09\\
3.55	-1.03696309779319e-09\\
3.55799999999992	-1.00380675926494e-09\\
3.56	-9.95835701639571e-10\\
3.56000000000003	-9.95835701639459e-10\\
3.56699999999996	-9.68915623529864e-10\\
3.56699999999999	-9.68915623529758e-10\\
3.56999999999998	-9.57836976468083e-10\\
3.57000000000001	-9.5783697646798e-10\\
3.57299999999999	-9.47028729125771e-10\\
3.57599999999997	-9.36487720987094e-10\\
3.57999999999997	-9.22843432539048e-10\\
3.58	-9.22843432538952e-10\\
3.58599999999997	-9.03242089233888e-10\\
3.58999999999997	-8.90741346406942e-10\\
3.59	-8.90741346406855e-10\\
3.59599999999997	-8.7253181517215e-10\\
3.596	-8.72531815172065e-10\\
3.6	-8.60614031577025e-10\\
3.60000000000003	-8.60614031576941e-10\\
3.60400000000003	-8.48866529660353e-10\\
3.60499999999998	-8.45955543126237e-10\\
3.605	-8.45955543126155e-10\\
3.60900000000001	-8.34412315383547e-10\\
3.61	-8.31551219250916e-10\\
3.61000000000003	-8.31551219250835e-10\\
3.61400000000003	-8.20202889176267e-10\\
3.61800000000004	-8.09003832791715e-10\\
3.61999999999997	-8.03458459906052e-10\\
3.62	-8.03458459905974e-10\\
3.62499999999999	-7.89747204825074e-10\\
3.62500000000002	-7.89747204824996e-10\\
3.62999999999998	-7.76244456608703e-10\\
3.63000000000001	-7.76244456608627e-10\\
3.63499999999997	-7.62939247002319e-10\\
3.63999999999993	-7.49820768251035e-10\\
3.63999999999997	-7.49820768250945e-10\\
3.64	-7.49820768250854e-10\\
3.64999999999993	-7.24207312104656e-10\\
3.64999999999997	-7.24207312104532e-10\\
3.65	-7.24207312104461e-10\\
3.65399999999996	-7.14199676270916e-10\\
3.65399999999999	-7.14199676270846e-10\\
3.65799999999995	-7.04320052976704e-10\\
3.65999999999997	-6.99426638454597e-10\\
3.66	-6.99426638454527e-10\\
3.66399999999996	-6.89729423847566e-10\\
3.66799999999993	-6.80147500623019e-10\\
3.66999999999997	-6.7539821434639e-10\\
3.67	-6.75398214346323e-10\\
3.67499999999998	-6.63530308088823e-10\\
3.67500000000001	-6.63530308088755e-10\\
3.67999999999998	-6.51598674527837e-10\\
3.68000000000001	-6.51598674527769e-10\\
3.68299999999996	-6.44405075781089e-10\\
3.68299999999999	-6.4440507578102e-10\\
3.68599999999995	-6.37182943714001e-10\\
3.68899999999991	-6.29930166474168e-10\\
3.68999999999998	-6.27505397728293e-10\\
3.69000000000001	-6.27505397728224e-10\\
3.69599999999992	-6.12875912693833e-10\\
3.69999999999997	-6.03040084635203e-10\\
3.7	-6.03040084635133e-10\\
3.70599999999992	-5.88149301894815e-10\\
3.71	-5.78123226900805e-10\\
3.71000000000003	-5.78123226900734e-10\\
3.71199999999996	-5.73091018610462e-10\\
3.71199999999999	-5.7309101861039e-10\\
3.71399999999993	-5.68061752500122e-10\\
3.71599999999986	-5.63034774957454e-10\\
3.71999999999973	-5.52985072567303e-10\\
3.71999999999997	-5.52985072566681e-10\\
3.72	-5.5298507256661e-10\\
3.72799999999974	-5.3288439466214e-10\\
3.72999999999997	-5.27855150479493e-10\\
3.73	-5.27855150479421e-10\\
3.73799999999974	-5.07702264901786e-10\\
3.73999999999997	-5.02651792441529e-10\\
3.74	-5.02651792441457e-10\\
3.74099999999996	-5.00124267731814e-10\\
3.74099999999999	-5.00124267731742e-10\\
3.74199999999998	-4.97595107903534e-10\\
3.74299999999998	-4.95064230773284e-10\\
3.74499999999996	-4.89996995638558e-10\\
3.74500000000001	-4.89996995638443e-10\\
3.74899999999997	-4.79838295671523e-10\\
3.74999999999997	-4.7729309159716e-10\\
3.75	-4.77293091597088e-10\\
3.75399999999997	-4.67167345432277e-10\\
3.75799999999994	-4.57157637888492e-10\\
3.75999999999997	-4.52194670157669e-10\\
3.76	-4.52194670157599e-10\\
3.76799999999994	-4.32607030425359e-10\\
3.76999999999996	-4.27772996260101e-10\\
3.76999999999999	-4.27772996260032e-10\\
3.77799999999993	-4.08669513813313e-10\\
3.77999999999996	-4.03948703377196e-10\\
3.77999999999999	-4.03948703377129e-10\\
3.78799999999993	-3.85302707978616e-10\\
3.78999999999996	-3.80699704610836e-10\\
3.78999999999999	-3.80699704610771e-10\\
3.79799999999993	-3.62503725045045e-10\\
3.79899999999996	-3.60252311388797e-10\\
3.79899999999999	-3.60252311388733e-10\\
3.79999999999996	-3.58005782683979e-10\\
3.79999999999999	-3.58005782683915e-10\\
3.80099999999998	-3.55764065952219e-10\\
3.80199999999997	-3.53527088350126e-10\\
3.80399999999995	-3.49067059971549e-10\\
3.80799999999991	-3.40200684932814e-10\\
3.80999999999996	-3.357931859718e-10\\
3.80999999999999	-3.35793185971737e-10\\
3.81499999999998	-3.24844771344557e-10\\
3.81500000000001	-3.24844771344495e-10\\
3.81999999999999	-3.13989727099737e-10\\
3.82000000000002	-3.13989727099675e-10\\
3.825	-3.03219235708126e-10\\
3.82799999999996	-2.96793884088281e-10\\
3.82799999999999	-2.9679388408822e-10\\
3.82999999999999	-2.92524548346873e-10\\
3.83000000000002	-2.92524548346812e-10\\
3.83200000000002	-2.88294575907205e-10\\
3.83400000000002	-2.84132042260303e-10\\
3.83800000000001	-2.76007136190375e-10\\
3.83999999999997	-2.72043707829588e-10\\
3.84	-2.72043707829532e-10\\
3.84799999999999	-2.56838238915802e-10\\
3.84999999999997	-2.53196462675009e-10\\
3.85	-2.53196462674958e-10\\
3.85699999999996	-2.40942373466437e-10\\
3.85699999999999	-2.40942373466389e-10\\
3.85999999999997	-2.35921562356343e-10\\
3.86	-2.35921562356296e-10\\
3.86299999999998	-2.31037176023923e-10\\
3.86599999999997	-2.26287786191116e-10\\
3.86999999999997	-2.20162866228495e-10\\
3.87	-2.20162866228452e-10\\
3.87599999999997	-2.11436141630538e-10\\
3.87999999999997	-2.05930893025605e-10\\
3.88	-2.05930893025567e-10\\
3.88500000000001	-1.99395925955059e-10\\
3.88500000000003	-1.99395925955023e-10\\
3.88599999999996	-1.98134716651213e-10\\
3.88599999999999	-1.98134716651178e-10\\
3.88699999999998	-1.96888671952929e-10\\
3.88799999999997	-1.95657751375489e-10\\
3.88999999999995	-1.93241123104214e-10\\
3.89	-1.93241123104155e-10\\
3.89399999999996	-1.88587629117252e-10\\
3.89799999999992	-1.84171815543064e-10\\
3.9	-1.82052316731245e-10\\
3.90000000000003	-1.82052316731215e-10\\
3.90799999999995	-1.7415736946297e-10\\
3.91	-1.72328112096307e-10\\
3.91000000000003	-1.72328112096282e-10\\
3.91499999999996	-1.67954190973339e-10\\
3.91499999999999	-1.67954190973314e-10\\
3.91999999999993	-1.63832922568453e-10\\
3.92	-1.63832922568392e-10\\
3.92000000000003	-1.63832922568369e-10\\
3.92499999999997	-1.59960959181594e-10\\
3.9299999999999	-1.56335155636481e-10\\
3.93000000000003	-1.56335155636389e-10\\
3.93000000000006	-1.56335155636369e-10\\
3.93999999999993	-1.49810444747907e-10\\
3.93999999999998	-1.49810444747878e-10\\
3.94	-1.49810444747861e-10\\
3.94399999999996	-1.4746816167273e-10\\
3.94399999999999	-1.47468161672714e-10\\
3.94799999999995	-1.45276922640747e-10\\
3.94999999999998	-1.44237585596833e-10\\
3.95	-1.44237585596819e-10\\
3.95399999999996	-1.42231618535389e-10\\
3.95499999999998	-1.417410974314e-10\\
3.95500000000001	-1.41741097431386e-10\\
3.95899999999997	-1.39822417542315e-10\\
3.96	-1.39353520779953e-10\\
3.96000000000003	-1.39353520779939e-10\\
3.96399999999999	-1.37520569529186e-10\\
3.96799999999995	-1.35755125881683e-10\\
3.97	-1.34897431240399e-10\\
3.97000000000003	-1.34897431240387e-10\\
3.97299999999996	-1.33641859002911e-10\\
3.97299999999999	-1.33641859002899e-10\\
3.97599999999992	-1.32423124893467e-10\\
3.97899999999986	-1.31240872529443e-10\\
3.97999999999997	-1.30854835404772e-10\\
3.98	-1.30854835404761e-10\\
3.98599999999987	-1.28542098160873e-10\\
3.98999999999997	-1.26989954659746e-10\\
3.99	-1.26989954659735e-10\\
3.99599999999987	-1.24644244083695e-10\\
3.99999999999997	-1.23067587919665e-10\\
4	-1.23067587919654e-10\\
4.00199999999993	-1.22275098261698e-10\\
4.00199999999999	-1.22275098261675e-10\\
4.00399999999992	-1.21479697021197e-10\\
4.00599999999986	-1.20681280826466e-10\\
4.00999999999972	-1.19074988120872e-10\\
4.00999999999995	-1.19074988120781e-10\\
4.01	-1.19074988120759e-10\\
4.01799999999973	-1.15821629814547e-10\\
4.01999999999995	-1.14999179950413e-10\\
4.02	-1.14999179950389e-10\\
4.02500000000001	-1.12925945169124e-10\\
4.02500000000006	-1.129259451691e-10\\
4.02999999999995	-1.10826917688256e-10\\
4.03	-1.10826917688232e-10\\
4.03099999999999	-1.10404490438879e-10\\
4.03100000000005	-1.10404490438855e-10\\
4.03200000000004	-1.09982195687722e-10\\
4.03300000000003	-1.09560019714196e-10\\
4.035	-1.08715969237917e-10\\
4.03899999999996	-1.07028690300309e-10\\
4.03999999999994	-1.06606961869765e-10\\
4.04	-1.06606961869741e-10\\
4.04799999999991	-1.03232311661412e-10\\
4.04999999999994	-1.02387918024115e-10\\
4.05	-1.02387918024091e-10\\
4.05799999999991	-9.90041266748656e-11\\
4.05999999999994	-9.81560749432171e-11\\
4.06	-9.8156074943193e-11\\
};
\end{axis}
\end{tikzpicture}%}
  \caption{The angular displacement of pendulum $P_3$ as a function of time.
    \texttt{Blue}: $C_3 = 6$ ms, \texttt{Red}: $C_3 = 10$ ms}
  \label{fig:01.5.6_10.3}
\end{figure}

Figure \ref{fig:01.5.3} shows the schedule calculated for each pendulum with all
jobs having execution time $C_i = 10$ ms.  Figure \ref{fig:01.5.4} illustrates
that the schedule is not feasible by ploting the overall usage of the CPU over
the length of a schedule period, which is at all times $100\%$, indicative of
the excessive processing load demanded.


\noindent\makebox[\textwidth][c]{%
\begin{minipage}{\linewidth}
  \begin{minipage}{0.45\linewidth}
    \begin{figure}[H]\centering
      \scalebox{0.7}{\input{./figures/01.rate_monotonic_scheduling/5/5.3.tex}}
      \caption{The calculated schedule for the three penduli.
        \texttt{Blue}: $P_1$, \texttt{Red}: $P_2$, \texttt{Orange}: $P_3$.
        $C_i = 10$ ms.}
      \label{fig:01.5.3}
    \end{figure}
  \end{minipage}
  \hfill
  \begin{minipage}{0.45\linewidth}
    \begin{figure}[H]\centering
    \scalebox{0.7}{\input{./figures/01.rate_monotonic_scheduling/5/5.4.tex}}
    \caption{The overall processing usage. Notice that it always at $100\%$.
      $C_i = 10$ ms.}
      \label{fig:01.5.4}
  \end{figure}
\end{minipage}
\end{minipage}
}
