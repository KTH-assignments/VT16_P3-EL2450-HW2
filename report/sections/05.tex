\section{Question 5}
In the case where $T_1 = 20, T_2 = 29, T_3 = 35$ ms and $C_i = 10$ ms,
$i=\{1,2,3\}$, $U=1.131 > 1$. Hence tasks $J_1, J_2, J_3$ are not schedulable
in any scheduling scheme.


\begin{figure}[H]\centering
  \scalebox{0.7}{\begin{ganttchart}[vgrid, hgrid]{0}{49}
%\gantttitle{2016}{12}\\
\gantttitlelist{10,20,...,50}{10}\\
\ganttset{progress label text={},
       bar incomplete/.append style={fill=black!40},
       group/.append style={draw=black, fill=black},}
\ganttbar{Task 1}{0}{9}
\ganttbar{}{20}{29}
\ganttbar{}{40}{49}\\

\ganttbar[progress=00]{Task 2}{0}{9}
\ganttbar{}{10}{19}
\ganttbar[progress=00]{}{29}{29}
\ganttbar{}{30}{39}\\

\ganttbar[progress=00]{Task 3}{0}{49}
\end{ganttchart}
}
  \caption{A portion of the RM schedule $\sigma$ for tasks $J_1, J_2, J_3$ for
    $C_i = 10$ ms.  Shaded areas denote the waiting time. Notice that $J_3$
    misses its deadlines consecutively, indicative of the inability of
    schedulability.}
\end{figure}

All penduli are still stable. However, due to the increased execution time, the
control input is not as swift as before, hence the increased magnitude of the
overshoot.

Figure \ref{fig:5.2} shows the angular displacement of each pendulum.



\begin{figure}[H]\centering
  \scalebox{1}{% This file was created by matlab2tikz.
%
%The latest updates can be retrieved from
%  http://www.mathworks.com/matlabcentral/fileexchange/22022-matlab2tikz-matlab2tikz
%where you can also make suggestions and rate matlab2tikz.
%
\definecolor{mycolor1}{rgb}{0.00000,0.44700,0.74100}%
\definecolor{mycolor2}{rgb}{0.85000,0.32500,0.09800}%
\definecolor{mycolor3}{rgb}{0.92900,0.69400,0.12500}%
%
\begin{tikzpicture}

\begin{axis}[%
width=4.133in,
height=3.26in,
at={(0.693in,0.44in)},
scale only axis,
xmin=0,
xmax=1,
xlabel={Time (seconds)},
xmajorgrids,
ymin=-0.15,
ymax=0.2,
ymajorgrids,
axis background/.style={fill=white},
title style={font=\bfseries},
]
\addplot [color=mycolor1,solid,forget plot]
  table[row sep=crcr]{%
0	0.15314\\
3.15544362088405e-30	0.15314\\
0.000656101980281985	0.153143230512962\\
0.00393661188169191	0.153256312778436\\
0.00999999999999994	0.153891071773171\\
0.01	0.153891071773171\\
0.0199999999999999	0.150048203824684\\
0.02	0.150048203824684\\
0.0289999999999998	0.137414337712804\\
0.029	0.137414337712803\\
0.03	0.135470213386942\\
0.0300000000000002	0.135470213386942\\
0.0349999999999996	0.124115011067004\\
0.035	0.124115011067003\\
0.0399999999999993	0.110014119663841\\
0.04	0.110014119663839\\
0.0449999999999993	0.0939630779639858\\
0.0499999999999987	0.0767526455719492\\
0.05	0.0767526455719445\\
0.0500000000000004	0.0767526455719429\\
0.0579999999999996	0.0466980443355424\\
0.058	0.0466980443355407\\
0.0599999999999996	0.0386819498575326\\
0.06	0.0386819498575308\\
0.0619999999999995	0.0306155753047838\\
0.0639999999999991	0.0226526210813104\\
0.0679999999999982	0.00702452540129683\\
0.0699999999999991	-0.000646743531092999\\
0.07	-0.000646743531096385\\
0.0779999999999982	-0.0304497245417899\\
0.0799999999999991	-0.0376945518218964\\
0.08	-0.0376945518218996\\
0.087	-0.0613629561565828\\
0.0870000000000009	-0.0613629561565856\\
0.09	-0.0705722498255595\\
0.0900000000000009	-0.0705722498255621\\
0.0929999999999999	-0.0792329294508874\\
0.095999999999999	-0.0873526353855747\\
0.0999999999999991	-0.0973497076333374\\
0.1	-0.0973497076333395\\
0.104999999999999	-0.108387871323025\\
0.105	-0.108387871323027\\
0.109999999999999	-0.117699946526397\\
0.11	-0.117699946526398\\
0.114999999999999	-0.125308754745586\\
0.115999999999999	-0.126627838237935\\
0.116	-0.126627838237936\\
0.119999999999999	-0.131232943213937\\
0.12	-0.131232943213938\\
0.123999999999999	-0.13476926202664\\
0.127999999999998	-0.137242340899987\\
0.129999999999998	-0.138081442501912\\
0.13	-0.138081442501912\\
0.137999999999998	-0.138771927810782\\
0.139999999999998	-0.138276983003393\\
0.14	-0.138276983003392\\
0.144999999999998	-0.135869686909084\\
0.145	-0.135869686909083\\
0.149999999999998	-0.131785845077204\\
0.15	-0.131785845077202\\
0.154999999999998	-0.126461814037968\\
0.159999999999996	-0.120330910865585\\
0.16	-0.12033091086558\\
0.169999999999996	-0.105586372774747\\
0.17	-0.105586372774741\\
0.173999999999998	-0.0990022340863097\\
0.174	-0.0990022340863068\\
0.174999999999998	-0.0973523335757812\\
0.175	-0.0973523335757782\\
0.176	-0.0957005634817941\\
0.177	-0.0940467619077872\\
0.179000000000001	-0.0907324157456936\\
0.179999999999998	-0.0890715463112649\\
0.18	-0.0890715463112619\\
0.184000000000001	-0.0823996248083219\\
0.188000000000002	-0.0756743350020158\\
0.189999999999998	-0.0722883843683618\\
0.19	-0.0722883843683588\\
0.198000000000002	-0.058558154107107\\
0.199999999999998	-0.0550722607951608\\
0.2	-0.0550722607951577\\
0.202999999999998	-0.0499542634358041\\
0.203	-0.0499542634358012\\
0.205999999999998	-0.045090375635058\\
0.208999999999996	-0.0404763063780602\\
0.209999999999998	-0.0389930887103154\\
0.21	-0.0389930887103128\\
0.215999999999996	-0.0305025648988234\\
0.219999999999998	-0.0251966015169791\\
0.22	-0.0251966015169768\\
0.225999999999996	-0.0177484283157274\\
0.229999999999998	-0.0131121370239185\\
0.23	-0.0131121370239165\\
0.231999999999998	-0.0108899737028716\\
0.232	-0.0108899737028697\\
0.233999999999998	-0.00873062873450069\\
0.235999999999997	-0.00663325550056961\\
0.239999999999993	-0.00262115909906458\\
0.239999999999996	-0.0026211590990612\\
0.24	-0.00262115909905783\\
0.244999999999998	0.00188687770292411\\
0.245	0.00188687770292559\\
0.249999999999998	0.00568827569695383\\
0.25	0.00568827569695508\\
0.254999999999999	0.00879235129348301\\
0.259999999999997	0.0112067117758103\\
0.26	0.0112067117758117\\
0.260999999999996	0.0116103534305937\\
0.261	0.0116103534305951\\
0.262	0.0119926581405718\\
0.263	0.0123536634279823\\
0.265	0.0130119151800953\\
0.269	0.0140742429758504\\
0.269999999999997	0.0142870265310786\\
0.27	0.0142870265310794\\
0.278	0.015231929494095\\
0.279999999999996	0.0152581815612772\\
0.28	0.0152581815612772\\
0.288	0.0148381646770985\\
0.289999999999996	0.0146213487868583\\
0.29	0.0146213487868579\\
0.298	0.0133042875626112\\
0.299999999999996	0.0128619274084787\\
0.3	0.0128619274084778\\
0.308	0.01063490283314\\
0.309999999999996	0.00996265931864272\\
0.31	0.00996265931864148\\
0.314999999999997	0.00820076427699913\\
0.315	0.00820076427699786\\
0.319	0.00675620801608072\\
0.319000000000004	0.00675620801607942\\
0.319999999999996	0.00638990890522511\\
0.32	0.0063899089052238\\
0.321	0.00602147428209788\\
0.321999999999999	0.00565086803499177\\
0.323999999999998	0.00490299515937\\
0.327999999999996	0.00337957699652842\\
0.329999999999996	0.00260343440859324\\
0.33	0.00260343440859186\\
0.337999999999996	-0.000386934584574294\\
0.339999999999996	-0.00109385045098048\\
0.34	-0.00109385045098172\\
0.347999999999996	-0.00376712537690133\\
0.348	-0.0037671253769025\\
0.349999999999996	-0.00439815708383284\\
0.35	-0.00439815708383395\\
0.351999999999996	-0.00501477739830985\\
0.353999999999993	-0.00561722810735759\\
0.357999999999985	-0.00678056001263201\\
0.359999999999996	-0.00734189732970837\\
0.36	-0.00734189732970936\\
0.367999999999985	-0.00927534065693867\\
0.369999999999996	-0.00967033707065848\\
0.37	-0.00967033707065915\\
0.377	-0.0108624199510937\\
0.377000000000004	-0.0108624199510943\\
0.379999999999997	-0.011295123023204\\
0.38	-0.0112951230232045\\
0.382999999999993	-0.0116814835302473\\
0.384999999999997	-0.0119134813252346\\
0.385	-0.011913481325235\\
0.387999999999993	-0.0122233415981324\\
0.389999999999997	-0.0124046089407074\\
0.39	-0.0124046089407077\\
0.392999999999993	-0.0126387283665326\\
0.395999999999986	-0.0128276905077501\\
0.399999999999997	-0.0130096777594896\\
0.4	-0.0130096777594898\\
0.405999999999986	-0.0131344066698936\\
0.406	-0.0131344066698937\\
0.406000000000004	-0.0131344066698937\\
0.41	-0.0131194192205885\\
0.410000000000004	-0.0131194192205884\\
0.414	-0.013025910252926\\
0.417999999999996	-0.0128537331127516\\
0.419999999999997	-0.0127380644190282\\
0.42	-0.012738064419028\\
0.427999999999993	-0.0121511608627946\\
0.429999999999997	-0.0119777204409489\\
0.43	-0.0119777204409485\\
0.435	-0.0114966423981403\\
0.435000000000004	-0.01149664239814\\
0.439999999999997	-0.0109467768969577\\
0.44	-0.0109467768969573\\
0.444999999999993	-0.010355451915085\\
0.449999999999986	-0.00974989383042669\\
0.449999999999993	-0.00974989383042581\\
0.45	-0.00974989383042494\\
0.454999999999997	-0.00912861851892244\\
0.455	-0.00912861851892199\\
0.459999999999997	-0.00849010351173955\\
0.46	-0.00849010351173909\\
0.463999999999997	-0.00797875498394609\\
0.464	-0.00797875498394564\\
0.467999999999997	-0.00748042273240351\\
0.469999999999997	-0.00723589270700806\\
0.47	-0.00723589270700762\\
0.473999999999997	-0.00675562548054035\\
0.477999999999993	-0.00628645622411489\\
0.479999999999997	-0.00605580272715413\\
0.48	-0.00605580272715372\\
0.487999999999993	-0.00515729754952254\\
0.489999999999997	-0.00493825812254068\\
0.49	-0.0049382581225403\\
0.492999999999997	-0.00462164187662262\\
0.493	-0.00462164187662226\\
0.495999999999997	-0.00432559297658477\\
0.498999999999993	-0.00404985024110665\\
0.499999999999997	-0.00396240592092304\\
0.5	-0.00396240592092273\\
0.505999999999993	-0.00348408566948026\\
0.509999999999993	-0.00320878304145653\\
0.51	-0.00320878304145607\\
0.515999999999993	-0.00285520465084399\\
0.519999999999993	-0.00265634789090958\\
0.52	-0.00265634789090925\\
0.521999999999993	-0.00256785689554054\\
0.522	-0.00256785689554024\\
0.523999999999993	-0.00248661028412352\\
0.524999999999993	-0.00244869355699522\\
0.525	-0.00244869355699495\\
0.526999999999993	-0.00237825467839496\\
0.528999999999986	-0.00231498584786251\\
0.529999999999993	-0.00228603233951768\\
0.53	-0.00228603233951748\\
0.533999999999986	-0.00218245323015712\\
0.537999999999972	-0.00209610281135348\\
0.539999999999993	-0.00205934498713197\\
0.54	-0.00205934498713185\\
0.547999999999972	-0.00195477946619444\\
0.549999999999993	-0.00193920328278263\\
0.55	-0.00193920328278258\\
0.550999999999993	-0.00193299477772194\\
0.551	-0.0019329947777219\\
0.551999999999997	-0.00192783848496902\\
0.552999999999993	-0.00192373389857262\\
0.554999999999986	-0.00191867833871959\\
0.558999999999972	-0.00192117689668293\\
0.559999999999993	-0.00192442875915677\\
0.56	-0.0019244287591568\\
0.567999999999972	-0.00197140268441475\\
0.57	-0.001988401838639\\
0.570000000000007	-0.00198840183863906\\
0.577999999999979	-0.0020592072895557\\
0.579999999999993	-0.00207649525762915\\
0.58	-0.00207649525762921\\
0.587999999999972	-0.00214419694775721\\
0.589999999999993	-0.00216079300364885\\
0.59	-0.00216079300364891\\
0.594999999999993	-0.00220177812842115\\
0.595	-0.00220177812842121\\
0.599999999999993	-0.00224212195069527\\
0.6	-0.00224212195069533\\
0.604999999999993	-0.00227749864227557\\
0.608999999999993	-0.00229909690584826\\
0.609	-0.0022990969058483\\
0.609999999999993	-0.00230357020084521\\
0.61	-0.00230357020084524\\
0.610999999999997	-0.00230767391869955\\
0.611999999999993	-0.00231140846167129\\
0.613999999999986	-0.00231777145091609\\
0.617999999999972	-0.00232608101572473\\
0.619999999999993	-0.00232803084947865\\
0.62	-0.00232803084947866\\
0.627999999999972	-0.00232051978163345\\
0.629999999999993	-0.00231477188658906\\
0.63	-0.00231477188658903\\
0.637999999999972	-0.00227623216538157\\
0.637999999999993	-0.00227623216538144\\
0.638	-0.00227623216538139\\
0.639999999999993	-0.00226269131337631\\
0.64	-0.00226269131337626\\
0.641999999999993	-0.00224783108334466\\
0.643999999999986	-0.00223189833618672\\
0.647999999999971	-0.00219678988174738\\
0.649999999999993	-0.0021776004092714\\
0.65	-0.00217760040927134\\
0.657999999999971	-0.00208976029042206\\
0.659999999999993	-0.00206498665594992\\
0.66	-0.00206498665594983\\
0.664999999999993	-0.00199802433862666\\
0.665	-0.00199802433862656\\
0.666999999999993	-0.00196920055372759\\
0.667	-0.00196920055372749\\
0.668999999999993	-0.00193919501446106\\
0.669999999999993	-0.0019237454289288\\
0.67	-0.00192374542892869\\
0.671999999999993	-0.00189246372396009\\
0.673999999999986	-0.00186100751792911\\
0.677999999999972	-0.00179752219965694\\
0.679999999999993	-0.00176546819611356\\
0.68	-0.00176546819611344\\
0.687999999999972	-0.00163488227406933\\
0.689999999999993	-0.00160157931998955\\
0.69	-0.00160157931998944\\
0.695999999999993	-0.00150276358338633\\
0.696	-0.00150276358338622\\
0.699999999999993	-0.00143846754788163\\
0.7	-0.00143846754788152\\
0.703999999999993	-0.00137531848145053\\
0.707999999999986	-0.00131321734327865\\
0.709999999999993	-0.00128252924894784\\
0.71	-0.00128252924894773\\
0.717999999999986	-0.00116191364707716\\
0.719999999999993	-0.00113223482981509\\
0.72	-0.00113223482981499\\
0.724999999999993	-0.00106143079390376\\
0.725	-0.00106143079390366\\
0.729999999999993	-0.000996873375159433\\
0.730000000000001	-0.000996873375159346\\
0.734999999999994	-0.000936717601452401\\
0.735000000000001	-0.000936717601452317\\
0.739999999999994	-0.000879129290255161\\
0.740000000000001	-0.000879129290255081\\
0.744999999999994	-0.000823967307569958\\
0.749999999999987	-0.000771096465700289\\
0.750000000000001	-0.000771096465700144\\
0.753999999999993	-0.00073036215460301\\
0.754	-0.000730362154602938\\
0.757999999999993	-0.000690947265248271\\
0.759999999999993	-0.000671715205610819\\
0.76	-0.000671715205610751\\
0.763999999999993	-0.000635005393658975\\
0.767999999999986	-0.00060114792352203\\
0.77	-0.000585272049199614\\
0.770000000000007	-0.000585272049199559\\
0.777999999999993	-0.000528645554914799\\
0.779999999999993	-0.000516180430556586\\
0.78	-0.000516180430556543\\
0.782999999999993	-0.000498598482522057\\
0.783	-0.000498598482522017\\
0.785999999999993	-0.000482235829801109\\
0.788999999999986	-0.000467078036854374\\
0.79	-0.000462290844231463\\
0.790000000000007	-0.000462290844231429\\
0.795999999999993	-0.000436322274410771\\
0.8	-0.000421602903029317\\
0.800000000000007	-0.000421602903029293\\
0.804999999999993	-0.000405445605427315\\
0.805	-0.000405445605427293\\
0.809999999999987	-0.00039117882537358\\
0.809999999999997	-0.000391178825373552\\
0.810000000000007	-0.000391178825373524\\
0.811999999999993	-0.00038599347258466\\
0.812	-0.000385993472584642\\
0.813999999999987	-0.00038110292454074\\
0.815999999999973	-0.000376505263846724\\
0.819999999999945	-0.000368181508555571\\
0.819999999999987	-0.00036818150855549\\
0.82	-0.000368181508555463\\
0.827999999999944	-0.00035497694563609\\
0.829999999999993	-0.000352385372545459\\
0.830000000000001	-0.00035238537254545\\
0.837999999999945	-0.000344115707971929\\
0.839999999999993	-0.000342523970614159\\
0.84	-0.000342523970614154\\
0.840999999999993	-0.000341798980015613\\
0.841000000000001	-0.000341798980015608\\
0.841999999999997	-0.000341121145212625\\
0.842999999999993	-0.000340490399724983\\
0.844999999999986	-0.000339369934026441\\
0.848999999999972	-0.000337691662864691\\
0.849999999999993	-0.000337389069357229\\
0.85	-0.000337389069357227\\
0.857999999999972	-0.000335259240190865\\
0.859999999999993	-0.000334758414659414\\
0.86	-0.000334758414659412\\
0.867999999999972	-0.000332875747277424\\
0.869999999999993	-0.000332434316465353\\
0.87	-0.000332434316465352\\
0.874999999999994	-0.000331068242922347\\
0.875000000000001	-0.000331068242922344\\
0.879999999999994	-0.000329146222069652\\
0.880000000000001	-0.000329146222069649\\
0.884999999999994	-0.000326663543542365\\
0.889999999999987	-0.000323614122940048\\
0.890000000000001	-0.000323614122940039\\
0.890000000000008	-0.000323614122940034\\
0.899	-0.000316486733966596\\
0.899000000000008	-0.000316486733966589\\
0.9	-0.000315554720433605\\
0.900000000000007	-0.000315554720433599\\
0.901000000000004	-0.000314594375941841\\
0.902	-0.000313605606321269\\
0.903999999999993	-0.0003115424012594\\
0.907999999999979	-0.00030707043771199\\
0.909999999999993	-0.00030465992587264\\
0.910000000000001	-0.000304659925872631\\
0.917999999999972	-0.000293831753429642\\
0.919999999999993	-0.000290822872510042\\
0.920000000000001	-0.000290822872510032\\
0.927999999999972	-0.000278102617137219\\
0.927999999999994	-0.000278102617137184\\
0.928000000000001	-0.000278102617137172\\
0.929999999999994	-0.000274779995718834\\
0.930000000000001	-0.000274779995718822\\
0.931999999999994	-0.000271402704015578\\
0.933999999999987	-0.000267974341572245\\
0.937999999999972	-0.000260959007541334\\
0.939999999999993	-0.000257369285387744\\
0.940000000000001	-0.000257369285387731\\
0.944999999999994	-0.000248144342019191\\
0.945000000000001	-0.000248144342019178\\
0.949999999999994	-0.000238543151774327\\
0.950000000000001	-0.000238543151774313\\
0.954999999999994	-0.000228542184423157\\
0.956999999999994	-0.000224424403949383\\
0.957000000000001	-0.000224424403949368\\
0.959999999999993	-0.000218116930134359\\
0.960000000000001	-0.000218116930134344\\
0.962999999999993	-0.000211787464606918\\
0.965999999999986	-0.000205570352027769\\
0.969999999999993	-0.000197446174189792\\
0.970000000000001	-0.000197446174189778\\
0.975999999999986	-0.000185588360519031\\
0.979999999999993	-0.000177884039190569\\
0.980000000000001	-0.000177884039190555\\
0.985999999999986	-0.000166598677044058\\
0.985999999999993	-0.000166598677044044\\
0.986000000000001	-0.000166598677044031\\
0.989999999999993	-0.000159238640787061\\
0.990000000000001	-0.000159238640787048\\
0.993999999999993	-0.000152203250136343\\
0.997999999999986	-0.000145688781985193\\
1	-0.000142623669560464\\
};
\addplot [color=mycolor2,solid,forget plot]
  table[row sep=crcr]{%
0	0.15313\\
3.15544362088405e-30	0.15313\\
0.000656101980281985	0.153131614989962\\
0.00393661188169191	0.153188143215565\\
0.00999999999999994	0.153505321610126\\
0.01	0.153505321610126\\
0.0199999999999999	0.154633126647887\\
0.02	0.154633126647887\\
0.0289999999999998	0.153421137821655\\
0.029	0.153421137821655\\
0.03	0.152969434437754\\
0.0300000000000002	0.152969434437754\\
0.0349999999999996	0.14975837658894\\
0.035	0.14975837658894\\
0.0399999999999993	0.144956580293631\\
0.04	0.14495658029363\\
0.0449999999999993	0.138558162777922\\
0.0499999999999987	0.130555285180189\\
0.05	0.130555285180187\\
0.0500000000000004	0.130555285180186\\
0.0579999999999996	0.115637633000987\\
0.058	0.115637633000986\\
0.0599999999999996	0.111647492594861\\
0.06	0.11164749259486\\
0.0619999999999995	0.107551527943661\\
0.0639999999999991	0.103348936174503\\
0.0679999999999982	0.0946205553357658\\
0.0699999999999991	0.090093055415954\\
0.07	0.0900930554159519\\
0.0779999999999982	0.0708733669527515\\
0.0799999999999991	0.0657863137092437\\
0.08	0.0657863137092414\\
0.087	0.0470707264257497\\
0.0870000000000009	0.0470707264257473\\
0.09	0.038608115811774\\
0.0900000000000009	0.0386081158117714\\
0.0929999999999999	0.0301187078522066\\
0.095999999999999	0.0218422822023367\\
0.0999999999999991	0.0111320584192047\\
0.1	0.0111320584192024\\
0.104999999999999	-0.0017451412647055\\
0.105	-0.00174514126470775\\
0.109999999999999	-0.0140697219629554\\
0.11	-0.0140697219629576\\
0.114999999999999	-0.0258567828338202\\
0.115999999999999	-0.028150974796736\\
0.116	-0.0281509747967381\\
0.119999999999999	-0.0371207645085133\\
0.12	-0.0371207645085153\\
0.123999999999999	-0.045606556121459\\
0.127999999999998	-0.0534569316870435\\
0.129999999999998	-0.0571457813072444\\
0.13	-0.0571457813072476\\
0.137999999999998	-0.0703421052010178\\
0.139999999999998	-0.0732546507332732\\
0.14	-0.0732546507332757\\
0.144999999999998	-0.0798659456623785\\
0.145	-0.0798659456623807\\
0.149999999999998	-0.0855263385064872\\
0.15	-0.0855263385064891\\
0.154999999999998	-0.0902427640154369\\
0.159999999999996	-0.0940210004002968\\
0.16	-0.0940210004002991\\
0.169999999999996	-0.0986317226348858\\
0.17	-0.0986317226348867\\
0.173999999999998	-0.0993526370731427\\
0.174	-0.0993526370731429\\
0.174999999999998	-0.0994327300022726\\
0.175	-0.0994327300022727\\
0.176	-0.0994727794614947\\
0.177	-0.0994727874134758\\
0.179000000000001	-0.0993526768352302\\
0.179999999999998	-0.0992325524195822\\
0.18	-0.099232552419582\\
0.184000000000001	-0.0985283580974968\\
0.188000000000002	-0.0975365687988356\\
0.189999999999998	-0.0969325918476496\\
0.19	-0.096932591847649\\
0.198000000000002	-0.093793207822887\\
0.199999999999998	-0.0928267236024457\\
0.2	-0.0928267236024448\\
0.202999999999998	-0.0912400352354355\\
0.203	-0.0912400352354345\\
0.205999999999998	-0.0894883650625657\\
0.208999999999996	-0.0875709405558083\\
0.209999999999998	-0.0868948207118116\\
0.21	-0.0868948207118103\\
0.215999999999996	-0.0824474621646144\\
0.219999999999998	-0.0791078050053071\\
0.22	-0.0791078050053055\\
0.225999999999996	-0.0738479984213533\\
0.229999999999998	-0.0703120493504639\\
0.23	-0.0703120493504624\\
0.231999999999998	-0.0685342136045761\\
0.232	-0.0685342136045745\\
0.233999999999998	-0.0667493395513993\\
0.235999999999997	-0.0649570773420509\\
0.239999999999993	-0.0613489817813503\\
0.239999999999996	-0.0613489817813471\\
0.24	-0.061348981781344\\
0.244999999999998	-0.0567910135981784\\
0.245	-0.0567910135981768\\
0.249999999999998	-0.0521746653123061\\
0.25	-0.0521746653123044\\
0.254999999999999	-0.0476242423318699\\
0.259999999999997	-0.0432641308592881\\
0.26	-0.0432641308592851\\
0.260999999999996	-0.0424144718283425\\
0.261	-0.0424144718283395\\
0.262	-0.0415721692525926\\
0.263	-0.0407371818383915\\
0.265	-0.0390889891928653\\
0.269	-0.0358786093708109\\
0.269999999999997	-0.0350937022935772\\
0.27	-0.0350937022935744\\
0.278	-0.0290632043510933\\
0.279999999999996	-0.0276233282023782\\
0.28	-0.0276233282023757\\
0.288	-0.0221263490325744\\
0.289999999999996	-0.0208163887975932\\
0.29	-0.0208163887975909\\
0.298	-0.0160570438801743\\
0.299999999999996	-0.0150006033982667\\
0.3	-0.0150006033982649\\
0.308	-0.0113022115444085\\
0.309999999999996	-0.010508549962371\\
0.31	-0.0105085499623696\\
0.314999999999997	-0.00875179114672262\\
0.315	-0.00875179114672148\\
0.319	-0.00757916087479443\\
0.319000000000004	-0.00757916087479348\\
0.319999999999996	-0.00731820843277085\\
0.32	-0.00731820843276995\\
0.321	-0.00706827602449595\\
0.321999999999999	-0.0068275159056266\\
0.323999999999998	-0.00637346579579224\\
0.327999999999996	-0.00557494401156089\\
0.329999999999996	-0.00523031581749665\\
0.33	-0.00523031581749607\\
0.337999999999996	-0.00421452794011216\\
0.339999999999996	-0.00405101331592819\\
0.34	-0.00405101331592793\\
0.347999999999996	-0.00375772258736748\\
0.348	-0.00375772258736748\\
0.349999999999996	-0.00377451998418073\\
0.35	-0.00377451998418079\\
0.351999999999996	-0.00381647679652969\\
0.353999999999993	-0.0038727166390681\\
0.357999999999985	-0.00402809230915985\\
0.359999999999996	-0.00412725859153486\\
0.36	-0.00412725859153505\\
0.367999999999985	-0.00466759973152374\\
0.369999999999996	-0.00483873640420929\\
0.37	-0.00483873640420961\\
0.377	-0.0055520172634314\\
0.377000000000004	-0.00555201726343181\\
0.379999999999997	-0.0059124410860251\\
0.38	-0.00591244108602555\\
0.382999999999993	-0.00628759771240643\\
0.384999999999997	-0.00653578599841538\\
0.385	-0.00653578599841582\\
0.387999999999993	-0.00690531584631486\\
0.389999999999997	-0.00714990444139671\\
0.39	-0.00714990444139714\\
0.392999999999993	-0.0075142602310904\\
0.395999999999986	-0.00787572623178794\\
0.399999999999997	-0.00835346102560518\\
0.4	-0.0083534610256056\\
0.405999999999986	-0.00906178009991046\\
0.406	-0.00906178009991213\\
0.406000000000004	-0.00906178009991255\\
0.41	-0.00952901067364299\\
0.410000000000004	-0.00952901067364341\\
0.414	-0.00997687415267218\\
0.417999999999996	-0.0103899124737084\\
0.419999999999997	-0.0105834744637126\\
0.42	-0.010583474463713\\
0.427999999999993	-0.0112722069208244\\
0.429999999999997	-0.0114231799044292\\
0.43	-0.0114231799044295\\
0.435	-0.011763825106213\\
0.435000000000004	-0.0117638251062133\\
0.439999999999997	-0.0120522432341001\\
0.44	-0.0120522432341002\\
0.444999999999993	-0.0122887876395593\\
0.449999999999986	-0.0124737481190688\\
0.449999999999993	-0.012473748119069\\
0.45	-0.0124737481190693\\
0.454999999999997	-0.0126006355076542\\
0.455	-0.0126006355076543\\
0.459999999999997	-0.012662889470727\\
0.46	-0.0126628894707271\\
0.463999999999997	-0.0126662104915079\\
0.464	-0.0126662104915079\\
0.467999999999997	-0.012628219059769\\
0.469999999999997	-0.0125937231314455\\
0.47	-0.0125937231314454\\
0.473999999999997	-0.0124936970420011\\
0.477999999999993	-0.0123522232406856\\
0.479999999999997	-0.0122659100556399\\
0.48	-0.0122659100556397\\
0.487999999999993	-0.0118547784292342\\
0.489999999999997	-0.0117378233332886\\
0.49	-0.0117378233332884\\
0.492999999999997	-0.0115516712471934\\
0.493	-0.0115516712471932\\
0.495999999999997	-0.0113525824375043\\
0.498999999999993	-0.0111404691009665\\
0.499999999999997	-0.0110668543137872\\
0.5	-0.0110668543137869\\
0.505999999999993	-0.0105943507351475\\
0.509999999999993	-0.0102497139092464\\
0.51	-0.0102497139092458\\
0.515999999999993	-0.00968763143281645\\
0.519999999999993	-0.00928239649471308\\
0.52	-0.00928239649471234\\
0.521999999999993	-0.00907396621485642\\
0.522	-0.00907396621485568\\
0.523999999999993	-0.00886621779288999\\
0.524999999999993	-0.0087625865484552\\
0.525	-0.00876258654845447\\
0.526999999999993	-0.00855578460269177\\
0.528999999999986	-0.00834956294861666\\
0.529999999999993	-0.00824665709535903\\
0.53	-0.0082466570953583\\
0.533999999999986	-0.00783628259099601\\
0.537999999999972	-0.007427665240691\\
0.539999999999993	-0.00722391533061898\\
0.54	-0.00722391533061826\\
0.547999999999972	-0.00641170964803699\\
0.549999999999993	-0.00620915772116902\\
0.55	-0.0062091577211683\\
0.550999999999993	-0.00610839228405739\\
0.551	-0.00610839228405667\\
0.551999999999997	-0.00600857782898421\\
0.552999999999993	-0.00590970946386365\\
0.554999999999986	-0.00571479167132077\\
0.558999999999972	-0.00533609908161176\\
0.559999999999993	-0.00524372026360131\\
0.56	-0.00524372026360066\\
0.567999999999972	-0.00453702881423189\\
0.57	-0.00436918072026969\\
0.570000000000007	-0.0043691807202691\\
0.577999999999979	-0.00373210030225669\\
0.579999999999993	-0.00358125209794317\\
0.58	-0.00358125209794264\\
0.587999999999972	-0.00301944684051335\\
0.589999999999993	-0.002889803828792\\
0.59	-0.00288980382879155\\
0.594999999999993	-0.002584329428027\\
0.595	-0.00258432942802658\\
0.599999999999993	-0.00230517829132263\\
0.6	-0.00230517829132225\\
0.604999999999993	-0.00205200842116886\\
0.608999999999993	-0.00186796954658788\\
0.609	-0.00186796954658757\\
0.609999999999993	-0.0018245096526498\\
0.61	-0.00182450965264949\\
0.610999999999997	-0.00178199627320724\\
0.611999999999993	-0.00174035824342754\\
0.613999999999986	-0.00165970011456107\\
0.617999999999972	-0.00150880114248635\\
0.619999999999993	-0.00143853072140514\\
0.62	-0.00143853072140489\\
0.627999999999972	-0.00119166260307253\\
0.629999999999993	-0.00113843848024014\\
0.63	-0.00113843848023996\\
0.637999999999972	-0.000959200605772057\\
0.637999999999993	-0.000959200605771646\\
0.638	-0.000959200605771511\\
0.639999999999993	-0.000922761876475018\\
0.64	-0.000922761876474895\\
0.641999999999993	-0.000889656176771994\\
0.643999999999986	-0.000859877017096072\\
0.647999999999971	-0.000810275621453308\\
0.649999999999993	-0.000790443662997569\\
0.65	-0.000790443662997505\\
0.657999999999971	-0.000737416471409183\\
0.659999999999993	-0.000730300623093721\\
0.66	-0.0007303006230937\\
0.664999999999993	-0.000723240435327344\\
0.665	-0.000723240435327345\\
0.666999999999993	-0.000724706680384332\\
0.667	-0.000724706680384341\\
0.668999999999993	-0.0007286246977068\\
0.669999999999993	-0.000731503342169344\\
0.67	-0.000731503342169367\\
0.671999999999993	-0.000739100607974611\\
0.673999999999986	-0.000749152467286245\\
0.677999999999972	-0.000776628328711077\\
0.679999999999993	-0.000794057716305854\\
0.68	-0.00079405771630592\\
0.687999999999972	-0.000873105436707512\\
0.689999999999993	-0.000894259205442502\\
0.69	-0.000894259205442578\\
0.695999999999993	-0.000961126830509811\\
0.696	-0.000961126830509893\\
0.699999999999993	-0.00100858781550804\\
0.7	-0.00100858781550812\\
0.703999999999993	-0.00105839800415164\\
0.707999999999986	-0.00111059645024464\\
0.709999999999993	-0.00113760398563549\\
0.71	-0.00113760398563559\\
0.717999999999986	-0.0012413546296336\\
0.719999999999993	-0.00126559425618062\\
0.72	-0.00126559425618071\\
0.724999999999993	-0.00132327427264495\\
0.725	-0.00132327427264502\\
0.729999999999993	-0.00137684013859725\\
0.730000000000001	-0.00137684013859733\\
0.734999999999994	-0.00142635747936927\\
0.735000000000001	-0.00142635747936934\\
0.739999999999994	-0.00147188695989593\\
0.740000000000001	-0.001471886959896\\
0.744999999999994	-0.00151348435948468\\
0.749999999999987	-0.00155120064015199\\
0.750000000000001	-0.00155120064015209\\
0.753999999999993	-0.00157735379664861\\
0.754	-0.00157735379664865\\
0.757999999999993	-0.00159855985748476\\
0.759999999999993	-0.00160731307452895\\
0.76	-0.00160731307452897\\
0.763999999999993	-0.00162112846046538\\
0.767999999999986	-0.00163003107373084\\
0.77	-0.00163264242228832\\
0.770000000000007	-0.00163264242228833\\
0.777999999999993	-0.00163402803412833\\
0.779999999999993	-0.00163230962377641\\
0.78	-0.0016323096237764\\
0.782999999999993	-0.00162818244094053\\
0.783	-0.00162818244094052\\
0.785999999999993	-0.00162219409866477\\
0.788999999999986	-0.00161434195603207\\
0.79	-0.00161130982246483\\
0.790000000000007	-0.0016113098224648\\
0.795999999999993	-0.0015887511829903\\
0.8	-0.00156954007733136\\
0.800000000000007	-0.00156954007733132\\
0.804999999999993	-0.00154080649094398\\
0.805	-0.00154080649094393\\
0.809999999999987	-0.00150679563310058\\
0.809999999999997	-0.0015067956331005\\
0.810000000000007	-0.00150679563310043\\
0.811999999999993	-0.0014919412420881\\
0.812	-0.00149194124208805\\
0.813999999999987	-0.00147670638900483\\
0.815999999999973	-0.00146108808774405\\
0.819999999999945	-0.0014286888199843\\
0.819999999999987	-0.00142868881998395\\
0.82	-0.00142868881998383\\
0.827999999999944	-0.0013591491013104\\
0.829999999999993	-0.00134075666164557\\
0.830000000000001	-0.00134075666164551\\
0.837999999999945	-0.00126304869651438\\
0.839999999999993	-0.00124256811471934\\
0.84	-0.00124256811471927\\
0.840999999999993	-0.00123222570545369\\
0.841000000000001	-0.00123222570545361\\
0.841999999999997	-0.00122189271573142\\
0.842999999999993	-0.0012115686391217\\
0.844999999999986	-0.00119094520212441\\
0.848999999999972	-0.00114978305289414\\
0.849999999999993	-0.00113950722460618\\
0.85	-0.00113950722460611\\
0.857999999999972	-0.00105743389112843\\
0.859999999999993	-0.0010369341522691\\
0.86	-0.00103693415226903\\
0.867999999999972	-0.000954889015545967\\
0.869999999999993	-0.000934346084458745\\
0.87	-0.000934346084458672\\
0.874999999999994	-0.000882889345164364\\
0.875000000000001	-0.000882889345164291\\
0.879999999999994	-0.000831240134318731\\
0.880000000000001	-0.000831240134318657\\
0.884999999999994	-0.000780901096501069\\
0.889999999999987	-0.000733376481736367\\
0.890000000000001	-0.000733376481736241\\
0.890000000000008	-0.000733376481736175\\
0.899	-0.000654740828904449\\
0.899000000000008	-0.000654740828904391\\
0.9	-0.000646541003483311\\
0.900000000000007	-0.000646541003483253\\
0.901000000000004	-0.000638447160425065\\
0.902	-0.000630458902985742\\
0.903999999999993	-0.000614797584413385\\
0.907999999999979	-0.000584725096243026\\
0.909999999999993	-0.000570308032111906\\
0.910000000000001	-0.000570308032111856\\
0.917999999999972	-0.000516911344203481\\
0.919999999999993	-0.000504630022639575\\
0.920000000000001	-0.000504630022639532\\
0.927999999999972	-0.000459703911601104\\
0.927999999999994	-0.000459703911600993\\
0.928000000000001	-0.000459703911600956\\
0.929999999999994	-0.000449511170401967\\
0.930000000000001	-0.000449511170401931\\
0.931999999999994	-0.000439729826664271\\
0.933999999999987	-0.000430357963039267\\
0.937999999999972	-0.000412835408365774\\
0.939999999999993	-0.000404681282702018\\
0.940000000000001	-0.00040468128270199\\
0.944999999999994	-0.000386054087615957\\
0.945000000000001	-0.000386054087615933\\
0.949999999999994	-0.000369920603371119\\
0.950000000000001	-0.000369920603371098\\
0.954999999999994	-0.000356151362573767\\
0.956999999999994	-0.000351271066955804\\
0.957000000000001	-0.000351271066955787\\
0.959999999999993	-0.000344619794471207\\
0.960000000000001	-0.000344619794471192\\
0.962999999999993	-0.000338768945938035\\
0.965999999999986	-0.00033371594102731\\
0.969999999999993	-0.000328215890113132\\
0.970000000000001	-0.000328215890113124\\
0.975999999999986	-0.000321412951872142\\
0.979999999999993	-0.000317305783940397\\
0.980000000000001	-0.00031730578394039\\
0.985999999999986	-0.000311779167283901\\
0.985999999999993	-0.000311779167283894\\
0.986000000000001	-0.000311779167283888\\
0.989999999999993	-0.000308513300192605\\
0.990000000000001	-0.0003085133001926\\
0.993999999999993	-0.000305579242748409\\
0.997999999999986	-0.000302974694493122\\
1	-0.000301795338108548\\
};
\addplot [color=mycolor3,solid,forget plot]
  table[row sep=crcr]{%
0	0.10153\\
3.15544362088405e-30	0.10153\\
0.000656101980281985	0.101530709989553\\
0.00393661188169191	0.101555560666546\\
0.00999999999999994	0.101694978093407\\
0.01	0.101694978093407\\
0.0199999999999999	0.102190448599525\\
0.02	0.102190448599525\\
0.0289999999999998	0.10292025163065\\
0.029	0.10292025163065\\
0.03	0.103018021716247\\
0.0300000000000002	0.103018021716247\\
0.0349999999999996	0.103192147327812\\
0.035	0.103192147327812\\
0.0399999999999993	0.102720100364795\\
0.04	0.102720100364795\\
0.0449999999999993	0.101601497399549\\
0.0499999999999987	0.0998354297964996\\
0.05	0.099835429796499\\
0.0500000000000004	0.0998354297964988\\
0.0579999999999996	0.0956592171978746\\
0.058	0.0956592171978743\\
0.0599999999999996	0.0943546352961654\\
0.06	0.0943546352961651\\
0.0619999999999995	0.0929455231467101\\
0.0639999999999991	0.0914316976162889\\
0.0679999999999982	0.0880891058325206\\
0.0699999999999991	0.0862599051673818\\
0.07	0.086259905167381\\
0.0779999999999982	0.0783631580954561\\
0.0799999999999991	0.0762726013656067\\
0.08	0.0762726013656058\\
0.087	0.0685830418499841\\
0.0870000000000009	0.0685830418499831\\
0.09	0.065107935293602\\
0.0900000000000009	0.065107935293601\\
0.0929999999999999	0.0615236275464586\\
0.095999999999999	0.0578290704987055\\
0.0999999999999991	0.0527296235853229\\
0.1	0.0527296235853218\\
0.104999999999999	0.0460729370657436\\
0.105	0.0460729370657424\\
0.109999999999999	0.0390974387049891\\
0.11	0.0390974387049878\\
0.114999999999999	0.0320420412716109\\
0.115999999999999	0.0306502167315683\\
0.116	0.0306502167315671\\
0.119999999999999	0.0251455926314253\\
0.12	0.0251455926314241\\
0.123999999999999	0.0197391081211328\\
0.127999999999998	0.0144279526059265\\
0.129999999999998	0.011807258483974\\
0.13	0.0118072584839717\\
0.137999999999998	0.00154912692971668\\
0.139999999999998	-0.000960911168698507\\
0.14	-0.000960911168700728\\
0.144999999999998	-0.00705216971721133\\
0.145	-0.00705216971721343\\
0.149999999999998	-0.0128321349015648\\
0.15	-0.0128321349015668\\
0.154999999999998	-0.0183055017781117\\
0.159999999999996	-0.0234767163352839\\
0.16	-0.0234767163352874\\
0.169999999999996	-0.0329292486698219\\
0.17	-0.032929248669825\\
0.173999999999998	-0.0363834968265353\\
0.174	-0.0363834968265368\\
0.174999999999998	-0.0372182447500583\\
0.175	-0.0372182447500598\\
0.176	-0.0380415218348881\\
0.177	-0.0388533548293884\\
0.179000000000001	-0.0404427936812492\\
0.179999999999998	-0.0412204511793037\\
0.18	-0.0412204511793051\\
0.184000000000001	-0.0442179202275956\\
0.188000000000002	-0.0470354872327416\\
0.189999999999998	-0.0483772688386875\\
0.19	-0.0483772688386887\\
0.198000000000002	-0.0531152697228068\\
0.199999999999998	-0.0541316071613497\\
0.2	-0.0541316071613506\\
0.202999999999998	-0.0555305499846143\\
0.203	-0.0555305499846151\\
0.205999999999998	-0.0567791771522262\\
0.208999999999996	-0.057877853791348\\
0.209999999999998	-0.0582108137877957\\
0.21	-0.0582108137877963\\
0.215999999999996	-0.0598600144100443\\
0.219999999999998	-0.0606281454731618\\
0.22	-0.0606281454731621\\
0.225999999999996	-0.0612843363763174\\
0.229999999999998	-0.0613914581691176\\
0.23	-0.0613914581691176\\
0.231999999999998	-0.0613569673420705\\
0.232	-0.0613569673420704\\
0.233999999999998	-0.0612784599866984\\
0.235999999999997	-0.061155925900476\\
0.239999999999993	-0.0607787081136675\\
0.239999999999996	-0.0607787081136671\\
0.24	-0.0607787081136667\\
0.244999999999998	-0.0600591304216388\\
0.245	-0.0600591304216385\\
0.249999999999998	-0.0590633795606718\\
0.25	-0.0590633795606714\\
0.254999999999999	-0.0577906466877398\\
0.259999999999997	-0.0562398979683205\\
0.26	-0.0562398979683193\\
0.260999999999996	-0.0558962794551987\\
0.261	-0.0558962794551974\\
0.262	-0.0555414794859283\\
0.263	-0.0551754865306404\\
0.265	-0.0544098737339776\\
0.269	-0.0527437846910428\\
0.269999999999997	-0.0522990874689763\\
0.27	-0.0522990874689747\\
0.278	-0.0486075012896723\\
0.279999999999996	-0.0476563426766285\\
0.28	-0.0476563426766268\\
0.288	-0.0437349511915884\\
0.289999999999996	-0.042724777029565\\
0.29	-0.0427247770295632\\
0.298	-0.0385608363448714\\
0.299999999999996	-0.037488363729724\\
0.3	-0.0374883637297221\\
0.308	-0.0332378303916068\\
0.309999999999996	-0.0321949371574688\\
0.31	-0.032194937157467\\
0.314999999999997	-0.029620762633946\\
0.315	-0.0296207626339442\\
0.319	-0.0275943241006294\\
0.319000000000004	-0.0275943241006276\\
0.319999999999996	-0.0270921464080909\\
0.32	-0.0270921464080891\\
0.321	-0.0265917087750128\\
0.321999999999999	-0.0260929949421357\\
0.323999999999998	-0.0251006739202311\\
0.327999999999996	-0.0231360082320983\\
0.329999999999996	-0.0221634082319868\\
0.33	-0.0221634082319851\\
0.337999999999996	-0.0183349540470543\\
0.339999999999996	-0.0173927050188045\\
0.34	-0.0173927050188028\\
0.347999999999996	-0.0137940267762798\\
0.348	-0.0137940267762782\\
0.349999999999996	-0.0129435092749043\\
0.35	-0.0129435092749028\\
0.351999999999996	-0.0121124277201381\\
0.353999999999993	-0.0113006740988322\\
0.357999999999985	-0.00973473117336946\\
0.359999999999996	-0.00898033835435596\\
0.36	-0.00898033835435464\\
0.367999999999985	-0.00615103237994316\\
0.369999999999996	-0.00549031258752466\\
0.37	-0.0054903125875235\\
0.377	-0.00332268163643367\\
0.377000000000004	-0.00332268163643262\\
0.379999999999997	-0.00246208994964999\\
0.38	-0.002462089949649\\
0.382999999999993	-0.00164216481576596\\
0.384999999999997	-0.00111802170159374\\
0.385	-0.00111802170159282\\
0.387999999999993	-0.000365347197260713\\
0.389999999999997	0.000114170794730868\\
0.39	0.000114170794731703\\
0.392999999999993	0.000793425379057254\\
0.395999999999986	0.00141940400203702\\
0.399999999999997	0.00217148258534534\\
0.4	0.00217148258534597\\
0.405999999999986	0.00312346260189153\\
0.406	0.00312346260189354\\
0.406000000000004	0.00312346260189404\\
0.41	0.00364117185530289\\
0.410000000000004	0.00364117185530331\\
0.414	0.00406564691138126\\
0.417999999999996	0.00439710843664848\\
0.419999999999997	0.0045280148603674\\
0.42	0.00452801486036761\\
0.427999999999993	0.00481984009384359\\
0.429999999999997	0.00483489369551712\\
0.43	0.00483489369551712\\
0.435	0.00480207239530616\\
0.435000000000004	0.00480207239530611\\
0.439999999999997	0.00468619349116265\\
0.44	0.00468619349116254\\
0.444999999999993	0.00448716285586529\\
0.449999999999986	0.0042048188178285\\
0.449999999999993	0.00420481881782804\\
0.45	0.00420481881782758\\
0.454999999999997	0.00383893202858688\\
0.455	0.00383893202858659\\
0.459999999999997	0.00338920528286627\\
0.46	0.00338920528286592\\
0.463999999999997	0.00296881592364295\\
0.464	0.00296881592364255\\
0.467999999999997	0.00249431796400048\\
0.469999999999997	0.00223670227654669\\
0.47	0.00223670227654622\\
0.473999999999997	0.00170503456494174\\
0.477999999999993	0.00116753023061576\\
0.479999999999997	0.000896502130446468\\
0.48	0.000896502130445985\\
0.487999999999993	-0.000203607417958384\\
0.489999999999997	-0.000482812805593269\\
0.49	-0.000482812805593767\\
0.492999999999997	-0.000904901238874969\\
0.493	-0.000904901238875472\\
0.495999999999997	-0.00133103588832417\\
0.498999999999993	-0.00176134136389649\\
0.499999999999997	-0.00190572508498463\\
0.5	-0.00190572508498514\\
0.505999999999993	-0.00275260833343492\\
0.509999999999993	-0.00329432190879493\\
0.51	-0.00329432190879587\\
0.515999999999993	-0.00407328344997044\\
0.519999999999993	-0.0045705789638816\\
0.52	-0.00457057896388246\\
0.521999999999993	-0.0048127208643742\\
0.522	-0.00481272086437505\\
0.523999999999993	-0.00505056767222024\\
0.524999999999993	-0.0051678900978277\\
0.525	-0.00516789009782853\\
0.526999999999993	-0.00539935205137182\\
0.528999999999986	-0.00562659515125581\\
0.529999999999993	-0.00573864388133106\\
0.53	-0.00573864388133186\\
0.533999999999986	-0.00617643788416967\\
0.537999999999972	-0.00659775989691432\\
0.539999999999993	-0.00680231268562629\\
0.54	-0.00680231268562701\\
0.547999999999972	-0.00755442208911304\\
0.549999999999993	-0.00772442179205999\\
0.55	-0.00772442179206058\\
0.550999999999993	-0.00780673324749164\\
0.551	-0.00780673324749222\\
0.551999999999997	-0.00788725602367983\\
0.552999999999993	-0.00796599273723251\\
0.554999999999986	-0.00811811815101259\\
0.558999999999972	-0.00840104490687427\\
0.559999999999993	-0.00846734757065569\\
0.56	-0.00846734757065616\\
0.567999999999972	-0.00893431802287783\\
0.57	-0.00903350441148642\\
0.570000000000007	-0.00903350441148676\\
0.577999999999979	-0.0093604118139374\\
0.579999999999993	-0.00942473223335562\\
0.58	-0.00942473223335584\\
0.587999999999972	-0.00961263898270645\\
0.589999999999993	-0.00964230246399086\\
0.59	-0.00964230246399095\\
0.594999999999993	-0.00968719012414359\\
0.595	-0.00968719012414363\\
0.599999999999993	-0.00969084781869305\\
0.6	-0.00969084781869302\\
0.604999999999993	-0.00965327851924945\\
0.608999999999993	-0.00959352026206663\\
0.609	-0.0095935202620665\\
0.609999999999993	-0.00957445170845045\\
0.61	-0.00957445170845031\\
0.610999999999997	-0.00955373040056255\\
0.611999999999993	-0.0095313556651646\\
0.613999999999986	-0.00948164295030244\\
0.617999999999972	-0.00936234080248655\\
0.619999999999993	-0.00929273586475684\\
0.62	-0.00929273586475658\\
0.627999999999972	-0.00894773586386479\\
0.629999999999993	-0.00884478475601719\\
0.63	-0.00884478475601681\\
0.637999999999972	-0.00839861878260049\\
0.637999999999993	-0.00839861878259923\\
0.638	-0.00839861878259882\\
0.639999999999993	-0.00828046686037032\\
0.64	-0.0082804668603699\\
0.641999999999993	-0.00815964057471369\\
0.643999999999986	-0.00803612422239629\\
0.647999999999971	-0.00778095675744499\\
0.649999999999993	-0.00764927248252216\\
0.65	-0.00764927248252169\\
0.657999999999971	-0.00709479492220257\\
0.659999999999993	-0.00694915034179599\\
0.66	-0.00694915034179547\\
0.664999999999993	-0.00658266984273853\\
0.665	-0.00658266984273801\\
0.666999999999993	-0.00643670081249966\\
0.667	-0.00643670081249914\\
0.668999999999993	-0.00629105939838003\\
0.669999999999993	-0.00621835563086651\\
0.67	-0.00621835563086599\\
0.671999999999993	-0.00607317016424685\\
0.673999999999986	-0.00592826506924562\\
0.677999999999972	-0.00563922070073368\\
0.679999999999993	-0.00549504386215189\\
0.68	-0.00549504386215137\\
0.687999999999972	-0.00492020190943284\\
0.689999999999993	-0.00477686438809373\\
0.69	-0.00477686438809322\\
0.695999999999993	-0.00434744880791587\\
0.696	-0.00434744880791536\\
0.699999999999993	-0.00406148324064083\\
0.7	-0.00406148324064033\\
0.703999999999993	-0.00378043682998459\\
0.707999999999986	-0.00350900689331351\\
0.709999999999993	-0.00337685381341326\\
0.71	-0.00337685381341279\\
0.717999999999986	-0.00287159023084518\\
0.719999999999993	-0.00275102943883562\\
0.72	-0.00275102943883519\\
0.724999999999993	-0.00245952629397335\\
0.725	-0.00245952629397294\\
0.729999999999993	-0.00218197628778351\\
0.730000000000001	-0.00218197628778313\\
0.734999999999994	-0.00191815396685727\\
0.735000000000001	-0.0019181539668569\\
0.739999999999994	-0.0016678450296479\\
0.740000000000001	-0.00166784502964756\\
0.744999999999994	-0.00143328760783166\\
0.749999999999987	-0.00121673262796016\\
0.750000000000001	-0.00121673262795958\\
0.753999999999993	-0.00105633170278198\\
0.754	-0.00105633170278171\\
0.757999999999993	-0.000907256099697013\\
0.759999999999993	-0.000836940845916116\\
0.76	-0.000836940845915871\\
0.763999999999993	-0.000704709741995797\\
0.767999999999986	-0.000583621164516536\\
0.77	-0.00052723542596818\\
0.770000000000007	-0.000527235425967985\\
0.777999999999993	-0.000329249340358925\\
0.779999999999993	-0.000286609872933135\\
0.78	-0.000286609872932989\\
0.782999999999993	-0.000227770046491974\\
0.783	-0.000227770046491842\\
0.785999999999993	-0.000175058309258881\\
0.788999999999986	-0.000128459247121661\\
0.79	-0.000114282192066528\\
0.790000000000007	-0.000114282192066429\\
0.795999999999993	-4.22451738027334e-05\\
0.8	-6.41055769183859e-06\\
0.800000000000007	-6.41055769178359e-06\\
0.804999999999993	2.47021951101581e-05\\
0.805	2.47021951101915e-05\\
0.809999999999987	4.06373850288279e-05\\
0.809999999999997	4.06373850288453e-05\\
0.810000000000007	4.06373850288627e-05\\
0.811999999999993	4.27650568151917e-05\\
0.812	4.27650568151949e-05\\
0.813999999999987	4.24668040450288e-05\\
0.815999999999973	3.97425879671313e-05\\
0.819999999999945	2.70145344093136e-05\\
0.819999999999987	2.70145344091291e-05\\
0.82	2.70145344090681e-05\\
0.827999999999944	-1.49686270216265e-05\\
0.829999999999993	-2.7600809865971e-05\\
0.830000000000001	-2.76008098660174e-05\\
0.837999999999945	-8.67243667958422e-05\\
0.839999999999993	-0.000103663567035473\\
0.84	-0.000103663567035535\\
0.840999999999993	-0.00011245835966093\\
0.841000000000001	-0.000112458359660993\\
0.841999999999997	-0.000121470325897288\\
0.842999999999993	-0.000130699758575071\\
0.844999999999986	-0.000149812229775815\\
0.848999999999972	-0.000190660437977041\\
0.849999999999993	-0.000201420928829159\\
0.85	-0.000201420928829236\\
0.857999999999972	-0.000295458030977637\\
0.859999999999993	-0.000321190591083767\\
0.86	-0.00032119059108386\\
0.867999999999972	-0.000424070723010823\\
0.869999999999993	-0.000449229809659529\\
0.87	-0.000449229809659618\\
0.874999999999994	-0.000511181751801168\\
0.875000000000001	-0.000511181751801255\\
0.879999999999994	-0.000571822715192518\\
0.880000000000001	-0.000571822715192603\\
0.884999999999994	-0.000631201958367439\\
0.889999999999987	-0.000689367714786955\\
0.890000000000001	-0.000689367714787112\\
0.890000000000008	-0.000689367714787194\\
0.899	-0.000791158298843613\\
0.899000000000008	-0.000791158298843692\\
0.9	-0.000802246811158136\\
0.900000000000007	-0.000802246811158215\\
0.901000000000004	-0.000813224446872723\\
0.902	-0.000824023665330358\\
0.903999999999993	-0.000845088248142637\\
0.907999999999979	-0.000885091457673038\\
0.909999999999993	-0.000904035283336731\\
0.910000000000001	-0.000904035283336797\\
0.917999999999972	-0.000972815107728738\\
0.919999999999993	-0.0009882723676427\\
0.920000000000001	-0.000988272367642754\\
0.927999999999972	-0.00104321089333297\\
0.927999999999994	-0.0010432108933331\\
0.928000000000001	-0.00104321089333314\\
0.929999999999994	-0.00105523182111481\\
0.930000000000001	-0.00105523182111485\\
0.931999999999994	-0.00106657052520826\\
0.933999999999987	-0.00107722847928501\\
0.937999999999972	-0.00109650758952085\\
0.939999999999993	-0.00110513125125947\\
0.940000000000001	-0.0011051312512595\\
0.944999999999994	-0.00112367898475253\\
0.945000000000001	-0.00112367898475255\\
0.949999999999994	-0.00113790357842256\\
0.950000000000001	-0.00113790357842258\\
0.954999999999994	-0.00114781658692478\\
0.956999999999994	-0.00115057646770512\\
0.957000000000001	-0.00115057646770513\\
0.959999999999993	-0.00115342606255304\\
0.960000000000001	-0.00115342606255305\\
0.962999999999993	-0.00115472809295829\\
0.965999999999986	-0.00115448293965585\\
0.969999999999993	-0.00115174914931009\\
0.970000000000001	-0.00115174914931008\\
0.975999999999986	-0.00114248734805354\\
0.979999999999993	-0.00113286738899141\\
0.980000000000001	-0.00113286738899139\\
0.985999999999986	-0.00111457699106654\\
0.985999999999993	-0.00111457699106651\\
0.986000000000001	-0.00111457699106649\\
0.989999999999993	-0.00110038722398059\\
0.990000000000001	-0.00110038722398057\\
0.993999999999993	-0.00108459203725712\\
0.997999999999986	-0.00106718321966508\\
1	-0.00105787090403666\\
};
\end{axis}
\end{tikzpicture}%
}
  \caption{The angular displacement of each pendulum as function of time.
    \texttt{Blue}: $P_1$, \texttt{Red}: $P_2$, \texttt{Orange}: $P_3$.
    $C_i = 10$ ms.}
  \label{fig:5.2}
\end{figure}

Figure \ref{fig:5.3} shows the schedule calculated for each pendulum with all
jobs having execution time $C_i = 10$ ms.  Figure \ref{fig:5.4} illustrates that
the schedule is not feasible by ploting the overall usage of the CPU over the
the length of a schedule period, which is at all times $100\%$.


\noindent\makebox[\textwidth][c]{%
\begin{minipage}{\linewidth}
  \begin{minipage}{0.45\linewidth}
    \begin{figure}[H]\centering
      \scalebox{0.7}{% This file was created by matlab2tikz.
%
%The latest updates can be retrieved from
%  http://www.mathworks.com/matlabcentral/fileexchange/22022-matlab2tikz-matlab2tikz
%where you can also make suggestions and rate matlab2tikz.
%
\definecolor{mycolor1}{rgb}{0.00000,0.44700,0.74100}%
\definecolor{mycolor2}{rgb}{0.92900,0.69400,0.12500}%
\definecolor{mycolor3}{rgb}{0.85000,0.32500,0.09800}%
%
\begin{tikzpicture}

\begin{axis}[%
width=4.133in,
height=0.863in,
at={(0.693in,2.837in)},
scale only axis,
xmin=0,
xmax=2000,
xmajorgrids,
ymin=0,
ymax=1,
ymajorgrids,
axis background/.style={fill=white}
]
\pgfplotsset{max space between ticks=50}
\addplot [color=mycolor1,solid,forget plot]
  table[row sep=crcr]{%
1	0\\
2	1\\
3	1\\
4	1\\
5	1\\
6	0\\
7	0\\
8	1\\
9	1\\
10	1\\
11	1\\
12	0\\
13	0\\
14	0\\
15	0\\
16	1\\
17	1\\
18	1\\
19	1\\
20	0\\
21	0\\
22	0\\
23	0\\
24	1\\
25	1\\
26	1\\
27	1\\
28	1\\
29	0\\
30	0\\
31	0\\
32	0\\
33	0\\
34	1\\
35	1\\
36	1\\
37	1\\
38	1\\
39	1\\
40	0\\
41	0\\
42	0\\
43	0\\
44	1\\
45	1\\
46	1\\
47	1\\
48	0\\
49	0\\
50	0\\
51	0\\
52	0\\
53	1\\
54	1\\
55	1\\
56	1\\
57	0\\
58	0\\
59	0\\
60	0\\
61	1\\
62	1\\
63	1\\
64	1\\
65	1\\
66	1\\
67	0\\
68	0\\
69	0\\
70	0\\
71	1\\
72	1\\
73	1\\
74	0\\
75	0\\
76	0\\
77	0\\
78	0\\
79	0\\
80	0\\
81	0\\
82	0\\
83	1\\
84	1\\
85	1\\
86	1\\
87	0\\
88	0\\
89	0\\
90	0\\
91	0\\
92	1\\
93	1\\
94	1\\
95	1\\
96	1\\
97	1\\
98	0\\
99	0\\
100	0\\
101	1\\
102	1\\
103	1\\
104	0\\
105	0\\
106	0\\
107	0\\
108	0\\
109	0\\
110	0\\
111	1\\
112	1\\
113	1\\
114	1\\
115	1\\
116	0\\
117	0\\
118	0\\
119	0\\
120	0\\
121	0\\
122	0\\
123	1\\
124	1\\
125	1\\
126	1\\
127	1\\
128	1\\
129	1\\
130	1\\
131	0\\
132	0\\
133	0\\
134	1\\
135	1\\
136	1\\
137	0\\
138	0\\
139	0\\
140	1\\
141	1\\
142	1\\
143	0\\
144	0\\
145	0\\
146	0\\
147	0\\
148	0\\
149	1\\
150	1\\
151	1\\
152	1\\
153	1\\
154	1\\
155	0\\
156	0\\
157	0\\
158	0\\
159	0\\
160	1\\
161	1\\
162	1\\
163	1\\
164	1\\
165	1\\
166	1\\
167	1\\
168	0\\
169	0\\
170	0\\
171	0\\
172	0\\
173	1\\
174	1\\
175	1\\
176	0\\
177	0\\
178	0\\
179	0\\
180	1\\
181	1\\
182	1\\
183	1\\
184	1\\
185	1\\
186	0\\
187	0\\
188	0\\
189	0\\
190	0\\
191	0\\
192	1\\
193	1\\
194	1\\
195	1\\
196	1\\
197	1\\
198	0\\
199	0\\
200	0\\
201	0\\
202	1\\
203	1\\
204	1\\
205	0\\
206	0\\
207	0\\
208	0\\
209	1\\
210	1\\
211	1\\
212	1\\
213	0\\
214	0\\
215	0\\
216	0\\
217	1\\
218	1\\
219	1\\
220	1\\
221	1\\
222	0\\
223	0\\
224	0\\
225	0\\
226	1\\
227	1\\
228	1\\
229	0\\
230	0\\
231	0\\
232	0\\
233	0\\
234	0\\
235	1\\
236	1\\
237	1\\
238	0\\
239	0\\
240	0\\
241	0\\
242	0\\
243	0\\
244	1\\
245	1\\
246	1\\
247	1\\
248	1\\
249	1\\
250	1\\
251	1\\
252	1\\
253	0\\
254	0\\
255	0\\
256	0\\
257	1\\
258	1\\
259	1\\
260	0\\
261	0\\
262	0\\
263	0\\
264	0\\
265	0\\
266	0\\
267	0\\
268	1\\
269	1\\
270	1\\
271	0\\
272	0\\
273	0\\
274	0\\
275	0\\
276	0\\
277	0\\
278	0\\
279	1\\
280	1\\
281	1\\
282	0\\
283	0\\
284	0\\
285	0\\
286	1\\
287	1\\
288	1\\
289	1\\
290	1\\
291	0\\
292	0\\
293	0\\
294	0\\
295	0\\
296	0\\
297	1\\
298	1\\
299	1\\
300	0\\
301	0\\
302	0\\
303	0\\
304	1\\
305	1\\
306	1\\
307	1\\
308	1\\
309	0\\
310	0\\
311	0\\
312	0\\
313	0\\
314	1\\
315	1\\
316	1\\
317	1\\
318	1\\
319	1\\
320	1\\
321	1\\
322	1\\
323	0\\
324	0\\
325	0\\
326	0\\
327	0\\
328	1\\
329	1\\
330	1\\
331	0\\
332	0\\
333	0\\
334	0\\
335	1\\
336	1\\
337	1\\
338	1\\
339	0\\
340	0\\
341	0\\
342	0\\
343	1\\
344	1\\
345	1\\
346	1\\
347	1\\
348	1\\
349	0\\
350	0\\
351	0\\
352	0\\
353	0\\
354	0\\
355	1\\
356	1\\
357	1\\
358	1\\
359	0\\
360	0\\
361	0\\
362	0\\
363	0\\
364	1\\
365	1\\
366	1\\
367	1\\
368	0\\
369	0\\
370	0\\
371	0\\
372	0\\
373	1\\
374	1\\
375	1\\
376	1\\
377	1\\
378	1\\
379	0\\
380	0\\
381	0\\
382	1\\
383	1\\
384	1\\
385	1\\
386	1\\
387	0\\
388	0\\
389	0\\
390	0\\
391	0\\
392	0\\
393	0\\
394	1\\
395	1\\
396	1\\
397	0\\
398	0\\
399	0\\
400	0\\
401	0\\
402	0\\
403	0\\
404	1\\
405	1\\
406	1\\
407	1\\
408	1\\
409	1\\
410	1\\
411	1\\
412	0\\
413	0\\
414	0\\
415	1\\
416	1\\
417	1\\
418	0\\
419	0\\
420	0\\
421	0\\
422	1\\
423	1\\
424	1\\
425	0\\
426	0\\
427	0\\
428	0\\
429	1\\
430	1\\
431	1\\
432	1\\
433	1\\
434	1\\
435	0\\
436	0\\
437	0\\
438	0\\
439	0\\
440	1\\
441	1\\
442	1\\
443	1\\
444	1\\
445	1\\
446	1\\
447	1\\
448	0\\
449	0\\
450	0\\
451	0\\
452	0\\
453	1\\
454	1\\
455	1\\
456	1\\
457	0\\
458	0\\
459	0\\
460	0\\
461	0\\
462	1\\
463	1\\
464	1\\
465	1\\
466	0\\
467	0\\
468	0\\
469	0\\
470	0\\
471	1\\
472	1\\
473	1\\
474	1\\
475	1\\
476	1\\
477	0\\
478	0\\
479	0\\
480	0\\
481	1\\
482	1\\
483	1\\
484	0\\
485	0\\
486	0\\
487	0\\
488	1\\
489	1\\
490	1\\
491	0\\
492	0\\
493	0\\
494	0\\
495	1\\
496	1\\
497	1\\
498	1\\
499	1\\
500	0\\
501	0\\
502	0\\
503	0\\
504	1\\
505	1\\
506	1\\
507	0\\
508	0\\
509	0\\
510	0\\
511	0\\
512	0\\
513	1\\
514	1\\
515	1\\
516	1\\
517	0\\
518	0\\
519	0\\
520	0\\
521	0\\
522	0\\
523	1\\
524	1\\
525	1\\
526	1\\
527	1\\
528	1\\
529	1\\
530	0\\
531	0\\
532	0\\
533	1\\
534	1\\
535	1\\
536	0\\
537	0\\
538	0\\
539	0\\
540	0\\
541	0\\
542	0\\
543	0\\
544	1\\
545	1\\
546	1\\
547	0\\
548	0\\
549	0\\
550	0\\
551	0\\
552	0\\
553	0\\
554	0\\
555	0\\
556	1\\
557	1\\
558	1\\
559	1\\
560	0\\
561	0\\
562	0\\
563	1\\
564	1\\
565	1\\
566	1\\
567	1\\
568	0\\
569	0\\
570	0\\
571	0\\
572	0\\
573	0\\
574	1\\
575	1\\
576	1\\
577	0\\
578	0\\
579	0\\
580	0\\
581	0\\
582	1\\
583	1\\
584	1\\
585	1\\
586	1\\
587	1\\
588	1\\
589	0\\
590	0\\
591	0\\
592	0\\
593	0\\
594	0\\
595	1\\
596	1\\
597	1\\
598	1\\
599	1\\
600	1\\
601	0\\
602	0\\
603	0\\
604	0\\
605	1\\
606	1\\
607	1\\
608	0\\
609	0\\
610	0\\
611	0\\
612	1\\
613	1\\
614	1\\
615	1\\
616	0\\
617	0\\
618	0\\
619	0\\
620	0\\
621	0\\
622	0\\
623	0\\
624	1\\
625	1\\
626	1\\
627	1\\
628	1\\
629	1\\
630	0\\
631	0\\
632	0\\
633	0\\
634	1\\
635	1\\
636	1\\
637	0\\
638	0\\
639	0\\
640	0\\
641	0\\
642	1\\
643	1\\
644	1\\
645	1\\
646	0\\
647	0\\
648	0\\
649	0\\
650	0\\
651	1\\
652	1\\
653	1\\
654	1\\
655	1\\
656	1\\
657	1\\
658	1\\
659	0\\
660	0\\
661	0\\
662	0\\
663	1\\
664	1\\
665	1\\
666	0\\
667	0\\
668	0\\
669	0\\
670	0\\
671	0\\
672	0\\
673	1\\
674	1\\
675	1\\
676	0\\
677	0\\
678	0\\
679	0\\
680	0\\
681	0\\
682	1\\
683	1\\
684	1\\
685	1\\
686	1\\
687	1\\
688	1\\
689	1\\
690	0\\
691	0\\
692	0\\
693	0\\
694	1\\
695	1\\
696	1\\
697	1\\
698	0\\
699	0\\
700	0\\
701	1\\
702	1\\
703	1\\
704	0\\
705	0\\
706	0\\
707	0\\
708	1\\
709	1\\
710	1\\
711	1\\
712	1\\
713	1\\
714	0\\
715	0\\
716	0\\
717	0\\
718	0\\
719	1\\
720	1\\
721	1\\
722	1\\
723	1\\
724	1\\
725	1\\
726	1\\
727	1\\
728	0\\
729	0\\
730	0\\
731	0\\
732	0\\
733	1\\
734	1\\
735	1\\
736	0\\
737	0\\
738	0\\
739	0\\
740	1\\
741	1\\
742	1\\
743	1\\
744	0\\
745	0\\
746	0\\
747	0\\
748	0\\
749	1\\
750	1\\
751	1\\
752	1\\
753	1\\
754	0\\
755	0\\
756	0\\
757	0\\
758	0\\
759	0\\
760	1\\
761	1\\
762	1\\
763	1\\
764	0\\
765	0\\
766	0\\
767	0\\
768	1\\
769	1\\
770	1\\
771	1\\
772	0\\
773	0\\
774	0\\
775	0\\
776	1\\
777	1\\
778	1\\
779	1\\
780	1\\
781	0\\
782	0\\
783	0\\
784	0\\
785	1\\
786	1\\
787	1\\
788	1\\
789	0\\
790	0\\
791	0\\
792	0\\
793	0\\
794	0\\
795	1\\
796	1\\
797	1\\
798	0\\
799	0\\
800	0\\
801	0\\
802	0\\
803	0\\
804	1\\
805	1\\
806	1\\
807	1\\
808	1\\
809	1\\
810	1\\
811	0\\
812	0\\
813	0\\
814	1\\
815	1\\
816	1\\
817	0\\
818	0\\
819	0\\
820	0\\
821	0\\
822	0\\
823	0\\
824	0\\
825	0\\
826	0\\
827	1\\
828	1\\
829	1\\
830	1\\
831	0\\
832	0\\
833	0\\
834	0\\
835	0\\
836	0\\
837	0\\
838	0\\
839	1\\
840	1\\
841	1\\
842	0\\
843	0\\
844	0\\
845	1\\
846	1\\
847	1\\
848	1\\
849	1\\
850	0\\
851	0\\
852	0\\
853	0\\
854	0\\
855	0\\
856	1\\
857	1\\
858	1\\
859	1\\
860	0\\
861	0\\
862	0\\
863	0\\
864	0\\
865	1\\
866	1\\
867	1\\
868	1\\
869	1\\
870	0\\
871	0\\
872	0\\
873	0\\
874	0\\
875	1\\
876	1\\
877	1\\
878	1\\
879	1\\
880	1\\
881	0\\
882	0\\
883	0\\
884	0\\
885	1\\
886	1\\
887	1\\
888	0\\
889	0\\
890	0\\
891	0\\
892	0\\
893	0\\
894	0\\
895	0\\
896	0\\
897	1\\
898	1\\
899	1\\
900	1\\
901	0\\
902	0\\
903	0\\
904	0\\
905	1\\
906	1\\
907	1\\
908	1\\
909	1\\
910	1\\
911	0\\
912	0\\
913	0\\
914	0\\
915	1\\
916	1\\
917	1\\
918	0\\
919	0\\
920	0\\
921	0\\
922	0\\
923	1\\
924	1\\
925	1\\
926	1\\
927	1\\
928	1\\
929	0\\
930	0\\
931	0\\
932	0\\
933	0\\
934	1\\
935	1\\
936	1\\
937	1\\
938	1\\
939	1\\
940	0\\
941	0\\
942	0\\
943	1\\
944	1\\
945	1\\
946	0\\
947	0\\
948	0\\
949	0\\
950	0\\
951	0\\
952	0\\
953	1\\
954	1\\
955	1\\
956	0\\
957	0\\
958	0\\
959	0\\
960	0\\
961	0\\
962	0\\
963	0\\
964	0\\
965	1\\
966	1\\
967	1\\
968	1\\
969	1\\
970	1\\
971	1\\
972	1\\
973	0\\
974	0\\
975	0\\
976	1\\
977	1\\
978	1\\
979	0\\
980	0\\
981	0\\
982	1\\
983	1\\
984	1\\
985	0\\
986	0\\
987	0\\
988	0\\
989	0\\
990	1\\
991	1\\
992	1\\
993	1\\
994	1\\
995	1\\
996	1\\
997	1\\
998	0\\
999	0\\
1000	0\\
1001	0\\
1002	0\\
1003	0\\
1004	1\\
1005	1\\
1006	1\\
1007	1\\
1008	1\\
1009	1\\
1010	1\\
1011	1\\
1012	0\\
1013	0\\
1014	0\\
1015	0\\
1016	0\\
1017	1\\
1018	1\\
1019	1\\
1020	0\\
1021	0\\
1022	0\\
1023	0\\
1024	1\\
1025	1\\
1026	1\\
1027	1\\
1028	0\\
1029	0\\
1030	0\\
1031	0\\
1032	0\\
1033	0\\
1034	0\\
1035	1\\
1036	1\\
1037	1\\
1038	1\\
1039	1\\
1040	1\\
1041	0\\
1042	0\\
1043	0\\
1044	0\\
1045	1\\
1046	1\\
1047	1\\
1048	0\\
1049	0\\
1050	0\\
1051	0\\
1052	1\\
1053	1\\
1054	1\\
1055	1\\
1056	0\\
1057	0\\
1058	0\\
1059	0\\
1060	1\\
1061	1\\
1062	1\\
1063	1\\
1064	1\\
1065	1\\
1066	1\\
1067	1\\
1068	1\\
1069	0\\
1070	0\\
1071	0\\
1072	0\\
1073	1\\
1074	1\\
1075	1\\
1076	0\\
1077	0\\
1078	0\\
1079	0\\
1080	0\\
1081	0\\
1082	1\\
1083	1\\
1084	1\\
1085	0\\
1086	0\\
1087	0\\
1088	0\\
1089	0\\
1090	0\\
1091	1\\
1092	1\\
1093	1\\
1094	1\\
1095	1\\
1096	1\\
1097	1\\
1098	0\\
1099	0\\
1100	0\\
1101	0\\
1102	1\\
1103	1\\
1104	1\\
1105	1\\
1106	0\\
1107	0\\
1108	0\\
1109	0\\
1110	0\\
1111	0\\
1112	0\\
1113	0\\
1114	1\\
1115	1\\
1116	1\\
1117	0\\
1118	0\\
1119	0\\
1120	0\\
1121	0\\
1122	0\\
1123	0\\
1124	0\\
1125	1\\
1126	1\\
1127	1\\
1128	0\\
1129	0\\
1130	0\\
1131	1\\
1132	1\\
1133	1\\
1134	1\\
1135	1\\
1136	1\\
1137	0\\
1138	0\\
1139	0\\
1140	0\\
1141	0\\
1142	0\\
1143	1\\
1144	1\\
1145	1\\
1146	0\\
1147	0\\
1148	0\\
1149	0\\
1150	0\\
1151	1\\
1152	1\\
1153	1\\
1154	1\\
1155	1\\
1156	0\\
1157	0\\
1158	0\\
1159	0\\
1160	0\\
1161	1\\
1162	1\\
1163	1\\
1164	1\\
1165	1\\
1166	1\\
1167	0\\
1168	0\\
1169	0\\
1170	0\\
1171	0\\
1172	0\\
1173	1\\
1174	1\\
1175	1\\
1176	1\\
1177	0\\
1178	0\\
1179	0\\
1180	0\\
1181	0\\
1182	1\\
1183	1\\
1184	1\\
1185	1\\
1186	0\\
1187	0\\
1188	0\\
1189	0\\
1190	1\\
1191	1\\
1192	1\\
1193	1\\
1194	1\\
1195	1\\
1196	0\\
1197	0\\
1198	0\\
1199	0\\
1200	1\\
1201	1\\
1202	1\\
1203	1\\
1204	0\\
1205	0\\
1206	0\\
1207	0\\
1208	0\\
1209	1\\
1210	1\\
1211	1\\
1212	1\\
1213	0\\
1214	0\\
1215	0\\
1216	0\\
1217	0\\
1218	1\\
1219	1\\
1220	1\\
1221	1\\
1222	1\\
1223	1\\
1224	0\\
1225	0\\
1226	0\\
1227	1\\
1228	1\\
1229	1\\
1230	0\\
1231	0\\
1232	0\\
1233	0\\
1234	0\\
1235	0\\
1236	0\\
1237	0\\
1238	0\\
1239	1\\
1240	1\\
1241	1\\
1242	1\\
1243	0\\
1244	0\\
1245	0\\
1246	0\\
1247	0\\
1248	0\\
1249	0\\
1250	1\\
1251	1\\
1252	1\\
1253	1\\
1254	1\\
1255	1\\
1256	1\\
1257	1\\
1258	0\\
1259	0\\
1260	0\\
1261	1\\
1262	1\\
1263	1\\
1264	0\\
1265	0\\
1266	0\\
1267	1\\
1268	1\\
1269	1\\
1270	1\\
1271	0\\
1272	0\\
1273	0\\
1274	0\\
1275	0\\
1276	1\\
1277	1\\
1278	1\\
1279	1\\
1280	1\\
1281	1\\
1282	0\\
1283	0\\
1284	0\\
1285	0\\
1286	0\\
1287	1\\
1288	1\\
1289	1\\
1290	1\\
1291	1\\
1292	1\\
1293	1\\
1294	0\\
1295	0\\
1296	0\\
1297	0\\
1298	0\\
1299	1\\
1300	1\\
1301	1\\
1302	0\\
1303	0\\
1304	0\\
1305	0\\
1306	0\\
1307	0\\
1308	0\\
1309	1\\
1310	1\\
1311	1\\
1312	1\\
1313	1\\
1314	0\\
1315	0\\
1316	0\\
1317	0\\
1318	1\\
1319	1\\
1320	1\\
1321	1\\
1322	1\\
1323	0\\
1324	0\\
1325	0\\
1326	0\\
1327	1\\
1328	1\\
1329	1\\
1330	0\\
1331	0\\
1332	0\\
1333	0\\
1334	0\\
1335	1\\
1336	1\\
1337	1\\
1338	1\\
1339	1\\
1340	1\\
1341	0\\
1342	0\\
1343	0\\
1344	0\\
1345	0\\
1346	1\\
1347	1\\
1348	1\\
1349	1\\
1350	1\\
1351	0\\
1352	0\\
1353	0\\
1354	0\\
1355	1\\
1356	1\\
1357	1\\
1358	0\\
1359	0\\
1360	0\\
1361	0\\
1362	0\\
1363	0\\
1364	1\\
1365	1\\
1366	1\\
1367	0\\
1368	0\\
1369	0\\
1370	0\\
1371	0\\
1372	0\\
1373	0\\
1374	0\\
1375	1\\
1376	1\\
1377	1\\
1378	1\\
1379	1\\
1380	1\\
1381	1\\
1382	0\\
1383	0\\
1384	0\\
1385	1\\
1386	1\\
1387	1\\
1388	0\\
1389	0\\
1390	0\\
1391	0\\
1392	0\\
1393	0\\
1394	0\\
1395	0\\
1396	1\\
1397	1\\
1398	1\\
1399	0\\
1400	0\\
1401	0\\
1402	0\\
1403	0\\
1404	0\\
1405	0\\
1406	0\\
1407	1\\
1408	1\\
1409	1\\
1410	1\\
1411	0\\
1412	0\\
1413	0\\
1414	1\\
1415	1\\
1416	1\\
1417	1\\
1418	0\\
1419	0\\
1420	0\\
1421	0\\
1422	0\\
1423	0\\
1424	1\\
1425	1\\
1426	1\\
1427	0\\
1428	0\\
1429	0\\
1430	0\\
1431	0\\
1432	1\\
1433	1\\
1434	1\\
1435	1\\
1436	1\\
1437	0\\
1438	0\\
1439	0\\
1440	0\\
1441	0\\
1442	0\\
1443	1\\
1444	1\\
1445	1\\
1446	1\\
1447	1\\
1448	1\\
1449	0\\
1450	0\\
1451	0\\
1452	0\\
1453	1\\
1454	1\\
1455	1\\
1456	0\\
1457	0\\
1458	0\\
1459	0\\
1460	0\\
1461	1\\
1462	1\\
1463	1\\
1464	1\\
1465	0\\
1466	0\\
1467	0\\
1468	0\\
1469	1\\
1470	1\\
1471	1\\
1472	1\\
1473	1\\
1474	1\\
1475	0\\
1476	0\\
1477	0\\
1478	0\\
1479	1\\
1480	1\\
1481	1\\
1482	0\\
1483	0\\
1484	0\\
1485	0\\
1486	0\\
1487	1\\
1488	1\\
1489	1\\
1490	1\\
1491	0\\
1492	0\\
1493	0\\
1494	0\\
1495	0\\
1496	1\\
1497	1\\
1498	1\\
1499	1\\
1500	1\\
1501	1\\
1502	0\\
1503	0\\
1504	0\\
1505	0\\
1506	1\\
1507	1\\
1508	1\\
1509	1\\
1510	0\\
1511	0\\
1512	0\\
1513	0\\
1514	0\\
1515	0\\
1516	0\\
1517	1\\
1518	1\\
1519	1\\
1520	0\\
1521	0\\
1522	0\\
1523	0\\
1524	0\\
1525	0\\
1526	0\\
1527	1\\
1528	1\\
1529	1\\
1530	1\\
1531	1\\
1532	1\\
1533	1\\
1534	1\\
1535	0\\
1536	0\\
1537	0\\
1538	1\\
1539	1\\
1540	1\\
1541	1\\
1542	0\\
1543	0\\
1544	0\\
1545	1\\
1546	1\\
1547	0\\
1548	0\\
1549	0\\
1550	0\\
1551	1\\
1552	1\\
1553	1\\
1554	1\\
1555	1\\
1556	1\\
1557	0\\
1558	0\\
1559	0\\
1560	0\\
1561	0\\
1562	1\\
1563	1\\
1564	1\\
1565	1\\
1566	1\\
1567	1\\
1568	1\\
1569	1\\
1570	0\\
1571	0\\
1572	0\\
1573	0\\
1574	0\\
1575	0\\
1576	0\\
1577	1\\
1578	1\\
1579	1\\
1580	1\\
1581	0\\
1582	0\\
1583	0\\
1584	0\\
1585	1\\
1586	1\\
1587	1\\
1588	1\\
1589	0\\
1590	0\\
1591	0\\
1592	0\\
1593	0\\
1594	1\\
1595	1\\
1596	1\\
1597	1\\
1598	1\\
1599	1\\
1600	0\\
1601	0\\
1602	0\\
1603	0\\
1604	1\\
1605	1\\
1606	1\\
1607	1\\
1608	0\\
1609	0\\
1610	0\\
1611	0\\
1612	1\\
1613	1\\
1614	1\\
1615	0\\
1616	0\\
1617	0\\
1618	0\\
1619	1\\
1620	1\\
1621	1\\
1622	1\\
1623	1\\
1624	0\\
1625	0\\
1626	0\\
1627	0\\
1628	1\\
1629	1\\
1630	1\\
1631	0\\
1632	0\\
1633	0\\
1634	0\\
1635	0\\
1636	0\\
1637	0\\
1638	0\\
1639	1\\
1640	1\\
1641	1\\
1642	1\\
1643	0\\
1644	0\\
1645	0\\
1646	0\\
1647	0\\
1648	0\\
1649	1\\
1650	1\\
1651	1\\
1652	1\\
1653	1\\
1654	1\\
1655	1\\
1656	0\\
1657	0\\
1658	0\\
1659	1\\
1660	1\\
1661	1\\
1662	0\\
1663	0\\
1664	0\\
1665	0\\
1666	0\\
1667	0\\
1668	0\\
1669	0\\
1670	1\\
1671	1\\
1672	1\\
1673	1\\
1674	1\\
1675	0\\
1676	0\\
1677	0\\
1678	0\\
1679	0\\
1680	0\\
1681	0\\
1682	0\\
1683	1\\
1684	1\\
1685	1\\
1686	0\\
1687	0\\
1688	0\\
1689	1\\
1690	1\\
1691	1\\
1692	1\\
1693	1\\
1694	0\\
1695	0\\
1696	0\\
1697	0\\
1698	0\\
1699	0\\
1700	1\\
1701	1\\
1702	1\\
1703	0\\
1704	0\\
1705	0\\
1706	0\\
1707	0\\
1708	0\\
1709	1\\
1710	1\\
1711	1\\
1712	1\\
1713	1\\
1714	0\\
1715	0\\
1716	0\\
1717	0\\
1718	0\\
1719	1\\
1720	1\\
1721	1\\
1722	1\\
1723	1\\
1724	1\\
1725	0\\
1726	0\\
1727	0\\
1728	0\\
1729	1\\
1730	1\\
1731	1\\
1732	0\\
1733	0\\
1734	0\\
1735	0\\
1736	1\\
1737	1\\
1738	1\\
1739	1\\
1740	1\\
1741	1\\
1742	0\\
1743	0\\
1744	0\\
1745	0\\
1746	0\\
1747	1\\
1748	1\\
1749	1\\
1750	1\\
1751	1\\
1752	1\\
1753	0\\
1754	0\\
1755	0\\
1756	0\\
1757	1\\
1758	1\\
1759	1\\
1760	0\\
1761	0\\
1762	0\\
1763	0\\
1764	0\\
1765	1\\
1766	1\\
1767	1\\
1768	1\\
1769	0\\
1770	0\\
1771	0\\
1772	0\\
1773	0\\
1774	0\\
1775	0\\
1776	0\\
1777	0\\
1778	1\\
1779	1\\
1780	1\\
1781	1\\
1782	1\\
1783	1\\
1784	0\\
1785	0\\
1786	0\\
1787	1\\
1788	1\\
1789	1\\
1790	0\\
1791	0\\
1792	0\\
1793	0\\
1794	0\\
1795	0\\
1796	1\\
1797	1\\
1798	1\\
1799	0\\
1800	0\\
1801	0\\
1802	0\\
1803	0\\
1804	0\\
1805	1\\
1806	1\\
1807	1\\
1808	1\\
1809	1\\
1810	1\\
1811	1\\
1812	1\\
1813	1\\
1814	1\\
1815	0\\
1816	0\\
1817	0\\
1818	0\\
1819	1\\
1820	1\\
1821	1\\
1822	0\\
1823	0\\
1824	0\\
1825	1\\
1826	1\\
1827	1\\
1828	0\\
1829	0\\
1830	0\\
1831	0\\
1832	0\\
1833	1\\
1834	1\\
1835	1\\
1836	1\\
1837	1\\
1838	1\\
1839	0\\
1840	0\\
1841	0\\
1842	0\\
1843	0\\
1844	0\\
1845	1\\
1846	1\\
1847	1\\
1848	1\\
1849	1\\
1850	1\\
1851	1\\
1852	1\\
1853	0\\
1854	0\\
1855	0\\
1856	0\\
1857	0\\
1858	1\\
1859	1\\
1860	1\\
1861	0\\
1862	0\\
1863	0\\
1864	0\\
1865	1\\
1866	1\\
1867	1\\
1868	1\\
1869	0\\
1870	0\\
1871	0\\
1872	0\\
1873	0\\
1874	1\\
1875	1\\
1876	1\\
1877	1\\
1878	1\\
1879	1\\
1880	1\\
1881	1\\
1882	1\\
1883	1\\
1884	0\\
1885	0\\
1886	0\\
1887	0\\
1888	1\\
1889	1\\
1890	1\\
1891	0\\
1892	0\\
1893	0\\
1894	0\\
1895	1\\
1896	1\\
1897	1\\
1898	0\\
1899	0\\
1900	0\\
1901	0\\
1902	1\\
1903	1\\
1904	1\\
1905	1\\
1906	1\\
1907	0\\
1908	0\\
1909	0\\
1910	0\\
1911	0\\
1912	0\\
1913	1\\
1914	1\\
1915	1\\
1916	1\\
1917	0\\
1918	0\\
1919	0\\
1920	0\\
1921	0\\
1922	0\\
1923	1\\
1924	1\\
1925	1\\
1926	0\\
1927	0\\
1928	0\\
1929	0\\
1930	0\\
1931	0\\
1932	1\\
1933	1\\
1934	1\\
1935	1\\
1936	1\\
1937	1\\
1938	1\\
1939	0\\
1940	0\\
1941	0\\
1942	1\\
1943	1\\
1944	1\\
1945	1\\
1946	0\\
1947	0\\
1948	0\\
1949	0\\
1950	0\\
1951	0\\
1952	0\\
1953	0\\
1954	1\\
1955	1\\
1956	1\\
1957	0\\
1958	0\\
1959	0\\
1960	0\\
1961	0\\
1962	0\\
1963	0\\
1964	0\\
1965	1\\
};
\end{axis}

\begin{axis}[%
width=4.133in,
height=0.863in,
at={(0.693in,0.44in)},
scale only axis,
xmin=0,
xmax=2000,
xmajorgrids,
ymin=0,
ymax=1,
ymajorgrids,
axis background/.style={fill=white}
]
\pgfplotsset{max space between ticks=50}
\addplot [color=mycolor2,solid,forget plot]
  table[row sep=crcr]{%
1	0\\
2	0.5\\
3	0.5\\
4	0.5\\
5	0.5\\
6	0.5\\
7	0.5\\
8	0.5\\
9	0.5\\
10	0.5\\
11	0.5\\
12	0.5\\
13	0.5\\
14	0.5\\
15	0.5\\
16	0.5\\
17	0.5\\
18	0.5\\
19	0.5\\
20	1\\
21	1\\
22	0.5\\
23	0.5\\
24	0.5\\
25	0.5\\
26	0.5\\
27	0.5\\
28	0.5\\
29	0.5\\
30	0.5\\
31	0.5\\
32	1\\
33	1\\
34	0.5\\
35	0.5\\
36	0.5\\
37	0.5\\
38	0.5\\
39	0.5\\
40	0.5\\
41	0.5\\
42	0.5\\
43	0.5\\
44	0.5\\
45	0.5\\
46	0.5\\
47	0.5\\
48	1\\
49	1\\
50	1\\
51	0.5\\
52	0.5\\
53	0.5\\
54	0.5\\
55	0.5\\
56	0.5\\
57	0.5\\
58	0.5\\
59	1\\
60	1\\
61	0.5\\
62	0.5\\
63	0.5\\
64	0.5\\
65	0.5\\
66	0.5\\
67	0.5\\
68	0.5\\
69	0.5\\
70	0.5\\
71	0.5\\
72	0.5\\
73	0.5\\
74	1\\
75	1\\
76	0.5\\
77	0.5\\
78	0.5\\
79	0.5\\
80	0.5\\
81	0.5\\
82	0.5\\
83	0.5\\
84	0.5\\
85	0.5\\
86	0.5\\
87	0.5\\
88	0.5\\
89	1\\
90	1\\
91	1\\
92	0.5\\
93	0.5\\
94	0.5\\
95	0.5\\
96	0.5\\
97	0.5\\
98	0.5\\
99	0.5\\
100	0.5\\
101	0.5\\
102	0.5\\
103	0.5\\
104	1\\
105	1\\
106	0.5\\
107	0.5\\
108	0.5\\
109	0.5\\
110	0.5\\
111	0.5\\
112	0.5\\
113	0.5\\
114	0.5\\
115	0.5\\
116	0.5\\
117	0.5\\
118	1\\
119	1\\
120	1\\
121	1\\
122	1\\
123	0.5\\
124	0.5\\
125	0.5\\
126	0.5\\
127	0.5\\
128	0.5\\
129	0.5\\
130	0.5\\
131	0.5\\
132	0.5\\
133	0.5\\
134	0.5\\
135	0.5\\
136	0.5\\
137	0.5\\
138	0.5\\
139	0.5\\
140	0.5\\
141	0.5\\
142	0.5\\
143	1\\
144	1\\
145	1\\
146	1\\
147	0.5\\
148	0.5\\
149	0.5\\
150	0.5\\
151	0.5\\
152	0.5\\
153	0.5\\
154	0.5\\
155	0.5\\
156	0.5\\
157	0.5\\
158	1\\
159	1\\
160	0.5\\
161	0.5\\
162	0.5\\
163	0.5\\
164	0.5\\
165	0.5\\
166	0.5\\
167	0.5\\
168	0.5\\
169	0.5\\
170	0.5\\
171	0.5\\
172	0.5\\
173	0.5\\
174	0.5\\
175	0.5\\
176	1\\
177	1\\
178	0.5\\
179	0.5\\
180	0.5\\
181	0.5\\
182	0.5\\
183	0.5\\
184	0.5\\
185	0.5\\
186	0.5\\
187	0.5\\
188	0.5\\
189	0.5\\
190	1\\
191	1\\
192	0.5\\
193	0.5\\
194	0.5\\
195	0.5\\
196	0.5\\
197	0.5\\
198	0.5\\
199	0.5\\
200	0.5\\
201	0.5\\
202	0.5\\
203	0.5\\
204	0.5\\
205	1\\
206	1\\
207	0.5\\
208	0.5\\
209	0.5\\
210	0.5\\
211	0.5\\
212	0.5\\
213	0.5\\
214	0.5\\
215	1\\
216	1\\
217	0.5\\
218	0.5\\
219	0.5\\
220	0.5\\
221	0.5\\
222	0.5\\
223	0.5\\
224	0.5\\
225	0.5\\
226	0.5\\
227	0.5\\
228	0.5\\
229	1\\
230	1\\
231	0.5\\
232	0.5\\
233	0.5\\
234	0.5\\
235	0.5\\
236	0.5\\
237	0.5\\
238	0.5\\
239	0.5\\
240	1\\
241	1\\
242	1\\
243	1\\
244	0.5\\
245	0.5\\
246	0.5\\
247	0.5\\
248	0.5\\
249	0.5\\
250	0.5\\
251	0.5\\
252	0.5\\
253	0.5\\
254	0.5\\
255	0.5\\
256	0.5\\
257	0.5\\
258	0.5\\
259	0.5\\
260	1\\
261	1\\
262	0.5\\
263	0.5\\
264	0.5\\
265	0.5\\
266	0.5\\
267	0.5\\
268	0.5\\
269	0.5\\
270	0.5\\
271	0.5\\
272	0.5\\
273	1\\
274	1\\
275	1\\
276	1\\
277	1\\
278	1\\
279	0.5\\
280	0.5\\
281	0.5\\
282	0.5\\
283	0.5\\
284	0.5\\
285	0.5\\
286	0.5\\
287	0.5\\
288	0.5\\
289	0.5\\
290	0.5\\
291	0.5\\
292	0.5\\
293	0.5\\
294	0.5\\
295	0.5\\
296	0.5\\
297	0.5\\
298	0.5\\
299	0.5\\
300	1\\
301	1\\
302	0.5\\
303	0.5\\
304	0.5\\
305	0.5\\
306	0.5\\
307	0.5\\
308	0.5\\
309	0.5\\
310	0.5\\
311	0.5\\
312	1\\
313	1\\
314	0.5\\
315	0.5\\
316	0.5\\
317	0.5\\
318	0.5\\
319	0.5\\
320	0.5\\
321	0.5\\
322	0.5\\
323	0.5\\
324	0.5\\
325	0.5\\
326	0.5\\
327	0.5\\
328	0.5\\
329	0.5\\
330	0.5\\
331	1\\
332	1\\
333	0.5\\
334	0.5\\
335	0.5\\
336	0.5\\
337	0.5\\
338	0.5\\
339	0.5\\
340	0.5\\
341	1\\
342	1\\
343	0.5\\
344	0.5\\
345	0.5\\
346	0.5\\
347	0.5\\
348	0.5\\
349	0.5\\
350	0.5\\
351	0.5\\
352	0.5\\
353	0.5\\
354	0.5\\
355	0.5\\
356	0.5\\
357	0.5\\
358	0.5\\
359	1\\
360	1\\
361	0.5\\
362	0.5\\
363	0.5\\
364	0.5\\
365	0.5\\
366	0.5\\
367	0.5\\
368	0.5\\
369	0.5\\
370	1\\
371	1\\
372	1\\
373	0.5\\
374	0.5\\
375	0.5\\
376	0.5\\
377	0.5\\
378	0.5\\
379	0.5\\
380	0.5\\
381	0.5\\
382	0.5\\
383	0.5\\
384	0.5\\
385	0.5\\
386	0.5\\
387	1\\
388	1\\
389	0.5\\
390	0.5\\
391	0.5\\
392	0.5\\
393	0.5\\
394	0.5\\
395	0.5\\
396	0.5\\
397	0.5\\
398	0.5\\
399	1\\
400	1\\
401	1\\
402	1\\
403	1\\
404	0.5\\
405	0.5\\
406	0.5\\
407	0.5\\
408	0.5\\
409	0.5\\
410	0.5\\
411	0.5\\
412	0.5\\
413	0.5\\
414	0.5\\
415	0.5\\
416	0.5\\
417	0.5\\
418	0.5\\
419	0.5\\
420	0.5\\
421	0.5\\
422	0.5\\
423	0.5\\
424	0.5\\
425	1\\
426	1\\
427	0.5\\
428	0.5\\
429	0.5\\
430	0.5\\
431	0.5\\
432	0.5\\
433	0.5\\
434	0.5\\
435	0.5\\
436	0.5\\
437	0.5\\
438	1\\
439	1\\
440	0.5\\
441	0.5\\
442	0.5\\
443	0.5\\
444	0.5\\
445	0.5\\
446	0.5\\
447	0.5\\
448	0.5\\
449	0.5\\
450	0.5\\
451	0.5\\
452	0.5\\
453	0.5\\
454	0.5\\
455	0.5\\
456	0.5\\
457	1\\
458	1\\
459	1\\
460	0.5\\
461	0.5\\
462	0.5\\
463	0.5\\
464	0.5\\
465	0.5\\
466	0.5\\
467	0.5\\
468	0.5\\
469	1\\
470	1\\
471	0.5\\
472	0.5\\
473	0.5\\
474	0.5\\
475	0.5\\
476	0.5\\
477	0.5\\
478	0.5\\
479	0.5\\
480	0.5\\
481	0.5\\
482	0.5\\
483	0.5\\
484	1\\
485	1\\
486	0.5\\
487	0.5\\
488	0.5\\
489	0.5\\
490	0.5\\
491	0.5\\
492	0.5\\
493	1\\
494	1\\
495	0.5\\
496	0.5\\
497	0.5\\
498	0.5\\
499	0.5\\
500	0.5\\
501	0.5\\
502	0.5\\
503	0.5\\
504	0.5\\
505	0.5\\
506	0.5\\
507	1\\
508	1\\
509	0.5\\
510	0.5\\
511	0.5\\
512	0.5\\
513	0.5\\
514	0.5\\
515	0.5\\
516	0.5\\
517	0.5\\
518	0.5\\
519	1\\
520	1\\
521	1\\
522	1\\
523	0.5\\
524	0.5\\
525	0.5\\
526	0.5\\
527	0.5\\
528	0.5\\
529	0.5\\
530	0.5\\
531	0.5\\
532	0.5\\
533	0.5\\
534	0.5\\
535	0.5\\
536	1\\
537	1\\
538	0.5\\
539	0.5\\
540	0.5\\
541	0.5\\
542	0.5\\
543	0.5\\
544	0.5\\
545	0.5\\
546	0.5\\
547	0.5\\
548	0.5\\
549	1\\
550	1\\
551	1\\
552	1\\
553	1\\
554	1\\
555	1\\
556	0.5\\
557	0.5\\
558	0.5\\
559	0.5\\
560	0.5\\
561	0.5\\
562	0.5\\
563	0.5\\
564	0.5\\
565	0.5\\
566	0.5\\
567	0.5\\
568	0.5\\
569	0.5\\
570	0.5\\
571	0.5\\
572	0.5\\
573	0.5\\
574	0.5\\
575	0.5\\
576	0.5\\
577	1\\
578	1\\
579	1\\
580	0.5\\
581	0.5\\
582	0.5\\
583	0.5\\
584	0.5\\
585	0.5\\
586	0.5\\
587	0.5\\
588	0.5\\
589	0.5\\
590	0.5\\
591	0.5\\
592	0.5\\
593	1\\
594	1\\
595	0.5\\
596	0.5\\
597	0.5\\
598	0.5\\
599	0.5\\
600	0.5\\
601	0.5\\
602	0.5\\
603	0.5\\
604	0.5\\
605	0.5\\
606	0.5\\
607	0.5\\
608	1\\
609	1\\
610	0.5\\
611	0.5\\
612	0.5\\
613	0.5\\
614	0.5\\
615	0.5\\
616	0.5\\
617	0.5\\
618	0.5\\
619	0.5\\
620	1\\
621	1\\
622	1\\
623	1\\
624	0.5\\
625	0.5\\
626	0.5\\
627	0.5\\
628	0.5\\
629	0.5\\
630	0.5\\
631	0.5\\
632	0.5\\
633	0.5\\
634	0.5\\
635	0.5\\
636	0.5\\
637	1\\
638	1\\
639	0.5\\
640	0.5\\
641	0.5\\
642	0.5\\
643	0.5\\
644	0.5\\
645	0.5\\
646	0.5\\
647	0.5\\
648	1\\
649	1\\
650	1\\
651	0.5\\
652	0.5\\
653	0.5\\
654	0.5\\
655	0.5\\
656	0.5\\
657	0.5\\
658	0.5\\
659	0.5\\
660	0.5\\
661	0.5\\
662	0.5\\
663	0.5\\
664	0.5\\
665	0.5\\
666	1\\
667	1\\
668	0.5\\
669	0.5\\
670	0.5\\
671	0.5\\
672	0.5\\
673	0.5\\
674	0.5\\
675	0.5\\
676	0.5\\
677	0.5\\
678	1\\
679	1\\
680	1\\
681	1\\
682	0.5\\
683	0.5\\
684	0.5\\
685	0.5\\
686	0.5\\
687	0.5\\
688	0.5\\
689	0.5\\
690	0.5\\
691	0.5\\
692	0.5\\
693	0.5\\
694	0.5\\
695	0.5\\
696	0.5\\
697	0.5\\
698	0.5\\
699	0.5\\
700	0.5\\
701	0.5\\
702	0.5\\
703	0.5\\
704	1\\
705	1\\
706	0.5\\
707	0.5\\
708	0.5\\
709	0.5\\
710	0.5\\
711	0.5\\
712	0.5\\
713	0.5\\
714	0.5\\
715	0.5\\
716	0.5\\
717	1\\
718	1\\
719	0.5\\
720	0.5\\
721	0.5\\
722	0.5\\
723	0.5\\
724	0.5\\
725	0.5\\
726	0.5\\
727	0.5\\
728	0.5\\
729	0.5\\
730	0.5\\
731	0.5\\
732	0.5\\
733	0.5\\
734	0.5\\
735	0.5\\
736	1\\
737	1\\
738	0.5\\
739	0.5\\
740	0.5\\
741	0.5\\
742	0.5\\
743	0.5\\
744	0.5\\
745	0.5\\
746	0.5\\
747	1\\
748	1\\
749	0.5\\
750	0.5\\
751	0.5\\
752	0.5\\
753	0.5\\
754	0.5\\
755	0.5\\
756	0.5\\
757	0.5\\
758	0.5\\
759	0.5\\
760	0.5\\
761	0.5\\
762	0.5\\
763	0.5\\
764	1\\
765	1\\
766	0.5\\
767	0.5\\
768	0.5\\
769	0.5\\
770	0.5\\
771	0.5\\
772	0.5\\
773	0.5\\
774	1\\
775	1\\
776	0.5\\
777	0.5\\
778	0.5\\
779	0.5\\
780	0.5\\
781	0.5\\
782	0.5\\
783	0.5\\
784	0.5\\
785	0.5\\
786	0.5\\
787	0.5\\
788	0.5\\
789	1\\
790	1\\
791	0.5\\
792	0.5\\
793	0.5\\
794	0.5\\
795	0.5\\
796	0.5\\
797	0.5\\
798	0.5\\
799	0.5\\
800	1\\
801	1\\
802	1\\
803	1\\
804	0.5\\
805	0.5\\
806	0.5\\
807	0.5\\
808	0.5\\
809	0.5\\
810	0.5\\
811	0.5\\
812	0.5\\
813	0.5\\
814	0.5\\
815	0.5\\
816	0.5\\
817	1\\
818	1\\
819	0.5\\
820	0.5\\
821	0.5\\
822	0.5\\
823	0.5\\
824	0.5\\
825	0.5\\
826	0.5\\
827	0.5\\
828	0.5\\
829	0.5\\
830	0.5\\
831	0.5\\
832	0.5\\
833	1\\
834	1\\
835	1\\
836	1\\
837	1\\
838	1\\
839	0.5\\
840	0.5\\
841	0.5\\
842	0.5\\
843	0.5\\
844	0.5\\
845	0.5\\
846	0.5\\
847	0.5\\
848	0.5\\
849	0.5\\
850	0.5\\
851	0.5\\
852	0.5\\
853	0.5\\
854	0.5\\
855	0.5\\
856	0.5\\
857	0.5\\
858	0.5\\
859	0.5\\
860	1\\
861	1\\
862	1\\
863	0.5\\
864	0.5\\
865	0.5\\
866	0.5\\
867	0.5\\
868	0.5\\
869	0.5\\
870	0.5\\
871	0.5\\
872	0.5\\
873	1\\
874	1\\
875	0.5\\
876	0.5\\
877	0.5\\
878	0.5\\
879	0.5\\
880	0.5\\
881	0.5\\
882	0.5\\
883	0.5\\
884	0.5\\
885	0.5\\
886	0.5\\
887	0.5\\
888	1\\
889	1\\
890	1\\
891	1\\
892	0.5\\
893	0.5\\
894	0.5\\
895	0.5\\
896	0.5\\
897	0.5\\
898	0.5\\
899	0.5\\
900	0.5\\
901	0.5\\
902	0.5\\
903	1\\
904	1\\
905	0.5\\
906	0.5\\
907	0.5\\
908	0.5\\
909	0.5\\
910	0.5\\
911	0.5\\
912	0.5\\
913	0.5\\
914	0.5\\
915	0.5\\
916	0.5\\
917	0.5\\
918	1\\
919	1\\
920	0.5\\
921	0.5\\
922	0.5\\
923	0.5\\
924	0.5\\
925	0.5\\
926	0.5\\
927	0.5\\
928	0.5\\
929	0.5\\
930	0.5\\
931	1\\
932	1\\
933	1\\
934	0.5\\
935	0.5\\
936	0.5\\
937	0.5\\
938	0.5\\
939	0.5\\
940	0.5\\
941	0.5\\
942	0.5\\
943	0.5\\
944	0.5\\
945	0.5\\
946	1\\
947	1\\
948	0.5\\
949	0.5\\
950	0.5\\
951	0.5\\
952	0.5\\
953	0.5\\
954	0.5\\
955	0.5\\
956	0.5\\
957	0.5\\
958	1\\
959	1\\
960	1\\
961	1\\
962	1\\
963	1\\
964	1\\
965	0.5\\
966	0.5\\
967	0.5\\
968	0.5\\
969	0.5\\
970	0.5\\
971	0.5\\
972	0.5\\
973	0.5\\
974	0.5\\
975	0.5\\
976	0.5\\
977	0.5\\
978	0.5\\
979	0.5\\
980	0.5\\
981	0.5\\
982	0.5\\
983	0.5\\
984	0.5\\
985	1\\
986	1\\
987	1\\
988	0.5\\
989	0.5\\
990	0.5\\
991	0.5\\
992	0.5\\
993	0.5\\
994	0.5\\
995	0.5\\
996	0.5\\
997	0.5\\
998	0.5\\
999	0.5\\
1000	0.5\\
1001	0.5\\
1002	1\\
1003	1\\
1004	0.5\\
1005	0.5\\
1006	0.5\\
1007	0.5\\
1008	0.5\\
1009	0.5\\
1010	0.5\\
1011	0.5\\
1012	0.5\\
1013	0.5\\
1014	0.5\\
1015	0.5\\
1016	0.5\\
1017	0.5\\
1018	0.5\\
1019	0.5\\
1020	1\\
1021	1\\
1022	0.5\\
1023	0.5\\
1024	0.5\\
1025	0.5\\
1026	0.5\\
1027	0.5\\
1028	0.5\\
1029	0.5\\
1030	0.5\\
1031	0.5\\
1032	1\\
1033	1\\
1034	1\\
1035	0.5\\
1036	0.5\\
1037	0.5\\
1038	0.5\\
1039	0.5\\
1040	0.5\\
1041	0.5\\
1042	0.5\\
1043	0.5\\
1044	0.5\\
1045	0.5\\
1046	0.5\\
1047	0.5\\
1048	1\\
1049	1\\
1050	0.5\\
1051	0.5\\
1052	0.5\\
1053	0.5\\
1054	0.5\\
1055	0.5\\
1056	0.5\\
1057	0.5\\
1058	1\\
1059	1\\
1060	0.5\\
1061	0.5\\
1062	0.5\\
1063	0.5\\
1064	0.5\\
1065	0.5\\
1066	0.5\\
1067	0.5\\
1068	0.5\\
1069	0.5\\
1070	0.5\\
1071	0.5\\
1072	0.5\\
1073	0.5\\
1074	0.5\\
1075	0.5\\
1076	1\\
1077	1\\
1078	0.5\\
1079	0.5\\
1080	0.5\\
1081	0.5\\
1082	0.5\\
1083	0.5\\
1084	0.5\\
1085	0.5\\
1086	0.5\\
1087	1\\
1088	1\\
1089	1\\
1090	1\\
1091	0.5\\
1092	0.5\\
1093	0.5\\
1094	0.5\\
1095	0.5\\
1096	0.5\\
1097	0.5\\
1098	0.5\\
1099	0.5\\
1100	0.5\\
1101	0.5\\
1102	0.5\\
1103	0.5\\
1104	0.5\\
1105	0.5\\
1106	1\\
1107	1\\
1108	0.5\\
1109	0.5\\
1110	0.5\\
1111	0.5\\
1112	0.5\\
1113	0.5\\
1114	0.5\\
1115	0.5\\
1116	0.5\\
1117	0.5\\
1118	0.5\\
1119	1\\
1120	1\\
1121	1\\
1122	1\\
1123	1\\
1124	1\\
1125	0.5\\
1126	0.5\\
1127	0.5\\
1128	0.5\\
1129	0.5\\
1130	0.5\\
1131	0.5\\
1132	0.5\\
1133	0.5\\
1134	0.5\\
1135	0.5\\
1136	0.5\\
1137	0.5\\
1138	0.5\\
1139	0.5\\
1140	0.5\\
1141	0.5\\
1142	0.5\\
1143	0.5\\
1144	0.5\\
1145	0.5\\
1146	1\\
1147	1\\
1148	1\\
1149	0.5\\
1150	0.5\\
1151	0.5\\
1152	0.5\\
1153	0.5\\
1154	0.5\\
1155	0.5\\
1156	0.5\\
1157	0.5\\
1158	0.5\\
1159	1\\
1160	1\\
1161	0.5\\
1162	0.5\\
1163	0.5\\
1164	0.5\\
1165	0.5\\
1166	0.5\\
1167	0.5\\
1168	0.5\\
1169	0.5\\
1170	0.5\\
1171	0.5\\
1172	0.5\\
1173	0.5\\
1174	0.5\\
1175	0.5\\
1176	0.5\\
1177	1\\
1178	1\\
1179	1\\
1180	0.5\\
1181	0.5\\
1182	0.5\\
1183	0.5\\
1184	0.5\\
1185	0.5\\
1186	0.5\\
1187	0.5\\
1188	1\\
1189	1\\
1190	0.5\\
1191	0.5\\
1192	0.5\\
1193	0.5\\
1194	0.5\\
1195	0.5\\
1196	0.5\\
1197	0.5\\
1198	0.5\\
1199	0.5\\
1200	0.5\\
1201	0.5\\
1202	0.5\\
1203	0.5\\
1204	1\\
1205	1\\
1206	0.5\\
1207	0.5\\
1208	0.5\\
1209	0.5\\
1210	0.5\\
1211	0.5\\
1212	0.5\\
1213	0.5\\
1214	0.5\\
1215	1\\
1216	1\\
1217	1\\
1218	0.5\\
1219	0.5\\
1220	0.5\\
1221	0.5\\
1222	0.5\\
1223	0.5\\
1224	0.5\\
1225	0.5\\
1226	0.5\\
1227	0.5\\
1228	0.5\\
1229	0.5\\
1230	1\\
1231	1\\
1232	0.5\\
1233	0.5\\
1234	0.5\\
1235	0.5\\
1236	0.5\\
1237	0.5\\
1238	0.5\\
1239	0.5\\
1240	0.5\\
1241	0.5\\
1242	0.5\\
1243	0.5\\
1244	0.5\\
1245	1\\
1246	1\\
1247	1\\
1248	1\\
1249	1\\
1250	0.5\\
1251	0.5\\
1252	0.5\\
1253	0.5\\
1254	0.5\\
1255	0.5\\
1256	0.5\\
1257	0.5\\
1258	0.5\\
1259	0.5\\
1260	0.5\\
1261	0.5\\
1262	0.5\\
1263	0.5\\
1264	0.5\\
1265	0.5\\
1266	0.5\\
1267	0.5\\
1268	0.5\\
1269	0.5\\
1270	0.5\\
1271	1\\
1272	1\\
1273	1\\
1274	0.5\\
1275	0.5\\
1276	0.5\\
1277	0.5\\
1278	0.5\\
1279	0.5\\
1280	0.5\\
1281	0.5\\
1282	0.5\\
1283	0.5\\
1284	0.5\\
1285	1\\
1286	1\\
1287	0.5\\
1288	0.5\\
1289	0.5\\
1290	0.5\\
1291	0.5\\
1292	0.5\\
1293	0.5\\
1294	0.5\\
1295	0.5\\
1296	0.5\\
1297	0.5\\
1298	0.5\\
1299	0.5\\
1300	0.5\\
1301	0.5\\
1302	1\\
1303	1\\
1304	1\\
1305	1\\
1306	0.5\\
1307	0.5\\
1308	0.5\\
1309	0.5\\
1310	0.5\\
1311	0.5\\
1312	0.5\\
1313	0.5\\
1314	0.5\\
1315	0.5\\
1316	1\\
1317	1\\
1318	0.5\\
1319	0.5\\
1320	0.5\\
1321	0.5\\
1322	0.5\\
1323	0.5\\
1324	0.5\\
1325	0.5\\
1326	0.5\\
1327	0.5\\
1328	0.5\\
1329	0.5\\
1330	1\\
1331	1\\
1332	0.5\\
1333	0.5\\
1334	0.5\\
1335	0.5\\
1336	0.5\\
1337	0.5\\
1338	0.5\\
1339	0.5\\
1340	0.5\\
1341	0.5\\
1342	0.5\\
1343	1\\
1344	1\\
1345	1\\
1346	0.5\\
1347	0.5\\
1348	0.5\\
1349	0.5\\
1350	0.5\\
1351	0.5\\
1352	0.5\\
1353	0.5\\
1354	0.5\\
1355	0.5\\
1356	0.5\\
1357	0.5\\
1358	1\\
1359	1\\
1360	0.5\\
1361	0.5\\
1362	0.5\\
1363	0.5\\
1364	0.5\\
1365	0.5\\
1366	0.5\\
1367	0.5\\
1368	0.5\\
1369	1\\
1370	1\\
1371	1\\
1372	1\\
1373	1\\
1374	1\\
1375	0.5\\
1376	0.5\\
1377	0.5\\
1378	0.5\\
1379	0.5\\
1380	0.5\\
1381	0.5\\
1382	0.5\\
1383	0.5\\
1384	0.5\\
1385	0.5\\
1386	0.5\\
1387	0.5\\
1388	1\\
1389	1\\
1390	0.5\\
1391	0.5\\
1392	0.5\\
1393	0.5\\
1394	0.5\\
1395	0.5\\
1396	0.5\\
1397	0.5\\
1398	0.5\\
1399	0.5\\
1400	0.5\\
1401	1\\
1402	1\\
1403	1\\
1404	1\\
1405	1\\
1406	1\\
1407	0.5\\
1408	0.5\\
1409	0.5\\
1410	0.5\\
1411	0.5\\
1412	0.5\\
1413	0.5\\
1414	0.5\\
1415	0.5\\
1416	0.5\\
1417	0.5\\
1418	0.5\\
1419	0.5\\
1420	0.5\\
1421	0.5\\
1422	0.5\\
1423	0.5\\
1424	0.5\\
1425	0.5\\
1426	0.5\\
1427	1\\
1428	1\\
1429	1\\
1430	0.5\\
1431	0.5\\
1432	0.5\\
1433	0.5\\
1434	0.5\\
1435	0.5\\
1436	0.5\\
1437	0.5\\
1438	0.5\\
1439	0.5\\
1440	0.5\\
1441	1\\
1442	1\\
1443	0.5\\
1444	0.5\\
1445	0.5\\
1446	0.5\\
1447	0.5\\
1448	0.5\\
1449	0.5\\
1450	0.5\\
1451	0.5\\
1452	0.5\\
1453	0.5\\
1454	0.5\\
1455	0.5\\
1456	1\\
1457	1\\
1458	1\\
1459	0.5\\
1460	0.5\\
1461	0.5\\
1462	0.5\\
1463	0.5\\
1464	0.5\\
1465	0.5\\
1466	0.5\\
1467	1\\
1468	1\\
1469	0.5\\
1470	0.5\\
1471	0.5\\
1472	0.5\\
1473	0.5\\
1474	0.5\\
1475	0.5\\
1476	0.5\\
1477	0.5\\
1478	0.5\\
1479	0.5\\
1480	0.5\\
1481	0.5\\
1482	1\\
1483	1\\
1484	0.5\\
1485	0.5\\
1486	0.5\\
1487	0.5\\
1488	0.5\\
1489	0.5\\
1490	0.5\\
1491	0.5\\
1492	0.5\\
1493	1\\
1494	1\\
1495	1\\
1496	0.5\\
1497	0.5\\
1498	0.5\\
1499	0.5\\
1500	0.5\\
1501	0.5\\
1502	0.5\\
1503	0.5\\
1504	0.5\\
1505	0.5\\
1506	0.5\\
1507	0.5\\
1508	0.5\\
1509	0.5\\
1510	1\\
1511	1\\
1512	0.5\\
1513	0.5\\
1514	0.5\\
1515	0.5\\
1516	0.5\\
1517	0.5\\
1518	0.5\\
1519	0.5\\
1520	0.5\\
1521	0.5\\
1522	1\\
1523	1\\
1524	1\\
1525	1\\
1526	1\\
1527	0.5\\
1528	0.5\\
1529	0.5\\
1530	0.5\\
1531	0.5\\
1532	0.5\\
1533	0.5\\
1534	0.5\\
1535	0.5\\
1536	0.5\\
1537	0.5\\
1538	0.5\\
1539	0.5\\
1540	0.5\\
1541	0.5\\
1542	0.5\\
1543	0.5\\
1544	0.5\\
1545	0.5\\
1546	0.5\\
1547	1\\
1548	1\\
1549	0.5\\
1550	0.5\\
1551	0.5\\
1552	0.5\\
1553	0.5\\
1554	0.5\\
1555	0.5\\
1556	0.5\\
1557	0.5\\
1558	0.5\\
1559	0.5\\
1560	1\\
1561	1\\
1562	0.5\\
1563	0.5\\
1564	0.5\\
1565	0.5\\
1566	0.5\\
1567	0.5\\
1568	0.5\\
1569	0.5\\
1570	0.5\\
1571	0.5\\
1572	0.5\\
1573	0.5\\
1574	0.5\\
1575	0.5\\
1576	0.5\\
1577	0.5\\
1578	0.5\\
1579	0.5\\
1580	0.5\\
1581	1\\
1582	1\\
1583	0.5\\
1584	0.5\\
1585	0.5\\
1586	0.5\\
1587	0.5\\
1588	0.5\\
1589	0.5\\
1590	0.5\\
1591	0.5\\
1592	1\\
1593	1\\
1594	0.5\\
1595	0.5\\
1596	0.5\\
1597	0.5\\
1598	0.5\\
1599	0.5\\
1600	0.5\\
1601	0.5\\
1602	0.5\\
1603	0.5\\
1604	0.5\\
1605	0.5\\
1606	0.5\\
1607	0.5\\
1608	1\\
1609	1\\
1610	0.5\\
1611	0.5\\
1612	0.5\\
1613	0.5\\
1614	0.5\\
1615	0.5\\
1616	0.5\\
1617	1\\
1618	1\\
1619	0.5\\
1620	0.5\\
1621	0.5\\
1622	0.5\\
1623	0.5\\
1624	0.5\\
1625	0.5\\
1626	0.5\\
1627	0.5\\
1628	0.5\\
1629	0.5\\
1630	0.5\\
1631	1\\
1632	1\\
1633	0.5\\
1634	0.5\\
1635	0.5\\
1636	0.5\\
1637	0.5\\
1638	0.5\\
1639	0.5\\
1640	0.5\\
1641	0.5\\
1642	0.5\\
1643	0.5\\
1644	0.5\\
1645	1\\
1646	1\\
1647	1\\
1648	1\\
1649	0.5\\
1650	0.5\\
1651	0.5\\
1652	0.5\\
1653	0.5\\
1654	0.5\\
1655	0.5\\
1656	0.5\\
1657	0.5\\
1658	0.5\\
1659	0.5\\
1660	0.5\\
1661	0.5\\
1662	1\\
1663	1\\
1664	0.5\\
1665	0.5\\
1666	0.5\\
1667	0.5\\
1668	0.5\\
1669	0.5\\
1670	0.5\\
1671	0.5\\
1672	0.5\\
1673	0.5\\
1674	0.5\\
1675	0.5\\
1676	0.5\\
1677	1\\
1678	1\\
1679	1\\
1680	1\\
1681	1\\
1682	1\\
1683	0.5\\
1684	0.5\\
1685	0.5\\
1686	0.5\\
1687	0.5\\
1688	0.5\\
1689	0.5\\
1690	0.5\\
1691	0.5\\
1692	0.5\\
1693	0.5\\
1694	0.5\\
1695	0.5\\
1696	0.5\\
1697	0.5\\
1698	0.5\\
1699	0.5\\
1700	0.5\\
1701	0.5\\
1702	0.5\\
1703	1\\
1704	1\\
1705	1\\
1706	1\\
1707	0.5\\
1708	0.5\\
1709	0.5\\
1710	0.5\\
1711	0.5\\
1712	0.5\\
1713	0.5\\
1714	0.5\\
1715	0.5\\
1716	0.5\\
1717	1\\
1718	1\\
1719	0.5\\
1720	0.5\\
1721	0.5\\
1722	0.5\\
1723	0.5\\
1724	0.5\\
1725	0.5\\
1726	0.5\\
1727	0.5\\
1728	0.5\\
1729	0.5\\
1730	0.5\\
1731	0.5\\
1732	1\\
1733	1\\
1734	0.5\\
1735	0.5\\
1736	0.5\\
1737	0.5\\
1738	0.5\\
1739	0.5\\
1740	0.5\\
1741	0.5\\
1742	0.5\\
1743	0.5\\
1744	0.5\\
1745	1\\
1746	1\\
1747	0.5\\
1748	0.5\\
1749	0.5\\
1750	0.5\\
1751	0.5\\
1752	0.5\\
1753	0.5\\
1754	0.5\\
1755	0.5\\
1756	0.5\\
1757	0.5\\
1758	0.5\\
1759	0.5\\
1760	1\\
1761	1\\
1762	0.5\\
1763	0.5\\
1764	0.5\\
1765	0.5\\
1766	0.5\\
1767	0.5\\
1768	0.5\\
1769	0.5\\
1770	0.5\\
1771	1\\
1772	1\\
1773	1\\
1774	1\\
1775	1\\
1776	1\\
1777	1\\
1778	0.5\\
1779	0.5\\
1780	0.5\\
1781	0.5\\
1782	0.5\\
1783	0.5\\
1784	0.5\\
1785	0.5\\
1786	0.5\\
1787	0.5\\
1788	0.5\\
1789	0.5\\
1790	1\\
1791	1\\
1792	0.5\\
1793	0.5\\
1794	0.5\\
1795	0.5\\
1796	0.5\\
1797	0.5\\
1798	0.5\\
1799	0.5\\
1800	0.5\\
1801	1\\
1802	1\\
1803	1\\
1804	1\\
1805	0.5\\
1806	0.5\\
1807	0.5\\
1808	0.5\\
1809	0.5\\
1810	0.5\\
1811	0.5\\
1812	0.5\\
1813	0.5\\
1814	0.5\\
1815	0.5\\
1816	0.5\\
1817	0.5\\
1818	0.5\\
1819	0.5\\
1820	0.5\\
1821	0.5\\
1822	0.5\\
1823	0.5\\
1824	0.5\\
1825	0.5\\
1826	0.5\\
1827	0.5\\
1828	1\\
1829	1\\
1830	1\\
1831	0.5\\
1832	0.5\\
1833	0.5\\
1834	0.5\\
1835	0.5\\
1836	0.5\\
1837	0.5\\
1838	0.5\\
1839	0.5\\
1840	0.5\\
1841	0.5\\
1842	0.5\\
1843	1\\
1844	1\\
1845	0.5\\
1846	0.5\\
1847	0.5\\
1848	0.5\\
1849	0.5\\
1850	0.5\\
1851	0.5\\
1852	0.5\\
1853	0.5\\
1854	0.5\\
1855	0.5\\
1856	0.5\\
1857	0.5\\
1858	0.5\\
1859	0.5\\
1860	0.5\\
1861	1\\
1862	1\\
1863	0.5\\
1864	0.5\\
1865	0.5\\
1866	0.5\\
1867	0.5\\
1868	0.5\\
1869	0.5\\
1870	0.5\\
1871	0.5\\
1872	1\\
1873	1\\
1874	0.5\\
1875	0.5\\
1876	0.5\\
1877	0.5\\
1878	0.5\\
1879	0.5\\
1880	0.5\\
1881	0.5\\
1882	0.5\\
1883	0.5\\
1884	0.5\\
1885	0.5\\
1886	0.5\\
1887	0.5\\
1888	0.5\\
1889	0.5\\
1890	0.5\\
1891	1\\
1892	1\\
1893	0.5\\
1894	0.5\\
1895	0.5\\
1896	0.5\\
1897	0.5\\
1898	0.5\\
1899	0.5\\
1900	1\\
1901	1\\
1902	0.5\\
1903	0.5\\
1904	0.5\\
1905	0.5\\
1906	0.5\\
1907	0.5\\
1908	0.5\\
1909	0.5\\
1910	0.5\\
1911	0.5\\
1912	0.5\\
1913	0.5\\
1914	0.5\\
1915	0.5\\
1916	0.5\\
1917	1\\
1918	1\\
1919	0.5\\
1920	0.5\\
1921	0.5\\
1922	0.5\\
1923	0.5\\
1924	0.5\\
1925	0.5\\
1926	0.5\\
1927	0.5\\
1928	1\\
1929	1\\
1930	1\\
1931	1\\
1932	0.5\\
1933	0.5\\
1934	0.5\\
1935	0.5\\
1936	0.5\\
1937	0.5\\
1938	0.5\\
1939	0.5\\
1940	0.5\\
1941	0.5\\
1942	0.5\\
1943	0.5\\
1944	0.5\\
1945	0.5\\
1946	1\\
1947	1\\
1948	0.5\\
1949	0.5\\
1950	0.5\\
1951	0.5\\
1952	0.5\\
1953	0.5\\
1954	0.5\\
1955	0.5\\
1956	0.5\\
1957	0.5\\
1958	0.5\\
1959	1\\
1960	1\\
1961	1\\
1962	1\\
1963	1\\
1964	1\\
1965	0.5\\
};
\end{axis}

\begin{axis}[%
width=4.133in,
height=0.863in,
at={(0.693in,1.639in)},
scale only axis,
xmin=0,
xmax=2000,
xmajorgrids,
ymin=0,
ymax=1,
ymajorgrids,
axis background/.style={fill=white}
]
\pgfplotsset{max space between ticks=50}
\addplot [color=mycolor3,solid,forget plot]
  table[row sep=crcr]{%
1	0\\
2	0.5\\
3	0.5\\
4	0.5\\
5	0.5\\
6	1\\
7	1\\
8	0\\
9	0\\
10	0.5\\
11	0.5\\
12	1\\
13	1\\
14	1\\
15	1\\
16	0\\
17	0\\
18	0\\
19	0\\
20	0\\
21	0\\
22	1\\
23	1\\
24	0.5\\
25	0.5\\
26	0.5\\
27	0.5\\
28	0.5\\
29	1\\
30	1\\
31	1\\
32	0\\
33	0\\
34	0\\
35	0\\
36	0\\
37	0\\
38	0.5\\
39	0.5\\
40	1\\
41	1\\
42	1\\
43	1\\
44	0\\
45	0\\
46	0\\
47	0\\
48	0\\
49	0\\
50	0\\
51	1\\
52	1\\
53	0.5\\
54	0.5\\
55	0.5\\
56	0.5\\
57	1\\
58	1\\
59	0\\
60	0\\
61	0\\
62	0\\
63	0\\
64	0.5\\
65	0.5\\
66	0.5\\
67	1\\
68	1\\
69	1\\
70	1\\
71	0\\
72	0\\
73	0\\
74	0\\
75	0\\
76	1\\
77	1\\
78	1\\
79	1\\
80	1\\
81	1\\
82	1\\
83	0.5\\
84	0.5\\
85	0.5\\
86	0.5\\
87	1\\
88	1\\
89	0\\
90	0\\
91	0\\
92	0\\
93	0\\
94	0.5\\
95	0.5\\
96	0.5\\
97	0.5\\
98	1\\
99	1\\
100	1\\
101	0\\
102	0\\
103	0\\
104	0\\
105	0\\
106	1\\
107	1\\
108	1\\
109	1\\
110	1\\
111	0.5\\
112	0.5\\
113	0.5\\
114	0.5\\
115	0.5\\
116	1\\
117	1\\
118	0\\
119	0\\
120	0\\
121	0\\
122	0\\
123	0\\
124	0\\
125	0.5\\
126	0.5\\
127	0.5\\
128	0.5\\
129	0.5\\
130	0.5\\
131	1\\
132	1\\
133	1\\
134	0\\
135	0\\
136	0\\
137	1\\
138	1\\
139	1\\
140	0\\
141	0\\
142	0\\
143	0\\
144	0\\
145	0\\
146	0\\
147	1\\
148	1\\
149	0.5\\
150	0.5\\
151	0.5\\
152	0.5\\
153	0.5\\
154	0.5\\
155	1\\
156	1\\
157	1\\
158	0\\
159	0\\
160	0\\
161	0\\
162	0\\
163	0\\
164	0\\
165	0\\
166	0.5\\
167	0.5\\
168	1\\
169	1\\
170	1\\
171	1\\
172	1\\
173	0\\
174	0\\
175	0\\
176	0\\
177	0\\
178	1\\
179	1\\
180	0.5\\
181	0.5\\
182	0.5\\
183	0.5\\
184	0.5\\
185	0.5\\
186	1\\
187	1\\
188	1\\
189	1\\
190	0\\
191	0\\
192	0\\
193	0\\
194	0\\
195	0\\
196	0.5\\
197	0.5\\
198	1\\
199	1\\
200	1\\
201	1\\
202	0\\
203	0\\
204	0\\
205	0\\
206	0\\
207	1\\
208	1\\
209	0.5\\
210	0.5\\
211	0.5\\
212	0.5\\
213	1\\
214	1\\
215	0\\
216	0\\
217	0\\
218	0\\
219	0.5\\
220	0.5\\
221	0.5\\
222	1\\
223	1\\
224	1\\
225	1\\
226	0\\
227	0\\
228	0\\
229	0\\
230	0\\
231	1\\
232	1\\
233	1\\
234	1\\
235	0.5\\
236	0.5\\
237	0.5\\
238	1\\
239	1\\
240	0\\
241	0\\
242	0\\
243	0\\
244	0\\
245	0\\
246	0.5\\
247	0.5\\
248	0.5\\
249	0.5\\
250	0.5\\
251	0.5\\
252	0.5\\
253	1\\
254	1\\
255	1\\
256	1\\
257	0\\
258	0\\
259	0\\
260	0\\
261	0\\
262	1\\
263	1\\
264	1\\
265	1\\
266	1\\
267	1\\
268	0.5\\
269	0.5\\
270	0.5\\
271	1\\
272	1\\
273	0\\
274	0\\
275	0\\
276	0\\
277	0\\
278	0\\
279	0.5\\
280	0.5\\
281	0.5\\
282	1\\
283	1\\
284	1\\
285	1\\
286	0\\
287	0\\
288	0\\
289	0.5\\
290	0.5\\
291	1\\
292	1\\
293	1\\
294	1\\
295	1\\
296	1\\
297	0\\
298	0\\
299	0\\
300	0\\
301	0\\
302	1\\
303	1\\
304	0.5\\
305	0.5\\
306	0.5\\
307	0.5\\
308	0.5\\
309	1\\
310	1\\
311	1\\
312	0\\
313	0\\
314	0\\
315	0\\
316	0\\
317	0\\
318	0\\
319	0\\
320	0.5\\
321	0.5\\
322	0.5\\
323	1\\
324	1\\
325	1\\
326	1\\
327	1\\
328	0\\
329	0\\
330	0\\
331	0\\
332	0\\
333	1\\
334	1\\
335	0.5\\
336	0.5\\
337	0.5\\
338	0.5\\
339	1\\
340	1\\
341	0\\
342	0\\
343	0\\
344	0\\
345	0\\
346	0.5\\
347	0.5\\
348	0.5\\
349	1\\
350	1\\
351	1\\
352	1\\
353	1\\
354	1\\
355	0\\
356	0\\
357	0\\
358	0\\
359	0\\
360	0\\
361	1\\
362	1\\
363	1\\
364	0.5\\
365	0.5\\
366	0.5\\
367	0.5\\
368	1\\
369	1\\
370	0\\
371	0\\
372	0\\
373	0\\
374	0\\
375	0.5\\
376	0.5\\
377	0.5\\
378	0.5\\
379	1\\
380	1\\
381	1\\
382	0\\
383	0\\
384	0\\
385	0\\
386	0\\
387	0\\
388	0\\
389	1\\
390	1\\
391	1\\
392	1\\
393	1\\
394	0.5\\
395	0.5\\
396	0.5\\
397	1\\
398	1\\
399	0\\
400	0\\
401	0\\
402	0\\
403	0\\
404	0\\
405	0\\
406	0.5\\
407	0.5\\
408	0.5\\
409	0.5\\
410	0.5\\
411	0.5\\
412	1\\
413	1\\
414	1\\
415	0\\
416	0\\
417	0\\
418	1\\
419	1\\
420	1\\
421	1\\
422	0\\
423	0\\
424	0\\
425	0\\
426	0\\
427	1\\
428	1\\
429	0.5\\
430	0.5\\
431	0.5\\
432	0.5\\
433	0.5\\
434	0.5\\
435	1\\
436	1\\
437	1\\
438	0\\
439	0\\
440	0\\
441	0\\
442	0\\
443	0\\
444	0\\
445	0\\
446	0.5\\
447	0.5\\
448	1\\
449	1\\
450	1\\
451	1\\
452	1\\
453	0\\
454	0\\
455	0\\
456	0\\
457	0\\
458	0\\
459	0\\
460	1\\
461	1\\
462	0.5\\
463	0.5\\
464	0.5\\
465	0.5\\
466	1\\
467	1\\
468	1\\
469	0\\
470	0\\
471	0\\
472	0\\
473	0\\
474	0\\
475	0.5\\
476	0.5\\
477	1\\
478	1\\
479	1\\
480	1\\
481	0\\
482	0\\
483	0\\
484	0\\
485	0\\
486	1\\
487	1\\
488	0.5\\
489	0.5\\
490	0.5\\
491	1\\
492	1\\
493	0\\
494	0\\
495	0\\
496	0\\
497	0.5\\
498	0.5\\
499	0.5\\
500	1\\
501	1\\
502	1\\
503	1\\
504	0\\
505	0\\
506	0\\
507	0\\
508	0\\
509	1\\
510	1\\
511	1\\
512	1\\
513	0.5\\
514	0.5\\
515	0.5\\
516	0.5\\
517	1\\
518	1\\
519	0\\
520	0\\
521	0\\
522	0\\
523	0\\
524	0\\
525	0.5\\
526	0.5\\
527	0.5\\
528	0.5\\
529	0.5\\
530	1\\
531	1\\
532	1\\
533	0\\
534	0\\
535	0\\
536	0\\
537	0\\
538	1\\
539	1\\
540	1\\
541	1\\
542	1\\
543	1\\
544	0.5\\
545	0.5\\
546	0.5\\
547	1\\
548	1\\
549	0\\
550	0\\
551	0\\
552	0\\
553	0\\
554	0\\
555	0\\
556	0.5\\
557	0.5\\
558	0.5\\
559	0.5\\
560	1\\
561	1\\
562	1\\
563	0\\
564	0\\
565	0\\
566	0.5\\
567	0.5\\
568	1\\
569	1\\
570	1\\
571	1\\
572	1\\
573	1\\
574	0\\
575	0\\
576	0\\
577	0\\
578	0\\
579	0\\
580	1\\
581	1\\
582	0.5\\
583	0.5\\
584	0.5\\
585	0.5\\
586	0.5\\
587	0.5\\
588	0.5\\
589	1\\
590	1\\
591	1\\
592	1\\
593	0\\
594	0\\
595	0\\
596	0\\
597	0\\
598	0\\
599	0.5\\
600	0.5\\
601	1\\
602	1\\
603	1\\
604	1\\
605	0\\
606	0\\
607	0\\
608	0\\
609	0\\
610	1\\
611	1\\
612	0.5\\
613	0.5\\
614	0.5\\
615	0.5\\
616	1\\
617	1\\
618	1\\
619	1\\
620	0\\
621	0\\
622	0\\
623	0\\
624	0\\
625	0\\
626	0\\
627	0.5\\
628	0.5\\
629	0.5\\
630	1\\
631	1\\
632	1\\
633	1\\
634	0\\
635	0\\
636	0\\
637	0\\
638	0\\
639	1\\
640	1\\
641	1\\
642	0.5\\
643	0.5\\
644	0.5\\
645	0.5\\
646	1\\
647	1\\
648	0\\
649	0\\
650	0\\
651	0\\
652	0\\
653	0.5\\
654	0.5\\
655	0.5\\
656	0.5\\
657	0.5\\
658	0.5\\
659	1\\
660	1\\
661	1\\
662	1\\
663	0\\
664	0\\
665	0\\
666	0\\
667	0\\
668	1\\
669	1\\
670	1\\
671	1\\
672	1\\
673	0.5\\
674	0.5\\
675	0.5\\
676	1\\
677	1\\
678	0\\
679	0\\
680	0\\
681	0\\
682	0\\
683	0\\
684	0.5\\
685	0.5\\
686	0.5\\
687	0.5\\
688	0.5\\
689	0.5\\
690	1\\
691	1\\
692	1\\
693	1\\
694	0\\
695	0\\
696	0\\
697	0\\
698	1\\
699	1\\
700	1\\
701	0\\
702	0\\
703	0\\
704	0\\
705	0\\
706	1\\
707	1\\
708	0.5\\
709	0.5\\
710	0.5\\
711	0.5\\
712	0.5\\
713	0.5\\
714	1\\
715	1\\
716	1\\
717	0\\
718	0\\
719	0\\
720	0\\
721	0\\
722	0\\
723	0\\
724	0\\
725	0\\
726	0.5\\
727	0.5\\
728	1\\
729	1\\
730	1\\
731	1\\
732	1\\
733	0\\
734	0\\
735	0\\
736	0\\
737	0\\
738	1\\
739	1\\
740	0.5\\
741	0.5\\
742	0.5\\
743	0.5\\
744	1\\
745	1\\
746	1\\
747	0\\
748	0\\
749	0\\
750	0\\
751	0\\
752	0.5\\
753	0.5\\
754	1\\
755	1\\
756	1\\
757	1\\
758	1\\
759	1\\
760	0\\
761	0\\
762	0\\
763	0\\
764	0\\
765	0\\
766	1\\
767	1\\
768	0.5\\
769	0.5\\
770	0.5\\
771	0.5\\
772	1\\
773	1\\
774	0\\
775	0\\
776	0\\
777	0\\
778	0.5\\
779	0.5\\
780	0.5\\
781	1\\
782	1\\
783	1\\
784	1\\
785	0\\
786	0\\
787	0\\
788	0\\
789	0\\
790	0\\
791	1\\
792	1\\
793	1\\
794	1\\
795	0.5\\
796	0.5\\
797	0.5\\
798	1\\
799	1\\
800	0\\
801	0\\
802	0\\
803	0\\
804	0\\
805	0\\
806	0.5\\
807	0.5\\
808	0.5\\
809	0.5\\
810	0.5\\
811	1\\
812	1\\
813	1\\
814	0\\
815	0\\
816	0\\
817	0\\
818	0\\
819	1\\
820	1\\
821	1\\
822	1\\
823	1\\
824	1\\
825	1\\
826	1\\
827	0.5\\
828	0.5\\
829	0.5\\
830	0.5\\
831	1\\
832	1\\
833	0\\
834	0\\
835	0\\
836	0\\
837	0\\
838	0\\
839	0.5\\
840	0.5\\
841	0.5\\
842	1\\
843	1\\
844	1\\
845	0\\
846	0\\
847	0\\
848	0.5\\
849	0.5\\
850	1\\
851	1\\
852	1\\
853	1\\
854	1\\
855	1\\
856	0\\
857	0\\
858	0\\
859	0\\
860	0\\
861	0\\
862	0\\
863	1\\
864	1\\
865	0.5\\
866	0.5\\
867	0.5\\
868	0.5\\
869	0.5\\
870	1\\
871	1\\
872	1\\
873	0\\
874	0\\
875	0\\
876	0\\
877	0\\
878	0\\
879	0.5\\
880	0.5\\
881	1\\
882	1\\
883	1\\
884	1\\
885	0\\
886	0\\
887	0\\
888	0\\
889	0\\
890	0\\
891	0\\
892	1\\
893	1\\
894	1\\
895	1\\
896	1\\
897	0.5\\
898	0.5\\
899	0.5\\
900	0.5\\
901	1\\
902	1\\
903	0\\
904	0\\
905	0\\
906	0\\
907	0\\
908	0.5\\
909	0.5\\
910	0.5\\
911	1\\
912	1\\
913	1\\
914	1\\
915	0\\
916	0\\
917	0\\
918	0\\
919	0\\
920	1\\
921	1\\
922	1\\
923	0.5\\
924	0.5\\
925	0.5\\
926	0.5\\
927	0.5\\
928	0.5\\
929	1\\
930	1\\
931	0\\
932	0\\
933	0\\
934	0\\
935	0\\
936	0.5\\
937	0.5\\
938	0.5\\
939	0.5\\
940	1\\
941	1\\
942	1\\
943	0\\
944	0\\
945	0\\
946	0\\
947	0\\
948	1\\
949	1\\
950	1\\
951	1\\
952	1\\
953	0.5\\
954	0.5\\
955	0.5\\
956	1\\
957	1\\
958	0\\
959	0\\
960	0\\
961	0\\
962	0\\
963	0\\
964	0\\
965	0\\
966	0\\
967	0.5\\
968	0.5\\
969	0.5\\
970	0.5\\
971	0.5\\
972	0.5\\
973	1\\
974	1\\
975	1\\
976	0\\
977	0\\
978	0\\
979	1\\
980	1\\
981	1\\
982	0\\
983	0\\
984	0\\
985	0\\
986	0\\
987	0\\
988	1\\
989	1\\
990	0.5\\
991	0.5\\
992	0.5\\
993	0.5\\
994	0.5\\
995	0.5\\
996	0.5\\
997	0.5\\
998	1\\
999	1\\
1000	1\\
1001	1\\
1002	0\\
1003	0\\
1004	0\\
1005	0\\
1006	0\\
1007	0\\
1008	0\\
1009	0\\
1010	0.5\\
1011	0.5\\
1012	1\\
1013	1\\
1014	1\\
1015	1\\
1016	1\\
1017	0\\
1018	0\\
1019	0\\
1020	0\\
1021	0\\
1022	1\\
1023	1\\
1024	0.5\\
1025	0.5\\
1026	0.5\\
1027	0.5\\
1028	1\\
1029	1\\
1030	1\\
1031	1\\
1032	0\\
1033	0\\
1034	0\\
1035	0\\
1036	0\\
1037	0\\
1038	0\\
1039	0.5\\
1040	0.5\\
1041	1\\
1042	1\\
1043	1\\
1044	1\\
1045	0\\
1046	0\\
1047	0\\
1048	0\\
1049	0\\
1050	1\\
1051	1\\
1052	0.5\\
1053	0.5\\
1054	0.5\\
1055	0.5\\
1056	1\\
1057	1\\
1058	0\\
1059	0\\
1060	0\\
1061	0\\
1062	0.5\\
1063	0.5\\
1064	0.5\\
1065	0.5\\
1066	0.5\\
1067	0.5\\
1068	0.5\\
1069	1\\
1070	1\\
1071	1\\
1072	1\\
1073	0\\
1074	0\\
1075	0\\
1076	0\\
1077	0\\
1078	1\\
1079	1\\
1080	1\\
1081	1\\
1082	0.5\\
1083	0.5\\
1084	0.5\\
1085	1\\
1086	1\\
1087	0\\
1088	0\\
1089	0\\
1090	0\\
1091	0\\
1092	0\\
1093	0.5\\
1094	0.5\\
1095	0.5\\
1096	0.5\\
1097	0.5\\
1098	1\\
1099	1\\
1100	1\\
1101	1\\
1102	0\\
1103	0\\
1104	0\\
1105	0\\
1106	0\\
1107	0\\
1108	1\\
1109	1\\
1110	1\\
1111	1\\
1112	1\\
1113	1\\
1114	0.5\\
1115	0.5\\
1116	0.5\\
1117	1\\
1118	1\\
1119	0\\
1120	0\\
1121	0\\
1122	0\\
1123	0\\
1124	0\\
1125	0.5\\
1126	0.5\\
1127	0.5\\
1128	1\\
1129	1\\
1130	1\\
1131	0\\
1132	0\\
1133	0\\
1134	0\\
1135	0.5\\
1136	0.5\\
1137	1\\
1138	1\\
1139	1\\
1140	1\\
1141	1\\
1142	1\\
1143	0\\
1144	0\\
1145	0\\
1146	0\\
1147	0\\
1148	0\\
1149	1\\
1150	1\\
1151	0.5\\
1152	0.5\\
1153	0.5\\
1154	0.5\\
1155	0.5\\
1156	1\\
1157	1\\
1158	1\\
1159	0\\
1160	0\\
1161	0\\
1162	0\\
1163	0\\
1164	0\\
1165	0.5\\
1166	0.5\\
1167	1\\
1168	1\\
1169	1\\
1170	1\\
1171	1\\
1172	1\\
1173	0\\
1174	0\\
1175	0\\
1176	0\\
1177	0\\
1178	0\\
1179	0\\
1180	1\\
1181	1\\
1182	0.5\\
1183	0.5\\
1184	0.5\\
1185	0.5\\
1186	1\\
1187	1\\
1188	0\\
1189	0\\
1190	0\\
1191	0\\
1192	0\\
1193	0.5\\
1194	0.5\\
1195	0.5\\
1196	1\\
1197	1\\
1198	1\\
1199	1\\
1200	0\\
1201	0\\
1202	0\\
1203	0\\
1204	0\\
1205	0\\
1206	1\\
1207	1\\
1208	1\\
1209	0.5\\
1210	0.5\\
1211	0.5\\
1212	0.5\\
1213	1\\
1214	1\\
1215	0\\
1216	0\\
1217	0\\
1218	0\\
1219	0\\
1220	0.5\\
1221	0.5\\
1222	0.5\\
1223	0.5\\
1224	1\\
1225	1\\
1226	1\\
1227	0\\
1228	0\\
1229	0\\
1230	0\\
1231	0\\
1232	1\\
1233	1\\
1234	1\\
1235	1\\
1236	1\\
1237	1\\
1238	1\\
1239	0.5\\
1240	0.5\\
1241	0.5\\
1242	0.5\\
1243	1\\
1244	1\\
1245	0\\
1246	0\\
1247	0\\
1248	0\\
1249	0\\
1250	0\\
1251	0\\
1252	0.5\\
1253	0.5\\
1254	0.5\\
1255	0.5\\
1256	0.5\\
1257	0.5\\
1258	1\\
1259	1\\
1260	1\\
1261	0\\
1262	0\\
1263	0\\
1264	1\\
1265	1\\
1266	1\\
1267	0\\
1268	0\\
1269	0\\
1270	0\\
1271	0\\
1272	0\\
1273	0\\
1274	1\\
1275	1\\
1276	0.5\\
1277	0.5\\
1278	0.5\\
1279	0.5\\
1280	0.5\\
1281	0.5\\
1282	1\\
1283	1\\
1284	1\\
1285	0\\
1286	0\\
1287	0\\
1288	0\\
1289	0\\
1290	0\\
1291	0\\
1292	0.5\\
1293	0.5\\
1294	1\\
1295	1\\
1296	1\\
1297	1\\
1298	1\\
1299	0\\
1300	0\\
1301	0\\
1302	0\\
1303	0\\
1304	0\\
1305	0\\
1306	1\\
1307	1\\
1308	1\\
1309	0.5\\
1310	0.5\\
1311	0.5\\
1312	0.5\\
1313	0.5\\
1314	1\\
1315	1\\
1316	0\\
1317	0\\
1318	0\\
1319	0\\
1320	0\\
1321	0.5\\
1322	0.5\\
1323	1\\
1324	1\\
1325	1\\
1326	1\\
1327	0\\
1328	0\\
1329	0\\
1330	0\\
1331	0\\
1332	1\\
1333	1\\
1334	1\\
1335	0.5\\
1336	0.5\\
1337	0.5\\
1338	0.5\\
1339	0.5\\
1340	0.5\\
1341	1\\
1342	1\\
1343	0\\
1344	0\\
1345	0\\
1346	0\\
1347	0\\
1348	0.5\\
1349	0.5\\
1350	0.5\\
1351	1\\
1352	1\\
1353	1\\
1354	1\\
1355	0\\
1356	0\\
1357	0\\
1358	0\\
1359	0\\
1360	1\\
1361	1\\
1362	1\\
1363	1\\
1364	0.5\\
1365	0.5\\
1366	0.5\\
1367	1\\
1368	1\\
1369	0\\
1370	0\\
1371	0\\
1372	0\\
1373	0\\
1374	0\\
1375	0\\
1376	0\\
1377	0.5\\
1378	0.5\\
1379	0.5\\
1380	0.5\\
1381	0.5\\
1382	1\\
1383	1\\
1384	1\\
1385	0\\
1386	0\\
1387	0\\
1388	0\\
1389	0\\
1390	1\\
1391	1\\
1392	1\\
1393	1\\
1394	1\\
1395	1\\
1396	0.5\\
1397	0.5\\
1398	0.5\\
1399	1\\
1400	1\\
1401	0\\
1402	0\\
1403	0\\
1404	0\\
1405	0\\
1406	0\\
1407	0.5\\
1408	0.5\\
1409	0.5\\
1410	0.5\\
1411	1\\
1412	1\\
1413	1\\
1414	0\\
1415	0\\
1416	0.5\\
1417	0.5\\
1418	1\\
1419	1\\
1420	1\\
1421	1\\
1422	1\\
1423	1\\
1424	0\\
1425	0\\
1426	0\\
1427	0\\
1428	0\\
1429	0\\
1430	1\\
1431	1\\
1432	0.5\\
1433	0.5\\
1434	0.5\\
1435	0.5\\
1436	0.5\\
1437	1\\
1438	1\\
1439	1\\
1440	1\\
1441	0\\
1442	0\\
1443	0\\
1444	0\\
1445	0\\
1446	0\\
1447	0.5\\
1448	0.5\\
1449	1\\
1450	1\\
1451	1\\
1452	1\\
1453	0\\
1454	0\\
1455	0\\
1456	0\\
1457	0\\
1458	0\\
1459	1\\
1460	1\\
1461	0.5\\
1462	0.5\\
1463	0.5\\
1464	0.5\\
1465	1\\
1466	1\\
1467	0\\
1468	0\\
1469	0\\
1470	0\\
1471	0\\
1472	0.5\\
1473	0.5\\
1474	0.5\\
1475	1\\
1476	1\\
1477	1\\
1478	1\\
1479	0\\
1480	0\\
1481	0\\
1482	0\\
1483	0\\
1484	1\\
1485	1\\
1486	1\\
1487	0.5\\
1488	0.5\\
1489	0.5\\
1490	0.5\\
1491	1\\
1492	1\\
1493	0\\
1494	0\\
1495	0\\
1496	0\\
1497	0\\
1498	0.5\\
1499	0.5\\
1500	0.5\\
1501	0.5\\
1502	1\\
1503	1\\
1504	1\\
1505	1\\
1506	0\\
1507	0\\
1508	0\\
1509	0\\
1510	0\\
1511	0\\
1512	1\\
1513	1\\
1514	1\\
1515	1\\
1516	1\\
1517	0.5\\
1518	0.5\\
1519	0.5\\
1520	1\\
1521	1\\
1522	0\\
1523	0\\
1524	0\\
1525	0\\
1526	0\\
1527	0\\
1528	0\\
1529	0.5\\
1530	0.5\\
1531	0.5\\
1532	0.5\\
1533	0.5\\
1534	0.5\\
1535	1\\
1536	1\\
1537	1\\
1538	0\\
1539	0\\
1540	0\\
1541	0\\
1542	1\\
1543	1\\
1544	1\\
1545	0\\
1546	0\\
1547	0\\
1548	0\\
1549	1\\
1550	1\\
1551	0.5\\
1552	0.5\\
1553	0.5\\
1554	0.5\\
1555	0.5\\
1556	0.5\\
1557	1\\
1558	1\\
1559	1\\
1560	0\\
1561	0\\
1562	0\\
1563	0\\
1564	0\\
1565	0\\
1566	0\\
1567	0\\
1568	0.5\\
1569	0.5\\
1570	1\\
1571	1\\
1572	1\\
1573	1\\
1574	1\\
1575	1\\
1576	1\\
1577	0\\
1578	0\\
1579	0\\
1580	0\\
1581	0\\
1582	0\\
1583	1\\
1584	1\\
1585	0.5\\
1586	0.5\\
1587	0.5\\
1588	0.5\\
1589	1\\
1590	1\\
1591	1\\
1592	0\\
1593	0\\
1594	0\\
1595	0\\
1596	0\\
1597	0\\
1598	0.5\\
1599	0.5\\
1600	1\\
1601	1\\
1602	1\\
1603	1\\
1604	0\\
1605	0\\
1606	0\\
1607	0\\
1608	0\\
1609	0\\
1610	1\\
1611	1\\
1612	0.5\\
1613	0.5\\
1614	0.5\\
1615	1\\
1616	1\\
1617	0\\
1618	0\\
1619	0\\
1620	0\\
1621	0.5\\
1622	0.5\\
1623	0.5\\
1624	1\\
1625	1\\
1626	1\\
1627	1\\
1628	0\\
1629	0\\
1630	0\\
1631	0\\
1632	0\\
1633	1\\
1634	1\\
1635	1\\
1636	1\\
1637	1\\
1638	1\\
1639	0.5\\
1640	0.5\\
1641	0.5\\
1642	0.5\\
1643	1\\
1644	1\\
1645	0\\
1646	0\\
1647	0\\
1648	0\\
1649	0\\
1650	0\\
1651	0.5\\
1652	0.5\\
1653	0.5\\
1654	0.5\\
1655	0.5\\
1656	1\\
1657	1\\
1658	1\\
1659	0\\
1660	0\\
1661	0\\
1662	0\\
1663	0\\
1664	1\\
1665	1\\
1666	1\\
1667	1\\
1668	1\\
1669	1\\
1670	0.5\\
1671	0.5\\
1672	0.5\\
1673	0.5\\
1674	0.5\\
1675	1\\
1676	1\\
1677	0\\
1678	0\\
1679	0\\
1680	0\\
1681	0\\
1682	0\\
1683	0.5\\
1684	0.5\\
1685	0.5\\
1686	1\\
1687	1\\
1688	1\\
1689	0\\
1690	0\\
1691	0\\
1692	0.5\\
1693	0.5\\
1694	1\\
1695	1\\
1696	1\\
1697	1\\
1698	1\\
1699	1\\
1700	0\\
1701	0\\
1702	0\\
1703	0\\
1704	0\\
1705	0\\
1706	0\\
1707	1\\
1708	1\\
1709	0.5\\
1710	0.5\\
1711	0.5\\
1712	0.5\\
1713	0.5\\
1714	1\\
1715	1\\
1716	1\\
1717	0\\
1718	0\\
1719	0\\
1720	0\\
1721	0\\
1722	0\\
1723	0.5\\
1724	0.5\\
1725	1\\
1726	1\\
1727	1\\
1728	1\\
1729	0\\
1730	0\\
1731	0\\
1732	0\\
1733	0\\
1734	1\\
1735	1\\
1736	0.5\\
1737	0.5\\
1738	0.5\\
1739	0.5\\
1740	0.5\\
1741	0.5\\
1742	1\\
1743	1\\
1744	1\\
1745	0\\
1746	0\\
1747	0\\
1748	0\\
1749	0\\
1750	0.5\\
1751	0.5\\
1752	0.5\\
1753	1\\
1754	1\\
1755	1\\
1756	1\\
1757	0\\
1758	0\\
1759	0\\
1760	0\\
1761	0\\
1762	1\\
1763	1\\
1764	1\\
1765	0.5\\
1766	0.5\\
1767	0.5\\
1768	0.5\\
1769	1\\
1770	1\\
1771	0\\
1772	0\\
1773	0\\
1774	0\\
1775	0\\
1776	0\\
1777	0\\
1778	0\\
1779	0\\
1780	0.5\\
1781	0.5\\
1782	0.5\\
1783	0.5\\
1784	1\\
1785	1\\
1786	1\\
1787	0\\
1788	0\\
1789	0\\
1790	0\\
1791	0\\
1792	1\\
1793	1\\
1794	1\\
1795	1\\
1796	0.5\\
1797	0.5\\
1798	0.5\\
1799	1\\
1800	1\\
1801	0\\
1802	0\\
1803	0\\
1804	0\\
1805	0\\
1806	0\\
1807	0.5\\
1808	0.5\\
1809	0.5\\
1810	0.5\\
1811	0.5\\
1812	0.5\\
1813	0.5\\
1814	0.5\\
1815	1\\
1816	1\\
1817	1\\
1818	1\\
1819	0\\
1820	0\\
1821	0\\
1822	1\\
1823	1\\
1824	1\\
1825	0\\
1826	0\\
1827	0\\
1828	0\\
1829	0\\
1830	0\\
1831	1\\
1832	1\\
1833	0.5\\
1834	0.5\\
1835	0.5\\
1836	0.5\\
1837	0.5\\
1838	0.5\\
1839	1\\
1840	1\\
1841	1\\
1842	1\\
1843	0\\
1844	0\\
1845	0\\
1846	0\\
1847	0\\
1848	0\\
1849	0\\
1850	0\\
1851	0.5\\
1852	0.5\\
1853	1\\
1854	1\\
1855	1\\
1856	1\\
1857	1\\
1858	0\\
1859	0\\
1860	0\\
1861	0\\
1862	0\\
1863	1\\
1864	1\\
1865	0.5\\
1866	0.5\\
1867	0.5\\
1868	0.5\\
1869	1\\
1870	1\\
1871	1\\
1872	0\\
1873	0\\
1874	0\\
1875	0\\
1876	0\\
1877	0\\
1878	0\\
1879	0.5\\
1880	0.5\\
1881	0.5\\
1882	0.5\\
1883	0.5\\
1884	1\\
1885	1\\
1886	1\\
1887	1\\
1888	0\\
1889	0\\
1890	0\\
1891	0\\
1892	0\\
1893	1\\
1894	1\\
1895	0.5\\
1896	0.5\\
1897	0.5\\
1898	1\\
1899	1\\
1900	0\\
1901	0\\
1902	0\\
1903	0\\
1904	0.5\\
1905	0.5\\
1906	0.5\\
1907	1\\
1908	1\\
1909	1\\
1910	1\\
1911	1\\
1912	1\\
1913	0\\
1914	0\\
1915	0\\
1916	0\\
1917	0\\
1918	0\\
1919	1\\
1920	1\\
1921	1\\
1922	1\\
1923	0.5\\
1924	0.5\\
1925	0.5\\
1926	1\\
1927	1\\
1928	0\\
1929	0\\
1930	0\\
1931	0\\
1932	0\\
1933	0\\
1934	0.5\\
1935	0.5\\
1936	0.5\\
1937	0.5\\
1938	0.5\\
1939	1\\
1940	1\\
1941	1\\
1942	0\\
1943	0\\
1944	0\\
1945	0\\
1946	0\\
1947	0\\
1948	1\\
1949	1\\
1950	1\\
1951	1\\
1952	1\\
1953	1\\
1954	0.5\\
1955	0.5\\
1956	0.5\\
1957	1\\
1958	1\\
1959	0\\
1960	0\\
1961	0\\
1962	0\\
1963	0\\
1964	0\\
1965	0.5\\
};
\end{axis}
\end{tikzpicture}%
}
      \caption{The calculated schedule for the three penduli.
        \texttt{Blue}: $P_1$, \texttt{Red}: $P_2$, \texttt{Orange}: $P_3$.
        $C_i = 10$ ms.}
      \label{fig:5.3}
    \end{figure}
  \end{minipage}
  \hfill
  \begin{minipage}{0.45\linewidth}
    \begin{figure}[H]\centering
    \scalebox{0.7}{% This file was created by matlab2tikz.
%
%The latest updates can be retrieved from
%  http://www.mathworks.com/matlabcentral/fileexchange/22022-matlab2tikz-matlab2tikz
%where you can also make suggestions and rate matlab2tikz.
%
\definecolor{mycolor1}{rgb}{0.00000,0.44700,0.74100}%
%
\begin{tikzpicture}

\begin{axis}[%
width=4.133in,
height=3.26in,
at={(0.693in,0.44in)},
scale only axis,
xmin=0,
xmax=1700,
xmajorgrids,
ymin=0,
ymax=1.25,
ymajorgrids,
axis background/.style={fill=white}
]
\addplot [color=mycolor1,solid,forget plot]
  table[row sep=crcr]{%
1	0\\
2	1\\
3	1\\
4	1\\
5	1\\
6	0.75\\
7	0.75\\
8	0.75\\
9	0.75\\
10	1\\
11	1\\
12	0.75\\
13	0.75\\
14	1\\
15	1\\
16	1\\
17	1\\
18	1\\
19	1\\
20	0.75\\
21	0.75\\
22	1\\
23	1\\
24	1\\
25	1\\
26	1\\
27	1\\
28	1\\
29	1\\
30	1\\
31	1\\
32	1\\
33	1\\
34	1\\
35	1\\
36	1\\
37	1\\
38	1\\
39	1\\
40	1\\
41	1\\
42	1\\
43	1\\
44	1\\
45	1\\
46	1\\
47	1\\
48	1\\
49	1\\
50	1\\
51	1\\
52	1\\
53	1\\
54	1\\
55	1\\
56	1\\
57	1\\
58	1\\
59	1\\
60	1\\
61	1\\
62	1\\
63	1\\
64	1\\
65	1\\
66	1\\
67	1\\
68	1\\
69	1\\
70	1\\
71	1\\
72	1\\
73	1\\
74	1\\
75	1\\
76	1\\
77	1\\
78	1\\
79	1\\
80	1\\
81	1\\
82	1\\
83	1\\
84	1\\
85	1\\
86	1\\
87	1\\
88	1\\
89	1\\
90	1\\
91	1\\
92	1\\
93	1\\
94	1\\
95	1\\
96	1\\
97	1\\
98	1\\
99	1\\
100	1\\
101	1\\
102	1\\
103	1\\
104	1\\
105	1\\
106	1\\
107	1\\
108	1\\
109	1\\
110	1\\
111	1\\
112	1\\
113	1\\
114	1\\
115	1\\
116	1\\
117	1\\
118	1\\
119	1\\
120	1\\
121	1\\
122	1\\
123	1\\
124	1\\
125	1\\
126	1\\
127	1\\
128	1\\
129	1\\
130	1\\
131	1\\
132	1\\
133	1\\
134	1\\
135	1\\
136	1\\
137	1\\
138	1\\
139	1\\
140	1\\
141	1\\
142	1\\
143	1\\
144	1\\
145	1\\
146	1\\
147	1\\
148	1\\
149	1\\
150	1\\
151	1\\
152	1\\
153	1\\
154	1\\
155	1\\
156	1\\
157	1\\
158	1\\
159	1\\
160	1\\
161	1\\
162	1\\
163	1\\
164	1\\
165	1\\
166	1\\
167	1\\
168	1\\
169	1\\
170	1\\
171	1\\
172	1\\
173	1\\
174	1\\
175	1\\
176	1\\
177	1\\
178	1\\
179	1\\
180	1\\
181	1\\
182	1\\
183	1\\
184	1\\
185	1\\
186	1\\
187	1\\
188	1\\
189	1\\
190	1\\
191	1\\
192	1\\
193	1\\
194	1\\
195	1\\
196	1\\
197	1\\
198	1\\
199	1\\
200	1\\
201	1\\
202	1\\
203	1\\
204	1\\
205	1\\
206	1\\
207	1\\
208	1\\
209	1\\
210	1\\
211	1\\
212	1\\
213	1\\
214	1\\
215	1\\
216	1\\
217	1\\
218	1\\
219	1\\
220	1\\
221	1\\
222	1\\
223	1\\
224	1\\
225	1\\
226	1\\
227	1\\
228	1\\
229	1\\
230	1\\
231	1\\
232	1\\
233	1\\
234	1\\
235	1\\
236	1\\
237	1\\
238	1\\
239	1\\
240	1\\
241	1\\
242	1\\
243	1\\
244	1\\
245	1\\
246	1\\
247	1\\
248	1\\
249	1\\
250	1\\
251	1\\
252	1\\
253	1\\
254	1\\
255	1\\
256	1\\
257	1\\
258	1\\
259	1\\
260	1\\
261	1\\
262	1\\
263	1\\
264	1\\
265	1\\
266	1\\
267	1\\
268	1\\
269	1\\
270	1\\
271	1\\
272	1\\
273	1\\
274	1\\
275	1\\
276	1\\
277	1\\
278	1\\
279	1\\
280	1\\
281	1\\
282	1\\
283	1\\
284	1\\
285	1\\
286	1\\
287	1\\
288	1\\
289	1\\
290	1\\
291	1\\
292	1\\
293	1\\
294	1\\
295	1\\
296	1\\
297	1\\
298	1\\
299	1\\
300	1\\
301	1\\
302	1\\
303	1\\
304	1\\
305	1\\
306	1\\
307	1\\
308	1\\
309	1\\
310	1\\
311	1\\
312	1\\
313	1\\
314	1\\
315	1\\
316	1\\
317	1\\
318	1\\
319	1\\
320	1\\
321	1\\
322	1\\
323	1\\
324	1\\
325	1\\
326	1\\
327	1\\
328	1\\
329	1\\
330	1\\
331	1\\
332	1\\
333	1\\
334	1\\
335	1\\
336	1\\
337	1\\
338	1\\
339	1\\
340	1\\
341	1\\
342	1\\
343	1\\
344	1\\
345	1\\
346	1\\
347	1\\
348	1\\
349	1\\
350	1\\
351	1\\
352	1\\
353	1\\
354	1\\
355	1\\
356	1\\
357	1\\
358	1\\
359	1\\
360	1\\
361	1\\
362	1\\
363	1\\
364	1\\
365	1\\
366	1\\
367	1\\
368	1\\
369	1\\
370	1\\
371	1\\
372	1\\
373	1\\
374	1\\
375	1\\
376	1\\
377	1\\
378	1\\
379	1\\
380	1\\
381	1\\
382	1\\
383	1\\
384	1\\
385	1\\
386	1\\
387	1\\
388	1\\
389	1\\
390	1\\
391	1\\
392	1\\
393	1\\
394	1\\
395	1\\
396	1\\
397	1\\
398	1\\
399	1\\
400	1\\
401	1\\
402	1\\
403	1\\
404	1\\
405	1\\
406	1\\
407	1\\
408	1\\
409	1\\
410	1\\
411	1\\
412	1\\
413	1\\
414	1\\
415	1\\
416	1\\
417	1\\
418	1\\
419	1\\
420	1\\
421	1\\
422	1\\
423	1\\
424	1\\
425	1\\
426	1\\
427	1\\
428	1\\
429	1\\
430	1\\
431	1\\
432	1\\
433	1\\
434	1\\
435	1\\
436	1\\
437	1\\
438	1\\
439	1\\
440	1\\
441	1\\
442	1\\
443	1\\
444	1\\
445	1\\
446	1\\
447	1\\
448	1\\
449	1\\
450	1\\
451	1\\
452	1\\
453	1\\
454	1\\
455	1\\
456	1\\
457	1\\
458	1\\
459	1\\
460	1\\
461	1\\
462	1\\
463	1\\
464	1\\
465	1\\
466	1\\
467	1\\
468	1\\
469	1\\
470	1\\
471	1\\
472	1\\
473	1\\
474	1\\
475	1\\
476	1\\
477	1\\
478	1\\
479	1\\
480	1\\
481	1\\
482	1\\
483	1\\
484	1\\
485	1\\
486	1\\
487	1\\
488	1\\
489	1\\
490	1\\
491	1\\
492	1\\
493	1\\
494	1\\
495	1\\
496	1\\
497	1\\
498	1\\
499	1\\
500	1\\
501	1\\
502	1\\
503	1\\
504	1\\
505	1\\
506	1\\
507	1\\
508	1\\
509	1\\
510	1\\
511	1\\
512	1\\
513	1\\
514	1\\
515	1\\
516	1\\
517	1\\
518	1\\
519	1\\
520	1\\
521	1\\
522	1\\
523	1\\
524	1\\
525	1\\
526	1\\
527	1\\
528	1\\
529	1\\
530	1\\
531	1\\
532	1\\
533	1\\
534	1\\
535	1\\
536	1\\
537	1\\
538	1\\
539	1\\
540	1\\
541	1\\
542	1\\
543	1\\
544	1\\
545	1\\
546	1\\
547	1\\
548	1\\
549	1\\
550	1\\
551	1\\
552	1\\
553	1\\
554	1\\
555	1\\
556	1\\
557	1\\
558	1\\
559	1\\
560	1\\
561	1\\
562	1\\
563	1\\
564	1\\
565	1\\
566	1\\
567	1\\
568	1\\
569	1\\
570	1\\
571	1\\
572	1\\
573	1\\
574	1\\
575	1\\
576	1\\
577	1\\
578	1\\
579	1\\
580	1\\
581	1\\
582	1\\
583	1\\
584	1\\
585	1\\
586	1\\
587	1\\
588	1\\
589	1\\
590	1\\
591	1\\
592	1\\
593	1\\
594	1\\
595	1\\
596	1\\
597	1\\
598	1\\
599	1\\
600	1\\
601	1\\
602	1\\
603	1\\
604	1\\
605	1\\
606	1\\
607	1\\
608	1\\
609	1\\
610	1\\
611	1\\
612	1\\
613	1\\
614	1\\
615	1\\
616	1\\
617	1\\
618	1\\
619	1\\
620	1\\
621	1\\
622	1\\
623	1\\
624	1\\
625	1\\
626	1\\
627	1\\
628	1\\
629	1\\
630	1\\
631	1\\
632	1\\
633	1\\
634	1\\
635	1\\
636	1\\
637	1\\
638	1\\
639	1\\
640	1\\
641	1\\
642	1\\
643	1\\
644	1\\
645	1\\
646	1\\
647	1\\
648	1\\
649	1\\
650	1\\
651	1\\
652	1\\
653	1\\
654	1\\
655	1\\
656	1\\
657	1\\
658	1\\
659	1\\
660	1\\
661	1\\
662	1\\
663	1\\
664	1\\
665	1\\
666	1\\
667	1\\
668	1\\
669	1\\
670	1\\
671	1\\
672	1\\
673	1\\
674	1\\
675	1\\
676	1\\
677	1\\
678	1\\
679	1\\
680	1\\
681	1\\
682	1\\
683	1\\
684	1\\
685	1\\
686	1\\
687	1\\
688	1\\
689	1\\
690	1\\
691	1\\
692	1\\
693	1\\
694	1\\
695	1\\
696	1\\
697	1\\
698	1\\
699	1\\
700	1\\
701	1\\
702	1\\
703	1\\
704	1\\
705	1\\
706	1\\
707	1\\
708	1\\
709	1\\
710	1\\
711	1\\
712	1\\
713	1\\
714	1\\
715	1\\
716	1\\
717	1\\
718	1\\
719	1\\
720	1\\
721	1\\
722	1\\
723	1\\
724	1\\
725	1\\
726	1\\
727	1\\
728	1\\
729	1\\
730	1\\
731	1\\
732	1\\
733	1\\
734	1\\
735	1\\
736	1\\
737	1\\
738	1\\
739	1\\
740	1\\
741	1\\
742	1\\
743	1\\
744	1\\
745	1\\
746	1\\
747	1\\
748	1\\
749	1\\
750	1\\
751	1\\
752	1\\
753	1\\
754	1\\
755	1\\
756	1\\
757	1\\
758	1\\
759	1\\
760	1\\
761	1\\
762	1\\
763	1\\
764	1\\
765	1\\
766	1\\
767	1\\
768	1\\
769	1\\
770	1\\
771	1\\
772	1\\
773	1\\
774	1\\
775	1\\
776	1\\
777	1\\
778	1\\
779	1\\
780	1\\
781	1\\
782	1\\
783	1\\
784	1\\
785	1\\
786	1\\
787	1\\
788	1\\
789	1\\
790	1\\
791	1\\
792	1\\
793	1\\
794	1\\
795	1\\
796	1\\
797	1\\
798	1\\
799	1\\
800	1\\
801	1\\
802	1\\
803	1\\
804	1\\
805	1\\
806	1\\
807	1\\
808	1\\
809	1\\
810	1\\
811	1\\
812	1\\
813	1\\
814	1\\
815	1\\
816	1\\
817	1\\
818	1\\
819	1\\
820	1\\
821	1\\
822	1\\
823	1\\
824	1\\
825	1\\
826	1\\
827	1\\
828	1\\
829	1\\
830	1\\
831	1\\
832	1\\
833	1\\
834	1\\
835	1\\
836	1\\
837	1\\
838	1\\
839	1\\
840	1\\
841	1\\
842	1\\
843	1\\
844	1\\
845	1\\
846	1\\
847	1\\
848	1\\
849	1\\
850	1\\
851	1\\
852	1\\
853	1\\
854	1\\
855	1\\
856	1\\
857	1\\
858	1\\
859	1\\
860	1\\
861	1\\
862	1\\
863	1\\
864	1\\
865	1\\
866	1\\
867	1\\
868	1\\
869	1\\
870	1\\
871	1\\
872	1\\
873	1\\
874	1\\
875	1\\
876	1\\
877	1\\
878	1\\
879	1\\
880	1\\
881	1\\
882	1\\
883	1\\
884	1\\
885	1\\
886	1\\
887	1\\
888	1\\
889	1\\
890	1\\
891	1\\
892	1\\
893	1\\
894	1\\
895	1\\
896	1\\
897	1\\
898	1\\
899	1\\
900	1\\
901	1\\
902	1\\
903	1\\
904	1\\
905	1\\
906	1\\
907	1\\
908	1\\
909	1\\
910	1\\
911	1\\
912	1\\
913	1\\
914	1\\
915	1\\
916	1\\
917	1\\
918	1\\
919	1\\
920	1\\
921	1\\
922	1\\
923	1\\
924	1\\
925	1\\
926	1\\
927	1\\
928	1\\
929	1\\
930	1\\
931	1\\
932	1\\
933	1\\
934	1\\
935	1\\
936	1\\
937	1\\
938	1\\
939	1\\
940	1\\
941	1\\
942	1\\
943	1\\
944	1\\
945	1\\
946	1\\
947	1\\
948	1\\
949	1\\
950	1\\
951	1\\
952	1\\
953	1\\
954	1\\
955	1\\
956	1\\
957	1\\
958	1\\
959	1\\
960	1\\
961	1\\
962	1\\
963	1\\
964	1\\
965	1\\
966	1\\
967	1\\
968	1\\
969	1\\
970	1\\
971	1\\
972	1\\
973	1\\
974	1\\
975	1\\
976	1\\
977	1\\
978	1\\
979	1\\
980	1\\
981	1\\
982	1\\
983	1\\
984	1\\
985	1\\
986	1\\
987	1\\
988	1\\
989	1\\
990	1\\
991	1\\
992	1\\
993	1\\
994	1\\
995	1\\
996	1\\
997	1\\
998	1\\
999	1\\
1000	1\\
1001	1\\
1002	1\\
1003	1\\
1004	1\\
1005	1\\
1006	1\\
1007	1\\
1008	1\\
1009	1\\
1010	1\\
1011	1\\
1012	1\\
1013	1\\
1014	1\\
1015	1\\
1016	1\\
1017	1\\
1018	1\\
1019	1\\
1020	1\\
1021	1\\
1022	1\\
1023	1\\
1024	1\\
1025	1\\
1026	1\\
1027	1\\
1028	1\\
1029	1\\
1030	1\\
1031	1\\
1032	1\\
1033	1\\
1034	1\\
1035	1\\
1036	1\\
1037	1\\
1038	1\\
1039	1\\
1040	1\\
1041	1\\
1042	1\\
1043	1\\
1044	1\\
1045	1\\
1046	1\\
1047	1\\
1048	1\\
1049	1\\
1050	1\\
1051	1\\
1052	1\\
1053	1\\
1054	1\\
1055	1\\
1056	1\\
1057	1\\
1058	1\\
1059	1\\
1060	1\\
1061	1\\
1062	1\\
1063	1\\
1064	1\\
1065	1\\
1066	1\\
1067	1\\
1068	1\\
1069	1\\
1070	1\\
1071	1\\
1072	1\\
1073	1\\
1074	1\\
1075	1\\
1076	1\\
1077	1\\
1078	1\\
1079	1\\
1080	1\\
1081	1\\
1082	1\\
1083	1\\
1084	1\\
1085	1\\
1086	1\\
1087	1\\
1088	1\\
1089	1\\
1090	1\\
1091	1\\
1092	1\\
1093	1\\
1094	1\\
1095	1\\
1096	1\\
1097	1\\
1098	1\\
1099	1\\
1100	1\\
1101	1\\
1102	1\\
1103	1\\
1104	1\\
1105	1\\
1106	1\\
1107	1\\
1108	1\\
1109	1\\
1110	1\\
1111	1\\
1112	1\\
1113	1\\
1114	1\\
1115	1\\
1116	1\\
1117	1\\
1118	1\\
1119	1\\
1120	1\\
1121	1\\
1122	1\\
1123	1\\
1124	1\\
1125	1\\
1126	1\\
1127	1\\
1128	1\\
1129	1\\
1130	1\\
1131	1\\
1132	1\\
1133	1\\
1134	1\\
1135	1\\
1136	1\\
1137	1\\
1138	1\\
1139	1\\
1140	1\\
1141	1\\
1142	1\\
1143	1\\
1144	1\\
1145	1\\
1146	1\\
1147	1\\
1148	1\\
1149	1\\
1150	1\\
1151	1\\
1152	1\\
1153	1\\
1154	1\\
1155	1\\
1156	1\\
1157	1\\
1158	1\\
1159	1\\
1160	1\\
1161	1\\
1162	1\\
1163	1\\
1164	1\\
1165	1\\
1166	1\\
1167	1\\
1168	1\\
1169	1\\
1170	1\\
1171	1\\
1172	1\\
1173	1\\
1174	1\\
1175	1\\
1176	1\\
1177	1\\
1178	1\\
1179	1\\
1180	1\\
1181	1\\
1182	1\\
1183	1\\
1184	1\\
1185	1\\
1186	1\\
1187	1\\
1188	1\\
1189	1\\
1190	1\\
1191	1\\
1192	1\\
1193	1\\
1194	1\\
1195	1\\
1196	1\\
1197	1\\
1198	1\\
1199	1\\
1200	1\\
1201	1\\
1202	1\\
1203	1\\
1204	1\\
1205	1\\
1206	1\\
1207	1\\
1208	1\\
1209	1\\
1210	1\\
1211	1\\
1212	1\\
1213	1\\
1214	1\\
1215	1\\
1216	1\\
1217	1\\
1218	1\\
1219	1\\
1220	1\\
1221	1\\
1222	1\\
1223	1\\
1224	1\\
1225	1\\
1226	1\\
1227	1\\
1228	1\\
1229	1\\
1230	1\\
1231	1\\
1232	1\\
1233	1\\
1234	1\\
1235	1\\
1236	1\\
1237	1\\
1238	1\\
1239	1\\
1240	1\\
1241	1\\
1242	1\\
1243	1\\
1244	1\\
1245	1\\
1246	1\\
1247	1\\
1248	1\\
1249	1\\
1250	1\\
1251	1\\
1252	1\\
1253	1\\
1254	1\\
1255	1\\
1256	1\\
1257	1\\
1258	1\\
1259	1\\
1260	1\\
1261	1\\
1262	1\\
1263	1\\
1264	1\\
1265	1\\
1266	1\\
1267	1\\
1268	1\\
1269	1\\
1270	1\\
1271	1\\
1272	1\\
1273	1\\
1274	1\\
1275	1\\
1276	1\\
1277	1\\
1278	1\\
1279	1\\
1280	1\\
1281	1\\
1282	1\\
1283	1\\
1284	1\\
1285	1\\
1286	1\\
1287	1\\
1288	1\\
1289	1\\
1290	1\\
1291	1\\
1292	1\\
1293	1\\
1294	1\\
1295	1\\
1296	1\\
1297	1\\
1298	1\\
1299	1\\
1300	1\\
1301	1\\
1302	1\\
1303	1\\
1304	1\\
1305	1\\
1306	1\\
1307	1\\
1308	1\\
1309	1\\
1310	1\\
1311	1\\
1312	1\\
1313	1\\
1314	1\\
1315	1\\
1316	1\\
1317	1\\
1318	1\\
1319	1\\
1320	1\\
1321	1\\
1322	1\\
1323	1\\
1324	1\\
1325	1\\
1326	1\\
1327	1\\
1328	1\\
1329	1\\
1330	1\\
1331	1\\
1332	1\\
1333	1\\
1334	1\\
1335	1\\
1336	1\\
1337	1\\
1338	1\\
1339	1\\
1340	1\\
1341	1\\
1342	1\\
1343	1\\
1344	1\\
1345	1\\
1346	1\\
1347	1\\
1348	1\\
1349	1\\
1350	1\\
1351	1\\
1352	1\\
1353	1\\
1354	1\\
1355	1\\
1356	1\\
1357	1\\
1358	1\\
1359	1\\
1360	1\\
1361	1\\
1362	1\\
1363	1\\
1364	1\\
1365	1\\
1366	1\\
1367	1\\
1368	1\\
1369	1\\
1370	1\\
1371	1\\
1372	1\\
1373	1\\
1374	1\\
1375	1\\
1376	1\\
1377	1\\
1378	1\\
1379	1\\
1380	1\\
1381	1\\
1382	1\\
1383	1\\
1384	1\\
1385	1\\
1386	1\\
1387	1\\
1388	1\\
1389	1\\
1390	1\\
1391	1\\
1392	1\\
1393	1\\
1394	1\\
1395	1\\
1396	1\\
1397	1\\
1398	1\\
1399	1\\
1400	1\\
1401	1\\
1402	1\\
1403	1\\
1404	1\\
1405	1\\
1406	1\\
1407	1\\
1408	1\\
1409	1\\
1410	1\\
1411	1\\
1412	1\\
1413	1\\
1414	1\\
1415	1\\
1416	1\\
1417	1\\
1418	1\\
1419	1\\
1420	1\\
1421	1\\
1422	1\\
1423	1\\
1424	1\\
1425	1\\
1426	1\\
1427	1\\
1428	1\\
1429	1\\
1430	1\\
1431	1\\
1432	1\\
1433	1\\
1434	1\\
1435	1\\
1436	1\\
1437	1\\
1438	1\\
1439	1\\
1440	1\\
1441	1\\
1442	1\\
1443	1\\
1444	1\\
1445	1\\
1446	1\\
1447	1\\
1448	1\\
1449	1\\
1450	1\\
1451	1\\
1452	1\\
1453	1\\
1454	1\\
1455	1\\
1456	1\\
1457	1\\
1458	1\\
1459	1\\
1460	1\\
1461	1\\
1462	1\\
1463	1\\
1464	1\\
1465	1\\
1466	1\\
1467	1\\
1468	1\\
1469	1\\
1470	1\\
1471	1\\
1472	1\\
1473	1\\
1474	1\\
1475	1\\
1476	1\\
1477	1\\
1478	1\\
1479	1\\
1480	1\\
1481	1\\
1482	1\\
1483	1\\
1484	1\\
1485	1\\
1486	1\\
1487	1\\
1488	1\\
1489	1\\
1490	1\\
1491	1\\
1492	1\\
1493	1\\
1494	1\\
1495	1\\
1496	1\\
1497	1\\
1498	1\\
1499	1\\
1500	1\\
1501	1\\
1502	1\\
1503	1\\
1504	1\\
1505	1\\
1506	1\\
1507	1\\
1508	1\\
1509	1\\
1510	1\\
1511	1\\
1512	1\\
1513	1\\
1514	1\\
1515	1\\
1516	1\\
1517	1\\
1518	1\\
1519	1\\
1520	1\\
1521	1\\
1522	1\\
1523	1\\
1524	1\\
1525	1\\
1526	1\\
1527	1\\
1528	1\\
1529	1\\
1530	1\\
1531	1\\
1532	1\\
1533	1\\
1534	1\\
1535	1\\
1536	1\\
1537	1\\
1538	1\\
1539	1\\
1540	1\\
1541	1\\
1542	1\\
1543	1\\
1544	1\\
1545	1\\
1546	1\\
1547	1\\
1548	1\\
1549	1\\
1550	1\\
1551	1\\
1552	1\\
1553	1\\
1554	1\\
1555	1\\
1556	1\\
1557	1\\
1558	1\\
1559	1\\
1560	1\\
1561	1\\
1562	1\\
1563	1\\
1564	1\\
1565	1\\
1566	1\\
1567	1\\
1568	1\\
1569	1\\
1570	1\\
1571	1\\
1572	1\\
1573	1\\
1574	1\\
1575	1\\
1576	1\\
1577	1\\
1578	1\\
1579	1\\
1580	1\\
1581	1\\
1582	1\\
1583	1\\
1584	1\\
1585	1\\
1586	1\\
1587	1\\
1588	1\\
1589	1\\
1590	1\\
1591	1\\
1592	1\\
1593	1\\
1594	1\\
1595	1\\
1596	1\\
1597	1\\
1598	1\\
1599	1\\
1600	1\\
1601	1\\
1602	1\\
1603	1\\
1604	1\\
1605	1\\
1606	1\\
1607	1\\
1608	1\\
1609	1\\
1610	1\\
1611	1\\
1612	1\\
1613	1\\
1614	1\\
1615	1\\
1616	1\\
1617	1\\
1618	1\\
1619	1\\
1620	1\\
1621	1\\
1622	1\\
1623	1\\
1624	1\\
1625	1\\
1626	1\\
1627	1\\
1628	1\\
1629	1\\
1630	1\\
1631	1\\
1632	1\\
1633	1\\
1634	1\\
1635	1\\
1636	1\\
1637	1\\
1638	1\\
1639	1\\
1640	1\\
1641	1\\
1642	1\\
1643	1\\
1644	1\\
1645	1\\
1646	1\\
1647	1\\
1648	1\\
1649	1\\
1650	1\\
1651	1\\
1652	1\\
1653	1\\
1654	1\\
1655	1\\
1656	1\\
1657	1\\
1658	1\\
1659	1\\
1660	1\\
1661	1\\
1662	1\\
1663	1\\
1664	1\\
1665	1\\
1666	1\\
1667	1\\
1668	1\\
1669	1\\
1670	1\\
1671	1\\
1672	1\\
1673	1\\
1674	1\\
1675	1\\
1676	1\\
1677	1\\
1678	1\\
1679	1\\
1680	1\\
1681	1\\
1682	1\\
1683	1\\
1684	1\\
1685	1\\
1686	1\\
1687	1\\
1688	1\\
1689	1\\
1690	1\\
1691	1\\
1692	1\\
1693	1\\
1694	1\\
1695	1\\
1696	1\\
1697	1\\
1698	1\\
1699	1\\
1700	1\\
1701	1\\
1702	1\\
1703	1\\
1704	1\\
1705	1\\
1706	1\\
1707	1\\
1708	1\\
1709	1\\
1710	1\\
1711	1\\
1712	1\\
1713	1\\
1714	1\\
1715	1\\
1716	1\\
1717	1\\
1718	1\\
1719	1\\
1720	1\\
1721	1\\
1722	1\\
1723	1\\
1724	1\\
1725	1\\
1726	1\\
1727	1\\
1728	1\\
1729	1\\
1730	1\\
1731	1\\
1732	1\\
1733	1\\
1734	1\\
1735	1\\
1736	1\\
1737	1\\
1738	1\\
};
\end{axis}
\end{tikzpicture}%
}
    \caption{The overall processing usage. Notice that it always at $100\%$.
      $C_i = 10$ ms.}
      \label{fig:5.4}
  \end{figure}
\end{minipage}
\end{minipage}
}
