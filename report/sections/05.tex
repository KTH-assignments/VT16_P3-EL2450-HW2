\section{Question 5}

In the case where $T_1 = 20, T_2 = 29, T_3 = 35$ ms and $C_i = 10$ ms,
$i=\{1,2,3\}$, $U=0.678$ and $n(2^{1/n} - 1) = 0.78$. Hence tasks $J_1, J_2, J_3$
are schedulable with RM.

\begin{figure}[H]\centering
  \scalebox{0.7}{\input{./figures/2.gantt.tex}}
  \caption{A portion of the RM schedule $\sigma$ for tasks $J_1, J_2, J_3$.
    Shaded areas denote the waiting time.}
\end{figure}
Figure \ref{fig:5.3} shows the schedule calculated for each pendulum with all
jobs having execution time $C_i = 10$ ms.  Figure \ref{fig:5.4} illustrates that
the schedule is again feasible by ploting the overall usage of the CPU over the
aforementioned time span.


\noindent\makebox[\textwidth][c]{%
\begin{minipage}{\linewidth}
  \begin{minipage}{0.45\linewidth}
    \begin{figure}[H]\centering
      \scalebox{0.7}{\input{./figures/5.3.tex}}
      \caption{The calculated schedule for the three penduli.
        \texttt{Blue}: $P_1$, \texttt{Red}: $P_2$, \texttt{Orange}: $P_3$.
        $C_i = 10$ ms.}
      \label{fig:5.3}
    \end{figure}
  \end{minipage}
  \hfill
  \begin{minipage}{0.45\linewidth}
    \begin{figure}[H]\centering
    \scalebox{0.7}{\input{./figures/5.4.tex}}
    \caption{The overall processing usage. Notice that it is at most at $100\%$.
      $C_i = 10$ ms.}
      \label{fig:5.4}
  \end{figure}
\end{minipage}
\end{minipage}
}
